\documentclass[11pt]{book}

%PAPIR - US TRADE
\paperwidth 15.24cm
\paperheight 22.86cm

%TEKST
\textwidth 11.9cm \textheight 19.4cm
\oddsidemargin=-0.5cm
\evensidemargin=-1.2cm
\topmargin=-15mm

\headheight=13.86pt

%\usepackage[slovene]{babel}
\usepackage[english]{babel}

%\usepackage[cp1250]{inputenc}
\usepackage[utf8]{inputenc}


\usepackage[T1]{fontenc}
\usepackage{amsmath}
\usepackage{color}
\usepackage{amsfonts}
\usepackage{makeidx}
\usepackage{calc}
\usepackage{gclc}
%\usepackage[dvips]{hyperref}
\usepackage{amssymb}
\usepackage[dvips]{graphicx}
\usepackage{fancyhdr}

%za slike
\usepackage{caption}
\DeclareCaptionFormat{empty}

\def\contentsname{Vsebina}

\makeindex

\newcommand{\ch}{\mathop {\mathrm{ch}}}
\newcommand{\sh}{\mathop {\mathrm{sh}}}
\newcommand{\tgh}{\mathop {\mathrm{th}}}
\newcommand{\tg}{\mathop {\mathrm{tg}}}
\newcommand{\ctg}{\mathop {\mathrm{ctg}}}
\newcommand{\arctg}{\mathop {\mathrm{arctg}}}
\newcommand{\arctgh}{\mathop {\mathrm{arcth}}}

\def\indexname{Indeks}

\definecolor{green1}{rgb}{0,0.5,0}
\definecolor{viol}{rgb}{0.5,0,0.5}
\definecolor{viol1}{rgb}{0.2,0,0.9}
\definecolor{viol3}{rgb}{0.3,0,0.6}
\definecolor{viol4}{rgb}{0.6,0,0.6}
\definecolor{grey}{rgb}{0.5,0.5,0.5}

 \def\qed{$\hfill\Box$}
\newcommand{\kdokaz}{\color{red}\qed\vspace*{2mm}\normalcolor}

\newcommand{\res}[1]{\color{green1}\textit{#1}\normalcolor}

\newtheorem{izrek}{Theorem}[section]
\newtheorem{lema}{Lemma}[section]
\newtheorem{definicija}{Definition}[section]
\newtheorem{aksiom}{Axiom}[section]
\newtheorem{zgled}{Exercise}[section]
\newtheorem{naloga}{Problem}
\newtheorem{trditev}{Proposition}[section]
\newtheorem{postulat}{Postulate}
\newtheorem{ekv}{E}


%BARVA

\newcommand{\pojem}[1]{\color{viol4}\textit{#1}\normalcolor}

%\newcommand{\pojemFN}[1]{\textit{#1}}

\newcommand{\blema}{\color{blue}\begin{lema}}
\newcommand{\elema}{\end{lema}\normalcolor}

\newcommand{\bizrek}{\color{blue}\begin{izrek}}
\newcommand{\eizrek}{\end{izrek}\normalcolor}

\newcommand{\bdefinicija}{\begin{definicija}}
\newcommand{\edefinicija}{\end{definicija}}

\newcommand{\baksiom}{\color{viol3}\begin{aksiom}}
\newcommand{\eaksiom}{\end{aksiom}\normalcolor}

\newcommand{\bzgled}{\color{green1}\begin{zgled}}
\newcommand{\ezgled}{\end{zgled}\normalcolor}

\newcommand{\bnaloga}{\color{red}\begin{naloga}}
\newcommand{\enaloga}{\end{naloga}\normalcolor}

\newcommand{\btrditev}{\color{blue}\begin{trditev}}
\newcommand{\etrditev}{\end{trditev}\normalcolor}


\newcommand{\del}[1]{\chapter{#1}}
\newcommand{\poglavje}[1]{\section{#1}}
%\newcommand{\naloge}[1]{\color{red}\section*{#1}\normalcolor}
\newcommand{\naloge}[1]{\section{#1}}
\newcommand{\ppoglavje}[1]{\subsection{#1}}

\setlength\arraycolsep{2pt}

\author{Milan Mitrovi\'c}
\title{\textsl{\Huge{\textbf{Euclidean Plane Geometry}}}}

\date{}

%_________________________________________________________________________________________

\begin{document}
\pagestyle{fancy}
\lhead[\thepage]{\textsl{\nouppercase{\rightmark}}}
\rhead[\textsl{\nouppercase{\leftmark}}]{\thepage}
\cfoot[]{}


 \vspace*{-12mm}

\hspace*{24mm} \textsl{\Huge{\textbf{Euclidean }}}\\

\hspace*{24mm} \textsl{\Huge{\textbf{Plane}}}\\

\hspace*{24mm} \textsl{\Huge{\textbf{Geometry}}}

 \vspace*{8mm}\hspace*{60mm}Milan Mitrovi\'c
% \normalcolor

 \vspace*{0mm}

\hspace*{-21mm}
\input{sl.NASL2A4.pic}

%\color{viol1}
 \hspace*{48mm}Sevnica

 \hspace*{49mm}
2013
% \normalcolor
  %slikaNova0-1-1
%\includegraphics[width=120mm]{slikaNaslov.pdf}

 \setcounter{section}{0}
 \thispagestyle{empty}
 \newpage


%\pagecolor{white}

%\color{viol1}
\vspace*{17mm}
 \hspace*{77mm} \textit{To Boris and Jasmina}
%\normalcolor
  %slikaNova0-1-1
%\includegraphics[width=120mm]{slikaNaslov.pdf}


 \thispagestyle{empty}
\newpage

%________________________________________________________________________

%PREDGOVOR
%{\hypertarget{Vsebina}\tableofcontents}
 %\printindex
\thispagestyle{empty}


%_______________________________________________________________________


\chapter*{Preface}

\thispagestyle{empty}


\thispagestyle{empty}

The present book is the result of the experience I gained as a professor in teaching geometry at the Mathematical Gymnasium in Belgrade for many years and preparing students in Slovenia for the International Mathematical Olympiad (IMO).

Formally, the substance is presented in such a way that it does not rely on prior knowledge of geometry. In the book, we will deal only with planar Euclidean geometry - all definitions and statements refer to the plane.

The first two chapters deal with the history and axiomatic design of geometry.
The consequences of the axioms of incidence,  congruence and parallelism are discussed in detail, while in the other two groups (axioms of order and continuity) the consequences are mostly not proven.
Chapters three and four deal with the relation of the congruence of figures, the use of the triangle congruence theorems, and a circle.
In the fifth chapter, a vectors are defined. Thales's theorem of proportion is proven.
Chapter six deals with isometries and their use. Their classification has been performed.
Chapters 7 and 8 deal with similarity transformations, figure similarity relation, and area of figures. The ninth chapter presents the inversion.
At the end of each chapter (except the introductory one) are exercises. Solutions and instructions can be found in the last, tenth chapter.


The book contains 341 theorems, 247 examples and 418 solved problems (28 of them from the IMO). In this sense, the book in front of you is at the same time a preparing guide for the IMO.

For some well-known theorems and problems, are given brief historical remarks. That can help high school and college students to better understand the development of geometry over the centuries.

A lot of help in writing the book was selflessly offered to me by Prof. Roman Drstvenšek, who read the manuscript in its entirety. With his professional and linguistic comments, he made a great contribution to the final version of the book.
In that work, he was assisted by Prof. Ana Kretič Mamič. I kindly thank both of them for the effort and time they have generously devoted to this book.

I would especially like to thank Prof. Kristjan Kocbek, who read the partial manuscript
 and with his critical remarks contributed to
 significant improvement of the book.

%I sincerely thank prof. Gordana Kenda Kozinc, who read and proofread the introductory chapter, and thanks also to prof. Alenka Brilej, who helped Roman with the language test with quite a few useful tips.

I also thank Prof. Dr Predrag Janiči\'c, who wrote the wonderful software package \textit {GCLC} for \ LaTeX {}. Almost all pictures in this book were made with this package.

Last but not least, I would like to thank the students of the Bežigrad Grammar School, the 1st Grammar School in Celje and the Brežice Grammar School for their inspiration and support. Students from these schools attended the renewed course of geometry which I have already taken before
years in Belgrade.

\thispagestyle{empty}

\vspace*{12mm} Sevnica, December 2013 \hfill Milan
Mitrovi\'c
%\newpage

%________________________________________________________________________

 \tableofcontents

 %\thispagestyle{empty}

%\newpage



% DEL 1 - - - - - - - - - - - - - - - - - - - - - - - - - - - - - - - - - - - - - - -
%________________________________________________________________________________
% O DEDUKTIVNI IN INDUKTIVNI METODI
%________________________________________________________________________________

 \del{Introduction} \label{pogUVOD}

%________________________________________________________________________________
\poglavje{Deductive and Inductive Method} \label{odd1DEDUKT}

Že v osnovni šoli spoznamo veliko geometrijskih pojmov, kot so:
trikotnik, krožnica, pravi kot itd. Kasneje se naučimo tudi
nekaj izrekov: izreki o skladnosti trikotnikov, Pitagorov in
Talesov izrek. V začetku izrekov ne dokazujemo, ampak dejstva
ugotavljamo na osnovi večih posameznih primerov. Ta način
sklepanja se imenuje induktivna metoda. Induktivna metoda (lat.
inductio -- uvajanje) je torej način sklepanja, pri katerem od
posameznih pridemo do splošnih zaključkov. Kasneje začnemo
posamezne izreke dokazovati. Skozi te dokaze se prvič srečamo s
t. i. deduktivnim načinom sklepanja oz. z dedukcijo. Dedukcija
(lat. deductio -- izvajanje) je način sklepanja, pri katerem se od
splošnih pride do posameznih zaključkov. Ideja pri tej
metodi torej je, da z dokazovanjem izpeljemo splošni zaključek in ga
potem uporabljamo v posameznih primerih. Ker pri induktivni
metodi ne moremo preveriti vseh primerov, saj je njihovo število
najpogosteje neskončno, lahko s to metodo  pridemo tudi do
napačnih zaključkov. Z deduktivno metodo dobimo vedno pravilne
zaključke, če so le predpostavke, ki jih v dokazu uporabljamo,
pravilne. Na naslednjem primeru analizirajmo obe omenjeni metodi.
Poskusimo priti do ugotovitve:
 \btrditev \label{TalesUvod}
 The diameter of a circle subtends a right angle to any point on the circle.
 \etrditev


\begin{figure}[!htb]
\centering
\input{sl.1.2.1.6.pic}
\caption{} \label{sl.sl.1.2.1.6.pic}
\end{figure}

  %slikaNova0-1-1
%\includegraphics[width=50mm]{slikaNova0-1-1.pdf}

Če bi uporabljali induktivno metodo, bi preverjali, ali ta trditev
velja v nekih posameznih primerih; npr.  v primeru, ko je  vrh
kota središče polkrožnice in podobno (Figure
\ref{sl.sl.1.2.1.6.pic}). Če bi le iz teh posameznih
primerov izpeljali splošno ugotovitev, seveda ne bi mogli biti
prepričani, da v katerem od primerov, ki ga nismo preverili, ta
trditev ne drži.

Uporabimo sedaj deduktivno metodo. Naj bo $AB$ polmer krožnice s
središčem $O$ in $L$ poljubna točka te krožnice, različna od
točk $A$ in $B$ (Figure \ref{sl.sl.1.2.1.6.pic}). Dokažimo, da je
kot $ALB$ pravi kot. Ker je
 $OA\cong OB\cong OL$, sledi,
da sta trikotnika $AOL$ in $BOL$ enakokraka, zato je
 $\angle ALO\cong\angle LAO=\alpha$ in $\angle BLO\cong\angle LBO=\beta$.
Tedaj je $\angle ALB=\alpha+\beta$.
 Vsota notranjih kotov v
trikotniku $ALB$ je enaka $180^0$,  torej je
$2\alpha+2\beta=180^0$. Iz tega sledi:
 $$\angle ALB=\alpha+\beta=90^0$$

Opazimo, da v primeru uporabe deduktivne metode oz. pri dokazovanju
trditve nismo obravnavali neke določene točke $L$ na krožnici,
ampak poljubno točko (v splošni legi). To pomeni, da trditev
velja za vsako točko krožnice (razen $A$ in $B$), če je seveda
dokaz pravilen. Toda ali je dokaz pravilen? V tem dokazu smo
uporabljali naslednji dve trditvi:
 \btrditev
 If two sides in a triangle are congruent, then the angles opposite the congruent sides are congruent angles.
 \etrditev
 \btrditev
  The sum of the interior angles of a triangle is equal to $180^0$.
   \etrditev
Uporabili smo tudi pojme, kot so: enakokraki trikotnik, skladnost
kotov; v sami trditvi pa tudi pojme: premer, krožnica, kot nad
premerom in pravi kot. Da smo prepričani, ali je trditev, ki
smo jo dokazovali, točna, moramo biti gotovi, da sta tudi trditvi,
ki smo ju uporabili v dokazu, točni. V našem primeru
predpostavljamo, da smo  omenjeni dve trditvi že dokazali in da
smo vpeljali vse omenjene pojme. Jasno je, da se ta problem pojavi
pri vsaki trditvi -- tudi pri dveh, na kateri smo se sklicevali v
dokazu. To  zahteva določeno sistematizacijo cele geometrije.
Zastavlja se vprašanje, kako začeti, če se v dokazu vsake
trditve zopet sklicujemo na prej dokazane. Ta proces bi se potem
lahko nadaljeval v neskončnost. Tako pridemo do potrebe po
začetnih trditvah -- \index{aksiomi} \pojem{aksiomih}. Isto velja
za pojme -- potrebujemo t. i. \pojem{začetne pojme}.\index{začetni
pojmi} Na ta način je vsaka geometrija (lahko jih je torej več),
ki jo obravnavamo, odvisna od izbire začetnih pojmov in aksiomov.
Ta pristop izgrajevanja neke geometrije imenujemo
\pojem{sintetični postopek}, sami geometriji pa pravimo, da je
\pojem{sintetična geometrija}\index{geometrija!sinteti\v{c}na}.


%________________________________________________________________________________
\poglavje{Basic Terms and Basic Theorems} \label{odd1POJMI}

V neki teoriji  (kot je geometrija) vsako vpeljavo novega
pojma naredimo z \index{definicija} \pojem{definicijo}, s katero ta pojem
opišemo s pomočjo
 nekih začetnih ali že definiranih pojmov.
 Povezave med pojmi in tudi njihove ustrezne lastnosti so dane z izjavami,
  ki jih imenujemo
\pojem{trditve teorije}. Kot smo že omenili, začetne trditve
imenujemo \index{aksiomi} \pojem{aksiomi}, trditve, ki so
iz njih izpeljane, pa  \pojem{izreki} te teorije. Formalno je
\pojem{dokaz} \index{dokaz izreka} nekega izreka $\tau$ zaporedje
trditev, ki logično sledijo ena iz druge, od katerih je vsaka
ali aksiom ali iz aksiomov izpeljana trditev (izrek), zadnja v tem
zaporedju pa je ravno trditev $\tau$.

 Čeprav izbira aksiomov ni enolično določena, ta ne more biti poljubna.
 Pri tej izbiri je
treba paziti, da aksiomi ne pripeljejo do protislovnih trditev
oziroma da ne pride do protislovja. To pomeni, da pri neki izbiri
aksiomov ne obstaja takšna trditev, da sta ta trditev in hkrati njena
negacija izreka v tej teoriji. Potrebno je imeti tudi dovolj
aksiomov, da bi za vsako trditev, ki jo lahko formuliramo v tej
teoriji, lahko ugotovili, ali velja ali ne. To pomeni, da je
bodisi trditev bodisi njena negacija izrek v tej teoriji. Za
sistem aksiomov, ki izpolnjuje prvo zahtevo, pravimo, da je
\index{sistem aksiomov!neprotisloven} \pojem{neprotisloven}, za
tistega, ki izpolnjuje drugo zahtevo, pa pravimo, da je \index{sistem
aksiomov!popoln} \pojem{popoln}. Pri izbiri aksiomov obstaja tudi
tretja zahteva -- da je sistem aksiomov \index{sistem
aksiomov!minimalen} \pojem{minimalen}, kar pomeni, da se noben od
aksiomov ne da izpeljati iz ostalih. Omenimo, da zadnja
 zahteva ni  tako pomembna, kot sta prvi dve.

Dodati moramo še, da evklidske geometrije ne gradimo
neodvisno od algebre in logike. Uporabljali bomo namreč pojme, kot
so npr. množica, funkcija, relacija z lastnostmi, ki za njih
veljajo. Uporabljali bomo tudi t. i. pravila sklepanja, kot je
npr. metoda protislovja. Za matematične discipline, ki jih na ta
način uporabljamo pri gradnji geometrije, pravimo, da so
\pojem{predpostavljene teorije}.



%________________________________________________________________________________
\poglavje{A Brief Historical Overview of the Development of Geometry}
\label{odd1ZGOD}

Z geometrijo so se ljudje začeli ukvarjati  že v rani zgodovini.
 V začetku je bilo to le opazovanje karakterističnih oblik, kot sta
 krožnica ali kvadrat. Po risbah, ki so bile odkrite na stenah starih jam, sklepamo,
 da so se ljudje že v prazgodovini  zanimali za simetrijo likov.

V nadaljnjem razvoju je človek ugotavljal razne lastnosti
geometrijskih likov. To je bilo zaradi praktičnih potreb, npr.
merjenja površine zemljišč -- tako je tudi nastala beseda
‘‘geometrija’’. V tem obdobju se je geometrija razvijala kot
induktivna znanost. To pomeni, da so do geometrijskih trditev
prihajali z izkušnjami -- s pomočjo meritev in s preverjanjem na
posameznih primerih. V tem smislu je bila geometrija razvita pri
vseh starih civilizacijah: kitajski, indijski in še posebej
egipčanski.

V Egiptu se je geometrija razvijala predvsem kot učenje o
meritvah. Ker je reka Nil večkrat poplavila, je
bilo potrebno zemljišča zelo pogosto znova premeriti. Razen tega so
znanje geometrije uporabljali tudi v gradbeništvu. Poznali so
npr. formulo za izračunavanje prostornine piramide in prisekane
piramide, čeprav so do nje prišli empirično. Tako je bila geometrija za
Egipčane predvsem pragmatična disciplina.
Najstarejši zapisi o tem segajo približno v leto 1700 pr. n. š.

Tudi v Mezopotamiji so imeli razvito geometrijo merjenja
ploščin. Geometrija trirazsežnega prostora ni bila
toliko obravnavana kot pri Egipčanih.

O kitajski geometriji ni toliko podatkov kot o egipčanski,
čeprav vemo, da je bila tudi ta  zelo razvita. V
najstarejših ohranjenih zapisih najdemo opis računanja
prostornin prizme, piramide, valja, stožca, prisekane piramide in
prisekanega stožca.

Indijska geometrija je precej mlajša od prejšnjih treh. Datira
približno v peto stoletje pr. n. š. V njej že vidimo prve
poskuse dokazovanja. Kasneje se je razvijala vzporedno s
starogrško geometrijo.

Preobrat v razvoju geometrije se je zgodil v Stari Grčiji. Tedaj
se je prvič v zgodovini pričela v geometriji uporabljati deduktivna metoda.
Prvi geometrijski dokazi so povezani s
Talesom\footnote{Starogrški filozof in matematik \textit{Tales}
\index{Tales} iz Mileta (640--546 pr. n. š.).}. Z njegovim imenom
povezujemo znani izrek o sorazmerju odsekov pri vzporednicah.
Dokazal je tudi izrek, da so koti nad premerom krožnice pravi,
čeprav je bila ta trditev brez dokaza znana že Babiloncem 1000 let
pred tem. Ta način razvoja geometrije so nadaljevali tudi drugi
starogrški filozofi, od katerih je bil Pitagora\footnote{Starogrški
filozof in matematik \textit{Pitagora} \index{Pitagora} z otoka
Samosa (ok. 580--490 pr. n. š.).} eden najpomembnejših.
Znamenit je seveda njegov  \index{izrek!Pitagorov}\pojem{Pitagorov
izrek}. Toda ta izrek so kot dejstvo poznali že Egipčani 3000 let
pr. n. š. (mogoče je bil izrek znan celo pred tem), pač pa je
Pitagora podal prvi znani dokaz. Arhimed\footnote{Starogrški filozof
in matematik \textit{Arhimed} \index{Arhimed} iz Sirakuze (287--212 pr. n. š.).}
je prvi predstavil teoretičen izračun števila $\pi$, tako da
je obravnaval v krožnico včrtane in očrtane večkotnike s $96$
stranicami. Dokazani so bili tudi izreki o skladnosti trikotnikov.
Ob hitrem napredku geometrije, ki se je odražal v velikem številu
dokazanih izrekov, se je pokazala potreba po sistematizaciji in s
tem po vpeljavi aksiomov. Potrebo po aksiomah sta prva opisala
Platon\footnote{Starogrški filozof in matematik \textit{Platon}
\index{Platon} (427--347 pr. n. š.).} in
Aristotel\footnote{Starogrški filozof in matematik
\textit{Aristotel} \index{Aristotel} iz Aten (384--322 pr. n. š.).}.
 Platon je v matematiki znan tudi po tem, da je raziskoval pravilne poliedre:
 tetraeder, kocko, oktaeder, dodekaeder in ikozaeder, zato jih po njem
 imenujemo tudi platonska telesa.

Enega  prvih poskusov aksiomatične zasnove geometrije -- in iz tega
časa edinega ohranjenega -- je dal najznamenitejši geometer tega časa,
Platonov učenec Evklid\footnote{Starogrški filozof in matematik
\textit{Evklid} \index{Evklid} iz Aleksandrije (ok. 330--270 pr. n. š.).}, v svojem znanem delu \textit{Elementi}, ki je sestavljeno iz
13 knjig. V njem je sistematiziral vse dotedanje znanje
geometrije. Začetne trditve je razdelil na aksiome in t. i.
postulate, od katerih so slednji čisto geometrične vsebine (danes
tudi njih imenujemo aksiomi). \textit{Elementi} so postali ena od
najpomembnejših in najvplivnejših knjig v zgodovini matematike.
Geometrija, ki jo je na ta način razvil, z manjšimi nepomembnimi
spremembami, je tista, ki se v šolah uči še danes. Dokazi, kot je
npr. ta o središčnem in obodnem kotu, so se obdržali v praktično
nespremenjeni obliki. Navedimo postulate, kot jih je podal Evklid
(Figure \ref{sl.sl.1.3.1.9.pic}):
\color{viol3}
\begin{postulat}
  We can draw a straight line from any point to any point.
 \end{postulat}
 \begin{postulat}
We can produce a finite straight line continuously in a straight line.
 \end{postulat}
 \begin{postulat}
We can describe a circle with any center and distance.
 \end{postulat}
  \begin{postulat}
All right angles are equal to one another.
 \end{postulat}
 \begin{postulat}
If a straight line falling on two straight lines makes the interior angles on the same side less than two right angles, the straight lines, if produced indefinitely, will meet on that side on which the angles are less that two right angles.
 \end{postulat}
\normalcolor

\begin{figure}[!htb]
\centering
\input{sl.1.3.1.9.pic}
\caption{} \label{sl.sl.1.3.1.9.pic}
\end{figure}

  %slikaNova1-3-3
%\includegraphics[width=100mm]{slikaNova1-3-3.pdf}


Vendar sistem aksiomov, ki jih je podal Evklid, ni bil popoln. V
nekaterih dokazih je določene dele sprejel kot očitne in jih ni
dokazoval. Seveda ne smemo biti preveč kritični, saj je bilo to delo za tiste
čase revolucionarno. \textit{Elementi} so bili stoletja
zgled in inspiracija matematikom in so začrtali nadaljnji razvoj
geometrije vse do danes. Za nadaljnji razvoj geometrije je bil posebej
pomemben zadnji aksiom, t. i. \index{aksiom!peti Evklidov}
\pojem{peti Evklidov aksiom}. Problem njegove neodvisnosti od
ostalih aksiomov je bil odprt naslednjih 2000 let!

Zadnji v nizu velikih starogrških matematikov so bili
Apolonij\footnote{Starogrški matematik \textit{Apolonij}
\index{Apolonij} iz Perge (262--190 pr. n. š.).},
Menelaj\footnote{Starogrški matematik \textit{Menelaj}
\index{Menelaj} iz Aleksandrije
  (ok. 70--130).} in Pappus\footnote{Starogrški matematik \textit{Pappus}
  \index{Pappus}
  iz
  Aleksandrije (ok. 290--350).}. Apolonij je v svoji knjigi \textit{Razprava
  o presekih stožca} definiral elipso, parabolo in hiperbolo kot
  preseke ravnine in krožnega (neskončnega) stožca. Tako je lahko določene
  njihove
  lastnosti  obravnaval hkrati, kar je bil za tisti čas precej
  sodoben pristop. Menelaj in Pappus sta dokazala določene izreke,
  ki so postali aktualni šele v 19.~stoletju z razvojem  projektivne geometrije.
  Torej so bile ideje teh treh matematikov zelo sodobne in na nek način
  lahko rečemo, da so bili na pragu odkritja prve neevklidske geometrije.

Po dokončnem padcu Stare Grčije pod Rimsko cesarstvo se je
obdobje slavne starogrške geometrije končalo. Čeprav so Stari Rimljani
prevzeli velik del starogrške kulture in so gradili ceste,
vodovode in tako dalje,  je zanimivo, da se nikoli niso preveč zanimali za
starogrško teoretično matematiko. Tako je njihov prispevek k razvoju
geometrije zelo skromen.

Pomembno vlogo v nadaljnjem razvoju geometrije so prevzeli Arabci.
Najprej je treba povedati, da so nam vsa dela Starih Grkov
vključno z Evklidovimi \textit{Elementi} danes znana zato, ker so jih
takrat prevedli in tako ohranili ravno Arabci. Od ustanovitve
Bagdada leta 762 so v naslednjih 100-tih letih prevedli večino del
starogrške in indijske matematike. Naredili so tudi sintezo
starogrškega pretežno geometričnega in indijskega pretežno
algebričnega pristopa. Omenimo, da je sama beseda \pojem{algebra}
arabskega izvora. Poleg tega so Arabci nadaljevali razvoj
\pojem{trigonometrije}, ki so jo zasnovali že Stari Grki. A. R. al-Biruni\footnote{Arabski matematik \textit{ A. R. al-Biruni} \index{al-Biruni, A. R.} (973--1048).} je dokazal danes znani \pojem{sinusni izrek}.

V Evropi se je razvoj geometrije začel v 12. stoletju, ko so preko
Španije in Sicilije znanja prinašali arabski in judovski matematiki.
(Evklidovi \textit{Elementi} so bili prevedeni iz arabskega jezika
približno leta 1200); toda pravi razcvet je Evropa doživela šele v
16. stoletju. V srednjeveškem obdobju se je namreč matematika zelo
počasi razvijala. V srednjem veku so se zahodnoevropski matematiki
šele učili starogrško geometrično dediščino iz arabskih prevodov,
vendar ta proces ni bil hiter. Ko se je to znanje akumuliralo in so
se družbeno-politični pogoji spremenili, se je v
razvoju geometrije začelo novo obdobje. Prve nove rezultate so dali italijanski
matematiki tistega časa, ki so veliko pozornosti posvetili
konstrukcijam s pomočjo ravnila in šestila.

Kot smo že prej omenili, je imel peti Evklidov aksiom zelo velik vpliv na
nadaljnji razvoj geometrije. Zaradi svoje
formulacije, ki ni tako enostavna kot pri prejšnjih aksiomih, in
tudi zaradi pomena je veliko matematikov v tistem obdobju menilo,
da ga ni potrebno obravnavati kot aksiom, ampak se ga lahko z ostalimi aksiomi dokaže kot
izrek. Če preberemo druge Evklidove začetne trditve, zares drži, da je
peti aksiom bolj zapleten.
 Problem neodvisnosti petega aksioma od ostalih je v naslednjih stoletjih okupiral mnoge
 matematike. Vse do druge polovice 19.
 stoletja problem ni bil rešen. V mnogih poskusih dokazovanja petega
 aksioma iz preostalih so bile uporabljene trditve, katerih dokaz
 je bil izpuščen.
Kasneje se je pokazalo, da se teh trditev niti ne da dokazati iz
ostalih aksiomov, če se iz njihovega seznama izpusti peti aksiom.
Podobno kot iz njih sledi peti aksiom, tudi te trditve
sledijo iz petega aksioma (seveda z uporabo ostalih aksiomov).
Zato jih imenujemo \pojem{ekvivalenti petega Evklidovega aksioma}.
Navedimo nekaj primerov teh ekvivalentov (Figure
\ref{sl.sl.1.3.1.9a.pic}):

\color{blue}
\begin{ekv}
If $ ABCD $ is a quadrilateral with two right angles on the side $BC$
and the sides $AB$ and $CD$ are congurent, then the two remain angles
of this quadrilateral are also right angles.\footnote{Ta ekvivalent je postavil
italijanski matematik \index{Saccheri, G. G.} \textit{G. G.
Saccheri} (1667--1733).}
\end{ekv}
\begin{ekv}
A line perpendicular to one arm of an acute angle intersects
his other arm.
\end{ekv}
\begin{ekv}
Every triangle can be circumscribed.
\end{ekv}
\begin{ekv}
If three angles of a quadrilateral are right angles, then the fourth angle is also a right angle.\footnote{\index{Lambert, J. H.}\textit{J.
H. Lambert} (1728--1777), francoski matematik.}
\end{ekv}
\begin{ekv}
The sum of the interior angles in every triangle is $180^0$.\footnote{\index{Legendre, A. M.} \textit{A. M. Legendre}
(1752--1833), francoski matematik.}
\end{ekv}
\begin{ekv}
For any given line $p$ and point $A$ not on $p$, in the plane containing both line $p$ and point $A$ there is just one line
 through point $A$ that do not intersect line $p$\footnote{\index{Playfair, J.} \textit{J. Playfair}
(1748--1819), škotski matematik.}.
 \end{ekv}
\normalcolor


\begin{figure}[!htb]
\centering
\input{sl.1.3.1.9a.pic}
\caption{} \label{sl.sl.1.3.1.9a.pic}
\end{figure}

Matematikom je torej uspelo dokazati peti Evklidov aksiom s
pomočjo vsake od teh trditev, toda sčasoma se je pokazalo, da se
 nobene od njih brez petega aksioma sploh ne da dokazati. Zato
so te trditve, kot smo že omenili, petemu aksiomu ekvivalentne.
Danes se najpogosteje uporablja Playfairjev ekvivalent, ki je bil kasneje
namesto petega aksioma dodan k Evklidovim aksiomom.

Toda kako so matematiki ugotovili, da se peti Evklidov aksiom ne
more izpeljati iz ostalih aksiomov? Samo dejstvo, da ga niso
uspeli dokazati, še ni pomenilo, da to ni mogoče. Odgovor na to
vprašanje je prišel konec 19. stoletja in je, kot bomo videli,
za razvoj geometrije prinesel veliko več kot samo dejstvo o
nedokazljivosti petega aksioma.

Naslednja prelomnica v razvoju geometrije je bilo odkritje
\index{geometrija!neevklidska}\pojem{neevklidskih geometrij} v 19. stoletju. Za začetnika tega razvoja
štejemo N.~I.~Lobačevskega\footnote{\index{Lobačevski, N.
I.}\textit{N. I. Lobačevski} (1792--1856), ruski matematik.}. Tudi on
je obravnaval problem neodvisnosti petega Evklidovega aksioma.
Izhajajoč iz njegove negacije oz. iz negacije Playfairjeve
ekvivalentne trditve je Lobačevski postavil predpostavko, da skozi
točko, ki ne leži na neki premici, obstajata vsaj dve premici, ki se s
to premico ne sekata, in sta komplanarni. V želji, da bi prišel do
protislovja (tako bi bil peti aksiom dokazan), je zgradil celo
zaporedje novih trditev. Ena med njimi je na primer ta, da je vsota notranjih kotov
trikotnika vedno manjša od iztegnjenega kota. Toda nobena teh trditev
ni bila v protislovju z ostalimi aksiomi, če seveda s seznama izključimo
peti Evklidov aksiom. Iz tega je dobil idejo, da je mogoče zgraditi
popolnoma novo geometrijo, ki je neprotislovna in temelji na vseh
Evklidovih aksiomih z izjemo petega, ki ga zamenjamo z njegovo
negacijo. Danes to geometrijo imenujemo
\index{geometrija!hiperbolična} \pojem{hiperbolična geometrija}
ali \pojem{geometrija Lobačevskega}.

Neodvisno od Lobačevskega je do istih rezultatov prišel tudi J.
Bolyai\footnote{\index{Bolyai, J.} \textit{J. Bolyai} (1802--1860),
madžarski matematik.}. Kot se to pogosto dogaja, ideje Lobačevskega v času njegovega življenja žal
niso bile sprejete. Popolno potrditev teh idej
oziroma dokaz neprotislovja te nove geometrije je konec 19. stoletja, tj. šele po smrti Lobačevskega, predstavil A.
Poincar\'{e}\footnote{\index{Poincar\'{e}, J. H.} \textit{J. H.
Poincar\'{e}} (1854--1912), francoski matematik.}.
Poincar\'{e} je zgradil model, na
osnovi katerega je pokazal, da bi morebitno protislovje geometrije
Lobačevskega pomenilo hkrati protislovje Evklidove geometrije.
Kasneje je prišlo do odkritja tudi drugih neevklidskih geometrij.

Čeprav je bil konec 19. in
v začetku 20. stoletja sistem aksiomov Evklidove geometrije že skoraj popolnoma zgrajen, je prvi
pravilen popolni sistem dal D. Hilbert\footnote{\index{Hilbert, D.}
\textit{D. Hilbert} (1862--1943), nemški matematik.} v svoji znani
knjigi \textit{Osnove geometrije}, objavljeni leta 1899. Zelo podoben
sistem aksiomov uporabljamo v skoraj nespremenjeni
obliki tudi danes.

Vzporedno z raziskovanjem problema petega Evklidovega aksioma in
razvojem neevklidskih geometrij so se v obravnavi geometrije razvile
tudi druge pomembne
metode. Že okoli leta 1637 je R.
Descartes\footnote{\index{Descartes, R.} \textit{R. Descartes}
(1596--1650) francoski matematik.} v svoji knjigi
\textit{Geometrija} pokazal, da lahko vsako točko v ravnini opišemo z
ustreznim parom dveh realnih števil in podobno v prostoru kot
trojico treh realnih števil. To je povezal s pojmom koordinatne
predstavitve odvisnosti ene količine (funkcije) od druge
(spremenljivke), ki je bil znan že prej. Danes takšen
način določanja točk v prostoru po njem imenujemo \pojem{kartezični
koordinatni sistem}. Premice in ravnine potem lahko opisujemo kot
množice rešitev ustreznih linearnih enačb, kjer so neznanke
koordinate točk.

Tako sta se pod vplivom idej F.
Vi\'{e}teja\footnote{\index{Vi\'{e}te, F.} \textit{F. Vi\'{e}te}
(1540--1603), francoski matematik.}, Descartesa in P.
Fermata\footnote{\index{Fermat, P.} \textit{P. Fermat} (1601--1665),
francoski matematik.} začeli razvijati dve zelo pomembni matematični
disciplini - najprej \index{geometrija!analitična}
\pojem{analitična geometrija}, nato še \index{linearna algebra}
\pojem{linearna algebra}, ki predstavljata povezavo med algebro in
geometrijo. Nadaljnji razvoj teh dveh disciplin je omogočal razvoj
\index{geometrija!večdimenzionalna} \pojem{večdimenzionalne
geometrije}, v kateri se lahko obravnavajo prostori, dimenzije, večje
od tri, saj v algebri ni takšnih omejitev, kot jih imamo v
geometrični percepciji prostora. Tako lahko definiramo t. i.
\pojem{politope} -- objekte večdimenzionalnega prostora, ki so analogija
dvodimenzionalnih večkotnikov in tridimenzionalnih poliedrov.

Kasneje je prišlo tudi do odkritja drugih neevklidskih geometrij.
V 19. stoletju se je razvila še t. i. \index{geometrija!projektivna}\pojem{projektivna geometrija}, vendar
njen razvoj ni potekal aksiomatično kot pri hiperbolični
geometriji, ustrezen sistem aksiomov je bil postavljen šele
kasneje. V tej geometriji v ravnini ni premic, ki se ne sekata.

Eden od prvih motivov za začetek razvoja
 projektivne geometrije izvira iz slikarstva oziroma iz želje, da se
 občutek trirazsežnega prostora prenese v ravnino. Že v zelo
 zgodnjem slikarstvu srečamo zelo pomembno lastnost -- da sta
 vzporednici na sliki predstavljeni kot premici, ki se sekata.

 V 15. stoletju so se italijanski umetniki zelo zanimali za
 geometrijo prostora. Teorijo perspektive je prvi obravnaval F.
 Brunellechi\footnote{\index{Brunellechi, F.} \textit{F.
 Brunellechi} (1377--1446), italijanski arhitekt.} leta 1425.
   Njegovo delo sta
nadaljevala L. B. Alberti\footnote{\index{Alberti, L. B.}
  \textit{L. B.
 Alberti} (1404--1472), italijanski matematik in slikar.} in A.
 D\"{u}rer\footnote{\index{D\"{u}rer, A.} \textit{A.
 D\"{u}rer} (1471--1528), nemški slikar.}. Albertijeva knjiga iz
 leta
 1435 predstavlja prvo predstavitev središčne projekcije.

 Za začetek razvoja projektivne geometrije kot matematične
 discipline smatramo obdobje, ko sta
 J. Kepler\footnote{\index{Kepler, J.} \textit{J. Kepler} (1571--1630),
  nemški astronom.}  in G. Desargues\footnote{\index{Desargues, G.}
  \textit{G. Desargues}
 (1591--1661), francoski arhitekt.}
 neodvisno drug od drugega
  vpeljala pojem točk v neskončnosti.
  Kepler je pokazal, da ima parabola dve gorišči, od katerih je eno
   točka v neskončnosti. Desargues
 je leta 1639 pisal: ‘‘Dve vzporednici imata skupni konec na
 neizmerni oddaljenosti.’’ Leta 1636 je napisal knjigo o perspektivi,
 in leta 1639 še o stožnicah. Znameniti \textit{Desarguesov izrek}
 je objavil  leta 1648.

  Z nadaljnjim razvojem projektivne geometrije povezujemo francoske matematike.
  Genialni B. Pascal\footnote{\index{Pascal, B.} \textit{B. Pascal} (1623--1662),
  francoski filozof in matematik.} je
  že kot šestnajstletnik dokazal pomemben izrek o
  stožnicah, ki ga danes po njem imenujemo \textit{Pascalov izrek}.
  Ta izrek, ki je eden izmed osnovnih izrekov
  projektivne geometrije, je bil objavljen leta 1640.
  G. Monge\footnote{\index{Monge, G.} \textit{G. Monge}
  (1746--1818), francoski matematik.}
  je bil med prvimi matematiki, ki ga lahko smatramo za specialista; je
  namreč prvi pravi geometer. \pojem{Opisno geometrijo}je razvil kot
  posebno disciplino. V njegovem raziskovanju v opisni geometriji
  najdemo veliko idej projektivne geometrije.
  Najbolj originalen Mongeov učenec je
  bil J. V. Poncelet\footnote{\index{Poncelet, J. V.}
  \textit{J. V. Poncelet} (1788--1867), francoski matematik.}.
  Čeprav je že Pappus\footnote{\index{Pappus} \textit{Pappus iz Aleksandrije} (3. stol.), starogrški matematik.}
  odkril prve projektivne izreke, jih je Poncelet
  s popolnoma  projektivnim načinom sklepanja dokazal šele v 19. stoletju.
  Leta 1822 je
  Poncelet objavil svoj znani ‘‘Traktat o projektivnih lastnostih likov’’,
  v katerem se pojavljajo vsi pomembni pojmi, karakteristični za
  projektivno
  geometrijo: harmonična četverica, perspektivnost, projektivnost,
  involucija itd. Poncelet je vpeljal premico v neskončnosti za vse
  ravnine, ki so vzporedne dani ravnini. Poncelet in J. D.
  Gergonne\footnote{\index{Gergonne, J. D.} \textit{J. D. Gergonne} (1771--1859), francoski matematik.} sta
  neodvisno
  drug od drugega proučevala dualnost v projektivni geometriji,
  C.~J.~Brianchon\footnote{\index{Brianchon, C. J.} \textit{C. J. Brianchon}
   (1783--1864), francoski matematik.} pa
 je dokazal izrek, ki je dualen Pascalovem izreku.
 M.~Chasles\footnote{\index{Chasles, M.} \textit{M. Chasles} (1793--1880), francoski matematik.} je bil zadnji iz
  velike šole
 francoskih projektivnih  geometrov tistega časa.

 Tipičen predstavnik t. i. čiste geometrije
 (danes bi rekli sintetične geometrije) je bil
 J. Steiner\footnote{\index{Steiner, J.} \textit{J. Steiner}
 (1796--1863),
 švicarski geometer.}.
 Steiner je razvijal projektivno geometrijo zelo sistematično,
  od perspektivnosti do projektivnosti in potem do stožnic.

 Sredi 19. stoletja so primat v razvoju projektivne geometrije
 prevzeli nemški matematiki.  Negovali so sintetični pristop
 h geometriji. Vsi matematiki do tedaj so projektivno geometrijo
  zasnovali na evklidski metrični geometriji -- z dodajanjem
  točk v neskončnosti. Toda C.~G.~C.~Staudt
  \footnote{\index{Staudt, K. G. C.} \textit{C. G. C. Staudt} (1798--1867),
  nemški matematik.}
   je bil prvi, ki jo je poskusil osamosvojiti ter
    zasnovati samo na incidenčnih aksiomih, brez pomoči metrike.
    Tako je prišlo do ukinitve razlike med točkami v neskončnosti in navadnimi
     točkami
    oziroma prehoda z razširjenega evklidskega na projektivni prostor.

  F. Klein\footnote{\index{Klein, F. C.} \textit{F. C. Klein} (1849--1925), nemški matematik.} je leta 1871
   projektivni geometriji postavil algebraične temelje s pomočjo
    t. i. \pojem{homogenih koordinat}, ki sta jih leta 1827 neodvisno drug od drugega,
     odkrila K. W. Feuerbach\footnote{\index{Feuerbach, K. W.} \textit{K. W. Feuerbach}
      (1800--1834), nemški matematik.}
      in A. F. M\"{o}bius\footnote{\index{M\"{o}bius, A. F.} \textit{A. F. M\"{o}bius}
       (1790--1868), nemški matematik.}.
     A. Cayley\footnote{\index{Cayley, A.} \textit{A. Cayley} (1821--1895), angleški matematik.} in
     Klein
     najdeta uporabo projektivne geometrije v drugih neevklidskih geometrijah.
     Odkrila sta model hiperbolične geometrije in modele drugih geometrij
     v projektivni.

   Prva, ki sta popolnoma aksiomatično zasnovala projektivno geometrijo, sta
    bila G. Fano\footnote{\index{Fano, G.} \textit{G. Fano} (1871--1952), italijanski matematik.}
     leta 1892 in M. Pieri\footnote{\index{Pieri, M.} \textit{M. Pieri} (1860--1913), italijanski matematik.}
      leta 1899.

Zaradi svoje relativne enostavnosti je bil razvoj klasične
(sintetične) projektivne geometrije konec 19. stoletja že skoraj
popolnoma zaključen. Njen razvoj se danes nadaljuje v okviru
drugih teorij -- posebej v algebri in algebraični geometriji kot
$n$-razsežna projektivna geometrija.



G. F. B. Riemann\footnote{\index{Riemann, G. F. B.} \textit{G. F.
B. Riemann} (1828--1866), nemški matematik.} je že v svoji knjigi
\textit{O domnevah, ki ležijo v osnovi geometrije} definiral
prostor poljubne dimenzije, ki ni vedno konstantne ukrivljenosti.
Po njem ga danes imenujemo \pojem{Riemannov metrični
prostor}\index{Riemannovi prostori}. Evklidovo geometrijo potem
dobimo kot poseben primer: če je ukrivljenost konstantna in enaka
0; hiperbolično geometrijo dobimo, če izberemo, da je
ukrivljenost konstantna in negativna. Če je ukrivljenost
konstantna in pozitivna, dobimo t. i.
\index{geometrija!eliptična} \pojem{eliptično geometrijo}.
Slednja geometrija je pravzaprav projektivna geometrija, če ji
dodamo metriko. To raziskovanje je bilo hkrati začetek razvoja
nove discipline v matematiki, t. i. \index{geometrija!diferencialna} \pojem{diferencialne geometrije}.

Če razmišljamo o neevklidskih geometrijah, se nam mogoče zdi čudno,
da se v matematiki sploh lahko obravnava več različnih teorij, kot
so npr. evklidska geometrija in hiperbolična geometrija, ki sta v
nasprotju druga z drugo. Za sodobno matematiko je največjega pomena,
da sta obe geometriji določeni s sistemoma aksiomov, ki sta (vsak zase) neprotislovna in popolna. Na vprašanje, katera od teh
dveh geometrij velja, je nesmiselno iskati odgovor v okviru
matematike. To je namreč odvisno od tega, za katere aksiome smo se
odločili. Takšno vprašanje bi bilo enako vprašanju, kateri aksiomi
veljajo. Toda aksiome po definiciji privzamemo brez dokaza. Seveda
lahko zastavimo vprašanje, kakšna je geometrija prostora v fizičnem
smislu in kako jo lahko opišemo z aksiomi.

Za odgovor na to vprašanje je potrebna fizikalna interpretacija
osnovnih geometričnih pojmov. Na primer, premico je najbolj naravno
interpretirati kot svetlobni žarek. V tem smislu se izkaže, da
fizični prostor ni evklidski. Določen ni niti s hiperbolično
geometrijo. S pojavom Einsteinove\footnote{\index{Einstein, A.}
\textit{A. Einstein} (1879--1955), slaven nemški fizik.} teorije
relativinosti v začetku 20. stoletja se je izkazalo, da je v
prostoru vesoljskih razsežnosti bolj ugodno uporabljati neevklidsko
geometrijo s spremenljivo ukrivljenostjo (Riemannovi metrični
prostori!). Lahko rečemo, da je geometrija vesolja lokalno različna
v vsaki točki, odvisno od bližine in velikosti neke mase.
Einsteinova teorija nam tudi pove, da sta prostor in čas medsebojno
povezana in da niti čas (kar je seveda presenetljivo) ne poteka
enako v vsaki točki vesolja. V zvezi z omenjeno povezavo prostora in
časa je pomemben t. i. \pojem{štirirazsežni prostor
Minkowskyega}\footnote{Ta prostor je odkril \index{Minkowsky, H.}
\textit{H. Minkowsky} (1864--1909), nemški matematik.}.

 Že od 20-ih let 20. stoletja in razvoja teorije prapoka vemo,
 da vesolje ni statično in da se širi. Od tega, kakšna
 je njegova usoda, je odvisno, katera geometrija ga globalno najbolje
 opisuje. Toda še vedno ne vemo dokončno, kakšna je oblika
 vesolja niti kakšna je njegova usoda. Niti ne vemo, ali je vesolje
 končno ali neskončno. Kot piše S.
 Hawking\footnote{\index{Hawking, S.} \textit{S. Hawking} (1942),
 angleški fizik.
 Eden najbolj značilnih teoretičnih fizikov našega časa,
 ki je največ vplival na sodobno predstavo o vesolju.}
 v svoji znani popularni knjigi iz leta 1988
 \textit{Kratka zgodovina časa} (\cite{KratkaZgodCasa}), je celo vesolje
 morda končno in neomejeno. Slednje se zdi paradoksalno,
 čeprav si lahko to predstavljamo ako, de si namesto trirazsežnega zamišljamo
 ‘‘dvorazsežno vesolje’’.
 Tako bi bitja tega dvorazsežnega vesolja lahko
 enkrat ugotovila, da njihovo vesolje pravzaprav ni
 ravnina ampak sfera, ki je končna, toda neomejena.
 Sfera je del trirazsežnega prostora. Tako si teoretično
 lahko predstavljamo vesolje kot trirazsežno sfero v
 štirirazsežnem prostoru. Trirazsežna sfera je ena od 3-mnogoterosti.
 V tem primeru bi bila geometrija vesolja globalno eliptična.


Čeprav je samo vprašanje oblike in usode vesolja vprašanje
teoretične fizike in kozmologije, vidimo, kako je sodobna
geometrija (neevklidske geometrije, geometrija večrazsežnih
prostorov itd.) tesno povezana s tem problemom (\cite{Oblika}).
Pomembno je razumeti, da se geometrija (in vsaka druga matematična
disciplina) razvija in obravnava kot abstraktna disciplina, v kateri
so nam fizične interpretacije le inspiracija -- v tem smislu nam
ostajajo le aksiomi in začetni pojmi, na katerih potem izgrajujemo
matematično teorijo.


Na koncu omenimo še enega najpomembnejših geometrov 20.
stoletja H. S. M. Coxeterja\footnote{\index{Coxeter, H. S. M.}
\textit{H. S. M. Coxeter} (1907--2003), kanadski matematik. Eden
največjih geometrov 20. stoletja.}. Coxeter je naprej raziskoval politope
v poljubnih razsežnostih, predvsem pravilne politope. Razen tega se
je veliko ukvarjal z grupami izometrij v hiperbolični geometriji in
z večrazsežno hiperbolično geometrijo.

Potrebno je dodati, da z vsem tem, kar smo
povedali, razvoj geometrije še zdaleč ni končan. Ravno obratno -- nasprotno običajni
predstavi -- geometrija in na splošno matematika se sedaj  razvijata
še hitreje kot kadar koli prej. Tudi danes obstaja v matematiki (v geometriji posebej iz neevklidskih
geometrij) veliko
problemov, ki so še vedno nerešeni.







% DEL 2 - - - - - - - - - - - - - - - - - - - - - - - - - - - - - - - - - - - - - - -
%________________________________________________________________________________
%  AKSIOMI RAVNINSKE EVKLIDSKE GEOMETRIJE
%________________________________________________________________________________


\del{Axioms of Planar Euclidean Geometry} \label{pogAKS}

V nadaljevanju bomo ilustrirali aksiomatično zasnovo ravninske
evklidske geometrije. Navedli bomo začetne pojme in začetne izreke -
aksiome, potem pa izpeljali še nekaj novih pojmov in izrekov.
Omenimo, da smo izbrali ravninske aksiome, ker se bomo v tej knjigi
ukvarjali le z geometrijo evklidske ravnine.

Naj bo $\mathcal{S}$ neprazna množica. Njene elemente imenujemo
\index{točka} \pojem{točke} in jih označujemo z $A, B, C, \ldots$
Določene podmnožice množice $\mathcal{S}$ imenujemo \index{premica}
\pojem{premice} in jih označujemo z $a, b, c, \ldots$ Množico
$\mathcal{S}$ (množico vseh točk) imenujemo tudi \index{ravnina}
\pojem{ravnina}. Razen teh sta začetna pojma tudi dve relaciji na
množici $\mathcal{S}$. Prva je \index{relacija!$\mathcal{B}$}
\pojem{relacija $\mathcal{B}$} in se nanaša na tri točke. Dejstvo,
da so točke $A$, $B$ in $C$ v tej relaciji, bomo označevali z
$\mathcal{B}(A,B,C)$ in brali: Točka $B$ je med točkama $A$ in $C$.
Druga je \index{relacija!skladnosti parov točk} \pojem{relacija
skladnosti parov točk}; dejstvo, da so pari točk $A, B$ in $C, D$ v
tej relaciji, bomo označevali z $(A,B) \cong (C,D)$ in brali: Par
točk $(A,B)$ je skladen s parom točk $(C,D)$.

 S pomočjo omenjenih začetnih pojmov lahko
definiramo tudi naslednje izpeljane pojme:

Če točka $A$ pripada premici $p$ ($A\in p$), oz. premica $p$
vsebuje točko $A$ ($p\ni A$), bomo rekli, da
 točka $A$\index{relacija!leži na premici} \pojem{leži na} premici $p$, oz. da premica $p$ \index{relacija!poteka skozi točko}\pojem{poteka
 skozi} točko $A$.
 Za tri ali
več točk pravimo, da so \index{kolinearne točke}\pojem{kolinearne},
če ležijo na isti premici, sicer so
\index{nekolinearne točke}\pojem{nekolinearne}. Dve
različni premici se \pojem{sekata}, če njun presek (presek dveh podmnožic)
ni prazna množica. Njun presek imenujemo \index{presečišče
dveh premic} \pojem{presečišče} dveh premic. Poljubno neprazno
podmnožico $\Phi$ množice $\mathcal{S}$ ($\Phi\subset\mathcal{S}$) imenujemo \index{lik} \pojem{lik}. Pravimo, da lika $\Phi_1$ in $\Phi_2$ \index{lika!sovpadata}\pojem{sovpadata} (oz. sta \index{lika!identična}\pojem{identična}), če je $\Phi_1=\Phi_2$.

 Sedaj bomo
navedli tudi osnovne trditve - aksiome. Po svoji naravi so
razdeljeni v pet skupin:

\begin{enumerate}
  \item incidence axioms (three axioms),
  \item ordering axioms (four axioms),
  \item congruence axioms (four axioms),
  \item continuity axiom (one axiom),
  \item Playfair's axiom (one axiom).
\end{enumerate}



%________________________________________________________________________________
 \poglavje{Incidence Axioms}
  \label{odd2AKSINC}

 Ker premice kot začetni pojmi predstavljajo določene množice točk,
 lahko za
točke in premice obravnavamo ustrezne relacije med elementi
in množicami: $\in$ in $\ni$ - relaciji imenujemo tudi
\index{relacija!incidencije}\pojem{relaciji incidence}. Aksiomi te
skupine opisujejo ravno osnovne lastnosti teh relacij  (Figure
\ref{sl.aks.2.1.1.pic}):

\vspace*{3mm}

        \baksiom \label{AksI1} For every pair of distinct points $A$ and $B$
        there is exactly one line $p$ such that $A$ and $B$  lie on $p$.
        \eaksiom

        \baksiom \label{AksI2}
        For every line there exist at least two distinct points such that both  lie on.
         \eaksiom

        \baksiom \label{AksI3} There exist three points that do not all lie on any one line.
        \eaksiom

\vspace*{3mm}


\begin{figure}[!htb]
\centering
\input{sl.aks.2.1.1.pic}
\caption{} \label{sl.aks.2.1.1.pic}
\end{figure}



 Iz prvih dveh aksiomov \ref{AksI1} in \ref{AksI2} sledi, da je vsaka premica določena s
 svojima dvema različnima točkama. Zato
premico $p$, ki je določena s točkama $A$ in $B$, imenujemo tudi
premica $AB$.

 Iz prvega aksioma \ref{AksI1} sledi, da je presečišče dveh
 premic, ki se sekata, ena sama točka. Če bi namreč dve premici
 imeli še eno skupno točko, bi po tem aksiomu sovpadali (bili bi identični),
 pri definiciji premic, ki se sekata, pa smo zahtevali, da
 sta različni.
 Dejstvo, da se premici $p$ in $q$ sekata v točki $A$, bomo
 zapisali $p\cap q=\{A\}$ ali krajše $p\cap q=A$ (Figure \ref{sl.aks.2.1.2.pic}).



\begin{figure}[!htb]
\centering
\input{sl.aks.2.1.2.pic}
\caption{} \label{sl.aks.2.1.2.pic}
\end{figure}



Tretji aksiom \ref{AksI3} lahko povemo tudi takole: Obstajajo
vsaj tri  točke, ki so nekolinearne.

 Tako smo izpeljali prve posledice aksiomov incidence;
 zaradi enostavnosti jih nismo izrazili v obliki izrekov. To so skoraj vse posledice, ki izhajajo iz prve skupine aksiomov.
 Zaradi tega je geometrija, ki  temelji le na aksiomih incidence,
 preveč enostavna. V njej bi lahko dokazali le obstoj treh točk in treh
 premic. Torej potrebujemo nove aksiome.



%________________________________________________________________________________
 \poglavje{Ordering Axioms}
 \label{odd2AKSURJ}

Aksiomi v tej skupini opisujejo osnovne karakteristike relacije
$\mathcal{B}$, ki smo jo navedli kot osnovni pojem.
\vspace*{3mm}


        \baksiom \label{AksII1} If $\mathcal{B} (A, B, C)$, then $A$, $B$
         and $C$ are three  distinct collinear points, and also
         $\mathcal{B} (C, B, A)$ (Figure \ref{sl.aks.2.2.1.pic}).
        \eaksiom


\begin{figure}[!htb]
\centering
\input{sl.aks.2.2.1.pic}
\caption{} \label{sl.aks.2.2.1.pic}
\end{figure}


        \baksiom \label{AksII2} If $A$, $B$ and $C$ are three distinct
         collinear points,
          exactly one of the relations holds: $\mathcal{B}(A,B,C)$,
        $\mathcal{B}(A,C,B)$, $\mathcal{B}(C,A,B)$
        (Figure \ref{sl.aks.2.2.2.pic}).
        \eaksiom


\begin{figure}[!htb]
\centering
\input{sl.aks.2.2.2.pic}
\caption{} \label{sl.aks.2.2.2.pic}
\end{figure}



        \baksiom \label{AksII3} Given a pair of distinct points $A$ and $B$ there is a point $C$ on line $AB$, so that
        is $\mathcal{B}(A,B,C)$
        (Figure \ref{sl.aks.2.2.3.pic}).
        \eaksiom

\vspace*{-1mm}

\begin{figure}[!htb]
\centering
\input{sl.aks.2.2.3.pic}
\caption{} \label{sl.aks.2.2.3.pic}
\end{figure}

        \baksiom \label{AksPascheva}\index{aksiom!Paschev}
        (Pasch's\footnote{\index{Pasch, M.}
         \textit{M. Pasch}
        (1843--1930), nemški matematik, ki je vpeljal pojem urejenosti točk
        v svojih ‘‘Predavanjih o novejši geometriji’’ iz leta 1882.  Te
        aksiome sta kasneje dopolnila italijanski matematik \index{Peano, G.} \textit{G. Peano}
        (1858--1932), v ‘‘Načelih geometrije’’, nato pa še nemški matematik
        \index{Hilbert, D.}\textit{D. Hilbert} (1862--1943) v svoji znani knjigi
         ‘‘Osnove geometrije’’ iz leta
        1899.} axiom)
        Let $A$, $B$ and $C$ be three noncollinear points and $l$ be a line that does not contain point $A$.
        If there is a point $P$ on $l$ that is $\mathcal{B}(B,P,C)$ then either $l$ contains a point $Q$ that is  $\mathcal{B}(A,Q,C)$ or $l$ contains a point $R$ that is $\mathcal{B}(A,R,B)$ (Figure \ref{sl.aks.2.2.4.pic}).
        \eaksiom


\begin{figure}[!htb]
\centering
\input{sl.aks.2.2.4.pic}
\caption{} \label{sl.aks.2.2.4.pic}
\end{figure}


V prejšnjem aksiomu nismo posebej poudarili, da premica $l$ leži v ravnini $ABC$, ker gradimo ravninsko evklidsko geometrijo, kjer vse točke ležijo v isti ravnini.

 Na tem mestu ne bomo dokazali vseh posledic aksiomov urejenosti.
Formalna izpeljava vseh dejstev namreč ni tako enostavna in bi vzela veliko prostora. Večino dokazov lahko bralec poišče v \cite{Lucic}.

Dokažimo prvo posledico aksiomov urejenosti.


        \bizrek \label{izrekAksUrACB}
        Given a pair of distinct points $A$ and $B$ there is a point $C$, so that
        is $\mathcal{B}(A,C,B)$.
        \eizrek


\begin{figure}[!htb]
\centering
\input{sl.aks.2.2.5.pic}
\caption{} \label{sl.aks.2.2.5.pic}
\end{figure}


\textbf{\textit{Proof.}} Po aksiomu \ref{AksI1} obstaja natanko ena premica, ki poteka skozi točki $A$ in $B$ - označimo jo z $AB$.
Po aksiomu \ref{AksI3} obstajajo vsaj tri nekolinearne točke.
Obstaja torej vsaj ena točka izven premice $AB$ - označimo jo z $D$
  (Figure \ref{sl.aks.2.2.5.pic}). Naprej po
aksiomu \ref{AksII3} obstaja  takšna točka $E$, da velja
$\mathcal{B}(B,D,E)$, nato pa še takšna točka $F$, da velja
$\mathcal{B}(A,E,F)$. $A$, $B$ in $E$ so nekolinearne točke,
saj bi sicer točka $D$ ležala na premici $AB$ (aksiom \ref{AksI1}).
Premica $FD$ ne poteka skozi točko $A$, ker bi bile po aksiomu \ref{AksI1} točke $F$, $D$, $A$ in $E$
kolinearne, z njimi pa tudi točka $B$. To pa ni možno, ker bi iz tega sledilo,  da točka $D$ leži na premici $AB$.
Uporabimo sedaj Paschev aksiom \ref{AksPascheva} na
točkah $A$, $B$ in $E$ ter premici $FD$. Premica $FD$ namreč seka premico $EB$ v takšni točki $D$,
da je $\mathcal{B}(B,D,E)$, zato seka bodisi premico $AE$ v takšni točki $F$, da je $\mathcal{B}(A,F,E)$ bodisi
premico $AB$ v takšni točki $C$, da je $\mathcal{B}(A,C,B)$. Ker je že $\mathcal{B}(A,E,F)$, po aksiomu \ref{AksII2} ne more biti tudi $\mathcal{B}(A,F,E)$. Torej premica $FD$ seka
premico $AB$ v taki točki $C$, za katero velja $\mathcal{B}(A,C,B)$.
\kdokaz

Relacija $\mathcal{B}$ in aksiomi urejenosti, ki se nanašajo nanjo, nam omogočajo definiranje novih pojmov.


 Naj bosta $A$ in $B$ poljubni različni točki.
  \index{daljica!odprta}\pojem{Odprta daljica} $AB$ z oznako $(AB)$ je množica vseh točk
  $X$, za
  katere velja $\mathcal{B}(A,X,C)$ (Figure \ref{sl.aks.2.2.6.pic}).

\begin{figure}[!htb]
\centering
\input{sl.aks.2.2.6.pic}
\caption{} \label{sl.aks.2.2.6.pic}
\end{figure}


  Če odprti daljici $AB$
  dodamo točki $A$ in $B$, dobimo \index{daljica}\pojem{daljico}
  (ali \index{daljica!zaprta} \pojem{zaprto daljico}) $AB$,
  ki jo označimo tudi z $[AB]$.
  Točki $A$ in $B$
  sta njeni \index{krajišče daljice}\pojem{krajišči}, ostale njene točke pa so \pojem{notranje točke} te daljice (Figure \ref{sl.aks.2.2.6.pic}). Še bolj formalno: daljica (oz. zaprta daljica) je unija odprte daljice in množice $\{A,B\}$ oz. $[AB]=(AB)\cup \{A,B\}$.

Na podoben način definiramo tudi
 \index{daljica!polodprta} \pojem{polodprto daljico}: $(AB]=(AB)\cup \{B\}$, oz.   $[AB)=(AB)\cup \{A\}$ (Figure \ref{sl.aks.2.2.6a.pic}).


\begin{figure}[!htb]
\centering
\input{sl.aks.2.2.6a.pic}
\caption{} \label{sl.aks.2.2.6a.pic}
\end{figure}

  Iz aksioma \ref{AksII1} neposrednono sledi, da sta daljici $AB$ in $BA$ isti. Iz istega aksioma sledi tudi, da je daljica $AB$ podmnožica premice $AB$. Zato pravimo, da daljica $AB$ \pojem{leži na premici} $AB$,
  premico $AB$
  pa imenujemo \index{nosilka!daljice}  \pojem{nosilka daljice} $AB$  (Figure \ref{sl.aks.2.2.6b.pic}).


\begin{figure}[!htb]
\centering
\input{sl.aks.2.2.6b.pic}
\caption{} \label{sl.aks.2.2.6b.pic}
\end{figure}

Po izreku \ref{izrekAksUrACB} ima daljica $AB$ razen svojih krajišč $A$ in $B$ vsaj še eno točko $C_1$. Na ta način lahko dobimo neskončno zaporedje točk $C_1$, $C_2$, ..., za katere velja $\mathcal{B}(A, C_n, C_{n-1})$ ($n\in \{2,3,\cdots\}$)  (Figure \ref{sl.aks.2.2.6c.pic}). Na tem mestu ne bomo formalno dokazovali dejstva, da so vse točke iz zaporedja različne in vse ležijo na daljici $AB$. Iz te trditve pa sledi, da ima vsaka daljica (posledično tudi premica) neskončno mnogo točk.

\begin{figure}[!htb]
\centering
\input{sl.aks.2.2.6c.pic}
\caption{} \label{sl.aks.2.2.6c.pic}
\end{figure}

Definirajmo še relacijo $\mathcal{B}$, ki se nanaša na več kot tri kolinearne točke. Pravimo, da je $\mathcal{B}(A_1,A_2,\ldots,A_n)$ ($n\in\{4,5,\ldots\}$), če za vsak $k\in\{1,2,\ldots,n-2\}$ velja $\mathcal{B}(A_k,A_{k+1},A_{k+2})$ (Figure \ref{sl.aks.2.2.6d.pic}).


\begin{figure}[!htb]
\centering
\input{sl.aks.2.2.6d.pic}
\caption{} \label{sl.aks.2.2.6d.pic}
\end{figure}

 Naj bo $S$ točka, ki leži na premici $p$. Na množici $p\setminus \{S\}$ (vseh točk premice $p$ brez točke $S$) definirajmo dve relaciji.
Pravimo, da sta točki $A$ in $B$ ($A,B\in p\setminus \{S\}$) \index{relacija!na različnih straneh točke} \pojem{na različnih straneh  točke} $S$ (kar označimo z $A,B\div S$), če je $B(A,S,B)$, sicer sta  točki $A$ in $B$ ($A,B\in p\setminus \{S\}$) \index{relacija!na isti strani točke} \pojem{na isti strani  točke} $S$ (kar označimo z $A,B\ddot{-} S$). Torej za točki $A,B\in p\setminus \{S\}$ velja $A,B\ddot{-} S$, če ni $A,B\div S$  (Figure \ref{sl.aks.2.2.7.pic}).


\begin{figure}[!htb]
\centering
\input{sl.aks.2.2.7.pic}
\caption{} \label{sl.aks.2.2.7.pic}
\end{figure}


Naj bosta $A$ in $B$ različni točki. Množico vseh takšnih točk $X$, za katere je $B,X\ddot{-} A$ vključno s točko $A$, imenujemo \index{poltrak}\pojem{poltrak} $AB$ z \pojem{začetno točko} ali \pojem{izhodiščem} $A$. Premica $AB$ je
\index{nosilka!poltraka}  \pojem{nosilka poltraka} $AB$ (Figure \ref{sl.aks.2.2.8.pic}).


\begin{figure}[!htb]
\centering
\input{sl.aks.2.2.8.pic}
\caption{} \label{sl.aks.2.2.8.pic}
\end{figure}

Že iz same definicije sledi, da je poltrak podmnožica svoje nosilke oz. da leži na svoji nosilki. Iz relacije $B,X\ddot{-} A$ sledi, da so $B$, $X$ in $A$ kolinearne točke, zato točka $X$ leži na premici $AB$.

Drugih pomembnih lastnosti daljice in poltraka, ki jih bomo kasneje uporabljali, na tem mestu ne bomo dokazovali. Povejmo nekaj teh lastnosti.

If $C$ is the an interior point of the line segment $AB$, then that line segment can be expressed as the union
the line segments $AC$ and $CB$.

            \bizrek \label{izrekAksIIDaljica}
            If $C$ is an interior point of the line segment $AB$, then that line segment can be expressed as the union
            the line segments $AC$ and $CB$ (Figure \ref{sl.aks.2.2.9.pic}).
            \eizrek

\begin{figure}[!htb]
\centering
\input{sl.aks.2.2.9.pic}
\caption{} \label{sl.aks.2.2.9.pic}
\end{figure}


            \bizrek \label{izrekAksIIPoltrak}
            Each point lying on the line determines exactly
            two rays on it. The union of these rays is equal to that line  (Figure \ref{sl.aks.2.2.9.pic}).
            \eizrek

Dokaz prejšnjega izreka temelji na dejstvu, da je relacija $\ddot{-} A$ ekvivalenčna relacija, ki ima dva razreda. Vsak od razredov je ustrezen odprti poltrak.

Poltraka iz prejšnjega izreka, ki sta določena z isto začetno točko na premici, imenujemo \index{poltrak!komplementarni}\pojem{komplementarna poltraka}.

Pojma daljica in poltrak nam omogočata definiranje novih pojmov.

Naj bodo $A_1$, $A_2$, ... $A_n$ takšne točke v ravnini, da nobene tri v zaporedju niso kolinearne. Unijo daljic $A_1A_2$, $A_2A_3$,... $A_{n-1}A_n$ imenujemo \index{lomljenka} \pojem{lomljenka} $A_1A_2\cdots A_n$ ali \index{poligonska
črta}\pojem{poligonska črta} $A_1A_2\cdots A_n$ (Figure \ref{sl.aks.2.2.10.pic}). Točke  $A_1$, $A_2$, ... $A_n$ so \index{oglišče!lomljenke} \pojem{oglišča lomljenke},
daljice $A_1A_2$, $A_2A_3$,... $A_{n-1}A_n$ pa \index{stranica!lomljenke} \pojem{stranice lomljenke}. Stranici lomljenke s skupnim ogliščem sta \index{sosednji stranici!lomljenke} \pojem{sosednji stranici lomljenke}.


\begin{figure}[!htb]
\centering
\input{sl.aks.2.2.10.pic}
\caption{} \label{sl.aks.2.2.10.pic}
\end{figure}

Če stranice lomljenke nimajo skupnih točk, razen sosednjih stranic, ki imata skupno oglišče, takšno lomljenko imenujemo \index{lomljenka!enostavna} \pojem{enostavna lomljenka} (Figure \ref{sl.aks.2.2.10a.pic}).

\begin{figure}[!htb]
\centering
\input{sl.aks.2.2.10a.pic}
\caption{} \label{sl.aks.2.2.10a.pic}
\end{figure}

Lomljenka $A_1A_2\cdots A_nA_{n+1}$, pri kateri je $A_{n+1}=A_1$ in so $A_n$, $A_1$ in $A_2$ nekolinearne točke, se imenuje \index{lomljenka!sklenjena} \pojem{sklenjena lomljenka} $A_1A_2\cdots A_n$ (Figure \ref{sl.aks.2.2.10b.pic}).

\begin{figure}[!htb]
\centering
\input{sl.aks.2.2.10b.pic}
\caption{} \label{sl.aks.2.2.10b.pic}
\end{figure}

Posebej nas bodo zanimale \pojem{enostavne sklenjene lomljenke} (Figure \ref{sl.aks.2.2.10b.pic}).

Naj bosta $p$ in $q$ dva poltraka s skupnim izhodiščem $O$ (Figure \ref{sl.aks.2.2.10c.pic}). Unijo teh dveh poltrakov imenujemo \index{kotna lomljenka} \pojem{kotna lomljenka} $pq$ (ali $pOq$).

\begin{figure}[!htb]
\centering
\input{sl.aks.2.2.10c.pic}
\caption{} \label{sl.aks.2.2.10c.pic}
\end{figure}


Za nek lik $\Phi$ pravimo, da je \index{lik!konveksen}\pojem{konveksen}, če je za poljubni njegovi točki $A,B\in \Phi$ daljica $AB$ podmnožica tega lika oz. če velja naslednje (Figure \ref{sl.aks.2.2.10d.pic}):
 $$(\forall A)(\forall B)\hspace*{1mm} (A,B\in \Phi \Rightarrow [AB]\subseteq \Phi).$$
Za lik, ki ni konveksen, pravimo, da je \index{lik!nekonveksen}\pojem{nekonveksen} (Figure \ref{sl.aks.2.2.10d.pic}).


\begin{figure}[!htb]
\centering
\input{sl.aks.2.2.10d.pic}
\caption{} \label{sl.aks.2.2.10d.pic}
\end{figure}

Neposredno iz definicije sledi, da je premica konveksen lik. Kot posledico aksiomov te skupine je mogoče dokazati, da sta tudi daljica in poltrak konveksna lika.


Za nek lik $\Phi$ pravimo, da je
\index{lik!povezan}\pojem{povezan lik}, če za vsaki njegovi točki $A,B\in \Phi$ obstaja lomljenka $AT_1T_2\cdots T_nB$, ki je podmnožica tega lika oz. če velja naslednje (Figure \ref{sl.aks.2.2.10e.pic}):
 $$(\forall A\in \Phi)(\forall B\in \Phi)(\exists T_1,T_2,\cdots , T_n)\hspace*{1mm}  AT_1T_2\cdots T_nB\subseteq \Phi.$$


\begin{figure}[!htb]
\centering
\input{sl.aks.2.2.10e.pic}
\caption{} \label{sl.aks.2.2.10e.pic}
\end{figure}

 Za lik, ki ni povezan, pravimo, da je \index{lik!nepovezan}\pojem{nepovezan}.

 Jasno je, da je vsak konveksen lik tudi povezan. Za lomljenko je dovolj vzeti kar daljico $AB$. Obratno seveda ne velja. Obstajajo liki, ki so povezani, niso pa konveksni, kar bomo ugotovili kasneje.



Sedaj bomo definirali dve relaciji, ki sta analogni z relacijama $\ddot{-} S$ in $\div S$.
 Naj bo $p$ premica, ki leži v ravnini $\alpha$ (ker aksiomatsko gradimo le evklidsko geometrijo ravnine, so pravzaprav vse točke, ki za nas obstajajo, v tej ravnini). Na množici $\alpha\setminus p$ (vseh točk razen točk premice $p$) definirajmo dve relaciji.
Pravimo, da sta točki $A$ in $B$ ($A,B\in \alpha\setminus p$) \index{relacija!na različnih bregovih premice} \pojem{na različnih bregovih  premice} $p$ (kar označimo z $A,B\div p$), če ima daljica $AB$ s premico $p$ skupno točko, sicer sta  točki $A$ in $B$ ($A,B\in \alpha\setminus p$) \index{relacija!na istem bregu premice} \pojem{na istem bregu  premice} $p$ (kar označimo z $A,B\ddot{-} p$). Torej za točki $A,B\in \alpha\setminus p$ je $A,B\ddot{-} p$, če ni $A,B\div p$ (Figure \ref{sl.aks.2.2.11.pic}).

\begin{figure}[!htb]
\centering
\input{sl.aks.2.2.11.pic}
\caption{} \label{sl.aks.2.2.11.pic}
\end{figure}


Naj bo $A$ točka, ki ne leži na premici $p$. Množico vseh takšnih točk $X$, za katere je $A,X\ddot{-} p$, imenujemo
\index{polravnina!odprta}\pojem{odprta polravnina} $pA$. Unija odprte polravnine $pA$ in premice $p$ je \index{polravnina!zaprta}\pojem{zaprta polravnina} oz. kar
\index{polravnina}\pojem{polravnina} $pA$. Premica $p$ je \index{rob!polravnine} \pojem{rob} te polravnine (Figure \ref{sl.aks.2.2.11a.pic}). Če točki $B$ in $C$ ležita na robu $p$ polravnine $pA$, bomo to ravnino imenovali tudi polravnina $BCA$. Razen tega bomo polravnine označevali tudi z grškimi črkami $\alpha$, $\beta$, $\gamma$,...

\begin{figure}[!htb]
\centering
\input{sl.aks.2.2.11a.pic}
\caption{} \label{sl.aks.2.2.11a.pic}
\end{figure}

Podobno kot pri poltraku je mogoče dokazati (kot posledico aksiomov te skupine), da vsaka premica $p$ v ravnini določa dve polravnini $\alpha$ in $\alpha'$, ki imata premico $p$ za rob (Figure \ref{sl.aks.2.2.11a.pic}). Pravimo, da sta si v tem primeru $\alpha$ in $\alpha'$ \index{polravnina!komplementarna}\pojem{komplementarni polravnini}.
Izkaže se, da je unija dveh komplementarnih polravnin cela ravnina. Podobno kot pri poltraku, tudi dokaz omenjenih trditev temelji na dejstvu, da je relacija $\ddot{-} p$ ekvivalenčna relacija z dvema razredoma. Vsak od razredov je ustrezna odprta polravnina.

Naj bo $pq$ oz. $pOq$ kotna lomljenka. Definirajmo novo relacijo na množici vseh točk ravnine razen točk, ki ležijo na lomljenki. Pravimo, da sta točki $A$ in $B$ na isti strani kotne lomljenke $pq$ (kar označimo z $A,B\ddot{-} pq$), če obstaja lomljenka $AT_1T_2\cdots T_nB$, ki kotne lomljenke $pq$ ne seka oz. z njo nima skupnih točk (Figure \ref{sl.aks.2.2.12.pic}).


\begin{figure}[!htb]
\centering
\input{sl.aks.2.2.12.pic}
\caption{} \label{sl.aks.2.2.12.pic}
\end{figure}

 Tudi relacija $\ddot{-} pq$ je ekvivalenčna relacija, ki ima dva razreda. Unijo vsakega od teh dveh razredov s kotno lomljenko $pq$ imenujemo
 \index{kot}\pojem{kot} $pq$, ki ga označimo z $\angle pq$, oz. $\angle pOq$.
  Kotna lomljenka torej določa dva kota. Dilemo, za katerega od kotov gre pri oznaki  $\angle pOq$, bomo kmalu odpravili.
 Poltraka $p$ in $q$ sta
\index{krak!kota}\pojem{kraka kota} in točka $O$ \index{vrh kota}\pojem{vrh kota}.
Če sta $P\in p$ in $Q\in q$ točki, ki ležita na krakih kota $pOq$ in se razlikujeta od njegovega vrha $O$, bomo kot imenovali tudi kot $POQ$ in označili z $\angle POQ$ (Figure \ref{sl.aks.2.2.12a.pic}). Če vemo, za kateri kot gre, ga bomo označevali kar z njegovim vrhom: $\angle O$. Razen tega bomo kote označevali tudi z grškimi črkami $\alpha$, $\beta$, $\gamma$,...

Vse točke kota $pOq$, ki ne ležijo na nobenem od obeh krakov $p$ in $q$, imenujemo \index{notranje točke!kota} \pojem{notranje točke kota}, množico vseh teh točk pa \index{notranjost!kota}\pojem{notranjost kota}. Jasno je, da gre za točke ustreznega razreda, ki jih določa relacija $\ddot{-} pq$. Točke drugega razreda so \index{zunanje!točke kota}\pojem{zunanje točke kota}, cel razred pa \index{zunanjost!kota}\pojem{zunanjost kota}. Točke, ki ležijo na krakih, oz. na kotni lomljenki $pOq$, ki kot $pOq$ določa, so \index{robne točke!kota}\pojem{robne točke kota}, cela lomljenka pa je \index{rob!kota}\pojem{rob kota}.

\begin{figure}[!htb]
\centering
\input{sl.aks.2.2.12a.pic}
\caption{} \label{sl.aks.2.2.12a.pic}
\end{figure}

Če sta kraka kota komplementarna poltraka, takšen kot imenujemo
\index{kot!iztegnjeni}\pojem{iztegnjeni kot} (Figure \ref{sl.aks.2.2.12b.pic}). Kot množica točk je ta kot v bistvu enak polravnini z robom, ki je nosilka obeh krakov.


\begin{figure}[!htb]
\centering
\input{sl.aks.2.2.12b.pic}
\caption{} \label{sl.aks.2.2.12b.pic}
\end{figure}

Če kotna lomljenka $pOq$ ne določa iztegnjenenega kota oz. ni enaka premici, se izkaže, da $pOq$ določa dva kota, ki predstavljata konveksen in nekonveknen lik - imenujemo ju \index{kot!konveksen}\pojem{konveksen (izbočeni) kot} in \index{kot!nekonveksen}\pojem{nekonveksen (vdrti) kot}. Formalen dokaz tega dejstva bomo na tem mestu izpustili. Če ne poudarimo drugače, bomo pod oznako $\angle pOq$ (oz. $\angle pq$ ali $\angle POQ$) vedno mislili na konveksen kot (Figure \ref{sl.aks.2.2.12b.pic}). V tem smislu je že iz definicije kota jasno, da (konveksna) kota $pOq$ in $qOp$ predstavljata isti kot.

Kota $pOq$ in $qOr$, ki imata skupen krak $q$, ki je hkrati njun presek (kot množice točk), sta \index{kot!sosednji}\pojem{sosednja kota} (Figure \ref{sl.aks.2.2.12c.pic}).  Če sta pri tem poltraka $p$ in $r$ še komplementarna (oz. določata iztegnjeni kot), pravimo, da sta $pOq$ in $qOr$ \index{kota!sokota}\pojem{sokota}  (Figure \ref{sl.aks.2.2.12c.pic}).

\begin{figure}[!htb]
\centering
\input{sl.aks.2.2.12c.pic}
\caption{} \label{sl.aks.2.2.12c.pic}
\end{figure}

Kota $pOq$ in $rOs$ sta \index{kota!sovršna}\pojem{sovršna kota}, če sta $p$ in $r$ oz. $q$ in $s$ para komplementarnih (dopolnilnih) poltrakov (Figure \ref{sl.aks.2.2.12d.pic}).

\begin{figure}[!htb]
\centering
\input{sl.aks.2.2.12d.pic}
\caption{} \label{sl.aks.2.2.12d.pic}
\end{figure}

 Naj bo $A_1A_2\cdots A_n$ ($n\in \{3,4,5,\cdots\}$) enostavna sklenjena lomljenka.
 Podobno kot pri kotni lomljenki, lahko na množici vseh točk ravnine razen točk, ki ležijo na lomljenki $A_1A_2\cdots A_n$, definiramo naslednjo relacijo: pravimo, da sta točki $B$ in $C$ na isti strani enostavne sklenjene lomljenke $A_1A_2\cdots A_n$ (kar označimo z $B,C\ddot{-} A_1A_2\cdots A_n$), če obstaja lomljenka $BT_1T_2\cdots T_nC$, ki enostavne sklenjene lomljenke $A_1A_2\cdots A_n$ ne seka oz. z njo nima skupnih točk (Figure \ref{sl.aks.2.2.13.pic}).


\begin{figure}[!htb]
\centering
\input{sl.aks.2.2.13.pic}
\caption{} \label{sl.aks.2.2.13.pic}
\end{figure}

 Tudi v tem primeru je mogoče dokazati, da gre za ekvivalenčno relacijo z dvema razredoma - pri tem za en razred obstaja premica, ki cela leži v njem, za drugi pa takšne premice ni. Unijo tistega razreda, ki ne vsebuje nobene premice (intuitivno - tistega, ki je omejen) in enostavne sklenjene lomljenke, $A_1A_2\cdots A_n$ imenujemo
\index{večkotnik}\pojem{večkotnik} $A_1A_2\cdots A_n$ ali
\index{$n$-kotnik}\pojem{$n$-kotnik} $A_1A_2\cdots A_n$ (Figure \ref{sl.aks.2.2.13a.pic}). Vse točke omenjenega razreda, ki ne vsebuje nobene premice, imenujemo \index{notranje točke!večkotnika} \pojem{notranje točke večkotnika}, cel razred pa \index{notranjost!večkotnika}\pojem{notranjost večkotnika}. Točke drugega razreda so \index{zunanje!točke večkotnika}\pojem{zunanje točke večkotnika}, cel razred pa \index{zunanjost!večkotnika}\pojem{zunanjost večkotnika}. Točke, ki ležijo na lomljenki $A_1A_2\cdots A_n$, ki določa večkotnik, so \index{robne točke!večkotnika}\pojem{robne točke večkotnika}, cela lomljenka pa je \index{rob!večkotnika}\pojem{rob večkotnika}.


\begin{figure}[!htb]
\centering
\input{sl.aks.2.2.13a.pic}
\caption{} \label{sl.aks.2.2.13a.pic}
\end{figure}


Za notranje točke večkotnika velja izrek, ki je pravzaprav ekvivalenten njihovi definiciji. Izrek bomo podali brez dokaza.

            \bizrek
            A point $N$ is an interior point of a polygon if and only
            if any ray from the point $N$, that does not contain the vertices of the
            polygon, intersects an odd number of sides of the polygon (Figure \ref{sl.aks.2.2.13b.pic}).
            \eizrek

\begin{figure}[!htb]
\centering
\input{sl.aks.2.2.13b.pic}
\caption{} \label{sl.aks.2.2.13b.pic}
\end{figure}

Točke $A_1$, $A_2$,..., $A_n$ (oglišča lomljenke) so \index{oglišče!večkotnika}\pojem{oglišča večkotnika}, daljice $A_1A_2$, $A_2A_3$, ... $A_{n-1}A_n$, $A_nA_1$ pa
\index{stranica!večkotnika}\pojem{stranice večkotnika}. Premice $A_1A_2$, $A_2A_3$, ... $A_{n-1}A_n$, $A_nA_1$ so \index{nosilka!stranice}\pojem{nosilke stranic} $A_1A_2$, $A_2A_3$, ... $A_{n-1}A_n$, $A_nA_1$. Stranici, ki vsebujeta skupno oglišče, sta \index{stranica!sosednja}\pojem{sosednji stranici}, sicer sta stranici  \index{stranica!nesosednja}\pojem{nesosednji}. Če sta oglišči hkrati krajišči iste stranice, pravimo, da sta oglišči \index{oglišče!sosednje}\pojem{sosednji}, sicer sta oglišči \index{oglišče!nesosednje}\pojem{nesosednji}.
Iz definicije je jasno, da ima vsako oglišče natanko dve sosednji oglišči. Prav tako ima vsaka starnica natanko dve sosednji stranici.
 Daljica, ki ju določata nesosednji oglišči, se imenuje
\index{diagonala!večkotnika}\pojem{diagonala večkotnika} (Figure \ref{sl.aks.2.2.13c.pic}).

\begin{figure}[!htb]
\centering
\input{sl.aks.2.2.13c.pic}
\caption{} \label{sl.aks.2.2.13c.pic}
\end{figure}

O diagonalah večkotnika govori naslednji izrek.

            \bizrek
            The number of diagonals of an $n$-gon is $\frac{n(n-3)}{2}$.
            \eizrek

Dokaz tega izreka bomo podali v razdelku \ref{odd3Helly}, kjer bomo posebej obravnavali kombinatorne lastnosti množic točk v ravnini.

Definirajmo še kote večkotnika. Naj bo $O$ poljubno oglišče večkotnika ter $P$ in $Q$ njegovi sosednji oglišči. Poltraka $OP$ in $OQ$ označimo s $p$ in $q$. V tem primeru kotna lomljenka $pOq$ določa dva kota. Tisti kot, za katerega velja, da vsak poltrak z začetno točko $O$, ki pripada temu kotu in ne vsebuje drugih oglišč večkotnika, seka rob večkotnika razen v točki $O$ še v lihem številu točk, imenujemo \index{kot!notranji večkotnika}\pojem{notranji kot večkotnika} ali krajše \index{kot!notranji}\pojem{kot večkotnika} ob oglišču $O$ (Figure \ref{sl.aks.2.2.13d.pic}). Če je notranji kot večkotnika konveksen, njegov sokot imenujemo \index{kot!zunanji večkotnika}\pojem{zunanji kot večkotnika} (Figure \ref{sl.aks.2.2.13d.pic}).
  Kota večkotnika sta
  \index{kot!sosednji}\pojem{sosednja kota}, če sta njuna vrha sosednji oglišči večkotnika, sicer sta kota \index{kot!nesosednji}\pojem{nesosednja}.


\begin{figure}[!htb]
\centering
\input{sl.aks.2.2.13d.pic}
\caption{} \label{sl.aks.2.2.13d.pic}
\end{figure}

Najbolj enostaven $n$-kotnik in hkrati eden najpogosteje uporabljenih likov v geometriji ravnine dobimo v primeru $n=3$ - \index{trikotnik}\pojem{trikotnik}.
V primeru trikotnika $ABC$ (označevali ga bomo z $\triangle ABC$) so torej točke $A$, $B$ in $C$ njegova \pojem{oglišča}, daljice $AB$, $BC$ in $CA$ pa njegove
\index{stranica!trikotnika}\pojem{stranice} (Figure \ref{sl.aks.2.2.14.pic}). Očitno sta vsaki dve stranici trikotnika sosednji. Prav tako sta sosednji tudi vsaki dve oglišči. Trikotnik torej nima nobene diagonale. Pravimo, da sta oglišče $A$ ($B$ in $C$) oz. kot $BAC$ ($ABC$ in $ACB$) \index{oglišče!nasprotno trikotnika}\pojem{nasprotno oglišče} oz. \index{kot!nasprotni trikotnika}\pojem{nasprotni kot} stranice $BC$ ($AC$ in $AB$) trikotnika $ABC$. In tudi stranica $BC$ ($AC$ in $AB$) je \index{stranica!nasprotna trikotnika}\pojem{nasprotna stranica} oglišča $A$ ($B$ in $C$) oz. kota $BAC$ ($ABC$ in $ACB$) trikotnika $ABC$. Kote (notranje) trikotnika $ABC$ ob ogliščih $A$, $B$ in $C$ pogosto označimo z $\alpha$, $\beta$ in $\gamma$, ustrezne zunanje pa z $\alpha_1$, $\beta_1$ in $\gamma_1$.

\begin{figure}[!htb]
\centering
\input{sl.aks.2.2.14.pic}
\caption{} \label{sl.aks.2.2.14.pic}
\end{figure}

Trikotnik $ABC$ lahko definiramo tudi kot presek polravnin $ABC$, $ACB$ in $BCA$. Na tem mestu ekvivalentnosti dveh definicij ne bomo dokazovali.

 Paschev aksiom v terminih trikotnikov lahko sedaj izrazimo v krajši obliki:

If a line in the plane of a triangle intersects one of its sides
and does not pass through any of its vertices, then intersects exactly one more side of this triangle



             \bizrek \label{PaschIzrek}
           If a line, not passing through any vertex of a triangle, intersects one side of the triangle
           then the line intersects exactly one more side of this triangle (Figure \ref{sl.aks.2.2.14a.pic}).
             \eizrek

\begin{figure}[!htb]
\centering
\input{sl.aks.2.2.14a.pic}
\caption{} \label{sl.aks.2.2.14a.pic}
\end{figure}

  V primeru $n=4$ za $n$-kotnik dobimo
\index{štirikotnik}\pojem{štirikotnik}. Ker ima vsako oglišče štirikotnika natanko eno nesosednje oglišče, bomo to oglišče imenovali tudi \index{oglišče!nasprotno štirikotnika}\pojem{nasprotno oglišče} štirikotnika. Podobno bosta nesosednji stranici \index{stranica!nasprotna štirikotnika}\pojem{nasprotni stranici} štirikotnika. Diagonalo štirikotnika torej določata nasprotni oglišči.
 Iz same definicije je jasno, da ima štirikotnik dve diagonali\index{diagonala!štirikotnika} (Figure \ref{sl.aks.2.2.14b.pic}).

\begin{figure}[!htb]
\centering
\input{sl.aks.2.2.14b.pic}
\caption{} \label{sl.aks.2.2.14b.pic}
\end{figure}

Za nesosednja kota štirikotnika pravimo tudi, da sta
 \index{kot!nasprotni štirikotnika}\pojem{nasprotna kota} štirikotnika. Kote (notranje) štirikotnika $ABCD$ ob ogliščih $A$, $B$, $C$ in $D$ ponavadi označimo z $\alpha$, $\beta$, $\gamma$ in $\delta$, ustrezne zunanje (tiste, ki obstajajo) pa z $\alpha_1$, $\beta_1$, $\gamma_1$ in $\delta_1$ (Figure \ref{sl.aks.2.2.14c.pic}).

\begin{figure}[!htb]
\centering
\input{sl.aks.2.2.14c.pic}
\caption{} \label{sl.aks.2.2.14c.pic}
\end{figure}



Kot posledico aksiomov urejenosti bomo na koncu tega razdelka vpeljali še pojma orientacije trikotnika in orientacije kota.

Trikotnik $ABC$, pri katerem so oglišča urejena trojica $(A,B,C)$, imenujemo
  \index{orientacija!trikotnika} \pojem{orientirani trikotnik}.
Pravimo, da sta orientirana trikotnika $ABC$ in $BCA'$ \pojem{iste orientacije}, če velja $A,A'\ddot{-} BC$, in nasprotne orientacije, če je $A,A'\div BC$
 (Figure \ref{sl.aks.2.2.15.pic}).

\begin{figure}[!htb]
\centering
\input{sl.aks.2.2.15.pic}
\caption{} \label{sl.aks.2.2.15.pic}
\end{figure}

Ko govorimo o orientaciji dveh trikotnikov, bomo v bodoče vedno mislili na orientirana trikotnika (besedo orientirana bomo pogosto izpustili). Trikotnika $ABC$ in $A'B'C'$ sta \pojem{iste orientacije} oz. sta \pojem{enako orientirana}, če obstaja takšno zaporedje trikotnikov: $\triangle ABC=\triangle P_1P_2P_3$, $\triangle P_2P_3P_4$, $\triangle P_3P_4P_5$, ..., $\triangle P_{n-2}P_{n-1}P_n=\triangle A'B'C'$, da je v tem zaporedju število sprememb orientacije dveh sosednjih trikotnikov sodo
 (Figure \ref{sl.aks.2.2.15a.pic}).

\begin{figure}[!htb]
\centering
\input{sl.aks.2.2.15a.pic}
\caption{} \label{sl.aks.2.2.15a.pic}
\end{figure}

Mogoče je dokazati, da je relacija iste orientacije dveh trikotnikov ekvivalenčna relacija, ki ima dva razreda. Za dva trikotnika, ki nista v istem razredu, pravimo, da sta \pojem{različne orientacije} oz. \pojem{različno orientirana}. Vsak od dveh razredov določa \index{orientacija!ravnine}\pojem{orientacijo ravnine}. Imenujemo ju \pojem{pozitivna orientacija} in \pojem{negativna orientacija}. Za lažjo predstavo se dogovorimo, naj bo orientacija, ki ustreza smeri vrtenja urinega kazalca, negativna, njej nasprotna pa pozitivna orientacija.
 Če ne bomo drugače poudarili, bomo vedno uporabljali pozitivno orientacijo ravnine oz. trikotnikov
 (Figure \ref{sl.aks.2.2.15b.pic}).

\begin{figure}[!htb]
\centering
\input{sl.aks.2.2.15b.pic}
\caption{} \label{sl.aks.2.2.15b.pic}
\end{figure}


Definirajmo še orientacijo kotov. Kota $ASB$ in $A'S'B'$, od katerih nobeden ni iztegnjeni kot, sta \pojem{iste orientacije}, če:
\begin{itemize}
  \item sta oba konveksna ali oba nekonveksna, trikotnika $ASB$ in $A'S'B'$ pa sta iste orientacije (Figure \ref{sl.aks.2.2.16.pic}),
  \item je en kot konveksen in drugi nekonveksen, tikotnika $ASB$ in $A'S'B'$ pa sta nasprotne orientacije (Figure \ref{sl.aks.2.2.16c.pic}).
\end{itemize}

\begin{figure}[!htb]
\centering
\input{sl.aks.2.2.16.pic}
\caption{} \label{sl.aks.2.2.16.pic}
\end{figure}

\begin{figure}[!htb]
\centering
\input{sl.aks.2.2.16c.pic}
\caption{} \label{sl.aks.2.2.16c.pic}
\end{figure}

 Če je $\angle ASB$ iztegnjeni kot, $\angle A'S'B'$ pa konveksen kot, sta kota $ASB$ in $A'S'B'$ iste orientacije, če obstaja takšna točka $C$ v notranjosti kota $ASB$, da sta kota $ASC$ in $A'S'B'$ iste orientacije (Figure \ref{sl.aks.2.2.16d.pic}). Podobno naredimo tudi, če je kot $A'S'B'$ nekonveksen ali če sta oba kota $ASB$ in $A'S'B'$ iztegnjena.

\begin{figure}[!htb]
\centering
\input{sl.aks.2.2.16d.pic}
\caption{} \label{sl.aks.2.2.16d.pic}
\end{figure}


Tudi v tem primeru se izkaže, da je relacija iste orientacije kotov ekvivalenčna relacija, ki ima dva razreda. Pri tem pozitivno orientacijo kota predstavlja tisti razred, pri katerem ima trikotnik $ASB$ za konveksni kot
 $ASB$ iz tega razreda  negativno orientacijo
 (Figure \ref{sl.aks.2.2.16a.pic}).
V tem smislu sta kota $ASB$ in $BSA$ nasprotno orientirana.
   \index{orientacija!kota} \pojem{Orientirani kot} $ASB$  bomo označili z $\measuredangle ASB$.


\begin{figure}[!htb]
\centering
\input{sl.aks.2.2.16a.pic}
\caption{} \label{sl.aks.2.2.16a.pic}
\end{figure}

\begin{figure}[!htb]
\centering
\input{sl.aks.2.2.16b.pic}
\caption{} \label{sl.aks.2.2.16b.pic}
\end{figure}

Če je $C$ poljubna točka, ki ne leži na robu kota $ASB$, bomo definirali vsoto orientiranih kotov $\measuredangle ASC$ in $\measuredangle CSB$
 (Figure \ref{sl.aks.2.2.16b.pic}):
 \begin{eqnarray}
 \measuredangle ASC+\measuredangle CSB = \measuredangle ASB.
 \label{orientKotVsota}
 \end{eqnarray}


%________________________________________________________________________________
 \poglavje{Congruence Axioms}
 \label{odd2AKSSKL}


Naslednji aksiomi so potrebni, da lahko vpeljemo pojem in lastnosti
skladnosti likov. S prejšnjimi aksiomi smo namreč lahko vpeljali in
obravnavali pojme: daljica, poltrak, kot, večkotnik, ... ne pa
še pojmov, ki so povezani s skladnostjo: krožnica, pravi kot,
skladnost trikotnikov,~...

Intuitivna ideja skladnosti likov, ki smo jo uporabljali že v
osnovni šoli, je povezana z gibanjem, ki prvi lik preslika v
drugega. To idejo bomo sedaj uporabili, da bi formalno definirali
pojem skladnosti in njene lastnosti.
 Najprej bomo začeli z osnovnim, že omenjenim, pojmom skladnosti
 parov točk $(A,B)\cong (C,D)$ (Figure \ref{sl.aks.2.3.1.pic})
 ter formalno definirali pojem
 ‘‘gibanja’’.

\begin{figure}[!htb]
\centering
\input{sl.aks.2.3.1.pic}
\caption{} \label{sl.aks.2.3.1.pic}
\end{figure}

 S pomočjo skladnosti parov točk najprej definirajmo skladnost $n$-terice točk.
 Pravimo, da
sta dve $n$-terici točk skladni (Figure \ref{sl.aks.2.3.2.pic}) oz.
$$(A_1 , A_2,\ldots ,A_n ) \cong ( A'_1 , A'_2 ,\ldots , A'_n ),$$
če je: $(A_i,A_j)\cong (A'_i,A'_j)$ za vsako $i,j\in \{1,2,\ldots,
n\}$.

\begin{figure}[!htb]
\centering
\input{sl.aks.2.3.2.pic}
\caption{} \label{sl.aks.2.3.2.pic}
\end{figure}


Bijektivna preslikava ravnine v ravnino
$\mathcal{I}:\mathcal{S}\rightarrow \mathcal{S}$ je
\index{izometrija}\pojem{izometrija} ali \pojem{izometrijska
transformacija}, če ohranja relacijo skladnosti parov točk
(Figure \ref{sl.aks.2.3.3.pic}) oz. če za vsaki dve točki $A$ in
$B$ velja:
 $$(\mathcal{I}(A),\mathcal{I}(B))\cong (A,B).$$

\begin{figure}[!htb]
\centering
\input{sl.aks.2.3.3.pic}
\caption{} \label{sl.aks.2.3.3.pic}
\end{figure}


Z naslednjimi aksiomi bomo vpeljali lastnosti na novo definirane
preslikave.


\vspace*{3mm}


            \baksiom \label{aksIII1} Isometries preserve the relation
            $\mathcal{B}$  (Figure \ref{sl.aks.2.3.4.pic}), which means that for every
              isometry $\mathcal{I} $ holds:
            $$\mathcal{I}: A, B,C\mapsto A',B',C'\hspace*{2mm}\wedge \hspace*{2mm}
            \mathcal{B}(A,B,C)
             \hspace*{1mm}\Rightarrow\hspace*{1mm} \mathcal{B}(A',B',C').$$
             \eaksiom


\begin{figure}[!htb]
\centering
\input{sl.aks.2.3.4.pic}
\caption{} \label{sl.aks.2.3.4.pic}
\end{figure}


            \baksiom  \label{aksIII2} If $ABC$ and $A'B'C'$ are two half-planes
             (Figure \ref {asl.aks.2.3.5.pic}), then there is a single isometry
             $\mathcal{I}$, which maps:

            \begin{itemize}
            \item point $A$ to point $A'$,
               \item half-line $AB$ in half-line $A'B'$,
              \item half-plane $ABC$ to half-plane $A'B'C'$.
            \end{itemize}
            If $(A,B)\cong (A',B')$ holds,
            then it is $\mathcal{I}(B)=B'$.
            \\ If in addition
            $(A,B,C)\cong (A',B',C')$ also holds,
             then it is $\mathcal{I}(B)=B'$ and $\mathcal{I}(C)=C'$.
            \eaksiom


\begin{figure}[!htb]
\centering
\input{sl.aks.2.3.5.pic}
\caption{} \label{asl.aks.2.3.5.pic}
\end{figure}

            \baksiom  \label{aksIII3} For every two points $A$ and $B$ there exists
             isometry such that holds $$\mathcal{I}: A, B\mapsto B,A.$$
             If $(S,A)\cong (S,B)$ and $S\in AB$, then for each isometry $\mathcal{I}$ with this property  holds $\mathcal{I}(S)=S$ (Figure \ref{sl.aks.2.3.6.pic}).
            \eaksiom

\begin{figure}[!htb]
\centering
\input{sl.aks.2.3.6.pic}
\caption{} \label{sl.aks.2.3.6.pic}
\end{figure}

            \baksiom  \label{aksIII4} The set of all isometries with respect to the composition of mappings form a group, which means that:
            \begin{itemize}
            \item composition of two isometries  $\mathcal{I}_2\circ \mathcal{I}_1$ is isometry,
            \item identity map $\mathcal{E}$ is isometry,
            \item if $\mathcal{I}$ is  isometry, then its inverse transformation
            $\mathcal{I}^{-1}$ is also isometry.
            \end{itemize}
             \eaksiom

\vspace*{3mm}

 Omenimo, da je pri strukturi grupe zahtevana tudi lastnost
 asociativnosti oz. $\mathcal{I}_1\circ (\mathcal{I}_2\circ \mathcal{I}_3)=
  (\mathcal{I}_1\circ \mathcal{I}_2)\circ \mathcal{I}_3$ (za poljubne izometrije
  $\mathcal{I}_1$, $\mathcal{I}_2$ in $\mathcal{I}_3$), ki pa je pri
  operaciji kompozituma funkcij
 avtomatično izpolnjena. Omenimo še, da je \pojem{identiteta} \index{identiteta}
 $\mathcal{E}$ iz prejšnjega aksioma preslikava, za katero je
 $\mathcal{E}(A)=A$ za vsako točko ravnine. Preslikava
 $\mathcal{I}^{-1}$ je \pojem{inverzna preslikava} za izometrijo
 $\mathcal{I}$, če velja $\mathcal{I}^{-1}\circ \mathcal{I}
 =\mathcal{I}\circ\mathcal{I}^{-1}=\mathcal{E}$. Po prejšnjem
 aksiomu sta torej identiteta in inverzna preslikava vsake izometrije tudi
  izometriji.



Dokažimo prve posledice aksiomov skladnosti. Najprej bomo
obravnavali naslednje lastnosti izometrij.



            \bizrek \label{izrekIzoB} Isometry maps a line to a line, a line segment to a line segment, a ray to a ray,
            a half-plane to a half-plane, an angle to an angle and an $n$-gon to an $n$-gon.
             \eizrek

\textbf{\textit{Proof.}}
 Po aksiomu \ref{aksIII1} izometrije ohranjajo relacijo
 $\mathcal{B}$. Zato se vse točke daljice $AB$ pri izometriji $I$ preslikajo
  v točke, ki ležijo na daljici $A'B'$, kjer je $A'=\mathcal{I}(A)$ in
  $B'=\mathcal{I}(B)$. Ker je tudi inverzna preslikava $\mathcal{I}^{-1}$
  izometrija (aksiom \ref{aksIII4}), je vsaka točka daljice
  $A'B'$ slika neke točke, ki leži na daljici $AB$. Torej se z
  izometrijo $\mathcal{I}$
  daljica $AB$ preslika v daljico $A'B'$.

 Tudi preostale like iz izreka smo definirali s pomočjo relacije
 $\mathcal{B}$, tako da je dokaz podoben kot za daljico.
 \kdokaz

Iz dokaza prejšnjega izreka sledi, da se krajišči daljice $AB$
z izometrijo preslikata v krajišči slike $A'B'$. Na podoben
način se z izometrijo izhodišče poltraka preslika v izhodišče
poltraka, rob polravnine v rob polravnine, vrh kota v vrh kota in
oglišče večkotnika v oglišče večkotnika.

Izometrije so definirane kot bijektivne preslikave, ki ohranjajo
skladnost parov točk. Ali velja tudi, da za skladne pare točk
obstaja izometrija, ki prvi par preslika v drugega? Odgovor podajmo
z naslednjim izrekom.


            \bizrek \label{izrekAB} If $(A,B)\cong (A',B')$, then there is an isometry
             $\mathcal{I}$, which maps the points $A$ and $B$ to the points
            $A'$ and $B'$,
             i.e.:
            $$\mathcal{I}: A, B\mapsto A',B'.$$
            \eizrek


\begin{figure}[!htb]
\centering
\input{sl.aks.2.3.7.pic}
\caption{} \label{sl.aks.2.3.7.pic}
\end{figure}


\textbf{\textit{Proof.}}
 Naj bo $C$ točka, ki ne leži na premici $AB$, in $C'$ točka,
 ki ne leži na premici $A'B'$ (Figure \ref{sl.aks.2.3.7.pic}).
 Po aksiomu \ref{aksIII2} obstaja ena sama izometrija $\mathcal{I}$, ki
 preslika točko $A$ v točko $A'$, poltrak $AB$ v poltrak $A'B'$
 in polravnino $ABC$ v polravnino $A'B'C'$. Ker je po predpostavki
 $(A,B)\cong (A',B')$ iz istega aksioma \ref{aksIII2}, sledi še
 $\mathcal{I}(B)=B'$.
 \kdokaz

Podoben je dokaz naslednjega izreka, ki bo kasneje v drugačni obliki
podan kot prvi izrek o skladnosti trikotnikov.



            \bizrek \label{IizrekABC} Let $(A,B,C)$ and $(A',B',C')$ be
            triplets of non-collinear points such that $$(A,B,C)\cong (A',B',C'),$$
            then there is a single isometry  $\mathcal{I}$, that maps the points
             $A$, $B$ and $C$ into the points $A'$, $B'$ and $C'$, i.e.:
            $$\mathcal{I}: A, B,C\mapsto A',B',C'.$$
            \eizrek


\begin{figure}[!htb]
\centering
\input{sl.aks.2.3.5.pic}
\caption{} \label{sl.aks.2.3.5.pic}
\end{figure}


 \textbf{\textit{Proof.}}
 Po aksiomu \ref{aksIII2} obstaja ena sama izometrija $\mathcal{I}$, ki
 preslika točko $A$ v točko $A'$, poltrak $AB$ v poltrak $A'B'$
 in polravnino $ABC$ v polravnino $A'B'C'$ (Figure \ref{sl.aks.2.3.5.pic}).
  Ker je po predpostavki
 $(A,B,C)\cong (A',B',C')$ iz istega aksioma \ref{aksIII2}, sledi še
 $\mathcal{I}(B)=B'$ in $\mathcal{I}(C)=C'$.

  Potrebno je dokazati, da je $\mathcal{I}$ edina takšna izometrija.
  Predpostavimo, da obstaja takšna izometrija $\mathcal{\widehat{I}}$, da
  velja $\mathcal{\widehat{I}}: A, B,C\mapsto A',B',C'$. Po
  izreku \ref{izrekIzoB} izometrija $\mathcal{\widehat{I}}$
  tudi poltrak $AB$ preslika v poltrak $A'B'$ in polravnino $ABC$
  v polravnino $A'B'C'$. Iz aksioma \ref{aksIII2} sledi
   $\mathcal{\widehat{I}}=\mathcal{I}$.
 \kdokaz

Direktna posledica je naslednji izrek.


                \bizrek \label{IizrekABCident} Let $A$, $B$ and $C$ be three non-collinear points, then the identity map
                 $\mathcal{E}$ is the only isometry that maps points $A$, $B$, and $C$ to the same points
                $A$, $B$ and $C$.
                \eizrek


\begin{figure}[!htb]
\centering
\input{sl.aks.2.3.5a.pic}
\caption{} \label{sl.aks.2.3.5a.pic}
\end{figure}


\textbf{\textit{Proof.}} (Figure \ref{sl.aks.2.3.5a.pic})

Najprej je identična preslikava $\mathcal{E}$, ki točke $A$, $B$
in $C$ preslika v  točke $A$, $B$ in $C$, izometrija po aksiomu
\ref{aksIII4}. Iz prejšnjega izreka \ref{IizrekABC} sledi, da je
takšna izometrija ena sama.
 \kdokaz

Za točko $A$ pravimo, da je \index{točka!fiksna} \pojem{fiksna
točka} (ali \index{točka!negibna} \pojem{negibna
točka}) izometrije $\mathcal{I}$, če velja $\mathcal{I}(A)=A$.
Prejšnji izrek nam pove, da so edine izometrije, ki imajo fiksne tri
nekolinearne točke, identitete.


 Izometrije bomo podrobneje obravnavali v poglavju
 \ref{pogIZO}, na tem mestu pa jih bomo uporabili predvsem za pomoč
 pri vpeljavi skladnosti likov. Dva lika $\Phi$ in $\Phi'$ sta
 \index{lika!skladna}\pojem{skladna}
 (označimo
 $\Phi\cong \Phi'$),
 če obstaja izometrija $I$, ki lik $\Phi$ preslika v lik $\Phi'$.

 Direktna posledica aksioma \ref{aksIII4} je naslednji izrek.


            \bizrek
             Congruence of figures is an equivalence relation. \label{sklRelEkv}
            \eizrek

\textbf{\textit{Proof.}}

\textit{Refleksivnost.} Za vsak lik $\Phi$ velja $\Phi \cong
\Phi$, ker je identična preslikava $\mathcal{E}$ izometrija
(aksiom \ref{aksIII4}) in $\mathcal{E}:\Phi\rightarrow\Phi$.

\textit{Simetričnost.} Iz $\Phi \cong \Phi_1$ sledi,  da obstaja
izometrija $\mathcal{I}$, ki preslika lik $\Phi$ v lik $\Phi_1$.
Inverzna preslikava $\mathcal{I}^{-1}$, ki je po aksiomu
\ref{aksIII4} izometrija, preslika lik $\Phi_1$ v lik $\Phi$,
zato velja $\Phi_1 \cong \Phi$.

\textit{Tranzitivnost.} Iz $\Phi \cong \Phi_1$ in $\Phi_1 \cong
\Phi_2$ sledi,  da obstajata takšni izometriji $\mathcal{I}$ in
$\mathcal{I}'$, da velja $\mathcal{I}:\Phi\rightarrow\Phi_1$ in
$\mathcal{I}':\Phi_1\rightarrow\Phi_2$.
 Potem  kompozitum $\mathcal{I}'\circ\mathcal{I}$,
  ki je po aksiomu \ref{aksIII4} izometrija, preslika lik $\Phi$
v lik $\Phi_2$, zato velja $\Phi \cong \Phi_2$.
\kdokaz



  Pojem skladnosti likov se nanaša tudi na daljice. Intuitivno
  smo skladnost daljic povezovali s skladnostjo parov točk.
  Sedaj bomo dokazali ekvivalentnost obeh relacij.

            \bizrek  \label{izrek(A,B)} $AB \cong A'B' \Leftrightarrow
            (A,B)\cong (A',B')$
             \eizrek

\textbf{\textit{Proof.}}

 ($\Rightarrow$) Če je $(A,B)\cong
(A',B')$, po izreku \ref{izrekAB} obstaja izometrija
$\mathcal{I}$, ki preslika točki $A$ in $B$ v točki $A'$ in
$B'$. Iz izreka \ref{izrekIzoB} sledi, da izometrija $\mathcal{I}$
preslika daljico $AB$ v daljico $A'B'$ oz. $AB \cong A'B'$.

($\Leftarrow$) Če je $AB \cong A'B'$, obstaja izometrija
$\mathcal{I}$, ki preslika daljico $AB$ v daljico $A'B'$. Po
posledici izreka \ref{izrekIzoB} se krajišče daljice preslika v
krajišče daljice. To pomeni, da velja bodisi
$\mathcal{I}:A,B\mapsto A',B'$ bodisi $\mathcal{I}:A,B\mapsto
B',A'$. Iz prve relacije sledi $(A,B)\cong (A',B')$ iz druge pa
$(A,B)\cong (B',A')$. Toda tudi iz drugega primera dobimo
$(A,B)\cong (A',B')$, kar je posledica aksiomov \ref{aksIII3} in
\ref{aksIII4}.
\kdokaz

 Zaradi prejšnjega izreka bomo v nadaljevanju namesto
  relacije $(A,B)\cong (A',B')$ vedno pisali $AB\cong
 A'B'$.


            \bizrek \label{ABnaPoltrakCX}
            For each line segment $AB$ and each ray $CX$, there is exactly
            one point $D$ on the ray
            $CX$ that $AB\cong CD$ holds.
            \eizrek


\begin{figure}[!htb]
\centering
\input{sl.aks.2.3.5b.pic}
\caption{} \label{sl.aks.2.3.5b.pic}
\end{figure}



 \textbf{\textit{Proof.}} Naj bo $P$ točka, ki ne leži na premici $AB$,
 in $Q$ točka,
 ki ne leži na premici $CX$ (Figure \ref{sl.aks.2.3.5b.pic}).
  Po aksiomu \ref{aksIII2} obstaja ena sama
 izometrija $\mathcal{I}$, ki
 preslika točko $A$ v točko $C$, poltrak $AB$ v poltrak $CX$
 in polravnino $ABP$ v polravnino $CXQ$.
 Naj bo $D=\mathcal{I}(C)$, potem je $AB \cong CD$.

 Predpostavimo,
 da na poltraku $CX$ obstaja še ena točka $\widehat{D}$, za
 katero
 velja $AB \cong C\widehat{D}$. Ker poltraka
 $CX$ in $CD$ sovpadata, izometrija $\mathcal{I}$ pa preslika točko
 $A$ v točko $C$, poltrak $AB$ v poltrak $CD$
 in polravnino $ABP$ v polravnino $CDQ$,
 iz aksioma \ref{aksIII2} sledi $\mathcal{I}(C)=\widehat{D}$ oz.
 $\widehat{D}=D$.
 \kdokaz


                \bizrek \label{izomEnaC'} Let $A$, $B$, $C$ be three non-collinear points
                 and $A'$, $B'$ points of the edge of a half-plane $\pi$ such that $AB \cong A'B'$.
                  Then there is exactly one point $C'$ in the half-plane $\pi$ such that $AC \cong A'C'$ and $BC \cong B'C'$.
                 \eizrek


\begin{figure}[!htb]
\centering
\input{sl.aks.2.3.11a.pic}
\caption{} \label{sl.aks.2.3.11a.pic}
\end{figure}


 \textbf{\textit{Proof.}} (Figure \ref{sl.aks.2.3.11a.pic})

 Po aksiomu \ref{aksIII2} obstaja ena sama
 izometrija $\mathcal{I}$, ki
 preslika točko $A$ v točko $A'$, poltrak $AB$ v poltrak $A'B'$
 in polravnino $ABC$ v polravnino $\pi$ ter velja $\mathcal{I}(B)=B'$.
 Naj bo $C'=\mathcal{I}(C)$, potem velja $AC \cong A'C'$ in
 $BC \cong B'C'$. Predpostavimo, da obstaja takšna točka
 $\widehat{C}'$, ki leži v polravnini $\pi$ ter velja $AC \cong A'\widehat{C}'$ in
 $BC \cong B'\widehat{C}'$. Ker je še $AB \cong
A'B'$, po izreku \ref{IizrekABC} obstaja ena sama izometrija
$\mathcal{\widehat{I}}$, ki preslika točke $A$, $B$ in $C$ v
točke $A'$, $B'$ in $\widehat{C}'$. Toda le-ta preslika tudi
poltrak $AB$ v poltrak $A'B'$ in polravnino $ABC$ v polravnino
$A'B'\widehat{C}'=\pi$. Po aksiomu \ref{aksIII2} je
$\mathcal{\widehat{I}}=\mathcal{I}$ in zato tudi
$\widehat{C}'=\mathcal{\widehat{I}}(C)=\mathcal{I}(C)=C'$.
 \kdokaz


            \bizrek \label{izoABAB} If $\mathcal{I}$ is an isometry that maps a points $A$ and $B$ into the same points
            $A$ and $B$ (i.e. $\mathcal{I}(A)=A$ and $\mathcal{I}(B)=B$), then it also holds for each point $X$ on the line
            $AB$ (i.e. $\mathcal{I}(X)=X$).
             \eizrek

\begin{figure}[!htb]
\centering
\input{sl.aks.2.3.8.pic}
\caption{} \label{sl.aks.2.3.8.pic}
\end{figure}


%

\textbf{\textit{Proof.}} Označimo z $X$ poljubno točko premice
$AB$. Brez škode za splošnost predpostavimo, da točka $X$ leži
na poltraku $AB$ (Figure \ref{sl.aks.2.3.8.pic}).  Dokažimo, da
velja $\mathcal{I}(X)=X$.

Naj bo $P$ točka, ki ne leži na premici $AB$ in
$P'=\mathcal{I}(P)$. Izometrija $\mathcal{I}$
 preslika točko $A$ v točko $A$, poltrak $AB$ v poltrak $AB$
 (oz. poltrak $AX$ v poltrak $AX$)
 in polravnino $ABP$ v polravnino $ABP'$
 (oz. polravnino $AXP$ v polravnino $AXP'$).
 Po aksiomu \ref{aksIII2}
 iz $AX\cong AX$ sledi $\mathcal{I}(X)=X$.
 \kdokaz

 Vpeljimo nove pojme, ki se nanašajo na daljice.

Pravimo, da je daljica $EF$ \index{vsota!daljic}\pojem{vsota daljic}
$AB$ in $CD$, kar označimo $EF=AB+CD$, če obstaja takšna točka $P$
na daljici $EF$, da velja $AB \cong EP$ in $CD \cong PF$ (Figure
\ref{sl.aks.2.3.9.pic}).

\begin{figure}[!htb]
\centering
\input{sl.aks.2.3.9.pic}
\caption{} \label{sl.aks.2.3.9.pic}
\end{figure}


Daljica $EF$ je \index{razlika!daljic}\pojem{razlika daljic} $AB$
in $CD$, kar označimo $EF=AB-CD$, če je $AB=EF+CD$ (Figure
\ref{sl.aks.2.3.9.pic}).

 Na podoben način lahko definiramo tudi
  množenje daljice z naravnim in s pozitivnim racionalnim
  številom. Za daljici $AB$ in $CD$ je $AB=n\cdot CD$
  ($n\in \mathbb{N}$), če  obstajajo takšne točke
  $X_1$, $X_2$,..., $X_{n-1}$, da je
  $\mathcal{B}(X_1,X_2,\ldots,X_{n-1})$ in
  $AX_1 \cong X_1X_2 \cong X_{n-1}B \cong CD$ (Figure
\ref{sl.aks.2.3.10.pic}).
  V tem primeru je tudi $CD=\frac{1}{n}\cdot AB$.

  Na tem mestu ne bomo formalno dokazovali dejstva, da za vsako daljico $PQ$ in vsako naravno število $n$ obstaja daljica $AB$, za katero je $AB=n\cdot PQ$, ter daljica $CD$, za katero je $CD=\frac{1}{n}\cdot PQ$.

\begin{figure}[!htb]
\centering
\input{sl.aks.2.3.10.pic}
\caption{} \label{sl.aks.2.3.10.pic}
\end{figure}


  Množenje daljice s pozitivnim
  racionalnim številom vpeljemo na naslednji način. Za
  $q=\frac{n}{m} \in \mathbb{Q^+}$ je:
$$q\cdot AB=\frac{n}{m}\cdot AB = n\cdot\left(\frac{1}{m}\cdot AB\right)$$

Če za točko $P$ daljice $AB$ velja $AP=\frac{n}{m}\cdot PB$,
pravimo, da točka $P$ deli daljico $AB$ v \index{razmerje}
\pojem{razmerju} $n:m$, kar zapišemo $AP:PB=n:m$.

Daljica $AB$ je \index{relacija!urejenosti daljic}\pojem{daljša}
od daljice $CD$, kar označimo $AB>CD$, če obstaja takšna točka
$P\neq B$ na daljici $AB$, da velja $CD \cong AP$ (Figure
\ref{sl.aks.2.3.11.pic}). V tem primeru pravimo tudi, da je
daljica $CD$ \pojem{krajša} od daljice $AB$ (oznaka $CD<AB$).

\begin{figure}[!htb]
\centering
\input{sl.aks.2.3.11.pic}
\caption{} \label{sl.aks.2.3.11.pic}
\end{figure}

Ni težko dokazati, da za daljici $AB$ in $CD$ velja natanko ena
od relacij $AB>CD$ ali $AB<CD$ ali $AB \cong CD$. To je posledica
izreka \ref{ABnaPoltrakCX}.

Točka $S$ je \index{središče!daljice} \pojem{središče (razpolovišče) daljice} $AB$,
če leži na tej daljici in velja $AS \cong SB$ (Figure
\ref{sl.aks.2.3.12.pic}). Jasno je, da središče deli daljico v
razmerju $1:1$. Potrebno je še dokazati, da takšna točka vedno
obstaja.

\begin{figure}[!htb]
\centering
\input{sl.aks.2.3.12.pic}
\caption{} \label{sl.aks.2.3.12.pic}
\end{figure}

                \bizrek
              For every line segment, there is exactly one midpoint.
                 \eizrek


\begin{figure}[!htb]
\centering
\input{sl.aks.2.3.13.pic}
\caption{} \label{sl.aks.2.3.13.pic}
\end{figure}

\textbf{\textit{Proof.}}
 Naj bo $AB$ daljica in $C$ poljubna točka, ki ne leži na
 premici $AB$
(Figure \ref{sl.aks.2.3.13.pic}). Označimo s $\pi$ polravnino
$ABC$ in s $\pi'$ komplementarno polravnino polravnine $\pi$. Po
aksiomu \ref{aksIII2} obstaja ena sama
 izometrija $\mathcal{I}$, ki
 preslika točko $A$ v točko $B$, poltrak $AB$ v poltrak $BA$
 in polravnino $\pi$ v polravnino $\pi'$. Iz $AB\cong BA$ (posledica
 aksioma \ref{aksIII3}) po istem
 aksiomu sledi $\mathcal{I}(B)=A$.

 Naj bo $C'=\mathcal{I}(C)$, potem velja $AC \cong B'C'$ in
 $BC \cong A'C'$. Ker sta $C$ in $C'$ na različnih straneh premice
 $AB$, premica $CC'$ seka premico $AB$ v neki točki $S$.
 Če je $\widehat{C}=\mathcal{I}(C')$, velja $A'C' \cong B\widehat{C}$ in
 $B'C' \cong A\widehat{C}$. Ker je tudi $AC \cong B'C'$ in
 $BC \cong A'C'$, po izreku \ref{izomEnaC'} je $\widehat{C}=C$ oz.
 $\mathcal{I}(C')=C$. Torej izometrija $\mathcal{I}$ preslika premici $AB$
 in $CC'$ sami vase, zato je:
 $$\mathcal{I}(S)=\mathcal{I}(AB\cap CC')=
 \mathcal{I}(AB)\cap \mathcal{I}(CC')=
 AB\cap CC'=S.$$
 Sedaj iz $\mathcal{I}:A,S\mapsto B,S$ sledi $AS\cong SB$.

 Da bi dokazali, da je točka $S$ res središče daljice $AB$, je potrebno
 dokazati še, da točka $S$ leži na daljici $AB$. Predpostavimo
 nasprotno. Brez škode za splošnost naj bo $\mathcal{B}(A,B,S)$. Toda v
 tem primeru na poltraku $SA$ obstajata dve takšni točki $A$ in $B$,
 da velja $SA\cong SB$, kar nasprotuje izreku \ref{ABnaPoltrakCX}.

  Dokažimo še, da ima daljica eno samo središče. Naj bo
  $\widehat{S}\neq S$ točka daljice $AB$ in  $A\widehat{S}\cong
  \widehat{S}B$. Iz aksioma \ref{aksIII3} sledi
  $\mathcal{I}(A\widehat{S})=A\widehat{S}$. To pomeni (izrek
  \ref{izoABAB}), da  za vsako točko $X\in AB$ velja
  $\mathcal{I}(X)=X$ in tudi $\mathcal{I}(A)=A$, kar pa ni mogoče.
  Torej je $\widehat{S}= S$.
 \kdokaz

Pravimo, da je točka $A$ \index{simetričnost!glede na točko}\pojem{simetrična} točki $B$ glede na točko $S$, če je $S$ središče daljice $AB$. Simetričnost glede na preslikavo čez točko (t. i. središčno zrcaljenje) bomo
podrobneje obravnavali v razdelku \ref{odd6SredZrc}.

 Sedaj bomo vpeljali pojme in izpeljali lastnosti, ki se nanašajo
 na kote in so analogni tistim, ki smo jih vpeljali za daljice.
 Če se izrek \ref{ABnaPoltrakCX} intuitivno nanaša na prenos daljice s šestilom
  na
 poltrak, bo naslednji izrek predstavljal prenos kota k danemu
 poltraku.



                 \bizrek \label{KotNaPoltrak}
                 For each angle $\alpha$ and each ray $Sp$ lies on a line $p$,
                there is exactly one ray $Sq$ in one of the half-planes determined by the line $p$, such that
                $\alpha \cong \angle pSq$.
                \eizrek

\begin{figure}[!htb]
\centering
\input{sl.aks.2.3.14.pic}
\caption{} \label{sl.aks.2.3.14.pic}
\end{figure}

 \textbf{\textit{Proof.}}
 Naj bo $\alpha=\angle BAC$ in $\pi'$ ena od polravnin, ki
  jo določa nosilka poltraka $Sp$ (Figure \ref{sl.aks.2.3.14.pic}).

 Po izreku \ref{ABnaPoltrakCX} obstaja
 ena sama točka $P$ na poltraku $Sp$, da velja $AB \cong SP$.
 Po aksiomu \ref{aksIII2} obstaja ena sama izometrija $\mathcal{I}$, ki
 preslika točko $A$ v točko $S$, poltrak $AB$ v poltrak $Sp$
 in polravnino $ABC$ v polravnino $\pi'$.
  Če je $Q=\mathcal{I}(C)$, se poltrak $AC$ s to izometrijo
  preslika v poltrak $SQ$. Zato poltrak $SQ=Sq$ leži v polravnini
  $\pi'$ in velja $\angle BAC\cong pSq$.

 Predpostavimo, da je tudi $S\widehat{q}$ poltrak, ki leži v polravnini
  $\pi'$ in velja $\angle BAC\cong pS\widehat{q}$. Iz definicije skladnosti
   sledi, da obstaja neka izometrija $\mathcal{\widehat{I}}$, ki
   preslika kot $BAC$ v kot $\angle BAC\cong pS\widehat{q}$. Ker
   tudi
   izometrija $\mathcal{\widehat{I}}$
 preslika točko $A$ v točko $S$, poltrak $AB$ v poltrak $Sp$
 in polravnino $ABC$ v polravnino $\pi'$, je po aksiomu
 \ref{aksIII2} $\mathcal{\widehat{I}}=\mathcal{I}$. Torej je
  $$S\widehat{q}=\mathcal{\widehat{I}}(AB)=\mathcal{I}(AB)=Sq,$$ kar je bilo potrebno dokazati. \kdokaz

 Ker nosilka poltraka $Sp$ iz prejšnjega izreka določa dve
 polravnini, obstajata dva kota s krakom $Sp$, ki sta skladna kotu
 $\alpha$. Omenjena kota sta različno orientirana. To pomeni, da ima
  za orientirani kot $\alpha$  le eden izmed teh dveh kotov isto
 orientacijo kot $\alpha$ (Figure \ref{sl.aks.2.3.15.pic}).

\begin{figure}[!htb]
\centering
\input{sl.aks.2.3.15.pic}
\caption{} \label{sl.aks.2.3.15.pic}
\end{figure}


Podobno kot pri daljicah definiramo določene operacije in
relacije tudi med koti.

 Kot $pq$ z vrhom $S$ je \index{vsota!kotov}\pojem{vsota kotov} $ab$ in $cd$ oz.
 $\angle pq = \angle ab + \angle cd$, če obstaja poltrak
 $s=SX$, ki leži v kotu $pq$ ter velja $\angle ps \cong \angle ab$
 in $\angle sq \cong \angle cd$ (Figure \ref{sl.aks.2.3.16.pic}).
  V tem primeru pravimo tudi, da je
 kot $ab$ \index{razlika!kotov}\pojem{razlika kotov} $pq$ in $cd$, oz.
  $ \angle ab= \angle pq - \angle cd$.

\begin{figure}[!htb]
\centering
\input{sl.aks.2.3.16.pic}
\caption{} \label{sl.aks.2.3.16.pic}
\end{figure}

Analogno kot pri daljicah za kot $ab$ definiramo kote
$n\cdot \angle ab$ in  $\frac{1}{n}\cdot \angle ab$ ($n\in
\mathbb{N}$) ter $q\cdot \angle ab$ ($q\in \mathbb{Q}$).

Pravimo, da je kot $ab$ z vrhom $S$ \index{relacija!urejenosti
kotov}\pojem{večji} od kota $cd$ ($\angle ab > \angle cd$), če
obstaja v kotu $ab$ poltrak $s=SX$, da velja $\angle as \cong
\angle cd$ (Figure \ref{sl.aks.2.3.17.pic}). V tem primeru je tudi
kot $cd$ \pojem{manjši} od kota $ab$ ($\angle cd< \angle ab$). Ni
težko dokazati, da za dva kota $ab$ in $cd$ velja natanko ena od
relacij: $\angle ab > \angle cd$, $\angle ab < \angle cd$ ali
$\angle ab \cong \angle cd$.


\begin{figure}[!htb]
\centering
\input{sl.aks.2.3.17.pic}
\caption{} \label{sl.aks.2.3.17.pic}
\end{figure}


Kota sta \index{kota!suplementarna}\pojem{suplementarna}, če je
njuna vsota enaka iztegnjenem kotu  (Figure
\ref{sl.aks.2.3.18.pic}).

\begin{figure}[!htb]
\centering
\input{sl.aks.2.3.18.pic}
\caption{} \label{sl.aks.2.3.18.pic}
\end{figure}


Poltrak $s=SX$ je \index{bisektrisa kota}\pojem{bisektrisa kota}
$\angle pSq=\alpha$ (Figure \ref{sl.aks.2.3.19.pic}), če leži v tem kotu in
velja $\angle ps \cong \angle sq$. Nosilka te bisektrise  je \index{simetrala!kota}\pojem{simetrala kota} $pSq$ (premica $s_{\alpha}$).

\begin{figure}[!htb]
\centering
\input{sl.aks.2.3.19.pic}
\caption{} \label{sl.aks.2.3.19.pic}
\end{figure}



Podobno kot za središče daljice velja za bisektriso kota naslednji izrek.

            \bizrek \label{izrekSimetralaKota}
             An angle has exactly one bisector.
             %(oz. eno samo simetralo).
            \eizrek

\textbf{\textit{Proof.}}
 Naj bo $\alpha=pSq$ poljubni kot, $P$ poljubna točka, ki leži na kraku
 $Sp$ ($P\neq S$) ter $Q$ točka, ki leži na kraku $Sq$ in velja
 $SP\cong SQ$.

\begin{figure}[!htb]
\centering
\input{sl.aks.2.3.20.pic}
\caption{} \label{sl.aks.2.3.20.pic}
\end{figure}



 Predpostavimo, da je kot $\alpha$ iztegnjeni kot
 (Figure \ref{sl.aks.2.3.20.pic}), ki določa
 polravnino $\pi$. Naj bo $A$ njena poljubna točka. Po izreku
 \ref{izomEnaC'} v polravnini $\pi$ obstaja ena sama točka $B$,
 da velja $(P,Q,A)\cong (Q,P,B)$. Iz izreka \ref{IizrekABC} sledi,
 da obstaja ena sama izometrija $\mathcal{I}$, ki točke $P$, $Q$ in
 $A$ preslika v točke $Q$, $P$ in $B$. Naj bo
 $\mathcal{I}(B)=\widehat{A}$. Ker je
 $(Q,P,B)\cong(P,Q,\widehat{A})$, je po izreku \ref{izomEnaC'}
 $\widehat{A}=A$. Torej:
  $$\mathcal{I}:P,Q,A,B\mapsto Q,P,B,A.$$
Zato se središči $S$ in $L$ daljic $PQ$ in $AB$ preslikata vase
(aksiom \ref{aksIII4}), kar potem velja tudi za poltrak $s=SL$ in
vsako njegovo točko (izrek \ref{izoABAB}). Torej izometrija
$\mathcal{I}$ preslika kot $pSs$ v kot $sSq$, zato
  je $pSs\cong sSq$ oz. poltrak $s$ je bisektrisa kota $pSq$.

  Dokažimo, da je $s$ edina bisektrisa kota $\alpha$. Naj bo
  $\widehat{s}=S\widehat{L}$
  poltrak, ki leži v kotu $\alpha$ in velja $pS\widehat{s}\cong
  \widehat{s}Sq$. Potem obstaja izometrija $\mathcal{\widehat{I}}$, ki
  preslika kot $pS\widehat{s}$ v kot $\widehat{s}Sq$. Ta izometrija
  preslika točko $S$ v točko $S$, poltrak $p$ v poltrak $q$ in
  polravnino $\pi$ v polravnino $\pi$, zato je po aksiomu
  \ref{aksIII2} $\mathcal{\widehat{I}}=\mathcal{I}$. Torej
  $\mathcal{I}(\widehat{s})=
  \mathcal{\widehat{I}}(\widehat{s})=\widehat{s}$. Če $\widehat{L} \notin
  s$, izometrija $\mathcal{I}$ preslika tri nekolinearne točke
  $S$, $L$ in $\widehat{L}$ vase in je $\mathcal{I}$ identična preslikava
  (izrek \ref{IizrekABCident}), kar ni mogoče. Torej $\widehat{L} \in
  s$ oz. $\widehat{s}=s$.


 Če je $\alpha$ neiztegnjeni konveksni kot  (Figure \ref{sl.aks.2.3.20.pic}),
  so točke $S$, $P$ in $Q$
 nekolinearne, zato po izreku \ref{IizrekABC} obstaja ena sama
 izometrija $\mathcal{I}$, ki  točke $P$, $S$ in $Q$ preslika v
 točke $Q$, $S$ in $P$. Z $L$ označimo središče daljice $PQ$.
 Po aksiomu \ref{aksIII3} je $\mathcal{I}(L)=L$. Potem se tudi vse
 točke poltraka $s=SL$ preslikajo vase (izrek \ref{izoABAB}). Kot
$\alpha$ je konveksni kot, kar pomeni, da točka $L$ in potem tudi
poltrak $s$ ležita v tem kotu.
 Torej izometrija $\mathcal{I}$ preslika kot $pSs$ v kot $sSq$, zato
  je $pSs\cong sSq$ oz. poltrak $s$ je bisektrisa kota $pSq$.

Na podoben način kot v prejšnjem primeru dokažemo, da kot
$\alpha$ nima drugih bisektris.

Če je $\alpha$ nekonveksni kot, bisektriso dobimo kot
komplementarni (dopolnilni) poltrak poltraka $s$.
 \kdokaz

Dokažimo dva izreka, ki se nanašata na sokote in sovršne kote.\index{kota!sokota} \index{kota!sovršna}



               \bizrek
              The adjacent supplementary angles  of two congruent angles are also congruent. \label{sokota}
             \eizrek

\begin{figure}[!htb]
\centering
\input{sl.aks.2.3.20a.pic}
\caption{} \label{sl.aks.2.3.20a.pic}
\end{figure}



\textbf{\textit{Proof.}} Naj bosta $\alpha'=\angle P'OQ$ in $\alpha_1'=\angle P_1'O_1Q_1$ sokota
 dveh skladnih kotov $\alpha=\angle POQ$ in $\alpha_1=\angle P_1O_1Q_1$ (Figure \ref{sl.aks.2.3.20a.pic}). Po aksiomu \ref{aksIII2} obstaja ena sama izometrija $\mathcal{I}$, ki preslika točko $O$ v točko $O_1$, poltrak $OP$ v poltrak $O_1P_1$ in polravnino $POQ$ v polravnino $P_1O_1Q_1$. Naj bo $Q_2=\mathcal{I}(Q)$. Potem je $\angle P_1O_1Q_2\cong \angle POQ$. Izometrija $\mathcal{I}$ preslika polravnino $POQ$ v polravnino $P_1O_1Q_1$, zato točka $Q_2$ (in tudi poltrak $O_1Q_2$) leži v polravnini $P_1O_1Q_1$. Ker je po predpostavki še  $\angle POQ\cong\angle P_1O_1Q_1$, po izreku  \ref{KotNaPoltrak} $OQ_1$ in $OQ_2$ predstavljata isti poltrak. Torej točka $Q_2$ leži na poltraku $O_1Q_1$. Naj bo $P_2'=\mathcal{I}(P')$. Ker izometrije poltrak preslikajo v poltrak (izrek \ref{izrekIzoB}), leži točka $P_2'$ na poltraku $O_1P_1'$. Iz $\mathcal{I}:P',O,Q\mapsto P_2',O_1,Q_2$ sledi, da izometrija $\mathcal{I}$ preslika kot $P'OQ$ v kot $P_2'O_1Q_2$  (izrek \ref{izrekIzoB}), zato je
 $\angle P'OQ\cong \angle P_2'O_1Q_2=\angle P_1'O_1Q_1$.
\kdokaz


                \bizrek \label{sovrsnaSkladna}
               Vertical angles are congruent.
                \eizrek

\begin{figure}[!htb]
\centering
\input{sl.aks.2.3.20b.pic}
\caption{} \label{sl.aks.2.3.20b.pic}
\end{figure}


\textbf{\textit{Proof.}} Naj bosta $\alpha=\angle POQ$ in $\alpha'=\angle P'OQ'$ sovršna kota, kjer so točke $P$, $O$, $P'$ (oz. $Q$, $O$, $Q'$) kolinearne (Figure \ref{sl.aks.2.3.20b.pic}). Kot $\beta=\angle QOP'$ je sokot za oba kota $\alpha$ in $\alpha'$. Ker je še $\beta\cong\beta$, je po prejšnjem izreku \ref{sokota} tudi $\alpha\cong\alpha'$.
\kdokaz



            \bizrek \label{sredZrcObstoj}
            For each point $S$ there exists an isometry $\mathcal{I}$ such that $\mathcal{I}(S)=S$.
            In addition, for each point $X\neq S$ the following holds:\\ if $\mathcal{I}(X)=X'$, then $S$ is the midpoint of the line segment $XX'$.
            \eizrek

\begin{figure}[!htb]
\centering
\input{sl.aks.2.3.20c.pic}
\caption{} \label{sl.aks.2.3.20c.pic}
\end{figure}


\textbf{\textit{Proof.}} Naj bo $P$ poljubna točka različna od $S$  (Figure \ref{sl.aks.2.3.20c.pic}). Po aksiomu \ref{AksII3} obstaja na premici $SP$ takšna točka $Q$, da velja $\mathcal{B}(P,S,Q)$. Označimo polravnini, ki ju določa rob $SP$ z $\alpha$ in $\alpha'$. Po aksiomu \ref{aksIII2} obstaja (ena sama) izometrija $\mathcal{I}$, ki preslika točko $S$ v točko $S$, poltrak $SP$ v poltrak $SQ$ in polravnino $\alpha$ v polravnino $\alpha'$.

Označimo s $p$ premico $SP$.
Točka $P'=\mathcal{I}(P)$ leži na poltraku $SQ$ oz. premici $p$. Ker je torej  $\mathcal{I}:S,P \mapsto S,P'$, se po aksiomu \ref{aksIII1} premica $SP$ preslika v premico $SP'$ oz. $\mathcal{I}:p\rightarrow p$.
Slika polravnine $\alpha'$ z robom $p$ je torej polravnina z istim robom (izrek \ref{izrekIzoB}). Ta polravnina ne more biti $\alpha'$, saj je izometrija  $\mathcal{I}$ bijektivna preslikavain po predpostavki  preslika polravnino  $\alpha$ v polravnino $\alpha'$.  Torej je $\mathcal{I}:\alpha'\rightarrow \alpha$.

Sedaj je jasno, da je brez škode za splošnost dovolj, če izpeljemo dokaz le za točke, ki ležijo v polravnini $\alpha$ (brez roba oz. le poltraka $SP$).

Naj bo $X\in \alpha\setminus p$ in $X'=\mathcal{I}(X)$. Takoj vidimo,da je $X'\in \alpha'\setminus p$. Po aksiomu \ref{AksII3} obstaja na premici $SX$ takšna točka $X_1$, da velja $\mathcal{B}(X,S,X_1)$. Ker sta $\angle PSX$ in $\angle P'SX_1$ sovršna kota, sta po izreku \ref{sovrsnaSkladna} tudi skladna. Toda iz $\mathcal{I}:S,P,X \mapsto S,P',X'$ sledi tudi
 $\angle PSX \cong \angle P'SX'$. Torej velja $\angle P'SX_1\cong \angle P'SX'$ (izrek \ref{sklRelEkv})), zato sta po izreku \ref{KotNaPoltrak} poltraka $SX_1$ in $SX'$ identična. To pomeni, da točka $X'$ leži na poltraku $SX_1$ oz. velja $\mathcal{B}(X,S,X')$. Ker je zaradi $\mathcal{I}:S,X \mapsto S,X'$ še $SX\cong SX'$, je po definiciji točka $S$ središče daljice $XX'$.

 Naj bo na koncu $Y$ poljubna točka poltraka $SP$, ki se razlikuje od točke $S$, in $Y'=\mathcal{I}(Y)$. Točka $Y'$ leži na poltraku $SQ$, zato je $\mathcal{B}(Y,S,Y')$. Ker je zaradi $\mathcal{I}:S,Y \mapsto S,Y'$ še $SY\cong SY'$, je po definiciji točka $S$ središče daljice $YY'$.
\kdokaz

V razdelku \ref{odd6SredZrc} bomo izometrijo, ki je omenjena v prejšnjem izreku \ref{sredZrcObstoj}, posebej obravnavali.


 Definirajmo nove vrste kotov.
 Konveksni kot je \index{kot!ostri}\pojem{ostri kot}, \index{kot!pravi}
 \pojem{pravi kot} oz.
 \index{kot!topi}\pojem{topi kot}, če je
 manjši, enak oz. večji od svojega sokota (Figure \ref{sl.aks.2.3.21.pic}).


\begin{figure}[!htb]
\centering
\input{sl.aks.2.3.21.pic}
\caption{} \label{sl.aks.2.3.21.pic}
\end{figure}



Iz definicije sledi, da so ostri (oz. topi) koti tisti konveksni
koti, ki so manjši (oz. večji) od pravega kota.

Iz izreka \ref{izrekSimetralaKota} sledi, da pravi kot
obstaja, saj bisektrisa iztegnjeni kot razdeli na dva skladna
sokota.

Ni težko dokazati, da sta vsaka dva prava kota skladna ter da je
kot, ki je skladen s pravim kotom, tudi pravi kot.

Če je vsota dveh kotov  pravi kot, pravimo, da sta kota
\index{kota!komplementarna}\pojem{komplementarna} (Figure
\ref{sl.aks.2.3.22.pic}).


\begin{figure}[!htb]
\centering
\input{sl.aks.2.3.22.pic}
\caption{} \label{sl.aks.2.3.22.pic}
\end{figure}



Sedaj bomo vpeljali izjemno pomembno relacijo med premicama. Če
premici $p$ in $q$ vsebujeta kraka pravega kota, pravimo, da sta
$p$ in $q$ \pojem{pravokotni}, kar označimo $p \perp q$ (Figure
\ref{sl.aks.2.3.23.pic}).

\begin{figure}[!htb]
\centering
\input{sl.aks.2.3.23.pic}
\caption{} \label{sl.aks.2.3.23.pic}
\end{figure}

Iz same definicije je jasno, da je pravokotnost simetrična relacija
oz. iz $p \perp q$ sledi $q \perp p$. Če je $p \perp q$ in $p \cap q=S$,
pravimo, da je premica $p$
\index{pravokotni!premici}\pojem{pravokotna} na premico $q$ v točki
$S$ oz. da je $p$ \index{pravokotnica}\pojem{pravokotnica}
premice $q$ v tej točki.



Naslednji izrek je najpomembnejši izrek, ki karakterizira relacijo
pravokotnosti.



                \bizrek \label{enaSamaPravokotnica}
                For each point $A$ and each line $p$, there is a unique line $n$
            going through the point $A$, which is perpendicular on the line $p$.
                \eizrek

\textbf{\textit{Proof.}}
Predpostavimo, da točka $A$ ne leži na premici $p$. Naj bosta
$B$ in $C$ poljubni točki, ki ležita na premici $p$
 (Figure \ref{sl.aks.2.3.24.pic}). S $\pi$ označimo polravnino $BCA$,
komplementarno polravnino pa s $\pi_1$. Po izreku \ref{izomEnaC'}
obstaja ena sama točka $A_1\in \pi_1$, za katero velja $(A,B,C)
\cong (A_1,B,C)$. Iz izreka \ref{IizrekABC} sledi, da obstaja ena
sama izometrija $\mathcal{I}$, ki preslika točke $A$, $B$ in $C$
v točke $A_1$, $B$ in $C$. Premico $AA_1$ označimo z $n$. Ker
sta $A$ in $A_1$ na različnih bregovih premice $p$, premica $n$
seka premico $p$ v neki točki $S$. Iz $\mathcal{I}:B,C \mapsto
B,C$ sledi $\mathcal{I}(S)=S$ (izrek \ref{izoABAB}). Torej
izometrija $\mathcal{I}$ preslika kot $ASB$ v kot $A_1SB$. Sledi,
da sta $\angle ASB$ in $\angle A_1SB$ skladna sokota, zato sta
tudi prava kota. Torej je $n \perp p$.


\begin{figure}[!htb]
\centering
\input{sl.aks.2.3.24.pic}
\caption{} \label{sl.aks.2.3.24.pic}
\end{figure}

Dokažimo, da je $n$ edina pravokotnica premice $p$ skozi točko
$A$. Naj bo $\widehat{n}$ premica, za katero je tudi $A\in
\widehat{n}$ in $\widehat{n} \perp p$. Naj bo točka $\widehat{S}$
presečišče premic $\widehat{n}$ in $p$. Po predpostavki je
$\angle A\widehat{S}B$ pravi kot in je skladen s svojim sokotom
$\angle B\widehat{S}A_2$ ($A_2$ je takšna točka, da velja
$\mathcal{B}(A,\widehat{S},A_2)$), ki je tudi pravi kot.

Iz $\mathcal{I}:B,C \mapsto B,C$ sledi
$\mathcal{I}(\widehat{S})=\widehat{S}$ (izrek \ref{izoABAB}).
Torej izometrija $\mathcal{I}$ preslika kot $A\widehat{S}B$ v kot
$A_1\widehat{S}B$. Sledi, da sta $\angle A\widehat{S}B$ in $\angle
A_1\widehat{S}B$ skladna, zato je tudi $\angle A_1\widehat{S}B$
pravi kot. Torej sta kota $A_1\widehat{S}B$ in $A_2\widehat{S}B$
prava kota in sta zato skladna. Iz tega sledi, da sta poltraka
$\widehat{S}A_1$ in $\widehat{S}A_2$ ista, zato je $A_1 \in
\widehat{S}A_2=\widehat{n}$ oz. $\widehat{n}=AA_1=n$.

 V primeru, ko točka $A$ leži na premici $p$, je pravokotnica $n$
 simetrala pripadajočega iztegnjenega kota (izrek \ref{izrekSimetralaKota}).
\kdokaz


Prejšnji izrek ima za posledico zelo pomembno dejstvo - obstoj parov disjunktnih premic v ravnini - oz. takšnih, ki nimata skupnih točk. To je vsebina naslednjih dveh izrekov.


            \bizrek \label{absolGeom1}
             Let $p$ and $q$ be a lines perpendicular on a line $PQ$ in the points $P$ and $Q$.
            Then the lines $p$ and $q$ do not have a common points i.e. $p\cap q=\emptyset$.
            \eizrek

\begin{figure}[!htb]
\centering
\input{sl.aks.2.3.25b.pic}
\caption{} \label{sl.aks.2.3.25b.pic}
\end{figure}

\textbf{\textit{Proof.}} Izrek je direktna posledica prejšnjega izreka \ref{enaSamaPravokotnica}. Če bi se namreč premici $p$ in $q$ sekali v neki točki $S$, bi iz točke $S$ imeli dve pravokotnici na premico $PQ$ (Figure \ref{sl.aks.2.3.25b.pic}), kar je v protislovju z omenjenim izrekom.
 \kdokaz



            \bizrek \label{absolGeom}
            If $A$ is a point that does not lie on a line $p$, then there exists at least
            one line (in the same plane) passing through the point $A$ and not intersecting the line
            $p$ (Figure \ref{sl.aks.2.3.25a.pic}).
            \eizrek


\begin{figure}[!htb]
\centering
\input{sl.aks.2.3.25a.pic}
\caption{} \label{sl.aks.2.3.25a.pic}
\end{figure}


\textbf{\textit{Proof.}} Po izreku \ref{enaSamaPravokotnica} obstaja (natanko ena) pravokotnica $n$ premice $p$, ki poteka skozi točko $A$. Označimo z $A'$ presečišče premic $p$ in $n$. Iz istega izreka sledi, da obstaja še pravokotnica $q$ premice $n$ v točki $A$. Po prejšnjem izreku \ref{absolGeom1} je $q$ premica, ki poteka skozi točko $A$ in nima skupnih točk s premico $p$.
 \kdokaz


Točka $A'$ je \index{nožišče}\pojem{nožišče} ali
\index{pravokotna projekcija}\pojem{pravokotna projekcija} točke
$A$ na premico $p$, če pravokotnica premice $p$ skozi točko $A$
seka to premico v točki $A'$. Označili jo bomo  z $A'=pr_{\perp
p}(A)$ (Figure \ref{sl.aks.2.3.25.pic}).
 Iz prejšnjega izreka sledi, da za vsako točko in premico
 obstaja ena sama pravokotna projekcija.

\begin{figure}[!htb]
\centering
\input{sl.aks.2.3.25.pic}
\caption{} \label{sl.aks.2.3.25.pic}
\end{figure}

Premico, ki poteka skozi središče $S$ daljice $AB$ in je pravokotna
na premico $AB$, imenujemo \index{simetrala!daljice}\pojem{simetrala daljice} $AB$ in jo označimo s $s_{AB}$
(Figure \ref{sl.aks.2.3.26.pic}). Lastnosti simetrale daljice bomo
obravnavali v naslednjem poglavju.

\begin{figure}[!htb]
\centering
\input{sl.aks.2.3.26.pic}
\caption{} \label{sl.aks.2.3.26.pic}
\end{figure}


Pravimo, da je točka $A$ \index{simetričnost!glede na premico}\pojem{simetrična} točki $B$ glede na premico $s$, če je $s$ simetrala daljice $AB$. Simetričnost glede na premico (kot preslikavo - t. i. osno zrcaljenje) bomo podrobneje obravnavali v razdelku \ref{odd6OsnZrc}.

Naj bo $S$ točka in $AB$ daljica. Množico vseh točk $X$, za
katere velja $SX \cong AB$, imenujemo
\index{krožnica}\pojem{krožnica} s
\index{središče!krožnice}\pojem{središčem} $S$ in \index{polmer
krožnice}\pojem{polmerom} $AB$; označimo jo s $k(S,AB)$ (Figure
\ref{sl.aks.2.3.27.pic})  oz.:
 $$k(S,AB)=\{X;\hspace*{1mm}SX \cong AB\}.$$


\begin{figure}[!htb]
\centering
\input{sl.aks.2.3.27.pic}
\caption{} \label{sl.aks.2.3.27.pic}
\end{figure}

 Seveda je krožnica množica
 točk v ravnini, ker v tej knjigi obravnavamo le ravninsko
 geometrijo (vse točke in vsi liki pripadajo isti ravnini).

 Iz definicije je jasno, da za polmer lahko izberemo poljubno daljico,
 ki je skladna z
daljico $AB$, torej katerokoli daljico $SP$, kjer je $P$ poljubna
točka na krožnici. Ker polmer ni vezan na
določeno daljico, ga običajno označujemo z malo črko $r$. Torej lahko krožnico zapišemo tudi takole:
 $$k(S,r)=\{X;\hspace*{1mm}SX \cong r\}.$$
 Množico

$$\{X;\hspace*{1mm}SX \leq r\}$$
imenujemo \index{krog}\pojem{krog} s središčem $S$ in polmerom $r$ (Figure \ref{sl.aks.2.3.28.pic}) označimo ga s $\mathcal{K}(S,r)$.
Množica
 $$\{X;\hspace*{1mm}SX < r\}$$
 je \index{notranjost!kroga}
 \pojem{notranjost kroga} $\mathcal{K}(S, r)$, njene točke so pa
 \pojem{notranje točke kroga}.
 To pomeni, da je krog pravzaprav unija svoje notranjosti in pripadajoče krožnice.

Množico
 $$\{X;\hspace*{1mm}SX > r\}$$
 imenujemo \index{zunanjost!kroga}\pojem{zunanjost kroga} $\mathcal{K}(S, r)$
  in njene točke \pojem{zunanje točke kroga}.

 Iz praktičnih razlogov bomo notranjost kroga $\mathcal{K}(S, r)$ imenovali tudi \index{notranjost!krožnice}\pojem{notranjost} pripadajoče krožnice $k(S,r)$, zunanjost kroga $\mathcal{K}(S, r)$ pa  \index{zunanjost!krožnice}\pojem{zunanjost} pripadajoče krožnice $k(S,r)$. Enako definiramo tudi \pojem{notranje} oz. \pojem{zunanje} točke krožnice.

\begin{figure}[!htb]
\centering
\input{sl.aks.2.3.28.pic}
\caption{} \label{sl.aks.2.3.28.pic}
\end{figure}

  Če sta $P$ in $Q$ dve točki krožnice $k(S, r)$, daljico $PQ$
   imenujemo \index{tetiva krožnice} \pojem{tetiva} te
  krožnice. Če
tetiva vsebuje središče krožnice, jo imenujemo
\index{premer krožnice}\pojem{premer} ali \index{diameter
krožnice}\pojem{diameter} te krožnice (Figure
\ref{sl.aks.2.3.29.pic}).

\begin{figure}[!htb]
\centering
\input{sl.aks.2.3.29.pic}
\caption{} \label{sl.aks.2.3.29.pic}
\end{figure}

Dokažimo naslednji izrek.


             \bizrek \label{premerInS}
            The centre $S$ of the circle $k(S, r)$ is at the same time the midpoint of each
            diameter of that circle.
            \eizrek

\textbf{\textit{Proof.}}

\begin{figure}[!htb]
\centering
\input{sl.aks.2.3.30.pic}
\caption{} \label{sl.aks.2.3.30.pic}
\end{figure}

 Če je $PQ$ premer  krožnice $k(S, r)$, točki $P$ in $Q$ ležita na
krožnici, kar pomeni: $SP \cong SQ \cong r$ (Figure
\ref{sl.aks.2.3.30.pic}). Ker točka $S$ leži na daljici $PQ$,
sledi, da je točka $S$ središče te daljice.
 \kdokaz

 Iz prejšnjega izreka sledi, da je premer enak dvema polmeroma, ker je:
$PQ = PS + SQ = 2\cdot PS = 2\cdot r$. To pa pomeni,
da so vsi premeri neke krožnice med seboj skladni.


Naj bosta $P$ in $Q$ poljubni točki krožnice  $k(S, r)$. Presek
krožnice $k$ z eno od polravnin (v ravnini te krožnice) z robom
$s=PQ$ imenujemo \index{krožni!lok} \pojem{krožni lok} $PQ$ (ali
krajše \pojem{lok}) s krajiščema $P$ in $Q$.

Torej vsaka tetiva $PQ$ na neki krožnici $k$ določa dva loka. Predpostavimo, da središče $S$ ne leži na robu polravnine, ki generira krožni lok.
Če ta polravnina
vsebuje središče $S$ krožnice $k$,
gre za \pojem{veliki lok} $PQ$, sicer je to \pojem{mali
lok} $PQ$.
 Če pa je
  središče $S$ na samem robu $PQ$ polravnine, potem je vsak od obeh lokov $PQ$
 \index{polkrožnica}\pojem{polkrožnica} $PQ$ (Figure
\ref{sl.skk.4.2.1.pic}).

\begin{figure}[!htb]
\centering
\input{sl.skk.4.2.1.pic}
\caption{} \label{sl.skk.4.2.1.pic}
\end{figure}

Ker lok le s svojima krajiščema ni enolično določen, moramo na krožnici poznati še
vsaj eno točko, ki temu loku pripada oz. ne pripada.

Na podoben način definiramo tudi določene dele kroga.

Naj bosta $P$ in $Q$ poljubni točki krožnice  $k(S, r)$. Presek
kroga $\mathcal{K}(S, r)$ z eno od polravnin (v ravnini te krožnice) z robom
$s=PQ$ imenujemo
\index{krožni!odsek} \pojem{krožni odsek}.
 Torej vsaka tetiva $PQ$ na neki krožnici $k(S, r)$ določa na krogu $\mathcal{K}(S, r)$ dva krožna odseka. Predpostavimo, da središče $S$ ne leži na robu polravnine, ki generira krožni odsek.
Če ta polravnina
vsebuje  središče $S$ krožnice $k$,
gre za \pojem{večji krožni odsek} $PQ$, sicer pa za \pojem{manjši
krožni odsek} $PQ$.
 Če je
  središče $S$ na samem robu $PQ$ polravnine, potem je vsak od obeh krožnih odsekov
\index{polkrog}\pojem{polkrog}.
 Iz definicije je jasno, da je rob krožnega odseka unija tetive $PQ$ in ustreznega loka (Figure
\ref{sl.skk.4.2.1b.pic}).

\begin{figure}[!htb]
\centering
\input{sl.skk.4.2.1b.pic}
\caption{} \label{sl.skk.4.2.1b.pic}
\end{figure}


Definirajmo še en pojem, ki je povezan s krogom.
Naj bosta $P$ in $Q$ poljubni točki krožnice  $k(S, r)$. Presek
kroga $\mathcal{K}(S, r)$ z enim od kotov $PSQ$ imenujemo
\index{krožni!izsek} \pojem{krožni izsek}. Tudi v tem primeru imamo dva krožna izseka. Če je kot $PSQ$ iztegnjeni kot, dobimo dva polkroga, sicer pa konveksni in nekonveksni krožni izsek, odvisno od tega, ali je kot $PSQ$ konveksen ali nekonveksen (Figure
\ref{sl.skk.4.2.1c.pic}).

\begin{figure}[!htb]
\centering
\input{sl.skk.4.2.1c.pic}
\caption{} \label{sl.skk.4.2.1c.pic}
\end{figure}





%________________________________________________________________________________
 \poglavje{Continuity Axiom} \label{odd2AKSZVE}

 Že v osnovni šoli smo pri uvajanju številske premice in
 koordinatnega sistema izvedeli,
 da je možno
vzpostaviti povezavo, pri kateri vsaki točki neke premice pripada
določeno realno število in obratno, vsakemu realnemu številu lahko
priredimo točko, ki leži na tej premici. S tem je povezan naslednji
aksiom.

 \baksiom \label{aksDed}\index{aksiom!Dedekindov}
  (Dedekind’s\footnote{\index{Dedekind, R.}
 \textit{R. Dedekind} (1831--1916),
 nemški matematik.}
  axiom)
  Suppose that all points on open line segment $AB$ are divided into the union of two nonempty disjoint sets $\mathcal{U}$ and
$\mathcal{V}$ such that no point of $\mathcal{U}$ is
  between two points of  $\mathcal{V}$ and vice versa: no point of $\mathcal{V}$ is
  between two points of  $\mathcal{U}$. Then there is a unique point $C$ on open line segment $AB$ such that
  $B(A',C,B')$ for any two points $A'\in
\mathcal{U}\setminus{C}$ and $B'\in \mathcal{V} \setminus {C}$
(Figure \ref{sl.aks.2.4.1.pic}).
 \eaksiom


\begin{figure}[!htb]
\centering
\input{sl.aks.2.4.1.pic}
\caption{} \label{sl.aks.2.4.1.pic}
\end{figure}

Povejmo brez dokazov dve pomembni posledici aksioma
zveznosti\footnote{Vse do 19. stoletja matematiki niso
čutili potrebe, da bi dokazali ti dve trditvi, oz. potrebe za
uvajanjem aksioma zveznosti. Celo \index{Evklid} Evklid iz
Aleksandrije (3. stol. pr. n. š.) v svojem znanem delu
‘‘Elementi’’, navaja konstrukcijo enakostraničnega trikotnika, pri kateri ne
dokazuje, da se določeni krožnici sekata.}.



             \bizrek \label{DedPoslKrozPrem} Let $k$ be a circle and $P$ a point inside that circle.
            Then any line $p$ passing through the point $P$ and the circle $k$ has exactly two common points (Figure
            \ref{sl.aks.2.4.2.pic}).
             \eizrek

\begin{figure}[!htb]
\centering
\input{sl.aks.2.4.2.pic}
\caption{} \label{sl.aks.2.4.2.pic}
\end{figure}



            \bizrek \label{DedPoslKrozKroz} If $k$ and $l$ are circles such that $l$ contains at least one point inside and one point outside $k$,
             then the circles has exactly two points (Figure \ref{sl.aks.2.4.3.pic}).
            \eizrek

\begin{figure}[!htb]
\centering
\input{sl.aks.2.4.3.pic}
\caption{} \label{sl.aks.2.4.3.pic}
\end{figure}

Dedekindov aksiom se v nekoliko drugačni obliki uporablja pri
zasnovi množice realnih števil. To nas spominja na že omenjeno
povezavo med množico točk neke premice in množico realnih števil.

Operacijo množenja daljice $AB$ s poljubnim
pozitivnim racionalnim številom $q$ smo že definirali. Sedaj lahko razširimo pojem
množenja za poljubno pozitivno realno število $\lambda$. Definicija daljice
$\lambda\cdot AB$, ($\lambda \in \mathbb{R}^+$), ki jo tu formalno ne
bomo izpeljali do konca, je povezana z dvema množicama točk
na poltraku $CD$:
 \begin{eqnarray*}
&& \{X\in CD;\hspace*{1mm}CX=q\cdot
AB,\hspace*{1mm}q\leq\lambda,\hspace*{1mm}q\in
\mathbb{Q}^+ \} \hspace*{1mm}\textrm{ in}\\
&& \{X\in CD;\hspace*{1mm}CX=q\cdot
AB,\hspace*{1mm}q>\lambda,\hspace*{1mm}q\in \mathbb{Q}^+ \}
 \end{eqnarray*}
 ter
Dedekindovim aksiomom \ref{aksDed}.

 S pomočjo aksioma zveznosti \ref{aksDed} lahko vpeljemo tudi
 pojme merjenja daljic in kotov.

 Pri merjenju daljic bomo vsaki daljici $AB$ priredili
 pozitivno realno  število $\textsl{m}(AB)$ na naslednji način.
  Naj bo $\mathcal{D}$ množica vseh daljic in $\mathbb{R}^+$ množica vseh pozitivnih realnih števil. Preslikavo $\textsl{m}:\mathcal{D}\rightarrow\mathbb{R}^+$, ki izpolnjuje naslednje lastnosti:
  \begin{itemize}
    \item $(\exists A_0B_0\in\mathcal{D})\hspace*{1mm}\textsl{m}(A_0B_0)=1$,
    \item $(\forall AB, CD\in\mathcal{D})\hspace*{1mm}(AB\cong CD \Rightarrow\textsl{m}(AB)=\textsl{m}(CD))$,
    \item $(\forall AB, CD, EF\in\mathcal{D})\hspace*{1mm}(AB+CD=EF\Rightarrow \textsl{m}(AB)+\textsl{m}(CD)=\textsl{m}(EF))$,
  \end{itemize}
  imenujemo \index{dolžina!daljice}\pojem{dolžina daljice} ali \index{mera!daljice}\pojem{mera daljice}, trojico $\textsl{M}=(\mathcal{D},\mathbb{R}^+,\textsl{m})$ pa \index{sistem merjenja!daljic}\pojem{sistem merjenja daljic}.

  Dolžino daljice $AB$ (oz. $\textsl{m}(AB)$) bomo navadno označevali z $|AB|$.

  Intuitivno je jasno, da obstaja neskončno mnogo sistemov merjenja, ki pa so odvisni od izbire enotske daljice $A_0B_0$ - tiste, ki ima dolžino enako 1, oz. $\textsl{m}(A_0B_0)=1$. V nekem sistemu merjenja je torej dolžina poljubne daljice $AB$ predstavljena s pozitivnim realnim  številom $x$, za katerega je $AB=x\cdot A_0B_0$. Torej je $\textsl{m}(AB)=x$ natanko tedaj, ko je $AB=x\cdot A_0B_0$ (Figure \ref{sl.aks.2.4.4.pic}). Sedaj je jasno, zakaj potrebujemo aksiom zveznosti - brez njega bi imeli težave z definicijo dolžine diagonale kvadrata, ki ima za stranico enotsko daljico (dolžine 1)\footnote{Stari Grki so si dolžino vedno predstavljali kot  racionalno število, zato so stranico
in diagonalo kvadrata imenovali \pojem{neprimerljivi daljici}.}.


\begin{figure}[!htb]
\centering
\input{sl.aks.2.4.4.pic}
\caption{} \label{sl.aks.2.4.4.pic}
\end{figure}


        \bizrek \label{meraDalj1}
        Za poljuben sistem merjenja daljic velja:

          (\textit{i}) $AB<CD\Rightarrow |AB|<|CD|$;

          (\textit{ii}) $|AB|=|CD|\Rightarrow AB\cong CD$.
        \eizrek

    \textbf{\textit{Proof.}}

\begin{figure}[!htb]
\centering
\input{sl.aks.2.4.5.pic}
\caption{} \label{sl.aks.2.4.5.pic}
\end{figure}

          (\textit{i}) Iz $AB<CD$ sledi, da na daljici $CD$ obstaja takšna točka $T$, da velja $CT\cong AB$. Ker je $\mathcal{B}(C,T,D)$, je jasno $CD=CT+TD$ (Figure \ref{sl.aks.2.4.5.pic}). Iz definicije mere je potem: $|CD|=|CT|+|TD|=|AB|+|TD|>|AB|$.

\begin{figure}[!htb]
\centering
\input{sl.aks.2.4.6.pic}
\caption{} \label{sl.aks.2.4.6.pic}
\end{figure}

    (\textit{ii}) Predpostavimo, da ni $AB\cong CD$. Brez škode za splošnost naj bo $AB<CD$. Toda v tem primeru iz dokazanega (\textit{i}) sledi $|AB|<|CD|$, kar je v protislovju s predpostavko $|AB|=|CD|$. Torej velja $AB\cong CD$ (Figure \ref{sl.aks.2.4.6.pic}).
    \kdokaz

Ker sta po definiciji dolžine in prejšnjem izreku \ref{meraDalj1} daljici skladni natanko tedaj, ko imata enako dolžino, bomo pri krožnici $k(S,r)$  na njen \index{polmer krožnice}polmer $r$ pogosto gledali kot na dolžino tega polmera.

Naslednji izrek bomo podali brez dokaza.

            \bizrek \label{meraDaljice}
            Naj bo $\textsl{m}:\mathcal{D}\rightarrow\mathbb{R}^+$ mera daljice. Preslikava $\textsl{m}_1:\mathcal{D}\rightarrow\mathbb{R}^+$ tudi predstavlja  mero natanko tedaj, ko obstaja takšno pozitivno realno število $\mu$, da za vsako daljico $AB$ velja:
            $$\textsl{m}_1(AB)=\mu\cdot\textsl{m}(AB).$$
            \eizrek

Mera daljice nam omogoča definicijo novega pojma.
 \index{razmerje!daljic}\pojem{Razmerje daljic} $AB$ in $CD$ z oznakama $AB:CD$ oz. $\frac{AB}{CD}$ je količnik dolžin teh dveh daljic (Figure \ref{sl.aks.2.4.7a.pic}). Torej:
  $$AB:CD=\frac{AB}{CD}=\frac{|AB|}{|CD|}.$$

\begin{figure}[!htb]
\centering
\input{sl.aks.2.4.7a.pic}
\caption{} \label{sl.aks.2.4.7a.pic}
\end{figure}

  Jasno je, da mora biti razmerje dveh daljic vedno isto število, neodvisno od sistema merjenja.

 Korektnost prejšnje definicije bomo torej potrdili z naslednjim izrekom.

            \bizrek
            Razmerje dveh daljic ni odvisno od sistema merjenja.
            \eizrek


    \textbf{\textit{Proof.}} Naj bosta $(\mathcal{D},\mathbb{R}^+,\textsl{m})$ in $(\mathcal{D},\mathbb{R}^+,\textsl{m}_1)$ dva sistema merjenja. Po prejšnjem izreku \ref{meraDaljice} obstaja nek $\mu\in\mathbb{R}^+$, da je $\textsl{m}_1(PQ)=\mu\cdot\textsl{m}(PQ)$ za poljubno daljico $PQ$. Za poljubni daljici $AB$ in $CD$ zato velja:
     $$\frac{\textsl{m}_1(AB)}{\textsl{m}_1(CD)}=
     \frac{\mu\cdot\textsl{m}(AB)}{\mu\cdot\textsl{m}(CD)}=
     \frac{\textsl{m}(AB)}{\textsl{m}(CD)},$$ kar je bilo treba dokazati. \kdokaz


 Pojem delitve daljice v danem razmerju lako razširimo, tako da bo razmerje pozitivno realno število.
  Pravimo, da točka $T$ deli daljico $AB$ v \index{delitev daljice!v razmerju}\pojem{razmerju} $\lambda \in \mathbb{R}^+$, če je $\mathcal{B}(A,T,B)$ in $\frac{AT}{TB}=\lambda$ (Figure \ref{sl.aks.2.4.7.pic}).

\begin{figure}[!htb]
\centering
\input{sl.aks.2.4.7.pic}
\caption{} \label{sl.aks.2.4.7.pic}
\end{figure}

 Enakost dveh razmerij bomo imenovali \pojem{sorazmerje}. Torej so daljice $AB$, $CD$, $EF$ in $GH$ (v tem vrstnem redu)  sorazmerne, če je:
  $$\frac{AB}{CD}=\frac{EF}{GH}.$$


Na podoben način definiramo tudi mero kota.

  Naj bo $\mathcal{K}$ množica vseh kotov in $\mathbb{R}^+$ množica vseh realnih pozitivnih števil. Preslikavo $\textsl{l}:\mathcal{K}\rightarrow\mathbb{R}^+$, ki izpolnjuje naslednje lastnosti:
  \begin{itemize}
    \item $(\exists \alpha_0\in\mathcal{K})\hspace*{1mm}\textsl{l}(\alpha_0)=1$,
    \item $(\forall \alpha, \beta\in\mathcal{K})\hspace*{1mm}(\alpha\cong \beta \Rightarrow\textsl{l}(\alpha)=\textsl{l}(\beta))$,
    \item $(\forall \alpha, \beta, \gamma\in\mathcal{K})\hspace*{1mm}(\alpha+\beta=\gamma\Rightarrow \textsl{l}(\alpha)+\textsl{l}(\beta)=\textsl{l}(\gamma))$.
  \end{itemize}
  imenujemo \index{mera!kota}\pojem{mera kota}, trojico $\textsl{L}=(\mathcal{K},\mathbb{R}^+,\textsl{l})$ pa \index{sistem merjenja!kotov}\pojem{sistem merjenja kotov}.

  Dejstvo, da je mera kota $\alpha$ enaka $x$ (oz. $\textsl{l}(\alpha)=x$) bomo bolj pogosto zapisali v obliki $\alpha=x$.


  Podobno kot pri merjenju daljic obstaja neskončno mnogo sistemov merjenja kotov, ki pa so odvisni od enotskega kota $\alpha_0$ - tistega, za katerega je mera enaka 1, oz. $\textsl{l}(\alpha_0)=1$. Tako je v nekem sistemu merjenja mera poljubnega kota $\alpha$  pozitivno realno  število $x$, za katero je $\alpha=x\cdot \alpha_0$. Torej je $\textsl{l}(\alpha)=x$ natanko tedaj, ko je $\alpha=x\cdot \alpha_0$. Seveda moramo tudi v tem primeru  uporabiti aksiom zveznosti, da lahko vpeljemo množenje kota s pozitivnim realnim številom.

  Od vseh sistemov merjenja bomo izpostavili dva.
  \begin{itemize}
    \item Pri prvem sistemu, ki ga bomo uporabljali najpogosteje, je enotski kot 180-ti del iztegnjenega kota. Za ta kot bomo rekli, da meri \pojem{eno kotno stopinjo} in to mero označili z $1^0$ (Figure \ref{sl.aks.2.4.8.pic}). Če torej uporabimo lastnosti funkcije mere $\textsl{l}$,  v tem sistemu iztegnjeni kot meri $180^0$, pravi kot pa $90^0$.
    \item V drugem sistemu merjenja z \pojem{radiani}, z oznako $[\textrm{rad}]$, meri iztegnjeni kot $\pi$ (kjer je $\pi$ iracionalno število - $\pi\doteq 3,14$), pravi kot pa $\frac{\pi}{2}$. V tem sistemu merjenja kotov oznake $[\textrm{rad}]$ ponavadi ne pišemo.
  \end{itemize}


\begin{figure}[!htb]
\centering
\input{sl.aks.2.4.8.pic}
\caption{} \label{sl.aks.2.4.8.pic}
\end{figure}

  Torej lahko v obeh sistemih mero iztegnjenega kota $\alpha$  zapišemo  $\alpha=180^0=\pi$, pravega kota $\beta$ pa $\beta=90^0=\frac{\pi}{2}$ (Figure \ref{sl.aks.2.4.8a.pic}).

\begin{figure}[!htb]
\centering
\input{sl.aks.2.4.8a.pic}
\caption{} \label{sl.aks.2.4.8a.pic}
\end{figure}

Splošno zvezo med obema sistemoma merjenja kotov lahko zapišemo s formulama:

$$1\hspace*{1mm}\textrm{rad}=\frac{180^0}{\pi}, \textrm{ oz. }
1^0=\frac{\pi}{180^0}\hspace*{1mm}\textrm{rad}.$$



%________________________________________________________________________________
 \poglavje{Playfair's Axiom} \label{odd2AKSVZP}


Kot posledico aksiomov prejšnjih štirih skupin smo že dokazali (izrek \ref{absolGeom}), da za točko $A$, ki ne leži na premici $p$, obstaja (v tej ravnini) vsaj ena premica $q$, ki poteka skozi točko $A$ in ne seka premice $p$ (Figure \ref{sl.aks.2.5.0.pic}).
 Toda ali je takšna premica ena sama? Intuitivno je odgovor pritrdilen. Toda ali lahko to dokažemo z dosedanjimi aksiomi?\footnote{To vprašanje je povezano z že omenjenim problemom petega evklidovega postulata in je bilo odprto skoraj 2000 let.} Izkaže se,  da na to vprašanje ne moremo odgovoriti, če ostanemo le pri prvih štirih skupinah aksiomov.\footnote{Tega dejstva sta se prva začela zavedati ruski matematik \textit{N. I. Lobačevski} (1792--1856) in madžarski matematik  \textit{J. Bolyai} (1802--1860). Prvi, ki je to formalno dokazal, je bil francoski matematik \index{Poincar\'{e}, J. H.} \textit{J. H.
Poincar\'{e}} (1854--1912).}  Dosedanji aksiomi tvorijo en nepopoln sistem aksiomov, ker obstaja trditev, ki jo v tej
teoriji lahko formuliramo, ne moremo pa ugotoviti, če velja ali ne velja.
  Torej je potrebno dodati nov aksiom, s katerim se bomo odločili, ali obstaja le ena takšna premica $q$ ali pa je takšnih premic več.


\begin{figure}[!htb]
\centering
\input{sl.aks.2.5.0.pic}
\caption{} \label{sl.aks.2.5.0.pic}
\end{figure}


        \baksiom \label{Playfair}\index{aksiom!Playfairjev}
         (Playfair's\footnote{
        \index{Playfair, J.}
        \textit{J. Playfair}
        (1748--1819), škotski matematik je predlagal to trditev kot ekvivalent petega evklidovega postulata.} axiom)
        For any given line $p$ and point $A$ not on $p$ (in the plane containing both line $p$ and point $A$), there is
        just one line  through the point $A$ that do not intersect the line $p$
         (Figure \ref{sl.aks.2.5.1.pic}).
        \eaksiom

\begin{figure}[!htb]
\centering
\input{sl.aks.2.5.1.pic}
\caption{} \label{sl.aks.2.5.1.pic}
\end{figure}

Drugo možnost ponuja naslednji aksiom.


        \baksiom \label{Lobac}\index{aksiom!Lobačevskega}
         (Lobachevsky's\footnote{Ruski matematik \textit{N. I. Lobačevski} (1792--1856) in madžarski matematik  \textit{J. Bolyai} (1802--1860) sta neodvisno drug od drugega zgradila prvo neevklidsko geometrijo, ki temelji na
         tem aksiomu in  aksiomih prvih štirih skupin.} axiom)
        For any given line $p$ and point $A$ not on $p$ (in the plane containing both line $p$ and point $A$), there are
        at least two lines  through the point $A$ that do not intersect the line $p$
         (Figure \ref{sl.aks.2.5.1.pic}).
        \eaksiom




Na ta način dobimo dve geometriji, od katerih je vsaka zase neprotislovna.
Prvo geometrijo, ki je določena z aksiomi prvih štirih skupin in s Playfairjevim aksiomom \ref{Playfair}, imenujemo
\index{geometrija!evklidska}\pojem{ravninska evklidska geometrija}. Drugo geometrijo, ki je določena z aksiomi prvih štirih skupin in
aksiomom Lobačevskega \ref{Lobac}, imenujemo \index{geometrija!hiperbolična}\pojem{ravninska hiperbolična
geometrija}.

Čeprav sta očitno različni, imata omenjeni geometriji veliko skupnega. To je jasno že zaradi tega, ker imata  prve štiri skupine aksiomov enake - razlikujeta se le v petem. Posledice prvih štirih
skupin aksiomov, ki smo jih do sedaj obravnavali, veljajo tako v
evklidski geometriji kot tudi v hiperbolični geometriji.
Geometrijo, ki temelji
 samo na  prvih štirih skupinah aksiomov, imenujemo
\index{geometrija!absolutna}\pojem{ravninska absolutna geometrija}. Le-ta določa skupne lastnosti evklidske in hiperbolične geometrije. Ker je sistem aksiomov, ki jo določa, nepopoln, pravimo, da je absolutna geometrija \pojem{nepopolna teorija}.

 Kljub podobnosti veljajo v
hiperbolični geometriji  na prvi pogled nenavadne trditve, kar je seveda posledica aksioma Lobačevskega \ref{Lobac} oz. negacija Playfairjevega aksioma \ref{Playfair}. Vsota notranjih kotov trikotnika je v hiperbolični geometriji
vedno manjša od $180^0$ in ni konstantna; pravokotnica enega kraka ostrega kota ne seka vedno drugega kraka, obstajajo celo trikotniki, za katere ne obstaja včrtana krožnica itd.
 Seveda se nam te trditve zdijo protislovne, toda protislovne so le v evklidski geometriji, v hiperbolični pa ne. Izkaže se, da je hiperbolična geometrija - enako kot evklidska  - neprotislovna teorija\footnote{Neprotislovnost hiperbolične geometrije je prvi dokazal francoski matematik \index{Poincar\'{e}, J. H.} \textit{J. H.
Poincar\'{e}} (1854--1912), ki je zgradil model hiperbolične geometrije v evklidski geometriji. Tako bi protislovje hiperbolične geometrije pomenilo protislovje tudi v evklidski geometriji.}.

Razen omenjenih geometrij obstajajo tudi druge neevklidske geometrije. Geometrija, v kateri se vsaki dve premici ravnine sekata, je t. i. \index{geometrija!eliptična}\pojem{eliptična geometrija\footnote{To geometrijo je razvijal nemški matematik \index{Riemann, G. F. B.} \textit{G. F.
B. Riemann} (1828--1866).}}. Iz že omenjenega izreka \ref{absolGeom} je jasno, da  eliptične geometrije ni možno graditi na aksiomih prvih štirih skupin oz. absolutne geometrije. V tem smislu se ta geometrija bolj razlikuje od evklidske in hiperbolične geometrije.

Omenimo še \index{geometrija!projektivna}\pojem{projektivno geometrijo}. Na določen način je ta od vseh omenjenih geometrij najbolj enostavna, saj temelji samo na treh skupinah aksiomov. Tudi v tej geometriji se vsaki dve premici sekata, toda za razliko od eliptične geometrije v njej nimamo definirane metrike oz. ni relacije skladnosti. V projektivni geometriji je mogoče narediti modele vseh treh geometrij: evklidske, hiperbolične in eliptične.


Še enkrat poudarimo, da v tej knjigi gradimo le ravninsko evklidsko geometrijo. Tako  smo že v začetku množico vseh točk $\mathcal{S}$ imenovali ravnina. V nadaljevanju  jo bomo označevali še z $\mathbb{E}^2$ (oz. $\mathbb{E}^2=\mathcal{S}$). Izbrali smo samo ravninske aksiome in tako dobili sistem aksiomov ravninske evklidske geometrije. Z nekoliko drugačno izbiro osnovnih pojmov (osnovno množico vseh točk $\mathcal{S}$ imenujemo prostor, določene podmnožice so premice in ravnine) in z dodajanjem novih aksiomov (potrebno bi jih bilo dodati že v prvi skupini) bi dobili \pojem{evklidsko geometrijo prostora} ali krajše \index{geometrija!evklidska}\pojem{evklidsko geometrijo}. Na podoben način bi dobili tudi \pojem{hiperbolično geometrijo}\index{geometrija!hiperbolična} in
\index{geometrija!absolutna}\pojem{absolutno geometrijo}.

Obravnavali bomo le evklidsko geometrijo ravnine, zato bomo vedno predpostavljali, da vse točke, premice in liki ležijo v isti ravnini (ker v tej geometriji drugih pravzaprav niti ni). Zaradi razumljivosti pa bomo  to dejstvo občasno ponovno poudarili.


V evklidski geometriji ravnine je sedaj možno vpeljati nov pojem. Pravimo, da sta premici $p$ in $q$ \index{vzporednost!premic}\pojem{vzporedni}, kar označimo $p\parallel q$, če sovpadata ali pa nimata skupnih točk.
 $$p\parallel q \hspace*{2mm} \Leftrightarrow \hspace*{2mm}p=q \hspace*{1mm}\vee \hspace*{1mm} p\cap q=\emptyset .$$

Seveda bi v evklidski geometriji (prostora) morali dodati še pogoj, da $p$ in $q$ ležita v isti ravnini.

Sedaj lahko Playfairjev aksiom  izrazimo tudi v naslednji obliki.



                \bizrek \label{Playfair1}
                For any given line $p$ and point $A$ not on $p$, there is
                just one line  through the point $A$ that is parallel to the line $p$
                (Figure \ref{sl.aks.2.5.1a.pic}).
                \eizrek


\begin{figure}[!htb]
\centering
\input{sl.aks.2.5.1a.pic}
\caption{} \label{sl.aks.2.5.1a.pic}
\end{figure}

 Dokažimo pomembno lastnost definirane relacije vzporednosti.



        \bizrek
        The relation of being parallel is an equivalence relation.
        \eizrek

\begin{figure}[!htb]
\centering
\input{sl.aks.2.5.2.pic}
\caption{} \label{sl.aks.2.5.2.pic}
\end{figure}

 \textbf{\textit{Proof.}} Potrebno  (in dovolj) je dokazati, da je relacija refleksivna, simetrična in tranzitivna (Figure \ref{sl.aks.2.5.2.pic}).


 (\textit{R}) Relacija je refleksivna že po sami definiciji, ker za vsako premico $p$ velja $p\parallel p$.

 (\textit{S}) Če je $p\parallel q$, je direktno po definiciji tudi $q\parallel p$, kar pomeni, da je relacija simetrična.

 (\textit{T}) Predpostavimo, da velja $p\parallel q$ in $q\parallel r$ (vse tri premice so v isti ravnini). Dokažimo, da velja tudi $p\parallel r$. Če vsaj dve od treh premic sovpadata, je dokaz trivialen. Predpostavimo, da se premici $p$ in $r$ sekata v neki točki $A$. V tem primeru potekata skozi točko $A$  vsaj dve premici, ki ne sekata premice $q$, kar je v nasprotju s Playfairjevim aksiomom \ref{Playfair}. Torej velja
 $p\parallel r$, kar pomeni, da je relacija tranzitivna.
\kdokaz

Relacijo vzporednosti bomo definirali tudi za poltrake in daljice.  Poltraka (oz. daljici) sta \index{vzporednost!poltrakov}\index{vzporednost!daljic}\pojem{vzporedna}, če sta njuni nosilki vzporedni premici.

Dokažimo pomemben izrek evklidske geometrije.



        \bizrek \label{KotiTransverzala}
        Let $l$ be a line intersecting two distant lines $a$ and $b$ in points $A$ and $B$, respectively.
        If $X\in a$ and $Y\in b$ are points such that $X,Y\div l$, then:
         $$\angle XAB\cong\angle YBA\hspace*{1mm} \Leftrightarrow \hspace*{1mm}  a\parallel b.$$
        \eizrek


\begin{figure}[!htb]
\centering
\input{sl.aks.2.5.3.pic}
\caption{} \label{sl.aks.2.5.3.pic}
\end{figure}

 \textbf{\textit{Proof.}}

 ($\Rightarrow$) Predpostavimo najprej, da je $\angle XAB\cong\angle YBA$. Označimo s $S$ središče daljice $AB$ in z $N$ pravokotno projekcijo točke $S$ na premico $a$ (Figure \ref{sl.aks.2.5.3.pic}).


 Po izreku \ref{sredZrcObstoj} obstaja takšna izometrija $\mathcal{I}$, ki preslika točko $S$ v točko $S$, za vsako točko $T\neq S$ in njeno sliko $T'=\mathcal{I}(T)$ pa velja, da je $S$ središče daljice $TT'$. Torej je najprej $\mathcal{I}:A,B\mapsto B,A$.
  Naj bo $\mathcal{I}(X)=X'$ in $\mathcal{I}(N)=M$. Označimo z $n$ premico $NM$. Dokažimo, da je $n$ skupna pravokotnica premic $a$ in $b$.

 Dokažimo najprej $\mathcal{I}:a\rightarrow b$ in $M\in b$. Ker je $S$ središče daljice $XX'$, je $X,X'\div S$ oz. $X,X'\div l$ in $X',Y\ddot{-} l$. Iz $\mathcal{I}:B,A,X\mapsto A,B,X'$ sledi $\angle XAB\cong \angle X'BA$. Ker je po predpostavki $\angle XAB\cong\angle YBA$, je tudi $\angle X'BA\cong\angle YBA$. Ker sta še točki $Y$ in $X'$  v isti polravnini $ABY$, po izreku \ref{KotNaPoltrak} poltraka $BX'$ in $BY$ sovpadata. To pomeni, da točka $X'$ leži na poltraku $BY$, zato je tudi $X'\in b$. Iz $\mathcal{I}:A,X\mapsto B,X'$ sedaj sledi (aksiom \ref{aksIII1}) $\mathcal{I}:a\rightarrow b$.
Ker je $N\in a$, je potem tudi $\mathcal{I}(N)\in \mathcal{I}(a)$ oz. $M\in b$.

Iz $\mathcal{I}:S,N\mapsto S,M$ po aksiomu \ref{aksIII1} sledi $\mathcal{I}:n\rightarrow n$.
Torej  velja $\mathcal{I}:a,n\rightarrow b,n$, zato je $\angle b,n\cong a,n=90^0$ oz. $b\perp n$.
Ker je $n$ skupna pravokotnica različnih premic $a$ in $b$, po izreku  \ref{absolGeom} premici $a$ in $b$ nimata skupnih točk, kar pomeni, da je $a\parallel b$.

\begin{figure}[!htb]
\centering
\input{sl.aks.2.5.3a.pic}
\caption{} \label{sl.aks.2.5.3a.pic}
\end{figure}

 ($\Leftarrow$) Naj bo sedaj $a\parallel b$ (Figure \ref{sl.aks.2.5.3a.pic}). Predpostavimo nasprotno - torej, da ni $\angle XAB\cong\angle YBA$. Po izreku \ref{KotNaPoltrak} obstaja tak poltrak $BZ$ v polravnini $ABY$, da velja $\angle ZBA\cong\angle XAB$. Premico $ZB$ označimo z $b'$. Iz prvega dela dokaza ($\Rightarrow$) sledi, da je $b'\parallel a$. Torej $b$ in $b'$ obe potekata skozi točko $B$ in sta vzporedni s premico $a$ oz. z njo nimata skupnih točk (ker je $a\neq b$). Po Playfairovem aksiomu to ni mogoče, kar pomeni, da je $b=b'$. Torej točka $Z$ leži na poltraku $BY$, zato je
 $\angle YBA=\angle ZBA \cong\angle XAB$.
  \kdokaz


Premico $l$, ki seka premici $a$ in $b$, imenujemo njuna \pojem{transverzala}. Kote, ki jih določa premica $l$ s premicama $a$ in $b$, pa imenujemo \index{koti!ob transverzali}\pojem{koti ob transverzali}. Iz prejšnjega izreka \ref{KotiTransverzala} in izreka \ref{sovrsnaSkladna} sledi, da sta vsaka dva kota ob transverzali $l$ vzporednih premic $a$ in $b$  bodisi skladna bodisi suplementarna (Figure \ref{sl.aks.2.5.3b.pic}).
Torej velja naslednja trditev.


\begin{figure}[!htb]
\centering
\input{sl.aks.2.5.3b.pic}
\caption{} \label{sl.aks.2.5.3b.pic}
\end{figure}


            \bizrek \label{KotiTransverzala1}
            If two parallel lines $a$ and $b$  are cut by a transversal $l$,
            then the angles on the transversal are either congruent or supplementary.
             \index{kota!z vzporednimi kraki}
             \eizrek



 Let's prove a generalization of the previous theorem.



            \bizrek \label{KotaVzporKraki}
            Angles with parallel sides are either congruent or supplementary.
             \index{kota!z vzporednimi kraki}
             \eizrek


\begin{figure}[!htb]
\centering
\input{sl.aks.2.5.4.pic}
\caption{} \label{sl.aks.2.5.4.pic}
\end{figure}


 \textbf{\textit{Proof.}} Naj bosta $\angle aSb$ in $\angle a'S'b'$ takšna kota, da velja $a\parallel a'$ in $b\parallel b'$ (Figure \ref{sl.aks.2.5.4.pic}).
  Če je $b'\parallel a$, po Playfairjevem aksiomu nosilki krakov $a'$ in $b'$ sovpadata. Iz tega enako sledi tudi za kraka $a$ in $b$, kar pomeni, da sta kota $\angle aSb$ in $\angle a'S'b'$ oba iztegnjena in skladna.
  S $S_1$ označimo presečišče nosilk krakov $a$ in $b'$. Po prejšnjem izreku \ref{KotiTransverzala1} sta vsaka dva kota ob transverzali $SS_1$ vzporednic, ki sta nosilki krakov $b$ in $b'$, ali skladna ali suplementarna. Enako velja tudi za poljubna kota ob transverzali $S_1S'$ vzporednic, ki sta nosilki krakov $a$ in $a'$. Iz tega sledi, da sta tudi kota $\angle aSb$ in $\angle a'S'b'$  ali skladna ali suplementarna.
 \kdokaz


V nadaljevanju se bomo ukvarjali z notranjimi in zunanjimi koti večkotnika. Začnimo s trikotnikom.


         \bizrek \label{VsotKotTrik} The sum of the interior angles of a triangle is equal to
         $180^0$.
          \eizrek

\begin{figure}[!htb]
\centering
\input{sl.aks.2.5.7.pic}
\caption{} \label{sl.aks.2.5.7.pic}
\end{figure}

 \textbf{\textit{Proof.}} Naj bo $ABC$ poljubni trikotnik  (Figure \ref{sl.aks.2.5.7.pic}). Po posledici Playfairjevega aksioma \ref{Playfair1} obstaja ena sama premica $l$, ki je vzporedna s premico $BC$. Naj bosta $Y$ in $Z$ takšni točki premice $l$, da velja $Y,B\div AC$ in $Z,C\div AB$. Premica $AB$ je transverzala vzporednic $BC$ in $l$. Ker je še  $Z,C\div AB$, je po izreku \ref{KotiTransverzala}
 $\angle ABC\cong\angle ZAB$. Podobno je premica $AC$ transverzala vzporednic $BC$ in $l$, zato iz  $Y,B\div AC$ in izreka \ref{KotiTransverzala} sledi $\angle BCA\cong\angle YAC$. Na koncu je:
  $$\angle ABC +\angle BAC +\angle BCA = \angle ZAB +\angle BAC +\angle CAY=\angle ZAY=180^0,$$ kar je bilo treba dokazati. \kdokaz

 Zelo koristna sta naslednja izreka.


          \bizrek \label{zunanjiNotrNotr}
          An exterior angle of a triangle is equal to the sum of the two opposite interior angles.
           \eizrek

 \textbf{\textit{Proof.}}  Naj bo $ABC$ poljubni trikotnik  (Figure \ref{sl.aks.2.5.6.pic}). Označimo njegove notranje kote ob ogliščih $A$, $B$ in $C$ z $\alpha$, $\beta$ in $\gamma$, ustrezne zunanje kote pa z  $\alpha'$, $\beta'$ in $\gamma'$. Brez škode za splošnost je dovolj dokazati, da velja $\alpha+\beta=\gamma'$.
 Iz prejšnjega izreka \ref{VsotKotTrik} sledi: $$\alpha+\beta+\gamma=180^0.$$ Ker za sta  zunanji in njemu priležni notranji kot po definiciji sokota, je tudi $$\gamma'+\gamma=180^0.$$ Iz prejšnjih dveh relacij dobimo:
  $$\alpha+\beta=\gamma',$$ kar je bilo treba dokazati. \kdokaz

\begin{figure}[!htb]
\centering
\input{sl.aks.2.5.6.pic}
\caption{} \label{sl.aks.2.5.6.pic}
\end{figure}

A direct consequence is the following theorem.


       \bizrek \label{zunanjiNotrNotrVecji}
       An exterior angle of a triangle is greater than either opposite interior angle.
        \eizrek

\textbf{\textit{Proof.}} Vpeljimo iste oznake kot v prejšnjem izreku \ref{zunanjiNotrNotr} (Figure \ref{sl.aks.2.5.6.pic}). Brez škode za splošnost  je dovolj dokazati, da velja $\gamma'>\alpha$ in $\gamma'>\beta$. Relaciji pa sta direktna posledica dokazane neenakosti $\alpha+\beta=\gamma'$ iz prejšnjega izreka \ref{zunanjiNotrNotr}.
 \kdokaz

 Glede na notranje kote lahko obravnavamo tri vrste trikotnikov.
 Dokazali smo že,
  da je v poljubnem trikotniku vsota notranjih kotov enaka $180^0$
  (izrek  \ref{VsotKotTrik}).
To pomeni, da je največ eden od teh kotov topi kot ali pravi kot,
oz. sta vsaj dva ostra. Torej imamo tri možnosti (Figure
\ref{sl.aks.2.6.4a.pic}):
\begin{itemize}
  \item Trikotnik je \index{trikotnik!ostrokotni}
  \pojem{ostrokotni}, če ima vse notranje kote ostre.
  \item Trikotnik je \index{trikotnik!pravokotni}\pojem{pravokotni},
    če ima en notranji kot pravi. Stranico, ki je nasproti
pravemu kotu pravokotnega trikotnika, imenujemo
\index{hipotenuza}\pojem{hipotenuza}, ostali dve stranici sta
\index{kateta}\pojem{kateti}. Ker je vsota notranjih kotov v vsakem
  trikotniku enaka $180^0$, sta
kota pri hipotenuzi pravokotnega trikotnika komplementarna.
  \item Trikotnik je \index{trikotnik!topokotni}\pojem{topokotni},
  če ima en notranji kot topi.
  Ostala dva notranja kota sta potem ostra.
\end{itemize}

\begin{figure}[!htb]
\centering
\input{sl.aks.2.6.4a.pic}
\caption{} \label{sl.aks.2.6.4a.pic}
\end{figure}

%slika 33.1

 Dokažimo še izreka, ki se nanašata na poljubne večkotnike.



         \bizrek \label{VsotKotVeck}
         The sum of the interior angles of any $n$-gon is equal to
         $(n - 2) \cdot 180^0$.
          \eizrek




\begin{figure}[!htb]
\centering
\input{sl.aks.2.5.5c.pic}
\caption{} \label{sl.aks.2.5.5c.pic}
\end{figure}


 \textbf{\textit{Proof.}} Predpostavimo najprej, da je $A_1A_2\ldots A_n$ konveksen $n $-kotnik (Figure \ref{sl.aks.2.5.5c.pic}). Njegovih $n-3$ diagonal $A_1A_3$, $A_1A_4$, ... $A_1A_{n-1}$ razdeli ta večkotnik na $n-2$ trikotnikov $\triangle_1$, $\triangle_2$, ..., $\triangle_{n-2}$. Ker se pri tem tudi vsak notranji kot $n $-kotnika  $A_1A_2\ldots A_n$ razdeli na ustrezne notranje kote omenjenih trikotnikov, je vsota vseh notranjih kotov
 $n $-kotnika  $A_1A_2\ldots A_n$ enaka vsoti vseh kotov trikotnikov  $\triangle_1$, $\triangle_2$, ..., $\triangle_{n-2}$. Po izreku \ref{VsotKotTrik} je na koncu ta vsota enaka ravno $(n - 2) \cdot 180^0$.

 Na tem mestu ne bomo dokazovali dejstva, da se tudi v primeru nekonveksnega $n $-kotnika  $A_1A_2\ldots A_n$ le-ta lahko razdeli na $n-2$ trikotnikov.
 \kdokaz


         \bizrek \label{VsotKotVeckZuna}
         The sum of the exterior angles of any $n$-gon is equal to
         $360^0$.
          \eizrek



\begin{figure}[!htb]
\centering
\input{sl.aks.2.5.5b.pic}
\caption{} \label{sl.aks.2.5.5b.pic}
\end{figure}


 \textbf{\textit{Proof.}} Označimo z $\alpha_1$, $\alpha_2$,..., $\alpha_n$ notranje in $\alpha'_1$, $\alpha'_2$, ..., $\alpha'_n$ pripadajoče zunanje kote ob ogliščih $A_1$, $A_2$,...,$A_n$ konveksnega $n $-kotnika  $A_1A_2\ldots A_n$. (Figure \ref{sl.aks.2.5.5b.pic}). Za ustrezni notranji in zunanji kot velja:

 \begin{eqnarray*}
 & & \alpha_1+\alpha'_1=180^0\\
 & & \alpha_2+\alpha'_2=180^0\\
 & & \vdots\\
 & & \alpha_n+\alpha'_n=180^0
 \end{eqnarray*}

 Če seštejemo vse enakosti in upoštevamo dokazano enakost iz prejšnjega izreka \ref{VsotKotVeck} $$\alpha_1+\alpha_2+\cdots + \alpha_n=(n - 2) \cdot 180^0,$$
  dobimo $(n - 2) \cdot 180^0+\alpha'_1+\alpha'_2+\cdots + \alpha'_n=n\cdot 180^0$ oz.
  $$\alpha'_1+\alpha'_2+\cdots + \alpha'_n=2 \cdot 180^0=360^0,$$ kar je bilo treba dokazati. \kdokaz

 Iz izreka \ref{VsotKotVeck}  sledi, da je vsota vseh notranjih kotov poljubnega štirikotnika enaka $360^0$, iz izreka \ref{VsotKotVeckZuna} pa, da je tudi vsota vseh zunanjih kotov konveksnega štirikotnika enaka $360^0$ (Figure \ref{sl.aks.2.5.5a.pic}).



\begin{figure}[!htb]
\centering
\input{sl.aks.2.5.5a.pic}
\caption{} \label{sl.aks.2.5.5a.pic}
\end{figure}



 Ker je trikotnik (kot presek treh polravnin) konveksen lik, je vsota  vseh zunanjih kotov poljubnega trikotnika enaka $360^0$ (Figure \ref{sl.aks.2.5.5d.pic}).

\begin{figure}[!htb]
\centering
\input{sl.aks.2.5.5d.pic}
\caption{} \label{sl.aks.2.5.5d.pic}
\end{figure}


Podobna trditev glede na izrek \ref{KotaVzporKraki}, ki se nanaša na kota z vzporednimi kraki, velja tudi za kota s pravokotnimi kraki.



        \bizrek \label{KotaPravokKraki}
        Angles with perpendicular sides are either congruent or supplementary.
         \index{kota!s pravokotnimi kraki}
        \eizrek



\begin{figure}[!htb]
\centering
\input{sl.aks.2.5.5.pic}
\caption{} \label{sl.aks.2.5.5.pic}
\end{figure}


 \textbf{\textit{Proof.}} Naj bosta $\angle aSb$ in $\angle a'S'b'$ takšna kota, da velja $a\perp a'$ in $b\perp b'$ (Figure \ref{sl.aks.2.5.5.pic}). Naj bo $A$ presečišče nosilk krakov $a$ in $a'$ ter $B$  presečišče nosilk krakov $b$ in $b'$. V štirikotniku $SAS'B$ merita notranja kota ob ogliščih $A$ in $B$ vsak $90^0$. Po izreku \ref{VsotKotVeck} je vsota vseh notranjih kotov tega štirikotnika enaka $360^0$. Zato notranja kota $BSA$ in $AS'B$ tega štirikotnika merita skupaj $180^0$, kar pomeni, da sta dva kota, ki ju določata nosilki poltrakov $a$ in $b$ oz. $a'$ in $b'$ suplementarna. Če pa enega od teh kotov zamenjamo z njegovim sokotom, sta ustrezna kota skladna.
 \kdokaz


%________________________________________________________________________________
\naloge{Exercises}
\begin{enumerate}

\item Naj bodo $P$, $Q$ in $R$ notranje točke stranic trikotnika
$ABC$. Dokaži, da so $P$, $Q$ in $R$ nekolinearne.

\item Naj bosta $P$ in $Q$ točki stranic $BC$ in $AC$ trikotnika $ABC$ in hkrati
različni od njegovih oglišč. Dokaži, da se daljici $AP$ in $BQ$
sekata v eni točki.

\item  Točke $P$, $Q$ in $R$ ležijo po vrsti na stranicah $BC$, $AC$ in $AB$ trikotnika
 $ABC$ in so različne od njegovih oglišč. Dokaži, da se daljici $AP$
in $QR$ sekata v eni točki.

\item Premica $p$, ki leži v ravnini štirikotnika, seka njegovo
diagonalo $AC$ in ne poteka skozi nobeno oglišče tega štirikotnika.
Dokaži, da premica $p$ seka natanko dve stranici tega
štirikotnika.

\item Dokaži, da je polravnina konveksen lik.

\item  Dokaži, da je presek dveh konveksnih likov
konveksen lik.

\item  Dokaži, da je poljuben trikotnik konveksen lik.

\item  Če je $\mathcal{B}(A,B,C)$ in $\mathcal{B}(D,A,C)$, je tudi
$\mathcal{B}(B,A,D)$. Dokaži.

\item  Naj bodo $A$, $B$, $C$ in $D$ takšne kolinearne točke, da je
$\neg\mathcal{B}(B,A,C)$ in $\neg\mathcal{B}(B,A,D)$. Dokaži, da
velja $\neg\mathcal{B}(C,A,D)$.

\item Naj bo $A_1A_2\ldots,A_{2k+1}$ poljubni večkotnik z lihim
številom oglišč. Dokaži, da ne obstaja premica, ki seka vse
njegove stranice.

\item Če izometrija $\mathcal{I}$ preslika lika $\Phi_1$ in $\Phi_2$
v lika  $\Phi'_1$ in $\Phi'_2$, potem se  presek
$\Phi_1\cap\Phi_2$ s to izometrijo preslika v presek
$\Phi'_1\cap\Phi'_2$. Dokaži.

\item  Dokaži, da sta poljubna poltraka neke
ravnine med seboj skladna.

\item  Dokaži, da sta poljubni premici neke
ravnine med seboj skladni.


\item  Naj bosta $k$ in $k'$ dve krožnici
neke ravnine s središčema $O$ in $O'$ ter polmeroma $AB$ in $A'B'$.
Dokaži ekvivalenco: $k\cong k' \Leftrightarrow AB\cong A'B'$.

\item  Naj bo $\mathcal{I}$
neidentična izometrija ravnine z dvema negibnima točkama $A$ in
$B$. Če je $p$ premica te ravnine, ki je vzporedna s premico
$AB$ in $A\notin p$. Dokaži, da na premici $p$ ni negibnih točk
izometrije $\mathcal{I}$.

\item   Naj bo $S$ edina negibna točka
izometrije $\mathcal{I}$ v neki ravnini. Dokaži, da če ta izometrija
preslika  premico $p$ vase, je $S\in p$.

\item  Dokaži, da se poljubni dve premici
neke ravnine ali sekata ali sta vzporedni.

\item  Če neka
premica v ravnini seka eno od dveh vzporednic iste ravnine,
 potem  seka tudi drugo vzporednico. Dokaži.

\item  Dokaži, da vsaka izometrija preslika vzporednici v vzporednici.

\item  Naj bodo $p$, $q$ in $r$ takšne premice neke ravnine, tako da velja
$p\parallel q$ in $r\perp p$. Dokaži, da je $r\perp q$.

\item Dokaži, da konveksen $n$-kotnik ne more imeti več kot treh
ostrih kotov.

\end{enumerate}



% DEL 3 - - - - - - - - - - - - - - - - - - - - - - - - - - - - - - - - - - - - - - -
%________________________________________________________________________________
% SKLADNOST TRIKOTNIKOV. VEČKOTNIKI
%________________________________________________________________________________

 \del{Congruence. Triangles and Polygons} \label{pogSKL}

%________________________________________________________________________________
 \poglavje{Triangle Congruence Theorems} \label{odd3IzrSkl}

Iz splošne definicije skladnosti likov sledi, da sta trikotnika
skladna, če obstaja izometrija, ki preslika prvi trikotnik v
drugega. Jasno je, da iz skladnosti dveh trikotnikov sledi
skladnost pripadajočih stranic in notranjih kotov. Nas pa zanima obraten problem:
Kdaj iz skladnosti nekaterih od
pripadajočih stranic in kotov sledi skladnost dveh trikotnikov? O
tem govorijo naslednji \index{izrek!o skladnosti
trikotnikov}\pojem{triangle congruence theorems}\footnote{Prvi, tretji in četrti izrek o skladnosti
trikotnikov pripisujejo \index{Pitagora} \textit{Pitagori z otoka
Samosa} (6. stol. pr. n. š.), za drugi izrek pa se
predpostavlja, da je bil znan že \index{Tales} \textit{Talesu iz
Mileta} (7.--6. stol. pr. n. š.). Vse štiri
navaja \index{Evklid} \textit{Evklid iz Aleksandrije}
(3. stol. pr. n. š.) v prvi knjigi svojih ‘‘Elementov’’.}:

                \bizrek \label{SSS} (\textit{SSS})
                 Triangles are congruent if three sides of one triangle are congruent
                 to the corresponding sides of the other triangle,
                i.e. (Figure
                \ref{sl.skl.3.1.1.pic}):
           \begin{eqnarray*}
            \left.
             \begin{array}{l}
              AB \cong A'B'\\
             BC \cong B'C'\\
             AC \cong A'C'
            \end{array}
            \right\}\hspace*{1mm}\hspace*{1mm}\Rightarrow\triangle ABC \cong \triangle A'B'C'
            \end{eqnarray*}
             \eizrek

\begin{figure}[!htb]
\centering
\input{sl.skl.3.1.1.pic}
\caption{} \label{sl.skl.3.1.1.pic}
\end{figure}

\textbf{\textit{Proof.}}
 Iz skladnosti daljic po izreku \ref{izrek(A,B)} je tudi $(A,B)
 \cong (A',B')$, $(B,C) \cong (B',C')$
 in $(A,C)
 \cong (A',C')$ oz $(A,B,C) \cong (A',B',C')$. Iz izreka
 \ref{IizrekABC} sledi, da obstaja izometrija $\mathcal{I}$, ki
 preslika točke $A$, $B$ in $C$ v točke $A'$, $B'$ in $C'$.
 Le-ta preslika trikotnik $ABC$ v trikotnik $A'B'C'$, kar je
 posledica izreka \ref{izrekIzoB}. Torej sta trikotnika $ABC$ in
 $A'B'C'$ skladna.
 \kdokaz



                \bizrek \label{SKS} (\textit{SAS})
                Triangles are congruent if two pairs of sides and the included angle of one triangle
                 are congruent to the corresponding sides and angle of the other triangle,
           i.e. (Figure
         \ref{sl.skl.3.1.2.pic}):
           \begin{eqnarray*}
            \left.
             \begin{array}{l}
              AB \cong A'B'\\
             AC \cong A'C'\\
             \angle BAC \cong \angle B'A'C'
            \end{array}
            \right\}\hspace*{1mm}\hspace*{1mm}\Rightarrow\triangle ABC \cong \triangle A'B'C'
            \end{eqnarray*}
            \eizrek


\begin{figure}[!htb]
\centering
\input{sl.skl.3.1.2.pic}
\caption{} \label{sl.skl.3.1.2.pic}
\end{figure}

\textbf{\textit{Proof.}}
  Ker sta kota $BAC$ in $B'A'C'$ skladna,
  obstaja izometrija $\mathcal{I}$, ki preslika kot $BAC$ v kot $B'A'C'$.
  Ta izometrija preslika vrh $A$ v vrh $A'$ ter kraka $AB$ in $AC$ v $A'B'$ in $A'C'$.
  Naj bo $\mathcal{I}(B)=\widehat{B}'$ in  $\mathcal{I}(C)=\widehat{C}'$.
  Iz tega
  sledi $AB \cong A'\widehat{B}'$ in $AC \cong A'\widehat{C}'$,
  toda po izreku \ref{ABnaPoltrakCX} je $\widehat{B}'=B'$ in
  $\widehat{C}'=C'$.
 Torej izometrija $\mathcal{I}$
 preslika točke $A$, $B$ in $C$ v točke $A'$, $B'$ in $C'$,
 oz. trikotnik $ABC$ v trikotnik $A'B'C'$, kar pomeni, da sta trikotnika $ABC$ in
 $A'B'C'$ skladna.
\kdokaz



                \bizrek \label{KSK} (\textit{ASA})
                Triangles are congruent if two pairs of angles and the included side of one triangle
                 are congruent to the corresponding angles and side of the other triangle
                i.e. (Figure
                \ref{sl.skl.3.1.3.pic}):
           \begin{eqnarray*}
            \left.
             \begin{array}{l}
             AB \cong A'B'\\
             \angle BAC \cong \angle B'A'C'\\
             \angle ABC \cong \angle A'B'C'
            \end{array}
            \right\}\hspace*{1mm}\hspace*{1mm}\Rightarrow\triangle ABC \cong \triangle A'B'C'
            \end{eqnarray*}
            \eizrek

\begin{figure}[!htb]
\centering
\input{sl.skl.3.1.3.pic}
\caption{} \label{sl.skl.3.1.3.pic}
\end{figure}

\textbf{\textit{Proof.}}
 Iz aksioma \ref{aksIII2} sledi, da obstaja izometrija  $\mathcal{I}$,
  ki preslika
 točko $A$ v točko $A'$, poltrak $AB$ v poltrak $A'B'$ in
 polravnino $ABC$ v polravnino $A'B'C'$.
 Ker je $AB \cong A'B'$, iz istega aksioma sledi
 $\mathcal{I}(B)=B'$. Naj bo $\mathcal{I}(C)=\widehat{C}'$.
 Potem je $\angle BAC \cong \angle B'A'\widehat{C}'$ in
   $\angle ABC \cong \angle A'B'\widehat{C}'$.
  Ker je po predpostavki tudi $\angle BAC \cong \angle B'A'C'$ in
   $\angle ABC \cong \angle A'B'C'$, iz izreka
   \ref{KotNaPoltrak} sledi, da sta poltraka $A'\widehat{C}'$ in
   $A'\widehat{C}'$ (oz. $B'C'$ in
   $B'\widehat{C}'$) enaka. Zato je $\widehat{C}'=A'\widehat{C}'\cap
   A'\widehat{C}'=A'C'\cap
   A'C'=C'$. Torej $\mathcal{I}:A,B,C \mapsto A',B',C'$, zato sta
   trikotnika $ABC$ in $A'B'C'$ skladna.
 \kdokaz

Dokaz četrtega izreka o skladnosti trikotnikov bomo izpustili.



            \bizrek \label{SSK} (\textit{SSA})
            Triangles are congruent if two pairs of sides and the angle opposite to the longer side of one triangle
                 are congruent to the corresponding sides and angle of the other triangle.
            (Figure \ref{sl.skl.3.1.4.pic}).
            \eizrek

\begin{figure}[!htb]
\centering
\input{sl.skl.3.1.4.pic}
\caption{} \label{sl.skl.3.1.4.pic}
\end{figure}


Ena od pomembnejših posledic izrekov o skladnosti trikotnikov je
naslednja trditev.



             \bizrek \label{enakokraki}
             If two sides of a triangle are congruent, then angles opposite those sides are congruent.\\
             And vice versa:\\
            If angles opposite those sides are congruent, then two sides of a triangle are congruent.
            \eizrek


\begin{figure}[!htb]
\centering
\input{sl.skl.3.1.5.pic}
\caption{} \label{sl.skl.3.1.5.pic}
\end{figure}


 \textbf{\textit{Proof.}}
 Naj bo $ABC$ takšen trikotnik, da je $AB \cong AC$
 (Figure \ref{sl.skl.3.1.5.pic}). Ker velja
še $AC \cong AB$ in $BC \cong CB$, iz izreka \textit{SSS} sledi, da
sta trikotnika $ABC$ in $ACB$ skladna (ta dva trikotnika imata
različno orientacijo). Zato je $\angle ABC \cong \angle CBA$. Na
enak način bi lahko dokazali tudi obratno trditev. V tem primeru bi
uporabljali izrek \textit{ASA}.
 \kdokaz

 Trikotnik (kot je trikotnik $ABC$ iz prejšnjega izreka),
 ki ima vsaj dve
stranici skladni, imenujemo
\index{trikotnik!enakokraki}\pojem{enakokraki trikotnik}. Vsaka od
dveh skladnih stranic je \index{krak!enakokrakega
trikotnika}\pojem{krak}, tretja stranica pa je
\index{osnovnica!enakokrakega trikotnika}\pojem{osnovnica} tega
trikotnika. Torej sta po prejšnjem izreku notranja kota ob
osnovnici enakokrakega trikotnika skladna. In obratno - če sta dva
notranja kota trikotnika skladna, je ta trikotnik enakokrak.



             \bzgled
            Let the $E$ and $F$ be a points lies on the line containing  the hypotenuse $AB$ of a perpendicular
            triangle $ABC$ and let $B(E,A,B,F)$, $EA\cong AC$ and
              $FB\cong BC$. What is the measure of the angle $ACB$?
            \ezgled


\begin{figure}[!htb]
\centering
\input{sl.skl.3.1.6.pic}
\caption{} \label{sl.skl.3.1.6.pic}
\end{figure}


 \textbf{\textit{Solution.}} (Figure \ref{sl.skl.3.1.6.pic})

 Notranja kota trikotnika $ABC$ ob ogliščih
 $A$ in $B$ označimo z $\alpha$ in
  $\beta$.
 Trikotnika $EAC$ in $CBF$ sta
 enakokraka trikotnika z osnovnicama $CE$ in $CF$, zato je $\angle CEA \cong \angle ACE$ in
 $\angle CFB \cong \angle BFC$ (izrek \ref{enakokraki}). Kot
 $\alpha$ je zunanji kot trikotnika $EAC$, zato je po izreku \ref{zunanjiNotrNotr}:
  $\alpha = 2\angle ECA$ oz.  $\angle ECA = \frac{1}{2} \alpha$.
  Podobno je tudi $\angle FCB = \frac{1}{2} \beta$. Torej:
\begin{eqnarray*}
   \angle ECF&=&\angle ECA+\angle ACB+\angle BCF=\\
   &=&\frac{1}{2}
   \cdot\alpha+90^0+
    \frac{1}{2} \cdot\beta=90^0+
    \frac{1}{2} \cdot\left(\alpha+\beta\right)=\\
    &=&90^0+
    \frac{1}{2}\cdot 90^0=135^0,
    \end{eqnarray*}
oz. $\angle ECF=135^0$. \kdokaz

Trikotnik, pri katerem so vse stranice enake, imenujemo
\index{trikotnik!enakostranični}\pojem{enakostranični trikotnik}
(Figure \ref{sl.skl.3.1.7.pic}), ki  je poseben primer
enakokrakega trikotnika. Zato iz omenjenega izreka sledi,
da so vsi notranji koti enakostraničnega trikotnika enaki. Ker je
njihova vsota enaka $180^0$, vsak od teh kotov meri $60^0$. Velja
tudi obratno: če sta vsaj dva kota trikotnika enaka $60^0$ (in
zaradi tega tudi tretji), potem je ta trikotnik enakostraničen.


\begin{figure}[!htb]
\centering
\input{sl.skl.3.1.7.pic}
\caption{} \label{sl.skl.3.1.7.pic}
\end{figure}




            \bzgled
              Let $ABC$ be an equilateral triangle and $P$, $Q$ and $R$
            points such that $\mathcal{B}(A,B,R)$, $\mathcal{B}(B,C,Q)$, $\mathcal{B}(C,A,P)$ and
            $BR\cong CQ\cong AP$. Prove that $PQR$ is also an equilateral triangle.
            \ezgled

\begin{figure}[!htb]
\centering
\input{sl.skl.3.1.8.pic}
\caption{} \label{sl.skl.3.1.8.pic}
\end{figure}


 \textbf{\textit{Proof.}} (Figure \ref{sl.skl.3.1.8.pic})

 Iz danih pogojev je najprej
  $AR\cong BQ\cong CP$. Trikotnik $ABC$ je  enakostranični trikotnik,
  zato vsi trije notranji koti merijo $60^0$. Iz tega sledi
  $\angle PAR\cong \angle RBQ\cong QCP$. Po izreku \textit{SAS} so
  trikotniki $PAR$, $RBQ$ in $QCP$ skladni in je $PR\cong RQ\cong QP$.
  To pomeni, da je tudi $PQR$  enakostranični trikotnik.
 \kdokaz


V prejšnjem poglavju smo definirali simetralo daljice kot premico,
ki je pravokotnica daljice  in poteka skozi njeno središče. Dokažimo
sedaj ekvivalentno definicijo simetrale daljice, ki bo zelo
pomembna v nadaljevanju. \index{simetrala!daljice}


             \bizrek \label{simetrala}
             The perpendicular bisector of a line segment $AB$ is the set of all points $X$
              that are equidistant from its endpoints, i.e. $AX \cong BX$.
             \eizrek

\begin{figure}[!htb]
\centering
\input{sl.skl.3.1.9.pic}
\caption{} \label{sl.skl.3.1.9.pic}
\end{figure}


 \textbf{\textit{Proof.}}
   Naj bo premica $s$ simetrala daljice
$AB$ v neki ravnini. Po definiciji je $s$ pravokotnica na  daljico
$AB$ skozi njeno središče – točko $S$. Označimo z $\mathcal{M}$
množico vseh točk $X$ te ravnine, za katere velja $AX \cong BX$.
Potrebno je dokazati, da je $s =\mathcal{M}$. To bomo dokazali z
dvema inkluzijama (Figure \ref{sl.skl.3.1.9.pic}).

($s\subseteq \mathcal{M}$). Naj bo $X \in s$. Dokažimo, da potem
velja $X \in \mathcal{M}$. Iz relacij $AS \cong BS$, $XS \cong XS$
in $\angle ASX \cong \angle BSX = 90^0$ sledi, da sta trikotnika
$ASX$ in $BSX$ skladna (izrek \textit{SAS}). Zato je $AX \cong BX$
oz. $X \in \mathcal{M}$.

($\mathcal{M}\subseteq s$). Naj bo sedaj $X \in \mathcal{M}$.
Dokažimo, da velja $X \in s$. Iz $X \in \mathcal{M}$ sledi $AX \cong BX$.
Sedaj iz $AS \cong BS$, $XS \cong XS$ in $AX \cong BX$ sledi,
da sta trikotnika $ASX$ in $BSX$ skladna (izrek \textit{SSS}). Zato sta kota
$ASX$ in $BSX$ skladna in kot sokota sta oba prava kota. To pomeni,
da je premica $XS$ pravokotnica daljice $AB$ v njenem središču.
Torej, premica $XS$ je simetrala $s$ oz. $X \in s$.
 \kdokaz

The next problem is an example of multiple use of the theorem of an isosceles triangle (\ref{enakokraki}).

      \bnaloga\footnote{42. IMO USA - 2001, Problem 5.}
      In a triangle $ABC$, let $AP$ bisect $\angle BAC$, with $P$ on $BC$, and let $BQ$ bisect $\angle ABC$, with $Q$ on $CA$.
It is known that $\angle BAC=60^0$ and that $|AB|+|BP|=|AQ|+|QB|$.
What are the possible measures of  interior angles of triangle $ABC$?
        \enaloga


\begin{figure}[!htb]
\centering
\input{sl.skk.4.9.IMO1.pic}
\caption{} \label{sl.skk.4.9.IMO1.pic}
\end{figure}

\textbf{\textit{Solution.}} Označimo notranje kote trikotnika $ABC$ z
$\alpha=60^0$, $\beta$ in $\gamma$. Naj bosta $D$ in $E$ takšni
točki, da velja: $BD\cong BP$, $\mathcal{B}(A,B,D)$, $QE \cong QB$
in $\mathcal{B}(A,Q,E)$ (Figure \ref{sl.skk.4.9.IMO1.pic}). Iz teh
pogojev sledi, da sta $DBP$ in $BQE$ enakokraka trikotnika z
osnovnicama $DP$ in $BE$. Iz danega pogoja $|AB|+|BP|=|AQ|+|QB|$
sledi še $AD\cong AE$, kar pomeni, da je tudi $ADE$ enakokraki
trikotnik z osnovnico $DE$.

Ker je $DBP$ enakokraki trikotnik, iz izrekov \ref{enakokraki} in
\ref{zunanjiNotrNotr} sledi: $\angle BDP\cong \angle BPD
=\frac{1}{2}\angle ABC=\frac{1}{2}\beta$.
Ker je tudi $BQE$ enakokraki trikotnik, je $\angle QBE\cong\angle QEB$.
 Iz skladnosti trikotnikov $ADP$ in $AEP$ (izrek \textit{SAS}
\ref{SKS}) sledi $\angle ADP\cong\angle AEP$ in $PD\cong PE$.

Če povežemo dokazane relacije, velja:
 \begin{eqnarray*}
&& \angle AEP\cong \angle BDP
=\frac{1}{2}\beta\cong \angle QBP\\
&&\textrm{ in } \angle AEB\cong\angle QBE
 \end{eqnarray*}

 Predpostavimo najprej, da velja $QB>QC$ oz.
 $\mathcal{B}(Q,C,E)$. V tem primeru je:
 \begin{eqnarray*}
 \angle PEB &=&\angle AEB-\angle AEP=\\
 &=&\angle QBE-\angle QBP=\\
 &=&\angle PBE.
 \end{eqnarray*}
To pomeni, da je $PBE$ enakokraki trikotnik z osnovnico $BE$ oz.
$PE\cong PB$. Toda iz že dokazanega $PE\cong PD$ in predpostavke
$PB\cong BD$ sledi $PD\cong PB\cong BD$, zatorej je $BDP$ enakostranični
trikotnik. Iz tega sledi $\beta=2\angle BDP=2\cdot 60^0=120^0$, oz. $\alpha+\beta=60^0+120^0=180^0$, kar ni možno (izrek
\ref{VsotKotTrik}). Zato relacija $QB>QC$ ni mogoča.

Na podoben način pripelje do protislovja tudi relacija $QB>QC$. To
pomeni, da je možno le $QB\cong QC$. V tem primeru je $C=E$ in
velja $\gamma=\angle ACB=\angle AEB\cong AEP=\frac{1}{2}\beta$. Iz
$\alpha+\beta+\gamma=180^0$, sledi
$60^0+\beta+\frac{1}{2}\beta=180^0$ oz. $\beta=80^0$.

Dokazali smo, da iz pogojev iz naloge sledi $\beta=80^0$. Torej je
edina možna rešitev  $\beta=80^0$. Potrebno je še dokazati, da
$\beta=80^0$ je rešitev, oz. da iz $\alpha=60^0$, $\beta=80^0$
sledi $|AB|+|BP|=|AQ|+|QB|$.
 Najprej iz $\angle QCB=\gamma=\frac{1}{2}\beta=40^0=\angle QBC$ sledi
 (izrek \ref{enakokraki}) $QC\cong QB$ oz.
 $|AQ|+|QB|=|AQ|+|QC|=|AC|$.
Če točko $D$ definiramo na isti način kot v prvem delu, spet
dobimo $\angle ADP\cong \angle BPD=\frac{1}{2}\angle ABC=40^0=\angle ACB=\angle ACP$.
To pomeni, da sta trikotnika
$ADP$ in $ACP$ skladna (izrek \textit{ASA} \ref{KSK}) oz.
$AD\cong AC$. Zato je na koncu:
 $$|AB|+|BP|=|AB|+|BD|=|AD|=|AC|=|AQ|+|QB|,$$ kar je bilo treba dokazati. \kdokaz



%________________________________________________________________________________
 \poglavje{Constructions in Geometry} \label{odd3NacrtNaloge}

Ob naslednjem zgledu bomo opisali t. i. načrtovalne naloge.


             \bzgled \label{načrt1odd3}
              Two congruent line segments $AB$ and $A'B'$ in a plane are given.
              Construct a point $C$ such that $\triangle ABC \cong \triangle A'B'C$.
             \ezgled


\begin{figure}[!htb]
\centering
\input{sl.skl.3.1.10.pic}
\caption{} \label{sl.skl.3.1.10.pic}
\end{figure}

 \textbf{\textit{Solution.}} Predpostavimo, da je $C$ točka v ravnini daljic, za
 katero je $\triangle ABC \cong \triangle A'B'C$. Potem je $AC\cong A'C$
 in  $BC\cong B'C$ oz. točka $C$ leži na simetralah daljic $AA'$
 in $BB'$ (izrek  \ref{simetrala}). To dejstvo nam omogoča konstrukcijo
  (Figure \ref{sl.skl.3.1.10.pic}).

 Načrtajmo simetrali daljic $AA'$
 in $BB'$. Točko $C$ dobimo v njunem presečišču.

 Dokažimo, da je $C$ iskana točka oz. da izpolnjuje pogoje iz
 naloge. Po predpostavki je že $AB\cong A'B'$. Ker smo točko $C$ dobili kot
 presečišče simetral daljic $AA'$
 in $BB'$, je  $AC\cong A'C$ in $BC\cong B'C$. Iz izreka \ref{SSS} (SSS) sledi,
 da sta trikotnika $ABC$ in $A'B'C$
 skladna.

  Naloga ima rešitev (eno) natanko tedaj, ko se simetrali daljic
  $AA'$
 in $BB'$ sekata, oz. ko premici $AA'$
 in $BB'$ nista vzporedni.
  \kdokaz

  Prejšnji zgled je torej t. i. \index{načrtovalna naloga}
   \pojem{načrtovalna naloga}, pri kateri
  je za dane podatke potrebno načrtati oz. konstruirati nek novi element ali lik,
  ki v zvezi z danimi podatki
  izpolnjuje določene pogoje. \pojem{Načrtovanje} oz. \index{konstrukcije}
  \pojem{konstrukcija}
   pomeni
  uporabo ravnila in šestila oz. uporabo
  \index{konstrukcije!elementarne}
  \pojem{elementarne konstrukcije}:\label{elementarneKonstrukcije}
\begin{itemize}
  \item za dani točki $A$ in $B$ narišemo:
 \begin{itemize}
  \item premico $AB$,
  \item daljico $AB$,
  \item poltrak $AB$;
\end{itemize}
 \item narišemo krožnico $k$:
\begin{itemize}
  \item s središčem $S$, ki poteka skozi dano točko $A$,
  \item s središčem $S$ in polmerom, ki je skladen z dano
  daljico;
\end{itemize}
\item narišemo krožni lok z danima središčem in polmerom,
\item narišemo presečišče (oz. presečišči):
\begin{itemize}
  \item dveh premic,
  \item premice in krožnice,
  \item dveh krožnic.
\end{itemize}
\end{itemize}

  Rešitev načrtovalne naloge (nariši lik $\Phi$, ki izpolnjuje pogoje
  $\mathcal{A}$) je formalno sestavljena iz štirih korakov:
\begin{itemize}
  \item \textit{analysis} - pri kateri predpostavimo, da je
  lik $\Phi$ že načrtan in izpolnjuje pogoje $\mathcal{A}$, nato
  pa iščemo nove pogoje $\mathcal{B}$, ki jih lik izpolnjuje.
  Ti sledijo iz pogojev $\mathcal{A}$ in so bolj ugodni za
  konstrukcijo lika $\Phi$. Dokažemo
  $\mathcal{A}\Rightarrow \mathcal{B}$.
  \item \textit{construction} - načrtamo lik $\Phi'$, ki izpolnjuje pogoje
   $\mathcal{B}$. Natanko opišemo potek načrtovanja.
  \item \textit{proof} - dokažemo, da je $\Phi' = \Phi$ oz.
  $\mathcal{B}\Rightarrow \mathcal{A}$.
  \item \textit{discussion} - raziščemo
  število rešitev naloge, odvisno od pogojev  $\mathcal{A}$.
\end{itemize}


        \bzgled \label{konstrTrik1}
        Line segments $a$, $l$ and an angle $\alpha$ are given. Construct a triangle
            $ABC$, such that the side $BC$ is congruent to the line segment $a$, the sum of the sides $AB + AC$
             equal to the line segment $l$ and the interior angle $BAC$ congruent to the angle $\alpha$  ($a$, $b+c$, $\alpha$).
        \ezgled


\begin{figure}[!htb]
\centering
\input{sl.skl.3.1.10a.pic}
\caption{} \label{sl.skl.3.1.10a.pic}
\end{figure}


 \textbf{\textit{Analysis.}} Naj bo $ABC$ trikotnik, pri katerem je $BC \cong a$, vsota $AB+AC$ enaka dani daljici $l$ in
$\angle BAC\cong \alpha$ (Figure \ref{sl.skl.3.1.10a.pic}). Naj bo $D$ takšna točka na poltraku $BA$, da je $AD\cong AC$ in točki $B$ in $D$ na različnih straneh
točke $A$. Torej velja $BD=BA+AD=AB+AC=l$ oz. $BD\cong l$. Trikotnik $ACD$ je enakokrak, zato sta  kota $ADC$ in $ACD$
po izreku \ref{enakokraki} skladna. Ker sta hkrati notranja
kota trikotnika $CAD$, sta oba enaka polovici zunanjega kota $BAC$ tega trikotnika (izrek  \ref{zunanjiNotrNotr}), oz.
 $\angle BDC=\angle ADC\cong \angle ACD=\frac{1}{2}\angle BAC=\frac{1}{2}\alpha$.
 To dejstvo nam omogoča
konstrukcijo trikotnika $BCD$.

\textbf{\textit{Construction.}} Načrtajmo najprej trikotnik $BCD$, kjer je
$\angle BDC=\frac{1}{2}\alpha$,
 $BC\cong a$ in $BD\cong l$,
nato pa točko $A$ kot presečišče simetrale daljice $CD$ z daljico $BD$. Dokažimo da je $ABC$
iskani trikotnik.

\textbf{\textit{Proof.}} Najprej je $BC\cong a$, že po konstrukciji. Po konstrukciji točka $A$ leži na simetrali daljice
$CD$, zato je $AD\cong AC$ (izrek \ref{simetrala}). Torej $CAD$ je enakokraki trikotnik z osnovnico $CD$, zato je (izrek \ref{enakokraki}) tudi $\angle ADC\cong \angle ACD$. Zaradi tega je (izrek \ref{zunanjiNotrNotr}) $\angle BAC = 2 \cdot \angle BDC= 2\cdot\frac{1}{2}\alpha=\alpha$. Iz $AD\cong AC$  pa sledi
 $AB + AC = AB + AD = BD \cong l$
.

\textbf{\textit{Discussion.}} Naloga ima rešitev (in sicer eno ali dve) natanko tedaj,
ko poltrak $DC$ seka krožnico $k(B,a)$
in simetrala daljice $CD$ seka daljico $BD$.
 \kdokaz

V bodoče ne bomo pri vsaki načrtovalni nalogi izpeljevali vseh korakov.
V večini primerov bomo naredili le prvi korak in s tem nakazali
potek reševanja.




%________________________________________________________________________________
 \poglavje{Triangle inequality} \label{odd3NeenTrik}

Najprej bomo dokazali  dva pomembna izreka, ki sta posledici
izrekov o skladnosti trikotnikov.

            \bizrek \label{vecstrveckot}
            One side of a triangle is longer than another side of a triangle if and only if
            the measure of the angle opposite the longer side is greater than the angle opposite the shorter side.
            \eizrek

\begin{figure}[!htb]
\centering
\input{sl.skl.3.2.1.pic}
\caption{} \label{sl.skl.3.2.1.pic}
\end{figure}

 \textbf{\textit{Proof.}} Naj bo $ABC$ trikotnik, v katerem je $AC > AB$ (Figure \ref{sl.skl.3.2.1.pic}).
 Dokažimo, da je tedaj tudi $\angle ABC > \angle ACB$. Ker je $AC > AB$, obstaja
takšna točka $B'$ med točkama $A$ in $C$, za katero velja $AB \cong AB'$.
Tedaj je trikotnik $BAB’$ enakokrak in velja
$\angle ABB' \cong \angle AB'B$ (izrek \ref{enakokraki}).
Poltrak $BB'$ je znotraj kota $ABC$, zato je $\angle ABC > \angle ABB '$.
Potem je $\angle AB'B$ zunanji kot trikotnika $BCB'$. Po
 izreku  \ref{zunanjiNotrNotrVecji} je ta kot večji od njegovega nesosednjega
notranjega kota $B'CB$. Če uporabimo doslej dokazano, dobimo:
 $$\angle ABC >
\angle ABB ' \cong \angle AB'B > \angle B'CB \cong \angle ACB.$$
 Torej velja $\angle ABC
> \angle ACB$. Na podoben način dokažemo, da velja tudi obratno.
\kdokaz

 Če  z $a$, $b$ in $c$ označimo dolžine stranic $BC$, $AC$ in
 $AB$ trikotnika $ABC$ ter z $\alpha$, $\beta$ in $\gamma$
 mere nasprotnih kotov ob ogliščih $A$, $B$ in $C$, lahko prejšnji izrek
 zapišemo v obliki:
  $$a > b \Leftrightarrow \alpha > \beta,$$
  izrek o enakokrakem trikotniku \ref{enakokraki} pa v obliki:
 $$a = b \Leftrightarrow \alpha = \beta.$$


 To pomeni, da sta izraza $a-b$ in $\alpha-\beta$ oba pozitivna,
 oba negativna ali oba enaka nič. Tako smo dokazali naslednjo
 lastnost:

             \bzgled \label{vecstrveckotAlgeb}
            For each triangle $ABC$ is:
             $$(a-b)(\alpha-\beta)\geq 0, \hspace*{4mm}
             (b-c)(\beta-\gamma)\geq 0, \hspace*{4mm}
             (c-a)(\gamma-\alpha)\geq 0$$
            \ezgled

Iz izreka \ref{vecstrveckot} direktno sledi naslednja
trditev:


            \bizrek \label{vecstrveckotHipot}
            The hypotenuse of a right-angled triangle is longer than its
            two legs.
            The longest side of an obtuse triangle is the one opposite to the obtuse angle.
             \eizrek

\begin{figure}[!htb]
\centering
\input{sl.skl.3.2.2.pic}
\caption{} \label{sl.skl.3.2.2.pic}
\end{figure}


\textbf{\textit{Proof.}} (Figure \ref{sl.skl.3.2.2.pic})

 Vsota notranjih kotov trikotnika je enaka $180^0$. Zato je v
pravokotnem trikotniku največji kot ravno pravi kot. Hipotenuza
pravokotnega trikotnika je po prejšnjem izreku najdaljša stranica
tega trikotnika. Podobno dokažemo tudi v primeru topokotnega
trikotnika.
 \kdokaz


Daljica $AA'$ je \index{višina!trikotnika}\pojem{višina}
trikotnika $ABC$, če je $AA'\perp BC$ in $A'\in BC$. Zadnja od
dveh relacij pomeni, da točka $A'$ leži na premici $BC$, ne pa
nujno na daljici $BC$. Relacija $\mathcal{B}(B,A',C)$ velja
natanko tedaj, ko sta notranja kota pri ogliščih $B$ in $C$ oba
ostra (Figure \ref{sl.skl.3.2.3.pic}). To je posledica izreka o
vsoti notranjih kotov poljubnega trikotnika (izrek
\ref{VsotKotTrik}). V primeru, da je  $\angle ABC\geq 90^0$ in
$\mathcal{B}(B,A',C)$, bi bila vsota notranjih kotov v trikotniku $ABA'$
večja od $180^0$.
 Torej
 višina trikotnika ni vedno v notranjosti trikotnika.
 Pri pravokotnem trikotniku sta višini iz
dveh oglišč ob ostrih kotih enaki ustreznima katetama. Višine,
ki potekajo iz oglišč $A$, $B$ in $C$ ponavadi označimo z $v_a$,
$v_b$ in $v_c$. Iz prejšnjega izreka \ref{vecstrveckotHipot}
sledi, da je dolžina višina poljubnega trikotnika manjša ali enaka dolžini
nepripadajoče stranice tega trikotnika, npr.: $v_a\leq b$, $v_a\leq c$, ...


\begin{figure}[!htb]
\centering
\input{sl.skl.3.2.3.pic}
\caption{} \label{sl.skl.3.2.3.pic}
\end{figure}


Sedaj bomo rešili načrtovalno nalogo, v kateri kot podatek nastopa višina trikotnika.


        \bzgled
        	 Construct a triangle $ABC$ such that the sides $AB$,
            $AC$ and the altitude  from the vertex $B$ are congruent to the three given line segments $c$, $b$ and $v_b$.
        \ezgled



\begin{figure}[!htb]
\centering
\input{sl.skl.3.2.4a.pic}
\caption{} \label{sl.skl.3.2.4a.pic}
\end{figure}



\textbf{\textit{Analysis.}} Naj bo $ABC$ trikotnik, za katerega velja $AB\cong c$,
 $AC\cong b$ in $AD\cong v_b$ (kjer je $BD$ višina tega trikotnika iz oglišča $B$).
 V pravokotnem trikotniku $ABD$
sta torej znani hipotenuza $AB\cong c$ in kateta
$AD\cong v_b$, kar pomeni, da ga lahko načrtamo. Tretje oglišče $C$ trikotnika $ABC$ leži na premici $AD$ (Figure \ref{sl.skl.3.2.4a.pic}).

\textbf{\textit{Construction.}} Načrtajmo najprej pravokotni trikotnik
$ABD$ (s pogoji: $AB\cong c$, $\angle ADB=90^0$ in $BD\cong v_b$). Na premici $AD$ nato določimo  takšno točko $C$,
da velja $AC\cong b$. Dokažimo, da je $ABC$ iskani trikotnik.


\textbf{\textit{Proof.}} Najprej je $AB\cong c$ in
 $AC\cong b$ že po konstrukciji. Ker je še $\angle ADB=90^0$, je $BD$ višina trikotnika $ABC$ iz oglišča $B$ in je po konstrukciji skladna z daljico $v_b$.


\textbf{\textit{Discussion.}} Naloga ima rešitev natanko tedaj, ko je možna
konstrukcija trikotnika $ABD$ oz. $hb\leq c$. Pri konstrukciji točke $C$
obstajata dve možnosti - na različnih straneh točke $A$, kar pomeni, da imamo dve rešitvi za trikotnik $ABC$. V primeru  $hb\cong c$ sta rešitvi pravokotna in skladna trikotnika.
 \kdokaz



        \bzgled
        If $v_a$, $v_b$ and $v_c$ are altitudes corresponding
        to the sides $a$, $b$ and $c$ of a triangle, then:
        $$\frac{v_a}{b+c}+\frac{v_b}{a+c}+\frac{v_c}{a+b}<\frac{3}{2}.$$
        \ezgled

\textbf{\textit{Proof.}}
Če seštejemo neenakosti $v_a\leq b$,
$v_a\leq c$
 dobimo $2v_a\leq b+c$ oz. $\frac{v_a}{b+c}\leq\frac{1}{2}$.
 Analogno dobimo $\frac{v_b}{a+c}\leq\frac{1}{2}$ in
 $\frac{v_c}{a+b}\leq\frac{1}{2}$. Ker vse enakosti ne morejo veljati hkrati,
  s seštevanjem dobimo iskano neenakost.
 \kdokaz

Naslednjo lastnost trikotnika bomo imenovali \index{trikotniška
neenakost} \pojem{trikotniška neenakost}.


             \bizrek \label{neenaktrik}
             The sum of any two sides of a triangle is greater than the third side.
            \eizrek


\begin{figure}[!htb]
\centering
\input{sl.skl.3.2.4.pic}
\caption{} \label{sl.skl.3.2.4.pic}
\end{figure}


\textbf{\textit{Proof.}} Naj bo $ABC$ poljuben trikotnik.
Dokažimo, da velja \\ $AB + AC > BC$. Z $D$ označimo takšno točko,
da je $\mathcal{B}(B,A,D)$ in $AD \cong  AC$ (Figure
\ref{sl.skl.3.2.4.pic}). Po izreku \ref{enakokraki}
($\triangle CAD$ je enakokraki trikotnik z osnovnico $CD$) je  tudi  $\angle BDC=\angle ADC  \cong  \angle ACD$. Poltrak $CA$ je
znotraj kota $DCB$, zato je
 $\angle ACD <  \angle DCB$. Tedaj je tudi  $\angle BDC <  \angle DCB$.
 Iz izreka
 \ref{vecstrveckot} (glede na trikotnik $BCD$) sledi:
$$BC < BD = AB + AD = AB + AC,$$ kar je bilo treba dokazati. \kdokaz

Iz trikotniške neenakosti dobimo kriterij za obstoj takšnega
trikotnika, da so njegove stranice skladne s tremi danimi daljicami.


             \bzgled
            Let $a$, $b$ and $c$ be three line segments. A triangle with sides $a$,
            $b$ and $c$ exist if and only if:
             $$b + c > a,\hspace*{2mm}
             a + c > b \hspace*{1mm}\textrm{ in }\hspace*{1mm}
             a + b > c.$$
             \ezgled

\begin{figure}[!htb]
\centering
\input{sl.skl.3.2.5.pic}
\caption{} \label{sl.skl.3.2.5.pic}
\end{figure}

\textbf{\textit{Proof.}}  Če takšen trikotnik obstaja, potem so tri
relacije direktna posledice izreka o trikotniški neenakosti
\ref{neenaktrik}. Predpostavimo torej, da veljajo vse tri relacije.
Brez škode za splošnost naj bo npr. $a$ najdaljša stranica tega
trikotnika (dovolj je, da ni krajša od neke druge stranice) ter $B$
in $C$ poljubni točki, za kateri je $BC \cong a$ (Figure
\ref{sl.skl.3.2.5.pic}). Ker je po predpostavki $b + c > a$,
to pomeni, da se krožnici $k(B,c)$ in $k(C,b)$ sekata v neki točki
$A$ (posledica Dedekindovega aksioma - izrek
\ref{DedPoslKrozKroz}, ker vsaka od njiju vsebuje notranje točke
druge), ki ni na daljici $BC$. Trikotnik $ABC$ je potem iskani
trikotnik. \kdokaz

 Če vemo, katera od treh stranic je najdaljša,
 je dovolj preveriti le eno neenakost, saj
sta drugi dve avtomatično izpolnjeni. Dokaz
prejšnjega izreka lahko uporabimo tudi za naslednji, ekvivalenten
kriterij.

             \bzgled \label{neenaktrik1}
               Let $a$, $b$ and $c$ be three line segments, such that $a \geq b,c$. A triangle with sides $a$,
            $b$ and $c$ exist if and only if $b + c > a$.
              \ezgled

Tako npr. lahko ugotovimo, da obstaja trikotnik s stranicami, ki
imajo dolžine 7, 5 in 3 (ker je $5+3>7$), trikotnik s stranicami, ki imajo dolžine 9, 6 in 2,
pa ne obstaja (ker ni $6+2>9$).

 Oglejmo si še nekatere posledice prejšnjih izrekov.


             \bzgled
              If $X$ is an arbitrary point of the side $BC$ of a triangle $ABC$,
                then:
               $$AX < AB + AC.$$
             \ezgled

\begin{figure}[!htb]
\centering
\input{sl.skl.3.2.6.pic}
\caption{} \label{sl.skl.3.2.6.pic}
\end{figure}

\textbf{\textit{Proof.}} Če uporabimo trikotniško neenakost za
trikotnike $ABX$ in $AXC$ (Figure \ref{sl.skl.3.2.6.pic}), dobimo:
 $$AX < AB + BX \hspace*{1mm} \textrm{ in }\hspace*{1mm}  AX < AC + CX.$$
Po seštevanju teh dveh neenakosti in uporabi trikotniške
neenakosti  za trikotnik $ABC$ dobimo:
 $$2AX < AB + AC +
BC < 2(AB + AC),$$ kar je bilo treba dokazati. \kdokaz

%%  !!! Dosegel magično stran - 100!!! Wow Bravo!!!

Naslednja neenakost je posplošitev prejšnje. V tem smislu je
prejšnja trditev njena posledica in je ni bilo potrebno posebej
dokazovati.


            \bzgled
            Let $X$ be an arbitrary point of the side $BC$, different from the vertices $B$ and $C$ of a triangle $ABC$.
            Then the line segment $AX$ is shorter than at least one of the two line segments $AB$ and $AC$ i.e.:
            $$AX < \max\{AB, AC\}.$$
            \ezgled

\begin{figure}[!htb]
\centering
\input{sl.skl.3.2.7.pic}
\caption{} \label{sl.skl.3.2.7.pic}
\end{figure}

\textbf{\textit{Proof.}} Ker za točko $X$ velja $\mathcal{B}(B, X, C)$,
eden od sokotov $AXB$ in $AXC$ ni oster. Brez škode za
splošnost naj bo to kot $AXC$ (Figure \ref{sl.skl.3.2.7.pic}). Potem
je le-ta največji kot v trikotniku $AXB$, kar pomeni, da je \\
$AX < AC$ (izrek \ref{vecstrveckot}). Podobno, če kot $AXB$ ni oster,
velja $AX < AB$.
 \kdokaz

Posebej bomo obravnavali primer daljice $AX$, če je točka $X$ iz
prejšnjih dveh izrekov središče stranice $BC$ (Figure
\ref{sl.skl.3.2.8.pic}). Takšno daljico, ki je določena z ogliščem
in središčem nasprotne stranice trikotnika, imenujemo
\index{težiščnica trikotnika} \pojem{težiščnica} trikotnika.
Težiščnice, ki ustrezajo ogliščem $A$, $B$ in $C$ trikotnika $ABC$,
običajno označujemo s $t_a$, $t_b$ in $t_c$. Zadnji
dve trditvi lahko uporabimo tudi za težiščnice. Toda za težiščnice
velja še dodatna lastnost, ki jo bomo sedaj dokazali.

\begin{figure}[!htb]
\centering
\input{sl.skl.3.2.8.pic}
\caption{} \label{sl.skl.3.2.8.pic}
\end{figure}



             \bzgled \label{neenTezisZgl}  If $a$, $b$, $c$ are the sides and $t_a$
             the corresponding median of a triangle $ABC$, then:
            $$\frac{b+c-a}{2}<t_a<\frac{b+c}{2}.$$
              \ezgled

\begin{figure}[!htb]
\centering
\input{sl.skl.3.2.9.pic}
\caption{} \label{sl.skl.3.2.9.pic}
\end{figure}

\textbf{\textit{Proof.}} Označimo z $A_1$ središče stranice $BC$
trikotnika $ABC$. Tedaj je $t_a  =AA_1$  (Figure
\ref{sl.skl.3.2.9.pic}).

Če za trikotnika $ABX$ in $ACX$ uporabimo trikotniško
neenakost (izrek \ref{neenaktrik}), dobimo: $AA_1 + A_1B > AB$ in
$AA_1 + A_1C > AC$  oz.:
$$t_a+\frac{a}{2}>c \hspace*{2mm} \textrm{ in } \hspace*{2mm}
 t_a+\frac{a}{2}>b.$$
Če seštejemo te dve neenakosti, dobimo $\frac{b+c-a}{2}<t_a$.
Dokažimo še $t_a<\frac{b+c}{2}$.
 Naj bo $D$ točka, za katero je
 $A_1D \cong AA_1$ in $\mathcal{B}(A,A_1,D)$. Trikotnika $AA_1B$ in $DA_1C$
  sta
skladna (izrek \textit{SAS} \ref{SKS}), kar pomeni, da je tudi $AB \cong DC$. Če
uporabimo še trikotniško neenakost (izrek \ref{neenaktrik}) za
trikotnik $ACD$, dobimo:
$$b+c = AC + AB = AC + CD > AD = 2AA_1 = 2t_a,$$ kar je bilo treba dokazati. \kdokaz

 Pokažimo še nekaj primerov uporabe trikotniške neenakosti.


             \bzgled
             Let $M$ be an arbitrary point of the bisector of the exterior angle at
            the vertex $C$ of a triangle $ABC$. Then
            $$MA + MB \geq CA + CB.$$
             \ezgled

\begin{figure}[!htb]
\centering
\input{sl.skl.3.2.10.pic}
\caption{} \label{sl.skl.3.2.10.pic}
\end{figure}


 \textbf{\textit{Proof.}} Naj bo $D$ takšna točka poltraka $BC$, da je $AC \cong CD$
 in $\mathcal{B}(A,C,D)$ (Figure
\ref{sl.skl.3.2.10.pic}). Trikotnika $ACM$ in $DCM$ sta skladna,
po izreku \textit{SAS}  \ref{SKS} ($AC \cong DC$, $CM \cong CM$, $\angle ACM \cong
\angle DCM$). Zato je tudi $MA \cong MD$. Če sedaj uporabimo
trikotniško neenakost, dobimo: $$MA + MB = MD + MB \geq BD = DC +
CB = CA + CB.$$
 Seveda enakost velja v primeru,
kadar so točke $B$, $M$ in $D$ kolinearne oziroma $M = C$.
 \kdokaz



             \bzgled
            If the bisector of the interior angle at
            the vertex $A$ of a triangle $ABC$ intersects the side $BC$ in the point $E$, then
            $$AB > BE \hspace*{2mm} \textrm{ in }\hspace*{2mm}  AC > CE .$$
             \ezgled

\begin{figure}[!htb]
\centering
\input{sl.skl.3.2.11.pic}
\caption{} \label{sl.skl.3.2.11.pic}
\end{figure}

 \textbf{\textit{Proof.}}  Ker je $\mathcal{B}(B,E,C)$, je kot
 $AEB$ zunanji kot  trikotnika $AEC$ (Figure
\ref{sl.skl.3.2.11.pic}). Zato je $\angle BEA > \angle EAC \cong
\angle BAE$ (izrek \ref{zunanjiNotrNotrVecji}). Nasproti večjega
kota v trikotniku $BAE$ je večja stranica, oziroma velja $AB > BE$
(izrek \ref{vecstrveckot}). Podobno dokazujemo, da velja tudi
druga od dveh relacij.
 \kdokaz


             \bzgled \label{zgled3.2.9}
            If $M$ is the interior point of a triangle $ABC$, then \\
            $BA + AC > BM + MC.$
            \ezgled

\begin{figure}[!htb]
\centering
\input{sl.skl.3.2.12.pic}
\caption{} \label{sl.skl.3.2.12.pic}
\end{figure}

\textbf{\textit{Proof.}} Naj bo $N$ presečišče premic $BM$ in
$CA$ (Figure \ref{sl.skl.3.2.12.pic}). Ker je $M$ notranja točka
trikotnika $ABC$, velja $\mathcal{B}(B,M,N)$ in
$\mathcal{B}(A,N,C)$. Če sedaj dvakrat uporabimo trikotniško
neenakost (izrek \ref{neenaktrik}), dobimo:
 \begin{eqnarray*}
\hspace*{-4mm}BM + MC &<& BM + (MN + NC) = (BM + MN) + NC = BN + NC\\
 \hspace*{-4mm}&<& (BA +
AN) + NC = BA + (AN + NC) = BA + AC.
  \end{eqnarray*}

Definirajmo dva nova pojma. Vsoto vseh stranic nekega večkotnika
imenujemo njegov \index{obseg!večkotnika} \pojem{obseg}. Polovica
te vsote pa je \pojem{polobseg} tega večkotnika.



             \bzgled
            If $M$ is the interior point and $s$ the semiperimeter of a triangle $ABC$, then
             $$s < AM + BM + CM < 2s.$$
             \ezgled

\begin{figure}[!htb]
\centering
\input{sl.skl.3.2.13.pic}
\caption{} \label{sl.skl.3.2.13.pic}
\end{figure}

 \textbf{\textit{Proof.}}
  Prvo neenakost dobimo, če
trikrat uporabimo trikotniško neenakost za trikotnike $MAB$,
$MBC$ in $MCA$ in jih potem seštejemo. Drugo neenakost pa dobimo,
če trikrat uporabimo prejšnjo trditev (zgled \ref{zgled3.2.9})
in seštejemo ustrezne neenakosti (Figure \ref{sl.skl.3.2.13.pic}).
 \kdokaz


             \bzgled
             In each convex pentagon there exist three diagonals,
             which are congruent to the sides of a triangle.
             \ezgled

\begin{figure}[!htb]
\centering
\input{sl.skl.3.2.14.pic}
\caption{} \label{sl.skl.3.2.14.pic}
\end{figure}

\textbf{\textit{Proof.}}
 Naj bo $AD$ najdaljša diagonala
petkotnika $ABCDE$ (naj ne bo krajša od nobene druge
diagonale). Dokažimo, da so $AD$, $AC$ in $BD$ iskane diagonale,
torej tiste, za katere obstaja trikotnik, čigar stranice so
 s temi diagonalami skladne (Figure \ref{sl.skl.3.2.14.pic}). Ker je
 $AD\geq AC$ in $AD\geq BD$, je dovolj dokazati (zgled \ref{neenaktrik1}),
da velja $AC + BD > AD$. Petkotnik $ABCDE$ je konveksen,
zato se njegovi diagonali $AC$ in
$BD$ sekata v neki točki $S$. Tedaj je: $$AC + BD > AS + SD > AD,$$ kar je bilo treba dokazati. \kdokaz

Zelo pomembna je naslednja posledica izrekov o skladnosti
trikotnikov. Tudi v tem dokazu bomo potrebovali trikotniško
neenakost.


             \bizrek \label{SkladTrikLema}
             Let $ABC$ and $A'B'C'$ triangles such that $AB \cong A'B'$
              and $AC \cong A'C'$. Then $BC > B'C'$ if and only if
             $\angle BAC > \angle B' A'C'$ i.e.
             $$BC > B'C' \Leftrightarrow \angle BAC > \angle B'
            A'C'.$$
             \eizrek

\begin{figure}[!htb]
\centering
\input{sl.skl.3.2.15.pic}
\caption{} \label{sl.skl.3.2.15.pic}
\end{figure}

\textbf{\textit{Proof.}} (Figure \ref{sl.skl.3.2.15.pic})

 ($\Leftarrow$) Naj bo $\angle BAC > \angle B' A'C'$. Tedaj obstaja
znotraj kota $BAC$ takšen poltrak $l$, da je $\angle BA,l \cong \angle B'A'C'$.
S $C''$ označimo točko poltraka $l$, za katero je
$AC'' \cong A'C'$. Tedaj sta (po izreku $SKS$) trikotnika $ABC''$ in
$A'B'C'$ skladna in je $BC'' \cong B'C'$. Dovolj je dokazati, da
velja $BC > BC''$. Če  $C''$ leži na stranici $BC$, je to trivialno
izpolnjeno. Predpostavimo, da točka $C''$ ne leži na stranici $BC$.
Naj bo točka $E$ presečišče simetrale kota $CAC''$ in stranice $BC$.
Po izreku \textit{SAS} sta skladna tudi trikotnika $ACE$ in $AC''E$, zato
je $CE \cong C''E$. Sedaj je:
$$BC = BE + EC = BE + EC'' \hspace{0.1mm} > BC'' = B'C'.$$
 ($\Rightarrow$) Naj bo $BC > B'C'$. Relacija $\angle BAC \cong
  \angle B' A'C'$ ne velja,
  ker bi bila tedaj (po izreku \textit{SAS}) trikotnika
$ABC$ in $A'B'C'$  skladna in potem tudi $BC \cong B'C'$. Če
bi pa veljalo $\angle BAC < \angle B' A'C'$,  bi iz že dokazanega
sledilo $BC < B'C'$. Torej velja $\angle BAC > \angle B' A'C'$.
 \kdokaz

             \bizrek \label{neenakIzlLin}
             If $A_1A_2\ldots A_n$ ($n\in \mathbb{N}$, $n\geq 3$) is polygonal chain, then
             $$|A_1A_2|+|A_2A_2|+\cdots +|A_{n-1}A_n|\geq |A_1A_n|.$$
             \eizrek


\begin{figure}[!htb]
\centering
\input{sl.skl.3.2.16.pic}
\caption{} \label{sl.skl.3.2.16.pic}
\end{figure}

\textbf{\textit{Proof.}} Dokaz bomo izpeljali z indukcijo po $n$
(Figure \ref{sl.skl.3.2.16.pic}).

 V primeru $n=3$ dobimo trikotniško neenakost - izrek
 \ref{neenaktrik}.

 Predpostavimo, da neenakost velja za $n=k$ ($k\in \mathbb{N}$, $k> 3$) oz.
  $|A_1A_2|+|A_2A_2|+\cdots +|A_{k-1}A_k|\geq |A_1A_k|.$ Dokažimo,
  da potem neenakost velja tudi za $n=k+1$ oz.
  $|A_1A_2|+|A_2A_2|+\cdots +|A_kA_{k+1}|\geq |A_1A_{k+1}|.$ Če
  uporabimo najprej indukcijsko
  predpostavko, nato pa trikotniško neenakost, dobimo:
  \begin{eqnarray*}
   && |A_1A_2|+|A_2A_2|+\cdots +|A_{k-1}A_k|+|A_kA_{k+1}|\geq\\
   && \geq|A_1A_k|+|A_kA_{k+1}|\geq |A_1A_{k+1}|,
  \end{eqnarray*}
 kar je bilo treba dokazati. \kdokaz

Dokažimo še eno neenakost, ki velja v poljubnem trikotniku.


             \bzgled
             If $a$, $b$, $c$ are the sides and $\alpha$, $\beta$, $\gamma$
              the opposite interior angles of a triangle, then
              $$60^0\leq \frac{a\alpha+b\beta +c\gamma}{a+b+c} < 90^0.$$
               \ezgled

\textbf{\textit{Proof.}} Dokazali bomo vsako od neenakosti
posebej. Pri tem bomo uporabili izrek o vsoti notranjih kotov
trikotnika (izrek \ref{VsotKotTrik}) Najprej bomo dokazali drugo
neenakost:
 \begin{eqnarray*}
  \frac{a\alpha+b\beta +c\gamma}{a+b+c} < 90^0
  &\Leftrightarrow& a\alpha+b\beta +c\gamma - 90^0(a+b+c)<0\\
  &\Leftrightarrow& a(\alpha-90^0)+b(\beta-90^0) +c(\gamma-90^0)<0\\
  &\Leftrightarrow& a(180^0-2\alpha)+b(180^0-2\beta) +c(180^0-2\gamma)>0\\
  &\Leftrightarrow& a(\beta+\gamma-\alpha)+b(\alpha+\gamma-\beta) +
  c(\alpha+\beta-\gamma)>0\\
  &\Leftrightarrow& \alpha(b+c-a)+\beta(a+c-b) +
  \gamma(a+b-c)>0
 \end{eqnarray*}
 Zadnja neenakost je izpolnjena, zato ker po trikotniški neenakosti (izrek
 \ref{neenaktrik}) velja $b+c-a>0$, $a+c-b>0$ in  $a+b-c>0$.
 Dokažimo še prvo neenakost:
 \begin{eqnarray*}
  && \frac{a\alpha+b\beta +c\gamma}{a+b+c} \geq 60^0\Leftrightarrow\\
  &\Leftrightarrow& a\alpha+b\beta +c\gamma - 60^0(a+b+c)\geq 0\\
  &\Leftrightarrow& a(\alpha-60^0)+b(\beta-60^0) +c(\gamma-60^0)\geq 0\\
  &\Leftrightarrow& a(3\alpha-180^0)+b(3\beta-180^0) +c(3\gamma-180^0)\geq 0\\
  &\Leftrightarrow& a(2\alpha-\beta-\gamma)+b(2\beta-\alpha-\gamma) +
  c(2\gamma-\alpha-\beta)\geq 0\\
  &\Leftrightarrow& a(\alpha-\beta+\alpha-\gamma)+b(\beta-\alpha+\beta-\gamma) +
  c(\gamma-\alpha+\gamma-\beta)\geq 0\\
  &\Leftrightarrow& a(\alpha-\beta)+a(\alpha-\gamma)+
  b(\beta-\alpha)+b(\beta-\gamma) +
  c(\gamma-\alpha)+c(\gamma-\beta)\geq 0\\
  &\Leftrightarrow& (a-b)(\alpha-\beta)+(a-c)(\alpha-\gamma)+
  (b-c)(\beta-\gamma) \geq 0
 \end{eqnarray*}
 Zadnja neenakost je posledica trditve \ref{vecstrveckotAlgeb}.
 \kdokaz

 Naslednji izrek bo motivacija za definiranje razdalje točke od
 premice.

             \bizrek Let $A'=pr_{\perp p}(A)$ be the foot of the perpendicular from a point  $A$ on a line $p$.
            If $X\in p$ and $X\neq A'$, then $AX>AA'$.
            \eizrek

\begin{figure}[!htb]
\centering
\input{sl.skl.3.2.17.pic}
\caption{} \label{sl.skl.3.2.17.pic}
\end{figure}

  \textbf{\textit{Proof.}}
 Po definiciji je $AA'\perp p$
(Figure \ref{sl.skl.3.2.17.pic}), kar pomeni, da je $AA'X$
 pravokotni trikotnik s hipotenuzo $AX$. Iz izreka
 \ref{vecstrveckotHipot} sledi $AX>AA'$.
 \kdokaz

 Če je $A'=pr_{\perp
p}(A)$, pravimo, da je dolžina daljice $AA'$ \index{razdalja!točke
 od premice} \pojem{razdalja točke $A$ od premice $p$}.
Označimo jo z $d(A,p)$. Torej $d(A,p)=|AA'|$.




%________________________________________________________________________________
 \poglavje{Circle and Line} \label{odd3KrozPrem}

V nadaljevanju se bomo ukvarjali s krožnico ter z medsebojno lego
krožnice in premice. Dokažimo najprej eno lastnost premera
krožnice, ki je enostavna posledica trikotniške neenakosti.


            \bizrek \label{premerNajdTetiva}
               The longest chord of a circle is its diameter.
             \eizrek

\begin{figure}[!htb]
\centering
\input{sl.skl.3.3.1.pic}
\caption{} \label{sl.skl.3.3.1.pic}
\end{figure}

 \textbf{\textit{Proof.}}  Naj bosta
$AB$ poljubna tetiva krožnice, ki ni premer, in $C$ točka na
premici $AS$, za katero velja $CS\cong SA$ in $\mathcal{B}(A,S,C)$
(Figure \ref{sl.skl.3.3.1.pic}). Tedaj točka $C$ leži na
krožnici $k$ in je $AC$ njen premer. Dokazali smo že (posledica
izreka \ref{premerInS}), da so vsi premeri neke krožnice
medsebojno skladni. Zato je dovolj dokazati, da je $AC>AB$. To pa
sledi iz trikotniške neenakosti (trikotnik $ASB$). Velja:
 $$AC=AS+SC=AS+SB>AB,$$ kar je bilo treba dokazati. \kdokaz

Sledi še ena lastnost tetive krožnice, kot posledica izreka
\ref{vecstrveckotHipot}.


         \bzgled \label{tetivaNotrTocke}
        Every point that lies on a chord of a circle, except its endpoints,
         is an interior point of the circle.
         \ezgled

\begin{figure}[!htb]
\centering
\input{sl.skl.3.3.2.pic}
\caption{} \label{sl.skl.3.3.2.pic}
\end{figure}

\textbf{\textit{Proof.}} Naj bo $X$ notranja točka tetive $AB$
s krajščema na krožnici $k(S,r)$ (Figure \ref{sl.skl.3.3.2.pic}). Kota $AXS$ in
$BXS$ sta sokota, kar pomeni, da nista oba ostra kota. Brez škode
za splošnost predpostavimo, da kot $BXS$ ni ostri kot. Tedaj je v
trikotniku $SXB$ stranica $SB$ najdaljša  (izrek
\ref{vecstrveckotHipot}), kar pomeni, da je:
 $$SX<SB=r.$$
Zaradi tega je $X$ notranja točka krožnice  $k(S,r)$.
 \kdokaz

Na podoben način bomo dokazali naslednji pomemben izrek.


        \bizrek \label{KrogKonv}
         A circular disc is a convex set.
          \eizrek

\begin{figure}[!htb]
\centering
\input{sl.skl.3.3.3.pic}
\caption{} \label{sl.skl.3.3.3.pic}
\end{figure}

\textbf{\textit{Proof.}} Naj bosta $A$ in $B$ dve točki kroga
$\mathcal{K}(S,r)$ (Figure \ref{sl.skl.3.3.3.pic}). Potrebno je dokazati, da
cela daljica $AB$ leži v tem krogu, oz. da to velja za poljubno
točko $X$, za katero je $\mathcal{B}(A,X,B)$. Ker točki $A$ in $B$
ležita v krogu $\mathcal{K}$, potem velja
 $SA, SB\leq r$.
Podobno kot pri dokazu prejšnjega izreka zapišemo: ker je
$\mathcal{B}(A,X,B)$, potem vsaj eden izmed sokotov $AXS$ in $BXS$ ni
oster. Brez škode za splošnost naj bo $\angle BXS\geq 90^0$. Če
uporabimo izrek \ref{vecstrveckotHipot} za trikotnik $SXB$, dobimo:
$$SX<SB\leq r.$$
 Torej točka $X$ leži v krogu $\mathcal{K}$, kar pomeni, da je $\mathcal{K}$
konveksen lik.
 \kdokaz

Intuitivno je jasno, da imata lahko premica in krožnica  največ
dve skupni točki. To dejstvo sedaj lahko tudi dokažemo.

        \bizrek \label{KroznPremPresek}
        A line and a circle can have at most two common points.
        \eizrek

\begin{figure}[!htb]
\centering
\input{sl.skl.3.3.4.pic}
\caption{} \label{sl.skl.3.3.4.pic}
\end{figure}

\textbf{\textit{Proof.}} Predpostavimo nasprotno, da imata  krožnica
$k(S,r)$ in premica $p$ vsaj tri različne skupne točke: $A$,
$B$ in $C$, oz. $A,B,C\in p\cap k$ (Figure \ref{sl.skl.3.3.4.pic}). Če središče $S$ leži na premici $p$, potem na tej premici obstajata
le dve točki, ki sta od točke $S$ oddaljeni za polmer $r$ (izrek
\ref{ABnaPoltrakCX}). Naj bo $S\notin p$.
Iz pogoja $A,B,C\in p\cap k$ sledi  $SA=SB=SC=r$,
kar pomeni, da so trikotniki $ASC$, $ASB$ in $BSC$ enakokraki.
Brez škode za splošnost predpostavimo, da je
$\mathcal{B}(A,C,B)$. Iz tega sledi (izrek \ref{enakokraki}), da
so skladni tudi koti:
 $$\angle  SCA\cong \angle SAC \cong \angle SBC \cong \angle SCB.$$
Torej sta sokota $SCA$ in $SCB$ skladna in sta zaradi tega hkrati
prava kota. Potem je  tudi kot $SAC$ pravi kot. To pa ni mogoče,
ker bi v tem primeru trikotnik $SAC$ imel dva  prava notranja kota.
To pomeni, da predpostavka $A,B,C\in p\cap k$ odpade. \kdokaz

 Iz izreka \ref{KroznPremPresek} torej sledi, da imata lahko premica in krožnica  dve, eno ali pa
nobene skupne točke. V prvem primeru pravimo, da se premica
 in krožnica
 \pojem{sekata}, v drugem primeru se \pojem{dotikata}, v tretjem
 primeru pa sta
 \pojem{mimobežni} (Figure
\ref{sl.skl.3.3.5.pic}). Premica je v prvem primeru \index{sekanta
krožnice}\pojem{sekanta} ali \pojem{sečnica}, v drugem primeru
\index{tangenta krožnice}\pojem{tangenta} ali \pojem{dotikalnica}
in v tretjem primeru \index{mimobežnica
krožnice}\pojem{mimobežnica}. Točko, v kateri se tangenta dotika krožnice, imenujemo \index{dotikališče}
\pojem{dotikališče}.

\begin{figure}[!htb]
\centering
\input{sl.skl.3.3.5.pic}
\caption{} \label{sl.skl.3.3.5.pic}
\end{figure}

Za tangento pogosto uporabljamo naslednji kriterij, ki je
pravzaprav potreben in zadosten pogoj, da je neka premica tangenta
krožnice.


        \bizrek \label{TangPogoj}
        Let $T$ be a point lying on the circle $k(S, r)$. A line
        $PT$ (lying in the plane of the circle) is a tangent of the circle at the point $T$ if and only if $PT \perp TS$.
        \eizrek


\textbf{\textit{Proof.}}  ($\Rightarrow$) Naj bo $PT$ tangenta
krožnice $k$ v točki $T$. Če kot $PTS$ ni pravi kot, je  eden
izmed kotov, ki ga določata premici $PT$ in $TS$ oster. Brez škode za
splošnost naj bo $\angle STX =w < 90°$ (Figure
\ref{sl.skl.3.3.6.pic}). Z $l$ označimo
 poltrak z izhodiščem $S$, ki leži v polravnini $STX$, tako da je
 $\angle ST,l = 180° - 2w $. Če je $Y$ presečišče poltrakov $TX$ in $l$, je
trikotnik $STY$ enakokrak ($\angle STY = \angle SYT =w$ ) in je
$ST = SY = r$. To pa ni možno, ker je $PT$ tangenta krožnice $k$
in imata eno samo skupno točko. Torej je  $\angle PTS$ pravi
kot oz. $PT \perp TS$.

\begin{figure}[!htb]
\centering
\input{sl.skl.3.3.6.pic}
\caption{} \label{sl.skl.3.3.6.pic}
\end{figure}

($\Leftarrow$) Naj bo sedaj $PT \perp TS$ (Figure
\ref{sl.skl.3.3.6.pic}). Za vsako točko $T_1\in PT$  ($T_1 \neq T$)
je $STT_1$ pravokotni trikotnik s hipotenuzo $ST_1$ in potem
velja (izrek \ref{vecstrveckotHipot}):
 $$ST_1 > ST = r.$$
 Torej nobena od točk $T_1$ ($T_1 \neq T$), ki ležijo na
premici $PT$, ne leži na krožnici $k$. To pomeni, da je premica
$PT$ tangenta te krožnice.
 \kdokaz

Iz dokaza prejšnjega izreka ($\Leftarrow$) sledi, da so vse
točke, ki ležijo na tangenti krožnice (razen njenega
dotikališča), zunanje točke te krožnice. S pomočjo te
lastnosti bomo dokazali naslednjo trditev.


        \bzgled \label{tangKrozEnaStr}
        All points of a circle are on the one side of its tangent.
        \ezgled

\begin{figure}[!htb]
\centering
\input{sl.skl.3.3.7.pic}
\caption{} \label{sl.skl.3.3.7.pic}
\end{figure}

\textbf{\textit{Proof.}}  Naj bo $T$ dotikališče krožnice $k(S, r)$
in njene tangente $t$ (Figure \ref{sl.skl.3.3.7.pic}). Tangenta $t$
deli ravnino, v kateri ležita $k$ in $t$, na dve polravnini. Tisto
polravnino, v kateri leži točka $S$, označimo z $\alpha_1$, drugo
polravnino pa z $\alpha_2$. Dokažimo, da vse točke krožnice $k$
ležijo v polravnini $\alpha_1$. Naj bo $X$ poljubna točka polravnine
$\alpha_2$. Ker sta
 točki $S$ in $X$ na različnih straneh premice $t$, sledi da jo odprta
daljica $SX$ seka v neki točki $Y$. Potem velja:
 $$SX = SY + YX > SY \geq ST = r,$$
  kar pomeni, da točka $X$ ne leži na krožnici $k$ in je njena
zunanja točka. Torej nobena od točk polravnine $\alpha_2$ ne
leži na krožnici $k$, oz.  so vse  v  polravnini
$\alpha_1$ z robom $t$. \kdokaz

Direktna posledica izreka \ref{TangPogoj} je tudi ta, da v vsaki
točki krožnice lahko narišemo eno samo tangento. Če je $X$
notranja točka krožnice $k(S, r)$, potem skozi to točko ne
poteka nobena tangenta, saj so vse premice skozi $X$
sekante, kar je posledica Dedekindovega aksioma (izrek
\ref{DedPoslKrozPrem}). Kasneje (izrek \ref{tangentiKroznice})
bomo ugotovili, da lahko skozi vsako zunanjo točko krožnice
narišemo natanko dve tangenti. Zaenkrat dokažimo naslednjo
trditev (bralec se bo spomnil, da gre za trditev, ki smo jo
obravnavali že na samem začetku v uvodnem poglavju - trditev
\ref{TalesUvod}).


            \bizrek \label{TalesovIzrKroz} \index{izrek!Talesov za krožnico}
            Thales' theorem for a circle\footnote{Starogrški
            filozof in matematik \textit{Tales}
            \index{Tales} iz Mileta (640--546 pr. n. š.)
             ni prvi, ki je odkril to trditev. Kot empirično dejstvo so jo poznali
             že stari Egipčani in Babilonci. Izrek imenujemo po Talesu, ki ga je prvi dokazal.
             V dokazu je uporabljal lastnosti enakokrakih trikotnikov in dejstvo, da je
             vsota notranjih kotov trikotnika enaka vsoti dveh pravih kotov. Torej je
             dokaz enak temu, ki ga bomo izpeljali tukaj.}:\\
            Let $AB$ be a diameter of a circle $k$. Then for any point $X$ of this circle different from $A$ and $B$
            ($X\in k$ in $X\neq A$ in $X\neq B$) is $\angle AXB=90^0$
            \index{izrek!Talesov za krožnico}
            \eizrek

\begin{figure}[!htb]
\centering
\input{sl.skl.3.3.8.pic}
\caption{} \label{sl.skl.3.3.8.pic}
\end{figure}

\textbf{\textit{Proof.}} Naj bo $O$ središče krožnice $k$ (Figure
\ref{sl.skl.3.3.8.pic}). Ker $A,B,X\in k$, je
 $OA\cong OB\cong OX$. Torej sta
 trikotnika $AOX$ in $BOX$  enakokraka, zato je (izrek
\ref{enakokraki}):
 $\angle AXO\cong\angle XAO=\alpha$ in $\angle BXO\cong\angle XBO=\beta$.
Tedaj je $\angle AXB=\alpha+\beta$.
 Vsota notranjih kotov v
trikotniku $AXB$ je enaka $180^0$ (izrek \ref{VsotKotTrik}), torej
je $2\alpha+2\beta=180^0$. Iz tega sledi
 $$\angle AXB=\alpha+\beta=90^0,$$ kar je bilo treba dokazati. \kdokaz

 Dokažimo tudi obratno trditev.


             \bizrek \label{TalesovIzrKrozObrat}
            If $A$, $B$ and $X$ are non-collinear points such that
            $\angle AXB=90^0$, then the point $X$ lies on a circle with the diameter $AB$.
             \eizrek

\begin{figure}[!htb]
\centering
\input{sl.skl.3.3.9.pic}
\caption{} \label{sl.skl.3.3.9.pic}
\end{figure}


\textbf{\textit{Proof.}} Naj bo $O$ središče daljice $AB$ in $k$
krožnica s središčem $O$ in polmerom $OA$ oz. premerom $AB$
(Figure \ref{sl.skl.3.3.9.pic}). Iz $\angle AXB=90^0$ je po izreku
\ref{VsotKotTrik}:
 \begin{eqnarray}
 \angle XAB+ \angle XBA = 90^0 \label{relacija336}
 \end{eqnarray}

Dokažimo $X\in k$. Predpostavimo nasprotno, torej da točka $X$ ne
leži na krožnici $k$. V tem primeru je $OX\neq OA$. Naj bo $X_1$
točka na poltraku $OX$, za katero je $OX_1\cong OA$ (izrek
\ref{ABnaPoltrakCX}). To pomeni, da točka $X_1$ leži na krožnici
$k$ in po Talesovem izreku \ref{TalesovIzrKroz} velja $\angle
AX_1B=90^0$.

Po naši predpostavki $OX\neq OA$ je jasno, da $X\neq X_1$.
Obravnavali bomo dve možnosti:

\textit{1)} Naj bo $OX_1<OX$ oz. $\mathcal{B}(O,X_1,X)$. V tem
primeru je $X_1$ notranja točka kotov $XAB$ in $XBA$, zato je
$\angle X_1AB<\angle XAB$ in $\angle X_1BA<\angle XBA$. Iz tega in
relacije \ref{relacija336} sledi:
 $$\angle X_1AB+ \angle X_1BA<\angle XAB+ \angle XBA = 90^0.$$
 Ker je še $\angle AX_1B=90^0$, je vsota kotov v trikotniku $AX_1B$
 manjša od $180^0$, kar po izreku \ref{VsotKotTrik} ni mogoče.


\textit{2)} Naj bo $OX_1>OX$ oz. $\mathcal{B}(O,X,X_1)$. Podobno
kot v prvem primeru dobimo:
 $$\angle X_1AB+ \angle X_1BA>\angle XAB+ \angle XBA = 90^0.$$
 V tem primeru je vsota kotov v trikotniku $AX_1B$ večja od
$180^0$, kar  po izreku \ref{VsotKotTrik} ni mogoče.

Iz tega sledi, da je $OX=OX_1$  oz. $X\in k$.
 \kdokaz


 Uporabimo prejšnja izreka za načrtovanje tangent.


             \bzgled \label{tangKrozKonstr}
             Let $A$ be an exterior point of a circle $k(S,r)$.
             Construct all tangents of this circle passing through the point $A$.
             \ezgled

\begin{figure}[!htb]
\centering
\input{sl.skl.3.3.10.pic}
\caption{} \label{sl.skl.3.3.10.pic}
\end{figure}


 \textbf{\textit{Solution.}} Naj bo $l$ krožnica s premerom $SA$
(Figure \ref{sl.skl.3.3.10.pic}). Ker je $S$ notranja, $A$ pa zunanja
točka dane krožnice $k$, imata krožnici $k$ in $l$ natanko
dve skupni točki $T_1$ in $T_2$ (izrek \ref{DedPoslKrozKroz}).
Po Talesovem izreku \ref{TalesovIzrKroz} je
$\angle ST_1A\cong \angle ST_2A=90^0$. Ker sta $ST_1$ in  $ST_2$
polmera krožnice $k$, sta $AT_1$ in $AT_2$ tangenti krožnice $k$
skozi točko $A$ (izrek \ref{TangPogoj}).

 Dokažimo, da sta $AT_1$ in $AT_2$ edini tangenti krožnice $k$
 iz točke $A$. Če je $AT$ tangenta iz točke $A$, ki se krožnice $k$
 dotika v točki $T$, je po izreku \ref{TangPogoj} $\angle ATS=90^0$.
 To pomeni, da točka $T$ leži na krožnici $l$
 (izrek \ref{TalesovIzrKrozObrat}) oz. $T\in k\cap l$. Torej $T$ je ena
 od točk $T_1$ in $T_2$, zato sta $AT_1$ in $AT_2$ edini tangenti krožnice
  $k$ iz točke $A$.
  \kdokaz

  Iz prejšnje konstrukcije sledi naslednji izrek.



        \bizrek \label{tangentiKroznice}
         If $V$ is an exterior point of a circle $k(S,r)$, then there are exactly
       two tangents of the circle $k$ through the point $V$.
        \eizrek


 Dokažimo še nekaj lastnosti tangent krožnice.


            \bzgled \label{TangOdsek}
            If $VA$ and $VB$ are tangents of a circle $k(S,r)$ in a points
            $A$ and $B$ of the circle, then the centre $S$ lies on the bisector
            of the angle $AVB$ and $VA \cong VB$.
             \ezgled

\begin{figure}[!htb]
\centering
\input{sl.skl.3.3.11.pic}
\caption{} \label{sl.skl.3.3.11.pic}
\end{figure}


\textbf{\textit{Proof.}} Iz izreka \ref{TangPogoj} sledi: $VA
\perp AS$ in $VB \perp BS$
(Figure \ref{sl.skl.3.3.11.pic}). Torej sta $ASV$ in $BSV$  pravokotna
trikotnika s skupno hipotenuzo $SV$. Ker je še $SA \cong SB = r$,
sta ta dva trikotnika skladna (izrek \textit{SSA} \ref{SSK}). Torej sta tudi kota
$AVS$ in $BVS$  skladna, kar pomeni, da je premica $VS$
simetrala kota $AVB$. Iz skladnosti teh dveh trikotnikov sledi
tudi $VA \cong VB$.
 \kdokaz

Velja tudi obratna trditev.


             \bzgled \label{SimKotaKraka}
             If a point $S$ lies on the bisector of a convex angle,
            then it is the centre of a circle touching both sides of this angle.
            \ezgled

\begin{figure}[!htb]
\centering
\input{sl.skl.3.3.12.pic}
\caption{} \label{sl.skl.3.3.12.pic}
\end{figure}

\textbf{\textit{Proof.}} Naj bosta $A$ in $B$ pravokotni projekciji
točke $S$ na krakih danega kota z vrhom $V$ (Figure
\ref{sl.skl.3.3.12.pic}). Trikotnika $ASV$ in $BSV$ sta skladna
(izrek \textit{ASA} \ref{KSK}), ker imata skupno stranico $VS$ in
dva para skladnih kotov - iz $\angle AVS\cong \angle BVS$ in $\angle
SAV\cong \angle SBV=90^0$ sledi $\angle ASV\cong \angle BSV$. Zaradi
tega velja $SA\cong SB$ in je  $k(S,SA)$  iskana krožnica. Kraka
danega kota sta namreč  po izreku \ref{TangPogoj} tangenti krožnice.
 \kdokaz

Sedaj bomo dokazali še en kriterij o medsebojni legi premice in
krožnice v ravnini.



        \bizrek \label{TangSekMimobKrit}
        Let $P$ be the foot of the perpendicular from the centre of a circle  $k(S,r)$
            on a line $p$ (lying in the plane of the circle). Then the line $p$ is:

        (i) secant, if and only if $SP < r$;

        (ii) tangent, if and only if  $SP \cong r$;

         (iii) non-intersecting line,  if and only if  $SP> r$.
        \eizrek

\begin{figure}[!htb]
\centering
\input{sl.skl.3.3.13.pic}
\caption{} \label{sl.skl.3.3.13.pic}
\end{figure}

\textbf{\textit{Proof.}} (Figure \ref{sl.skl.3.3.13.pic})

Trditev (ii) sledi direktno iz kriterija
za tangento (izrek \ref{TangPogoj}).

(i) V dokazu direktne smeri ekvivalence uporabljamo dejstvo, da je
hipotenuza pravokotnega trikotnika daljša od obeh katet (izrek
\ref{vecstrveckotHipot}). Če sta $A$ in $B$ presečišči sekante $p$
in krožnice $k$, je $SA$ hipotenuza pravokotnega trikotnika $ASP$ in
velja:
 $r \cong SA > SP$.

 V dokazu obratne smeri
ekvivalence uporabimo  posledico Dedekindovega aksioma (izrek
\ref{DedPoslKrozPrem}). Ker je v tem primeru $P$ notranja točka te
krožnice, je vsaka premice te ravnine, ki gre skozi
točko $P$, sekanta krožnice $k$.

(iii) Sledi iz dokazanega (i) in (ii). Če je namreč $SP > r$,
potem ni niti $SP < r$ niti $SP \cong r$. Iz ekvivalenc (i) in (ii)
sledi, da premica $p$ ni niti sekanta niti tangenta. Zato je $p$
mimobežnica krožnice $k$. Na isti način dokazujemo tudi obratno smer
ekvivalence.
  \kdokaz

%________________________________________________________________________________
 \poglavje{Quadrilaterals} \label{odd3Stirik}


V razdelku \ref{odd2AKSURJ} smo  pojem štirikotnika vpeljali kot
večkotnik, ki ima štiri stranice in štiri oglišča.
Definirali smo pojme sosednji in nasprotni stranici, sosednji in
nasprotni koti ter diagonalo. Štirikotniku $ABCD$ dolžine
njegovih stranic $AB$, $BC$, $CD$ in $DA$ običajno označimo z $a$,
$b$, $c$ in $d$, dolžini njegovih diagonal $AC$ in $BD$ pa z $e$ in
$f$  (Figure \ref{sl.skl.3.4.1.pic}).

\begin{figure}[!htb]
\centering
\input{sl.skl.3.4.1.pic}
\caption{} \label{sl.skl.3.4.1.pic}
\end{figure}

 V istem razdelku  smo kot pojme vpeljali  notranje
  in zunanje kote štirikotnika. Omenili smo tudi, da notranje kote ob ogliščih $A$, $B$, $C$ in $D$ štirikotnika $ABCD$
   navadno označimo
   z $\alpha$, $\beta$, $\gamma$ in $\delta$, njegove zunanje kote
    pa z $\alpha'$, $\beta'$, $\gamma'$ in $\delta'$.
Dokazali smo (posledica splošnega izreka \ref{VsotKotVeck}), da je
vsota vseh štirih notranjih kotov poljubnega štirikotnika
enaka $360^0$ (Figure \ref{sl.skl.3.4.1.pic}). Enaka  je tudi vsota zunanjih kotov (v konveksnem štirikotniku). Torej:
 \begin{eqnarray*}
 \alpha+\beta+\gamma+\delta=360^0,\\
 \alpha'+\beta'+\gamma'+\delta'=360^0
 \end{eqnarray*}

Dodajmo še, da za notranja kota pravimo, da sta
\index{kota!sosednja} \pojem{sosednja} oz. \index{kota!nasprotna} \pojem{nasprotna}, če sta pripadajoči oglišči sosednji oz. nasprotni.



Sedaj bomo nekatere vrste štirikotnikov podrobneje obravnavali.

 Štirikotnik
$ABCD$ je \index{trapez}\pojem{trapez}, če je $AB\parallel CD$ (Figure \ref{sl.skl.3.4.2.pic}).
Stranici $AB$ in $CD$ sta \index{osnovnica!trapeza}
\pojem{osnovnici}, $BC$ in $AD$ pa
\index{krak!trapeza}\pojem{kraka} tega trapeza.
Daljica $PQ$ ($P\in AB$, $Q\in CD$ in $PQ\perp AB$) je \index{višina!trapeza}\pojem{višina trapeza}. Pogosto jo označimo z $v$.


Trapez je
 \index{trapez!enakokraki}\pojem{enakokraki},
 Če je $BC \cong AD$ in ni $BC \parallel AD$,
 oz. \index{trapez!pravokotni}\pojem{pravokotni}, če je vsaj
eden  notranji kot pravi.




\begin{figure}[!htb]
\centering
\input{sl.skl.3.4.2.pic}
\caption{} \label{sl.skl.3.4.2.pic}
\end{figure}


Dva notranja kota ob istem kraku trapeza sta suplementarna, ker
sta kota z vzporednima krakoma (izrek \ref{KotiTransverzala}). Suplementarnost teh kotov je tudi
zadosten pogoj, da je štirikotnik trapez. Iz tega sledi, da ima pravokotni trapez  vsaj dva prava notranja kota.


Kot posebno vrsto trapezov dobimo še eno skupino štirikotnikov.
To so \pojem{paralelogrami}. Možno jih je definirati na različne načine.
Izbrali bomo enega, za vse ostale pa bomo
dokazali, da so ekvivalentni.

Štirikotnik $ABCD$ je \index{paralelogram} \pojem{paralelogram}, če
velja $AB
\parallel CD$ in $AD \parallel BC$ (Figure \ref{sl.skl.3.4.3.pic}).
Daljici $PQ$ ($P\in AB$, $Q\in CD$ in $PQ\perp AB$) in $MN$ ($M\in BC$, $N\in AD$ in $MN\perp BC$) sta
  \index{višina!paralelograma}\pojem{višini paralelograma}. Pogosto ju označimo z $v_a$ in $v_b$.



\begin{figure}[!htb]
\centering
\input{sl.skl.3.4.3.pic}
\caption{} \label{sl.skl.3.4.3.pic}
\end{figure}

 Paralelogram je torej štirikotnik, ki ima dva para
vzporednih stranic. V definiciji paralelograma ni uporabljen pojem
skladnosti. Paralelograme (tudi trapeze) lahko obravnavamo
tudi v t. i. \index{geometrija!afina} \pojem{afini geometriji}.
To je geometrija, ki je zasnovana na vseh aksiomah evklidske
geometrije, če iz seznama izključimo aksiome tretje skupine -
aksiome skladnosti.

Dokažimo sedaj že omenjene ekvivalente za definicijo
paralelograma.


            \bizrek  \label{paralelogram}
            Let $ABCD$ be a convex quadrilateral.
            Then the following statements are equivalent:
            \begin{enumerate}
              \item The quadrilateral $ABCD$ is a parallelogram.
              \item Any two adjacent interior angles of  the quadrilateral $ABCD$ are supplementary.
              \item Any two opposite interior angles of  the quadrilateral $ABCD$ are congruent.
             \item $AB \parallel CD$ and $AB \cong CD$\footnote{Ta ekvivalent
                v nekoliko drugačni obliki navaja \index{Evklid}
                \textit{Evklid iz Aleksandrije} (3. stol. pr. n. š.) v
                prvi knjigi svojih ‘‘Elementov’’.}.
             \item $AB \cong CD$ and $AD \cong BC$.
             \item The diagonals of the quadrilateral $ABCD$ bisect each other, i.e.
               line segments $AC$ and $BD$ have a common midpoint.
            \end{enumerate}
             \eizrek

 \textbf{\textit{Proof.}}
Potrebno je dokazati ekvivalentnost vseh izjav $(1)-(6)$. Da bi se
izognili dokazovanju vseh ekvivalenc (po dve implikaciji - npr. z
izjavo (1), kar bi skupaj dalo 10 implikacij), bomo dokaz nekoliko
poenostavili, tako da dokažemo implikacije po naslednji shemi.

\vspace*{5mm}
\hspace*{25mm}
$\begin{array}{ccccccc}
  \textit{(1)} & \Leftarrow & \textit{(2)} & \Leftarrow & \textit{(3)} &   &   \\
  \Downarrow &   &   &   & \Uparrow &   &   \\
  \textit{(4)} &   & \Rightarrow &   & \textit{(5)} & \Leftrightarrow & \textit{(6)}
\end{array}$

\vspace*{5mm}

 Kot vidimo, je to dovolj, ker implikacija $\textit{(1)}
\Rightarrow \textit{(2)}$ sledi direktno iz: $\textit{(1)}\Rightarrow \textit{(4)}\Rightarrow
\textit{(5)}\Rightarrow \textit{(3)} \Rightarrow \textit{(2)}$.

Označimo z $\alpha$, $\beta$, $\gamma$ in $\delta$ notranje kote pri
ogliščih $A$, $B$, $C$ in $D$ štirikotnika $ABCD$. Štirikotnik
$ABCD$ je konveksen, kar pomeni, da se njegovi diagonali sekata v
neki točki $S$.

$\textit{(2)}\Rightarrow \textit{(1)}$. Naj bosta kota $\alpha$ in
$\beta$ suplementarna  (Figure \ref{sl.skl.3.4.4.pic}). Potem sta to
kota ob transverzali $AB$ premic $AD$ in $BC$, zato je $AD\parallel
BC$ (izrek \ref{KotiTransverzala}). Podobno iz suplementarnosti
kotov $\beta$ in $\gamma$ sledi $AB\parallel CD$. Torej je štirikotnik
$ABCD$ paralelogram.

\begin{figure}[!htb]
\centering
\input{sl.skl.3.4.4.pic}
\caption{} \label{sl.skl.3.4.4.pic}
\end{figure}

$\textit{(3)}\Rightarrow\textit{(2)}$. Naj bo $\alpha =\gamma$ in $\beta =\delta$ (Figure \ref{sl.skl.3.4.4.pic}).
Ker je $\alpha + \beta +\gamma +\delta = 360°$ (vsota vseh
notranjih kotov v štirikotniku je $360°$ - izrek \ref{VsotKotVeck}), sledi
$\alpha + \beta =180°$ in $\beta +\gamma = 180°$.


\begin{figure}[!htb]
\centering
\input{sl.skl.3.4.4a.pic}
\caption{} \label{sl.skl.3.4.4a.pic}
\end{figure}

$\textit{(1)}\Rightarrow\textit{(4)}$. Naj bo štirikotnik $ABCD$
paralelogram, oz. naj velja $AB \parallel CD$ in $AD \parallel BC$
(Figure \ref{sl.skl.3.4.4a.pic}). Dokažimo, da je potem tudi $AB \cong CD$.
Premica $AC$ je transverzala vzporednic $AB$ in $CD$, kar
pomeni, da sta kota $CAB$ in $ACD$ izmenična kota na tej
transverzali in sta zato skladna. Podobno iz vzporednosti premic
$AD$ in $BC$ sledi, da sta tudi kota $ACB$ in $CAD$ skladna. Ker je
$AC \cong AC$, sta trikotnika $ACB$ in $CAD$ skladna (izrek
\ref{KSK} - \textit{ASA}). Zato je  $AB \cong CD$.

\begin{figure}[!htb]
\centering
\input{sl.skl.3.4.4b.pic}
\caption{} \label{sl.skl.3.4.4b.pic}
\end{figure}

 $\textit{(4)}\Rightarrow \textit{(5)}$. Naj bo $ABCD$ takšen štirikotnik, da velja
 $AB \parallel CD$ in $AB \cong CD$ (Figure \ref{sl.skl.3.4.4b.pic}).
 Dokažimo, da je  $AD \cong BC$.
 Premica $AC$ je transverzala vzporednic $AB$ in
$CD$, kar pomeni, da sta kota $CAB$ in $ACD$ izmenična kota ob tej
transverzali in sta zato skladna. Ker je še $AC \cong AC$, sta
trikotnika $ACB$ in $CAD$ skladna (izrek \ref{SKS} - \textit{SAS}). Zato je
$BC \cong AD$.


\begin{figure}[!htb]
\centering
\input{sl.skl.3.4.4c.pic}
\caption{} \label{sl.skl.3.4.4c.pic}
\end{figure}

 $\textit{(5)}\Rightarrow \textit{(3)}$. Naj bo $ABCD$ takšen štirikotnik, da je $AB \cong CD$ in
 $AD \cong BC$ (Figure \ref{sl.skl.3.4.4c.pic}). Dokažimo, da
je potem $\beta =\delta$ in $\alpha =\gamma$. Ker je še $AC \cong
AC$, sta trikotnika $ACB$ in $CAD$ skladna (izrek \ref{SSS} - \textit{SSS}).
Iz tega sledi $\angle ABC \cong \angle CDA$ oz. $\beta =\delta$. Na
podoben način dokažemo tudi $\alpha =\gamma$.

\begin{figure}[!htb]
\centering
\input{sl.skl.3.4.4d.pic}
\caption{} \label{sl.skl.3.4.4d.pic}
\end{figure}

$\textit{(5)}\Leftrightarrow \textit{(6)}$. Naj bo $ABCD$ takšen štirikotnik, da
je $AB \cong CD$ in $AD \cong BC$ (Figure \ref{sl.skl.3.4.4d.pic}). Dokažimo, da je točka $S$
skupno središče njegovih diagonal $AC$ in $BD$. Ker je $AC \cong
AC$ je $\triangle ACB \cong \triangle CAD$ (izrek \ref{SSS} - \textit{SSS}). Zato
je $\angle ACB \cong \angle CAD$ oz. $\angle SCB \cong \angle
SAD$. Iz skladnosti sovršnih kotov $CSB$ in $ASD$ sledi, da sta
skladna tudi kota $SBC$ in $SDA$. Ker je še $AD \cong BC$, sledi
$\triangle CSB \cong \triangle ASD$ (izrek \ref{KSK} - \textit{ASA}). Zato je $SB
\cong SD$ in $SC \cong SA$ oz. točka $S$ je skupno središče
diagonal $AC$ in $BD$.


\begin{figure}[!htb]
\centering
\input{sl.skl.3.4.4e.pic}
\caption{} \label{sl.skl.3.4.4e.pic}
\end{figure}


Obratno, naj bo $S$ skupno središče diagonal $AC$ in $BD$ (Figure \ref{sl.skl.3.4.4e.pic}). Tedaj
je $SB \cong SD$ in $SC \cong SA$. Skladna sta tudi sovršna kota
$CSB$ in $ASD$, zato je $\triangle CSB \cong \triangle ASD$ (izrek
\ref{SKS} - \textit{SAS}). Iz tega sledi $AD \cong BC$. Na podoben način
dokažemo tudi $AB \cong CD$.
 \kdokaz

Bralcu priporočamo, da dokaže prejšnji izrek z uporabo neke
podobne sheme. To bo dobra vaja za uporabo izrekov o skladnosti
trikotnikov.

Definirajmo sedaj še neke vrste štirikotnikov, za katere se bo
izkazalo, da so posebni primeri paralelogramov (Figure
\ref{sl.skl.3.4.5.pic}).

\begin{figure}[!htb]
\centering
\input{sl.skl.3.4.5.pic}
\caption{} \label{sl.skl.3.4.5.pic}
\end{figure}

Štirikotnik, ki ima vse stranice skladne, imenujemo \index{romb}
\pojem{romb}.

Štirikotnik, pri katerem so vsi notranji koti skladni (in torej
enaki $90^0$, ker je njihova vsota $360^0$), je \index{pravokotnik}
\pojem{pravokotnik}.

Štirikotnik, pri katerem so vse stranice skladne in vsi notranji
koti skladni (in enaki $90^0$), imenujemo \index{kvadrat}
\pojem{kvadrat}.


Ni težko dokazati, da je vsak od teh štirikotnikov hkrati
paralelogram. To je direktna posledica prejšnjega izreka. Romb je
paralelogram zaradi $\textit{(5)}\Rightarrow\textit{(1)}$; pravokotnik pa zaradi
$\textit{(2)}\Rightarrow\textit{(1)}$ (ali $\textit{(3)}\Rightarrow\textit{(1)}$). Za kvadrat je
jasno, da je hkrati pravokotnik in romb, zato je tudi
paralelogram.

Iz prejšnjega izreka \ref{paralelogram} - ekvivalent (\textit{5}) sledi,
da je paralelogram, ki ima dve sosednji stranici skladni, romb. Prav tako
 po istem izreku iz ekvivalentov \textit{(2)} in \textit{(3)} sledi,
 da je paralelogram, ki ima vsaj en pravi kot, pravokotnik.

Naslednji izrek daje dodatne kriterije, kdaj je paralelogram hkrati romb,
pravokotnik oz. kvadrat. Ta izrek se nanaša na diagonale. Pri
paralelogramu se diagonali vedno razpolavljata, toda pri rombu,
pravokotniku in kvadratu bomo imeli še dodatne lastnosti.



        \bizrek \label{RombPravKvadr} $ $  (Figure \ref{sl.skl.3.4.5a.pic})

        a) A parallelogram is a rhombus if and only if their diagonals are perpendicular.

         b) A parallelogram is a rectangle if and only if their diagonals are congruent.

        c) A parallelogram is a square if and only if their diagonals are perpendicular and congruent.
        \eizrek

\begin{figure}[!htb]
\centering
\input{sl.skl.3.4.5a.pic}
\caption{} \label{sl.skl.3.4.5a.pic}
\end{figure}

 \textbf{\textit{Proof.}}
 Naj bo $ABCD$ paralelogram in $S$ presečišče njegovih diagonal $AC$ in $BD$. Po
prejšnjem izreku \ref{paralelogram} je točka $S$ njuno skupno
središče. Iz istega izreka sledi tudi $AB \cong CD$ in $AD \cong
BC$.

\begin{figure}[!htb]
\centering
\input{sl.skl.3.4.6.pic}
\caption{} \label{sl.skl.3.4.6.pic}
\end{figure}

\textit{a)}  (Figure \ref{sl.skl.3.4.6.pic})

 Če je $ABCD$ romb, ima vse stranice skladne. Zato sta
trikotnika $ABS$ in
  $ADS$
skladna (izrek \ref{SSS} - \textit{SSS}). Potem sta skladna tudi kota $ASB$
in $ASD$ in sta (kot sokota) oba prava kota. To pomeni, da sta
diagonali pravokotni.

Če sta diagonali paralelograma $ABCD$ pravokotni, sta trikotnika
$ABS$ in $ADS$ skladna (izrek \ref{SKS} - \textit{SAS}). Zato sta stranici
$AB$ in $AD$ skladni. Na podoben način dokazujemo, da so skladne
vse stranice tega paralelograma, kar pomeni, da je paralelogram
romb.


\begin{figure}[!htb]
\centering
\input{sl.skl.3.4.6a.pic}
\caption{} \label{sl.skl.3.4.6a.pic}
\end{figure}


\textit{b)}  (Figure \ref{sl.skl.3.4.6a.pic})

 Če je $ABCD$ pravokotnik, ima vse notranje kote skldne in prave. Tedaj
sta trikotnika $ABC$ in $DCB$ skladna (izrek \ref{SKS} - \textit{SAS}).
Zato je $AC \cong DB$.

Če pri paralelogramu $ABCD$ velja $AC \cong DB$, sta trikotnika
$ABC$ in $DCB$ skladna (izrek \ref{SSS} - \textit{SSS}). Iz tega sledi, da
sta skladna notranja kota pri ogliščih $B$ in $D$ paralelograma
$ABCD$. Po prejšnjem izreku \ref{paralelogram} sta kota
suplementarna, kar pomeni, da sta oba prava kota. Analogno so pravi
tudi vsi koti tega paralelograma, zato je paralelogram pravokotnik.

 \textit{c)} Paralelogram je kvadrat natanko tedaj, ko je hkrati romb in pravokotnik.
 Slednje je pa
izpolnjeno natanko tedaj, ko sta diagonali pravokotni in
skladni, kar sledi iz dokazanega  (\textit{a.} in \textit{b.}).
 \kdokaz


\begin{figure}[!htb]
\centering
\input{sl.skl.3.4.7.pic}
\caption{} \label{sl.skl.3.4.7.pic}
\end{figure}



Ker sta diagonali pravokotnika skladni in se razpolavljata,
obstaja krožnica, ki vsebuje vsa oglišča tega pravokotnika (Figure \ref{sl.skl.3.4.7.pic}). To
je t. i. \index{očrtana krožnica!pravokotnika} \pojem{očrtana
krožnica pravokotnika}. Njeno središče je presečišče
njegovih diagonal. Če je namreč točka $S$ presečišče
diagonal pravokotnika $ABCD$, potem iz prejšnjega izreka
\ref{RombPravKvadr} sledi:
 $$SA \cong SC \cong SB \cong SD.$$
Polmer te krožnice je enak polovici diagonale pravokotnika. Ker
je kvadrat posebna vrsta pravokotnika, tudi za njega obstaja
očrtana krožnica.

Dokažimo sedaj še pomembno lastnost enakokrakih trapezov.


     \bizrek \label{trapezEnakokraki}
     Interior base angles of an isosceles trapezium are congruent.
     The diagonals of an isosceles trapezium are congruent line segments.
     \eizrek


\begin{figure}[!htb]
\centering
\input{sl.skl.3.4.8.pic}
\caption{} \label{sl.skl.3.4.8.pic}
\end{figure}

  \textbf{\textit{Proof.}}
  Naj bo $ABCD$ enakokraki trapez z osnovnico $AB$ (Figure \ref{sl.skl.3.4.8.pic}). Brez škode za splošnost predpostavimo, da je $AB>CD$. Po definiciji enakokrakega trapeza je $BC\cong AD$. Označimo s $C'$ in $D'$ pravokotni projekciji oglišč $C$ in $D$ na premici $AB$. Štirikotnik $D'C'CD$ je paralelogram s pravim kotom, zato je pravokotnik. Že iz dejstva, da je $D'C'CD$ paralelogram, sledi $CC'\cong DD'$ (izrek \ref{paralelogram}). Ker je še $\angle CC'B\cong \angle DD'A=90^0$, sta trikotnika $CC'B$ in $DD'A$ skladna (izrek \textit{SSA} \ref{SSK}), zato je $\beta=\angle CBC'\cong \angle DAD'=\alpha$.

  Dokažimo še, da sta diagonali $AC$ in $BD$ skladni. To sledi iz skladnosti trikotnikov $ABC$ in $BAD$ (izrek \textit{SAS} \ref{SKS}).
  \kdokaz


            \bzgled
            Let $ABCD$, $AEBK$ and $CEFG$ be equally oriented squares in a plane.
            Then $B$, $D$ and $F$ are collinear points and the point $B$ is a midpoint of the line segment $DF$.
             \ezgled


\begin{figure}[!htb]
\centering
\input{sl.skl.3.4.9.pic}
\caption{} \label{sl.skl.3.4.9.pic}
\end{figure}

 \textbf{\textit{Proof.}} (Figure \ref{sl.skl.3.4.9.pic})

 Trikotnika $CAE$ in $FBE$ sta skladna, ker je $CE\cong FE$, $AE\cong BE$  in
 $\angle AEC=90^0-\angle CEB=\angle BEF$ (izrek \ref{SKS} - \textit{SAS}). Zato je
 $\angle EBF$ pravi kot  in so točke $D$, $B$ in $F$ kolinearne. Iz
 skladnosti teh dveh trikotnikov sledi tudi $BF\cong AC\cong BD$.
  \kdokaz


\begin{figure}[!htb]
\centering
\input{sl.skl.3.4.10.pic}
\caption{} \label{sl.skl.3.4.10.pic}
\end{figure}

Razen trapezov in paralelogramov bomo definirali še eno novo skupino
štirikotnikov. Štirikotnik $ABCD$ je
\index{deltoid}\pojem{deltoid}, če sta njegovi diagonali pravokotni in
ena od diagonal razpolavlja drugo (Figure \ref{sl.skl.3.4.10.pic}).
Naslednji izrek se nanaša na deltoid in je ekvivalent njegove
definicije.


        \bzgled
        A quadrilateral is a deltoid if and only if it has two pairs of congruent adjacent sides.
        \ezgled

\begin{figure}[!htb]
\centering
\input{sl.skl.3.4.11.pic}
\caption{} \label{sl.skl.3.4.11.pic}
\end{figure}

\textbf{\textit{Proof.}} (Figure \ref{sl.skl.3.4.11.pic})

 Naj bo štirikotnik $ABCD$ deltoid. Tedaj sta diagonali
$AC$ in $BD$ pravokotni in ena od diagonal razpolavlja drugo. Brez
škode za splošnost naj diagonala $BD$ razpolavlja diagonalo $AC$.
Sledi, da sta pravokotna trikotnika $ABS$ in $CBS$ skladna (izrek
\ref{SKS} - \textit{SAS}). Potem je $AB \cong CB$. Iz skladnosti
trikotnikov $ADS$ in $CDS$ pa sledi $AD \cong CD$.

 Naj bo $ABCD$ štirikotnik, v katerem velja $AB \cong CB$ in $AD \cong CD$.
 Trikotnika $ABD$ in $CBD$ sta
skladna (izrek \ref{SSS} - \textit{SSS}), zato sta skladna tudi kota $ADS$
in $CDS$. Iz tega sledi, da sta skladna tudi trikotnika $ADS$ in
$CDS$ (izrek \ref{SKS} - \textit{SAS}). Zato je $S$ središče diagonale $AC$, kota $DSA$ in $DSC$
pa sta  prava kota, ker sta skladna sokota.
 \kdokaz



            \bzgled
            Let $k_1(S_1,r)$, $k_2(S_2,r)$, $k_3(S_3,r)$ be congruent circles and
            $k_1\cap k_2=\{B,A_3\}$, $k_2\cap k_3=\{B,A_1\}$, $k_3\cap k_1=\{B,A_2\}$.
            Prove that the lines $S_1A_1$,
             $S_2A_2$ and $S_3A_3$ intersect at a single point .
            \ezgled

\begin{figure}[!htb]
\centering
\input{sl.skk.4.2.12.pic}
\caption{} \label{sl.skk.4.2.12.pic}
\end{figure}

\textbf{\textit{Proof.}}
 Dokazali bomo še več, da imajo daljice $O_1A_1$, $O_2A_2$ in $O_3A_3$
 isto središče, oziroma da so ustrezni štirikotniki
paralelogrami (Figure \ref{sl.skk.4.2.12.pic}). Ker so $k_1$, $k_2$
in $k_3$ skladne krožnice, sta štirikotnika $O_1A_2O_3B$ in $O_2A_1O_3B$
 romba. Zaradi tega sta daljici $O_1A_2$ in $O_2A_1$ skladni in
vzporedni, kar pomeni, da je štirikotnik $O_1A_2A_1O_2$ paralelogram
(izrek \ref{paralelogram}). Iz istega izreka sledi, da imata njegovi
diagonali $O_1A_1$ in $O_2A_2$  skupno središče. Na podoben
način dokazujemo, da imata tudi daljici $O_2A_2$ in $O_3A_3$  skupno
središče, kar pomeni, da to velja tudi za vse tri daljice $O_1A_1$,
$O_2A_2$ in $O_3A_3$ hkrati. \kdokaz


            \bzgled
            Construct a rectangle $ABCD$ if its diagonals and the difference of its sides
             are congruent with the two given line segments $d$ and $l$.
            \ezgled


\begin{figure}[!htb]
\centering
\input{sl.skl.3.4.10a.pic}
\caption{} \label{sl.skl.3.4.10a.pic}
\end{figure}

\textbf{\textit{Solution.}} Naj bo $ABCD$ pravokotnik, pri katerem je $AC\cong d$ in $AB-BC=l$ (Figure \ref{sl.skl.3.4.10a.pic}). Označimo z $E$ takšno točko stranice $AB$, da velja $EB\cong BC$. V tem primeru je $AE=AC-EB=AC-BC=l$. Ker je $EBC$ enakokraki pravokotni trikotnik, je $\angle CEB\cong\angle ECB=45^0$ (izreka \ref{enakokraki} in \ref{VsotKotTrik}) oz. $\angle AEC=135^0$.
To nam  omogoča najprej konstrukcijo  trikotnika $AEC$ ($AC\cong d$, $\angle AEC=135^0$ in $AE\cong l$), nato pa še pravokotnika $ABCD$.
 \kdokaz


        \bzgled
        Construct a triangle with given $b$, $c$ and $t_a$ (sides $AC$, $AB$ and triangle median $AA_1$).
        \ezgled



\begin{figure}[!htb]
\centering
\input{sl.skl.3.4.10b.pic}
\caption{} \label{sl.skl.3.4.10b.pic}
\end{figure}

\textbf{\textit{Solution.}} Naj bo $ABC$ trikotnik, pri katerem je $AC\cong b$, $AB\cong c$ in $AA_1\cong t_a$, kjer je $A_1$ središče daljice $BC$ (Figure \ref{sl.skl.3.4.10b.pic}). Označimo z $D$ takšno točko poltraka $AA_1$, da je $DA_1\cong AA_1$ in $\mathcal{B}(A, A_1,D)$. To pomeni, da je $A_1$ skupno središče daljic $BC$ in $AD$, zato je po izreku \ref{paralelogram} štirikotnik $ABDC$ paralelogram. Po istem izreku je $CD\cong AB\cong c$. To nam omogoča najprej konstrukcijo  trikotnika $ADC$ ($AC\cong b$, $CD\cong c$ in $AD=2t_a$), nato pa še točke $A$.
 \kdokaz

 Sedaj bomo vpeljali krajšo obliko zapisa podatkov za načrtovanje trikotnikov. Podobno, kot smo imeli pri prejšnji nalogi  zapis:  $b$, $c$, $t_a$, bomo za elemente trikotnika $ABC$ običajno uporabljali oznake:
 \begin{itemize}
   \item $a$, $b$, $c$ - stranice,
   \item $\alpha$, $\beta$, $\gamma$ - notranji koti,
   \item $v_a$, $v_b$, $v_c$ - višine,
   \item $t_a$, $t_b$, $t_c$ - težiščnice,
   \item $l_a$, $l_b$, $l_c$ - daljice, ki so določene z ogliščem in s presečiščem simetrale notranjega kota v tem oglišču z nasprotno stranico;
   \item $s$ - polobseg  ($s=\frac{a+b+c}{2}$),
   \item $R$ - polmer očrtane krožnice (glej razdelek \ref{odd3ZnamTock}),
   \item $r$ - polmer včrtane krožnice (glej razdelek \ref{odd3ZnamTock}),
   \item  $r_a$, $r_b$, $r_c$ - polmeri pričrtanih krožnic (glej razdelek \ref{odd4Pricrt}).
 \end{itemize}


        \bzgled
        Construct a trapezium if its sides are congruent with the four given line segments $a$, $b$, $c$ and $d$.
        \ezgled


\begin{figure}[!htb]
\centering
\input{sl.skl.3.4.10c.pic}
\caption{} \label{sl.skl.3.4.10c.pic}
\end{figure}

\textbf{\textit{Solution.}} Brez škode za splošnost predpostavimo najprej, da je $a\geq c$. Naj bo $ABCD$ trapez, v katerem so stranice $AB\cong a$, $BC\cong b$, $CD\cong c$ in $DA\cong d$ (Figure \ref{sl.skl.3.4.10c.pic}). V tem primeru je $AB\geq CD$, zato na stranici $AB$ obstaja takšna točka $E$, da velja $AE\cong CD$. Ker je še $AB\parallel CD$, je po izreku \ref{paralelogram} štirikotnik $AECD$ paralelogram, zato je po istem izreku tudi $CE\cong DA\cong d$. Velja tudi $EB=AB-AE=AB-CD=a-c$. To omogoča konstrukcijo trikotnika $EBC$ ($EB=a-c$, $BC\cong c$ in $CE\cong d$), nato pa še oglišč $A$ in $D$ (iz pogoja $AE\cong CD\cong c$).
 \kdokaz



%________________________________________________________________________________
 \poglavje{Regular Polygons}\label{odd3PravilniVeck}

 Pojem kvadrata se vklaplja v splošno definicijo nove vrste večkotnikov.
 Večkotnik je
\index{pravilni!večkotniki} \pojem{pravilen}, če ima vse
stranice skladne in so vsi notranji koti skladni (Figure
\ref{sl.skl.3.5.1.pic}).

\begin{figure}[!htb]
\centering
\input{sl.skl.3.5.1.pic}
\caption{} \label{sl.skl.3.5.1.pic}
\end{figure}

 Kvadrat je torej pravilni štirikotnik. Prav tako je enakostranični trikotnik
 \index{trikotnik!pravilni}\pojem{pravilni trikotnik}.
To je namreč posledica dejstva, da so pri enakostraničnem
trikotniku tudi vsi koti enaki.

Ugotovili smo že, da je vsota vseh notarnjih kotov poljubnega
$n$-kotnika enaka $(n - 2) \cdot 180^0$ (izrek \ref{VsotKotVeck}). Ker
so pri pravilnem $n$-kotniku vsi notranji koti skladni, lahko notranji kot
izračunamo tako, da vsoto vseh kotov delimo s številom $n$. Tako smo
dokazali naslednjo trditev (Figure \ref{sl.skl.3.5.2.pic}).


             \bizrek \label{pravVeckNotrKot}
             The measure of each interior angle of a regular $n$-gon is:
            $$\frac{(n - 2)\cdot 180^0}{n}.$$
            \eizrek

 Tako notranji kot pravilnega trikotnika meri $60^0$, štirikotnika $90^0$, petkotnika $108^0$, šestkotnika $120^0$, ...


\begin{figure}[!htb]
\centering
\input{sl.skl.3.5.2.pic}
\caption{} \label{sl.skl.3.5.2.pic}
\end{figure}

Dokažimo še dve pomembni lastnosti pravilnih večkotnikov.


        \bizrek \label{sredOcrtaneKrozVeck}
        For each regular polygon, there exists a circle passing through each of its vertices.
        \eizrek

\textbf{\textit{Proof.}} Naj bo $A_1A_2\ldots A_n$ pravilen
$n$-kotnik (Figure \ref{sl.skl.3.5.2.pic}). Potem ima vse
stranice skladne in vsi notranji koti so skladni ter enaki $\frac{(n
- 2)\cdot 180^0}{n}$. Naj bosta $s_1$ in $s_2$ simetrali stranic
$A_1A_2$ in $A_2A_3$ tega večkotnika ter točka $S$ njuno
presečišče. Iz izreka \ref{simetrala} sledi $SA_1 \cong SA_2$
in $SA_2 \cong SA_3$  oz.:
 $$SA_1 \cong SA_2 \cong SA_2 \cong SA_3.$$
 Sledi, da sta enakokraka trikotnika $A_1SA_2$ in $A_2SA_3$
skladna (izrek \ref{SSS} - \textit{SSS}). Potem so skladni tudi koti
$SA_1A_2$, $SA_2A_1$, $SA_2A_3$ in $SA_3A_2$. Iz $\angle SA_2A_1 \cong
\angle SA_2A_3$  sledi, da sta oba kota enaka polovici notranjega
kota tega večkotnika oz. $\frac{\alpha}{2}=\frac{(n-2)\cdot
180^0}{2n}$. Zato je tudi:
 $$\angle SA_3A_4 =\alpha - \frac{\alpha}{2}=\frac{\alpha}{2} = \angle
 SA_3A_2.$$
Torej sta trikotnika $A_2SA_3$ in $A_3SA_4$ skladna (izrek
\ref{SKS} - \textit{SAS}). Zaradi tega je   $SA_3 \cong SA_4$  oz.:
 $$SA_1 \cong SA_2 \cong SA_2 \cong SA_3\cong SA_4.$$
 Če ta postopek nadaljujemo, dobimo:
 $$SA_1 \cong SA_2 \cong SA_2 \cdots \cong SA_n,$$
kar pomeni, da je točka $S$ središče krožnice $k(S, SA_1)$, ki
vsebuje vsa njegova oglišča.
 \kdokaz

Krožnico iz prejšnjega izreka imenujemo \index{očrtana
krožnica!pravilnega večkotnika} \pojem{očrtana krožnica
pravilnega večkotnika}. Iz dokaza prejšnjega izreka je jasno, da
se njeno središče nahaja v presečišču simetral vseh njegovih
stranic.


Analogno dokazujemo tudi naslednji izrek.


        \bizrek \label{sredVcrtaneKrozVeck}
        For each regular polygon, there exists a circle  touching each of its sides.
         \eizrek


\begin{figure}[!htb]
\centering
\input{sl.skl.3.5.3.pic}
\caption{} \label{sl.skl.3.5.3.pic}
\end{figure}

\textbf{\textit{Proof.}} Naj bo $A_1A_2\ldots A_n$ pravilen $n$-kotnik
 (Figure \ref{sl.skl.3.5.3.pic}).
Točko $S$ definirajmo enako kot v dokazu prejšnjega izreka.
Dokazali smo, da velja:  $SA_1 \cong SA_2 \cong SA_2 \cdots \cong
SA_n$. Iz tega po izreku \ref{SSS} - \textit{SSS} sledi skladnost
enakokrakih trikotnikov:
 $$\triangle A_1SA_2 \cong \triangle A_2 SA_3 \cong \cdots \cong
  \triangle A_{n-1}SA_n \cong \triangle A_nSA_1.$$
Zaradi tega so skladni tudi vsi koti ob osnovnicah teh trikotnikov.
Torej so premice $SA_1$, $SA_2$,..., $SA_n$ simetrale notranjih
kotov večkotnika $A_1A_2\ldots A_n$. Naj bodo $P_1$, $P_2$,...,
$P_n$ nožišča višin iz oglišča $S$ omenjenih enakokrakih trikotnikov.
Iz skladnosti trikotnikov $\triangle A_1SP_1$, $\triangle A_2SP_1$,
$\triangle A_2SP_2$, ..., $\triangle A_1SP_n$ (izreka \ref{SSK} in
\ref{KSK}) sledi skladnost daljic $SP_1$, $SP_2$,..., $SP_n$. Po
izreku \ref{TangPogoj} se krožnica $k(S, SP_1)$ dotika vseh stranic
večkotnika $A_1A_2\ldots A_n$.
 \kdokaz

Krožnico iz prejšnjega izreka imenujemo \index{včrtana krožnica!pravilnega večkotnika} \pojem{včrtana krožnica pravilnega
večkotnika}. Iz dokaza tega izreka je jasno, da se središče včrtane
krožnice pravilnega večkotnika nahaja na presečišču simetral vseh
njegovih notranjih kotov. Iz dokaza je očitno tudi, da so točke, v
katerih se ta krožnica dotika stranic pravilnega večkotnika, hkrati
središča teh stranic. Središče očrtane in včrtane krožnice je ista
točka in jo zato imenujemo tudi \index{središče!pravilnega
večkotnika}\pojem{središče pravilnega večkotnika}.

 Videli smo tudi, da so vsi trikotniki,
 določeni s središčem
pravilnega $n$-kotnika in z njegovimi stranicami, enakokraki in vsi
skladni. Polmera očrtane in včrtane krožnice $n$-kotnika sta enaka
kraku oz. višini vsakega od teh trikotnikov. Koti ob vrhu teh
trikotnikov so tudi skladni in ker jih je skupaj $n$ (enako kot
stranic $n$-kotnika), vsak od njih meri (Figure \ref{sl.skl.3.5.4.pic}):
 $$\varphi = \frac{360^0}{n}.$$

\begin{figure}[!htb]
\centering
\input{sl.skl.3.5.4.pic}
\caption{} \label{sl.skl.3.5.4.pic}
\end{figure}

 Pri pravilnem šestkotniku, oz. za $n = 6$, velja:
 $$\varphi = \frac{360^0}{6}=60^0.$$

 To pomeni, da so omenjeni trikotniki pravilni.
Torej je pravilni šestkotnik sestavljen iz šestih
pravilnih trikotnikov  (Figure \ref{sl.skl.3.5.4.pic}).

V nadaljevanju bomo obravnavali lastnosti pravilnih $n$-kotnikov.

 Naj bo najprej $n$ sodo število in $k = \frac{n}{2}+1$.
Pravimo, da je $A_k$ \pojem{nasprotno oglišče} oglišča $A_1$
pravilnega $n$-kotnika $A_1A_2\ldots A_n$ (Figure
\ref{sl.skl.3.5.5.pic}). Analogno sta $A_2$ in $A_{k+1}$, $A_3$ in
$A_{k+2}$, ... , $A_{k-1}$ in $A_n$ nasprotni oglišči tega
$n$-kotnika. Podobno sta stranici $A_1A_2$ in $A_kA_{k+1}$, ... ,
$A_{k-1} A_k$ in $A_nA_1$ \pojem{nasprotni stranici} večkotnika
$A_1A_2\ldots A_n$. Opazimo, da velja:
 $$\angle A_1SA_k=\frac{n}{2}\varphi = \frac{n}{2}\cdot
  \frac{360^0}{n}=180^0,$$
kar pomeni, da diagonala $A_1A_k$ tega $n$-kotnika vsebuje njegovo
središče. Zato ta diagonala predstavlja premer očrtane krožnice.
Analogno to velja za vse diagonale, ki so določene z nasprotnimi
oglišči. Zaradi tega takšne diagonale imenujemo \index{velika diagonala
pravilnega $n$-kotnika} \pojem{velike diagonale} pravilnega
$n$-kotnika. Polmer očrtane krožnice je
 enak polovici velike diagonale.

\begin{figure}[!htb]
\centering
\input{sl.skl.3.5.5.pic}
\caption{} \label{sl.skl.3.5.5.pic}
\end{figure}

 Na podoben način se dokaže, da za sodo število $n$ središči
 nasprotnih stranic pravilnega $n$-kotnika
$A_1A_2\ldots A_n$ določata premere včrtane krožnice tega
$n$-kotnika. Če upoštevamo prejšnje oznake, dobimo:
 \begin{eqnarray*}
 \angle P_1SP_k&=&\angle P_1SA_2 + \angle A_2SA_{k-1} + \angle
 A_{k-1}SP_k\\
 &=&  \frac{\varphi}{2}+\frac{n-2}{2}\cdot \varphi+\frac{\varphi}{2}
 =\frac{n}{2}\cdot\varphi
 =180^0.
  \end{eqnarray*}


 Daljice, ki so določene s parom središč nasprotnih stranic $n$-kotnika
 $A_1A_2\ldots A_n$ oz. daljice $P_1P_k$,
$P_2P_{k+1}$, ... , $P_{k-1}P_n$, imenujemo \pojem{višine}
\index{višina!pravilnega $n$-kotnika} tega $n$-kotnika. Polmer
včrtane krožnice je enak polovici višine.

 Torej vsak pravilni $n$-kotnik, kjer je $n$ sodo število, vsebuje
$\frac{n}{2}$ velikih diagonal (enakih premeru očrtane krožnice) in
$\frac{n}{2}$ višin (enakih premeru včrtane krožnice). Vsaka od njih
gre skozi središče tega $n$-kotnika.


 Naj bo sedaj $n$ liho število (Figure \ref{sl.skl.3.5.6.pic}) in
  $k=\frac{n+1}{2}+1$.
 Potem je:
\begin{eqnarray*}
 \angle P_1SA_k=\angle P_1SA_2 + \angle A_2SA_k=
  \frac{\varphi}{2}+\frac{n-1}{2}\cdot \varphi
 =\frac{n}{2}\cdot\varphi
 =180^0.
  \end{eqnarray*}

\begin{figure}[!htb]
\centering
\input{sl.skl.3.5.6.pic}
\caption{} \label{sl.skl.3.5.6.pic}
\end{figure}

To pomeni, da daljica $P_1A_k$ vsebuje središče $S$ pravilnega
$n$-kotnika  $A_1A_2\ldots A_n$. To daljico imenujemo \index{višina!pravilnega $n$-kotnika}
\pojem{višina} tega $n$-kotnika, stranica
$A_1A_2$ in oglišče $A_k$ sta si \pojem{nasprotni}. Analogno
definiramo tudi preostalih $n$ višin in $n$ parov nasprotnih
stranic in oglišč. Na podoben način lahko dokažemo, da tudi ostale
višine tega $n$-kotnika  vsebujejo njegovo središče.

Pri pravilnem (enakostraničnem) trikotniku  imamo torej tri
višine, ki se sekajo v njegovem središču (Figure
\ref{sl.skl.3.5.7.pic}). Če to velja pri
poljubnem trikotniku, bomo ugotovili kasneje


\begin{figure}[!htb]
\centering
\input{sl.skl.3.5.7.pic}
\caption{} \label{sl.skl.3.5.7.pic}
\end{figure}

 V kvadratu sta njegovi  diagonali hkrati veliki
 diagonali in se sekata v
njegovem središču  (Figure \ref{sl.skl.3.5.7.pic}). Višini
kvadrata sta skladni z njegovo stranico, kar ni težko dokazati.

Omenili smo že, da je pravilni šestkotnik sestavljen iz šestih
trikotnikov, ki se stikajo v njegovem središču. Pravilni
šestkotnik bomo obširneje obravnavali v nadaljevanju. Dokazali bomo
tudi nekaj lastnosti pravilnega petkotnika, sedemkotnika,
devetkotnika in dvanajstkotnika. Posebej zanimiv bo problem
načrtovanja pravilnih $n$-kotnikov za poljubno število $n$.

Dokažimo sedaj zanimivo lastnost pravilnega devetkotnika.

             \bzgled
            If $a$ is a side and $d$ and $e$ are the shortest and longest
            diagonal of a regular nonagon ($9$-gon), then $e - d = a$.
             \ezgled


\begin{figure}[!htb]
\centering
\input{sl.skl.3.5.8.pic}
\caption{} \label{sl.skl.3.5.8.pic}
\end{figure}

\textbf{\textit{Proof.}} Naj bosta $d = CE$ in $e = BF$ najkrajša in
najdaljša diagonala pravilnega devetkotnika $ABCDEFGHI$ s stranico
$a$ ter $P$ presečišče premic $BC$ in $FE$  (Figure
\ref{sl.skl.3.5.8.pic}). Notranji kot tega devetkotnika meri
 $\angle CDE=\frac{9-2}{9}\cdot 180^0=140^0$,
zato je $\angle ECD = \angle CED = 20^0$. Iz tega sledi
 $\angle BCE = \angle FEC = 120^0$, oz.:
 $$\angle ECP = \angle CEP = 60^0.$$
 Torej je trikotnik $CPE$ pravilen. Ker je $CB = EF=a$
in $\angle BPF \cong \angle CPE = 60^0$, je pravilen tudi
trikotnik $BPF$. Torej:
 $$e = BF = BP = BC + CP = BC + CE = a + d,$$ kar je bilo treba dokazati. \kdokaz


%%________________________________________________________________________________
 \poglavje{Midsegment of Triangle} \label{odd3SrednTrik}

Sedaj bomo obravnavali zelo pomembno lastnost trikotnika, ki jo
bomo pogosto uporabljali. Naj bosta $P$ in $Q$ središči stranic
$AB$ in $AC$ trikotnika $ABC$. Daljico $PQ$ imenujemo
\index{srednjica!trikotnika} \pojem{srednjica trikotnika} $ABC$,
ki je pripadajoča stranici $BC$ (Figure \ref{sl.skl.3.6.1.pic}).
Pravimo tudi, da je $PQ$ srednjica trikotnika $ABC$ za osnovnico
$BC$. Dokažimo osnovno lastnost, ki se nanaša na srednjico.

\begin{figure}[!htb]
\centering
\input{sl.skl.3.6.1.pic}
\caption{} \label{sl.skl.3.6.1.pic}
\end{figure}


           \bizrek \label{srednjicaTrik}
             Let $PQ$ be the midsegment of a triangle $ABC$ corresponding to the side $BC$. Then:
            $$ PQ = \frac{1}{2} BC\hspace*{2mm}
            \textrm{ in } \hspace*{2mm} PQ \parallel BC.$$
           \eizrek



 \textbf{\textit{Proof.}} Naj bo $R$ takšna točka, da velja $PQ \cong QR$ in
$\mathcal{B}(P,Q,R)$ (Figure \ref{sl.skl.3.6.1.pic}). Daljici $AC$
in $PR$ imata skupno središče, zato je štirikotnik $APCR$
paralelogram (izrek \ref{paralelogram}). Zaradi tega sta daljici
$AP$ in $RC$ skladni in vzporedni. Točka $P$ je središče daljice
$AB$, zato sta tudi daljici $PB$ in $RC$ skladni in vzporedni. To
pomeni, da je tudi štirikotnik $PBCR$ paralelogram. Iz tega
sledi, da sta daljici $BC$ in $PR$ skladni in vzporedni. Končna
ugotovitev sledi iz dejstva, da je točka $Q$ središče daljice
$PR$.
 \kdokaz


            \bzgled
            Let $AB$ and $A'B'$ be congruent line segments, $C$ and $D$ the midpoints of the line segments
            $AA'$ and $BB'$. Suppose that $CD =\frac{1}{2}  AB$.
            What is a measure of the angle between the lines $AB$ and $A'B'$?
             \ezgled


\begin{figure}[!htb]
\centering
\input{sl.skl.3.6.2.pic}
\caption{} \label{sl.skl.3.6.2.pic}
\end{figure}



\textbf{\textit{Solution.}} Naj bo točka $S$ središče daljice
$A'B$ (Figure \ref{sl.skl.3.6.2.pic}). Daljici $CS$ in $DS$ sta
srednjici trikotnikov $A'AB$ in $BA'B'$, zato je: $$CS =
\frac{1}{2}AB = CD = \frac{1}{2}A'B'= DS,$$
 oziroma $SCD$ je pravilen trikotnik. Kota $\angle AB,A'B'$ in $\angle CSD$
  imata vzporedne krake. Torej je:
$\angle AB, A'B' \cong \angle CSD = 60^0$.
 \kdokaz

Naslednja posledica izreka \ref{srednjicaTrik} se nanaša na
trapez.


             \bizrek \label{srednjTrapez}
             Let $P$ and $Q$ be the midpoints of legs $BC$ and $DA$  of a trapezium $ABCD$.
            Suppose that $M$ and $N$ are the midpoints of the diagonals $AC$ and $BD$ of that trapezium.
            Then the points $M$ and $N$ lie on the line $PQ$, which is parallel to the bases $AB$ in $CD$ of the trapezium,
             and also:
             $$PQ = \frac{1}{2}( AB + CD)     \hspace*{2mm}
             \textrm{ in } \hspace*{2mm} MN=\frac{1}{2}( AB - CD).$$
            \eizrek

\begin{figure}[!htb]
\centering
\input{sl.skl.3.6.3.pic}
\caption{} \label{sl.skl.3.6.3.pic}
\end{figure}


 \textbf{\textit{Proof.}}  Daljice $PN$, $NQ$ in $PM$ so po vrsti srednjice
  trikotnikov
 $DAC$, $ACB$ in $ADB$ za
pripadajoče osnovnice $DC$, $AB$ in $AB$ (Figure
\ref{sl.skl.3.6.3.pic}). Zaradi tega so vse tri premice $PN$, $NQ$
in $PM$ vzporedne z osnovnicama $CD$ in $AB$. Ker skozi vsako
točko (najprej $N$, nato $P$) obstaja le ena vzporednica s premico
$AB$ (Playfairjev\footnote{\index{Playfair, J.}\textit{J.
Playfair} (1748--1819), škotski matematik.} aksiom
\ref{Playfair}), so točke $P$, $N$, $M$ in $Q$ kolinearne. Velja
še (izrek \ref{srednjicaTrik}):
 $PN = \frac{1}{2} CD$ in
  $NQ = PM = \frac{1}{2} AB$. Iz tega sledi:
   \begin{eqnarray*}
   PQ&=& PN+NQ=\frac{1}{2}CD+
   \frac{1}{2}AB=\frac{1}{2}\left(AB+CD\right)\\
  NM&=& PM-PN=\frac{1}{2}AB-
   \frac{1}{2}CD=\frac{1}{2}\left(AB-CD\right),
  \end{eqnarray*}
  kar je bilo potrebno dokazati.  \kdokaz

Daljico $PQ$ iz prejšnjega izreka imenujemo
\index{srednjica!trapeza} \pojem{srednjica trapeza}.

 Naslednje trditve se nanašajo na poljubni štirikotnik.


             \bizrek \label{Varignon}
             Let $ABCD$ be an arbitrary quadrilateral and $P$, $Q$, $K$ and $L$
            the midpoints of the sides $AB$, $CD$, $BC$ and $AD$, respectively. Then the quadrilateral $PKQL$ is
            a parallelogram (so-called \index{paralelogram!Varignonov}
              Varignon\footnote{\index{Varignon, P.}
              \textit{P. Varignon} (1654--1722),
             francoski matematik,
            ki je prvi dokazal to lastnost. Vendar je bil izrek  objavljen šele
            po njegovi smrti leta 1731. Glede na enostavnost pa je  prav
            presenetljivo, da je ta trditev toliko časa ‘‘čakala’’ na svoje
            odkritje.} parallelogram).
            \eizrek

\begin{figure}[!htb]
\centering
\input{sl.skl.3.6.4.pic}
\caption{} \label{sl.skl.3.6.4.pic}
\end{figure}


 \textbf{\textit{Proof.}} (Figure \ref{sl.skl.3.6.4.pic})
Daljici $PK$ in $LQ$ sta srednjici trikotnikov $ABC$ in $ADC$ za
isto osnovnico $AC$, zato sta skladni in
vzporedni. Torej je štirikotnik $PKQL$  paralelogram.
 \kdokaz

V posebnem primeru je lahko Varignonov paralelogram celo pravokotnik,
romb ali kvadrat. Kdaj je to možno? To vprašanje nam da idejo za
naslednji izrek. Vemo, da je paralelogram pravokotnik, če  ima
vsaj en notranji  kot pravi. Drugi (ekkvivalenten) pogoj  pa je,
da ima skladni diagonali. Podobno obravnavo lahko uporabimo tudi
pri rombu in kvadratu.


            \bzgled \label{VarignonPoslPravRomb}
            Let $ABCD$ be an arbitrary quadrilateral and $P$, $K$, $Q$ and $L$
            the midpoints of the sides $AB$, $BC$, $CD$ and
            $DA$, respectively. Then:

            a) $AC \perp BD \Leftrightarrow PQ \cong KL$;

             b) $AC \cong BD \Leftrightarrow PQ \perp KL$.
             \ezgled


\begin{figure}[!htb]
\centering
\input{sl.skl.3.6.5.pic}
\caption{} \label{sl.skl.3.6.5.pic}
\end{figure}


\textbf{\textit{Proof.}} (Figure \ref{sl.skl.3.6.5.pic})
 Iz prejšnjega izreka \ref{Varignon} sledi, da je
 štirikotnik $PKQL$ vedno paralelogram – Varignonov
paralelogram. Daljici $PL$ in $PK$ sta srednjici trikotnikov $ABD$
in $ABC$ za osnovnici $AD$ oz. $AC$. Zaradi tega velja:
 $PL= \frac{1}{2}BD$ in $PL \parallel BD$ ter $PK= \frac{1}{2}AC$ in $PK \parallel AC$.
 Zatorej velja:

 a) $AC \perp BD \Leftrightarrow PL \perp PK
\Leftrightarrow PKQL \textrm{ pravokotnik} \Leftrightarrow PQ
\cong KL$;

 b) $AC \cong BD \Leftrightarrow PL \cong PK \Leftrightarrow
  PKQL \textrm{ romb } \Leftrightarrow PQ \perp KL$.
 \kdokaz

Če je Varignonov paralelogram kvadrat, so izpolnjeni vsi štirje
pogoji iz prejšnjih ekvivalenc, oz. je v tem primeru: $AB \perp CD$, $AB \cong CD$, $PQ \perp KL$ in $PQ \cong KL$.

Oglejmo si še eno uporabo lastnosti Varignonovega paralelograma.


            \bzgled \label{VagnanPosl}
            Let $ABCD$ be an arbitrary quadrilateral. If $P$, $Q$, $K$, $L$, $M$ and
             $N$ are the midpoints of the line segments $AB$, $CD$, $BC$, $AD$, $AC$ and $BD$,  respectively,
             then the line segments $PQ$, $KL$ and
            $MN$  have a common midpoint.
             \ezgled


\begin{figure}[!htb]
\centering
\input{sl.skl.3.6.6.pic}
\caption{} \label{sl.skl.3.6.6.pic}
\end{figure}


\textbf{\textit{Proof.}}
 (Figure \ref{sl.skl.3.6.6.pic}) Štirikotnik $PKQL$ je Varignonov
paralelogram (izrek \ref{Varignon}). Na podoben način dokazujemo, da
je tudi štirikotnik $LNKM$ paralelogram (srednjice trikotnikov $ADB$
in $ACB$). Paralelograma $PKQL$ in $LPKQ$ imata skupno diagonalo
$LK$. Ker se diagonali poljubnega paralelograma razpolavljata (izrek
\ref{paralelogram}), imajo daljice $LK$, $PQ$ in $MN$  skupno
središče.
 \kdokaz

Točko iz prejšnjega primera, v kateri se daljice sekajo, imenujemo
\index{težišče!štirikotnika} \pojem{težišče štirikotnika}. Več
o tem bomo povedali v razdelku \ref{odd5TezVeck}.

Naslednja trditev je lep primer kombiniranja neenakosti trikotnika
in srednjice trikotnika. Trditev je pravzaprav posplošitev izreka
o srednjici trapeza \ref{srednjTrapez}.


              \bizrek
             If $P$ and $Q$ are the midpoints of the sides $AB$ and $CD$ of
             an arbitrary quadrilateral
            $ABCD$, then: $$PQ \leq \frac{1}{2}\left( BC + AD\right).$$
             \eizrek


\begin{figure}[!htb]
\centering
\input{sl.skl.3.6.7.pic}
\caption{} \label{sl.skl.3.6.7.pic}
\end{figure}


\textbf{\textit{Proof.}}
 Naj bo $S$ središče diagonale $AC$ štirikotnika $ABCD$
  (Figure \ref{sl.skl.3.6.7.pic}).
Če uporabimo izrek o srednjici trikotnika (\ref{srednjicaTrik}) in
trikotniško neenakost  (\ref{neenaktrik}), dobimo:
 $$BC + AD = 2PS
+ 2SQ = 2(PS + SQ) \geq 2PQ.$$
 Enakost velja v primeru, kadar so
točke $P$, $S$ in $Q$ kolinearne, oz. ko je štirikotnik $ABCD$
trapez z osnovnico $BC$.
 \kdokaz

 Omenimo še, da
neenakost iz prejšnjega primera velja tudi, če točke $A$, $B$,
$C$ in $D$ niso v isti ravnini, oz. če je $ABCD$
\index{tetraeder} \pojem{tetraeder}.


        \bzgled \label{TezisceSredisceZgled}
        Let $P$ be the midpoint of the median $AA_1$ of a triangle $ABC$ and
        $Q$
        the intersection of the side $AC$ and the line $BP$. Determine the ratios
        $AQ :QC$ and $BP : PQ$.
        \ezgled


\begin{figure}[!htb]
\centering
\input{sl.skl.3.6.8.pic}
\caption{} \label{sl.skl.3.6.8.pic}
\end{figure}


\textbf{\textit{Solution.}} Naj bo $R$ središče daljice $QC$
  (Figure \ref{sl.skl.3.6.8.pic}). Daljica $A_1R$ je srednjica
trikotnika $BCQ$ za osnovnico $BQ$, zato je (izrek
\ref{srednjicaTrik}) $BQ = 2A_1R$ in $BQ\parallel A_1R$. Iz te
vzporednosti in definicije točke $P$ sledi, da je $PQ$ srednjica
trikotnika $AA_1R$ za osnovnico $A_1R$, zato je (izrek
\ref{srednjicaTrik} in Playfairjev aksiom \ref{Playfair}) točka $Q$
središče daljice $AR$ in velja $A_1R = 2PQ$. Torej: $AQ \cong QR
\cong RC$ oz. $AQ:QC=1:2$. Na koncu je še $BQ=2A_1R=4PQ$ oz.
$BP:PQ=3:1$.
 \kdokaz



                 \bnaloga\footnote{19. IMO Yugoslavia - 1977, Problem 1.}
                Equilateral triangles $ABP$, $BCL$, $CDM$, $DAN$ are constructed inside the
                square $ABCD$. Prove that the midpoints of the segments $LM$, $MN$, $NP$, $PL$, $AN$, $LB$,
                  $BP$, $CM$, $CL$, $DN$, $DM$ in $AP$
                are the twelve vertices of a regular dodecagon ($12$-gon).
                \enaloga

\begin{figure}[!htb]
\centering
\input{sl.skl.3.6.IMO1.pic}
\caption{} \label{sl.skl.3.6.IMO1.pic}
\end{figure}

\textbf{\textit{Solution.}} Označimo z $a$ dolžino stranice in
z $O$ središče kvadrata $ABCD$ (Figure \ref{sl.skl.3.6.IMO1.pic}).

Dokažimo najprej, da je štirikotnik $MNPL$ tudi kvadrat z istim
središčem $O$. Ker so $ABP$, $BCL$, $CDM$ in $DAN$ vsi pravilni
trikotniki, ležita diagonali $MP$ in $LN$ štirikotnika $MNPL$ na
simetralah stranic $AB$ in $BC$ kvadrata $ABCD$. Iz tega sledi $MP \perp LN$.
Ker je še $d(M,AB)=d(P,CD)=a-v$ (kjer je $v$ dolžina
višine omenjenih pravilnih trikotnikov), velja tudi $OM\cong OP$.
Podobno je tudi $OL\cong ON$, kar pomeni, da je štirikotnik $MNPL$
res kvadrat z istim središčem~$O$.

Dokažimo sedaj, da je $LAM$ pravilni trikotnik. Ker je $AB\cong
AD\cong BL\cong DM=a$ in $\angle LBA\cong\angle MDA
=90^0-60^0=30^0$, sta trikotnika $LBA$ in $MDA$ skladna (izrek
\textit{SAS} \ref{SKS}). To pomeni, da je $LA\cong MA$ in $\angle
DAL= 90^0-\angle LAB=15^0$. Podobno je tudi $\angle BAM=15^0$ oz.
$\angle LAM = 90^0-2\cdot 15^0=60^0$. Torej je $LAM$ pravilni
trikotnik, zato ima stranica kvadrata $MNPL$  dolžino
$b=|LM|=|LA|$.

Označimo s $S$ središče daljice $LM$. Središča stranic kvadrata $MNPL$
ležijo na krožnici $k(O,\frac{b}{2})$, ki je včrtana krožnica tega
kvadrata. Dokažimo, da tudi točka $T$ - središče daljice $AN$ -
leži na tej krožnici. Daljica $OT$ je srednjica trikotnika $LAN$
za osnovnico $LA$, zato je $OT\parallel LA$ in
$|OT|=\frac{1}{2}|LA|=\frac{b}{2}$. Torej točka $T$ in analogno
tudi vse točke v nalogi definiranega $12$-kotnika ležijo na
krožnici $k(O,\frac{b}{2})$.

Dokažimo še, da je omenjeni $12$-kotnik pravilen. Brez škode za
splošnost zadošča dokazati, da je $\angle
SOT=\frac{360^0}{12}=30^0$. Toda iz že dokazanega dejstva
$OT\parallel LA$ sledi $\angle SOT\cong \angle
LAS=\frac{1}{2}\angle LAM=30^0$.
 \kdokaz

%________________________________________________________________________________
 \poglavje{Triangle Centers} \label{odd3ZnamTock}

 Našo raziskavo bomo nadaljevali s trikotnikom – najbolj enostavnim
 večkotnikom,
 ki pa je  hkrati
lik, ki ima nepričakovano veliko zanimivih lastnosti. Nekatere  bomo
obravnavali v tem razdelku, nekatere pa kasneje, ko se bomo ukvarjali
z drugimi pojmi, kot so izometrije in podobnost.

 Sedaj bomo obravnavali štiri \index{značilne točke
 trikotnika}\pojem{značilne točke
 trikotnika}\footnote{Te štiri točke omenjajo že Stari Grki, čeprav so
 (še posebej to velja za težišče) bile znane verjetno že veliko časa pred
tem.} in
 njihovo uporabo pri štiri- in večkotnikih.

Začnimo najprej s prvo med značilnimi točkami, ki je povezana s
 težiščnicami trikotnika. Težiščnico smo  že definirali
  v poglavju \ref{odd3NeenTrik}.



         \bizrek \label{tezisce}
         The medians of a triangle intersect at one point.
        That point divides  the medians in the ratio $2:1$
         (from the vertex to the midpoint of the opposite side).
        \eizrek

\begin{figure}[!htb]
\centering
\input{sl.skl.3.7.1.pic}
\caption{} \label{sl.skl.3.7.1.pic}
\end{figure}


\textbf{\textit{Proof.}}
  Naj bodo $AA_1$, $BB_1$ in $CC_1$ težiščnice trikotnika $ABC$
   (Figure \ref{sl.skl.3.7.1.pic}).
Zaradi Paschevega aksioma, \ref{AksPascheva} glede na trikotnik
$BCB_1$ in premico $AA_1$, premica $AA_1$ seka daljico $BB_1$.
Analogno premica $BB_1$ seka daljico $AA_1$, kar pomeni, da se
težiščnici $AA_1$ in $BB_1$ sekata v neki točki $T$. Naj bosta $A_2$
in $B_2$ središči daljic $AT$ in $BT$. Daljici $A_1B_1$ in $A_2B_2$
sta srednjici trikotnikov $ABC$ in $ABT$ z isto osnovnico $AB$. To
pomeni, da sta daljici $A_1B_1$ in $A_2B_2$ vzporedni in enaki
polovici stranice $AB$. Zaradi tega je štirikotnik $B_2A_1B_1A_2$
paralelogram (izrek \ref{paralelogram}), kar pomeni da se njegovi
diagonali $A_1A_2$ in $B_1B_2$ razpolavljata in je točka $T$
njuno skupno središče. Torej velja:
 $A_1T\cong TA_2 \cong A_2A$  in $B_1T\cong TB_2 \cong B_2B$
 oz. $AT:TA_1=2:1$ in $BT:TB_1=2:1$ ter
  $$A_1T=\frac{1}{3}A_1A \textrm{ in }    B_1T = \frac{1}{3}B_1B.$$
  Na enak način
 dokažemo, da se tudi težiščnici $AA_1$ in $CC_1$ sekata v neki
točki $T’$, za katero velja $AT':T'A_1=2:1$ in $CT':T'C_1=2:1$ oz.:
 $$A_1T'=\frac{1}{3}A_1A \textrm{ in }    C_1T' = \frac{1}{3}C_1C.$$
To pomeni, da sta $T$ in $T'$ točki poltraka $A_1A$, za kateri velja
$A_1T\cong A_1T'=\frac{1}{3}A_1A$, zato je po izreku
\ref{ABnaPoltrakCX} $T = T'$, kar pomeni, da se težiščnice $AA_1$,
$BB_1$ in $CC_1$ sekajo v točki $T$ in velja
$AT:TA_1=BT:TB_1=CT:TC_1=2:1$.
 \kdokaz


Točka iz prejšnjega izreka, v kateri se sekajo vse težiščnice
trikotnika, se imenuje \index{težišče!trikotnika}\pojem{težišče
trikotnika}.

Težišče trikotnika v fizičnem smislu predstavlja točko, ki je
središče mase tega trikotnika. To bo še bolj jasno, ko bomo
v razdelku \ref{odd8PloTrik} dokazali dejstvo, da težišče
 deli  trikotnik na trikotnike z enako ploščino.
  V naslednjem poglavju \ref{pogVEKT} (razdelek \ref{odd5TezVeck})
bomo obravnavali težišče poljubnega večkotnika.

V  razdelku \ref{odd3PravilniVeck} smo ugotovili, da za vsak
pravilni večkotnik obstajata očrtana krožnica (ki vsebuje vsa njegova
oglišča) in včrtana krožnica (ki se dotika vseh njegovih stranic) z
istim središčem. Ta lastnost se prenaša tudi na pravilne oz.
enakostranične trikotnike. Kako pa je s poljubnim trikotnikom?
Dokazali bomo, da obstajata omenjeni krožnici za poljuben trikotnik, le
da imata v splošnem primeru  različni središči.


        \bizrek \label{SredOcrtaneKrozn}
       The perpendicular bisectors of the sides of any triangle intersect at a single point,
             which is the centre of a circle containing all its vertices.
        \eizrek

\begin{figure}[!htb]
\centering
\input{sl.skl.3.7.2.pic}
\caption{} \label{sl.skl.3.7.2.pic}
\end{figure}


\textbf{\textit{Proof.}}
  Naj bodo $p$, $q$ in $r$  simetrale stranic $BC$, $AC$ in $AB$
   trikotnika $ABC$ (Figure \ref{sl.skl.3.7.2.pic}).
 Simetrali $p$
in $q$ nista vzporedni (ker bi bili v tem primeru po Playfairjevem aksiomu
\ref{Playfair}  vzporedni tudi premici $BC$ in $AC$) in se
sekata v neki točki $O$. Ker le-ta leži na simetralah $p$ in $q$
stranic $BC$ in $AC$,  je $OB \cong OC$ in $OC \cong OA$. Iz tega
sledi  $OA \cong OB$, kar pomeni, da točka $O$ leži tudi na
simetrali $r$ daljice $AB$. Torej se simetrale $p$, $q$ in $r$
sekajo v eni točki.

Ker je  $OA \cong OB \cong OC$, je točka $O$ središče krožnice
$k(O,OA)$, ki vsebuje vsa oglišča trikotnika $ABC$.
 \kdokaz

 Krožnico iz prejšnjega izreka, ki vsebuje vsa oglišča trikotnika,
  imenujemo  \index{očrtana krožnica!trikotnika} \pojem{trikotniku očrtana
 krožnica}, njeno središče pa
        \index{središče!očcrtane krožnice!trikotnika}
        \pojem{središče trikotniku očrtane krožnice}.



         \bizrek \label{SredVcrtaneKrozn}
          The bisectors of the interior angles of any triangle intersect at a single point,
             which is the centre of a circle touching all its sides.
        \eizrek

\begin{figure}[!htb]
\centering
\input{sl.skl.3.7.3.pic}
\caption{} \label{sl.skl.3.7.3.pic}
\end{figure}


\textbf{\textit{Proof.}}
  Naj bodo $p$, $q$ in $r$  simetrale notranjih kotov pri
  ogliščih  $A$, $B$ in $C$
   trikotnika $ABC$ (Figure \ref{sl.skl.3.7.3.pic}).
   Dokažimo, da simetrali $p$ in $q$ nista vzporedni. V nasprotnem
    bi bila po izreku \ref{KotiTransverzala} vsota polovic
   notranjih kotov pri ogliščih $A$ in $B$
 enaka $180°$, kar po izreku \ref{VsotKotTrik} ni mogoče. Torej se
$p$ in $q$  sekata v neki točki $S$. Ker točka $S$ leži na
simetralah notranjih kotov pri ogliščih $A$ in $B$, je  enako
oddaljena od nosilk stranic $AC$ in $AB$ oz. od nosilk stranic $BA$
in $BC$ (izrek \ref{SimKotaKraka}). Iz tega sledi, da je točka $S$
enako oddaljena od nosilk stranic $CA$ in $CB$, kar pomeni, da leži
na simetrali $r$ notranjega kota pri oglišču $C$. Torej se simetrale
$p$, $q$ in $r$ sekajo v točki $S$.

S $P$, $Q$ in $R$ označimo pravokotne projekcije točke $S$ na
stranicah $BC$, $CA$ in $AB$. Ker velja $\frac{1}{2}\angle CBA<90^0$
in $\frac{1}{2}\angle BCA<90^0$, je $\mathcal{B}(B,P,C)$. Podobno je
tudi  $\mathcal{B}(C,Q,A)$ in $\mathcal{B}(A,R,B)$. Zaradi že
dokazane lastnosti točke $S$ velja $SP \cong SQ \cong SR$. Torej je
točka $S$  središče krožnice $l$, ki poteka skozi točke $P$, $Q$ in
$R$. Zaradi pravokotnosti polmerov $SP$, $SQ$ in $SR$ na ustrezne
stranice,  so le-te tangente krožnice $l$. Ker je
$\mathcal{B}(B,P,C)$,  $\mathcal{B}(C,Q,A)$ in $\mathcal{B}(A,R,B)$
se krožnica $l$ dotika vseh
         stranic trikotnika $ABC$.
 \kdokaz


 Krožnico iz prejšnjega izreka, ki se dotika vseh
         stranic trikotnika, imenujemo  \index{včrtana krožnica!trikotnika} \pojem{trikotniku včrtana
 krožnica}, njeno središče pa
        \index{središče!včcrtane krožnice!trikotnika}
        \pojem{središče trikotniku včrtane krožnice}.

Ostala je še ena od štirih omenjenih značilnih točk trikotnika.
Nanaša se na višine trikotnika.


        \bizrek \label{VisinskaTocka}
        The lines containing the altitudes of a triangle  intersect at a single point.
        \eizrek

\begin{figure}[!htb]
\centering
\input{sl.skl.3.7.4.pic}
\caption{} \label{sl.skl.3.7.4.pic}
\end{figure}


\textbf{\textit{Proof.}}
  Naj bodo $p$, $q$ in $r$  nosilke višin
    $AA'$, $BB'$ in $CC'$
   trikotnika $ABC$ (Figure \ref{sl.skl.3.7.4.pic}).
Označimo z $a$, $b$ in $c$ premice, ki so v točkah $A$, $B$ in $C$
pravokotne na ustrezne višine. Ker so premice $a$, $b$ in $c$
vzporedne s stranicama trikotnika $ABC$, se vsaki dve med seboj
sekata. Označimo s $P$, $Q$ in $R$ po vrsti presečišča premic $b$ in
$c$, $a$ in $c$ ter $a$ in $b$. Štirikotnika $ABCQ$ in $RBCA$ sta
 paralelograma, kar pomeni, da je $RA \cong BC \cong AQ$, oz. je točka
$A$ središče daljice $RQ$. Premica $AA’$ je zato simetrala
stranice $RQ$ trikotnika $PQR$. Analogno sta $BB’$ in $CC’$
simetrali stranic $PR$ in $PQ$ istega trikotnika. Po izreku
\ref{SredOcrtaneKrozn} se simetrale $AA’$, $BB’$ in $CC’$ trikotnika
$PQR$ sekajo v neki točki $V$. Točka $V$ je torej presečišče
nosilk višin
    $AA'$, $BB'$ in $CC'$
   trikotnika $ABC$.
   \kdokaz

Točko iz prejšnjega izreka, v kateri se sekajo nosilke višin, imenujemo
\index{višinska točka trikotnika}
         \pojem{višinska točka trikotnika}. Trikotnik $A'B'C'$, ki ga
         določajo nožišča višin trikotnika $ABC$ imenujemo
\index{trikotnik!pedalni} \pojem{pedalni trikotnik} trikotnika
$ABC$.

Ugotovili smo, ima vsak trikotnik štiri značilne točke,
in sicer: težišče, središče očrtane krožnice, središče včrtane
krožnice in višinsko točko. Toda to niso edine značilne točke
trikotnika. Nekatere od njih bomo omenili kasneje. Za
neko točko v ravnini trikotnika na splošno rečemo, da je njegova
\pojem{značilna točka}, če je njena definicija simetrična glede na
oglišča tega trikotnika.


\begin{figure}[!htb]
\centering
\input{sl.skl.3.7.5.pic}
\caption{} \label{sl.skl.3.7.5.pic}
\end{figure}


 Jasno je, da se pri poljubnem trikotniku štiri značilne točke razlikujejo
 (Figure \ref{sl.skl.3.7.5.pic}).

          Pri enakokrakem trikotniku imajo težiščnica,
višina, simetrala osnovnice in simetrala notranjega kota nasproti
osnovnice  isto nosilko. Če je $A_1$ središče osnovnice
$BC$ enakokrakega trikotnika $ABC$, sta trikotnika $ABA_1$ in
$ACA_1$ skladna, kar pomeni, da je kot pri oglišču $A_1$ pravi kot
in sta kota $BAA_1$ in $CAA_1$ skladna. Torej je daljica
$AA_1$ hkrati težiščnica in višina, premica $AA_1$ pa hkrati
simetrala stranice $BC$ in simetrala notranjega kota pri oglišču $A$
trikotnika $ABC$. Iz tega sledi, da vse štiri značilne točke tega
trikotnika ležijo na eni premici $AA_1$ (Figure
\ref{sl.skl.3.7.6.pic}).

Če uporabimo že dokazano lastnost enakokrakega trikotnika za
enakostranični trikotnik, ugotovimo, da imajo  pri njem vse ustrezne
težiščnice, višine, simetrale stranic in simetrale notranjih kotov
iste nosilke. To pomeni, da se pri enakostraničnem trikotniku
vse štiri značilne točke prekrivajo (Figure \ref{sl.skl.3.7.6.pic}).
To je pravzaprav že definirano (razdelek \ref{odd3PravilniVeck})
središče tega enakostraničnega (oz. pravilnega) trikotnika.


\begin{figure}[!htb]
\centering
\input{sl.skl.3.7.6.pic}
\caption{} \label{sl.skl.3.7.6.pic}
\end{figure}

Pokazali bomo še, kakšno lego imajo značilne točke glede na vrsto
trikotnika.

\begin{figure}[!htb]
\centering
\input{sl.skl.3.7.7.pic}
\caption{} \label{sl.skl.3.7.7.pic}
\end{figure}

Težiščnice so vedno v notranjosti trikotnika. Zato je tudi težišče
notranja točka vsakega trikotnika (Figure \ref{sl.skl.3.7.7.pic}).
Ista ugotovitev velja tudi za središče včrtane krožnice (Figure
\ref{sl.skl.3.7.7s.pic}).

\begin{figure}[!htb]
\centering
\input{sl.skl.3.7.7s.pic}
\caption{} \label{sl.skl.3.7.7s.pic}
\end{figure}

Pri ostrokotnem trikotniku nožišča pravokotnic iz njegovih oglišč
ležijo na stranicah tega trikotnika, kar pomeni (Paschev aksiom
\ref{AksPascheva}), da se njegove višine sekajo v notranjosti. Torej pri
ostrokotnem trikotniku leži višinska točka  v njegovi notranjosti
(Figure \ref{sl.skl.3.7.7v.pic}). Pri pravokotnem trikotniku je višinska
točka oglišče pri pravemu kotu. To je zato, ker sta njegovi
kateti hkrati višini trikotnika. Višinska točka topokotnega
trikotnika leži v njegovi zunanjosti, ker
 niso vse višine v njegovi notranjosti. Ustrezna nožišča
pripadajo namreč nosilkam stranic, ne pa samim stranicam.

\begin{figure}[!htb]
\centering
\input{sl.skl.3.7.7v.pic}
\caption{} \label{sl.skl.3.7.7v.pic}
\end{figure}

Središče očrtane krožnice je notranja oz. zunanja točka trikotnika,
odvisno od tega, ali je trikotnik ostrokotni oz. topokotni (Figure
\ref{sl.skl.3.7.7o.pic}). Formalni dokaz tega dejstva  bomo
izpustili. Dokažimo le naslednji izrek, ki se nanaša na pravokotne
trikotnike.

\begin{figure}[!htb]
\centering
\input{sl.skl.3.7.7o.pic}
\caption{} \label{sl.skl.3.7.7o.pic}
\end{figure}


        \bizrek The circumcentre of a right-angled triangle is at the same time the midpoint of
        its hypotenuse.
        \eizrek

\begin{figure}[!htb]
\centering
\input{sl.skl.3.7.8.pic}
\caption{} \label{sl.skl.3.7.8.pic}
\end{figure}


\textbf{\textit{Proof.}} Označimo z $O$ središče hipotenuze $AB$ in
s $P$ središče katete $AC$ pravokotnega trikotnika $ABC$ (Figure
\ref{sl.skl.3.7.8.pic}). Daljica $OP$ je srednjica tega trikotnika,
ki ustreza kateti $BC$, zato je $OP\parallel BC$. Iz tega sledi $OP
\perp AC$. Torej sta trikotnika $OPC$ in $OPA$  skladna (izrek
\textit{SAS} \ref{SKS}) in je potem $OC \cong OA$.
Ker je še $OB \cong OA$,
 je točka $O$ središče očrtane krožnice tega trikotnika.
 \kdokaz

Daljica $OC$ iz prejšnjega izreka je težiščnica trikotnika. To
pomeni, da je težiščnica na hipotenuzo pravokotnega
trikotnika enaka polmeru očrtane krožnice tega trikotnika, hkrati pa
tudi polovici njegove hipotenuze.

Prejšnji izrek je povezan tudi s Talesovim izrekom za krožnico
\ref{TalesovIzrKroz} in njegovim obratnim izrekom
\ref{TalesovIzrKrozObrat}. Vse omenjene trditve bomo sedaj podali v
enem izreku.


          \bizrek Thales’ theorem for a circle (several forms - Figure
          \ref{sl.skl.3.7.9.pic}):
         \index{izrek!Talesov za krožnico}
           \label{TalesovIzrKroz2}
           \begin{enumerate}
            \item The circumcentre of a right-angled triangle is at the same time the midpoint of
                 its hypotenuse.
             \item If $t_c$ is the median of a right-angled triangle for its hypotenuse $c$ and $R$
               the circumradius of that triangle, then
            $R=t_c=\frac{c}{2}$.
               \item If $AB$ is a diameter of a circle $k$, then for any point $X\in k$  ($X\neq A$ and $X\neq B$) is
               $\angle AXB=90^0$.
              \item If $A$, $B$ in $X$ are three non-collinear points, such that $\angle AXB=90^0$, then the point $X$ lies on a circle with the diameter $AB$.
                 \end{enumerate}
            \index{izrek!Talesov za krožnico}
             \eizrek

\begin{figure}[!htb]
\centering
\input{sl.skl.3.7.9.pic}
\caption{} \label{sl.skl.3.7.9.pic}
\end{figure}

Poznavanje značilnih točk trikotnika in lastnosti srednjice
trikotnika omogočata dokazovanje raznih drugih lastnosti tako
trikotnika  kot tudi štirikotnika in $n$-kotnika.


              \bzgled
              Let $CD$ be the altitude  at the hypotenuse $AB$ of a right-angled triangle $ABC$.
            If $M$ and $N$ are the midpoints of the line segments $CD$ and $BD$, then $AM \perp CN$.
           \ezgled

\begin{figure}[!htb]
\centering
\input{sl.skl.3.7.10.pic}
\caption{} \label{sl.skl.3.7.10.pic}
\end{figure}

\textbf{\textit{Proof.}}  Daljica $NM$ je srednjica trikotnika
$BCD$, zato je po izreku \ref{srednjicaTrik} $NM \parallel BC$
(Figure \ref{sl.skl.3.7.10.pic}). Ker je kot pri oglišču $C$ pravi
kot, je tudi $NM \perp AC$. Zaradi tega je premica $NM$ nosilka
višine trikotnika $ANC$, Ker je še $CD$ višina tega trikotnika, je
 $M$ njegova višinska točka. Torej je premica $AM$ nosilka
tretje višine tega trikotnika in velja $AM \perp CN$.
 \kdokaz


        \bzgled
        If $P$ and $Q$ are the midpoints of the sides $BC$ and $CD$ of a parallelogram $ABCD$,
          then the lines $AP$ and $AQ$
         divide the diagonal $BD$ of this parallelogram into three congruent line segments.
        \ezgled

\begin{figure}[!htb]
\centering
\input{sl.skl.3.7.11.pic}
\caption{} \label{sl.skl.3.7.11.pic}
\end{figure}

\textbf{\textit{Proof.}} Označimo z $E$ in $F$ presečišči daljic
$AP$ in $AQ$ z diagonalo $BD$ paralelograma $ABCD$ ter s $S$
presečišče njegovih diagonal $AC$ in $BD$ (Figure
\ref{sl.skl.3.7.11.pic}). Diagonali paralelograma se razpolavljata
(izrek \ref{paralelogram}), zato je točka $S$ skupno središče daljic
$AC$ in $BD$. To pomeni, da sta točki $E$ in $F$ težišči trikotnikov
$ACB$ in $ACD$, zato delita težiščnici $SB$ in $SD$ teh trikotnikov
v razmerju $2:1$ (izrek \ref{tezisce}). Torej:
 \begin{eqnarray*}
     BE &=& \frac{2}{3}BS = \frac{2}{3}DS = FD,\\
     EF&=& ES+ SF= \frac{1}{3}SB+ \frac{1}{3}SD=
     \frac{1}{3}(SB +SD)=\frac{1}{3} BD,
 \end{eqnarray*}
  kar smo želeli dokazati.  \kdokaz


           \bzgled
            Let $BAKL$ and $ACPQ$ be positively oriented squares
        in the same plane. Prove that the lines $BP$ and $CL$ intersect at a point lying
        on the line containing the altitude $AA'$ of the triangle $ABC$.
           \ezgled

\begin{figure}[!htb]
\centering
\input{sl.skl.3.7.12.pic}
\caption{} \label{sl.skl.3.7.12.pic}
\end{figure}

\textbf{\textit{Proof.}}
 Naj bo  $AA'$
 višina trikotnika $ABC$ (Figure \ref{sl.skl.3.7.12.pic}). Označimo z $X$
  in $Y$ presečišči premice $AA'$ s
pravokotnicama na premico $CL$ skozi točko $B$ in na premico $BP$ skozi točko $C$.
Dokažimo da je $X = Y$. Trikotnika $BLC$ in $ABX$ sta po izreku \textit{ASA} \ref{KSK} skladna, ker je: $BL \cong AB$,
$\angle BLC\cong\angle ABX$ in $\angle BCL\cong\angle AXB$ (kota s pravokotnima krakoma -
izrek \ref{KotaPravokKraki}). Zaradi tega je $AX \cong BC$. Analogno
sta skladna tudi trikotnika $CPB$ in $ACY$, zato je $AY \cong BC$.
Torej velja $AX \cong AY$ oziroma $X = Y$. To pomeni, da so premice
$AA'$, $BP$ in $CL$ nosilke višin trikotnika $XBC$, zato se sekajo v
eni točki.
 \kdokaz


          \bzgled \label{zgledPravokotnik}
            Let $K$ be the midpoint of the side $CD$ of a rectangle $ABCD$.
          A point $L$ is the foot of the perpendicular from the vertex $B$ on the diagonal
         $AC$ and $S$ is the midpoint of
            the line segment $AL$. Prove that $\angle KSB$ is a right angle.
         \ezgled

\begin{figure}[!htb]
\centering
\input{sl.skl.3.7.13.pic}
\caption{} \label{sl.skl.3.7.13.pic}
\end{figure}

\textbf{\textit{Proof.}}
 Naj bo $V$ središče daljice $BL$ (Figure \ref{sl.skl.3.7.13.pic}).
 Daljica $SV$ je srednjica
 trikotnika $ABL$ za osnovnico $AB$, zato je $SV\parallel AB$ in
 $SV =\frac{1}{2} AB$ (izrek \ref{srednjicaTrik}). Iz
 prve relacije in $BC\perp AB$ sledi $SV\perp BC$
 (izrek \ref{KotiTransverzala}). To pomeni, da sta
 $BL$ in $SV$ nosilki višin trikotnika $CSB$. Torej je $V$  višinska
 točka tega trikotnika, zato je $CV$ nosilka njegove tretje višine
 (izrek \ref{VisinskaTocka}) oz. velja $CV\perp SB$. Iz $SV\parallel AB$ in
 $SV =\frac{1}{2} AB=KC$ sledi, da je štirikotnik $SVCK$
 paralelogram oz. $CV\parallel SK$. Iz tega in $CV\perp SB$ na
 koncu sledi (izrek \ref{KotiTransverzala})  $SK\perp SB$, torej je
$\angle KSB$ pravi kot.
 \kdokaz



              \bzgled
              Let $AP$, $BQ$ and $CR$ be the altitude, the median
            and the bisector of the angle $ACB$ ($R\in AB$) of a triangle $ABC$. Prove that if
            the triangle $PQR$ is regular, then the triangle $ABC$ is also regular.
             \ezgled


\begin{figure}[!htb]
\centering
\input{sl.skl.3.7.14.pic}
\caption{} \label{sl.skl.3.7.14.pic}
\end{figure}

\textbf{\textit{Proof.}} Naj bo $PQR$ pravilni trikotnik oz.
$PQ\cong QR\cong RP$ (Figure \ref{sl.skl.3.7.14.pic}). Točka $Q$ je
središče hipotenuze $AC$ pravokotnega trikotnika $APC$, zato je $QA
\cong QC \cong QP$ (izrek \ref{TalesovIzrKroz2}. Ker je v tem
primeru tudi $QR\cong QC\cong QP$, iz istega izreka sledi, da je tudi
$\angle ARC$ pravi kot. Iz skladnosti trikotnikov $ACR$ in $BCR$
(izrek \ref{KSK}) dobimo, da je točka $R$ središče stranice $AB$ in
da velja $AC\cong BC$. Ker je točka $R$ središče hipotenuze $AB$
pravokotnega trikotnika $APB$, je $AB = 2RP = 2PQ = 2AQ = AC$.
Torej velja  $AB\cong AC\cong BC$, kar pomeni, da je tudi $ABC$
pravilni trikotnik.
 \kdokaz


Ni težko dokazati, da je nek trikotnik enakokrak natanko tedaj, ko
sta ustrezni težiščnici skladni. Analogno velja tudi za višine. Ali
podobno velja tudi za t. i. odseke simetral kotov?
 Daljici $BB'$ in $CC'$, kjer sta $BB'$ in $CC'$ simetrali notranjih kotov
  trikotnika $ABC$ ter $B'\in AC$ in $C'\in AB$, imenujemo
 \index{odsek simetrale kota} \pojem{odseka simetral kotov}. Odseke simetral ozačimo z $l_a$, $l_b$ in $l_c$.
 Omenjena
  trditev  velja tudi v tem primeru, ampak
dokaz ni tako enostaven. O tem govori naslednji znani izrek.



            \bizrek \index{izrek!Steiner-Lemusov}
            (Steiner-Lehmus\footnote{\textit{D. C. L. Lehmus} (1780--1863),\index{Lehmus, D. C. L.} francoski
            matematik, ki je leta 1840 poslal to, na prvi pogled enostavno
            trditev, slavnem švicarskemu geometru \index{Steiner, J.} \textit{J. Steinerju} (1796--1863),
             ki  je izpeljal zelo obsežen dokaz tega izreka. Nato je sledilo
             več različnih rešitev tega problema in eno od njih je leta 1908
              objavil
             francoski matematik \index{Poincar\'{e}, J. H.} \textit{J. H. Poincar\'{e}} (1854--1912).})
            Let  $BB'$ ($B' \in AC$) and $CC'$ ($C' \in AB$) be the bisectors
              of the interior angles of a triangle $ABC$. Then:
             $$AB \cong AC \Leftrightarrow BB'\cong CC'.$$
            \eizrek


\begin{figure}[!htb]
\centering
\input{sl.skl.3.7.15.pic}
\caption{} \label{sl.skl.3.7.15.pic}
\end{figure}

\textbf{\textit{Proof.}} (Figure \ref{sl.skl.3.7.15.pic})

($\Rightarrow$) Iz $AB \cong AC$ sledi $\angle ABC \cong \angle
ACB$ (izrek \ref{enakokraki}) oz. $\angle B'BC \cong \angle C'CB$.
Po izreku \textit{ASA} \ref{KSK} sta trikotnika $B'BC$ in $C'CB$
skladna, zato je $BB'\cong CC'$.

($\Leftarrow$) Naj bo  $BB'\cong CC'$.
 Predpostavimo,  da ni $AB\cong AC$. Brez škode za splošnost
 naj bo $AB < AC$. V tem primeru je $\angle ACB < \angle ABC$
 (izrek \ref{vecstrveckot}) oz.
$\angle ACC'< \angle ABB'$. To pomeni, da v notranjosti kota $ABB'$
obstaja poltrak $p$ z izhodiščem $B$, ki stranico $AC$ seka v
takšni točki $D$, da hkrati velja  $\mathcal{B}(A,D,B')$ in $\angle
DBB'\cong \angle ACC'$. V trikotniku $BCD$ je $\angle ACB < \angle
DBC$ in zaradi tega tudi $BD < CD$ (izrek \ref{vecstrveckot}).
Torej obstaja takšna točka $E$, ki je med točkama $C$ in $D$, tako da je
$BD \cong CE$. Po izreku \textit{SAS} \ref{SKS} sta trikotnika
$BDB'$ in $CEC'$ skladna, zato sta skladna tudi kota $BDB'$ in
$CEC'$. Dokažimo, da to ni mogoče. Zaradi Paschevega aksioma
\ref{AksPascheva} (uporabljenega za trikotnik $AC'E$ in premico
$BD$) premica $BD$ seka daljico $C'E$ v neki točki $S$. V trikotniku
$SDE$ je kot $SEC$ (oz. kot $CEC'$) zunanji in po izreku
\ref{zunanjiNotrNotrVecji} ne more biti skladen nesosednjemu notranjemu
kotu $SDE$ (oz. kotu $BDB'$). To pomeni, da predpostavka $AB < AC$
(analogno tudi $AB > AC$) ni mogoča. Torej je $AB\cong AC$.
 \kdokaz


         \bzgled
          Let $A_1$ be the midpoint of the side $BC$ of a triangle $ABC$.
         Calculate the measure of the angle $AA_1C$, if
           $\angle BAC=45^0$ and $\angle ABC=30^0$.
         \ezgled

\begin{figure}[!htb]
\centering
\input{sl.skl.3.7.1a.pic}
\caption{} \label{sl.skl.3.7.1a.pic}
\end{figure}

\textbf{\textit{Proof.}} Naj bo $CC'$ višina trikotnika $ABC$ (Figure
\ref{sl.skl.3.7.1a.pic}).

Iz $\angle CC'B = 90^0$ sledi najprej
$\angle C'CB=60^0$, nato pa še, da točka $C'$ leži na krožnici nad
premerom $CB$ in s središčem $A_1$ (izrek \ref{TalesovIzrKroz}), zato
je $A_1C'\cong A_1C\cong A_1B$. Torej je trikotnik $CC'A_1$
enakokrak, oz. po izreku \ref{enakokraki} velja $\angle CC'A_1
\cong\angle C'CB=60^0$. To pomeni, da je trikotnik $CC'A_1$
pravilen in je $C'C\cong C'A_1$. Iz dejstva, da je $AC'C$ enakokraki
trikotnik ($\angle CAC'=\angle ACC'=45^0$), pa sledi $AC'\cong C'C$. Če to povežemo s prejšnjo relacijo, dobimo $AC'\cong C'A_1$, kar
pomeni, da je tudi trikotnik $AC'A_1$ enakokrak. Zato je (izreka
\ref{enakokraki} in \ref{zunanjiNotrNotr}):
 $$\angle C'A_1A\cong\angle C'AA_1=\frac{1}{2}\angle A_1C'B=
 \frac{1}{2}\angle C'BA_1=\frac{1}{2}\cdot 30^0=15^0.$$
 Na koncu je še:
  $$\angle AA_1C=\angle C'A_1C-\angle C'A_1A =60^0-15^0=45^0,$$ kar je bilo potrebno izračunati. \kdokaz


        \bzgled
        Let $P$ be the midpoint of the side $BC$ of an isosceles triangle $ABC$
            and $Q$ the foot of the perpendicular from the point $P$ on the leg
        $AC$ of that triangle. Let $S$ be the midpoint of the line segment $PQ$. Prove that
        $AS \perp BQ$.
        \ezgled


\begin{figure}[!htb]
\centering
\input{sl.skl.3.7.16.pic}
\caption{} \label{sl.skl.3.7.16.pic}
\end{figure}

\textbf{\textit{Proof.}} Iz skladnosti trikotnikov $ABP$ in $ACP$
(izrek \textit{SSS} \ref{SSS}) sledi skladnost sokotov $APB$ in
$APC$ oz. $AP\perp BC$ (Figure \ref{sl.skl.3.7.16.pic}). Označimo z
$R$ središče daljice $QC$. Daljica $SR$ je srednjica trikotnika
$QPC$ za osnovnico $PC$, zato je po izreku \ref{srednjicaTrik}
$SR\parallel CP$. Iz tega in $AP\perp BC$ sledi $SR\perp AP$. Torej je
$S$  višinska točka trikotnika $APR$, zato je po izreku
\ref{VisinskaTocka} tudi $AS\perp PR$. Toda daljica $PR$ je
srednjica trikotnika $BQC$ za osnovnico $BQ$, zato je $PR\parallel BQ$
(izrek \ref{srednjicaTrik}). Iz $AS\perp PR$ in $PR\parallel BQ$
dobimo $AS\perp BQ$.
 \kdokaz


       \bzgled \label{kotBSC}
      If $S$ is the incentre and $\alpha$, $\beta$,
       $\gamma$ the interior angles  at the vertices
        $A$, $B$, $C$  of a triangle
       $ABC$, then
       $$\angle BSC=90^0+\frac{1}{2}\cdot\alpha.$$
       \ezgled


\begin{figure}[!htb]
\centering
\input{sl.skl.3.7.1c.pic}
\caption{} \label{sl.skl.3.7.1c.pic}
\end{figure}

\textbf{\textit{Proof.}}  Po izreku \ref{SredVcrtaneKrozn} se
simetrale notranjih kotov trikotnika $ABC$ sekajo v središču včrtane
krožnice - v točki $S$ (Figure \ref{sl.skl.3.7.1c.pic}). Zato je
$\angle SBC =\frac{1}{2}\cdot \beta$ in $\angle SCB
=\frac{1}{2}\cdot \gamma$. Ker je po izreku \ref{VsotKotTrik} v vsakem
trikotniku vsota notranjih kotov enaka $180^0$, sledi:
 $$\angle BSC = 180^0-\frac{1}{2}\cdot\left( \beta+
  \gamma\right)=180^0-\frac{1}{2}\cdot\left( 180^0-
 \alpha\right)=90^0+\frac{1}{2}\cdot\alpha,$$ kar je bilo treba dokazati. \kdokaz


        \bzgled
        Construct a triangle with given $a$, $t_a$, $R$ (see the labels  in section \ref{odd3Stirik}).
        \ezgled


\begin{figure}[!htb]
\centering
\input{sl.skl.3.7.16a.pic}
\caption{} \label{sl.skl.3.7.16a.pic}
\end{figure}

\textbf{\textit{Proof.}} Konstrukcijo lahko izpeljemo tako, da najprej narišemo očrtano krožnico  $k(O,R)$, izberemo poljubno točko $B\in k$, načrtamo tetivo $BC\cong a$ te krožnice, središče $A_1$ tetive $BC$ in na koncu oglišče $A$ kot presečišče krožnic $k(O,R)$ in $k_1(A_1,t_a)$ (Figure \ref{sl.skl.3.7.16a.pic}). Jasno je, da ima naloga  rešitve natanko tedaj, ko je $a\leq 2R$ in presečišče krožnic $k(O,R)$ in $k_1(A_1,t_a)$ ni prazna množica. Število rešitev je v tem primeru odvisno od števila presečišč krožnic $k(O,R)$ in $k_1(A_1,t_a)$.
 \kdokaz

        \bzgled
        Construct a right-angled triangle if the hypotenuse and the altitude to that hypotenuse
         are congruent to the given line segments $c$ and $v_c$.
        \ezgled



\begin{figure}[!htb]
\centering
\input{sl.skl.3.7.16b.pic}
\caption{} \label{sl.skl.3.7.16b.pic}
\end{figure}

\textbf{\textit{Analysis.}}
Naj bo $ABC$ pravokotni trikotnik s pravim kotom v oglišču $C$,
pri katerem sta hipotenuza $AB$ in višina $CC'$ skladni z daljicama $c$ in $v_c$ (Figure \ref{sl.skl.3.7.16b.pic}).
Po izreku \ref{TalesovIzrKroz2} je središče $O$ hipotenuze $AB$ hkrati središče očrtane krožnice trikotnika $ABC$.
Torej oglišče $C$  leži na krožnici $k$ s premerom $AB$. Ker je še $CC'\cong v_c$, leži oglišče $C$ tudi na vzporednici premice $AB$,
ki je od nje oddaljena za $v_c$. Točka $C$ je potem presečišče te vzporednice in krožnice $k$.


\textbf{\textit{Construction.}}
Najprej načrtajmo  daljico $AB$, ki je skladna z dano daljico $c$, nato še središče $O$ daljice $AB$ in krožnico $k(O,OA)$. Potem načrtajmo  vzporednico $p$ premice $AB$ na razdalji $v_c$. Eno od presečišč premice $p$ in krožnice $k(O,OA)$ označimo s $C$. Dokažimo, da je $ABC$ iskani trikotnik.


\textbf{\textit{Proof.}}
 Po konstrukciji točka $C$ leži na krožnici s polmerom $AB$, zato je po izreku \ref{TalesovIzrKroz2} $\angle ACB=90^0$, kar pomeni, da je $ABC$ pravokotni trikotnik s hipotenuzo $AB$. Po konstrukciji je le-ta skladna z daljico $c$. Naj bo $CC'$ višina trikotnika $ABC$. Po konstrukciji leži točka $C$ na premici $p$, ki je od premice $AB$ oddaljena $v_c$, zato je tudi $|CC'|=d(C,AB)=d(p,AB)$ oz. $CC'\cong v_c$.


\textbf{\textit{Discussion.}}
 Število rešitev je odvisno od števila presečišč premice $p$ in krožnice $k(O,OA)$.
 \kdokaz



        \bnaloga\footnote{2. IMO Romania - 1960, Problem 4.}
         Construct triangle ABC, given $v_a$, $v_b$ (the altitudes from $A$ and $B$) and $t_a$,
         the median from vertex $A$.
         \enaloga


\begin{figure}[!htb]
\centering
\input{sl.skl.3.7.IMO1.pic}
\caption{} \label{sl.skl.3.7.IMO1.pic}
\end{figure}

\textbf{\textit{Solution.}} Naj bo $ABC$ takšen trikotnik, da sta
$AA' \cong v_a$ in $BB' \cong v_b$ njegovi višini ter $AA_1\cong t_a$
njegova težiščnica (Figure \ref{sl.skl.3.7.IMO1.pic}). Označimo
z $A'_1$ pravokotno projekcijo točke $A_1$ na premici $AC$.
Daljica $A_1A'_1$ je srednjica trikotnika $BB'C$ za osnovnico
$BB'$, zato je po izreku \ref{srednjicaTrik}:
 $$|A_1A'_1|=\frac{1}{2}\cdot|BB'|=\frac{1}{2}\cdot v_b
 \hspace*{1mm} \textrm{ in }  \hspace*{1mm} A_1A'_1
\parallel BB'.$$ Iz tega sledi, da sta premici
$A_1A'_1$ in $AC$ pravokotni v točki $A'_1$, torej je premica $AC$
tangenta krožnice $k(A_1,\frac{1}{2} v_b)$ \ref{TangPogoj}.
Dokazane lastnosti nam omogočajo konstrukcijo.

 Najprej lahko načrtamo pravokotni trikotnik $AA'A_1$ ($AA'\cong v_a$,
 $AA_1\cong t_a$ in $\angle AA'A_1 = 90^0$), nato pa krožnico
 $k(A_1,\frac{1}{2} v_b)$. Iz točke $A$ načrtamo tangenti na
 krožnico $k(A_1,\frac{1}{2} v_b)$. Presečišče ene od tangent
 s premico $A'A_1$ označimo s $C$. Na koncu načrtamo takšno točko
  $B$, da velja
  $BA_1 \cong CA_1$ in $\mathcal{B}(C,A_1,B)$.

 Dokažimo, da trikotnik $ABC$ izpolnjuje dane pogoje. Iz
 konstrukcije je  $AA'\cong v_a$ višina in $AA_1 \cong t_a$
 težiščnica (ker je $A_1$ središče daljice $BC$) trikotnika
 $ABC$. Naj bo $BB'$ višina tega trikotnika. Dokažimo še $BB' \cong
 v_b$.
 Premica $AC$ je po konstrukciji tangenta krožnice $k(A_1,\frac{1}{2}
 v_b)$. Njuno dotikališče označimo z $A'_1$. Po izreku
 \ref{TangPogoj} sta premici
$A_1A'_1$ in $AC$ pravokotni v točki $A'_1$, zato je daljica
$A_1A'_1$ srednjica trikotnika $BB'C$ za osnovnico $BB'$ in velja
$|BB'|= 2\cdot |A_1A'_1|=2\cdot\frac{1}{2}\cdot v_b=v_b$.

 Naloga nima rešitev, kadar je $v_a>t_a$. Če je  $v_a\leq
 t_a$, je število rešitev odvisno od števila tangent, ki jih
 lahko načrtamo iz točke $A$ na krožnico $k(A_1,\frac{1}{2}
 v_b)$. Pri tem mora tangenta sekati premico $A'A_1$.
 Če velja $\frac{1}{2} v_b<t_a$ in $\frac{1}{2} v_b\neq v_a$,
  ima naloga dve rešitvi, v primeru $\frac{1}{2} v_b<t_a$ in
  $\frac{1}{2} v_b = v_a$ je rešitev le ena,  v primeru
  $\frac{1}{2} v_b\geq t_a$ pa rešitve ni.
 \kdokaz


%________________________________________________________________________________
 \poglavje{Euler's Circle. Eight Point Circle}
 \label{odd3EulKroz}

Sedaj bomo obravnavali zanimivo lastnost, ki se nanaša na
\index{trikotnik!pedalni}pedalni trikotnik, ki smo ga že definirali
kot trikotnik, ki je določen z nožišči višin nekega trikotnika, in t.
i. \index{trikotnik!središčni} \pojem{središčni trikotnik}, ki ga
določajo središča stranic tega trikotnika. Dokazali bomo namreč, da
imata omenjena trikotnika  skupno očrtano krožnico (Figure
\ref{sl.skl.3.8.1.pic}). Pred tem pa dokažimo naslednji pomožni izrek.


\begin{figure}[!htb]
\centering
\input{sl.skl.3.8.1.pic}
\caption{} \label{sl.skl.3.8.1.pic}
\end{figure}



        \bizrek \label{EulerKroznicaLema}
        Let $V$ be the orthocentre of a triangle $ABC$. If $K$, $L$, $M$
         and $N$ are the midpoints of the line segments $AB$, $AC$, $VC$
        and $VB$, respectively, then the quadrilateral $KLMN$ is a rectangle.
        \eizrek


\begin{figure}[!htb]
\centering
\input{sl.skl.3.8.2.pic}
\caption{} \label{sl.skl.3.8.2.pic}
\end{figure}

\textbf{\textit{Proof.}}
 Naj bo $AA'$ višina tega trikotnika. Daljici $KN$
in $LM$ sta srednjici trikotnikov $ABA'$ in $CAA'$ za skupno
osnovnico $AA'$, zato je po izreku \ref{srednjicaTrik}
$KN=\frac{1}{2}AA'=LM$ in $KN\parallel AA'\parallel LM$ (Figure
\ref{sl.skl.3.8.2.pic}). Torej je štirikotnik $KLMN$  paralelogram.
Dovolj je dokazati, da ima vsaj en notranji kot pravi.
Daljica $KL$ je srednjica trikotnika $ABC$, zato je $KL\parallel
BC$. Ker je še $KN\parallel AA'$ in $AA'\perp BC$, je tudi
$KL\perp KN$ oz. $\angle LKN=90^0$, kar pomeni, da je
paralelogram $KLMN$ hkrati pravokotnik.
 \kdokaz

 Sedaj smo pripravljeni na dokaz osnovnega izreka.



        \bizrek \label{EulerKroznica}
        For any given triangle
        the midpoints of the sides, the foots of the altitudes and
         the midpoints of the line segments from each vertex of the triangle to the orthocentre
         lie on the common circle - so-called  \index{krožnica!Eulerjeva}
        \pojem{Euler’s circle}\color{blue}\footnote{Krožnico imenujemo po
        švicarskem matematiku \index{Euler, L.} \textit{L. Eulerju}
        (1707--1783), ki je že leta 1765 dokazal, da imata pedalni in središčni
        trikotnik  skupno očrtano krožnico.
        Ostale lastnosti te krožnice so  leta 1821
        najprej obravnavali francoski matematiki \index{Poncelet, J. V.}
        \index{Brianchon, C. J.} \index{Terquem, O.} \index{Feuerbach, K. W.}
        \textit{J. V.
        Poncelet} (1788--1867) \textit{C. Brianchon}  (1783--1864) in
        \textit{O. Terquem}
        (1782--1862), kasneje pa še nemški matematik
         \textit{K. W. Feuerbach} (1800--1834).} of that triangle.
        \eizrek

\begin{figure}[!htb]
\centering
\input{sl.skl.3.8.3.pic}
\caption{} \label{sl.skl.3.8.3.pic}
\end{figure}

\textbf{\textit{Proof.}}
 Uporabili bomo prejšnjo trditev \ref{EulerKroznica}. Če ohranimo
 iste oznake, smo že dokazali, da je štirikotnik $KLMN$ pravokotnik
 (Figure
\ref{sl.skl.3.8.3.pic}). Če s $P$ označimo središče
stranice $BC$ in s $Q$ središče daljice $AV$, je tudi $PMQK$
pravokotnik. Ker je $KM$ skupna diagonala teh dveh pravokotnikov, je
ta premer njune skupne očrtane krožnice $e$. Torej središča
stranic in središča daljic,
         ki povezujejo višinsko točko in oglišča,
        pripadajo isti krožnici $e$. Dokažimo še, da tudi nožišča višin
ležijo na tej krožnici. Točka $A'$, ki je nožišče višine iz oglišča
$A$, leži na krožnici $e$, ker je $\angle QA'P$ pravi kot in $PQ$
premer krožnice $e$ (Talesov izrek \ref{TalesovIzrKroz2}). Analogno na tej krožnici ležita tudi  nožišči $B'$ in $C'$ višin $BB'$
in $CC'$.
 \kdokaz

 Eulerjevo krožnico imenujemo tudi
 \index{krožnica!devetih točk} \index{krožnica!Feuerbachova}
 \pojem{Feuerbachova krožnica} in \pojem{krožnica devetih točk}.
 Še nekaj lastnosti Eulerjeve krožnice bomo obravnavali v razdelkih \ref{odd5EulPrem} in \ref{odd7SredRazteg}. Sedaj pa dokažimo analogno trditev za
 štirikotnike  t. i. \index{krožnica!osmih točk} \pojem{krožnica osmih točk}.



        \bizrek
        Let $ABCD$ be a quadrilateral with the perpendicular diagonals.
        Then the midpoints of the sides
         and the foots of the perpendiculars from these midpoints to the line
         containing the opposite sides of that quadrilateral lie on the same circle.
        \eizrek

\begin{figure}[!htb]
\centering
\input{sl.skl.3.8.4.pic}
\caption{} \label{sl.skl.3.8.4.pic}
\end{figure}

\textbf{\textit{Proof.}} Naj bo $ABCD$ štirikotnik s pravokotnima
diagonalama $AC$ in $BD$ (Figure \ref{sl.skl.3.8.4.pic}). Označimo z
$A_1$, $B_1$, $C_1$ in $D_1$ središča njegovih stranic $AB$, $BC$,
$CD$ in $DA$ ter z $A'$, $B'$, $C'$ in $D'$ pravokotne projekcije
teh središč na nosilkah nasprotnih stranic tega štirikotnika.
Štirikotnik $A_1B_1C_1D_1$ je paralelogram (Varignonov paralelogram
- izrek \ref{Varignon}). Ker sta diagonali $AC$ in $BD$ pravokotni,
je ta paralelogram  pravokotnik (izrek \ref{VarignonPoslPravRomb}),
zato njegova oglišča ležijo na isti krožnici. Diagonali $A_1C_1$ in
$B_1D_1$ sta premera te krožnice. Ker je $\angle C_1C'A_1=\angle
C_1A'A_1=\angle B_1D'D_1=\angle B_1B'D_1=90^0$, sledi (Talesov izrek
\ref{TalesovIzrKroz2}), da tudi točke $A'$, $B'$, $C'$ in $D'$ ležijo
na tej krožnici.
 \kdokaz

 Omenimo še, da lahko za poljuben trikotnik $ABC$ z višinsko točko $V$
  njegovo Eulerjevo krožnico  vidimo kot
krožnico osmih točk štirikotnika $ABVC$ (njegovi diagonali $AV$ in
$BC$ sta pravokotni - slika \ref{sl.skl.3.8.5.pic}), vendar se v tem
primeru dva para točk prekrivata in dejansko dobimo le šest
točk\footnote{To dejstvo je leta 1944 dokazal ameriški matematik
\index{Brand, L.} \textit{L. Brand} (1885--1971)}. Da bi dokazali, da
trditev (za Eulerjevo krožnico) velja za ostale tri točke, uporabimo
še krožnico osmih točk za štirikotnik $CAVB$. Ker imata dve krožnici
(za štirikotnika $ABVC$ in $CAVB$)  vsaj tri skupne točke, se
krožnici prekrivata – sta identični Eulerjevi krožnici trikotnika
$ABC$.

\begin{figure}[!htb]
\centering
\input{sl.skl.3.8.5.pic}
\caption{} \label{sl.skl.3.8.5.pic}
\end{figure}

 %_______________________________________________________________________________
 \poglavje{Tessellations} \label{odd3Tlakovanja}

V tem razdelku se bomo ukvarjali s prekrivanjem ravnine s skladnimi
liki. Takšno prekrivanje imenujemo \index{tlakovanja}
\pojem{tlakovanje} ali \index{teselacija} \pojem{teselacija}
ravnine. Lik, s katerim na ta način pokrivamo ravnino, je
\index{celica tlakovanja} \pojem{celica tlakovanja}. Najbolj znano
 tlakovanje je seveda prekrivanje ravnine s skladnimi
kvadrati. Obravnavali bomo še tlakovanje z drugimi liki. Najprej
bomo dali odgovor na vprašanje, katera so možna tlakovanja s
pravilnimi večkotniki.



        \bizrek \label{pravilnaTlakovanja}
        All possible tessellations $(n,m)$ of the plane with a regular $n$-gons,
        $m$ of them around each vertex, are
        (Figure \ref{sl.skl.3.9.1.pic})\footnote{Ta problem je rešil
        znameniti starogrški filozof in matematik \index{Pitagora}
        \textit{Pitagora z otoka Samosa}
         (582--497 pr. n. š.).}:
        $$(4,4),\hspace*{1mm} (6,3)\textrm{ in }
        (3,6).$$
        \eizrek

\begin{figure}[!htb]
\centering
\input{sl.skl.3.9.1aa.pic}
\input{sl.skl.3.9.1bb.pic}
\input{sl.skl.3.9.1.pic}
\caption{} \label{sl.skl.3.9.1.pic}
\end{figure}

\textbf{\textit{Proof.}} Naj bo $O$ središče in $AB$ ena stranica
celice tlakovanja $(n,m)$ - pravilnega $n$-kotnika (Figure
\ref{sl.skl.3.9.2.pic}). S $S$ označimo središče stranice $AB$. Ker
je začetni $n$-kotnik pravilen in je $m$ takšnih okrog oglišča $B$,
notranji koti trikotnika $OSB$ pri ogliščih $O$, $B$ in $S$  po vrsti
merijo  $\frac{360^0}{2n}$, $\frac{360^0}{2m}$ in $90^0$. Po izreku
\ref{VsotKotTrik} je $\frac{360^0}{2n}+\frac{360^0}{2m}+90^0=180^0$.
Če enakost poenostavimo, dobimo ekvivalentno enakost:
$$\frac{1}{n}+\frac{1}{m}=\frac{1}{2},$$
oz. $nm-2n-2m=0$ in na koncu:
 \begin{eqnarray}
(n-2)(m-2)=4. \label{teselRelEvk}
\end{eqnarray}

 Ker sta $n$ in $m$ naravni števili in večji kot $2$,
so edine rešitve zadnje enačbe: $(n,m)\in \{(4,4), (3,6), (6,3)\}$.
 \kdokaz

\begin{figure}[!htb]
\centering
\input{sl.skl.3.9.2.pic}
\caption{} \label{sl.skl.3.9.2.pic}
\end{figure}

Tlakovanja ravnine s pravilnimi večkotniki imenujemo
\index{tlakovanja!pravilna} \pojem{pravilna tlakovanja} ravnine. V
evklidski ravnini  obstajajo torej tri pravilna tlakovanja.

Ker je v hiperbolični geometriji vsota notranjih kotov v trikotniku
vedno manjša od $180^0$, relacija za trikotnik $OSB$ iz prejšnjega
\ref{pravilnaTlakovanja} izreka postane:
$\frac{360^0}{2n}+\frac{360^0}{2m}+90^0<180^0$ in potem namesto
relacije \ref{teselRelEvk} dobimo:
 \begin{eqnarray}
(n-2)(m-2)>4. \label{teselRelHyp}
\end{eqnarray}
 Ta neenačba ima neskončno mnogo rešitev v množici $\mathbb{N}^2$, kar pomeni,
da imamo v hiperbolični geometriji  neskončno mnogo pravilnih tlakovanj.
Dve od njih sta npr. $(3,7)$ in $(4,5)$ (Figure
\ref{sl.skl.3.9.2H.pic}\footnote{http://math.slu.edu/escher/index.php/Category:Hyperbolic-Tessellations}).
Pri slednji se pet kvadratov stika okrog enega oglišča. To je
mogoče, ker je v hiperbolični geometriji notranji kot pri kvadratu vedno
oster in ni konstanten. Izkaže se, da ima kvadrat z daljšo stranico
manjši notranji kot. Mogoče je izbrati takšno stranico kvadrata,
da je notranji kot enak $\frac{360^0}{5} =72^0$, kar ravno
ustreza tlakovanju $(4,5)$.

\begin{figure}[!htb]
\centering
\includegraphics[width=0.413\textwidth]{whyptess1.eps}\hspace*{4mm}
 \includegraphics[width=0.387\textwidth]{whyptess.eps}
\caption{} \label{sl.skl.3.9.2H.pic}
\end{figure}

V eliptični geometriji, kjer je vsota kotov v trikotniku vedno
večja od $180^0$, omenjena relacija za trikotnik $OSB$ postane:
$\frac{360^0}{2n}+\frac{360^0}{2m}+90^0>180^0$, oz.:
 \begin{eqnarray}
(n-2)(m-2)<4. \label{teselRelElipt}
\end{eqnarray}
Ta enačba ima v množici $\mathbb{N}^2$  rešitve $(3,3)$, $(4,3)$,
$(3,4$), $(5,3)$ in $(3,5)$. Ker se eliptična geometrija realizira
kot model na sferi, te rešitve pomenijo tlakovanja sfere s sfernimi
večkotniki. Stranice teh večkotnikov so loki velikih krožnic sfere.
Če v evklidskem prostoru z daljicami povežemo ustrezna oglišča teh
tlakovanj, dobimo t. i. \index{pravilni!poliedri} \pojem{pravilne
poliedre} (Figure
\ref{sl.skl.3.9.2E.pic}\footnote{http://www.upc.edu/ea-smi/personal/claudi/web3d/}):
\pojem{pravilni tetraeder}, \pojem{kocko} (oz. \pojem{pravilni
heksaeder}), \pojem{pravilni oktaeder}, \pojem{pravilni dodekaeder}
in \pojem{pravilni ikozaeder}. Npr. $(4,3)$ bi predstavljal kocko,
pri kateri se po trije kvadrati (pravilna 4-kotnika) stikajo v eni
točki.


\begin{figure}[!htb]
\centering
 \includegraphics[bb=0 0 11cm 6cm]{wpoliedri.eps}
\caption{} \label{sl.skl.3.9.2E.pic}
\end{figure}


Vrnimo se nazaj k evklidski ravnini. Če dopustimo možnost, da pri
prekrivanju ravnine uporabimo več (končno mnogo) vrst pravilnih
večkotnikov oz. več vrst celic, ki so enako razporejene v vsakem oglišču, potem razen treh pravilnih
tlakovanj iz izreka \ref{pravilnaTlakovanja} obstaja še osem t. i.
\index{tlakovanja!Arhimedova} \pojem{Arhimedovih tlakovanj}
\footnote{Dokaz te trditve je izvedel nemški astronom, matematik in
fizik \index{Kepler, J.} \textit{J. Kepler} (1571--1630).} (Figure
\ref{sl.skl.3.9.2A.pic}\footnote{http://commons.wikimedia.org/wiki/File\%3AArchimedean-Lattice.png}).



\begin{figure}[!htb]
\centering
 \includegraphics[bb=0 0 10cm 7.7cm]{Archimedean.eps}
\caption{} \label{sl.skl.3.9.2A.pic}
\end{figure}

Razen pravilnih in Arhimedovih tlakovanj obstajajo tudi druga
tlakovanja z večkotniki, ki niso pravilni. Najbolj enostaven primer
je tlakovanje s skladnimi paralelogrami (Figure
\ref{sl.skl.3.9.3a.pic}), ki ga dobimo, če pravilno
tlakovanje $(4,4)$ tako deformiramo, da se namesto kvadratov v enem oglišču stikajo štirje
paralelogrami. To tlakovanje je določeno z mrežo
dveh šopov vzporednic.

\begin{figure}[!htb]
\centering
\input{sl.skl.3.9.3aa.pic}
\input{sl.skl.3.9.3aaa.pic}
\caption{} \label{sl.skl.3.9.3a.pic}
\end{figure}

\begin{figure}[!htb]
\centering
\input{sl.skl.3.9.3bb.pic}
\input{sl.skl.3.9.3cc.pic}
\caption{} \label{sl.skl.3.9.3bc.pic}
\end{figure}

Če vse paralelograme razdelimo na
dva trikotnika z diagonalami, ki imajo isto smer,  dobimo tlakovanje ravnine s skladnimi trikotniki
(Figure \ref{sl.skl.3.9.3a.pic}). Trikotnik (osnovna celica) je lahko
poljuben, saj lahko dva takšna (skladna) trikotnika po skupni
stranici vedno povežemo  v paralelogram in tako dobimo tlakovanje s
paralelogrami. Tlakovanje s poljubnimi skladnimi trikotniki je
posplošitev pravilnega tlakovanja $(3,6)$ s pravilnimi trikotniki.


%\vspace*{-1mm}

Kot posebna primera tlakovanja s paralelogrami dobimo tlakovanje s
pravokotniki in tlakovanje z rombi (Figure \ref{sl.skl.3.9.3bc.pic}).
Dokazali bomo še bolj splošno
trditev, ki morda ni tako samoumevna. Mogoče je namreč tudi
tlakovanje s poljubnimi skladnimi štirikotniki.

%\vspace*{-1mm}



        \bzgled
        Let $ABCD$ be an arbitrary quadrilateral. Prove that it is
        possible to tessellate the plane with the cell $ABCD$ so that each vertex
         is surrounded by four such quadrilaterals.
        \ezgled


\begin{figure}[!htb]
\centering
\input{sl.skl.3.9.4.pic}
\caption{} \label{sl.skl.3.9.4.pic}
\end{figure}

 \textbf{\textit{Proof.}}  (Figure
\ref{sl.skl.3.9.4.pic})
 Naj bo $ABCD$ poljuben štirikotnik v
ravnini z notranjimi koti $\alpha$, $\beta$, $\gamma$ in $\delta$.
Točki $O$ in $S$ naj bosta središči njegovih stranic $AB$ in $BC$. S središčnima
zrcaljenjema
 $\mathcal{S}_O$ in $\mathcal{S}_S$ (za definicijo središčnega zrcaljenja
 glej razdelek \ref{odd6SredZrc})
 se štirikotnik  $ABCD$ preslika v štirikotnika
$BAC_1D_1$ in $A_2CBD_2$. Pri tem je $\angle ABD_1\cong\alpha$ in
$CBD_2\cong\gamma$, zato je tudi $\angle D_1BD_2\cong\delta$. Ker je
še $BD_1\cong AD$ in $BD_2\cong CD$, obstaja takšna točka $E$,  da
sta štirikotnika $D_1ED_2B$ in $ABCD$ skladna. Torej se okrog točke
$B$ stikajo štirje štirikotniki, ki so vsi skladni štirikotniku
$ABCD$. Postopek ‘‘prekrivanja’’ ravnine lahko nadaljujemo, če
uporabljamo središčne simetrije glede na središča stranic
novonastalih štirikotnikov.
 \kdokaz

 Torej obstajajo tlakovanja ravnine s poljubnim trikotnikom in poljubnim
  štirikotnikom. Jasno je, da za poljubni petkotnik, šestkotnik,...
  ta lastnost ne velja.
Za pravilni šestkotnik obstaja pravilno tlakovanje $(6,3)$. Če v
prejšnjem izreku sestavimo po dva ustrezna sosednja štirikotnika,
dobimo prekrivanje ravnine s skladnimi šestkotniki, ki niso nujno
pravilni, so pa vedno središčno simetrični.


 %_______________________________________________________________________________
 \poglavje{Sets of Points in a Plane. Sylvester's Problem}
 \label{odd3Silvester}

V tem razdelku bomo raziskovali probleme, ki so povezani z množicami
točk v ravnini in s premicami, ki jih te točke določajo. Na začetku bomo
obravnavali nekaj posledic prvih dveh skupin aksiomov (incidence
in urejenosti).
 Najprej bomo definirali nove pojme. Naj bo $\mathfrak{T}$ množica
 $n$ ($n>2$) točk v ravnini. S $\mathcal{P}(\mathfrak{T})$ označimo
 množico vseh premic, od katerih gre vsaka skozi vsaj dve točki iz
 množice $\mathfrak{T}$ (Figure
\ref{sl.skl.3.10.1.pic}). Ker množica $\mathfrak{T}$ vsebuje vsaj
dve točki,
 iz aksiomov incidence sledi, da je
 množica $\mathcal{P}(\mathfrak{T})$ neprazna. Zastavlja se
 vprašanje, koliko  premic je v množici
 $\mathcal{P}(\mathfrak{T})$.

\begin{figure}[!htb]
\centering
\input{sl.skl.3.10.1.pic}
\caption{} \label{sl.skl.3.10.1.pic}
\end{figure}



        \bizrek \label{stevPremic}
        Let $\mathfrak{T}$ be a set of $n$ ($n>2$) points in the plane such that
        no three of them are collinear. Then the number of lines of the set $\mathcal{P}(\mathfrak{T})$ is equal to
         $$\frac{n(n-1)}{2}.$$
         \eizrek

\begin{figure}[!htb]
\centering
\input{sl.skl.3.10.2.pic}
\caption{} \label{sl.skl.3.10.2.pic}
\end{figure}

\textbf{\textit{Proof.}} Skozi vsako od $n$ točk iz množice
$\mathfrak{T}$ poteka natanko $n -1$ premic iz množice
$\mathcal{P}(\mathfrak{T})$. Ker na ta način vsako premico štejemo
dvakrat (Figure \ref{sl.skl.3.10.2.pic}), moramo še deliti z 2. Torej je v množici
$\mathcal{P}(\mathfrak{T})$ ravno $\frac{n(n-1)}{2}$ premic.
 \kdokaz

Prejšnji izrek lahko rešimo tudi na naslednji način: skozi prvo
točko gre $n -1$ premic, skozi drugo $n - 2$ premic (ena manj, ker
premice, ki je določena s tema dvema točkama, ne štejemo dvakrat), $n
- 3$ premic skozi tretjo točko in tako naprej vse do ene premice
skozi predzadnjo točko. To je skupaj  $(n -1) + (n - 2) +\cdots+1$
premic. Seveda je to spet enako $\frac{n(n-1)}{2}$. Če vzamemo
$n-1= k$, dobimo znano formulo za vsoto prvih $k$ naravnih števil:
 $$1+ 2+ \cdots + k=\frac{k(k+1)}{2}.$$

 Iz prejšnjega izreka lahko izpeljemo formulo za število diagonal
 poljubnega $n$-kotnika.



            \bizrek
            If $D_n$ is the number of diagonals of an arbitrary $n$-gon,
            then
             $$D_n=\frac{n( n-3)}{2}.$$
            \eizrek

\begin{figure}[!htb]
\centering
\input{sl.skl.3.10.3.pic}
\caption{} \label{sl.skl.3.10.3.pic}
\end{figure}

\textbf{\textit{Proof.}} Označimo s $\mathfrak{O}$ množico vseh
oglišč poljubnega $n$-kotnika.
  Število diagonal je enako številu vseh premic
iz množice $\mathcal{P}(\mathfrak{O})$ (izreka \ref{stevPremic})
zmanjšano za število njegovih stranic (Figure
\ref{sl.skl.3.10.3.pic}). Torej:
 $$D_n=\frac{n(n- 1)}{2}-n=\frac{n^2-3n}{2}=\frac{n(n-3)}{2},$$ kar je bilo treba dokazati. \kdokaz

Omenimo še, da bi lahko formulo za število diagonal $n$-kotnika
izpeljali tudi direktno - s podobno obravnavo kot v dokazu izreka
\ref{stevPremic}. Iz vsakega od $n$ oglišč $n$-kotnika lahko
narišemo $n - 3$ diagonal. Na ta način vsako diagonalo štejemo
dvakrat, zato moramo še deliti z 2 in dobimo prejšnjo formulo.

Izrek \ref{stevPremic} se je nanašal na število premic, ki so
določene z množico točk v ravnini, ki so v takšni legi, da
nobene tri niso kolinearne. V naslednjem primeru bomo preverili, kaj se
zgodi, če dodamo nove pogoje.



            \bzgled
            Let $\mathfrak{T}$ be a set of $n$ ($n > 2$) points in the plane which are
            in such a position that $m$ ($m < n$) of them lies on the same line, but otherwise
            no other three points are collinear. What is the number of lines
            in the set $\mathcal{P}(\mathfrak{T})$?
            \ezgled

\begin{figure}[!htb]
\centering
\input{sl.skl.3.10.4.pic}
\caption{} \label{sl.skl.3.10.4.pic}
\end{figure}

\textbf{\textit{Solution.}}
 (Figure \ref{sl.skl.3.10.4.pic})

 Brez dodatnega pogoja bi v množici $\mathcal{P}(\mathfrak{T})$
  obstajalo  $\frac{n(n -1)}{2}$ premic (izrek \ref{stevPremic}).
  Dodatni pogoj v nalogi  to
število zmanjša, in sicer za $1$ manj kot je število premic, ki jih določa
množica $m$ točk v splošni legi. To je zaradi tega, ker bi sicer
te premice šteli večkrat. Torej je število premic množice
$\mathcal{P}(\mathfrak{T})$  enako:
$$\frac{(n-1)}{2}-\frac{(m -1)}{2}+1.$$
 \kdokaz

Naslednji enostaven primer bo uvod v zelo zanimiv problem odnosa
množice točk $\mathfrak{T}$ in množice vseh premic
$\mathcal{P}(\mathfrak{T})$, ki jo ta množica točk določa.


            \bzgled
            Construct nine points lying on ten lines
            in such a way, that each of those ten lines contains exactly three of these nine points\footnote{\index{Newton, I.}
            \textit{I. Newton} (1642--1727), znani angleški
            matematik
            in fizik, ki je zastavil ta problem v obliki:
            ‘‘How can you plant 9 trees in a garden with 10 rows and each row containing exactly 3 trees?’’}.
            \ezgled

\begin{figure}[!htb]
\centering
\input{sl.skl.3.10.5.pic}
\caption{} \label{sl.skl.3.10.5.pic}
\end{figure}

\textbf{\textit{Proof.}} Ena od možnosti je naslednja. Naj bo $ABCD$
poljubni pravokotnik, $P$ in $Q$ središči stranic $AB$ in $CD$, $S$
presečišče diagonal, $K$ presečišče premic $AP$ in $DQ$ ter $L$
presečišče premic $BP$ in $CQ$ (Figure \ref{sl.skl.3.10.5.pic}). Ker
je $ABCD$ pravokotnik, so točke $P$, $Q$ in $S$  kolinearne. Prav tako
so kolinearne tudi
 točke $K$, $L$ in $S$ (točki $K$ in $L$ sta središči pravokotnikov
  $AQPD$ in $QBCP$). Torej imamo devet točk $A$, $B$, $C$, $D$, $P$,
  $Q$, $S$, $K$ in $L$, od katerih po tri ležijo na vsaki od desetih premic $AB$,
   $CD$, $PQ$, $KL$, $AC$, $BD$,
$PA$, $PB$, $QC$ in $QD$.
 \kdokaz

 Če v prejšnjem zgledu množico devetih točk označimo s $\mathfrak{T}$, vidimo,
 da množica desetih premic ni množica $\mathcal{P}(\mathfrak{T})$.
  V množici $\mathcal{P}(\mathfrak{T})$ bi namreč imeli  šestnajst
  premic - naših deset in še dodatne premice $AD$, $BC$ $DL$, $AL$, $CK$ in $BK$.
  Toda vsaka od teh šestih premic vsebuje
le dve točki. Za njih torej ni izpolnjen pogoj, da vsebujejo
natanko tri točke začetne množice. Sedaj je logično zastaviti
naslednje vprašanje: Ali je v ravnini sploh mogoče postaviti končno
množico nekolinearnih točk $\mathfrak{T}$ tako, da vsaka premica iz množice
$\mathcal{P}(\mathfrak{T})$ vsebuje natanko tri točke iz množice
$\mathfrak{T}$? Jasno je namreč, da je to možno, če zahtevamo,
da vsaka premica iz $\mathcal{P}(\mathfrak{T})$ vsebuje natanko dve
točki iz $\mathfrak{T}$. Najbolj enostaven primer za to so oglišča
trikotnika in njegove nosilke, ali pa poljubna množica točk iz
izreka \ref{stevPremic}. Že omenjeni problem za tri točke ni tako
enostaven, odgovor pa bomo poiskali v nadaljevanju. Omenimo, da je
negativen. Celo več - odgovor je negativen tudi v primeru,
ko zahtevamo, da vsaka premica iz $\mathcal{P}(\mathfrak{T})$ vsebuje
vsaj tri točke iz  $\mathfrak{T}$. Najprej pa dokažimo  eno lemo
(pomožno trditev).



            \bizrek \label{SylvesterLema}
           Let $\mathfrak{T}$ be a finite set of points in the plane which do not all lie on the same line
            and $\mathcal{P}=\mathcal{P}(\mathfrak{T})$. If $T_0\in \mathfrak{T}$ and $p_0\in \mathcal{P}$ ($T_0\notin p_0$)
             are such that they determine the minimum distance, i.e.
            $$(\forall T\in \mathfrak{T})(\forall p\in \mathcal{P})
              ( T\notin p\Rightarrow d(T,p)\geq d(T_0,p_0)),$$
            then the line $p_0$ contains exactly two points from the set $\mathfrak{T}$.
            \eizrek

\begin{figure}[!htb]
\centering
\input{sl.skl.3.10.6.pic}
\caption{} \label{sl.skl.3.10.6.pic}
\end{figure}

\textbf{\textit{Proof.}} Predpostavimo nasprotno. Naj premica $p_0$
vsebuje vsaj tri različne točke $A$, $B$ in $C$ iz množice
$\mathfrak{T}$ (Figure \ref{sl.skl.3.10.6.pic}). Označimo s $T'_0$ pravokotno projekcijo točke $T_0$
na premici $p_0$. Če se točka $T'_0$ razlikuje od točk $A$, $B$ in
$C$, sta vsaj dve od teh treh točk (naj bosta to $B$ in $C$) na premici $p_0$ na isti
strani točke $T'_0$. Brez škode za splošnost naj bo
$\mathcal{B}(T'_0,B,C)$. Ker $T_0,C \in \mathfrak{T}$, potem tudi premica
$q= CT_0$  ($q \neq p_0$, ker $T_0\notin p_0$) pripada množici
$\mathcal{P}$. Naj bo $B'$ pravokotna projekcija točke  $B$ na
premici $q$. Ni težko dokazati, da v tem primeru velja:
 $$d(B,q)=|BB'|<T_0T'_0=d(T_0,p_0).$$
 Zadnja relacija je v
nasprotju s predpostavko, zato premica $p_0$ vsebuje natanko dve
točki.

Če  $T_0$ leži na eni od točk $A$, $B$ ali $C$, je dokaz
 podoben.
 \kdokaz

 Sedaj bomo dokazali napovedan izrek.


            \bizrek
            Let $\mathfrak{T}$ be a finite set of points in the plane which do not all lie on the same line.
            Then there is a line that contains exactly two points from the set
            $\mathfrak{T}$
            \index{problem!Sylvestrov}
             (\textit{Sylvester's\footnote{\index{Sylvester, J. J.} \index{Karamata, J.}
            \index{Erdös, P.} \index{Gallai, T.} \index{Kelly, L. M.}
            \index{Coxeter, H. S. M.}
             Angleški matematik \textit{J. J. Sylvester} (1814--1897)
              je zastavil ta problem že leta 1893,
              ki takrat ni bil rešen in so nanj pozabili.
                Šele čez štirideset let (leta 1933) sta sta ga ponovno ‘‘obudila’’ Madžarski matematik
            \textit{P. Erdös} (1913--1996) in srbski matematik \textit{J.
            Karamata} (1902--1967), madžarski matematik
            \textit{T. Gallai} (1912–-1992) pa ga je istega leta rešil. Nato je bilo
            objavljenih več različnih dokazov rešitve tega problema - najbolj eleganten
            iz leta 1948, ki ga je podal ameriški matematik \textit{L. M. Kelly}
            (1914-–2002), je prikazan tukaj. Leta 1961 je veliki kanadski
            geometer \textit{H. S. M. Coxeter} (1907--2003) dokazal to
            trditev brez uporabe skladnosti - le kot posledico aksiomov
            incidence in urejenosti.} problem}).
            \eizrek

\begin{figure}[!htb]
\centering
\input{sl.skl.3.10.7.pic}
\caption{} \label{sl.skl.3.10.7.pic}
\end{figure}

\textbf{\textit{Proof.}}
Naj bo $\mathcal{P}=\mathcal{P}(\mathfrak{T})$. Ker je množica
$\mathfrak{T}$ končna,  je končna tudi množica $\mathcal{P}$, nato
pa tudi množica vseh razdalj (Figure \ref{sl.skl.3.10.7.pic}):
$$\mathcal{D} = \{d(T,p);\hspace*{1mm}
T\in \mathfrak{T},\hspace*{1mm} p\in \mathcal{P},\hspace*{1mm} T\notin p\}.$$
 Ker je $\mathcal{D}$ končna množica pozitivnih realnih števil, ima svoj
minimalni element $d(T_0,p_0)$ (nobena razdalja iz
$\mathcal{D}$ ni manjša), ki se doseže za neko točko $T_0\in
\mathfrak{T}$ in neko premico $p_0\in \mathcal{P}$. Iz definicije
množice $\mathcal{D}$ sledi $T_0\notin p_0$. Po prejšnji lemi
\ref{SylvesterLema} ležita na premici $p_0$  natanko dve točki iz
množice $\mathfrak{T}$.
 \kdokaz

 Ta razdelek bomo končali še z dvema zanimivima primeroma.



            \bizrek
            Let $\mathfrak{T}$ be a finite set of points,
            such that distances between two points of this set are all different.
            If we connect each point with a line segment to its nearest point, then none of the points will be
            directly connected to more than five points from this set\footnote{Poljski matematik, astronom in fizik
              \index{Steinhaus, H.}\textit{H. Steinhaus} (1887--1972)
              je ta problem zapisal v obliki:
              ‘‘Vsako mesto na zemljevidu Evrope povežemo... ’’}.
            \eizrek

\begin{figure}[!htb]
\centering
\input{sl.skl.3.10.8.pic}
\caption{} \label{sl.skl.3.10.8.pic}
\end{figure}

\textbf{\textit{Proof.}}
 Najprej ugotovimo, da sta lahko dve poljubni točki $X$ in $Y$  povezani
 v eni od treh različnih
možnosti (Figure \ref{sl.skl.3.10.8.pic}):

 \textit{1)} točki $X$ je najbližja točka $Y$, ne pa obratno;

 \textit{2)} točki
$Y$ je najbližja točka $X$, ne pa obratno;

\textit{3)} točki $X$ je najbližja točka $Y$ in obratno - točki $Y$
je najbližja točka $X$.

Vsaki  točki je najbližja samo ena točka, toda ta je
lahko povezana z več točkami,  katerim je  najbližja. Potrebno
je dokazati, da takšnih točk, s katerimi je povezana, ni več kot
pet. Predpostavimo nasprotno. Naj bo $P$ točka dane množice
$\mathfrak{T}$ in je z daljico povezana z vsaj šestimi točkami $A$,
$B$, $C$, $D$, $E$ in $F$. Brez škode za splošnost lahko
predpostavimo, da so tako razporejene (oz. lahko jih tako označimo),
da bo $ABCDEF$ šestkotnik (Figure \ref{sl.skl.3.10.8.pic}). Ena izmed
točk $A$, $B$, $C$, $D$, $E$ in $F$ je najbližja točki $P$, naj bo
to točka $A$. Torej velja: $PA < PB, PC, PD, PE, PF$. Ker so tudi
točke $B$, $C$, $D$, $E$ in $F$ povezane s točko $P$, pomeni da je
točka $P$ najbližja vsaki od njih (ne pa obratno). Na osnovi tega je
najprej $BA > BP > PA$, zato je $\angle APB$ največji kot v
trikotniku $APB$ in je zaradi tega večji od $60^0$. Podobno je $CB >
BP,BC$, zato je tudi $\angle BPC > 60^0$. Enako bi veljalo tudi za
kote $CPD$, $DPE$, $EPF$ in $EPA$, kar pa ni mogoče, saj je njihova
vsota vedno enaka ali celo večja od $360^0$, ne glede na to, ali je
$P$ notranja ali zunanja točka šestkotnika $ABCDEF$. Torej točka $P$
ni povezana z več kot s petimi točkami.
 \kdokaz


            \bzgled
            What is the maximum number of regions in the plane that can be
            divided by $n$ lines\footnote{Ta problem je rešil švicarski geometer
            \index{Steiner, J.} \textit{J. Steiner} (1796--1863).}?
            \ezgled

\begin{figure}[!htb]
\centering
\input{sl.skl.3.10.9.pic}
\caption{} \label{sl.skl.3.10.9.pic}
\end{figure}

\textbf{\textit{Solution.}}  Opisali bomo postopek načrtovanja takšnih
$n$ premic (Figure \ref{sl.skl.3.10.9.pic}). Če je $n = 1$, oz. če
gre le za eno premico, je ravnina razdeljena na dve območji. Dve
premici, če nista vzporedni, razdelita ravnino na štiri območja. Če
dodamo tretjo premico $p_3$, ki z njima ni vzporedna in ne gre skozi
njuno skupno točko,  seka začetni dve premici v dveh točkah. Ti
dve točki delita premico $p_3$ na tri dele, vsak od njih pa je v
enem od treh od prejšnjih štirih območij ravnine. Torej s premico
$p_3$ dobimo še tri nove dele ravnine, torej skupaj sedem. Postopek
nadaljujemo. Če $n -1$ premic deli ravnino na $k$ delov, z
dodajanjem $n$-te premice $p_n$ (ki ni vzporednica nobene od
prejšnjih $n -1$ premic in ne vsebuje nobenega njihovega presečišča),
dobimo najprej $n -1$ presečišč na tej premici, nato $n$ njenih
delov oz. $n$ novih območij dane ravnine. Torej je največje možno
število območij za $n$ premic enako:
\begin{eqnarray*}
2+2+3+\cdots+n&=&1+1+2+3+\cdots+n=\\
&=&1+\frac{n(n+1)}{2}=\\&=&\frac{n^2+n+2}{2}.
\end{eqnarray*}
Formalni dokaz tega dejstva bi lahko izpeljali z matematično indukcijo.
 \kdokaz


 %_______________________________________________________________________________
 \poglavje{Helly's Theorem}
\label{odd3Helly}

Naslednji pomemben izrek je posledica le prvih dveh skupin aksiomov
oz. aksiomov incidence in aksiomov urejenosti. Uporabili ga bomo
tudi pri nalogah, ki so povezane s skladnostjo.



            \bizrek \label{Helly}
            Let $\Phi_1$, $\Phi_2$, ... , $\Phi_n$ ($n \geq 4$) be convex sets in the plane.
            If every three of these sets have a common point, then all $n$ sets have a common point
            \index{izrek!Hellyjev}(Helly's theorem\footnote{Avstrijski
            matematik  \index{Helly, E.} \textit{E. Helly} (1884--1943)
             je odkril to trditev v splošnem primeru $n$-razsežnega
             prostora $\mathbb{E}^n$
              leta 1913, objavil pa šele leta 1923. Alternativna dokaza sta
              medtem podala avstrijski matematik \index{Radon, J. K. A.}
              \textit{J. K. A. Radon} (1887-–1956) leta 1921 in
              madžarski matematik \index{Kőnig, D.}
              \textit{D. Kőnig} (1884–-1944) leta 1922.}).
            \eizrek

\begin{figure}[!htb]
\centering
\hspace*{10mm}
\input{sl.skl.3.11.1.pic}
\caption{} \label{sl.skl.3.11.1.pic}
\end{figure}

\textbf{\textit{Proof.}} Dokaz bomo izvedeli z indukcijo po $n$.

\textit{(A)} Naj bo najprej $n = 4$ in (Figure
\ref{sl.skl.3.11.1.pic}):
\begin{itemize}
  \item $P_4\in \Phi_1 \cap \Phi_2 \cap \Phi_3$,
  \item $P_3\in \Phi_1 \cap \Phi_2 \cap \Phi_4$,
  \item $P_2\in \Phi_1 \cap \Phi_3 \cap \Phi_4$,
  \item $P_1\in \Phi_2 \cap \Phi_3 \cap \Phi_4$.
\end{itemize}
Dokažimo, da obstaja točka, ki leži v vsakem od likov  $\Phi_1$,
$\Phi_2$, $\Phi_3$ in $\Phi_4$. Glede na medsebojno lego
točk $P_1$, $P_2$, $P_3$ in $P_4$ bomo od več možnih primerov
obravnavali samo dva najbolj
splošna (dokaz v ostalih primerih je
podoben).

\textit{1)} Štirikotnik, ki ga določajo točke $P_1$, $P_2$, $P_3$ in
$P_4$, je nekonveksen. V tem primeru je ena od točk $P_1$, $P_2$,
$P_3$ in $P_4$ notranja točka trikotnika, ki ga določajo preostale
tri točke. Brez škode za splošnost naj bo $P_4$ notranja točka
trikotnika $P_1P_2P_3$. Oglišča tega trikotnika ležijo v liku
$\Phi_4$. Ker je $\Phi_4$ konveksni lik, v njemu ležijo tudi vse
stranice in notranje točke trikotnika $P_1P_2P_3$, prav tako tudi
točka $P_4$. V tem primeru je točka $P_4$  skupna točka likov
$\Phi_1$, $\Phi_2$, $\Phi_3$ in $\Phi_4$.

\textit{2)}  Štirikotnik, ki ga določajo točke $P_1$, $P_2$, $P_3$
in $P_4$, je konveksen. Brez škode za splošnost naj bosta njegovi
diagonali $P_1P_2$ in $P_3P_4$. Ker je štirikotnik konveksen, se
njegovi diagonali sekata v neki točki $S$. Dani liki so konveksni,
zato iz $P_1, P_2\in \Phi_3,\Phi_4$ sledi, da diagonala $P_1P_2$
vsa leži v likih $\Phi_3$ in $\Phi_4$. Analogno iz $P_3, P_4\in
\Phi_1,\Phi_2$ sledi, da diagonala $P_3P_4$ vsa leži v likih
$\Phi_1$ in $\Phi_2$. Točka $S$, ki leži na obeh diagonalah
$P_1P_2$ in $P_3P_4$, leži v vseh štirih likih $\Phi_1$, $\Phi_2$,
$\Phi_3$ in $\Phi_4$.

S tem smo dokazali, da trditev velja za $n=4$.

\textit{(B)} Predpostavimo sedaj, da trditev velja za $n = k$ ($k\in
\mathbb{N}$ in $k>4$).
 Dokažimo, da trditev velja tudi za
$n = k +1$. Naj bodo $\Phi_1$, $\Phi_2$, $\ldots$ , $\Phi_{k-1}$,
$\Phi_k$ in $\Phi_{k+1}$ takšni liki, da ima vsaka poljubna trojica teh likov
vsaj eno skupno točko. Naj bo $\Phi'=\Phi_k\cap\Phi_{k+1}$. Dokažimo
najprej, da ima vsaka trojica likov $\Phi_1$, $\Phi_2$ ,$\ldots$,
$\Phi_{k-1}$, $\Phi'$ skupno točko. Za trojice likov izmed
$\Phi_1$, $\Phi_2$, $\ldots$, $\Phi_{k-1}$ je to izpolnjeno že po
predpostavki. Brez škode za splošnost je dovolj, če dokazažemo, da
imajo liki $\Phi_1$, $\Phi_2$ in $\Phi'=\Phi_k\cap\Phi_{k+1}$
skupno točko. To pa velja  (na osnovi dokazanega primera za $n = 4$),
ker ima vsaka trojica izmed likov $\Phi_1$, $\Phi_2$,  $\Phi_k$ in
$\Phi_{k+1}$  skupno točko. Iz indukcijske predpostavke
(za $n=k$) sledi, da imajo liki $\Phi_1$, $\Phi_2$, $\ldots$,
$\Phi_{k-1}$, $\Phi'$  skupno točko, ta točka pa hkrati leži v
vsakem od likov $\Phi_1$, $\Phi_2$, $\ldots$,  $\Phi_{k-1}$,
$\Phi_k$, $\Phi_{k+1}$.
 \kdokaz

V nadaljevanju bomo obravnavali nekaj posledic Hellyjevega izreka.



            \bzgled
           Let $\alpha_1$, $\alpha_2$, $\cdots$, $\alpha_n$
            ($n > 3$) be half-planes covering a plane $\alpha$.
            Prove that there are three of these half-planes that also cover the plane $\alpha$.
            \ezgled


\textbf{\textit{Proof.}} Naj bodo $\beta_1$, $\beta_2$, $\cdots$,
$\beta_n$ odprte polravnine, ki so določene s polravninami
$\alpha_1$, $\alpha_2$, $\cdots$, $\alpha_n$ kot  komplementarne
polravnine glede na ravnino $\alpha$ oz.
$\beta_k=\alpha\setminus\alpha_k$, $k\in\{1,2,\ldots,n\}$.
Za vsako točko $X$ ravnine $\alpha$ in vsak
$k\in\{1,2,\ldots,n\}$ velja ekvivalenca:
 $$X\in \alpha_k \Leftrightarrow X\notin \beta_k.$$
Predpostavimo nasprotno, da nobena trojica polravnin $\alpha_1$,
$\alpha_2$, $\cdots$, $\alpha_n$ ne prekriva ravnine $\alpha$. To
 pomeni, da za vsako trojico izmed njih obstaja točka v ravnini $\alpha$,
ki ne leži na nobeni od njih, oziroma da za vsako trojico izmed polravnin
$\beta_1$, $\beta_2$, $\cdots$, $\beta_n$ obstaja točka ravnine
$\alpha$, ki leži na vsaki od njih. Ker so polravnine konveksni
liki, iz Hellyjevega izreka \ref{Helly} sledi, da obstaja točka
$X$, ki leži  na vsaki od polravnin $\beta_1$, $\beta_2$, $\cdots$,
$\beta_n$. Ta točka torej leži v  ravnini $\alpha$, ne leži pa v
nobeni od polravnin $\alpha_1$, $\alpha_2$, $\cdots$, $\alpha_n$,
kar je v nasprotju z osnovno predpostavko, da polravnine $\alpha_1$,
$\alpha_2$, $\cdots$, $\alpha_n$ prekrivajo ravnino $\alpha$.
 Torej  obstaja vsaj ena trojica izmed
polravnin $\alpha_1$, $\alpha_2$, $\cdots$, $\alpha_n$, ki
prekrivajo ravnino $\alpha$.
 \kdokaz



        \bzgled \label{lemaJung}
        If for every three of $n$ ($n > 3$) points of a plane
        there is such a circle with radius $r$ containing these three points, then it exists
        a circle of equal radius containing all these $n$ points.
        \ezgled

\begin{figure}[!htb]
\centering
\input{sl.skl.3.11.2.pic}
\caption{} \label{sl.skl.3.11.2.pic}
\end{figure}

\textbf{\textit{Proof.}} (Figure \ref{sl.skl.3.11.2.pic})

 Naj bodo $A_1$, $A_2$,$\ldots$, $A_n$ točke z danimi lastnostmi. S
$\mathcal{K}_i$ ($i\in\{1,2,\ldots,n\}$) označimo kroge s središči $A_i$ in s
polmerom $r$. Naj bodo $A_p$, $A_q$ in $A_l$ poljubne točke iz
množice $\{A_1, A_2,\ldots, A_n\}$. Po predpostavki obstaja krog s
polmerom $r$, ki te tri točke vsebuje. Označimo središče tega kroga
z $O$. Iz tega sledi $|OA_p|, |OA_q|, |OA_l|\leq r$, kar pomeni, da
točka $O$ leži v vsakem od krogov $\mathcal{K}_p$, $\mathcal{K}_q$ in $\mathcal{K}_l$. Torej,
vsaki trije izmed krogov $\mathcal{K}_1$, $\mathcal{K}_2$,$\ldots$, $\mathcal{K}_n$ imajo vsaj eno
skupno točko. Ker so krogi konveksni liki (izrek \ref{KrogKonv}), po
Hellyjevem izreku obstaja točka $S$, ki leži v vsakem izmed krogov
$\mathcal{K}_1$, $\mathcal{K}_2$,$\ldots$, $\mathcal{K}_n$. Iz tega sledi, da je $\mathcal{K}(S, r)$ iskani
krog, saj velja $|SA_1|, |SA_2|,\ldots |SA_n|\leq r$.
 \kdokaz

 Zanimiva posledica zadnje trditve \ref{lemaJung} bo podana v razdelku
 \ref{odd7Pitagora} (izrek \ref{Jung}).

%________________________________________________________________________________
\naloge{Exercises}

\begin{enumerate}

 \item Naj bo $S$ točka, ki leži v kotu $pOq$, točki $A$ in $B$ pa
 pravokotni projekciji  točke $S$ na krakih $p$ in $q$ tega
kota. Dokaži, da je $SA\cong SB$ natanko tedaj, ko je premica
$OS$ simetrala kota $pOq$.

\item Dokaži, da je vsota diagonal konveksnega štirikotnika večja od
vsote dveh njegovih nasprotnih stranic.

  \item Dokaži, da je v vsakemu trikotniku
  največ ena stranica krajša od pripadajoče višine.

  \item Naj bo $AA_1$ težiščnica trikotnika $ABC$. Dokaži,
   da je od dveh kotov, ki jih težiščnica $AA_1$ določa s
stranicama $AB$ in $AC$, večji tisti, ki ga težiščnica določa s
krajšo stranico.

  \item Naj  bosta $BB_1$ in $CC_1$ težiščnici trikotnika $ABC$ ter
  $AB<AC$.
  Dokaži, da je $BB_1<CC_1$.

  \item Naj bodo $a$, $b$ in $c$ stranice, $t_a$, $t_b$ in $t_c$
   ustrezne težiščnice ter $s$ polobseg poljubnega trikotnika.
Dokaži, da velja:
 \begin{enumerate}
  \item $s <  t_a  + t_b +  t_c  < 2s$;
  \item $t_a + t_b + t_c  >  \frac{3}{4}(a + b + c)$.
 \end{enumerate}

\item Naj bo premica $p$ mimobežnica krožnice $k$. Dokaži, da so
vse točke te krožnice na istem bregu premice $p$.

\item Če krožnica $k$ leži v nekem konveksnem liku $\Phi$, potem tudi
krog, ki je določen s to krožnico, leži v tem liku. Dokaži.

\item Naj bosta $p$ in $q$ različni tangenti krožnice $k$, ki se jo dotikata v
točkah $P$ in $Q$. Dokaži ekvivalenco: $p \parallel q$ natanko
tedaj, ko je $AB$ premer krožnice $k$.

\item Če je $AB$ tetiva krožnice $k$, potem je presek premice $AB$ in
kroga, ki ga krožnica $k$ določa, enak tej tetivi. Dokaži.

\item Naj bo $S'$ pravokotna projekcija središča $S$ krožnice $k$ na
premici $p$. Dokaži, da je $S'$ zunanja točka te krožnice
natanko tedaj, ko premica $p$ krožnice ne seka.

\item Naj bo $V$ višinska točka trikotnika $ABC$, pri katerem velja
$CV \cong AB$. Določi velikost kota $ACB$.

\item Naj bo $CC'$ višina pravokotnega trikotnika $ABC$ ($\angle ACB =
90^0$). Če sta $O$ in $S$ središči včrtanih krožnic trikotnikov
$ACC'$ in $BCC'$, je simetrala notranjega kota $ACB$
pravokotna na premici $OS$. Dokaži.

\item Naj bo $ABC$ trikotnik, v katerem je $\angle ABC = 15^0$ in
$\angle ACB = 30^0$. Naj bo $D$ takšna točka stranice $BC$, da je
 $\angle BAD=90^0$. Dokaži, da je $BD = 2AC$.

\item Dokaži, da obstaja takšen petkotnik, da je mogoče prekriti ravnino
 s takšnimi petkotniki, ki so mu skladni.

\item Dokaži, da obstaja takšen desetkotnik, da je mogoče prekriti ravnino
 s takšnimi desetkotniki, ki so mu skladni.

\item V neki ravnini je vsaka točka pobarvana rdeče ali črno.
        Dokaži, da obstaja pravilni trikotnik, ki ima vsa
        oglišča iste barve.

\item Naj bodo $l_1,l_2,\ldots, l_n$ ($n > 3$) loki, ki vsi ležijo na isti
krožnici. Središčni kot vsakega loka je kvečjemu enak $180^0$.
 Dokaži, da obstaja točka, ki leži na vsakem  loku,
 če imajo vsaki trije loki vsaj eno skupno točko.

%drugi del

\item
Naj bosta $p$ in $q$ pravokotnici, ki  se sekata v točki $A$. Če
je $B, B'\in p$, $C, C'\in q$, $AB\cong AC'$, $AB'\cong AC$,
$\mathcal{B}(B,A,B')$ in $\mathcal{B}(C,A,C')$, potem
pravokotnica na premico $BC$ skozi točko $A$ poteka skozi središče
daljice $B'C'$. Dokaži.

\item
Dokaži, da se simetrale notranjih kotov pravokotnika, ki ni
kvadrat, sekajo v točkah, ki so oglišča kvadrata.

\item
 Dokaži, da se simetrale notranjih kotov paralelograma, ki ni
romb, sekajo v točkah, ki so oglišča pravokotnika. Dokaži še, da so diagonale
tega pravokotnika vzporedne s stranicami paralelograma in so
enake razliki sosednjih stranic tega paralelograma.

\item
Dokaži, da sta simetrali dveh sokotov med seboj pravokotni.

\item Naj bosta $B'$ in $C'$ nožišči višin iz oglišč $B$ in $C$ trikotnika
$ABC$. Dokaži ekvivalenco $AB\cong AC \Leftrightarrow BB'\cong
CC'$.

\item Dokaži, da je trikotnik pravilen,
če središče trikotniku očrtane krožnice in njegova višinska točka sovpadata.
Ali podobna trditev velja za poljubni dve značilni
točki tega trikotnika?

\item Dokaži, da sta ostrokotna trikotnika $ABC$ in $A'B'C'$ skladna natanko
tedaj, ko imata skladni višini $CD$ in $C'D'$, stranici $AB$ in
$A'B'$ ter kota $ACD$ in $A'C'D'$.

\item Če je $ABCD$ pravokotnik ter $AQB$ in $APD$ pravilna
  trikotnika z enako orientacijo, je daljica $PQ$
skladna z diagonalo tega pravokotnika. Dokaži.

\item Naj bosta $BB'$ in $CC'$ višini trikotnika $ABC$ ($AC>AB$) ter
 $D$ takšna točka poltraka $AB$, da velja $AD\cong AC$. Točka
$E$ je presečišče premice $BB'$ s premico, ki poteka skozi točko $D$ in je
vzporedna s premico $AC$. Dokaži, da je $BE=CC'-BB'$.

\item Naj bo $ABCD$ konveksni štirikotnik, pri katerem velja
 $AB\cong BC\cong CD$ in $AC\perp BD$. Dokaži, da je $ABCD$
 romb.

\item Naj bo $BC$ osnovnica enakokrakega trikotnika $ABC$. Če sta $K$ in
$L$ takšni točki, da je $\mathcal{B}(A,K,B)$, $\mathcal{B}(A,C,L)$ in $KB\cong LC$, potem
središče daljice $KL$ leži na osnovnici $BC$. Dokaži.

\item Naj bo $S$ središče trikotniku $ABC$ včrtane krožnice.
Premica, ki poteka skozi točko $S$ in je vzporedna s stranico $BC$
tega trikotnika, seka stranici $AB$ in $AC$ po vrsti v točkah
$M$ in $N$. Dokaži, da je $BM+NC=NM$.

\item Naj bo $ABCDEFG$ konveksni sedemkotnik. Izračunaj vsoto
konveksnih kotov, ki jih določa lomljenka $ACEGBDFA$.

\item Dokaži, da so središča stranic in nožišče poljubne višine trikotnika,
v katerem nobeni dve stranici nista skladni, oglišča
enakokrakega trapeza.

 \item Naj bo $ABC$ pravokotni trikotnik s pravim kotom pri oglišču $C$.
Točki $E$ in $F$ naj bosta presečišči simetral notranjih kotov pri
ogliščih $A$ in $B$ z nasprotnima katetama,  $K$ in $L$ pa
pravokotni projekciji točk $E$ in $F$ na hipotenuzi tega
trikotnika. Dokaži, da je $\angle LCK=45^0$.


\item Naj bo $M$ središče stranice $CD$ kvadrata $ABCD$ in $P$ takšna točka
 diagonale $AC$, da velja $3AP=PC$. Dokaži, da je $\angle BPM$
pravi kot.

 \item Naj bodo $P$, $Q$ in $R$ središča stranic $AB$, $BC$ in $CD$
  paralelograma $ABCD$. Premici $DP$ in $BR$ naj sekata daljico
$AQ$ v točkah $K$ in $L$. Dokaži, da je $KL= \frac{2}{5} AQ$.

 \item  Naj bo $D$ središče hipotenuze $AB$ pravokotnega
trikotnika $ABC$ ($AC>BC$). Točki $E$ in $F$ naj bosta presečišči
poltrakov $CA$ in $CB$ s premico, ki poteka skozi $D$ in je pravokotna
na premico $CD$. Točka $M$ naj bo središče daljice $EF$. Dokaži, da
je $CM\perp AB$.

\item Naj bosta $A_1$ in $C_1$ središči stranic $BC$ in $AB$ trikotnika $ABC$.
 Simetrala notranjega kota pri oglišču $A$ seka daljico
$A_1C_1$ v točki $P$. Dokaži, da je $\angle APB$  pravi kot.

 \item Naj bosta $P$ in $Q$ takšni točki stranic $BC$ in $CD$ kvadrata $ABCD$,
  da je premica $PA$ simetrala kota $BPQ$. Določi velikost kota
  $PAQ$.

\item Dokaži, da  središče očrtane krožnice leži najbliže najdaljši stranici
trikotnika.

 \item Dokaži, da je središče včrtane krožnice najbliže oglišču, ki je vrh
največjega notranjega kota trikotnika.

\item Naj bo $ABCD$ konveksen štirikotnik. Določi točko $P$, tako da
bo vsota $AP+BP+CP+DP$ minimalna.

 \item Diagonali $AC$ in $BD$ enakokrakega trapeza $ABCD$ z osnovnico $AB$
se sekata v točki $O$ in velja $\angle AOB=60^0$. Točke $P$, $Q$
in $R$ so po vrsti središča daljic $OA$, $OD$ in $BC$. Dokaži, da
je $PQR$ pravilni trikotnik.

\item Naj bo $P$ poljubna notranja točka trikotnika $ABC$, za katero velja
 $\angle PBA\cong \angle PCA$. Točki $M$ in $L$ sta pravokotni
projekciji točke $P$ na stranicah $AB$ in $AC$, točka $N$ pa
središče stranice $BC$. Dokaži, da je $NM\cong
NL$\footnote{Predlog za MMO 1982 (SL 9.)).}.

\item Naj bodo $P$, $Q$ in $R$ središča stranic $BC$, $AC$ in $AB$
 trikotnika $ABC$ ($AB<AC$) in $D$ nožišče višine iz oglišča
$A$. Dokaži, da je $\angle DRP\cong \angle DQP=\angle ABC-\angle ACB$.

\item Naj bo $AD$ simetrala notranjega kota pri oglišču $A$ ($D\in BC$) trikotnika
$ABC$ in $E$ takšna točka stranice $AB$, da velja $\angle
BDE\cong\angle BAC$. Dokaži, da je $DE\cong DC$.

\item Naj bo $O$ središče kvadrata $ABCD$ ter $P$, $Q$ in $R$ točke,
 ki razdelijo njegov obseg na tri enake dele. Dokaži, da se
minimum vsote $|OP|+|OQ|+|OR|$ doseže, kadar je ena od teh točk
središče stranice kvadrata.

\item Dano je končno število premic, ki ravnino razdelijo na območja.
Dokaži, da lahko ravnino pobarvamo z dvema barvama, tako da je
vsako območje pobarvano z eno barvo, sosednji območji pa vedno z
različnima barvama.


\item Načrtaj trikotnik $ABC$, če so dani podatki (glej oznake v razdelku \ref{odd3Stirik}):

 (\textit{a}) $\alpha$, $\beta$, $s$; \hspace*{2mm}
 (\textit{b}) $a-b$, $c$, $\gamma$; \hspace*{2mm}
 (\textit{c}) $a$, $\beta-\gamma$, $b-c$; \hspace*{2mm}

 (\textit{d}) $a$, $\beta-\gamma$, $b+c$; \hspace*{2mm}
 (\textit{e}) $b$, $c$, $v_a$; \hspace*{2mm}
 (\textit{f}) $b$, $v_a$, $v_b$; \hspace*{2mm}

(\textit{g}) $\alpha$, $v_a$, $v_b$; \hspace*{2mm}
 (\textit{h}) $c$, $a+b$, $\gamma$;
 (\textit{i}) $v_a$, $\alpha$, $\beta$; \hspace*{2mm}

 (\textit{j}) $b$, $a+c$, $v_c$; \hspace*{2mm}
 (\textit{k}) $b-c$, $v_b$, $\alpha$; \hspace*{2mm}
 (\textit{l}) $a$, $t_b$, $t_c$;\hspace*{2mm}
 (\textit{m}) $b$, $c$, $t_a$; \hspace*{2mm}

 (\textit{n}) $t_a$, $t_b$, $t_c$; \hspace*{2mm}
(\textit{o}) $c$, $v_a$, $l_a$; \hspace*{2mm}
 (\textit{p}) $c$, $v_a$, $t_b$;
 (\textit{r}) $b$, $l_a$, $\alpha$; \hspace*{2mm}

 (\textit{s}) $v_a$, $v_b$, $t_a$; \hspace*{2mm}
(\textit{t}) $t_a$, $v_b$, $b+c$; \hspace*{2mm}
 (\textit{u}) $a$, $b$, $\alpha-\beta$;

 \item Načrtaj enakokraki trikotnik $ABC$, če so dani:
        \begin{enumerate}
        \item osnovnica ter vsota kraka in višine na osnovnico,
        \item obseg in višina na osnovnico,
        \item obe višini,
        \item kot ob osnovnici in odsek njegove simetrale,
        \item krak in na njem nožišče pripadajoče višine,
        \item krak in pripadajoča višina.
        \end{enumerate}

 \item Načrtaj pravokotni trikotnik $ABC$ s pravim kotom v oglišču $C$, če so dani naslednji podatki:

 (\textit{a}) $\alpha$, $a+b$, \hspace*{2mm}
 (\textit{b}) $\alpha$, $a-b$, \hspace*{2mm}
 (\textit{c}) $a$, $b+c$,

 (\textit{d}) $c$, $a+b$,\hspace*{2mm}
 (\textit{e}) $t_a$, $t_c$, \hspace*{2mm}
 (\textit{f}) $a$, $c-b$,

(\textit{g}) $a+v_c$, $\alpha$, \hspace*{2mm}
 (\textit{h}) $t_c$, $v_c$,\hspace*{2mm}
 (\textit{i}) $a$, $t_a$,\hspace*{2mm}
 (\textit{j}) $v_c$, $l_c$.

 \item Načrtaj pravokotnik $ABCD$, če je dano:
        \begin{enumerate}
        \item diagonala in ena stranica,
        \item diagonala in obseg,
        \item ena stranica in kot, ki ga oklepata diagonali,
        \item obseg in kot, ki ga oklepata diagonali.
        \end{enumerate}

 \item Načrtaj romb $ABCD$, če je dano:
        \begin{enumerate}
        \item stranica in vsota diagonal,
        \item stranica in razlika diagonal,
        \item en kot in vsota diagonal,
        \item en kot in razlika diagonal.
        \end{enumerate}

 \item Načrtaj paralelogram $ABCD$, če je dano:
        \begin{enumerate}
        \item ena stranica in diagonali,
        \item ena stranica in višini,
        \item ena diagonala in višini,
        \item stranica $AB$, kot ob oglišču $A$ in vsota $BC+AC$.
        \end{enumerate}

 \item Načrtaj trapez $ABCD$, če je dano:
        \begin{enumerate}
        \item osnovnici, krak in manjši kot, ki ne leži ob tem kraku,
        \item osnovnici in diagonali,
        \item osnovnici in kota ob daljši osnovnici,
        \item vsota osnovnic, višina in kota ob daljši osnovnici.
        \end{enumerate}

 \item Načrtaj deltoid $ABCD$, če so dani: diagonala $AC$, ki leži na somernici deltoida, $\angle CAD$ in vsota $AD+DC$.


 \item Načrtaj štirikotnik $ABCD$, če je dano:
        \begin{enumerate}
        \item štiri stranice in en kot,
        \item štiri stranice in kot, ki ga oklepata nosilki nasprotnih stranic,
        \item tri stranice in kota ob četrti stranici,
        \item središča treh stranic in daljica, ki je skladna in vzporedna s četrto stranico.
        \end{enumerate}


\end{enumerate}



%%% Do tu pregledala tudi Ana.

% DEL 4 - - - - - - - - - - - - - - - - - - - - - - - - - - - - - - - - - - - - - - -
%________________________________________________________________________________
% SKLADNOST TRIKOTNIKOV IN KROŽNICA
%________________________________________________________________________________

  \del{Congruence and Circle} \label{pogSKK}


Nekatere lastnosti krožnice smo obravnavali že v prejšnjih
dveh poglavjih – določene lastnosti polmera, premera, tetive, odnos
krožnice in premice ter lastnosti tangente na krožnico. Videli smo, da
za vsak trikotnik obstajata  očrtana in včrtana krožnica. Dokazali smo, da za pravilne večkotnike in
nekatere štirikotnike (pravokotnik, kvadrat)
obstaja očrtana krožnica, za nekatere pa tudi včrtana
krožnica. V tem poglavju se bomo poglobili v nadalnje lastnosti
krožnice, ki so posledice skladnosti trikotnikov.

%________________________________________________________________________________
 \poglavje{Two Circles} \label{odd4DveKroz}

 Analogno obravnavi medsebojne lege krožnice in premice,
 ki smo jo izvedli v razdelku \ref{odd3KrozPrem}, bomo v tem
 razdelku podobno naredili za dve krožnici v isti ravnini. V
 nadaljevanju bomo predpostavljali, da sta krožnici, ki ju
 obravnavamo, v isti ravnini.

 Definirajmo najprej nekaj pojmov, ki se nanašajo na dve krožnici.
 Premica, ki jo določata središči dveh krožnic, je
 \index{centrala dveh krožnic}\pojem{centrala} teh dveh krožnic.
 Razdaljo med središčema dveh krožnic imenujemo
 \index{središčna razdalja dveh krožnic}
  \pojem{središčna razdalja dveh krožnic}
 (Figure \ref{sl.skk.4.1.1.pic}).


\begin{figure}[!htb]
\centering
\input{sl.skk.4.1.1.pic}
\caption{} \label{sl.skk.4.1.1.pic}
\end{figure}


  Krožnici v isti ravnini z istim
  središčem (njuna središčna razdalja je enaka $0$)
  imenujemo
\index{krožnici!koncentrični}
 \pojem{koncentrični krožnici}
 (Figure \ref{sl.skk.4.1.1.pic}). Če imata koncentrični krožnici  vsaj
 eno skupno točko, sta krožnici identični (sovpadata). Če je $X$ skupna točka
 koncentričnih krožnic $k_1(S,r_1)$ in $k_2(S,r_2)$, velja
 $|SX|=r_1=r_2$ oz. velja $r_1=r_2$, kar pomeni, da sta krožnici
 identični. Koncentrični krožnici sta bodisi identični bodisi nimata
 skupnih točk. Podobno kot za krožnico in premico se tudi tu zastavlja
 vprašanje, koliko skupnih točk lahko imata različni krožnici in
 kakšna je njuna medsebojna lega. To raziskavo bomo začeli
 z naslednjim izrekom.


            \bizrek
            Two different circles lying in the same plane
            have at most two common points.
            \eizrek


\begin{figure}[!htb]
\centering
\input{sl.skk.4.1.2.pic}
\caption{} \label{sl.skk.4.1.2.pic}
\end{figure}

\textbf{\textit{Proof.}} Predpostavimo nasprotno. Naj bodo $A$, $B$
in $C$ tri različne skupne točke dveh krožnic $k(O,r_1)$ in
$l(S,r_2)$ (Figure \ref{sl.skk.4.1.2.pic}). Te tri točke niso
kolinearne, kar bi pomenilo, da premica $AB$ seka krožnico (npr.
$k$) v treh različnih točkah, kar po izreku \ref{KroznPremPresek} ni
mogoče. Če pa so  $A$, $B$ in $C$ nekolinearne točke, določajo
trikotnik $ABC$, kar po izreku \ref{SredOcrtaneKrozn} pomeni, da je
$O=S$, oz. se obe točki nahajata v presečišču simetral stranic tega
trikotnika. Tudi polmera sta enaka, ker je $r_1=|OA|=|SA|=r_2$, zato
sta krožnici  $k(O,r_1)$ in $l(S,r_2)$ identični - obe predstavljata
trikotniku $ABC$ očrtano krožnico.
 \kdokaz

Torej lahko imata dve različni krožnici v isti ravnini dve
skupni točki, eno skupno točko ali pa nobene skupne točke. V prvem
primeru pravimo, da se \index{krožnici!se sekata} \pojem{krožnici
sekata}, v drugem se \index{krožnici!se dotikata} \pojem{krožnici
dotikata} v njunem \index{dotikališče!dveh krožnic}
\pojem{dotikališču}, v tretjem pa sta \index{krožnici!sta
mimobežni}\pojem{krožnici mimobežni} (Figure \ref{sl.skk.4.1.3.pic}).


\begin{figure}[!htb]
\centering
\input{sl.skk.4.1.3.pic}
\caption{} \label{sl.skk.4.1.3.pic}
\end{figure}


Če se  krožnici ne sekata, je notranjost vsaj ene od teh dveh
krožnic ali v notranjosti ali v zunanjosti druge krožnice. To je
posledica izreka \ref{DedPoslKrozKroz}.

Premica, ki je določena s presečiščema
 dveh krožnic, ki se
sekata, se imenuje \pojem{sekanta, ki je nosilka njune skupne
tetive}. V zvezi s tem dokažimo naslednji izrek.



            \bizrek \label{KroznPresABpravokOS}
            If two circles intersect at two points $A$ and $B$,
            then the line containing the centres of the two circles is perpendicular to the line $AB$.
             \eizrek


\begin{figure}[!htb]
\centering
\input{sl.skk.4.1.4.pic}
\caption{} \label{sl.skk.4.1.4.pic}
\end{figure}

\textbf{\textit{Proof.}} Naj bosta $A$ in $B$ presečišči krožnic
$k_1(S_1,r_1)$ in  $k_2(S_2,r_2)$ (Figure \ref{sl.skk.4.1.4.pic}).
 Ker je $S_1A\cong S_1B\cong r_1$ in $S_2A\cong S_2B\cong r_2$,
 je premica $S_1S_2$  simetrala daljice $AB$ (izrek \ref{simetrala}),
 zato je $S_1S_2\perp
 AB$.
  \kdokaz


 Dokazali bomo, da imata krožnici, ki se dotikata, skupno tangento v
njunem dotikališču.



            \bizrek \label{tangSkupnaDotikKrozn}
            Let $k_1$ and $k_2$ be different circles with centres $S_1$ and
            $S_2$ touching at a point $T$. Then:

            (i) $S_1$, $S_2$ and $T$  are collinear points;

             (ii) the tangent of the circle $k_1$ at the point $T$ is at the same time the tangent of the circle $k_2$ at
            the same point.
            \eizrek



\begin{figure}[!htb]
\centering
\input{sl.skk.4.1.5.pic}
\caption{} \label{sl.skk.4.1.5.pic}
\end{figure}

\textbf{\textit{Proof.}}  (Figure \ref{sl.skk.4.1.5.pic}).

(\textit{i}) Predpostavimo, da točke $S_1$, $S_2$ in $T$ niso
kolinearne. Premica $S_1S_2$ razdeli ravnino, v kateri ležita
krožnici, na dve polravnini. Tisto polravnino, ki vsebuje točko $T$,
označimo s $\pi_1$, drugo pa s $\pi_2$. Iz izreka \ref{izomEnaC'}
sledi, da v polravnini $\pi_2$ obstaja (ena sama) točka $T'$, za
katero je $S_1T'\cong S_1T$ in $S_2T'\cong ST_2$. To bi pomenilo,
da še točka $T'$, ki je različna od točke $T$, leži na krožnicah
$k_1$ in $k_2$, kar ni mogoče. Torej so točke $S_1$, $S_2$ in $T$
kolinearne.

 (\textit{ii}) Po izreku \ref{TangPogoj} je tangenta
krožnice $k_1$ v točki $T$ pravokotna na polmer $S_1T$. Podobno
je tangenta krožnice $k_2$ v isti točki $T$  pravokotna na polmer
$S_2T$. Ker iz (\textit{i}) premici $S_1T$ in $S_2T$ sovpadata,
sta tudi obe pravokotnici oz. tangenti identičnii.
 \kdokaz

Po izreku \ref{tangKrozEnaStr} so vse točke krožnice na isti strani
vsake njene tangente – na tisti strani, kjer je njeno središče. To
pomeni, da sta krožnici $k_1$ in $k_2$ iz prejšnjega izreka bodisi
 na isti strani bodisi na različnih straneh njune skupne tangente.
 Kadar je $B(S_1,T,S_2)$, sta krožnici na različnih straneh
njune skupne tangente  in pravimo, da se krožnici $k_1$
in $k_2$ \pojem{dotikata od zunaj}. Sicer sta krožnici na isti strani
te tangente in pravimo, da se  \pojem{dotikata od znotraj}. V
prvem primeru je notranjost ene od teh dveh krožnic v zunanjosti, v
drugem  pa v notranjosti druge krožnice (Figure
\ref{sl.skk.4.1.6.pic}).

\begin{figure}[!htb]
\centering
\input{sl.skk.4.1.6.pic}
\caption{} \label{sl.skk.4.1.6.pic}
\end{figure}

Ko se  krožnici $k_1(S_1,r_1)$ in $k_2(S_2,r_2)$ dotikata zunaj,
iz prejšnjega izreka sledi, da je $|S_1S_2| = r_1 + r_2$.
Če se krožnici dotikata od znotraj, je $|S_1S_2| = |r_1 - r_2|$.
Jasno je, da velja tudi obratno. Pogoja $|S_1S_2| = r_1 +
r_2$ oz. $|S_1S_2| = |r_1 - r_2|$ sta zadostna, da se krožnici
dotikata od zunaj oz. od znotraj. Na podoben način dobimo tudi ostale
kriterije za medsebojno lego dveh krožnic.

\begin{figure}[!htb]
\centering
\input{sl.skk.4.1.7.pic}
\caption{} \label{sl.skk.4.1.7.pic}
\end{figure}

            \bizrek
            Let $k_1(S_1,r_1)$
            and $k_2(S_2,r_2)$ ($r_1\geq r_2$) be two circles. Then
            (Figure \ref{sl.skk.4.1.7.pic}):

            (i) the circles $k_1(S_1,r_1)$
            and $k_2(S_2,r_2)$ are lying outside each other
             if and only if $|S_1S_2|>r_1+r_2$;

             (ii) the circles $k_1(S_1,r_1)$
            and $k_2(S_2,r_2)$ are touching each other externally
            if and only if $|S_1S_2|=r_1+r_2$;

             (iii) the circles $k_1(S_1,r_1)$
            and $k_2(S_2,r_2)$ are intersecting each other at two points
            if and only if $r_1-r_2<|S_1S_2|<r_1+r_2$;

            (iv) the circles $k_1(S_1,r_1)$
            and $k_2(S_2,r_2)$ are touching each other internally
             if and only if $|S_1S_2|=r_1-r_2$;

            (v) one of the circles $k_1(S_1,r_1)$
            and $k_2(S_2,r_2)$ is lying inside another
             if and only if $|S_1S_2|<r_1-r_2$.
            \eizrek


Sedaj bomo definirali še nekaj pojmov, ki se nanašata na dve
krožnici.

\pojem{Kot med krožnicama}, ki se sekata, je kot, ki ga določata
tangenti teh dveh krožnic v njuni skupni točki. Ni težko dokazati,
da ta kot ni odvisen od izbire skupne točke, oz. da sta kota med
tangentama v vsaki od dveh skupnih točk skladna (Figure
\ref{sl.skk.4.1.8.pic}).

Ko se krožnici dotikata, pravimo, da določata kot $0^0$.


\begin{figure}[!htb]
\centering
\input{sl.skk.4.1.8.pic}
\caption{} \label{sl.skk.4.1.8.pic}
\end{figure}

Krožnici sta \index{pravokotni!krožnici} \pojem{pravokotni}, če
določata kot $90^0$, oz. če sta njuni tangenti v skupni točki
pravokotni (Figure \ref{sl.skk.4.1.8.pic}).

Direktna posledica izreka \ref{TangPogoj} je naslednji pogoj
pravokotnosti dveh krožnic.



            \bizrek \label{pravokotniKroznici}
            Two circles are perpendicular if and only if
            the tangent of one of the circles at the points of intersection
            contains the centre of  another circle.
             \eizrek

V zgledu \ref{tangKrozKonstr} smo ugotovili, kako lahko narišemo
tangenti krožnice iz njene poljubne zunanje točke. V izreku
\ref{tangSkupnaDotikKrozn} pa smo  dokazali, da imata krožnici, ki se
dotikata,  vsaj eno skupno tangento. V naslednjem zgledu bomo
konstruirali \index{skupna tangenta}\pojem{skupne tangente} dveh
krožnic v splošni legi.

            \bzgled \label{tang2ehkroz}
            Construct a common tangent of two given circles
            lying in the same plane.
            \ezgled

\textbf{\textit{Solution.}} Naj bosta $k_1(S_1,r_1)$ in
$k_2(S_2,r_2)$ ($r_1\geq r_2$) poljubni krožnici v isti ravnini ter
$t$ njuna skupna tangenta, ki se krožnic $k_1$ in $k_2$
dotika po vrsti v točkah $T_1$ in $T_2$.

Obravnavali bomo dva primera:

\textit{1)} Predpostavimo najprej, da sta točki $T_1$ in $T_2$ na
istem bregu centrale $S_1S_2$
 (Figure \ref{sl.skk.4.1.9.pic}). Označimo $S'_2=pr_{\perp
S_1T_1}(S_2)$. Po izreku \ref{TangPogoj} je $\angle
S_1T_1T_2\cong\angle S_2T_2T_1=90^0$. To pomeni, da je štirikotnik
$S_2T_2T_1S'_2$ pravokotnik, zato je tudi $\angle S_2S'_2T_1=90^0$
in $|S'_2T_1|=|S_2T_2|=r_2$.
 Ker je po predpostavki $r_1\geq r_2$ in $|S_1T_1|=r_1$, je
 $|S_1S'_2|=r_1-r_2$. Označimo s $k$ krožnico s središčem $S_1$ in
 polmerom $r_1-r_2$. Krožnica $k$ poteka skozi točko $S'_2$.
 Če je $S_2$ zunanja točka krožnice $k$,
 iz $\angle S_2S'_2T_1=90^0$
 sledi, da je premica $S_2S_2'$ tangenta
 te krožnice.

\begin{figure}[!htb]
\centering
\input{sl.skk.4.1.9.pic}
\caption{} \label{sl.skk.4.1.9.pic}
\end{figure}

 Prejšnja analiza nam omogoča konstrukcijo. Najprej načrtamo
 krožnico $k(S_1,r_1-r_2)$, nato pa njeno tangento $S_1S'_2$ v
 dotikališču $S'_2$ (zgled \ref{tangKrozKonstr}), točko $T_1$ kot
 presečišče poltraka $S'_2T_1$ in krožnice $k_1$, četrto oglišče
 $T_2$ pravokotnika $T_1S'_2S_2T_2$ (ker je že iz konstrukcije
 $\angle S_2S'_2T_1=90^0$) in na koncu skupno tangento $t=T_1T_2$.


 Dokažimo, da je $t$ res skupna tangenta. Ker je že po
 konstrukciji $T_1\in k_1$ in $\angle S_1T_1T_2\cong
 \angle S_2T_2T_1=90^0$, je dovolj dokazati, da velja $T_2\in
 k_2$. To pa sledi iz dejstva, da je štirikotnik $T_1S'_2S_2T_2$
 pravokotnik oz. $|S_2T_2|=|S'_2T_1|=r_1-(r_1-r_2)=r_2$.

Razen načrtane tangente $t$ dobimo še tangento $t_1$,
ki je simetrična tangenti $t$ glede na centralo $S_1S_2$. Za
tangenti $t$ in $t_1$ pravimo, da sta \index{skupna
tangenta!zunanja}\pojem{zunanji tangenti}.

\textit{2)} Če predpostavimo, da sta $T_1$ in $T_2$ na različnih
bregovih centrale $S_1S_2$, dobimo v določenih primerih  še dve t. i. \index{skupna tangenta!notranja}\pojem{notranji tangenti} po
enakem postopku, le da krožnico $k(S_1,r_1-r_2)$ zamenjamo s krožnico
$k'(S_1, r_1+r_2)$.

\begin{figure}[!htb]
\centering
\input{sl.skk.4.1.10.pic}
\caption{} \label{sl.skk.4.1.10.pic}
\end{figure}

Obravnavajmo še število rešitev naloge (Figure
\ref{sl.skk.4.1.10.pic}). Kadar sta krožnici mimobežni in
nobena ni v notranjosti druge, imata krožnici vse štiri opisane
skupne tangente - dve zunanji in dve notranji tangenti. Če
se krožnici dotikata od zunaj, ima naloga tri rešitvi, ker se notranji
tangenti prekrivata in dobimo skupno tangento, ki je omenjena v
izreku \ref{tangSkupnaDotikKrozn}. Ko se krožnici sekata,
imamo le dve rešitvi - dve zunanji tangenti. Kadar se krožnici dotikata od
znotraj, obstaja le ena skupna tangenta (tista iz izreka
\ref{tangSkupnaDotikKrozn}). In na koncu, če sta krožnici mimobežni
in je ena v notranjosti druge, nimata
skupnih tangent.
\kdokaz



            \bzgled
            Let $A$, $B$, $C$ and $D$ be points in the plane such that do not all lie
            on the same circle nor all on the same line. Prove that there are two such circles
            $k$ and $l$, which have no common points, the first of them passes
            through the points $A$ and $B$, and the other one
             through the points $C$ and $D$.
            \ezgled

\begin{figure}[!htb]
\centering
\input{sl.skk.4.1.1a.pic}
\caption{} \label{sl.skk.4.1.1a.pic}
\end{figure}

\textbf{\textit{Proof.}}
 Naj bosta
$m$ in $n$ simetrali daljic $AB$ in $CD$. Obravnavali bomo dva
primera (Figure \ref{sl.skk.4.1.1a.pic}):

 \textit{1)}
Če se premici $m$ in $n$ sekata v točki $O$, sta iskani krožnici
$k(O,OA)$ in $l(O,OC)$, ker sta koncentrični in po predpostavki
različni.

\textit{2)} Če sta premici $m$ in $n$ vzporedni, sta vzporedni tudi
premici $AB$ in $CD$ (in po predpostavki različni). S $p$ označimo
poljubno vzporednico premic $AB$ in $CD$, tako da bosta $AB$ in $CD$
na različnih straneh premice $p$. Naj bosta $M$ in $N$ presečišči
premic $m$ in $n$ s premico $p$. Če je $M \neq N$, sta iskani
krožnici očrtani krožnici trikotnikov $ABM$ in $CDN$ (izrek
\ref{SredOcrtaneKrozn}), ker ležita na različnih bregovih premice
$p$. Če je $M = N$ oz. $m = n$, sta iskani krožnici spet
koncentrični $k(M,MA)$ in $l(M,MC)$.
 \kdokaz



%________________________________________________________________________________
 \poglavje{Center Angle and Circumferential Angle} \label{odd4SredObod}


 Definirajmo najprej pojma središčni in obodni kot
  (Figure \ref{sl.skk.4.2.1a.pic}).

\begin{figure}[!htb]
\centering
\input{sl.skk.4.2.1a.pic}
\caption{} \label{sl.skk.4.2.1a.pic}
\end{figure}

 \pojem{Središčni kot}
  \index{kot!središčni}  krožnice $k(S,r)$ je poljuben kot,
  ki leži v ravnini te krožnice  in ima vrh v točki $S$.
   \pojem{Obodni kot}
  \index{kot!obodni} te krožnice je poljuben kot z vrhom
  na krožnici $k$, njegova kraka pa vsebujeta dve tetivi te
krožnice. Presek krožnice in njegovega središčnega oz. obodnega kota
je lok, ki ga imenujemo \pojem{pripadajoči lok} tega kota. V tem
primeru za središčni oz. obodni kot pravimo, da je kot \pojem{nad tem lokom}.
 Vemo, da poljubna tetiva $PQ$ na krožnici $k(S,r)$ določa dva loka.
 Če bomo vedeli, za kateri
od obeh lokov gre, bomo včasih za kot nad pripadajočim lokom $PQ$
rekli, da je kot \pojem{nad tetivo} $PQ$.

V posebnem primeru, ko je tetiva premer, je pripadajoč središčni kot
nad to tetivo  enak $180^0$. Talesov izrek za krožnico (izrek
\ref{TalesovIzrKroz}) lahko v terminih obodnih kotov zapišemo v
naslednji obliki:



             \bizrek \label{TalesovIzrKroz2oblika}
            All inscribed angles subtending a diameter of a circle are right angles.
            \eizrek

\begin{figure}[!htb]
\centering
\input{sl.skk.4.2.2a.pic}
\caption{} \label{sl.skk.4.2.2a.pic}
\end{figure}

Torej je središčni kot nad premerom dvakrat večji od
obodnega kota nad tem premerom (Figure \ref{sl.skk.4.2.2a.pic}).
Dokazali bomo, da ta trditev velja tudi  za obodni in središčni kot nad
poljubno tetivo.


\begin{figure}[!htb]
\centering
\input{sl.skk.4.2.3.pic}
\caption{} \label{sl.skk.4.2.3.pic}
\end{figure}



         \bizrek
            \label{SredObodKot}
            In any circle, a central angle is twice of the measure
            of the circumferential angle subtending the
            same arc\footnote{Predpostavlja se, da je to trditev prvi dokazal
           \index{Tales}\textit{Tales} iz Mileta (7.--6. stol. pr. n. š.).} (Figure \ref{sl.skk.4.2.3.pic}).
          \eizrek


\begin{figure}[!htb]
\centering
\input{sl.skk.4.2.4.pic}
\caption{} \label{sl.skk.4.2.4.pic}
\end{figure}

\textbf{\textit{Proof.}} Naj bo $PQ=l$  lok krožnice $k(S, r)$ in
$V$ poljubna točka te krožnice, ki ne leži na tem loku. Dokazali
bomo, da za središčni kot $PSQ$ in obodni kot $PVQ$ velja:
$$\angle PSQ = 2\angle PVQ.$$ Obravnavali bomo tri različne možnosti
(Figure \ref{sl.skk.4.2.4.pic}):

\textit{1)} Središče $S$ krožnice $k$ leži na enem od krakov
obodnega kota $PVQ$. Brez škode za splošnost naj bo to krak $VQ$. V
tem primeru je $PSV$ enakokraki trikotnik ($SP \cong SV = r$), zato
je $\angle SPV \cong \angle PVS$ (izrek \ref{enakokraki}). V
trikotniku $PSQ$ je zunanji kot $PSQ$ enak vsoti nesosednjih
notranjih kotov (izrek \ref{zunanjiNotrNotr}). Torej:
 $$ \angle PSQ =
\angle SPV + \angle PVS = 2\angle PVS\\ = 2\angle PVQ.$$

\textit{2)} Središče $S$ krožnice $k$ leži v notranjosti obodnega
kota $PVQ$. V tem primeru drugo presečišče krožnice $k$ s premico
$VS$ – točka $V'$ – leži na loku $l$, ker le-ta predstavlja presek
obodnega kota  $PVQ$ in krožnice $k$. Če dvakrat uporabimo dokazano
dejstvo iz \textit{1)}, dobimo:
  \begin{eqnarray*}
\angle PSQ &=& \angle PSV'+\angle V'SQ =
2\angle PVV'+2\angle V'VQ=\\ &=& 2(\angle PVV'+\angle V'VQ) = 2\angle PVQ.
  \end{eqnarray*}
\textit{3)} Središče $S$ krožnice $k$ je zunanja točka obodnega
kota. Na enak način kot v \textit{2)} definirajmo točko $V'$. V tem
primeru točka $V'$ ne leži na loku $l$. Brez škode za splošnost
predpostavimo, da je $P$ notranja točka obodnega kota $\angle V'VQ$.
Spet uporabimo rezultat iz \textit{1)}:
  \begin{eqnarray*}
  \angle PSQ &=& \angle V'SQ
-\angle V'SP = 2\angle V'VQ - 2\angle V'VP=\\ &=& 2(\angle V'VQ -\angle
V'VP) = 2\angle PVQ,
  \end{eqnarray*}
 kar je bilo potrebno dokazati.  \kdokaz

Najpomembnejši sta naslednji posledici prejšnjega izreka.



        \bizrek
        \label{ObodObodKot}
        In a circle, different circumferential angles
        subtending the same arc are congruent.
        \eizrek


\begin{figure}[!htb]
\centering
\input{sl.skk.4.2.5.pic}
\caption{} \label{sl.skk.4.2.5.pic}
\end{figure}

\textbf{\textit{Proof.}} Iz prejšnjega izreka \ref{SredObodKot}
sledi, da so vsi obodni koti nad istim lokom enaki polovici
 središčnega kota nad tem lokom (Figure \ref{sl.skk.4.2.5.pic}).
  Zaradi tega so vsi
omenjeni obodni koti med seboj skladni.
 \kdokaz



        \bizrek
        \label{ObodObodKotNaspr}
        Two circumferential angles
        subtending the same chord of a circle, with the vertices lying
        on different sides of the line containing this chord,
        are supplementary.
        \eizrek


\textbf{\textit{Proof.}} Tetiva krožnice določa na njej dva loka, ki
se dopolnjujeta do cele krožnice (Figure \ref{sl.skk.4.2.5.pic}).
Vsak od obeh omenjenih obodih kotov ustreza enemu od teh dveh
lokov, kar pomeni, da je vsota ustreznih središčnih kotov enaka
$360^0$. Ker je vsota obodnih kotov po izreku \ref{SredObodKot}
enaka polovici vsote ustreznih središčnih kotov, sta obodna kota
suplementarna.
 \kdokaz

Kot smo že omenili, bomo pogosto govorili o obodnih in središčnih
kotih nad isto tetivo, če le vemo, za katerega od obeh pripadajočihih lokov te
tetive gre. V tem smislu formulirajmo naslednji trditvi.



        \bizrek
          \label{SklTetSklObKot}
          Two circumferential angles
        subtending the congruent chords of a circle
        are congruent.
         \eizrek

\begin{figure}[!htb]
\centering
\input{sl.skk.4.2.4a.pic}
\caption{} \label{sl.skk.4.2.4a.pic}
\end{figure}

\textbf{\textit{Proof.}}  Po izreku \textit{SSS} \ref{SSS} skladnima
tetivama ustrezata skladna središčna kota (Figure
\ref{sl.skk.4.2.4a.pic}). Trditev je potem direktna posledica izreka
\ref{SredObodKot}.  \kdokaz

Jasno je, da velja tudi naslednja trditev  (Figure
\ref{sl.skk.4.2.4b.pic}).


            \bizrek
          \label{SklTetSklObKot2}
          Two circumferential angles
        subtending the congruent chords of congruent circles
        are congruent.
         \eizrek


\begin{figure}[!htb]
\centering
\input{sl.skk.4.2.4b.pic}
\caption{} \label{sl.skk.4.2.4b.pic}
\end{figure}

Zelo zanimiva in uporabna je tudi naslednja posledica.



         \bizrek
          \label{ObodKotTang}
          The angle determined by a chord of a circle and the tangent of that circle
            in one of the endpoints of this chord is congruent to the circumferential
            angle subtending this chord.
         \eizrek

\begin{figure}[!htb]
\centering
\input{sl.skk.4.2.6.pic}
\caption{} \label{sl.skk.4.2.6.pic}
\end{figure}

\textbf{\textit{Proof.}}  Naj bo $PAQ$ obodni kot neke krožnice nad
tetivo $PQ$, $LP$ takšna tangenta te krožnice v točki $P$, da sta
točki $L$ in $A$ na različnih straneh premice $PQ$, in $PB$ premer te
krožnice (Figure \ref{sl.skk.4.2.6.pic}). Ker je $BP \perp PL$ (izrek
\ref{TangPogoj}) in $BQ \perp PQ$ (Talesov izrek
\ref{TalesovIzrKroz}), sta  kota $LPQ$ in $PBQ$ skladna (izrek
\ref{KotaPravokKraki}). Toda po izreku \ref{ObodObodKot} sta
 kota $PAQ$ in $PBQ$ skladna, zato je tudi $\angle LPQ \cong PAQ$.
 \kdokaz

Prejšnji rezultat omogoča realizacijo ene zelo pomembne konstrukcije
-- načrtovanje geometrijskega mesta točk v ravnini, iz katerih se
dana daljica ‘‘vidi’’ pod danim kotom. Pravzaprav gre za
naslednji problem.

%angle of view of a line segment!!!


        \bizrek
          \label{ObodKotGMT}
          Let $A$ and $B$ be two different points and $\omega$ a given angle.
        The set of all points $X$ in the plane from which the angle of view of the line segment $AB$ is
        $\omega$, i.e. $\angle AXB\cong \omega$, is the union of two open circular arcs, which are
        symmetric with respect to the line $AB$.
        \eizrek


\begin{figure}[!htb]
\centering
\input{sl.skk.4.2.7.pic}
\caption{} \label{sl.skk.4.2.7.pic}
\end{figure}

\textbf{\textit{Proof.}} Naj bo $p$ poltrak z izhodiščem $A$, tako da
je $\angle p,BA \cong \omega$, premica $n$ pravokotnica tega
poltraka v točki $A$ in $s$ simetrala daljice $AB$ (Figure
\ref{sl.skk.4.2.7.pic}). Presečišče premic $n$ in $s$ označimo s
$S$, krožnico s središčem $S$ in polmerom $SA$ pa s $k$. Z $l$
označimo lok, ki je presek krožnice $k$ in polravnine z robom $AB$,
v kateri ne leži poltrak $p$. V dopolnilni polravnini na podoben
način določimo lok $l'$, ki je skladen z lokom $l$. Dokažimo, da je
iskano geometrijsko mesto točk množica $l \cup l'\setminus\{A,B\}$.

Dejstvo, da se iz vsake točke loka $l$ (oz. $l'$), ki je različna od
točk $A$ in $B$, daljica $AB$ vidi pod kotom $\omega$, sledi iz
izrekov \ref{TangPogoj} in \ref{ObodKotTang}. Za poljubno točko
$P$ odprtega loka $l$ velja: $\angle APB \cong \angle p,AB \cong
\omega$.

Predpostavimo, da točka $M$ ne pripada množici $l \cup
l'\setminus\{A,B\}$. V primeru, kadar je $M$ ena od točk $A$ ali
$B$, kot $AMB$ sploh ne obstaja. Naj bo $M \neq A,B$ in brez škode
za splošnost točka $M$ v isti polravnini kot lok $l$.
Označimo z $N$ drugo presečišče poltraka $AM$ in loka $l$ ($N\neq A$).
Če je $\mathcal{B}(A,M,N)$, je v trikotniku $NMB$ zunanji kot $AMB$
večji od nesosednjega notranjega kota $MNB$ (izrek
\ref{zunanjiNotrNotrVecji}), ki je po že dokazanem enak $\omega$,
zato je $\angle AMB>\omega$. Če pa velja $\mathcal{B}(A,N,M)$, lahko s
podobnim sklepanjem  ugotovimo, da je v tem primeru $\angle
AMB<\omega$, kar pomeni, da za nobeno točko $M\notin l \cup
l'\setminus\{A,B\}$ ne velja $\angle AMB \cong \omega$.
 \kdokaz

Iz dokaza prejšnjega izreka dobimo tudi naslednjo ugotovitev.


            \bizrek \label{obodKotGMTZunNotr}
             Let $AB$ be an arc of a circle $k$, $\omega$ the corresponding circumferential angle
          of this arc and $M$ a point of the half-plane with the edge $AB$ not containing this arc (Figure
            \ref{sl.skk.4.2.8.pic}). Then:

             (i) $\angle AMB >\omega$ if and only if the point $M$ is an interior point of the circle $k$;

             (ii) $\angle AMB \cong\omega$ if and only if the point $M$ lies on the circle $k$;

             (iii) $\angle AMB <\omega$ if and only if the point $M$ is an exterior point of the circle $k$.
            \eizrek


\begin{figure}[!htb]
\centering
\input{sl.skk.4.2.8.pic}
\caption{} \label{sl.skk.4.2.8.pic}
\end{figure}

V naslednjih primerih bomo videli uporabo izreka o obodnem in
središčnem kotu in njegovih posledic.



         \bzgled
         Construct a triangle with given $a$, $\alpha$, $v_a$. \label{konstr_aalphava}
         \ezgled

\begin{figure}[!htb]
\centering
\input{sl.skk.4.3.1d.pic}
\caption{} \label{sl.skk.4.3.1d.pic}
\end{figure}

   \textbf{\textit{Analysis.}} Točka $A$ leži hkrati na geometrijskem mestu
točk, iz katerih se daljica $BC$ vidi pod kotom $\alpha$ (unija dveh krožnih lokov - izrek \ref{ObodKotGMT}) in vzporednici premice $BC$, ki je od nje
oddaljena $v_a$ (Figure \ref{sl.skk.4.3.1d.pic}). Torej je oglišče $A$ presečišče te
vzporednice in omenjenega geometrijskega mesta točk.

\textbf{\textit{Construction.}} Načrtajmo najprej daljico
$BC\cong a$, nato pa geometrijsko mesto točk $\mathcal{L}$, iz
katerih se ta daljica vidi pod kotom $\alpha$ (izrek \ref{ObodKotGMT}).
Nato načrtajmo  vzporednico $p$ premice $BC$ na razdalji $v_a$. Z $A$ označimo presečišče premice $p$
in omenjenega geometrijskega mesta točk $\mathcal{L}$. Dokažimo, da je $ABC$ iskani trikotnik.

\textbf{\textit{Proof.}} Že po konstrukciji je jasno, da je $BC\cong a$. Po konstrukciji točka $A$ leži na
geometrijskem mestu točk, iz katerih se daljica $BC$ vidi pod kotom $\alpha$, zato je tudi $BAC\cong\alpha$. Višina
trikotnika $ABC$ iz oglišča $A$ je skladna daljici $v_a$, ker točka $A$ po konstrukciji leži na premici $p$, ki je na
razdalji $v_a$ vzporedna premici $BC$.

\textbf{\textit{Discussion.}} Nujni pogoj je seveda $\alpha<180^0$. Število rešitev naloge je enako številu presečišč
premice $p$ in množice $\mathcal{L}$.
 \kdokaz



            \bzgled
           Let $p$, $q$ and $r$ be lines in the plane intersecting
            at one point and divide this plane into six congruent angles. Suppose that $P$, $Q$ and $R$
            are the foots of the perpendiculars from an arbitrary point $X$ of this plane on the lines $p$, $q$ and
            $r$, respectively. Prove that $PQR$ is a regular triangle.
            \ezgled



\begin{figure}[!htb]
\centering
\input{sl.skk.4.2.9.pic}
\caption{} \label{sl.skk.4.2.9.pic}
\end{figure}

\textbf{\textit{Proof.}}  (Figure \ref{sl.skk.4.2.9.pic}).
 Naj bo $S$
presečišče premic $p$, $q$ in $r$. Jasno je, da premice določajo
kote $60^0$. Ker je $\angle XPS \cong \angle XQS \cong \angle XRS =
90^0$, po izreku \ref{TalesovIzrKroz2} točke $S$, $X$, $P$, $Q$ in
$R$ ležijo na krožnici $k$ s premerom $SX$.
 Če uporabimo izrek \ref{ObodObodKot} za ustrezne loke $PQ$ in $QR$,
 je $\angle PRQ \cong \angle PSQ =
60^0$ in $\angle QPR \cong \angle QSR = 60^0$. Ker ima vse kote
enake $60^0$, je torej $PQR$ pravilni trikotnik.
 \kdokaz


            \bzgled
            Let $k(O,R)$ and $l(S,r)$ ($R = 2r$) be circles touching each other internally
            and $P$ an arbitrary point on the circle $l$. Which curve
            is described by the point $P$ if the circle $l$ rolls without slipping around the circle $k$\footnote{Ta problem je rešil poljski astronom
             \index{Copernicus, N.} \textit{N. Copernicus} (1473--1543).
             V splošnem primeru, kadar ni nujno $R = 2r$, krivuljo, po
             kateri se giblje točka $P$, imenujemo \index{hipocikloida}
              \pojem{hipocikloida}. V primeru zunanjega kotaljenja krožnice
              po drugi krožnici, krivuljo imenujemo \index{epicikloida}
              \pojem{epicikloida}, v primeru
              kotaljenja krožnice po premici pa je to \index{cikloida}
              \pojem{cikloida}. Cikloido je prvi raziskoval nemški matematik in
              filozof \index{Kuzanski, N.} \textit{N. Kuzanski} (1401--1464)
              in kasneje francoski matematik in filozof \index{Mersenne, M.}
              \textit{M. Mersenne} (1588--1648). Poimenoval jo je italijanski fizik,
              matematik, astronom in filozof \index{Galilei, G.}
               \textit{G. Galilei} (1564--1642) leta 1599.}?
            \ezgled

\begin{figure}[!htb]
\centering
\input{sl.skk.4.2.10.pic}
\caption{} \label{sl.skk.4.2.10.pic}
\end{figure}

\textbf{\textit{Solution.}}
 Naj bo
$P_0$ lega točke $P$ v trenutku, ko leži na krožnici $k$ (Figure
\ref{sl.skk.4.2.10.pic}). Takrat, ko je točka $P$ v legi $P_i$,
se krožnica $l$, ki je v legi $l_i$, dotika krožnice $k$ v neki
točki $T_i$. Ker gre za ‘‘gibanje brez spodrsavanja’’, sta dolžini
ustreznih lokov $P_0T_i$ in $P_iT_i$ krožnic $k$ in $l$ med seboj
enaki. Polmer krožnice $k$ je dvakrat večji od polmera krožnice $l$,
zato za pripadajoče središčne kote omenjenih lokov velja $\angle
T_iS_iP_i= 2\angle T_iOP_0$. Toda $\angle T_iOP_i$ je ustrezni
obodni kot krožnice $l$ za isti lok $P_iT_i$, zato je $2\angle
T_iOP_i = \angle T_iS_iP_i$ (izrek \ref{ObodObodKot}). Iz prejšnjih
dveh relacij dobimo $\angle T_iOP_i= \angle T_iOP_0$, kar pomeni,
da je točka $P_i$ kolinearna s točkama $O$ in $P_0$, zato iskani tir
točke $P$ predstavlja premer $P_0P'_0$  krožnice $k$.
 \kdokaz



            \bzgled
            Let $C$ be the midpoint of an arc $AB$ and $D$ an arbitrary point of this
            arc other than $C$. Prove that
                $$AC + BC > AD + BD.$$
            \ezgled

\begin{figure}[!htb]
\centering
\input{sl.skk.4.2.11.pic}
\caption{} \label{sl.skk.4.2.11.pic}
\end{figure}

\textbf{\textit{Proof.}}
  (Figure
\ref{sl.skk.4.2.11.pic}).
 Naj bosta $C'$ in $D'$ takšni točki poltrakov $AC$ in $AD$,
 da velja: $CC'\cong CB \cong CA$, $DD'\cong DB$,
$\mathcal{B}(A,C,C')$ in $\mathcal{B}(A,D,D')$. Trikotnika $C'CB$ in
$D'DB$ sta enakokraka, zato po izrekih \ref{enakokraki} in \ref{zunanjiNotrNotr},
sledi:
 $$\angle CC'B \cong \angle CBC' = \frac{1}{2}\angle ACB
\hspace*{2mm}\textit{ in }\hspace*{2mm}
 \angle DD'B \cong \angle DBD'
= \frac{1}{2}\angle ADB.$$
 Ker sta kota $ACB$ in $ADB$ obodna kota nad tetivo $AB$, sta skladna
  (izrek \ref{ObodObodKot}).
Skladna sta tudi kota $AC'B$ in $AD'B$,  točki $C'$ in $D'$ pa
pripadata pripadajočemu loku $l'$ nad tetivo $AB$ (izrek
\ref{ObodKotGMT}). Ker je $CC'\cong CB \cong CA$, je daljica $AC'$
premer krožnice, ki vsebuje lok $l'$, zato je $\angle AD'C'$ pravi
kot (izrek \ref{TalesovIzrKroz2}). Daljica $AC'$ je torej
hipotenuza pravokotnega trikotnika $AD'C'$ in zato po izreku
\ref{vecstrveckot} velja:
$$AC + CB = AC'> AD'= AD + DB,$$ kar je bilo treba dokazati. \kdokaz



      \bnaloga\footnote{35. IMO Hong Kong - 1994, Problem 2.}
      $ABC$ is an isosceles triangle with $BC \cong AC$. Suppose that:

        (i) $D$ is the midpoint of $AB$ and $E$ is the point on the line $CD$ such that
            $EA\perp AC$;

        (ii) $F$ is an arbitrary point on the segment $AB$ different from $A$ and $B$;

        (iii) $G$ lies on the line $CA$ and $H$ lies on the line $CB$ such that $G$, $F$, $H$ are
              distinct and collinear.

        Prove that $EF$ is perpendicular to $HG$ if and only if $GF\cong FH$.
        \enaloga


\begin{figure}[!htb]
\centering
\input{sl.skk.4.2.IMO1.pic}
\caption{} \label{sl.skk.4.2.IMO1.pic}
\end{figure}


\textbf{\textit{Proof.}} Najprej iz skladnosti trikotnikov $CAE$ in
$CBE$ (izrek \textit{SAS} \ref{SKS}) sledi $\angle EBC\cong\angle
EAC=90^0$ in $EA\cong EB$ (Figure \ref{sl.skk.4.2.IMO1.pic}).
Dokažimo ekvivalenco $EF\perp HG \Leftrightarrow GF\cong FH$ v obe
smeri.

 ($\Rightarrow$) Predpostavimo, da sta premici $EF$ in $HG$
 pravokotni oz. $\angle EFG\cong\angle EFH=90^0$.
 Naj bosta $k$ in $l$ krožnici s premeroma $EG$ in
$EH$. Ker je $\angle EAG\cong\angle EFG=90^0$ in $\angle
EBH\cong\angle EFH=90^0$, po izreku \ref{TalesovIzrKroz} velja
$A,F\in k$ in $B,F\in l$. Najprej iz $EA\cong EB$ sledi $\angle
EAB\cong\angle EBA$. Iz tega in izreka \ref{ObodObodKot} sledi:
 $$\angle EGF\cong\angle EAF\cong\angle EBF\cong\angle EHF.$$
 Torej je trikotnik $EGH$ enakokrak, zato je njegova višina $EF$
 hkrati težiščnica (skladnost trikotnikov $EFG$ in $EFH$, izrek
 \textit{ASA} \ref{KSK}) oz. velja $GF\cong FH$.

($\Leftarrow$) Naj bo sedaj $GF\cong FH$, oz. je točka $F$
središče daljice $GH$. Naj bo $k$ krožnica s premerom $EG$.
Poleg točke $A$ označimo s $\widehat{F}$ drugo presečišče te
krožnice s premico $AB$. Če se krožnica $k$ dotika premice
$AC$, sledi $G=A$ oz. $F=D$ in $H=B$, torej je v tem primeru že
izpolnjeno $GF\cong FH$.

Predpostavimo, da je $\widehat{F}\neq F$. Naj bo $\widehat{H}$
presečišče premic $G\widehat{F}$ in $CB$. Ker točka
$\widehat{F}$ leži na krožnici $k$ s premerom $EG$, je $\angle
G\widehat{F}E=90^0$, oz. velja $E\widehat{F}\perp G\widehat{H}$.
Torej so za točke $G$, $\widehat{F}$ in $\widehat{H}$ izpolnjene
predpostavke leve strani ekvivalence, zato iz že dokazanega
prvega dela trditve ($\Rightarrow$) sledi $G\widehat{F}\cong
\widehat{F}\widehat{H}$, oz. točka $\widehat{F}$ je središče
daljice $G\widehat{H}$. V trikotniku $GH\widehat{H}$ je
$\widehat{F}F$ srednjica, zato je $\widehat{F}F\parallel
\widehat{H}H$ in $AB\parallel BC$, kar pa ni mogoče. Tako
predpostavka $\widehat{F}\neq F$ odpade, torej velja $\widehat{F}=
F$, zato je tudi $\widehat{H}=H$. Na koncu iz $E\widehat{F}\perp
G\widehat{H}$ sledi $EF\perp GH$.
 \kdokaz





%________________________________________________________________________________
 \poglavje{More About Circumcircle and Incircle of a Triangle}
 \label{odd4OcrtVcrt}


Najprej bomo obravnavali nekatere pomembne točke, ki ležijo na
očrtani krožnici trikotnika.



        \bizrek \label{TockaN}
        The bisector of the side $BC$ and the bisector of the interior angle $BAC$ of a
        triangle $ABC$ ($AB\neq AC$) intersect at the Circumcircle $l(O,R)$
        of that triangle.
         \eizrek

\begin{figure}[!htb]
\centering
\input{sl.skk.4.3.1.pic}
\caption{} \label{sl.skk.4.3.1.pic}
\end{figure}


\textbf{\textit{Proof.}}
   (Figure \ref{sl.skk.4.3.1.pic}).

 Naj bo točka $N$ eno od presečišč očrtane krožnice trikotnika $ABC$ in
 simetrale stranice $BC$
(takšna, da je $A,N÷BC$). Ker leži točka $N$ na simetrali stranice
$BC$, velja $NB \cong NC$. Torej je $BNC$ enakokraki trikotnik,
zato sta  $\angle NBC$ in $\angle NCB$ skladna kota (izrek
\ref{enakokraki}). Ker leži točka $N$ tudi na očrtani krožnici
trikotnika $ABC$, po izreku \ref{ObodObodKot} velja:
 \begin{eqnarray*}
 \angle BAN
\cong \angle BCN \textrm{ (obodna kota za krajši lok }BN \textrm{) }\\
 \angle NAC \cong \angle NBC \textrm{ (obodna kota za krajši lok }
 CN \textrm{).}
 \end{eqnarray*}
 Torej je $\angle BAN \cong \angle NAC$ oz. premica $AN$ je simetrala kota
$BAC$, s tem pa je trditev dokazana.
 \kdokaz


 Točka $N$ iz prejšnjega izreka je središče tistega loka $BC$ očrtane
 krožnice trikotnika $ABC$, ki ne vsebuje oglišča $A$.
 Dokažimo še eno pomembno lastnost točke $N$.



        \bizrek \label{TockaN.NBNC}
        For the point $N$ from the previous theorem is $NB\cong NS\cong NC$,
        where $S$ is the incentre of the triangle $ABC$.
         \eizrek


\begin{figure}[!htb]
\centering
\input{sl.skk.4.3.2.pic}
\caption{} \label{sl.skk.4.3.2.pic}
\end{figure}


\textbf{\textit{Proof.}} Označimo z $\alpha$ in $\beta$ notranja
kota trikotnika $ABC$ ob ogliščih $A$ in  $B$  (Figure
\ref{sl.skk.4.3.2.pic}). $BNS$ je enakokraki trikotnik (izrek
\ref{enakokraki}), ker sta kota pri ogliščih $B$ in $S$ skladna.
Če uporabimo izreka \ref{zunanjiNotrNotr} in \ref{ObodObodKot},
dobimo:
 \begin{eqnarray*}
 \angle BSN &=& \angle ABS + \angle BAS =\frac{1}{2}\alpha+\frac{1}{2}\beta,\\
\angle SBN &=& \angle SBC +\angle CBN  = \angle SBC + \angle CAN =
\frac{1}{2}\beta+\frac{1}{2}\alpha.
 \end{eqnarray*}
Torej velja $NB \cong NS$  in podobno tudi $NC \cong NS$.
\kdokaz



           \bizrek \label{TockaNbetagama}
            Let $AA'$ be the altitude from the vertices $A$ and $AE$ bisector of the interior
           angle $BAC$ of a triangle $ABC$ ($A',E\in BC$). Suppose that $l(O,R)$ is the
            circumcircle of that triangle. If $\angle CBA=\beta\geq\angle ACB=\gamma$,
          then
           $$\angle A'AE\cong \angle EAO=\frac{1}{2}\left( \beta-\gamma\right).$$
           \eizrek

\begin{figure}[!htb]
\centering
\input{sl.skk.4.3.1b.pic}
\caption{} \label{sl.skk.4.3.1b.pic}
\end{figure}


\textbf{\textit{Proof.}}
 Naj bo $N$ točka, definirana kot v v prejšnjih trditvah
  (Figure \ref{sl.skk.4.3.1b.pic}). Premici
 $AA'$ in $ON$ sta vzporedni, ker sta obe pravokotni na premico $BC$.
 Ker je $OA\cong ON=R$, je $AON$ enakokraki trikotnik. Zaradi tega je
 (izreka \ref{KotiTransverzala} in \ref{enakokraki}) najprej:
 $$\angle A' AE \cong \angle ANO \cong \angle NAO =
\angle EAO,$$
 nato pa še:
 $$\angle A'AE=\frac{1}{2}\alpha-\left(90^0-\beta \right)=
 \frac{1}{2}\alpha-\left(\frac{\alpha+\beta+\gamma}{2}-\beta \right)=
 \frac{1}{2}\left( \beta-\gamma\right),$$ kar je bilo treba dokazati. \kdokaz




           \bzgled \label{tockaNtockePQR}
             Let $P$, $Q$ and $R$ be the midpoints of those arcs $BC$, $AC$ and $AB$ of
           the circumcircle of a triangle $ABC$ not containing the vertices $A$, $B$ and $C$
            of that triangle.
            If $E$ and $F$ are intersections of the line $QR$ with sides $AB$ and $AC$, respectively
             and $S$ the incentre of this triangle, then:

             (i) $AP \perp QR$,

              (ii) the quadrilateral $AESF$ is a rhombus.
             \ezgled


\begin{figure}[!htb]
\centering
\input{sl.skk.4.3.2a.pic}
\caption{} \label{sl.skk.4.3.2a.pic}
\end{figure}


\textbf{\textit{Proof.}} (Figure \ref{sl.skk.4.3.2a.pic})

(\textit{i}) Naj bo $L$ presecišče premic $AP$ in $QR$. Po izreku
\ref{TockaN} točke $P$, $Q$ in $R$ ležijo na simetralah $AS$, $BS$
in $CS$ notranjih kotov trikotnika $ABC$ ($S$ je središče trikotniku
$ABC$ včrtane krožnice). Če z $\alpha$, $\beta$ in $\gamma$ označimo
notranje kote trikotnika $ABC$, potem zaradi skladnosti ustreznih
obodnih kotov (izrek \ref{ObodObodKot}) dobimo:
 \begin{eqnarray*}
\angle RPL &=& \angle RPA = \angle RCA =\frac{1}{2}\gamma,\\
 \angle PRL &=& \angle PRQ = \angle PRC + \angle CRQ =
\angle PAC + \angle CBQ = \frac{1}{2}\alpha+ \frac{1}{2}\beta.
 \end{eqnarray*}
Torej je vsota kotov v trikotniku $PRL$  (izrek \ref{VsotKotTrik})
$180^0 =\frac{1}{2}\alpha+ \frac{1}{2}\beta +\frac{1}{2}\gamma
+\angle RLP = 90° + \angle RLP$. Zato je $\angle RLP = 90°$ oz. $AP
\perp QR$.

 (\textit{ii}) Po
izreku \ref{TockaN.NBNC} je $RA \cong RS$, zato iz skladnosti
pravokotnih trikotnikov $ALR$ in $SLR$ (izrek \textit{SSA}
\ref{SSK}) sledi, da je točka $L$ središče diagonale $AS$
štirikotnika $AESF$. Iz skladnosti trikotnikov $AEL$ in $AFL$ (izrek
\textit{ASA} \ref{KSK}) sledi, da je točka $L$ tudi središče
diagonale $EF$, zato je $AESF$ paralelogram (izrek
\ref{paralelogram}. Ker sta diagonali $AS$ in $EF$ še pravokotni, je
 štirikotnik $AESF$ romb (izrek \ref{RombPravKvadr}).
  \kdokaz

Iz prejšnje trditve dobimo direktno posledico.


          \bzgled \label{PedalniLemasPQR}
           Let $P$, $Q$ and $R$ be the midpoints of those arcs $BC$, $AC$ and $AB$ of
           the circumcircle of a triangle $ABC$ not containing the vertices $A$, $B$ and $C$
            of that triangle.
            Prove that the incentre of the  triangle $ABC$ is at the same time
            orthocentre  of the triangle $PQR$ (Figure \ref{sl.skk.4.3.2b.pic}).
          \ezgled


\begin{figure}[!htb]
\centering
\input{sl.skk.4.3.2b.pic}
\caption{} \label{sl.skk.4.3.2b.pic}
\end{figure}



Dokažimo nekaj posledic izreka \ref{ObodObodKot}, ki so povezane z
višinsko točko.



         \bizrek \label{TockaV'}
         Points that are symmetric to the orthocentre of an acute triangle
            with respect to its sides lie on the circumcircle of this triangle.
         \eizrek


\begin{figure}[!htb]
\centering
\input{sl.skk.4.3.3.pic}
\caption{} \label{sl.skk.4.3.3.pic}
\end{figure}


\textbf{\textit{Proof.}} Naj bo $V$ višinska točka trikotnika $ABC$,
$l$ očrtana krožnica trikotnika $ABC$ in $V_a$ drugo presečišče
krožnice $l$ z nosilko višine $AA'$ (Figure \ref{sl.skk.4.3.3.pic}).
Dokažimo da je točka $V_a$ simetrična točki $V$ glede na premico
$BC$. Dovolj je dokazati, da je $VA'\cong V_aA'$. Kota $V_aBC$ in
$V_aAC$ sta skladna (obodna kota za tetivo $V_aC$ - izrek
\ref{ObodObodKot}), kota $V_aAC$ in $CBV$  sta skladna kota s
pravokotnima krakoma (izrek \ref{KotaPravokKraki}). Torej sta skladna
 tudi kota $V_aBC$ in $CBV$, zato pa tudi trikotnika $V_aBA'$ in
$VBA'$ oz. velja $VA'\cong V_aA'$. Podobno velja tudi za drugi dve
višini.
 \kdokaz

            \bizrek \label{TockaV'a}
            Let $V$ be the orthocentre of a triangle
            $ABC$. The circumcircles of the triangles $VBC$, $AVB$ and $ACV$ are congruent to
            the circumcircle of the triangle $ABC$.
             \eizrek


\begin{figure}[!htb]
\centering
\input{sl.skk.4.3.3a.pic}
\caption{} \label{sl.skk.4.3.3a.pic}
\end{figure}


\textbf{\textit{Proof.}} Direktna posledica prejšnjega izreka
\ref{TockaV'}, kajti tri omenjene očrtane krožnice so simetrične
očrtani krožnici trikotnika $ABC$ glede na nosilke njegovih stranic
(Figure \ref{sl.skk.4.3.3a.pic}).



         \bizrek \label{TockaV1}
          Points that are symmetric to the orthocentre of an acute triangle
            with respect to the midpoints of its sides lie on the circumcircle of this triangle.
         \eizrek


\begin{figure}[!htb]
\centering
\input{sl.skk.4.3.3b.pic}
\caption{} \label{sl.skk.4.3.3b.pic}
\end{figure}


\textbf{\textit{Proof.}} Naj bo $V$ višinska točka trikotnika $ABC$,
$l$ očrtana krožnica (s središčem $O$) trikotnika $ABC$ in $V_{A_1}$ točka, ki je simetrična točki $A$ glede na točko $O$ (Figure \ref{sl.skk.4.3.3.pic}). Iz same definicije točke $V_{A_1}$ je jasno, da ta leži na krožnici $l$. Dokažimo še, da je $V_{A_1}$ simetrična točki $V$ glede na točko $A_1$, ki je središče daljice $BC$. Ker je $AV_{A_1}$ premer krožnice $l$, je po izreku \ref{TalesovIzrKroz2} $\angle ACV_{A_1}=90^0$ oz. $V_{A_1}C\perp AC$.  Premica $BV$ je nosilka višine trikotnika $ABC$, zato velja $BV\perp AC$. Iz zadnjih dveh relacij sledi $V_{A_1}C\parallel BV$. Analogno je tudi
$V_{A_1}B\parallel CV$. Torej je štirikotnik $V_{A_1}CVB$ paralelogram, zato imata njegovi diagonali $VV_{A_1}$ in $BC$ skupno središče. Središče daljice $BC$ je točka $A_1$, kar pomeni, da  je $V_{A_1}$ simetrična točki $V$ glede na točko $A_1$. Točka $V_{A_1}$ pa po konstrukciji leži na krožnici $l$.
\kdokaz


Dokažimo še nekaj posledic izreka \ref{ObodObodKot}, ki so povezane z očrtano
krožnico trikotnika.


        \bzgled \label{zgledTrikABCocrkrozP}
        Let  $k$ be the circumcircle of a regular triangle $ABC$.
         If $P$ is an arbitrary point lying
        on the shorter arc $BC$ of the circle $k$, then
         $$|PA|=|PB|+|PC|.$$
         \ezgled


\begin{figure}[!htb]
\centering
\input{sl.skk.4.3.4.pic}
\caption{} \label{sl.skk.4.3.4.pic}
\end{figure}


\textbf{\textit{Proof.}} Ker je $\angle ACP>60^0>\angle PAC$, je
po izreku \ref{vecstrveckot} $AP>PC$ (Figure
\ref{sl.skk.4.3.4.pic}). Zato na daljici $AP$ obstaja takšna točka
$Q$, da velja $PQ\cong PC$. Po izreku \ref{ObodObodKot} je $\angle
CPQ=\angle CPA\cong\angle CBA=60^0$, kar pomeni, da je $PCQ$
enakostranični trikotnik, zato je tudi $CQ\cong CP$ in $\angle
PCQ=60^0$. Iz tega sledi $\angle ACQ=\angle ACP-60^0=\angle BCP$.
Po izreku \textit{SAS} (izrek \ref{SKS}) sta torej skladna trikotnika
$ACQ$ in $BCP$, zato je tudi $AQ\cong BP$.

Na koncu je: $|PB|+|PC|=|AQ|+|PQ|=|AP|$.
\kdokaz



            \bzgled
            Three circles of equal radii $r$ intersect at point $O$.
            Furthermore each two of them intersect at one more point: $A$, $B$, and $C$.
            Prove that the radius of the circumcircle of the triangle $ABC$ is also $r$.
            \ezgled

\begin{figure}[!htb]
\centering
\input{sl.skk.4.3.5.pic}
\caption{} \label{sl.skk.4.3.5.pic}
\end{figure}


\textbf{\textit{Proof.}}
   (Figure \ref{sl.skk.4.3.5.pic})

Označimo s $k$, $l$, $j$ in $o$ očrtane krožnice trikotnikov $OBC$,
$OAC$, $OAB$ in $ABC$ ter s $P$ poljubno točko krožnice $k$ tako, da
sta točki $O$ in $P$ na različnih straneh premice $BC$. Po
predpostavki so krožnice $k$, $l$ in $j$ skladne. Kota $BAO$ in
$BCO$ sta tudi skladna, ker sta obodna kota skladnih krožnic $k$ in $j$
nad tetivo $BO$ (izrek \ref{SklTetSklObKot2}). Analogno sta skladna
tudi kota $CAO$ in $CBO$. Zaradi tega je:
 $$\angle BAC = \angle BAO + \angle CAO = \angle BCO
+ \angle CBO = 180° - \angle BOC = \angle BPC.$$
 Torej imata krožnici $k$ in $o$ skladna
obodna kota nad skupno tetivo $BC$, zato sta med seboj skladna.
   \kdokaz



             \bzgled \label{KvadratKonstr4tocke}
            Construct a square $ABCD$ such that given points $P$, $Q$, $R$ and $S$
             lyes on the sides $AB$, $BC$, $CD$ and $DA$ of this square, respectively.
             \ezgled



\begin{figure}[!htb]
\centering
\input{sl.skk.4.3.1c.pic}
\caption{} \label{sl.skk.4.3.1c.pic}
\end{figure}


\textbf{\textit{Solution.}}
Ker sta $\angle PAS$ in $\angle QCR$ prava kota, oglišči $A$ in $B$ ležita na krožnicah $k$ in $l$ s premeroma
$PS$ in $QR$ (Figure \ref{sl.skk.4.3.1c.pic}). Nosilka diagonale $AC$ kvadrata $ABCD$ je hkrati simetrala notranjih kotov $BAD$ in $BCD$, zato gre skozi
središči $N$ in $M$ ustreznih polkrožnic, ki sta določeni s $k$ in $l$ (izrek \ref{TockaN}).
Konstrukcijo  lahko torej  izpeljemo tako, da najprej načrtamo krožnici $k$ in $l$, nato premico $NM$, oglišči $A$ in $C$ in na koncu še
oglišči $B$ in $D$.
 \kdokaz




         \bzgled
         Construct a triangle $v_a$, $t_a$, $l_a$.
         \ezgled

\begin{figure}[!htb]
\centering
\input{sl.skk.4.3.1e.pic}
\caption{} \label{sl.skk.4.3.1e.pic}
\end{figure}

   \textbf{\textit{Solution.}}
   Naj bo $ABC$ trikotnik, v katerem so višina $AA'$,
težiščnica $AA_1$ in odsek simetrale $AE$ notranjega kota $BAC$ skladni z
daljicami $v_a$, $t_a$ in $l_a$. Z $O$ označimo središče trikotniku $ABC$ očrtane
krožnice $k$. Po izreku \ref{TockaN}
se premici $AE$ in $OA_1$ sekata v točki $N$, ki leži na
krožnici $k$ (Figure \ref{sl.skk.4.3.1e.pic}).

Torej lahko najprej načrtamo
pravokotni trikotnik $AA'E$ s kateto $v_a$ in hipotenuzo $l_a$ ter točko $A_1$ iz pogoja $AA_1\cong t_a$. Nato
načrtamo točko $N$ kot presečišče premice $AE$ in pravokotnice premice $A'E$ skozi
točko $A_1$. Središče $O$ je  presečišče premice $A_1N$ in
simetrale daljice $AN$ (ker je $AN$ tetiva krožnice $k$). Oglišči $B$ in $C$ sta presečišči
krožnice $k(O,OA)$ s premico $A'E$.
   \kdokaz




         \bzgled
         Construct a triangle  $R$, $r$, $a$. \label{konstr_Rra}
         \ezgled


\begin{figure}[!htb]
\centering
\input{sl.skk.4.3.1a.pic}
\caption{} \label{sl.skk.4.3.1a.pic}
\end{figure}


\textbf{\textit{Solution.}} Naj bo $ABC$ takšen trikotnik, da velja
$BC\cong a$ in sta $l(O,R)$ in $k(S,r)$ njegova očrtana oz. včrtana
krožnica
   (Figure \ref{sl.skk.4.3.1a.pic}). Označimo z $\alpha$, $\beta$ in $\gamma$
njegove notranje kote pri ogliščih $A$, $B$ in $C$. Po izreku
\ref{SredObodKot} je $\alpha = \angle BAC = \frac{1}{2}\cdot\angle BOC$.
Iz zgleda \ref{kotBSC} sledi $\angle BSC=90^0+\frac{1}{2}\cdot\alpha$.
Iz dveh relacij dobimo enakost
$\angle BSC=90^0+\frac{1}{4}\cdot\angle BOC$, ki omogoča konstrukcijo.

Najprej načrtamo enakokraki trikotnik $BOC$ ($BC\cong a$ in $OB\cong
OC\cong R$). Točko $S$ dobimo kot eno od presečišč loka s tetivo
$BC$ in obodnim kotom $90^0+\frac{1}{4}\cdot\angle BOC$ ter premico,
ki je na razdalji $r$ vzporedna s premico $BC$. Nato načrtamo
včrtano krožnico $k(S,r)$ in točko $A$ kot presečišče ostalih dveh tangent
te krožnice iz točk $B$ in $C$.
 \kdokaz



        \bnaloga\footnote{47. IMO Slovenia - 2006, Problem 1.}
        Let $ABC$ be a triangle with incentre $I$. A point $P$ in the interior of the
        triangle satisfies
        $$\angle PBA + \angle PCA = \angle PBC + \angle PCB.$$
        Show that $|AP| \geq |AI|$, and that equality holds if and only if $P = I$.
         \enaloga


\begin{figure}[!htb]
\centering
\input{sl.skl.4.3.IMO1.pic}
\caption{} \label{sl.skl.4.3.IMO1.pic}
\end{figure}


\textbf{\textit{Proof.}} Označimo z $\alpha$, $\beta$ in $\gamma$
notranje kote trikotnika $ABC$ pri ogliščih $A$, $B$ in $C$
(Figure \ref{sl.skl.4.3.IMO1.pic}). Pogoj $\angle PBA + \angle PCA
= \angle PBC + \angle PCB$ lahko preoblikujemo v obliko
$\beta-\angle PBC + \gamma-\angle PCB = \angle PBC + \angle PCB$
oziroma:
 $$\angle PBC + \angle PCB=\frac{1}{2}\left( \beta+\gamma\right).$$
Iz tega in dejstva, da je vsota notranjih kotov vsakega izmed
trikotnikov $BPC$ in $ABC$ enaka $180^0$ (izrek\ref{VsotKotTrik}),
sledi:
 $$\angle BPC =180^0-\frac{1}{2}\left( \beta+\gamma\right)=90^0+
 \frac{1}{2} \alpha.$$
Toda iz zgleda \ref{kotBSC} sledi $\angle BIC =90^0+
 \frac{1}{2}\cdot \alpha$, zato je $\angle BPC\cong \angle BIC$.
 Torej točki $P$ in $I$ ležita na istem loku $\mathcal{L}$ s
 tetivo $BC$ in obodnim kotom $90^0+
 \frac{1}{2} \alpha$. Naj bo točka $N$ presečišče
 simetrale stranice $BC$ in simetrale
 notranjega kota $BAC$ trikotnika $ABC$. Po izreku
 \ref{TockaN} leži točka $N$ na očrtani krožnici
  trikotnika $ABC$ in velja $NB\cong NI\cong NC$ (izrek
  \ref{TockaN.NBNC}). To pomeni, da je $N$ središče loka
  $\mathcal{L}$, zato je $NP\cong NI$ oz. $\angle NIP\cong\angle
  NPI<90^0$. Ker so točke $A$, $I$ in $N$ kolinearne (ležijo na
  simetrali notranjega kota $BAC$), je $\angle AIP =180^0-\angle
  NIP>90^0$. Iz izreka \ref{vecstrveckot} (za trikotnik $API$)
  sedaj sledi $|AP| \geq |AI|$ in enakost velja natanko tedaj, ko
  trikotnika $API$ ni oz. kadar je $P=I$.
  \kdokaz


        \bnaloga\footnote{43. IMO United Kingdom - 2002, Problem 2.}
          $BC$ is a diameter of a circle center $O$. $A$ is any point on
        the circle with $\angle AOC>60^0$. $EF$ is the chord which is the perpendicular
        bisector of $AO$. $D$ is the midpoint of the minor arc $AB$. The line through
        $O$ parallel to $AD$ meets $AC$ at $J$. Show that $J$ is the incenter of triangle
        $CEF$.
         \enaloga


\begin{figure}[!htb]
\centering
\input{sl.skk.4.3.IMO4.pic}
\caption{} \label{sl.skk.4.3.IMO4.pic}
\end{figure}


\textbf{\textit{Proof.}} Ker
točki $E$ in $F$ ležita na simetrali daljice $EF$, hkrati pa na
krožnici s središčem $O$, je $$AF\cong FO\cong AO\cong EO \cong EA.$$ To
pomeni, da je štirikotnik $EOFA$ romb, ki je sestavljen iz dveh enakostraničnih trikotnikov $AOF$ in $AEO$.

Brez škode za splošnost predpostavimo, da
je $\angle COE>\angle COF$ (Figure \ref{sl.skk.4.3.IMO4.pic}).
Najprej iz pogoja $\angle
        AOC>60^0$ sledi, da je $\angle COF=60^0- \angle
        AOC>0^0$, zato sta točki $A$ in $F$ na isti strani premice
        $BC$.

 Ker je $AOC$ enakokraki trikotnik z osnovnico $AC$, po izrekih
 \ref{enakokraki} in \ref{zunanjiNotrNotr} velja
  $\angle ACO =\frac{1}{2}\angle AOB$. Točka $D$ je središče
  loka $BD$, zato je $\angle AOD\cong\angle DOB$
  oz. $\angle DOB=\frac{1}{2}\angle AOB$. To pomeni, da je
   $\angle ACO\cong\angle DOB$ in sta po izreku
   \ref{KotiTransverzala} premici $AC$ in $DO$ vzporedni oz.
   $AJ\parallel DO$. Ker je po predpostavki $AD\parallel JO$, je
   štirikotnik $ADOJ$ paralelogram, zato je $AJ\cong OD$. Torej
   $$AJ\cong OD\cong OE\cong AF\cong AE.$$
   Iz $AF\cong AE$ sledi, da
  je $AJ$ simetrala notranjega kota pri oglišču $C$ trikotnika $CEF$
   (izrek \ref{SklTetSklObKot}). Ker je še
  $AJ\cong  AF\cong AE$,
  je po izreku \ref{TockaN.NBNC} točka $J$ središče
        včrtane krožnice tega trikotnika.
\kdokaz


%________________________________________________________________________________
 \poglavje{Cyclic Quadrilateral} \label{odd4Tetivni}

Za večkotnik pravimo, da je
\index{štirikotnik!tetiven}\index{večkotnik!tetiven}\pojem{tetiven},
če zanj obstaja očrtana krožnica, oz. če obstaja krožnica, ki
vsebuje vsa njegova oglišča (Figure \ref{sl.skk.4.5.10.pic}). Za
oglišča v tem primeru pravimo, da so \index{konciklične
točke}\pojem{konciklične točke}.  Ugotovili smo že, da je vsak
trikotnik tetiven (izrek \ref{SredOcrtaneKrozn}) in prav tako, da je
vsak pravilni večkotnik tetiven (izrek \ref{sredOcrtaneKrozVeck}).
Po drugi strani je jasno, da niso vsi večkotniki tetivni. Npr. romb
je štirikotnik, ki nima očrtane krožnice.
V tem razdelku se bomo torej ukvarjali s tetivnimi štirikotniki.

\begin{figure}[!htb]
\centering
\input{sl.skk.4.5.10.pic}
\caption{} \label{sl.skk.4.5.10.pic}
\end{figure}

Ker je kvadrat pravilni večkotnik, je hkrati tetivni štirikotnik. Ni
težko dokazati, da je tudi pravokotnik vrsta tetivnega štirikotnika
- središče očrtane krožnice je presečišče njegovih diagonal, ki sta
skladni in se razpolavljata. Toda kako bi na splošno ugotovili ali
je nek štirikotnik tetiven? Jasno je, da se pri tetivnem
štirikotniku (na splošno tudi večkotniku) simetrale vseh njegovih
stranic sekajo v eni točki (Figure \ref{sl.skk.4.5.10.pic}). Ta pogoj
je zadosten, da je štirikotnik tetiven, ni pa
preveč operativen v konkretnih primerih. Za tetivnost štirikotnikov
obstaja namreč potreben in zadosten pogoj, ki je uporabnejši.



             \bizrek \label{TetivniPogoj}
               A convex quadrilateral is cyclic if and only if
            its opposite interior angles are supplementary.
            Thus, if $\alpha$, $\beta$, $\gamma$ and $\delta$ are
            the interior angles of a convex quadrilateral $ABCD$,
            it is cyclic if and only if
                $$\alpha+\gamma=180^0.$$
            \eizrek

\begin{figure}[!htb]
\centering
\input{sl.skk.4.5.11.pic}
\caption{} \label{sl.skk.4.5.11.pic}
\end{figure}

\textbf{\textit{Proof.}}
 (Figure \ref{sl.skk.4.5.11.pic})

($\Rightarrow$) Predpostavimo najprej, da je štirikotnik $ABCD$
tetiven. Ker je konveksen, sta oglišči $A$ in $C$ na različnih
straneh premice $BD$. Po izreku \ref{ObodObodKotNaspr} je
$\alpha+\gamma=180^0$.

($\Leftarrow)$ Predpostavimo sedaj, da sta nasprotna kota
štirikotnika $ABCD$ suplementarna oz. $\alpha+\gamma=180^0$. Naj bo
$k$ očrtana krožnica trikotnika $ABD$. V tem primeru se iz četrtega
oglišča $C$ tetiva $BD$ vidi pod kotom, ki je suplementaren kotu pri
oglišču $A$, kar pomeni, da tudi točka $C$ leži na krožnici $k$
(izrek \ref{ObodKotGMT}).
  \kdokaz

  A direct consequence is the following theorem.



             \bizrek \label{TetivniPogojZunanji}
              A convex quadrilateral is cyclic if and only if
            one of its interior angles is congruent to the opposite exterior angle.
            Thus, if $\alpha$, $\beta$, $\gamma$ and $\delta$ are
            the interior angles and
            $\alpha'$, $\beta'$, $\gamma'$ in $\delta'$ the exterior angles
             of a convex quadrilateral $ABCD$,
            it is cyclic if and only if
                $$\alpha\cong\gamma'.$$
            \eizrek


\begin{figure}[!htb]
\centering
\input{sl.skk.4.5.12.pic}
\caption{} \label{sl.skk.4.5.12.pic}
\end{figure}

Uporabimo kriterij iz izreka \ref{TetivniPogoj} za paralelogram in
trapez.

            \bizrek \label{paralelogramTetivEnakokr}
            A parallelogram is cyclic if and only if it is a rectangle.
            \eizrek


\textbf{\textit{Proof.}} Naj bodo $\alpha$, $\beta$, $\gamma$ in
$\delta$ notranji koti paralelograma $ABCD$
 (Figure \ref{sl.skk.4.5.13.pic}).

($\Leftarrow$) Če je paralelogram pravokotnik, je
$\alpha+\gamma=90^0+90^0=180^0$, kar pomeni, da je $ABCD$ tetiven
štirikotnik (izrek \ref{TetivniPogoj}).

 ($\Rightarrow$) Predpostavimo, da je $ABCD$ tetivni paralelogram.
 Ker je $ABCD$ paralelogram, je po izreku \ref{paralelogram}
 $\alpha\cong\gamma$. Ker je tudi tetiven, je po izreku \ref{TetivniPogoj}
$\alpha+\gamma=180^0$. Torej velja $\alpha\cong\gamma=90^0$, zato je
  $ABCD$ pravokotnik.
 \kdokaz

\begin{figure}[!htb]
\centering
\input{sl.skk.4.5.13.pic}
\caption{} \label{sl.skk.4.5.13.pic}
\end{figure}



            \bizrek \label{trapezTetivEnakokr}
            A trapezium is cyclic if and only if it is isosceles.
            \eizrek

\textbf{\textit{Proof.}} Naj bo $ABCD$ trapez z osnovnico $AB$ in z
notranjimi koti $\alpha$, $\beta$, $\gamma$ in $\delta$
 (Figure \ref{sl.skk.4.5.13.pic}). V poljubnem trapezu
  velja $\alpha+\delta=180^0$ in $\beta+\gamma=180^0$.


($\Leftarrow$) Predpostavimo, da je trapez $ABCD$ enakokrak oz. $AD
\cong BC$. Po izreku \ref{trapezEnakokraki} je v tem primeru
$\alpha\cong\beta$. Torej $\alpha+\gamma=\beta+\gamma=180^0$, zato
je po izreku \ref{TetivniPogoj} $ABCD$ tetivni štirikotnik.

($\Rightarrow$) Naj bo trapez $ABCD$ tetivni štirikotnik in $k$
njegova očrtana krožnica. Osnovnici $AB$ in $CD$ sta vzporedni
tetivi te krožnice, zato imata skupno simetralo, ki poteka skozi
središče $S$ krožnice $k$ in je pravokotna na tetivi $AB$ in
$CD$. To pomeni, da sta kraka $AD$ in $BC$ simetrična glede na to
simetralo, zato sta med seboj skladna in je trapez $ABCD$  enakokrak.
 \kdokaz

 Posebej so zanimivi tetivni štirikotniki s pravokotnima
 diagonalama\footnote{\index{Brahmagupta}\textit{Brahmagupta} (598--660), indijski matematik, ki je
 preučeval takšne štirikotnike.}.
 Na te štirikotnike se nanaša
naslednji primer.


            \bzgled \label{TetivniLemaBrahm}
            Suppose that the diagonals of a cyclic quadrilateral $ABCD$ are perpendicular and intersect
            at a point $S$. Prove that the foot of the perpendicular from the point $S$ on the line $AB$
            contains the midpoint of the line $CD$.
            \ezgled

\begin{figure}[!htb]
\centering
\input{sl.skk.4.5.14.pic}
\caption{} \label{sl.skk.4.5.14.pic}
\end{figure}

\textbf{\textit{Proof.}}
 (Figure \ref{sl.skk.4.5.14.pic})

Označimo z $N$ in $M$ presečišči pravokotnice na premico $AB$ skozi
točko $S$ s stranicama $AB$ in $CD$ štirikotnika $ABCD$. Tedaj velja:
 \begin{eqnarray*}
 \angle CDB &\cong& \angle CAB \hspace*{3mm}
 \textrm{(obodna kota za ustrezni lok } CB
 \textrm{ - izrek \ref{ObodObodKot}})\\
      &\cong& \angle NSB  \hspace*{3mm}
 \textrm{ (kota s
pravokotnima krakoma - izrek \ref{KotaPravokKraki})}\\
     &\cong& \angle MSD  \hspace*{3mm}
 \textrm{(sovršna kota)}
 \end{eqnarray*}
 Ker je $\angle CDB\cong \angle MSD$, je $MD \cong MS$ (izrek \ref{enakokraki}).
 Analogno je tudi $MC \cong MS$. Torej je
 $MD \cong MC$, kar pomeni, da je $M$ središče stranice $CD$.
 \kdokaz

Eno lastnost tetivnih štirikotnikov s pravokotnima diagonalama
bomo obravnavali še v zgledu \ref{HamiltonPoslTetiv}.



             \bzgled \label{TetŠtirZgl0}
             Let $k$ be the circumcircle of a cyclic quadrilateral $ABCD$
            and $N$, $M$, $L$ and $P$ the midpoints of those arcs $AB$, $B$C, $CD$ and $AD$
            of the circle $k$, not containing the third vertices of this quadrilateral.
            Prove that $NL\perp PM$.
            \ezgled


\begin{figure}[!htb]
\centering
\input{sl.skk.4.5.0.pic}
\caption{} \label{sl.skk.4.5.0.pic}
\end{figure}

\textbf{\textit{Proof.}} Naj bo $S$ presečišče premic $NL$ in $PM$
(Figure \ref{sl.skk.4.5.0.pic}).
 Če dvakrat uporabimo izreka \ref{ObodObodKot} in \ref{TockaN}, dobimo:

 \begin{eqnarray*}
  \angle PNS &=& \angle PND +\angle DNL =
\angle PBD +\angle DBL =\\
 &=& \frac{1}{2} \angle ABD +\frac{1}{2}\angle CBD = \frac{1}{2}\angle
 ABC.
 \end{eqnarray*}

 Na enak način dokažemo tudi $\angle NPS = \frac{1}{2}\angle
 ADC$. Zato je po izreku \ref{TetivniPogoj}:
 $$\angle PNS +\angle NPS = \frac{1}{2} \left(\angle ABC+\angle
 ADC\right)=90^0.$$
 Če izrek \ref{VsotKotTrik} uporabimo za trikotnik $PSN$, dobimo
 $\angle PSN = 90^0$.
  \kdokaz


              \bzgled \label{TetivniVcrtana}
             Let $ABCD$ be a cyclic quadrilateral.
             Prove that incentres of the triangles $BCD$, $ACD$, $ABD$ and $ABC$
             are the vertices of a rectangle.
             \ezgled

\begin{figure}[!htb]
\centering
\input{sl.skk.4.5.1.pic}
\caption{} \label{sl.skk.4.5.1.pic}
\end{figure}

\textbf{\textit{Proof.}} Označimo z $A_1$, $B_1$, $C_1$ in $D_1$
središča včrtanih krožnic trikotnikov $BCD$, $ACD$, $ABD$ in $ABC$
ter z $N$, $M$, $L$ in $P$ središča tistih lokov $AB$, $BC$, $CD$ in
$AD$ očrtane krožnice štirikotnika $ABCD$, ki ne vsebujejo ostalih
oglišč tega štirikotnika (Figure \ref{sl.skk.4.5.1.pic}). Iz zgleda
\ref{TockaN} sledi, da sta $BL$ in $DM$ simetrali kotov $CBD$ in
$BDC$, zato je točka $A_1$ presečišče premic $BL$ in $DM$. Analogno
je točka $B_1$ presečišče premic $CP$ in $AL$. Po zgledu
\ref{TockaN.NBNC} je
 $LC\cong LA_1\cong LB_1\cong LD$, torej je $A_1LB_1$ enakokraki trikotnik
 z osnovnico $A_1B_1$. Iz zgleda \ref{TockaN} sledi tudi,
 da je $LN$ simetrala kota $ALB$ oz. $B_1LA_1$. V enakokrakem
 trikotniku $A_1LB_1$ simetrala kota $B_1LA_1$ vsebuje višino
 tega trikotnika iz točke $L$. To pomeni, da velja $LN\perp
 A_1B_1$. Analogno sledi tudi $LN\perp C_1D_1$,  $PM\perp A_1D_1$
 in
 $PM\perp C_1B_1$. Iz prejšnjega zgleda \ref{TetŠtirZgl0} je $LN\perp PM$,
 zato je štirikotnik $A_1B_1C_1D_1$ pravokotnik.
  \kdokaz

Omenili smo že, da je pravokotnik tetivni štirikotnik. Sedaj bomo
dokazali zanimivo lastnost pravokotnika, ki se nanaša na točke,
ki ležijo na njegovi očrtani krožnici.



            \bzgled
             Let $P$ be an arbitrary point on the shorter arc $AB$ of the circumcircle of a rectangle $ABCD$.
            Suppose that $L$ and $M$ are the foots of the perpendiculars from the point $P$ on the
            diagonals $AC$ and $BD$, respectively. Prove that the length of the line segment $LM$ does not depend
            on the position of the point $P$.
            \ezgled


\begin{figure}[!htb]
\centering
\input{sl.skk.4.5.15.pic}
\caption{} \label{sl.skk.4.5.15.pic}
\end{figure}

\textbf{\textit{Proof.}}
 Naj bo točka $O$ središče krožnice $k$ (Figure \ref{sl.skk.4.5.15.pic}).
Štirikotnik $PMOL$ je tetiven, ker je $\angle OLP + \angle OMP
=90^0+90^0= 180^0$ (izrek \ref{TetivniPogoj}). Označimo z $l$
očrtano krožnico tega štirikotnika. Ker sta kota $OLP$ in $OMP$ oba
prava, je daljica $OP$ (oz. polmer krožnice $k$) premer krožnice
$l$. Potem je $LM$ tetiva krožnice $l$, ki pripada obodnemu kotu
$\angle LOM =\angle AOB$, ki je konstanten. Ne glede na izbiro
točke $P$ je daljica $LM$ tetiva krožnice s konstantnim premerom
$OA$, ki pripada konstantnem obodnem kotu $AOB$ (oziroma ustreznemu
konstantnemu središčnem kotu te krožnice). Tetivi, ki pripadata
skladnima središčnima kotoma skladnih krožnic, sta med seboj skladni,
zato dolžina daljice $LM$ ni odvisna od lege točke $P$.
 \kdokaz

  Lastnosti tetivnega štirikotnika pogosto uporabljamo tudi za dokazovanje
  različnih
lastnosti trikotnika.


            \bzgled \label{PedalniVS}
            The orthocentre of an acute triangle is the incentre of its \index{trikotnik!pedalni} pedal triangle.
            \ezgled

\begin{figure}[!htb]
\centering
\input{sl.skk.4.5.16.pic}
\caption{} \label{sl.skk.4.5.16.pic}
\end{figure}

\textbf{\textit{Proof.}}
 Naj bodo $AA'$, $BB'$ in $CC'$ višine trikotnika $ABC$, ki se sekajo
 v višinski točki $V$ tega
trikotnika (Figure \ref{sl.skk.4.5.16.pic}). Če je $A_1$ središče
stranice $BC$, točki $B'$ in $C'$ ležita na krožnici $k(A_1,A_1B)$
(izrek \ref{TalesovIzrKroz2}). Torej je štirikotnik $BC'B'C$
tetiven, zato je po izreku \ref{TetivniPogojZunanji} $\angle
AC'B'\cong \angle ACB = \gamma$. Analogno dokažemo, da je
štirikotnik $AC'A'C$ tetiven, zato je tudi $\angle BC'A'\cong
\angle ACB = \gamma$. Torej sta kota $AC'B'$ in $BC'A'$ skladna. Ker
je $CC'\perp AB$, sta skladna tudi kota $CC'B'$ in $CC'A'$. To
pomeni, da je premica $C'C$ simetrala kota $A'C'B'$. Analogno sta
tudi premici $A'A$ in $B'B$ simetrali ustreznih notranjih kotov
trikotnika $A'B'C'$, zato je točka $V$ središče trikotniku $A'B'C'$
včrtane krožnice.
 \kdokaz

 Iz dokaza prejšnje trditve (\ref{PedalniVS}) lahko ugotovimo, da so koti, ki jih določajo stranice
pedalnega trikotnika $A'B'C'$  s stranicami trikotnika
$ABC$, enaki ustreznim kotom trikotnika $ABC$. To dejstvo bomo
uporabili v naslednjem primeru.



            \bzgled \label{PedalniLemaOcrtana}
             Let $O$ be the circumcentre of a triangle $ABC$.
            Prove that the lines $OA$, $OB$ and $OC$ are perpendicular to the corresponding sides of the
            pedal triangle $A'B'C'$.
            \ezgled


\begin{figure}[!htb]
\centering
\input{sl.skk.4.5.17.pic}
\caption{} \label{sl.skk.4.5.17.pic}
\end{figure}

\textbf{\textit{Proof.}} (Figure \ref{sl.skk.4.5.17.pic}).

Z $L$ označimo presečišče premic $OA$ in $B'C'$. Dovolj je dokazati,
da je notranji kot pri oglišču $L$ trikotnika $C'LA$ pravi kot.
Izračunajmo druga dva kota tega trikotnika. Iz prejšnjega zgleda
\ref{PedalniVS} je kot pri oglišču $C'$ enak $\gamma$. Trikotnik
$AOB$ je enakokrak in $\angle AOB=2\gamma$ (izrek \ref{SredObodKot}).
Torej velja (izreka \ref{enakokraki} in \ref{VsotKotTrik})
 $\angle C' AL=\angle BAO =90^0-\gamma$,
zato je $\angle ALC'=90^0$.
 \kdokaz

Direktna posledica trditev \ref{PedalniVS} in \ref{PedalniLemasPQR}
je naslednja trditev.



            \bzgled \label{PedalniLemasLMN}
            Let $P$, $Q$ and $R$ be the midpoints of those arcs $BC$, $AC$ and $AB$
            of the circumcircle of a triangle $ABC$ not containing the vertices $A$, $B$ and $C$.
            Suppose that the point $S$ is the incentre of the triangle $ABC$
             and $L=SA\cap QR$, $M=SB\cap PR$ and $N=SC\cap PQ$. Then the triangles $LMN$ and $ABC$ have
            the common incentre.
             (Figure \ref{sl.skk.4.5.18.pic}).
            \ezgled


\begin{figure}[!htb]
\centering
\input{sl.skk.4.5.18.pic}
\caption{} \label{sl.skk.4.5.18.pic}
\end{figure}



             \bzgled \label{Miquelova točka}
             Let $P$, $Q$ and $R$ be an arbitrary points on the sides $BC$, $AC$ and $AB$
             of the triangle $ABC$, respectively. Prove that the circumcircles of triangles
              $AQR$, $PBR$ and $PQC$ intersect at in one point (so-called \index{točka!Miquelova}
            \pojem{Miquel point}\color{green1}\footnote{Točko imenujemo po
            francoskem matematiku \index{Miquel, A.} \textit{A. Miquelu} (1816–-1851), ki je
            to trditev objavil leta 1838 kot članek v Liouvilleovem
            (\index{Liouville, J.}\textit{J. Liouville} (1809–-1882), francoski
            matematik) časopisu. Toda, kot je to pogosto pri matematiki, Miquel
            ni bil prvi, ki je omenjeni izrek dokazal. Že deset let pred njim je
            to dejstvo odkril in objavil znani švicarski matematik
            \index{Steiner, J.} \textit{J. Steiner} (1769--1863).}).
            \ezgled


\begin{figure}[!htb]
\centering
\input{sl.skk.4.5.2.pic}
\caption{} \label{sl.skk.4.5.2.pic}
\end{figure}

\textbf{\textit{Proof.}} (Figure \ref{sl.skk.4.5.2.pic})

Označimo  s $k_A$, $k_B$ in $k_C$ očrtane krožnice trikotnikov
$AQR$, $PBR$ in $PQC$ ter notranje kote trikotnika $ABC$ po vrsti z
$\alpha$, $\beta$ in $\gamma$. Naj bo $S$ drugo presečišče krožnic
$k_B$ in $k_C$ (dokaz je podoben tudi v primeru, če je $S = P$).
Štirikotnika $BPSR$ in $PCQS$ sta tetivna, zato je $\angle RSP =
180^0 - \beta$ in $\angle QSP = 180^0 -\gamma$ (izrek
\ref{TetivniPogoj}). Iz tega sledi $\angle RSQ = \beta +\gamma$ in
potem tudi $\angle RAQ + \angle RSQ =\alpha + \beta +\gamma =
180^0$. Tudi štirikotnik $ARSQ$ je tetiven (izrek
\ref{TetivniPogoj}) oz. ima svojo očrtano krožnico, ki je
pravzaprav  krožnica $k_A$, ki je očrtana trikotniku $AQR$.
To pomeni, da se krožnice $k_A$, $k_B$ in $k_C$ sekajo v točki $S$.
 \kdokaz

 V poglavju \ref{pogINV} bomo dokazali eno posplošitev prejšnje
 trditve (zgled \ref{MiquelKroznice}).



        \bnaloga\footnote{45. IMO Greece - 2004, Problem 1.}
          Let $ABC$ be an acute-angled triangle with $AB\neq AC$. The
circle with diameter $BC$ intersects the sides $AB$ and $AC$ at $M$ and $N$,
respectively. Denote by $O$ the midpoint of the side $BC$. The bisectors of
the angles $\angle BAC$ and $\angle MON$ intersect at $L$. Prove that the circumcircles
of the triangles $BML$ and $CNL$ have a common point lying on the side
$BC$.
         \enaloga


\begin{figure}[!htb]
\centering
\input{sl.skk.4.4.IMO1.pic}
\caption{} \label{sl.skk.4.4.IMO1.pic}
\end{figure}


\textbf{\textit{Proof.}} Označimo z $E$ presečišče simetrale kota
$BAC$ s stranico $BC$ trikotnika $ABC$ (Figure
\ref{sl.skk.4.4.IMO1.pic}). Dokažimo, da je $E$ iskana točka  oz.
da leži na očrtanih krožnicah obeh trikotnikov $BML$
        in $CNL$.

Ker iz konstrukcije točk $M$ in $N$ sledi $OM\cong ON$, je $OMN$
enakokraki trikotnik z osnovnico $MN$. To pomeni, da je simetrala
$OL$ kota $MON$ hkrati simetrala stranice $MN$ (sledi iz
skladnosti trikotnikov $MSO$ in $NSO$, kjer je $S$ središče
daljice $MN$). Torej točka $L$ leži na simetrali daljice $MN$
 trikotnika, zato po izreku \ref{TockaN} leži na očrtani krožnici
 $k$
 trikotnika $AMN$. Pogoj $AB\neq AC$ nam pove, da se simetrali kota $BAC$ in
 stranice $MN$ (oz. kota $MON$) razlikujeta, zato je njun presek
 točka.

  Če  uporabimo izreka \ref{TetivniPogojZunanji} in \ref{ObodObodKot},
 dobimo:
  \begin{eqnarray*}
   \angle BCA &\cong& AMN \cong\angle ALN,\\
   \angle ABC &\cong& ANM \cong\angle ALM.
  \end{eqnarray*}
Iz teh relacij in izreka  \ref{TetivniPogojZunanji} sledi, da sta
 $NLEC$ in $LMBE$ tetivna
štirikotnika. Torej točka $E$ leži na  očrtanih krožnicah
obeh trikotnikov $BML$
        in $CNL$.
 \kdokaz

%________________________________________________________________________________
  \poglavje{Tangential Quadrilateral} \label{odd4Tangentni}



Za večkotnik pravimo, da je
\index{štirikotnik!tangenten}\index{večkotnik!tangenten}\pojem{tangenten},
če za njega obstaja včrtana krožnica, oziroma če obstaja takšna
krožnica, da so nosilke vseh stranic večkotnika njene tangente
(Figure \ref{sl.skk.4.6.1.pic}). Ugotovili smo že, da je vsak
trikotnik tangenten (izrek \ref{SredVcrtaneKrozn}) in prav tako
je tangenten tudi vsak pravilni večkotnik (izrek
\ref{sredVcrtaneKrozVeck}). Po drugi strani pa je jasno, da vsi
večkotniki niso tangentni. Na primer pravokotnik je štirikotnik, ki nima
včrtane krožnice. V tem razdelku se bomo posebej ukvarjali s
tangentnimi štirikotniki.

\begin{figure}[!htb]
\centering
\input{sl.skk.4.6.1.pic}
\caption{} \label{sl.skk.4.6.1.pic}
\end{figure}

Ker je kvadrat pravilni večkotnik, je tudi tangentni štirikotnik.
 Kako pa bi na
splošno ugotovili, ali je nek štirikotnik tangenten? Jasno je, da se
pri tangentnem štirikotniku (na splošno tudi večkotniku) simetrale
vseh njegovih notranjih kotov sekajo v eni točki (Figure
\ref{sl.skk.4.6.1.pic}). Ta pogoj je za tangentnost večkotnika tudi zadosten.
Žal pa ta pogoj ni preveč uporaben v
konkretnih primerih. Obstaja namreč uporabnejši pogoj, ki je za tangentnost štirikotnikov potreben in
hkrati zadosten.



             \bizrek \label{TangentniPogoj}
              A quadrilateral $ABCD$ is tangential if and only if
               $$|AB| + |CD| = |BC| + |AD|.$$
            \eizrek

\begin{figure}[!htb]
\centering
\input{sl.skk.4.6.2.pic}
\caption{} \label{sl.skk.4.6.2.pic}
\end{figure}


\textbf{\textit{Proof.}}  (Figure \ref{sl.skk.4.6.2.pic})

 ($\Rightarrow$) Predpostavimo najprej, da je štirikotnik $ABCD$ tangenten in
  $k$ njegova včrtana krožnica.
 Naj bodo $P$, $Q$, $R$ in $S$
dotikališča stranic $AB$, $BC$, $CD$ in $DA$ s krožnico $k$. Ker so
ustrezne tangentne daljice skladne (izrek \ref{TangOdsek}), velja:
$AP \cong AS$, $BP \cong BQ$, $CQ \cong CR$ in $DR \cong DS$. Zato
je:
 \begin{eqnarray*}
|AB| + |CD|&=&|AP| + |PB| + |CR| + |RD| \\&=& |AS| + |SD| + |BQ| +
|QC|\\&=&|AD| + |BC|.
 \end{eqnarray*}

 ($\Leftarrow$) Dokažimo obratno trditev. Predpostavimo, da sta v štirikotniku
 $ABCD$ vsoti parov nasprotnih stranic enaki, oz. da velja
 $|AB| + |CD| = |BC| + |AD|$. Obstaja krožnica $k$, ki se dotika
stranic $AB$, $BC$ in $DA$ tega štirikotnika (njeno središče je
presečišče simetral notranjih kotov pri ogliščih $A$ in $B$ tega
štirikotnika). Dokažimo še, da se ta krožnica dotika tudi stranice
$CD$ štirikotnika $ABCD$. Naj bo $D'$ presečišče druge tangente iz
točke $C$ krožnice $k$ in premice $AD$. Predpostavimo, da je $D'\neq
D$. Brez škode za splošnost naj bo $\mathcal{B}(A,D',D)$. Ker je
štirikotnik $ABCD'$ tangenten z včrtano krožnico $k$, po že dokazanem
delu izreka velja $|AB| + |CD'| = |AD'|+|BC|$. Ker pa je po
predpostavki še $|AB| + |CD| = |AD| + |BC|$, velja tudi
$|CD|-|CD'| = |DA| - |D'A|=|DD'|$ oz.  $|CD|= |CD'| + |DD'|$. To pa
ni mogoče zaradi trikotniške neenakosti \ref{neenaktrik} (točke $C$,
$D$ in $D'$ ne morejo biti kolinearne, ker bi sicer veljalo $C\in
AD$). Na podoben način pridemo do protislovja tudi v primeru, če je
$\mathcal{B}(A,D,D')$. Torej velja $D'= D$, zato je $ABCD$ tangenten
štirikotnik.
 \kdokaz

 Direktna posledica prejšnjega kriterija sta naslednja izreka.

        \bizrek \label{TangDeltoidRomb}
        A rhombus, a deltoid, and a square are
         tangential quadrilaterals (Figure \ref{sl.skk.4.6.3.pic}).
        \eizrek

\begin{figure}[!htb]
\centering
\input{sl.skk.4.6.3.pic}
\caption{} \label{sl.skk.4.6.3.pic}
\end{figure}

         \bizrek \label{TangParalelogram}
          A parallelogram is a tangential quadrilateral
          if and only if it is a rhombus (Figure \ref{sl.skk.4.6.3.pic}).
        \eizrek

V naslednjih dveh primerih bomo hkrati obravnavali
tetivne in tangentne štirikotnike.



        \bzgled Let $k_A$, $k_B$, $k_C$ and $k_D$ circles with centres $A$,
        $B$, $C$ and $D$, such that two in a row (also $k_A$ and $k_D$)
        are touching each other externally. Prove that the quadrilateral defined by
        the touching points of circles is cyclic, and the quadrilateral $ABCD$ is tangential.
        \ezgled

\begin{figure}[!htb]
\centering
\input{sl.skk.4.6.4.pic}
\caption{} \label{sl.skk.4.6.4.pic}
\end{figure}


\textbf{\textit{Proof.}}
 Naj bodo $P$, $Q$, $R$ in $S$ po vrsti dotikališča
danih krožnic ter $p$ in $r$ skupni tangenti ustreznih
krožnic v točkah $P$ in $R$ (Figure \ref{sl.skk.4.6.4.pic}).

Najprej velja:
 \begin{eqnarray*}
 |AD| + |BC| &=& |AP| + |PD| + |BR| + |RC| =\\
  &=& |AQ| + |SD| + |QB| + |SC| =\\
  &=&  |AB| +
|CD|.
 \end{eqnarray*}
  Zaradi tega je $ABCD$ tangentni štirikotnik (izrek \ref{TangentniPogoj}).

Tangenti $p$ in $r$ delita notranja kota pri ogliščih $P$ in $R$
štirikotnika $PQRS$ na kote, od katerih je vsak enak polovici
ustreznega središčnega kota (izrek \ref{ObodKotTang}). Omenjeni
središčni koti pa so notranji koti štirikotnika $ABCD$. Označimo
jih z $\alpha$, $\beta$, $\gamma$ in $\delta$. Zaradi tega je (izrek
\ref{VsotKotVeck}):
 \begin{eqnarray*}
 \angle QPS+ \angle SRQ&=& \angle QP,p+\angle p,PS+ \angle SR,r+\angle r,RQ=\\
  &=& \frac{1}{2}\alpha+\frac{1}{2}\delta+\frac{1}{2}\gamma+\frac{1}{2}\beta=\\
  &=& \frac{1}{2}\left(\alpha+\delta+\gamma+\beta\right)=\\
  &=& \frac{1}{2}\cdot360^0=180^0,
 \end{eqnarray*}
kar pomeni, da je $PQRS$ tetivni štirikotnik.
  \kdokaz

 Jasno je, da je včrtana
 krožnica štirikotnika $ABCD$ hkrati očrtana krožnica štirikotnika $PQRS$.
 Ker so namreč po predpostavki
 $PAQ$, $QBR$, $RCS$ in $SDP$  enakokraki trikotniki z osnovnicami
 $PQ$, $QR$, $RS$ in $SP$, so simetrale notranjih kotov štirikotnika $ABCD$
 hkrati simetrale stranic štirikotnika $PQRS$ (Figure
 \ref{sl.skk.4.6.4a.pic}). To je tudi drugi (enostavnejši)
 način dokaza drugega dela prejšnjega zgleda - trditve, da je $PQRS$
 tetivni štirikotnik.

\begin{figure}[!htb]
\centering
\input{sl.skk.4.6.4a.pic}
\caption{} \label{sl.skk.4.6.4a.pic}
\end{figure}



            \bzgled \label{tetivTangLema}
            Let $L$ be the intersection of the diagonals of a cyclic quadrilateral  $ABCD$.
            Prove that the foots of the perpendiculars from the point $L$ on the sides of
            this quadrilateral are the vertices of a tangential quadrilateral.
            \ezgled


\begin{figure}[!htb]
\centering
\input{sl.skk.4.6.5.pic}
\caption{} \label{sl.skk.4.6.5.pic}
\end{figure}


        \textbf{\textit{Proof.}}
  Naj bodo $P$, $Q$, $R$ in $S$ pravokotne projekcije iz točke $L$ na stranicah
   $AB$, $BC$, $CD$ in $DA$
štirikotnika $ABCD$ (Figure \ref{sl.skk.4.6.5.pic}).
Zaradi ustreznih pravih kotov sta $PBQL$ in $APLS$
tetivna štirikotnika (izrek \ref{TetivniPogoj}). Po predpostavki je tetiven tudi štirikotnik $ABCD$. Če to
uporabimo, dobimo enakost ustreznih obodnih kotov (izrek \ref{ObodObodKot}). Torej:
$$\angle SPL \cong \angle SAL = \angle DAC \cong \angle DBC
= \angle LBQ \cong \angle LPQ.$$
Iz tega sledi, da je premica $PL$ simetrala notranjega kota pri
oglišču $P$ štirikotnika $PQRS$. Podobno so premice $QL$, $RL$ in $SL$
simetrale ostalih treh notranjih kotov tega štirikotnika. Torej je $L$ središče včrtane krožnice štirikotnika $PQRS$,
zato je ta tangenten.
        \kdokaz

Dokažimo še eno zanimivo lastnost tangentnih štirikotnikov.



            \bzgled
            Prove that the incircles of triangles $ABC$ and $ACD$ touch each
            other if and only if $ABCD$ is a tangential quadrilateral.
            \ezgled


\begin{figure}[!htb]
\centering
\input{sl.skk.4.6.6.pic}
\caption{} \label{sl.skk.4.6.6.pic}
\end{figure}


        \textbf{\textit{Proof.}}
   (Figure \ref{sl.skk.4.6.6.pic}).

   Dokažimo najprej nekatere relacije, ki veljajo pri poljubnem konveksnem štirikotniku $ABCD$.
Naj bodo $P$, $Q$ in $X$ točke, v katerih se včrtana krožnica $k$ trikotnika $ABC$ dotika njegovih stranic
$AB$, $BC$ in $CA$, ter $R$, $S$ in $Y$  točke, v katerih se včrtana krožnica $l$ trikotnika $ACD$ dotika njegovih stranice $CD$, $DA$ in $AC$. Najprej velja (izrek \ref{TangOdsek}):
 \begin{eqnarray*}
 |AX| &=& |AP|=\frac{1}{2}\left(|AX|+|AP|\right)=\frac{1}{2}\left(|AC|-|CX|+|AB|-|BP|\right)\\
  &=& \frac{1}{2}\left(|AC|-|CQ|+|AB|-|BQ|\right)=
  \frac{1}{2}\left(|AC|+|AB|-|BC|\right),
 \end{eqnarray*}
 torej velja:
  $$|AX|=\frac{1}{2}\left(|AC|+|AB|-|BC|\right).$$
  Na enak način dokažemo, da velja tudi:
  $$|AY|=\frac{1}{2}\left(|AC|+|AD|-|DC|\right).$$
  Sedaj lahko začnemo z dokazovanjem ekvivalence.

Krožnici $k$ in $l$ se dotikata natanko tedaj, ko je $X = Y$ oz. $|AX| = |AY|$. Zadnja enakost velja natanko tedaj, ko je: $$\frac{1}{2}\left(|AC|+|AB|-|BC|\right)=\frac{1}{2}\left(|AC|+|AD|-|DC|\right)$$
 oz. $|AB| + |DC| = |AD| + |BC|$, ta pa je izpolnjena natanko tedaj, ko je $ABCD$ tangenten štirikotnik (izrek \ref{TangentniPogoj}).
   \kdokaz

Posledica tega je naslednja trditev.



            \bzgled
            Let $ABCD$ be a tangential quadrilateral.
            Then the  incircles of the triangles $ABC$ and $ACD$ touch each other
            if and only if
            the  incircles of the triangles $ABD$ and $CBD$ touch each
            other (Figure \ref{sl.skk.4.6.6a.pic}).
            \ezgled


\begin{figure}[!htb]
\centering
\input{sl.skk.4.6.6a.pic}
\caption{} \label{sl.skk.4.6.6a.pic}
\end{figure}


        \textbf{\textit{Proof.}} Trditvi, da se včrtani krožnici trikotnikov $ABC$ in $ACD$ oz. trikotnikov $ABD$ in $CBD$ dotikata, sta
        ekvivalentni s trditvijo, da je štirikotnik $ABCD$ tangenten. To pomeni, da sta tudi začetni trditvi ekvivalentni.
         \kdokaz

%_______________________________________________________________________________
 \poglavje{Bicentric Quadrilateral} \label{odd4TetivniTangentni}

  Nekateri štirikotniki so hkrati tetivni in tangentni. Imenujemo jih \index{štirikotnik!tetivnotangentni}\pojem{tetivnotangentni} ali \index{štirikotnik!bicentrični}\pojem{bicentrični} štirikotniki. Kateri
 štirikotniki so to? Kvadrat je vsekakor eden tak. Ali je edini? Odgovor je negativen. Štirikotnik, ki ga dobimo iz zgleda \ref{tetivTangLema}, je vedno tangenten. V določenem primeru bo veljalo, da je ta tudi tetiven.
 Velja namreč naslednja trditev.



                \bizrek \label{tetivTangIzrek}
                If $L$ is the intersection of the perpendicular diagonals of a cyclic quadrilateral
            $ABCD$, then the foots of the perpendiculars from the point $L$ on the sides of
            this quadrilateral are the vertices of a bicentric quadrilateral.
                \eizrek



\begin{figure}[!htb]
\centering
\input{sl.skk.4.7.2.pic}
\caption{} \label{sl.skk.4.7.2.pic}
\end{figure}


        \textbf{\textit{Proof.}}
   (Figure \ref{sl.skk.4.7.2.pic})

Vpeljimo iste oznake kot v zgledu \ref{tetivTangLema}. Dokazali smo že, da je $PQRS$ tangenten  štirikotnik. Dokažimo še, da je tetiven. Iz dokaza trditve iz omenjenega zgleda \ref{tetivTangLema} sledi:
\begin{eqnarray*}
        \angle SPQ &=& \angle SPL+\angle LPQ = \angle SAL+\angle LBQ =\\
          &=& \angle DAC + \angle DBC
        = 2\cdot\angle DBC
\end{eqnarray*}
 oziroma $\angle SPQ= 2\cdot\angle DBC$. Analogno je tudi $\angle SRQ= 2\cdot\angle ACB$.
   Ker je po predpostavki $AC\perp BD$, je $CLB$ pravokotni trikotnik, zato velja:
  $$\angle SRQ+\angle SRQ=2\cdot(\angle DBC+\angle ACB)=2\cdot 90^0=180^0.$$
   Po izreku \ref{TetivniPogoj} je $PQRS$ tetivni štirikotnik.
   \kdokaz

 Ni se težko prepričati, da velja tudi obratna trditev.



        \bizrek
        Let $PQRS$ be a bicentric quadrilateral. Suppose that point $L$ is
        the incentre of this quadrilateral and at the same time the intersection of the diagonals
        of a cyclic quadrilateral $ABCD$. If $P$, $Q$, $R$ and $S$ the foots of the perpendiculars
        from the point $L$ on the sides of quadrilateral $ABCD$, then $AC\perp BD$.
        \eizrek

V naslednji nalogi bomo videli, da za tri nekolinearne točke $A$, $B$ in $C$ obstaja ena sama točka $D$, tako da je $ABCD$ tetivnotangentni štirikotnik.


            \bnaloga\footnote{4.
            IMO Czechoslovakia - 1962, Problem 5.}
             On the circle $k$ there are given three distinct points $A$, $B$, $C$. Construct (using
            only straightedge and compasses) a fourth point $D$ on $k$ such that a circle
            can be inscribed in the quadrilateral thus obtained.
            \enaloga

\begin{figure}[!htb]
\centering
\input{sl.skk.4.5.IMO1.pic}
\caption{} \label{sl.skk.4.5.IMO1.pic}
\end{figure}


\textbf{\textit{Solution.}} Brez škode za splošnost predpostavimo,
da je $AB\geq BC$.

Naj bo $D$ točka, ki izpolnjuje pogoje iz naloge oz. takšna, da je
$ABCD$ tetivnotangentni štirikotnik (Figure
\ref{sl.skk.4.5.IMO1.pic}). Označimo z $a$, $b$, $c$ in $d$ po vrsti
stranice $AB$, $BC$, $CD$ in $DA$ tega štirikotnika ter z $\alpha$,
$\beta$, $\gamma$ in $\delta$ njegove notranje kote pri ogliščih
$A$, $B$, $C$ in $D$. Iz pogoja tetivnosti štirikotnika $ABCD$
(izrek \ref{TetivniPogoj}) sledi $\delta=180^0-\beta$, iz njegove
tangentnosti (izrek \ref{TangentniPogoj}) pa $a+c=b+d$ oz.
$d-c=a-b$. Na ta način nalogo prevedemo v načrtovanje tretjega
oglišča $D$ trikotnika $ACD$, kjer so dani stranica $AC$, kot
$\angle ADC=180^0-\beta$ in razlika stranic $AD-CD=AB-BC=a-b$. Naj bo
$E$ točka na daljici $AD$, za katero velja $DE\cong DC$. Potem je
$AE=AD-DE=AD-CD=a-b$. Trikotnik $ECD$ je enakokrak, zato po izreku
\ref{enakokraki} sledi $\angle CED\cong\angle DCE$. Zato je $\angle
AEC=180^0-\angle DEC=180^0-\frac{1}{2}\beta$. To nam omogoča
konstrukcijo trikotnika $ACE$ oz. točke $E$.

Najprej načrtajmo točko $E$ kot presečišče loka $l$ (glej konstrukcijo,
ki je opisana v izreku \ref{ObodKotGMT}) $180^0-\frac{1}{2}\angle ABC$) ter krožnice
$j(A,AB-BC)$ (če je $AB\cong AC$, privzamemo $E=A$). Točko $D$
nato lahko načrtamo kot presečišče poltraka $AE$ in simetrale
$s_{EC}$ daljice $EC$ (v primeru $AB\cong AC$, oz. $E=A$ je $D$
drugo presečišče simetrale $s_{EC}=s_{AC}$ s krožnico $k$).

Dokažimo, da točka $D$ izpolnjuje pogoje iz naloge oz. da je $ABCD$
tetivnotangentni štirikotnik. Najprej bomo obravnavali primer, ko
je $AB>BC$.

Po konstrukciji točka $D$ leži na simetrali daljice $EC$, zato je
$DE\cong DC$ in tudi (izrek \ref{enakokraki}) $\angle DEC\cong\angle
DCE$. Točko $E$ smo načrtali tako, da leži na loku $l$ s tetivo $AC$ in
obodnim kotom $180^0-\frac{1}{2}\angle ABC$, zato je $\angle
AEC=180^0-\frac{1}{2}\angle ABC$. Ker je po konstrukciji
$\mathcal{B}(A,E,D)$, je $\angle DCE\cong\angle
DEC=\frac{1}{2}\angle ABC$. Iz enakokrakega trikotnika $EDC$
 po izreku \ref{VsotKotTrik} sledi $\angle ADC=\angle EDC=180^0-\angle
 ABC$. Torej velja $\angle EDC+\angle ABC=180^0$,
 zato je po izreku \ref{TetivniPogoj} $ABCD$ tetivni
 štirikotnik oz. $D\in k$.

 Dokažimo še, da je $ABCD$ tangentni štirikotnik. V prvem delu
  dokaza (tetivnost) smo že
 videli, da velja $DE\cong DC$. Točka $E$ po konstrukciji leži
 na krožnici $j(A,|AB-BC|)$, zato je $|AE|=|AB|-|BC|$. Ker velja še
 $\mathcal{B}(A,E,D)$, je $|AD|-|CD|=|AD|-|DE|=|AE|=|AB|-|BC|$. Iz tega sledi
 $|AD|+|BC|=|AB|+|CD|$ in je po izreku \ref{TangPogoj} štirikotnik
 $ABCD$ tangenten.

 Če je $AB\cong BC$, točka $D$ že po
 konstrukciji leži na krožnici $k$. Ker točki $B$ in $D$ obe ležita na simetrali
 daljice $AC$, je štirikotnik $ABCD$ deltoid, zato je tudi
 tangenten (izrek \ref{TangDeltoidRomb}).

 Raziščimo še število rešitev naloge. Krožnica $k(A,AB-AC)$ in
 lok $l$ se vedno sekata v eni točki $E$. Ker je $ABC<180^0$, je
 $\angle AEC=180^0-\frac{1}{2}\beta>90^0$.
 To pomeni, da simetrala $s_{EC}$ vedno seka poltrak $AE$ v eni točki
 $D$ in velja $\mathcal{B}(A,E,D)$. To pomeni, da ima naloga vedno eno samo
 rešitev.
  \kdokaz




  %______________________________________________________________________________
 \poglavje{Simson Line} \label{odd4Simson}

Dokažimo najprej osnovni izrek.


             \bizrek \label{SimpsPrem}
             The foots of the perpendiculars
            from an arbitrary point lying on the circumcircle of a triangle
            to the  lines containing the sides of this triangle  are three collinear points.
            The line containing these points is the so-called
            \index{premica!Simsonova}
            \pojem{Simson\footnote{Premico imenujemo po škotskem matematiku
            \index{Simson, R.}
            \textit{R. Simsonu} (1687--1768), čeprav je to lastnost
            prvi objavil škotski matematik \index{Wallace, W.}
             \textit{W. Wallace}
            (1768--1843) šele leta 1799.} line}\color{blue}.
            \eizrek


\begin{figure}[!htb]
\centering
\input{sl.skk.4.7.1a.pic}
\caption{} \label{sl.skk.4.7.1a.pic}
\end{figure}

 \textbf{\textit{Proof.}} Naj bo $S$ poljubna točka očrtane krožnice
 $k$
trikotnika $ABC$ ter $P$, $Q$ in $R$ pravokotne projekcije točke $S$
na nosilke stranic $BC$, $AC$ in $AB$ (Figure \ref{sl.skk.4.7.1a.pic}). Brez
škode za splošnost predpostavimo, da velja
 $\mathcal{B}(B,P,C)$, $\mathcal{B}(A,Q,C)$ in $\mathcal{B}(A,B,R)$.
  V tem primeru sta točki $Q$ in $R$ na različnih bregovih premice $BC$,
zato je dovolj dokazati $\angle BPR \cong \angle CPQ$. Zaradi
ustreznih pravih kotov in lege točke $S$ so po izreku
\ref{TetivniPogoj} štirikotniki $BRSP$, $ABSC$, $ARSQ$ in $SPQC$
tetivni, zato je
 (izrek \ref{TetivniPogoj}:
 \begin{eqnarray*}
 \angle BPR &=& \angle BSR = \angle RSC - \angle BSC=\\
&=& \angle RSC - (180° - \angle BAC) =\\
&=& \angle RSC - \angle RSQ = \angle CSQ = \angle CPQ,
 \end{eqnarray*}
 kar pomeni, da so točke $P$, $Q$ in $R$ kolinearne. \kdokaz

 V nadaljevanju bomo obravnavali nadaljnje zanimive lastnosti Simsonove
 premice. Ker vsaka točka $X$, ki leži na očrtani krožnici nekega
 trikotnika, določa Simsonovo premico, bomo to premico označili z
 $x$. Na ta način vsak trikotnik določa eno preslikavo $X\mapsto x$.



            \bzgled \label{SimsZgled1}
            Let $P$ be an arbitrary point of the circumcircle $k$ of a triangle
            $ABC$. Suppose that $P_A$ is the intersection of the perpendicular line of
            the line $BC$ through the point $P$ with the circle $k$.
            Prove that the line $AP_A$ is parallel to the Simson line $p$
            of the triangle at the point $P$.
            \ezgled


\begin{figure}[!htb]
\centering
\input{sl.skk.4.7.1b.pic}
\caption{} \label{sl.skk.4.7.1b.pic}
\end{figure}

 \textbf{\textit{Proof.}}
 Naj bodo $X$, $Y$ in $Z$ pravokotne projekcije točke $P$ na premice
  $BC$, $AC$ in $AB$ (Figure \ref{sl.skk.4.7.1b.pic}). Po izreku \ref{SimpsPrem} je Simsonova premica
  $p$ določena s točkami $X$, $Y$ in $Z$. Podobno kot v izreku
  \ref{SimpsPrem} je
štirikotnik $PYXC$ tetiven, zato je $\angle YXP \cong \angle ACP$.
Po izreku \ref{ObodObodKot} sta kota $ACP$ in $AP_AP$ nad
tetivo $AP$ skladna, zaradi tega je tudi $\angle YXP \cong \angle AP_AP$
oz. $XY\parallel AP$ (izrek \ref{KotiTransverzala}).
  \kdokaz



            \bzgled \label{SimsZgled2}
            Let $P$ and $Q$ be arbitrary points lying on the circumcircle
            $k(O,r)$ of a triangle $ABC$ and $p$ and $q$ their Simson lines. Prove that
             $$\angle pq = \frac{1}{2}\angle POQ.$$
           \ezgled

\begin{figure}[!htb]
\centering
\input{sl.skk.4.7.1c.pic}
\caption{} \label{sl.skk.4.7.1c.pic}
\end{figure}

 \textbf{\textit{Proof.}}
  Naj bosta $X_P$ in $X_Q$ nožišči pravokotnic iz točk $P$ in $Q$
   na premici $BC$ ter $P_A$ in $Q_A$
presečišči teh pravokotnic s  krožnico $k$ (Figure
\ref{sl.skk.4.7.1c.pic}). Simsonovi premici $p$ in $q$ sta po vrsti
vzporedni s premicami $AP_A$ in $AQ_A$ (zgled \ref{SimsZgled1}).
Torej je kot, ki ga določata premici $p$ in $q$, skladen obodnemu
kotu $Q_AAP_A$, ta pa je enak polovici središčnega kota $Q_AOP_A$
(izrek \ref{SredObodKot}) oziroma polovici kota $QOP$ (ker je
trapez $PP_AQ_AQ$ tetiven in po izreku \ref{trapezTetivEnakokr} tudi
enakokrak oz. $PQ \cong P_AQ_A$).
 \kdokaz


            \bzgled \label{SimsZgled3}
            Let $P$ be an arbitrary point of the circumcircle $k$ of a triangle
            $ABC$, $p$ its Simson line and $V$ the orthocentre of this triangle.
             Prove that the line $p$ bisects the line segment $PV$.
            \ezgled

\begin{figure}[!htb]
\centering
\input{sl.skk.4.7.1d.pic}
\caption{} \label{sl.skk.4.7.1d.pic}
\end{figure}

 \textbf{\textit{Proof.}}
Naj bosta $P_A$ in $X$ točki, definirani kot v zgledu
\ref{SimsZgled1} (Figure \ref{sl.skk.4.7.1d.pic}). Naj bosta še $V'$
in $P'$ točki, ki sta simetrični točkam $V$ in $P$ glede na premico
$BC$. Točka $V'$ leži na očrtani krožnici $k$ trikotnika $ABC$
(izrek \ref{TockaV'}). Zaradi lastnosti simetrije
oz. osnega zrcaljenja (glej razdelek \ref{odd6OsnZrc}), izreka
\ref{KotiTransverzala}, izreka \ref{ObodObodKot} in zgleda
\ref{SimsZgled1} velja:
   $$\angle VP'P\cong\angle V'PP'\cong\angle AV'P
   \cong\angle AP_AP\cong\angle p,PP'.$$
 Torej je $\angle VP'P \cong \angle p,PP'$, zato sta po izreku
 \ref{KotiTransverzala} premici $VP'$ in $p$ vzporedni. Ker
je točka $X$ središče daljice $PP'$, premica $p$ vsebuje srednjico
trikotnika $PVP'$ (izrek \ref{srednjicaTrik}) oz. središče njegove
stranice $PV$.
 \kdokaz

V razdelkih \ref{odd5Hamilton} in \ref{odd7SredRazteg} bomo dokazali še dve lastnosti
Simsonovih premic (zgleda \ref{HamiltonSimson} in \ref{SimsEuler}), ki sta povezani s Hamiltonovim izrekom oz.
Eulerjevo krožnico trikotnika.




%________________________________________________________________________________
 \poglavje{Torricelli Point} \label{odd4Torricelli}

V tem razdelku bomo podali še eno znamenito točko trikotnika.


             \bizrek \label{izrekTorichelijev}
             On each side of a triangle $ABC$ the equilateral triangles $BEC$, $CFA$ and $AGB$
             are externally erected. Prove:
            \begin{enumerate}
              \item $AE$, $BF$ and $CG$ are congruent line segments;
              \item the lines $AE$, $BF$ and $CG$ intersect at one point
              (so-called \pojem{Torricelli\footnote{Problem je prvi zastavil francoski matematik \index{Fermat, P.} \textit{P.
            Fermat} (1601--1665) kot izziv italijanskem matematiku in fiziku \index{Torricelli, E.} \textit{E.
            Torricelliju} (1608--1647). Torricellijevo rešitev je objavil njegov učenec - italijanski matematik in fizik \textit{V. Viviani} (1622–-1703) - leta 1659. Omenjeno točko imenujemo tudi \index{točka!Fermatova}\pojem{Fermat
            point}.}
             point} \color{blue}of this triangle) and every two of them determine an angle with measure $60^0$.
            \end{enumerate}
             \index{točka!Torricellijeva}
            \eizrek


\begin{figure}[!htb]
\centering
\input{sl.skk.4.7.1.pic}
\caption{} \label{sl.skk.4.7.1.pic}
\end{figure}

 \textbf{\textit{Proof.}}
  (Figure \ref{sl.skk.4.7.1.pic})

 (\textit{i}) Trikotnika $AEC$ in $FBC$ sta skladna po izreku \textit{SAS} \ref{SKS} ($AC \cong FC$ , $CE \cong CB$ in
$\angle ACE \cong \angle FCB = \angle ACB + 60°$), zato je $AE\cong BF$. Analogno je tudi $AE\cong CG$.

(\textit{ii}) Naj bodo $k$, $l$ in $j$ očrtane krožnice trikotnikov $BEC$, $CFA$ in $AGB$. Dokažimo najprej, da se te krožnice
sekajo v eni točki. S $T$ označimo drugo presečišče krožnic $k$ in $l$ ($T\neq C$). Štirikotnika $BECT$ in $CFAT$ sta tetivna, zato (izrek \ref{TetivniPogoj}) oba kota $BTC$ in
$ATC$ merita $120^0$. Torej tudi kot $ATB$ meri $120^0$, kar pomeni, da je štirikotnik $AGBT$ tetiven (izrek \ref{TetivniPogoj}) oz. točka $T$ tudi leži
  na  krožnici $j$.

Dokažimo, da vsaka od daljic $AE$, $BF$, $CG$ poteka skozi točko $T$. Iz enakosti ustreznih obodnih
kotov (izrek \ref{ObodObodKot}) dobimo:

\begin{eqnarray*}
\angle ATE&=&\angle ATF+\angle FTC+\angle CTE=\\
&=&\angle ACF+\angle FAC+\angle CBE
=3\cdot 60^0=180^0.
\end{eqnarray*}

Torej so $A$, $T$ in $E$ kolinearne točke, oz. točka $T$ leži na daljici $AE$. Analogno leži točka $T$ tudi
na daljicah $BF$ in $CG$. Jasno je, da velja tudi:
$\angle AE,BF\cong\angle ATF\cong\angle ACF=60^0$.
 \kdokaz

  V razdelku \ref{odd9MetrInv} (izrek \ref{izrekToricheliFerma}) bomo dokazali še eno zanimivo lastnost Torricellijeve točke.


%________________________________________________________________________________
 \poglavje{Excircles of a Triangle} \label{odd4Pricrt}

 Dokazali smo že, da za poljubni trikotnik obstajata
 očrtana in včrtana krožnica. Prva vsebuje vsa oglišča
trikotnika, druga pa se dotika vseh njegovih stranic. Sedaj bomo
pokazali, da obstajajo tudi krožnice, ki se dotikajo po ene
stranice in dveh nosilk stranic trikotnika.


            \bizrek
            The bisector of the interior angle at vertex $A$ and the bisectors of
            the exterior angles at vertices $B$ and $C$ of a triangle $ABC$ intersect at one point,
            which is the centre of the circle touching the side $BC$ and the lines containing the sides $AB$ and
            $AC$. It is so-called \index{pričrtane krožnice trikotnika} \pojem{excircle of the triangle}\color{blue}.
            \eizrek



\begin{figure}[!htb]
\centering
\input{sl.27.1.94_veliki_zadatak_lema.pic}
\caption{} \label{sl.27.1.94_veliki_zadatak_lema.pic}
\end{figure}


 \textbf{\textit{Proof.}} Izrek dokažemo podobno kot pri včrtani
 krožnici trikotnika. Simetrali
zunanjih kotov pri ogliščih $B$ in $C$ nista vzporedni in se
 sekata v neki točki - označimo jo s $S_a$ (Figure
\ref{sl.27.1.94_veliki_zadatak_lema.pic}). Ker točka $S_a$ leži na
teh dveh simetralah, velja $A,S_a\div BC$ in je $S_a$ enako oddaljena od premic $AB$, $BC$ in
$AC$. Torej $S_a$ pripada tudi simetrali notranjega kota pri
oglišču $A$ in je središče krožnice, ki se dotika stranice $BC$ ter
nosilk stranic $AB$ in $AC$.
 \kdokaz

 Sedaj smo pripravljeni dokazati t. i. \index{velika naloga}
  \pojem{‘‘veliko nalogo’’}, ki je zelo uporabna pri
 načrtovanju trikotnikov.




              \bizrek \label{velikaNaloga}
              Let $P$, $Q$, $R$ be the touching points of the incircle $k(S,r)$
               of a triangle $ABC$ with the sides $BC=a$, $AC=b$, $AB=c$ ($b>c$) and $P_i$, $Q_i$, $R_i$
                ($i\in \{a,b,c\}$) the touching points of the excircles
              $k_i(S_i,r_i)$ with lines $BC$, $AC$ in $AB$. Let $l(O,R)$
               be the circumcircle of this triangle with the semiperimeter
                $s=\frac{a+b+c}{2}$, $A_1$ the midpoint of the line segments $BC$, $M$ and $N$,
                intersections of the line $OA_1$ with the circle $l$ ($N,A\div BC$) and
                $M’$, $N’$ the foots of the perpendiculars  from these points on the line $AB$ (Figure
                \ref{sl.27.1.94_veliki_zadatak.pic}). Then:
                \vspace*{2mm}

                (\textit{i}) $AQ_a\cong AR_a=s$, \hspace*{0.4mm} (\textit{ii})
                $AQ\cong AR=s-a$, \hspace*{0.4mm} (\textit{iii}) $QQa\cong RRa\cong a$,
                % \vspace*{1mm}

                 (\textit{iv}) $PPa=b-c$,\hspace*{1mm}
                (\textit{v}) $P_bP_c=b+c$,
                % \vspace*{1mm}

                 (\textit{vi})
                  $A_1$ is the midpoint of the line segment $PP_a$ in $P_bP_c$,
                %\vspace*{1mm}

                 (\textit{vii}) $A_1N= \frac{r_a- r}{2}$,\hspace*{2mm}
               (\textit{viii}) $A_1M= \frac{r_b + r_c }{2} $,\hspace*{1mm}


              (\textit{ix}) $r_a +r_b +r_c =4R+r$,\footnote{To lastnost
               trikotnika je leta 1790 odkril francoski matematik \index{L'Huilier, S. A. J.}
                \textit{S. A. J. L'Huilier}
               (1750--1840).}
             %\vspace*{1mm}

                (\textit{x}) $NN’= \frac{r_a +r}{2}$, \hspace*{1mm} (\textit{xi})
                $MM’= \frac{r_b -r_c }{2 }$,\hspace*{1mm} (\textit{xii})
                $N’B\cong AM’=\frac{ b - c}{2}$,
                % \vspace*{1mm}

                (\textit{xiii}) $AN’\cong BM’= \frac{b + c}{2} $, \hspace*{1mm} (\textit{xiv}) $M’N’\cong b$.
                 \eizrek


\begin{figure}[!htb]
\centering
\input{sl.27.1.94_veliki_zadatak.pic}
\caption{} \label{sl.27.1.94_veliki_zadatak.pic}
\end{figure}

 \textbf{\textit{Proof.}}
  Preden začnemo z dokazovanjem, omenimo, da po izreku \ref{TockaN}
  točka $N$ leži na simetrali notranjega kota $BAC$ trikotnika
  $ABC$.

  (\textit{i}) Če uporabimo enakost tangentnih daljic
  (izrek \ref{TangOdsek}), dobimo:
 \begin{eqnarray*}
 AQ_a&\cong &AR_a= \frac{1}{2}\left(AQ_a+AR_a\right)=
 \frac{1}{2}\left(AB+BR_a+AC+CQ_a\right)\\
 &=& \frac{1}{2}\left(AB+BP_a+AC+CP_a\right)=
\frac{1}{2}\left(AB+BC+AC\right)=s.
 \end{eqnarray*}

 (\textit{ii}) Na podoben način dobimo:
 \begin{eqnarray*}
 AQ&\cong &AR= \frac{1}{2}\left(AQ+AR\right)=
 \frac{1}{2}\left(AB-BR+AC-CQ\right)\\
 &=& \frac{1}{2}\left(AB-BP+AC-CP\right)=
\frac{1}{2}\left(AB+AC-BC\right)=s-a.
 \end{eqnarray*}

 (\textit{iii}) $QQ_a\cong AQ_a-AQ=a$.

 (\textit{iv}) Enakost dokažemo tako, da
 najprej izračunamo:\\ $BP$ in $CP_a\cong CQ_a$.

 (\textit{v}) $P_bP_c\cong CP_c+BP_b-a=2s-a=b+c$.

 (\textit{vi}) Sledi iz $BP\cong CP_a=s-b$.

 (\textit{vii}) Točki $A_1$ in $N$ sta središči diagonal trapeza
 $SPS_aP_a$ z
 osnovnicama $SP\cong r$ in
$S_aP_a\cong r_a$. Enakost sledi iz izreka \ref{srednjTrapez}.

 (\textit{viii}) Premici $NA$ in $S_cS_b$
  sta pravokotni (simetrali notranjega
 in zunanjega kota pri oglišču
$A$), zato točka $M$ pripada premici $S_cS_b$. Iskana enakost sledi
iz dejstva, da je daljica $A_1M$ srednjica trapeza
$S_cP_cP_bS_b$ (izrek \ref{srednjTrapez}).

 (\textit{ix}) Sledi neposredno iz $2\cdot R=NM=NA_1+A_1M$ in
 (\textit{vii}) ter (\textit{viii}).

(\textit{x}) Sledi iz izreka \ref{srednjTrapez} in dejstva, da  je
$NN'$ srednjica trapeza $SRR_aS_a$.

(\textit{xi}) Točki $M$ in $M'$ sta središči diagonal trapeza
$R_cS_cR_bS_b$ z osnovnicama $R_cS_c\cong r_c$ in $R_bR_b\cong r_b$. Enakost
sledi iz izreka \ref{srednjTrapez}.

 (\textit{xii}) Točka $N$ je središče daljice $RR_a$, zato je:
  $$N'B=AN'-AB= \frac{1}{2}\left(AR_a+AR\right)-c=
  \frac{1}{2}\left(s+(s-a)\right)-c= \frac{1}{2}\left(b-c\right).$$

 (\textit{xiii}) in (\textit{xiv}) sledita
  direktno iz dokazane enakosti (\textit{xii}).
 \kdokaz

%KONSTRUKCIJA (VN)

        \bzgled
         Construct a triangle $ABC$,  with given:

         (\textit{a}) $a$, $b-c$, $r$, \hspace*{3mm} (\textit{b})
        $b-c$, $r$, $v_b$, \hspace*{3mm} (\textit{c})  $a$, $b+c$, $r$,
        \hspace*{3mm}(\textit{č}) $R$, $r$, $r_a$.
         \ezgled



\textbf{\textit{Solution.}}
 Pri vsaki konstrukciji bomo uporabili veliko nalogo - \ref{velikaNaloga}.
 Uporabili bomo
  enake oznake.

\begin{figure}[!htb]
\centering
\input{sl.27.1.94_veliki_zadatak_konstr.pic}
\caption{} \label{sl.27.1.94_veliki_zadatak_konstr.pic}
\end{figure}

\begin{figure}[!htb]
\centering
\input{sl.27.1.94_veliki_zadatak_konstr2.pic}
\caption{} \label{sl.27.1.94_veliki_zadatak_konstr2.pic}
\end{figure}



(\textit{a}) Ker je $PP_a=b-c$ in točka $A_1$ skupno središče
stranice $BC$ in
 daljice $PP_a$, velja tudi $PA_1 = \frac{1}{2}( b - c)$ (Figure
\ref{sl.27.1.94_veliki_zadatak_konstr.pic}).
Najprej načrtamo stranico $BC$, nato pa njeno središče
$A_1$, točko $P$, včrtano krožnico trikotnika, tangenti iz
oglišč $B$ in $C$ ter na koncu še oglišče $A$.



(\textit{b}) Podobno kot pri prejšnjem primeru. Najprej načrtamo
daljico $PA_1$, nato včrtano krožnico trikotnika $ABC$ (Figure
\ref{sl.27.1.94_veliki_zadatak_konstr.pic}).
Potrebno je uporabiti še pogoj višine iz oglišča $B$. Z $L$
označimo pravokotno projekcijo iz točke $A_1$ na premici $AC$.
Daljica $A_1L$ je srednjica trikotnika $CBB'$, zato je $A_1L
=\frac{1}{2}BB'=\frac{1}{2} v_b$. Torej premico $AC$ lahko
konstruiramo kot skupno tangento včrtane krožnice in krožnice
$k(A_1, \frac{1}{2}v_b)$. Tako dobimo oglišče $C$, nato pa tudi
oglišči $B$ in $A$.


(\textit{c}) Vemo, da velja $RR_a\cong a$ in $AN'=\frac{1}{2}(b+c)$,
točka $N'$ pa je središče daljice $RR_a$ (Figure \ref{sl.27.1.94_veliki_zadatak_konstr2.pic}). Torej iz danih
podatkov
 najprej konstruiramo točke $A$, $N'$, $R$ in $R_a$, nato pa tudi $S$, $N$
  in
$S_a$. Na koncu narišemo včrtano in pričrtano krožnico -
stranice trikotnika pa ležijo na njunih skupnih tangentah.


(\textit{č}) Ker je $A_1N =\frac{1}{2}(r_a -r)$ in $MN = 2R$, lahko
načrtamo najprej točke $N$, $A_1$ in $M$, nato pa še očrtano
krožnico trikotnika $ABC$ in stranico $BC$ (Figure \ref{sl.27.1.94_veliki_zadatak_konstr2.pic}). Konstrukcijo lahko
končamo na dva načina. V prvem primeru nalogo prevedemo v že znano
konstrukcijo trikotnika: $a$, $R$, $r$ (zgled \ref{konstr_Rra}), v
drugem pa uporabimo enakost $RR_a\cong a$.
 \kdokaz




%________________________________________________________________________________
\naloge{Exercises}

\begin{enumerate}

\item   Dolžine stranic nekega trikotnika so $6$, $7$ in $9$. Naj bodo $k_1$, $k_2$ in $k_3$ krožnice s središči
v ogliščih tega trikotnika. Krožnice se med seboj dotikajo tako, da
 se krožnica s središčem v oglišču najmanjšega kota trikotnika z ostalima krožnicama dotika od znotraj,
preostali dve krožnici pa se dotikata od zunaj. Izračunaj dolžine polmerov teh treh krožnic.

\item Dokaži, da je kot, ki ga določata sečnici krožnice, ki se med seboj sekata v zunanjosti krožnice, enak polovici razlike središčnih kotov,  prirejenih lokoma, ki ležita med krakoma tega kota.

\item   Vrh kota $\alpha$ je zunanja točka krožnice $k$. Med krakoma tega kota ležita na krožnici dva
loka, ki sta v razmerju $3:10$. Večji od teh lokov ustreza središčnemu kotu $40^0$. Določi
mero kota $\alpha$.

\item  Dokaži, da je kot, ki ga oklepata tangenti krožnice, enak polovici razlike središčnih kotov, prirejenih lokoma, ki ležita med krakoma tega kota.

\item Naj bo $L$  pravokotna projekcija poljubne točke $K$ krožnice $k$
 na njeni tangenti $t$ skozi točko $T\in k$ in $X$ točka, ki je
simetrična točki $L$ glede na premico $KT$. Določi geometrijsko
mesto točk $X$.

\item Naj bosta $BB'$ in $CC'$ višini trikotnika $ABC$ in $t$
tangenta očrtane krožnice tega trikotnika v točki $A$. Dokaži, da
je  $B'C'\parallel t$.

\item V pravokotnem trikotniku $ABC$ je nad kateto $AC$ kot premerom
načrtana krožnica, ki seka hipotenuzo $AB$ v točki $E$. Tangenta te
krožnice v točki $E$ seka drugo kateto $BC$ v točki $D$. Dokaži, da je $BDE$
enakokraki trikotnik.

\item V pravi kot z vrhom $A$ je včrtana krožnica, ki se dotika krakov tega kota v
točkah $B$ in $C$. Poljubna tangenta te krožnice seka premici $AB$ in $AC$, po vrsti
v točkah $M$ in $N$ (tako, da je je $\mathcal{B}(A,M,B)$). Dokaži, da velja:
$$\frac{1}{3}\left(|AB|+|AC|\right) < |MB|+|NC| <
\frac{1}{2}\left(|AB|+|AC|\right).$$

\item Dokaži, da je pri pravokotnem trikotniku vsota katet enaka vsoti
premerov očrtane in včrtane krožnice.

\item Naj se simetrale notranjih kotov konveksnega štirikotnika sekajo v šestih različnih točkah.
Dokaži, da so štiri od teh točk  oglišča tetivnega štirikotnika.

\item Naj bodo: $c$ dolžina hipotenuze, $a$ in $b$ dolžini
katet ter $r$ polmer včrtane krožnice pravokotnega trikotnika. Dokaži da velja:
\begin{enumerate}
 \item $2r + c \geq 2 \sqrt{ab}$, \item $a + b + c > 8r$.
\end{enumerate}

\item Naj bosta $P$ in $Q$ središči krajših lokov $AB$ in $AC$
pravilnemu trikotniku $ABC$ očrtane krožnice. Dokaži, da stranici $AB$ in $AC$ tega trikotnika razdelita
tetivo $PQ$ na tri skladne daljice.

\item Naj bodo $k_1$, $k_2$, $k_3$, $k_4$ štiri krožnice, od katerih se vsaka od zunaj dotika ene stranice   in dveh nosilk stranic poljubnega konveksnega
štirikotnika. Dokaži, da so središča teh krožnic konciklične točke.

\item Krožnici $k$ in $l$ se dotikata od zunaj v točki $A$. Točki $B$ in $C$ sta dotikališči
skupne zunanje tangente teh dveh krožnic. Dokaži, da je $\angle BAC$ pravi kot.

\item Naj bo $ABCD$ deltoid ($AB\cong AD$ in $CB\cong CD$). Dokaži:
\begin{enumerate}
 \item $ABCD$ je tangenten štirikotnik,
 \item  $ABCD$ je tetiven štirikotnik natanko tedaj, ko je $AB\perp BC$.
\end{enumerate}

\item Krožnici $k$ in $k_1$  se dotikata od zunaj v točki $T$, v kateri se sekata premici $p$ in $q$. Premica $p$ ima
s krožnicama še presečišči $P$ in  $P_1$, premica $q$ pa  $Q$ in $Q_1$.  Dokaži, da je $PQ\parallel P_1Q_1$.

\item Naj bo $MN$ skupna tangenta krožnic $k$ in $l$ ($M$ in $N$ sta dotikališči),
ki se sekata v točkah $A$ in $B$. Izračunaj mero vsote $\angle MAN+\angle MBN$.

\item Naj bo $t$ tangenta trikotniku $ABC$ očrtane krožnice v točki $A$. Premica, ki je vzporedna s tangento $t$, seka stranici $AB$ in $AC$ v točkah $D$ in $E$.
Dokaži, da so točke $B$, $C$, $D$ in $E$ konciklične.

\item Naj bosta $D$ in $E$ poljubni točki polkrožnice, ki je načrtana nad premerom $AB$. Naj bo $AD\cap BE= \{F\}$ in $AE\cap BD= \{G\}$.
Dokaži, da je $FG\perp AB$.

\item Naj bo $M$ točka krožnice $k(O,r)$. Določi geometrijsko mesto središč
 vseh tetiv te krožnice, ki imajo eno krajišče v točki $M$.

\item Naj bosta $M$ in $N$ točki, ki sta simetrični nožišču $A'$ višine $AA'$ trikotnika $ABC$ glede na stranici $AB$ in $AC$, ter $K$ presečišče premic $AB$ in $MN$. Dokaži,
da so točke $A$, $K$, $A'$, $C$ in $N$ konciklične.

\item Naj bo $D$ nožišče višine iz oglišča $A$ ostrokotnega trikotnika $ABC$ in $O$
središče očrtane krožnice tega trikotnika. Dokaži, da je  $\angle CAD\cong \angle BAO$.

\item Naj bo $ABCD$ tetivni štirikotnik, $E$ višinska točka trikotnika $ABD$ in $F$
 višinska točka trikotnika $ABC$. Dokaži, da je štirikotnik $CDEF$
paralelogram.

\item Krožnici s središčema  $O_1$ in $O_2$ se sekata v točkah $A$ in $B$. Premica $p$,
ki poteka skozi točko $A$, seka ti dve krožnici še v točkah $M_1$ in $M_2$. Dokaži, da je
$\angle O_1M_1B\cong\angle O_2M_2B$.

\item   Polkrožnica s središčem $O$ je načrtana nad premerom $AB$.
Naj bosta $C$ in $D$ takšni točki na daljici $AB$, da velja $CO\cong OD$. Vzporedni premici skozi točki $C$ in $D$ sekata polkrožnico v točkah $E$ in $F$.
Dokaži, da sta premici $CE$ in $DF$ pravokotni na premico $EF$.

\item Na tetivi $AB$ krožnice $k$ s središčem $O$ naj leži točka $C$, točka $D$ pa naj bo
drugo presečišče krožnice $k$ z očrtano krožnico trikotnika $ACO$. Dokaži, da je
$CD\cong CB$.

\item Naj bo $AB$ mimobežnica krožnice $k$. Premici $AC$ in $BD$ naj bosta tangenti
krožnice $k$ v točkah $C$ in $D$.
Dokaži, da velja:
 $$||AC|-|BD||< |AB| < |AC|+|BD|.$$

\item Naj bo $S$ presečišče nosilk krakov $AD$ in $BC$
trapeza $ABCD$ z osnovnico $AB$. Dokaži, da se očrtani krožnici trikotnikov $SAB$ in
$SCD$ dotikata v točki $S$.

\item Premici $PB$ in $PD$ se dotikata krožnice $k(O,r)$ v točkah $B$ in $D$.
Premica $PO$ seka krožnico $k$ v točkah $A$ in $C$ ($\mathcal{B}(P,A,C)$). Dokaži, da je
premica $BA$ simetrala kota $PBD$.

\item Štirikotnik $ABCD$ je včrtan v krožnico s središčem $O$. Diagonali $AC$ in
$BD$ sta pravokotni. Naj bo $M$ pravokotna projekcija središča $O$
na premici $AD$. Dokaži, da je
 $$|OM|=\frac{1}{2}|BC|.$$

\item Daljici $AB$ in $BC$ sta sosednji stranici pravilnega devetkotnika, ki je včrtan krožnici $k$ s središčem $O$.
Točka $M$ je središče stranice $AB$, točka $N$ pa središče
polmera $OX$ krožnice $k$, ki je pravokoten na premico $BC$. Dokaži, da je
$\angle OMN=30^0$.

\item Krožnici $k_1$ in $k_2$ se sekata v točkah $A$ in $B$. Naj bo $p$ premica, ki poteka skozi točko $A$, krožnico $k_1$ seka še v točki $C$, krožnico $k_2$ pa še v točki $D$, ter $q$ premica, ki poteka skozi točko $B$, krožnico $k_1$ seka še v točki $E$, krožnico $k_2$ pa še v točki $F$. Dokaži, da
je $\angle CBD\cong\angle EAF$.

\item Krožnici $k_1$ in $k_2$ se sekata v točkah $A$ in $B$. Načrtaj premico $p$, ki poteka skozi točko $A$, tako da bo
dolžina daljice $MN$, kjer sta $M$ in $N$ presečišči premice $p$
s krožnicama $k_1$ in $k_2$,  maksimalna.

\item Naj bo $L$ pravokotna projekcija poljubne točke $K$ krožnice $k$ na njeni
tangenti skozi točko $T\in k$ ter $X$ točka, ki je simetrična točki $L$ glede na premico
$KT$. Določi geometrijsko mesto točk $X$.

\item Dokaži, da je tetivni večkotnik z lihim številom oglišč, ki ima vse notranje kote skladne, pravilni večkotnik.

\item Dve krožnici se dotikata od znotraj v točki $A$. Daljica $AB$ je premer večje
krožnice, tetiva $BK$ večje krožnice pa se dotika manjše krožnice v točki $C$. Dokaži
da je premica $AC$ simetrala kota $BAK$.

\item Naj bo $BC$ tetiva krožnice $k$. Določi geometrijsko mesto višinskih točk
vseh trikotnikov $ABC$, kjer je $A$ poljubna točka, ki leži na krožnici $k$.

\item Imejmo štirikotnik s tremi topimi notranjimi koti. Dokaži, da daljša
diagonala poteka iz oglišča, ki pripada ostrem kotu.

\item Naj bo $ABCDEF$ tetivni šestkotnik ter $AB\cong DE$ in $BC\cong EF$.
Dokaži, da je $CD\parallel AF$.

\item Naj bo $ABCD$ konveksni štirikotnik, pri katerem je $\angle ABD=50^0$, $\angle ADB=80^0$, $\angle ACB=40^0$ in $\angle DBC=\angle BDC +30^0$. Izračunaj mero kota $\angle DBC$.

\item Naj bo $M$ poljubna notranja točka kota z vrhom $A$, točki $P$ in $Q$ pravokotni projekciji točke $M$ na krakih tega kota, točka $K$
pa  pravokotna projekcija vrha $A$ na premici $PQ$. Dokaži, da je $\angle MAP\cong \angle QAK$.


\item Pri tetivnem osemkotniku $A_1A_2\ldots A_8$ velja $A_1A_2\parallel A_5A_6$, $A_2A_3\parallel A_6A_7$,
$A_3A_4\parallel A_7A_8$. Dokaži, da je $A_8A_1\cong A_4A_5$.

\item Neka krožnica vsako stranico štirikotnika seka v dveh točkah in tako na vseh stranicah štirikotnika določa skladne tetive.
Dokaži, da je ta štirikotnik tangenten.

\item Dolžine stranic tangentnega petkotnika $ABCDE$ so naravna števila in hkrati velja $|AB|=|CD|=1$.
Včrtana krožnica petkotnika se dotika stranice $BC$ v točki $K$.
Izračunaj dolžino daljice $BK$.

\item Dokaži, da krožnica, ki poteka skozi  sosednji oglišči $A$ in $B$ pravilnega
petkotnika $ABCDE$ in njegovo središče $O$, poteka tudi skozi presečišče
njegovih diagonal $AD$ in $BE$.

\item Naj bo $H$ višinska točka trikotnika $ABC$, $l$ krožnica nad  premerom $AH$
ter $P$ in $Q$ presečišči te krožnice s stranicama $AB$ in $AC$. Dokaži, da se
tangenti krožnice $k$ skozi točki $P$ in $Q$ sekata na stranici $BC$.

\item Krožnica $l$ se od znotraj dotika krožnice $k$ v točki $C$. Naj bo $M$ poljubna točka
krožnice $l$ (različna od $C$). Tangenta krožnice $l$ v točki $M$ seka krožnico $k$ v
točkah $A$ in $B$. Dokaži, da je $\angle ACM \cong \angle MCB$.

\item Naj bo $k$ očrtana krožnica trikotnika $ABC$ in $R$ središče tistega loka $AB$ te krožnice,
ki ne vsebuje točke $C$. Daljici $RP$ in $RQ$ sta tetivi te krožnice. Prva je
vzporedna, druga pa pravokotna na simetralo notranjega kota  $\angle BAC$. Dokaži:
\begin{enumerate}
\item premica $BQ$ je simetrala notranjega kota $\angle CBA$,
\item  trikotnik, ki ga določajo premice $AB$, $AC$ in $PR$, je pravilni trikotnik.
 \end{enumerate}

\item Naj bo $X$ takšna notranja točka trikotnika $ABC$, da velja:
 $\angle BXC =\angle BAC+60^0$,  $\angle AXC =\angle ABC+60^0$ in  $\angle AXB =\angle AC B+60^0$. Naj bodo
$P$, $Q$ in $R$ druga presečišča  premic $AX$, $BX$ in $CX$ z očrtano krožnico trikotnika $ABC$. Dokaži, da je  trikotnik $PQR$ pravilen.


\item Dokaži, da dotikališča včrtane krožnice s trikotnikom $ABC$ delijo njegove stranice na
odseke dolžin $s-a$, $s-b$ in $s-c$ ($a$, $b$ in $c$ so dolžine stranic, $s$ pa je polobseg
trikotnika).


 \item Krožnice $k$, $l$ in $j$ se paroma od zunaj dotikajo v nekolinearnih točkah $A$, $B$ in $C$. Dokaži, da je očrtana krožnica trikotnika $ABC$ pravokotna na krožnice $k$, $l$ in $j$.


 \item Naj bo $ABCD$ tetivni štirikotnik s središčem očrtane krožnice v točki $O$. Z $E$ označimo presečišče njegovih diagonal $AC$ in $BD$ ter z $F$, $M$ in $N$ središča daljic $OE$, $AD$ in $BC$. Če so $F$, $M$ in $N$ kolinearne točke, velja $AC\perp BD$ ali $AB\cong CD$. Dokaži.



\item Načrtaj trikotnik $ABC$ (glej oznake v razdelku \ref{odd3Stirik}):

 (\textit{a}) $a$, $\alpha$, $r$, \hspace*{2mm}
 (\textit{b}) $a$, $\alpha$, $r_a$, \hspace*{2mm}
 (\textit{c}) $a$, $v_b$, $v_c$, \hspace*{2mm}

 (\textit{d}) $\alpha$, $v_a$, $s$, \hspace*{2mm}
 (\textit{e}) $v_a$, $l_a$, $r$, \hspace*{2mm}
(\textit{f}) $\alpha$, $v_a$, $l_a$, \hspace*{2mm}

 (\textit{g}) $\alpha$, $\beta$, $R$, \hspace*{2mm}
 (\textit{h}) $c$, $r$, $R$, \hspace*{2mm}
 (\textit{i}) $a$, $v_b$, $R$, \hspace*{2mm}

 \item Načrtaj krožnico $k$ tako, da:

  \begin{enumerate}
    \item se dotika dveh danih nevzporednih premic $p$ in $q$, tetiva, ki jo določata dotikališči, pa je skladna z dano daljico $t$,
    \item je središče dana točka $S$, dana premica $p$ pa na njej določa tetivo, ki je skladna z dano daljico $t$,
    \item poteka skozi dani točki $A$ in $B$, središče pa leži na dani krožnici $l$,
    \item ima dan polmer $r$ in se dotika dveh danih krožnic $l$ in $j$,
    \item se dotika premice $p$ v točki $P$ in poteka skozi dano točko $A$.
  \end{enumerate}

  \item Načrtaj kvadrat $ABCD$, če je dano oglišče $B$ ter dve točki $E$ in $F$, ki ležita na nosilkah stranic $AD$ in $CD$.

  \item Dana je premica $CD$ ter točki $A$ in $B$ ($A,B\notin CD$). Načrtaj na premici $CD$ takšno točko $M$, da velja $\angle AMC\cong2\angle BMD$.


  \item Načrtaj trikotnik $ABC$ s podatki:

   (\textit{a}) $v_a$, $t_a$, $\beta-\gamma$, \hspace*{2mm}
   (\textit{b}) $v_a$, $l_a$, $R$, \hspace*{2mm}
   (\textit{c}) $R$, $\beta-\gamma$, $t_a$, \hspace*{2mm}

   (\textit{d}) $R$, $\beta-\gamma$, $v_a$, \hspace*{2mm}
   (\textit{e}) $R$, $\beta-\gamma$, $a$. \hspace*{2mm}


\item Na nosilki stranice $AB$ pravokotnika $ABCD$ načrtaj točko $E$, iz katere se stranici $AD$ in $DC$ vidita pod enakim kotom. Kdaj ima naloga rešitev?

    \item V konveksnem štirikotniku $ABCD$ velja $BC\cong CD$. Načrtaj ta štirikotnik, če so dani: stranici $AB$ in $AD$ ter notranja kota ob ogliščih $B$ in $D$.

    \item V dano krožnico $k$ včrtaj trikotnik $ABC$, če so dani: oglišče $A$, premica $p$, ki je vzporedna z višino $AA'$, in presečišče $B_2$ nosilke višine $BB'$ in te krožnice.

    \item Načrtaj pravilni trikotnik $ABC$, če je njegova stranica $BC$ skladna z daljico $a$, nosilki stranic $AB$ in $AC$ ter simetrala notranjega kota $BAC$ pa gredo po vrsti skozi dane točke $M$, $N$ in $P$.

  \item Načrtaj trikotnik $ABC$, če so dani:
  \begin{enumerate}
    \item oglišče $A$, središče očrtane krožnice $O$ in središče včrtane krožnice $S$,
    \item središče očrtane krožnice $O$, središče včrtane krožnice $S$ in središče pričrtane krožnice $S_a$,
    \item  oglišče $A$, središče očrtane krožnice $O$ in višinska točka $V$,
    \item oglišči $B$ in $C$ ter simetrala notranjega kota $BAC$,
    \item  oglišče $A$, središče očrtane krožnice $O$ in presečišče $E$ stranice $BC$ s simetralo notranjega kota $BAC$,
    \item točke $M$, $P$ in $N$, v katerih nosilki višine in težiščnice iz oglišča $A$ ter simetrala notranjega kota $BAC$ sekajo očrtano krožnico trikotnika,
    \item oglišče $A$, središče očrtane krožnice $O$, točka $N$, v kateri simetrala notranjega kota $BAC$ sekaj očrtano krožnico trikotnika, in daljica $a$, ki je skladna s stranico $BC$.
  \end{enumerate}

  \item Načrtaj trikotnik $ABC$ s podatki:

   (\textit{a}) $a$, $b$, $\alpha=3\beta$, \hspace*{3mm}
   (\textit{b}) $t_a$, $t_c$, $v_b$.\hspace*{3mm}

\item Skozi točko $M$, ki leži v notranjosti dane krožnice $k$, načrtaj takšno tetivo, da je razlika njenih odsekov (od točke $M$) enaka dani daljici $a$.

\item Načrtaj trikotnik $ABC$, če poznaš:

 (\textit{a}) $b-c$, $r$, $r_a$, \hspace*{1.8mm}
 (\textit{b}) $a$, $r$, $r_a$, \hspace*{1.8mm}
 (\textit{c}) $a$, $r_b+r_c$, $v_a$, \hspace*{1.8mm}
 (\textit{d}) $b+c$, $r_b$, $r_c$,

 (\textit{e}) $R$, $r_b$, $r_c$, \hspace*{1.8mm}
 (\textit{f}) $b$, $R$, $r+r_a$, \hspace*{1.8mm}
(\textit{g}) $a$, $v_a$, $r_a-r$, \hspace*{1.8mm}
 (\textit{h}) $\alpha$, $r$, $b+c$.


 \item Dani so: krožnica $k$, njen premer $AB$ in točka $M\notin k$. Samo z  ravnilom načrtaj pravokotnico iz točke $M$ na premico $AB$.

 \item Dani so: kvadrat $ABCD$ in takšni točki $M$ in $N$ na stranicah $BC$ in $CD$, da velja $\angle MAN=45^0$.
 Samo z ravnilom načrtaj pravokotnico iz točke $A$ na premico $MN$.

\end{enumerate}





% DEL 5 - - - - - - - - - - - - - - - - - - - - - - - - - - - - - - - - - - - - - - -
%________________________________________________________________________________
% VEKTORJI
%________________________________________________________________________________

  \del{Vectors} \label{pogVEKT}


%________________________________________________________________________________
\poglavje{Vector Definition. The Sum of Vectors} \label{odd5DefVekt}

Intuitivno je vektor usmerjena daljica, ki jo lahko vzporedno premikamo\footnote{Pojem vektorja je bil znan že Starim Grkom. Sodoben koncept vektorjev, ki je povezan z \index{linearna algebra} linearno algebro in \index{geometrija!analitična} analitično geometrijo, se je začel razvijati v 19. stoletju, najprej kot posplošitev kompleksnih števil. V tem smislu je angleški matematik \index{Hamilton, W. R.}\textit{W. R. Hamilton} (1805--1865) definiral t. i. \index{kvaternioni}\pojem{kvaternione} $q = w + ix + jy + kz$, $i^2 = j^2 = k^2 = -ijk = -1$ kot posplošitev kompleksnih števil v štirirazsežnem prostoru}. V tem smislu bi vektor $\overrightarrow{AB}$ predstavljala cela množica daljic, ki so skladne, vzporedne in isto usmerjene kot neka dana daljica $AB$.  Lahko bi pisali tudi $\overrightarrow{CD}=\overrightarrow{AB}$ za vsako daljico $CD$ iz te množice daljic  (Figure \ref{sl.vek.5.1.1.pic}).

\begin{figure}[!htb]
\centering
\input{sl.vek.5.1.1.pic}
\caption{} \label{sl.vek.5.1.1.pic}
\end{figure}

Na ta način dobimo idejo za  formalno definicijo vektorjev. Najprej vpeljimo relacijo $\varrho$ na množici parov točk. Naj bodo $A$, $B$, $C$ in $D$ točke v isti ravnini
(Figure \ref{sl.vek.5.1.2.pic}). Pravimo, da je $(A,B)\varrho (C,D)$, če je izpolnjen eden od treh pogojev\footnote{Na ta način - v \index{geometrija!afina}afini geometriji (brez aksiomov skladnosti) - je vektorje definiral srbski matematik \index{Veljković, M.}\textit{M. Veljković} (1954--2008), profesor na Matematični gimnaziji v Beogradu.}\index{relacija!$\varrho$}:

\begin{enumerate}
  \item štirikotnik $ABDC$ je paralelogram,
  \item obstajata takšni točki $P$ in $Q$, da sta štirikotnika $ABQP$ in $CDQP$ paralelograma,
  \item $A=B$ in $C=D$.
\end{enumerate}

\begin{figure}[!htb]
\centering
\input{sl.vek.5.1.2.pic}
\caption{} \label{sl.vek.5.1.2.pic}
\end{figure}

Intuitivno je jasno, da je drugi pogoj potrebno dodati zaradi primera, kadar so točke $A$, $B$, $C$ in $D$ kolinearne. Tretji pogoj se bo nanašal na t. i. vektor nič.

Iz same definicije relacije $\varrho$ dobimo naslednji izrek.



                \bizrek \label{vektRelRo}
                If $(A,B)\varrho (C,D)$ and $A\neq B$, then the line segments $AB$ and $CD$
                 are congruent, parallel, and have the same direction.
                \eizrek

\textbf{\textit{Proof.}} Ker je $A\neq B$, ostaneta le prva dva pogoja iz definicije. V tem primeru sta relaciji $AB\parallel CD$ in $AB\cong CD$ direktni posledici definicije paralelograma in izreka \ref{paralelogram}.

Glede usmerjenosti daljic bi uporabili definicijo: vzporedni daljici $XY$ in $UV$ sta \pojem{enako usmerjeni}, če je izpolnjen eden od pogojev:
\begin{itemize}
  \item $Y,V\ddot{-} XU$;
  \item obstajata takšni točki $S$ in $T$, da je $ST\parallel XY$,  $Y,T\ddot{-} XS$ in $V,T\ddot{-} US$.
\end{itemize}
\kdokaz

Naslednji izrek potrebujemo, da bomo lahko definirali vektorje.


                \bizrek
                The relation $\varrho$ on the set of pairs of points in the plane is an equivalence relation.
                \eizrek


\textbf{\textit{Proof.}} Refleksivnost in simetričnost sta direktni posledici definicije relacije $\varrho$. Za tranzitivnost pa je potrebno preveriti vse možne kombinacije pogojev 1--3 iz definicije.
\kdokaz

\pojem{Vektor} \index{vektor} sedaj definiramo kot razred ekvivalenčne relacije $\varrho$:

$$\overrightarrow{AB}=[AB]_{\varrho}=\{(X,Y);\hspace*{1mm}(X,Y)\varrho(A,B)\}.$$


\begin{figure}[!htb]
\centering
\input{sl.vek.5.1.3.pic}
\caption{} \label{sl.vek.5.1.3.pic}
\end{figure}


Točko $A$ bomo imenovali \index{začetna točka vektorja}\pojem{začetna točka}, točko $B$ pa \index{končna točka!vektorja}\pojem{končna točka} vektorja $\overrightarrow{AB}$.

Jasno je, da sta v primeru $(A,B)\varrho (C,D)$ para $(A,B)$ in $(C,D)$ iz istega razreda, kar pomeni, da je tedaj $\overrightarrow{AB}=\overrightarrow{CD}$. Velja tudi obratno: relacija $\overrightarrow{AB}=\overrightarrow{CD}$ pomeni, da velja $(A,B)\varrho (C,D)$
(Figure \ref{sl.vek.5.1.3.pic}).

Če ni potrebno posebej poudariti, za katerega predstavnika razreda gre, bomo vektorje označevali tudi z $\overrightarrow{v}$, $\overrightarrow{u}$, $\overrightarrow{w}$, ...

Vektor $\overrightarrow{AA}$, ki  predstavlja množico parov sovpadajočih točk, bomo imenovali \index{vektor!nič}\pojem{vektor nič}. Vektor nič bomo označevali tudi z $\overrightarrow{0}$
(Figure \ref{sl.vek.5.1.4.pic}).

\begin{figure}[!htb]
\centering
\input{sl.vek.5.1.4.pic}
\caption{} \label{sl.vek.5.1.4.pic}
\end{figure}

Za vektor $\overrightarrow{BA}$ bomo rekli, da je \index{vektor!nasproten}\pojem{nasproten vektor} vektorju $\overrightarrow{AB}$ in ga označili z $-\overrightarrow{AB}$. Jasno je, da definicija nasprotnega vektorja ni odvisna od izbire predstavnika razreda; nasprotni vektor dobimo, če zamenjamo začetno in končno točko vektorja (oz. vsakega para točk ustreznega razreda). Torej če označimo $\overrightarrow{v}=\overrightarrow{AB}$ je
$-\overrightarrow{v}=-\overrightarrow{AB}=\overrightarrow{BA}$
(Figure \ref{sl.vek.5.1.4.pic}).


Množico vseh vektorjev ravnine označimo z $\mathcal{V}$.

Direktna posledica izreka \ref{vektRelRo} je naslednja trditev.


                \bizrek \label{vektVzpSkl}
                If $\overrightarrow{AB}=\overrightarrow{CD}$, then the line segments $AB$ and $CD$
                 are congruent, parallel, and have the same direction.
                \eizrek

Na osnovi prejšnjega izreka lahko sedaj korektno definiramo naslednje pojme:
\begin{itemize}
  \item \index{smer vektorja}\pojem{smer} vektorja $\overrightarrow{AB}\neq\overrightarrow{0}$ je določena s poljubno premico $p\parallel AB$,
  \item \index{dolžina!vektorja}\pojem{dolžina} ali \index{intenziteta vektorja}\pojem{intenziteta}  vektorja $\overrightarrow{AB}$ je $|\overrightarrow{AB}|=|AB|$ ($|\overrightarrow{0}|=0$),
  \item \index{usmerjenost vektorja}\pojem{usmerjenost} ali \index{orientacija!vektorja}\pojem{orientacija} vektorja $\overrightarrow{AB}\neq\overrightarrow{0}$ je določena z urejenim parom $(A,B)$, kjer je $A$ začetna točka, $B$ pa končna točka tega vektorja.
\end{itemize}

Če imata vektorja $\overrightarrow{v}$ in $\overrightarrow{u}$ isto smer, pravimo tudi, da sta \index{vektorja!vzporedna}\pojem{vzporedna} ali \index{vektorja!kolinearna}\pojem{kolinearna} in označimo z $\overrightarrow{v}\parallel\overrightarrow{u}$ (pri tem pravimo tudi, da je $\overrightarrow{0}$ kolinearen z vsakim vektorjem), sicer sta vektorja \index{vektorja!nekolinearna}\pojem{nekolinearna}.

Če za kolinearna vektorja $\overrightarrow{v}$ in $\overrightarrow{u}$ velja $\overrightarrow{v}=\overrightarrow{SA}$ in $\overrightarrow{u}=\overrightarrow{SB}$, pravimo, da sta vektorja  $\overrightarrow{v}$ in $\overrightarrow{u}$ \pojem{enako usmerjena} (oznaka $\overrightarrow{u}\rightrightarrows \overrightarrow{v}$), če je $A,B\ddot{-} S$, oz. \pojem{nasprotno usmerjena} (oznaka $\overrightarrow{u}\rightleftarrows \overrightarrow{v}$), če je $A,B\div S$.

Jasno je, da sta kolinearna vektorja $\overrightarrow{AB}$ in $\overrightarrow{CD}$ enako usmerjena natanko tedaj, ko sta enako usmerjeni daljici $AB$ in $CD$.


Iz teh definicij direktno sledi, da je  $|-\overrightarrow{v}|=|\overrightarrow{v}|$ in $-\overrightarrow{v}\parallel\overrightarrow{v}$; vektorja $\overrightarrow{v}$ in $-\overrightarrow{v}$ sta nasprotno usmerjena (če je $\overrightarrow{v}=\overrightarrow{AB}$, je $-\overrightarrow{v}=\overrightarrow{BA}$).

Zelo pomemben je naslednji izrek.



                \bizrek \label{vektABCObst1TockaD}
                For every three points there $A$, $B$ and $C$ there exists exactly one point $D$,
                 such that $\overrightarrow{AB}=\overrightarrow{CD}$, i.e.
                $$(\forall A, B, C)(\exists_1 D)\hspace*{1mm}\overrightarrow{AB}=\overrightarrow{CD}.$$
                \eizrek


\textbf{\textit{Proof.}} Obravnavali bomo tri različne možnosti (Figure \ref{sl.vek.5.1.5.pic}).

\begin{figure}[!htb]
\centering
\input{sl.vek.5.1.5.pic}
\caption{} \label{sl.vek.5.1.5.pic}
\end{figure}

\textit{1)} Če je $A=B$, je $D=C$ edina točka, za katero je $(A,B)\varrho (C,D)$ oz. $\overrightarrow{CD}=\overrightarrow{AB}=\overrightarrow{0}$.

\textit{2)} Naj bo $A\neq B$ in $C\notin AB$. Potem je $D$ četrto oglišče paralelograma $ABDC$ (ki obstaja in je eno samo po posledici
Playfairjevega aksioma \ref{Playfair1}) ter edina točka, za katero je $(A,B)\varrho (C,D)$ oz. $\overrightarrow{CD}=\overrightarrow{AB}$.

\textit{3)} Naj bo $A\neq B$ in $C\in AB$. Naj bo $P$ poljubna točka, ki ne leži na premici $AB$. Po dokazanem delu \textit{2)} obstaja natanko ena točka $Q$, za katero je $\overrightarrow{PQ}=\overrightarrow{AB}$, nato pa tudi  natanko ena točka $D$, za katero je $\overrightarrow{CD}=\overrightarrow{PQ}=\overrightarrow{AB}$.
\kdokaz

Prejšnji izrek lahko zapišemo tudi v drugačni obliki (Figure \ref{sl.vek.5.1.6.pic}).



                \bizrek \label{vektAvObst1TockaB}
                For every point $X$ and every vector $\overrightarrow{v}$ there exists exactly one point $Y$,
                 such that $\overrightarrow{XY}=\overrightarrow{v}$, i.e.
                $$(\forall X)(\forall \overrightarrow{v})(\exists_1 Y)\hspace*{1mm}\overrightarrow{XY}=\overrightarrow{v}.$$
                \eizrek

\begin{figure}[!htb]
\centering
\input{sl.vek.5.1.6.pic}
\caption{} \label{sl.vek.5.1.6.pic}
\end{figure}


Na osnovi dokazanega je jasno, da lahko za vsak vektor $\overrightarrow{v}$ izberemo poljubnega predstavnika njegovega razreda s poljubno začetno točko. Na podoben način bi lahko dokazali, da je to možno tudi glede na končno točko. To intuitivno pomeni, da lahko vektor vzporedno premikamo na en sam način do njegove začetne oz. končne točke.

Dva poljubna vektorja lahko torej postavimo tako, da imata isto začetno točko ali pa tako, da je začetna točka drugega hkrati končna točka prvega vektorja.
To dejstvo nam omogoča definicijo vsote dveh vektorjev.

Če je $\overrightarrow{v}=\overrightarrow{AB}$ in $\overrightarrow{u}=\overrightarrow{BC}$, je \index{vsota!vektorjev}\pojem{vsota vektorjev} $\overrightarrow{v}$ in $\overrightarrow{u}$ vektor $\overrightarrow{v}+\overrightarrow{u}=\overrightarrow{AC}$ (Figure \ref{asl.vek.5.1.7.pic}). Torej velja $\overrightarrow{AB}+\overrightarrow{BC}=\overrightarrow{AC}$.


\begin{figure}[!htb]
\centering
\input{sl.vek.5.1.7.pic}
\caption{} \label{asl.vek.5.1.7.pic}
\end{figure}


Potrebno je dokazati korektnost definicije - da vsota vektorjev ni odvisna od izbire predstavnikov dveh razredov ($\overrightarrow{v}$ in $\overrightarrow{u}$) oz. od izbire točke $A$.

                \bizrek \label{vektKorektDefSest}
                If $A$, $B$, $C$, $A'$, $B'$ and $C'$ are points such that $\overrightarrow{AB}=\overrightarrow{A'B'}$ and
                 $\overrightarrow{BC}=\overrightarrow{B'C'}$, then there is also $\overrightarrow{AC}=\overrightarrow{A'C'}$.
                \eizrek

\begin{figure}[!htb]
\centering
\input{sl.vek.5.1.7a.pic}
\caption{} \label{sl.vek.5.1.7a.pic}
\end{figure}


\textbf{\textit{Proof.}}  (Figure \ref{sl.vek.5.1.7.pic}).

V formalnem dokazu bi obravnavali različne možnosti glede na relacijo $\varrho$.
  \kdokaz

 Pravilo seštevanja vektorjev, pri katerem uporabljamo že omenjeno enakost:

\begin{itemize}
  \item $\overrightarrow{AB}+\overrightarrow{BC}=\overrightarrow{AC}$,
\end{itemize}

 imenujemo \index{pravilo!trikotniško}\pojem{trikotniško pravilo} ali \index{Chaslesova identiteta}\pojem{Chaslesova\footnote{\index{Chasles, M.} \textit{M. Chasles} (1793–-1880),  francoski matematik.} identiteta}.
 Pravilo se tako imenuje kljub temu, da velja tudi v primeru, kadar so $A$, $B$ in $C$ kolinearne točke (oz. $\overrightarrow{v}$ in $\overrightarrow{u}$ kolinearna vektorja); takrat namreč ne gre za trikotnik (Figure \ref{sl.vek.5.1.8.pic}).

\begin{figure}[!htb]
\centering
\input{sl.vek.5.1.8.pic}
\caption{} \label{sl.vek.5.1.8.pic}
\end{figure}



Izkaže se, da množica vseh vektorjev $\mathcal{V}$ glede na operacijo seštevanja
vektorjev predstavlja strukturo t. i. \index{group!commutative}\pojem{commutative group} (ali \index{group!Abelian}\pojem{Abelian\footnote{\index{Abel, N. H.}\textit{N. H. Abel} (1802--1829), norveški matematik.} grupe}). Pojem grupe (ki ni vedno komutativna) smo že omenili v razdelku \ref{odd2AKSSKL} pri izometrijah, podrobneje pa se bomo s to strukturo ukvarjali  v razdelku \ref{odd6Grupe}. Lastnosti  te strukture (komutativne grupe) so dane v zapisu
naslednjega izreka.



                \bizrek \label{vektKomGrupa} The ordered pair $(\mathcal{V},+)$ form a commutative  group, which means that:
                \begin{enumerate}
                  \item $(\forall \overrightarrow{v}\in\mathcal{V})(\forall \overrightarrow{v}\in\mathcal{V})\hspace*{1mm}
                      \overrightarrow{v}+\overrightarrow{v}\in\mathcal{V}$
                   (\index{grupoidnost}\textit{closure}),
                  \item $(\forall \overrightarrow{u}\in\mathcal{V})
                        (\forall \overrightarrow{v}\in\mathcal{V})
                        (\forall \overrightarrow{w}\in\mathcal{V})
                        \hspace*{1mm}
                      \left(\overrightarrow{u}+\overrightarrow{v}\right)+\overrightarrow{w}
                      = \overrightarrow{u}+\left(\overrightarrow{v}+\overrightarrow{w}\right)
                      $
                   (\index{asociativnost}\textit{associativity}),
                  \item $(\exists \overrightarrow{e}\in\mathcal{V})
                        (\forall \overrightarrow{v}\in\mathcal{V})
                        \hspace*{1mm}
                     \overrightarrow{v}+\overrightarrow{e}
                      = \overrightarrow{e}+\overrightarrow{v}
                      $
                   (\index{nevtralni element}\textit{identity element}),
                  \item  $(\forall \overrightarrow{v}\in\mathcal{V})
                        (\exists \overrightarrow{u}\in\mathcal{V})
                        \hspace*{1mm}
                     \overrightarrow{v}+\overrightarrow{u}
                      = \overrightarrow{u}+\overrightarrow{v}=\overrightarrow{e}
                      $
                   (\index{inverzni element}\textit{inverse element}),
                  \item   $(\forall \overrightarrow{v}\in\mathcal{V})
                        (\forall \overrightarrow{u}\in\mathcal{V})
                        \hspace*{1mm}
                     \overrightarrow{v}+\overrightarrow{u}
                      = \overrightarrow{u}+\overrightarrow{v}
                      $
                   (\index{komutativnost}\textit{commutativity}).
                \end{enumerate}
                \eizrek

\textbf{\textit{Proof.}}

 \textit{1)} Vsota dveh poljubnih vektorjev je vektor, kar sledi iz same definicije vsote dveh vektorjev.

 \textit{2)} Naj bodo $\overrightarrow{u}$, $\overrightarrow{v}$ in $\overrightarrow{w}$ poljubni vektorji ter $A$, $B$, $C$ in $D$ takšne točke, da velja $\overrightarrow{u}=\overrightarrow{AB}$, $\overrightarrow{v}=\overrightarrow{BC}$ in $\overrightarrow{w}=\overrightarrow{CD}$ (izrek \ref{vektAvObst1TockaB}) (Figure \ref{sl.vek.5.1.9a.pic}).


\begin{figure}[!htb]
\centering
\input{sl.vek.5.1.9a.pic}
\caption{} \label{sl.vek.5.1.9a.pic}
\end{figure}


  Po definiciji seštevanja vektorjev (in korektnosti te definicije - izrek \ref{vektKorektDefSest}) je:
 \begin{eqnarray*}
 & & \left(\overrightarrow{u}+\overrightarrow{v}\right)+\overrightarrow{w}=
     \left(\overrightarrow{AB}+\overrightarrow{BC}\right)+\overrightarrow{CD}=
     \overrightarrow{AC}+\overrightarrow{CD}=\overrightarrow{AD}\\
  & & \overrightarrow{u}+\left(\overrightarrow{v}+\overrightarrow{w}\right)                     =\overrightarrow{AB}+\left(\overrightarrow{BC}+\overrightarrow{CD}\right)=
     \overrightarrow{AB}+\overrightarrow{BD}=\overrightarrow{AD}
 \end{eqnarray*}
  Iz tega pa sledi asociativnost seštevanja vektorjev.

 \textit{3)}  V primeru seštevanja vektorjev je nevtralni element kar vektor nič oz. $\overrightarrow{e}=\overrightarrow{0}$, ker za vsak vektor $\overrightarrow{v}=\overrightarrow{AB}$ velja:

 \begin{eqnarray*}
 & & \overrightarrow{v}+\overrightarrow{0}=
     \overrightarrow{AB}+\overrightarrow{BB}=
     \overrightarrow{AB}=\overrightarrow{v}\\
  & & \overrightarrow{0}+\overrightarrow{v}=
     \overrightarrow{AA}+\overrightarrow{AB}=
     \overrightarrow{AB}=\overrightarrow{v}.
 \end{eqnarray*}


 \textit{4)}  V primeru seštevanja vektorjev je za poljubni vektor $\overrightarrow{v}=\overrightarrow{AB}$ inverzni elament nasproten vektor $\overrightarrow{u}=-\overrightarrow{v}=\overrightarrow{BA}$:

 \begin{eqnarray*}
 & & \overrightarrow{v}+\left(-\overrightarrow{v}\right)=
     \overrightarrow{AB}+\overrightarrow{BA}=
     \overrightarrow{AA}=\overrightarrow{0}\\
  & & -\overrightarrow{v}+\overrightarrow{v}=
     \overrightarrow{BA}+\overrightarrow{AB}=
     \overrightarrow{BB}=\overrightarrow{0}.
 \end{eqnarray*}

 \textit{5)} Naj bosta $v$ in $u$ poljubna vektorja ter $A$, $B$ in $C$ takšne točke, da velja
 $\overrightarrow{v}=\overrightarrow{AB}$ in $\overrightarrow{u}=\overrightarrow{BC}$ (izrek \ref{vektAvObst1TockaB}). Obravnavali bomo dva primera (Figure \ref{sl.vek.5.1.9b.pic}).


\begin{figure}[!htb]
\centering
\input{sl.vek.5.1.9b.pic}
\caption{} \label{sl.vek.5.1.9b.pic}
\end{figure}

 \textit{a)} Predpostavimo, da so točke $A$, $B$ in $C$ nekolinearne oz. $\overrightarrow{v}$ in $\overrightarrow{u}$ nekolinearna vektorja. Označimo z $D$ četrto oglišče paralelograma $ABCD$. Po definiciji relacije $\varrho$ je $(A,B)\varrho (D,C)$ in $(B,C)\varrho (A,D)$, iz definicije vektorjev pa nato sledi $\overrightarrow{AB}=\overrightarrow{DC}$ in $\overrightarrow{BC}=\overrightarrow{AD}$. Torej:
  $$\overrightarrow{v}+\overrightarrow{u}=
  \overrightarrow{AB}+\overrightarrow{BC}=
  \overrightarrow{AC}=
  \overrightarrow{AD}+\overrightarrow{DC}=
  \overrightarrow{BC}+\overrightarrow{AB}
                      = \overrightarrow{u}+\overrightarrow{v}.$$

 \textit{a)} Naj bodo $A$, $B$ in $C$ kolinearne točke oz. $\overrightarrow{v}$ in $\overrightarrow{u}$ kolinearna vektorja. Izrazimo vektorja $\overrightarrow{v}$ in $\overrightarrow{u}$ kot vsoti $\overrightarrow{v}=\overrightarrow{v}_1+\overrightarrow{v}_2$ in $\overrightarrow{u}=\overrightarrow{u}_1+\overrightarrow{u}_2$, tako da nobena dva od vektorjev $\overrightarrow{v}_1$,  $\overrightarrow{v}_2$,  $\overrightarrow{u}_1$ in $\overrightarrow{u}_2$ nista kolinearna. Če sedaj uporabimo dokazano v primeru \textit{a)}, dobimo:
 $$\overrightarrow{v}+\overrightarrow{u}=
  \overrightarrow{v}_1+\overrightarrow{v}_2+
  \overrightarrow{u}_1+\overrightarrow{u}_2=
   \overrightarrow{u}_1+\overrightarrow{u}_2+
  \overrightarrow{v}_1+\overrightarrow{v}_2=
                       \overrightarrow{u}+\overrightarrow{v},$$ kar je bilo treba dokazati. \kdokaz

Naslednja trditev je posledica prejšnjega izreka.

                \bzgled \label{vektABCD_ACBD}
                For arbitrary points $A$, $B$, $C$ and $D$ is
                $$\overrightarrow{AB}=\overrightarrow{CD}\Rightarrow \overrightarrow{AC}=\overrightarrow{BD}.$$
                \ezgled

\begin{figure}[!htb]
\centering
\input{sl.vek.5.1.10.pic}
\caption{} \label{sl.vek.5.1.10.pic}
\end{figure}

\textbf{\textit{Proof.}} Po definiciji seštevanja vektorjev in izreka \ref{vektKomGrupa} (komutativnost) je (Figure \ref{sl.vek.5.1.10.pic}):
$\overrightarrow{AC}=
\overrightarrow{AB}+ \overrightarrow{BC}=
\overrightarrow{CD}+ \overrightarrow{BC}=
\overrightarrow{BC}+\overrightarrow{CD}=
\overrightarrow{BD}.$
  \kdokaz

  Še ena posledica komutativnosti seštevanja vektorjev (izrek \ref{vektKomGrupa}) je naslednje pravilo seštevanja za nekolinearna vektorja.

  \begin{itemize}
    \item \textit{Za vsake tri nekolinearne točke $A$, $B$ in $C$ velja $\overrightarrow{AB}+\overrightarrow{AC}=\overrightarrow{AD}$ natanko tedaj, ko je $ABDC$ paralelogram.}
  \end{itemize}


\begin{figure}[!htb]
\centering
\input{sl.vek.5.1.11.pic}
\caption{} \label{sl.vek.5.1.11.pic}
\end{figure}


  To pravilo imenujemo \index{pravilo!paralelogramsko}\pojem{paralelogramsko pravilo}\footnote{Paralelogramsko pravilo je bilo verjetno znano že Starim Grkom. Predpostavlja se, da ga je omenjal starogrški matematik in filozof \index{Aristotel} \textit{Aristotel} (384--322 pr. n. š.)} (Figure \ref{sl.vek.5.1.11.pic}).

  Posledica asociativnosti iz izreka \ref{vektKomGrupa} je naslednje pravilo za seštevanje vektorjev:
  \begin{itemize}
    \item $\overrightarrow{A_1A_2}+\overrightarrow{A_2A_3}+\cdots +\overrightarrow{A_{n-1}A_n}=\overrightarrow{A_1A_n}$,
  \end{itemize}

ki ga imenujemo \index{pravilo!poligonsko}\pojem{poligonsko pravilo} (Figure \ref{asl.vek.5.1.12.pic}).

\begin{figure}[!htb]
\centering
\input{sl.vek.5.1.12.pic}
\caption{} \label{asl.vek.5.1.12.pic}
\end{figure}

Direktna posledica tega pravila je naslednja trditev (Figure \ref{sl.vek.5.1.13.pic}).

                \bzgled
                For arbitrary points $A_1$, $A_2$,..., $A_n$ in a plane is
                $$\overrightarrow{A_1A_2}+\overrightarrow{A_2A_3}+\cdots +\overrightarrow{A_{n-1}A_n}+\overrightarrow{A_nA_1}=\overrightarrow{0}.$$
                \ezgled

\begin{figure}[!htb]
\centering
\input{sl.vek.5.1.13.pic}
\caption{} \label{sl.vek.5.1.13.pic}
\end{figure}

Z naslednjim izrekom bomo podali ekvivalentno definicijo središča daljice.

                \bzgled \label{vektSredDalj}
                A point  $S$ is the midpoint of a line segment  $AB$ if and only if $\overrightarrow{SA}=-\overrightarrow{SB}$ i.e. $\overrightarrow{SA}+\overrightarrow{SB}=\overrightarrow{0}$.
                \ezgled

\begin{figure}[!htb]
\centering
\input{sl.vek.5.1.14.pic}
\caption{} \label{sl.vek.5.1.14.pic}
\end{figure}

 \textbf{\textit{Solution.}} (Figure \ref{sl.vek.5.1.14.pic})

 ($\Leftarrow$) Predpostavimo, da velja $\overrightarrow{SA}=-\overrightarrow{SB}$ oz. $\overrightarrow{AS}=\overrightarrow{SB}$. To pomeni, da sta $AS$ in $SB$ vzporedni, enako usmerjeni in skladni daljici (izrek \ref{vektVzpSkl}) oz. $SA\cong BS$ in $\mathcal{B}(A,S,B)$. Torej je $S$ središče daljice $AB$.

($\Rightarrow$) Naj bo sedaj $S$ središče daljice $AB$. Dovolj je dokazati, da velja $\overrightarrow{SA}=\overrightarrow{BS}$ oz. $\overrightarrow{BS}=\overrightarrow{SA}$.
 Predpostavimo, da velja $\overrightarrow{BS}=\overrightarrow{SA'}$.
 Toda v tem primeru je $SA'\cong BS$ in $\mathcal{B}(A',S,B)$. Po izreku \ref{ABnaPoltrakCX} je $A=A'$, zato je $\overrightarrow{BS}=\overrightarrow{SA'}=\overrightarrow{SA}$.
\kdokaz

Definirajmo še operacijo odštevanja dveh vektorjev. \pojem{Razlika vektorjev} \index{razlika!vektorjev} $\overrightarrow{v}$ in $\overrightarrow{u}$ je vsota vektorjev $\overrightarrow{v}$ in $-\overrightarrow{u}$ oz.
$$\overrightarrow{v}-\overrightarrow{u}=\overrightarrow{v}+(-\overrightarrow{u}).$$

            \bzgled \label{vektOdsev}
            For arbitrary three points $O$, $A$ and $B$ is
            $$\overrightarrow{OB}-\overrightarrow{OA}=\overrightarrow{AB}.$$
            \ezgled


\begin{figure}[!htb]
\centering
\input{sl.vek.5.1.15.pic}
\caption{} \label{sl.vek.5.1.15.pic}
\end{figure}

 \textbf{\textit{Solution.}} (Figure \ref{sl.vek.5.1.15.pic})

 Iz same definicije odštevanja in seštevanja vektorjev ter komutativnosti seštevanja
 (izrek \ref{vektKomGrupa})sledi:
 $$\overrightarrow{OB}-\overrightarrow{OA}=
 \overrightarrow{OB}+(-\overrightarrow{OA})=
 \overrightarrow{OB}+\overrightarrow{AO}=
\overrightarrow{AO}+\overrightarrow{OB}=
 \overrightarrow{AB},$$ kar je bilo treba dokazati. \kdokaz



          \bzgled \label{vektOPi}
           Suppose that a point $O$ lies on a line $p$ and let $P_1, P_2, \cdots, P_n$ be points
            lying in the same half-plane $\alpha$ with the edge $p$.
            If:
            $$\overrightarrow{OS_n}=\overrightarrow{OP_1}+\overrightarrow{OP_2}+
               \cdots+\overrightarrow{OP_n},$$
            then the point $S_n$ also lies in the half-plane $\alpha$.
           \ezgled

\begin{figure}[!htb]
\centering
\input{sl.vek.5.2.1a.pic}
\caption{} \label{sl.vek.5.2.1a.pic}
\end{figure}

 \textbf{\textit{Solution.}} Dokaz bomo izpeljali z indukcijo
 glede na $n$ (Figure \ref{sl.vek.5.2.1a.pic}).

 (\textit{i}) Za $n=1$ je jasno $\overrightarrow{OS_1}=\overrightarrow{OP_1}$,
 in $P_1\in\alpha\Rightarrow S_1=P_1\in\alpha$.

(\textit{ii}) Predpostavimo, da trditev velja za $n=k$. Dokažimo, da potem
 velja tudi za $n=k+1$. Naj bodo $P_1, P_2, \cdots, P_k,
 P_{k+1}\in\alpha$. Dokažimo, da je potem tudi $S_{k+1}\in
 \alpha$. Velja:
   $$\overrightarrow{OS_{k+1}}=\overrightarrow{OP_1}+\overrightarrow{OP_2}+
         \cdots+\overrightarrow{OP_k}+\overrightarrow{OP_{k+1}}=
         \overrightarrow{OS_k}+\overrightarrow{OP_{k+1}}.$$
Točka $S_k$ po indukcijski predpostavki leži v polravnini
$\alpha$. Ker  $S_k,P_{k+1}\in \alpha$ in $O\in p$, je kot $\angle
S_kOP_{k+1}$ konveksen in cel leži v polravnini $\alpha$. To
pomeni, da tudi točka $S_{n+1}$ (kot četrto oglišče
paralelograma $P_{k+1}OP_kS_{k+1}$) leži v notranjosti kota $\angle
S_kOP_{k+1}$ oz. v polravnini $\alpha$.

 \kdokaz



%________________________________________________________________________________
\poglavje{Linear Combination of Vectors} \label{odd5LinKombVekt}

S pomočjo pojma seštevanja vektorjev lahko definiramo pojem množenja vektorja s poljubnim naravnim številom:
 $$1\cdot \overrightarrow{v} =\overrightarrow{v},\hspace*{2mm} (n+1)\cdot \overrightarrow{v}=n\cdot\overrightarrow{v}+\overrightarrow{v},\hspace*{2mm} (n\in \mathbb{N}).$$
 Definicijo lahko razširimo tudi na cela števila: $0\cdot \overrightarrow{v}=\overrightarrow{0}$ in $-n\cdot \overrightarrow{v}=n\cdot(-\overrightarrow{v})$.
  Jasno je, da sta v tem primeru vektorja $\overrightarrow{v}$ in $l\cdot \overrightarrow{v}$ ($l\in \mathbb{Z}$) vedno kolinearna in velja
  $|l\cdot \overrightarrow{v}|=|l|\cdot |\overrightarrow{v}|$.
  Na ta način dobimo idejo za definicijo množenja vektorja s poljubnim realnim številom $\lambda\in \mathbb{R}$.

  Najprej za $\lambda=0$ definirajmo $0\cdot \overrightarrow{v}=\overrightarrow{0}$. Če
 je $\lambda\neq 0$ in $\overrightarrow{v}=\overrightarrow{AB}$, je $\lambda\cdot\overrightarrow{v}=\overrightarrow{AC}$, kjer je $C$ takšna točka na premici $AB$, da velja (Figure \ref{sl.vek.5.2.1.pic}):

 \begin{itemize}
   \item $|AC|=|\lambda| \cdot|AB|$,
   \item $C,B\ddot{-} A$, če je $\lambda>0$,
   \item $C,B\div A$, če je $\lambda<0$.
 \end{itemize}


\begin{figure}[!htb]
\centering
\input{sl.vek.5.2.1.pic}
\caption{} \label{sl.vek.5.2.1.pic}
\end{figure}



 Jasno je, da je točka $C$ za vsak $\lambda\neq 0$ enolično določena, vektor $\lambda\cdot \overrightarrow{v}$ pa ni odvisen od izbire točke $A$. To pomeni, da je definicija množenja vektorja z realnim številom korektna.

 Iz same definicije sledi, da za vsak vektor $\overrightarrow{v}$ in vsako realno število $\lambda\in \mathbb{R}$ velja $|\lambda \cdot \overrightarrow{v}|=|\lambda| \cdot |\overrightarrow{v}|$.

Množenje vektorja $\overrightarrow{v}$ z realnim številom $\lambda$ bomo imenovali tudi \index{množenje vektorja s skalarjem}\pojem{množenje vektorja $\overrightarrow{v}$ s skalarjem $\lambda$}.

Podobno kot pri množenju algebrskih izrazov bomo oznako $\cdot$ množenja vektorja z realnim številom običajno izpuščali, oz. bomo namesto $\lambda\cdot\overrightarrow{v}$  raje pisali kar $\lambda\overrightarrow{v}$.

 Naslednja trditev nam da potreben in zadosten pogoj za kolinearnost dveh vektorjev.



            \bizrek \label{vektKriterijKolin}
            Vectors $\overrightarrow{v}$ and $\overrightarrow{u}$ ($\overrightarrow{v},\overrightarrow{u}\neq \overrightarrow{0}$) are collinear if and only if there is such $\lambda\in \mathbb{R}$, that is $\overrightarrow{u}=\lambda\cdot\overrightarrow{v}$.
            \eizrek


\begin{figure}[!htb]
\centering
\input{sl.vek.5.2.2.pic}
\caption{} \label{sl.vek.5.2.2.pic}
\end{figure}

 \textbf{\textit{Proof.}} (Figure \ref{sl.vek.5.2.2.pic})

  Naj bosta $\overrightarrow{v}$ in $\overrightarrow{u}$ poljubna vektorja ($\overrightarrow{v},\overrightarrow{u}\neq \overrightarrow{0}$), $A$ poljubna točka ter $B$ in $C$ takšni točki, da velja $\overrightarrow{AB}=\overrightarrow{v}$ in $\overrightarrow{AC}=\overrightarrow{u}$ (izrek \ref{vektAvObst1TockaB}).


 ($\Leftarrow$) Predpostavimo najprej, da velja $\overrightarrow{u}=\lambda\cdot\overrightarrow{v}$ za nek $\lambda\in \mathbb{R}$. Po definiciji točka $C$ leži na premici $AB$, kar pomeni, da sta vektorja $\overrightarrow{AB}$ in $\overrightarrow{AC}$ oz. $\overrightarrow{v}$ in $\overrightarrow{u}$  kolinearna.

 ($\Rightarrow$) Predpostavimo sedaj, da sta vektorja $\overrightarrow{v}$ in $\overrightarrow{u}$ oz. $\overrightarrow{AB}$ in $\overrightarrow{AC}$ kolinearna.
  V tem primeru je:
  \begin{eqnarray*}
  \overrightarrow{u}&=&\overrightarrow{AC}=\frac{|AC|}{|AB|}\cdot \overrightarrow{AB}=
  \frac{|\overrightarrow{u}|}{|\overrightarrow{v}|}\cdot \overrightarrow{v},
  \hspace*{2mm} \textrm{ če je } C,B\ddot{-} A;\\
  \overrightarrow{u}&=&\overrightarrow{AC}=-\frac{|AC|}{|AB|}\cdot \overrightarrow{AB}=
  -\frac{|\overrightarrow{u}|}{|\overrightarrow{v}|}\cdot \overrightarrow{v},
  \hspace*{2mm} \textrm{ če je } C,B\div A,
  \end{eqnarray*}
  kar je bilo treba dokazati. \kdokaz

Če kombiniramo operaciji seštevanja vektorjev in množenja vektorja s skalarjem, dobimo t. i. linearno kombinacijo vektorjev. Še natančneje - za poljubno $n$-terico vektorjev $(\overrightarrow{a_1},\overrightarrow{a_2},\ldots,\overrightarrow{a_n})$ in poljubno $n$-terico realnih števil $(\alpha_1,\alpha_2,\ldots,\alpha_n)\in \mathbb{R}^n$ je vektor
 $$\overrightarrow{v}=\alpha_1\cdot \overrightarrow{a_1}+
 \alpha_2\cdot \overrightarrow{a_2}
 +\cdots +\alpha_n\cdot \overrightarrow{a_n}$$
 \index{linearna kombinacija vektorjev}\pojem{linearna kombinacija vektorjev} $\overrightarrow{a_1}$, $\overrightarrow{a_2}$, $\ldots$, $\overrightarrow{a_n}$.



                \bizrek \label{vektLinKombNicLema}
                Let $\overrightarrow{a}$ and $\overrightarrow{b}$ be two non-collinear non-zero vectors
                and  $(\alpha,\beta)\in \mathbb{R}^2$ two real numbers.
                If $\alpha\cdot\overrightarrow{a}=
                \beta\cdot\overrightarrow{b}$, then $\alpha=\beta=0$.
                \eizrek

 \textbf{\textit{Proof.}} Predpostavimo nasprotno - brez škode za splošnost, da velja $\alpha\neq0$. Potem je $\overrightarrow{a}=
 \frac{\beta}{\alpha}\cdot\overrightarrow{b}$, kar pomeni (izrek \ref{vektKriterijKolin}), da je vektor $\overrightarrow{b}$ kolinearen z vektorjem $\overrightarrow{a}$. To je v nasprotju s predpostavko, zato je $\alpha=\beta=0$.
\kdokaz

Direktna posledica prejšnjega izreka je naslednja trditev.



                 \bizrek \label{vektLinKombNic}
                If for the linear combination of two non-collinear non-zero vectors $\overrightarrow{a}$ and $\overrightarrow{b}$ holds
                $$\alpha\cdot\overrightarrow{a}+
                \beta\cdot\overrightarrow{b}=\overrightarrow{0},$$
                 then $\alpha=\beta=0$.
                \eizrek

 \textbf{\textit{Proof.}} Če relacijo $\alpha\cdot\overrightarrow{a}+
 \beta\cdot\overrightarrow{b}=\overrightarrow{0}$ zapišemo v obliki $\alpha\cdot\overrightarrow{a}=
 -\beta\cdot\overrightarrow{b}$, vidimo, da je trditev direktna posledica prejšnjega izreka.
\kdokaz

 Zelo pomemben je naslednji izrek, ki se nanaša na predstavljanje poljubnega vektorja kot linearne kombinacije dveh nekolinearnih neničelnih vektorjev.


                \bizrek \label{vektLinKomb1Razcep}
                Let $\overrightarrow{a}$ and $\overrightarrow{b}$ be two non-collinear non-zero vectors in the same plane.
                Each vector $\overrightarrow{v}$ in this plane can be express  in a single way as a linear combination
                of vectors  $\overrightarrow{a}$ and $\overrightarrow{b}$, i.e. there is exactly
                one pair of real numbers $(\alpha,\beta)\in \mathbb{R}^2$, such that
                $$\overrightarrow{v}=\alpha\cdot\overrightarrow{a}+
                \beta\cdot\overrightarrow{b}.$$
                \eizrek



\begin{figure}[!htb]
\centering
\input{sl.vek.5.2.3.pic}
\caption{} \label{sl.vek.5.2.3.pic}
\end{figure}

 \textbf{\textit{Proof.}}  (Figure \ref{sl.vek.5.2.3.pic})

 Naj bo $O$ poljubna točka ter $A$, $B$ in $V$ takšne točke, da velja $\overrightarrow{OA}=\overrightarrow{a}$, $\overrightarrow{OB}=\overrightarrow{b}$  in $\overrightarrow{OV}=\overrightarrow{v}$ (izrek \ref{vektAvObst1TockaB}). Označimo z $A'$ presečišče premice $OA$ in vzporednice premice $OB$ skozi točko $V$ ter  z $B'$ presečišče premice $OB$ in vzporednice premice $OA$ skozi točko $V$. Torej je štirikotnik $OA'VB'$ paralelogram. Po izreku \ref{vektKriterijKolin} obstajata takšni $\alpha,\beta\in \mathbb{R}$, da velja
 $\overrightarrow{OA'}=
 \alpha\cdot\overrightarrow{OA}=
 \alpha\cdot\overrightarrow{a}$ in
 $\overrightarrow{OB'}=
 \beta\cdot\overrightarrow{OB}=
 \beta\cdot\overrightarrow{b}$. Po paralelogramskem pravilu  je:

 $$\overrightarrow{v}=
 \overrightarrow{OV}=\overrightarrow{OA'}+\overrightarrow{OB'}=
 \alpha\cdot\overrightarrow{a}+\beta\cdot\overrightarrow{b}.$$

Dokažimo, da je $(\alpha,\beta)\in \mathbb{R}^2$ edini tak par. Predpostavimo, da za
$(\alpha',\beta')\in \mathbb{R}^2$ velja $\overrightarrow{v}=
 \alpha'\cdot\overrightarrow{a}+\beta'\cdot\overrightarrow{b}$. Iz
 $\overrightarrow{v}=
 \alpha'\cdot\overrightarrow{a}+\beta'\cdot\overrightarrow{b}=
 \alpha\cdot\overrightarrow{a}+\beta\cdot\overrightarrow{b}$ sledi
 $(\alpha'-\alpha)\cdot\overrightarrow{a}+(\beta'-\beta)\cdot\overrightarrow{b}=
\overrightarrow{0}$. Po prejšnjem izreku \ref{vektLinKombNic} je
  $\alpha'-\alpha=\beta'-\beta=0$ oz.  $\alpha'=\alpha$ in $\beta'=\beta$.
\kdokaz



                \bizrek \label{vektVektorskiProstor}
                The set $\mathcal{V}$ of all vectors in the plane form so-called
                 \index{vector space} vector space \footnote{ Prvi, ki je leta 1888 na sodoben način definiral pojem vektorskega prostora,
                 je bil italijanski matematik \index{Peano, G.}\textit{G. Peano} (1858–-1932).} over the field $\mathbb{R}$, which means that:
                \begin{enumerate}
                  \item the ordered pair $(\mathcal{V},+)$ form a commutative  group,
                  \item $(\forall \alpha \in \mathbb{R})(\forall \overrightarrow{v}\in \mathcal{V})\hspace*{1mm} \alpha\cdot \overrightarrow{v} \in \mathcal{V}$,
                  \item $(\forall \alpha,\beta \in \mathbb{R})(\forall \overrightarrow{v}\in \mathcal{V})\hspace*{1mm} \alpha\cdot (\beta\cdot\overrightarrow{v})=(\alpha\beta)\cdot \overrightarrow{v}$,
                  \item $(\forall \overrightarrow{v}\in \mathcal{V})\hspace*{1mm} 1\cdot \overrightarrow{v}=\overrightarrow{v}$,
                  \item $(\forall \alpha,\beta \in \mathbb{R})(\forall \overrightarrow{v}\in \mathcal{V})\hspace*{1mm} (\alpha+ \beta)\cdot\overrightarrow{v}=\alpha\cdot \overrightarrow{v}+\beta\cdot \overrightarrow{v}$,
                  \item $(\forall \alpha \in \mathbb{R})(\forall \overrightarrow{v},\overrightarrow{u}\in \mathcal{V})\hspace*{1mm} \alpha\cdot(\overrightarrow{v}+\overrightarrow{u})=
                      \alpha\cdot \overrightarrow{v}+\alpha\cdot \overrightarrow{u}$.
                \end{enumerate}
                \eizrek



 \textbf{\textit{Proof.}} Trditev $\textit{1}$ je v bistvu že dokazan izrek \ref{vektKomGrupa}. Trditve $\textit{2}-\textit{5}$ so direktna posledica definicije množenja vektorja s skalarjem. Dokaz trditve $\textit{6}$ pa ni tako enostavna - v dokazu bi morali uporabiti tudi Dedekindov aksiom zveznosti \ref{aksDed}.
 \kdokaz


Za strukturo vektorskega prostora na množici vseh vektorjev ravnine $\mathcal{V}$ nad obsegom $\mathbb{R}$ bomo uporabljali oznako $\mathcal{\overrightarrow{V}}^2$.
 Če bi obravnavali vektorje v evklidskem prostoru, bi prav tako dobili vektorski prostor. Toda vektorski prostor lahko obravnavamo tudi bolj abstraktno kot poljubni urejeni par $(\mathcal{\overrightarrow{V}},\mathcal{F})$, ker $(\mathcal{\overrightarrow{V}},+)$ predstavlja komutativno grupo, $(\mathcal{F},+,\cdot)$ obseg in so izpolnjeni vsi pogoji $\textit{2}-\textit{6}$ iz prejšnjega izreka.

Ker lahko vsak vektor v ravnini izrazimo kot linearno kombinacijo dveh nekolinearnih neničelnih vektorjev $\overrightarrow{a}$ in $\overrightarrow{b}$ te ravnine, pravimo, da sta takšna dva vektorja $\overrightarrow{a}$ in $\overrightarrow{b}$ t. i. \index{baza vektorskega prostora}\pojem{baza} vektorskega prostora $\mathcal{\overrightarrow{V}}^2$. Če za nek vektor $\overrightarrow{v}\in \mathcal{V}$ velja $\overrightarrow{v}=\alpha\cdot\overrightarrow{a}+
                \beta\cdot\overrightarrow{b}$, pravimo, da par $(\alpha,\beta)$ predstavlja \index{koordinate vektorja}\pojem{koordinati} vektorja $\overrightarrow{v}$ v bazi $(\overrightarrow{a},\overrightarrow{b})$. Iz izreka \ref{vektLinKomb1Razcep} sledi, da so v poljubni bazi koordinate vsakega vektorja enolično določene.



V vektorskem prostoru $\mathcal{\overrightarrow{V}}^2$ je torej neskončno mnogo baz,  število vektorjev v poljubni bazi je vedno 2, zato pravimo, da ima vektorski prostor $\mathcal{\overrightarrow{V}}^2$ \pojem{dimenzijo} 2 oz. je ravnina \pojem{dvorazsežna}.

Jasno je, da je dimenzija vektorskega prostora $\mathcal{\overrightarrow{V}}^3$, ki ga določajo vektorji v evklidskem prostoru, enaka 3. Evklidski prostor je torej \pojem{trirazsežen}, vsak vektor v neki bazi je določen s trojico realnih števil, ki predstavljajo koordinate tega vektorja.

V poljubnem vektorskem prostoru na podoben način določamo bazo in dimenzijo. Ker v splošnem primeru dimenzija prostora ni omejena na 3, nam to omogoča raziskovanje \pojem{$n$-razsežnih} evklidskih prostorov (za poljubno $n\in \mathbb{N}$).

Zelo uporabni so naslednji zgledi.

                \bzgled \label{vektSredOSOAOB}
                Let $O$, $A$ and $B$ be arbitrary points  and $S$ the midpoint of the line segment $AB$. Then
                $$\overrightarrow{OS}=\frac{1}{2}\cdot\left(\overrightarrow{OA}+
                \overrightarrow{OB}\right).$$
                \ezgled


\begin{figure}[!htb]
\centering
\input{sl.vek.5.2.4.pic}
\caption{} \label{sl.vek.5.2.4.pic}
\end{figure}

 \textbf{\textit{Proof.}}  (Figure \ref{sl.vek.5.2.4.pic})

  Po izreku \ref{vektSredDalj} je $\overrightarrow{SA}=-\overrightarrow{SB}$ oz. $\overrightarrow{AS}=-\overrightarrow{BS}$. Po trikotniškem pravilu  za seštevanje vektorjev je
  $\overrightarrow{OS}=\overrightarrow{OA}+\overrightarrow{AS}$ in
  $\overrightarrow{OS}=\overrightarrow{OB}+\overrightarrow{BS}$. Če seštejemo enakosti  in upoštevamo $\overrightarrow{AS}=-\overrightarrow{BS}$,  dobimo: $2\cdot\overrightarrow{OS}=\overrightarrow{OA}+\overrightarrow{AS}
  +\overrightarrow{OB}+\overrightarrow{BS}=\overrightarrow{OA}
  +\overrightarrow{OB}$
 oz.
$\overrightarrow{OS}=\frac{1}{2}\cdot\left(\overrightarrow{OA}+
                \overrightarrow{OB}\right)$.
\kdokaz

                \bzgled \label{vektDelitDaljice}
                Let $O$, $A$ and $B$ be arbitrary points  and  $P$ a point in the line segment $AB$ such that $|AP|:|PB|=n:m$. Then
                $$\overrightarrow{OP}=\frac{1}{n+m}\left(m\cdot\overrightarrow{OA}+
                n\cdot\overrightarrow{OB}\right).$$
                \ezgled


\begin{figure}[!htb]
\centering
\input{sl.vek.5.2.5.pic}
\caption{} \label{sl.vek.5.2.5.pic}
\end{figure}

 \textbf{\textit{Proof.}}  (Figure \ref{sl.vek.5.2.5.pic})

 Najprej iz $|AP|:|PB|=n:m$ sledi $|AP|=\frac{n}{n+m}\cdot|AB|$ in $|BP|=\frac{m}{n+m}\cdot|AB|$ oz. $\overrightarrow{AP}=\frac{n}{n+m}\cdot\overrightarrow{AB}$ in $\overrightarrow{BP}=\frac{m}{n+m}\cdot\overrightarrow{BA}$ (ker je $\mathcal{B}(A,P,B)$).
Nato je:
\begin{eqnarray*}
  \overrightarrow{OP}&=&\overrightarrow{OA}+\overrightarrow{AP}=\overrightarrow{OA}
  +\frac{n}{n+m}\cdot\overrightarrow{AB}\hspace*{3mm} \textrm{ in}\\
  \overrightarrow{OP}&=&\overrightarrow{OB}+\overrightarrow{BP}=\overrightarrow{OB}
  +\frac{m}{n+m}\cdot\overrightarrow{BA}.
\end{eqnarray*}
  Če prvo enakost množimo z $m$, drugo z $n$ in nato seštejemo,  dobimo:
\begin{eqnarray*}
(n+m)\cdot\overrightarrow{OP}&=&m\cdot\overrightarrow{OA}
  +n\cdot\overrightarrow{OB}+\frac{nm}{n+m}\cdot\left(\overrightarrow{AB}+
  \overrightarrow{BA}\right)=\\
  &=&m\cdot\overrightarrow{OA}
  +n\cdot\overrightarrow{OB}.
\end{eqnarray*}
 Če  dobljeno enakost delimo z $n+m$, dobimo iskano relacijo.
\kdokaz


                \bzgled \label{vektParamDaljica}
                Let $O$, $A$ and $B$ be arbitrary points. A point $X$ lies on the line segment $AB$ if and only if for some scalar $0\leq\lambda\leq1$ is
                $$\overrightarrow{OX}=(1-\lambda)\cdot\overrightarrow{OA}+
                \lambda\cdot\overrightarrow{OB}.$$
                \ezgled


 \textbf{\textit{Proof.}}  (Figure \ref{sl.vek.5.2.6.pic})

 Predpostavimo najprej, da točka $X$ leži na daljici $AB$. Potem za nek  $0\leq\lambda\leq1$ velja $\overrightarrow{AX}=\lambda\cdot \overrightarrow{AB}$.

 \begin{eqnarray*}
  \overrightarrow{OX}&=&\overrightarrow{OA}+\overrightarrow{AX}=\\
  &=& \overrightarrow{OA}+\lambda\cdot \overrightarrow{AB}=\\
  &=& \overrightarrow{OA}+\lambda\cdot \left(\overrightarrow{OB}-\overrightarrow{OA}\right)=\\
  &=& (1-\lambda)\cdot\overrightarrow{OA}+
                \lambda\cdot\overrightarrow{OB}.
\end{eqnarray*}

Predpostavimo sedaj, da za  nek $0\leq\lambda\leq1$ velja
                $\overrightarrow{OX}=(1-\lambda)\cdot\overrightarrow{OA}+
                \lambda\cdot\overrightarrow{OB}$. Potem je:

 \begin{eqnarray*}
  \overrightarrow{AX}&=&\overrightarrow{AO}+\overrightarrow{OX}=\\
  &=&\overrightarrow{AO}+(1-\lambda)\cdot\overrightarrow{OA}+
                \lambda\cdot\overrightarrow{OB}=\\
 &=&-\lambda\cdot\overrightarrow{OA}+
                \lambda\cdot\overrightarrow{OB}=\\
 &=&\lambda\cdot\left(\overrightarrow{AO}+
                \overrightarrow{OB}\right)=\\
 &=&\lambda\cdot\overrightarrow{AB}.
\end{eqnarray*}
Torej za nek  $0\leq\lambda\leq1$ velja $\overrightarrow{AX}=\lambda\cdot \overrightarrow{AB}$,
 kar pomeni, da točka $X$ leži na daljici $AB$.
\kdokaz

\begin{figure}[!htb]
\centering
\input{sl.vek.5.2.6.pic}
\caption{} \label{sl.vek.5.2.6.pic}
\end{figure}

A direct consequence is the following theorem.

  $O$, $A$ in $B$ . A point $X$ lies on the line segment $AB$ if and only if for some

                \bzgled
                Let $O$, $A$ and $B$ be arbitrary points. A point $X$ lies on the line segment $AB$ if and only if for some scalars $\alpha, \beta\in [0,1]$ ($\alpha+\beta=1$) is
                $$\overrightarrow{OX}=\alpha\cdot\overrightarrow{OA}+
                \beta\cdot\overrightarrow{OB}.$$
                \ezgled

V naslednjem izreku bomo podali \index{vektorska enačba premice} \pojem{enačbo premice v vektorski obliki}.

                \bzgled \label{vektParamPremica}
                Let $O$, $A$ in $B$ be arbitrary points. A point $X$ lies on the line $AB$ if and only if for some scalar $\lambda\in \mathbb{R}$ is
               $$\overrightarrow{OX}=(1-\lambda)\cdot\overrightarrow{OA}+
                \lambda\cdot\overrightarrow{OB}.$$
                \ezgled


 \textbf{\textit{Proof.}}  (Figure \ref{sl.vek.5.2.6.pic})

Dokaz je enak kot pri prejšnjem izreku, le da uporabimo dejstvo, da točka $X$ leži na premici $AB$ natanko tedaj, ko sta vektorja $\overrightarrow{AX}$ in $\overrightarrow{AB}$ kolinearna, oz. za nek
$\lambda\in \mathbb{R}$ velja $\overrightarrow{AX}=\lambda\cdot \overrightarrow{AB}$ (izrek \ref{vektKriterijKolin}).
\kdokaz

\index{količnik kolinearnih vektorjev}\index{razmerje!kolinearnih vektorjev}
Če za kolinearna vektorja $\overrightarrow{v}$ in $\overrightarrow{u}$ velja $\overrightarrow{v}=\lambda \overrightarrow{u}$ ($\lambda\in\mathbb{R}$) in $\overrightarrow{u}\neq \overrightarrow{0}$, potem lahko definiramo \pojem{količnik kolinearnih vektorjev} ali \pojem{razmerje kolinearnih vektorjev}  (Figure \ref{sl.vek.5.2.7.pic}):
$$\overrightarrow{v}:\overrightarrow{u}
=\frac{\overrightarrow{v}}{\overrightarrow{u}}=\lambda.$$


\begin{figure}[!htb]
\centering
\input{sl.vek.5.2.7.pic}
\caption{} \label{sl.vek.5.2.7.pic}
\end{figure}

Iz definicije množenja vektorja z realnim številom  dobimo naslednjo trditev.

                    \bizrek \label{vektKolicnDolz}
                    Let $\overrightarrow{v}$ and $\overrightarrow{u}$ be collinear vectors and $\overrightarrow{u}\neq \overrightarrow{0}$. Then
                    $$\frac{\overrightarrow{v}}{\overrightarrow{u}}=
                    \left\{ \begin{array}{lll}
                    \frac{|\overrightarrow{v}|}{|\overrightarrow{u}|},   &
                    \textrm{ if } \overrightarrow{u}\rightrightarrows \overrightarrow{v}\\
                     -\frac{|\overrightarrow{v}|}{|\overrightarrow{u}|},      &
                     \textrm{ if } \overrightarrow{u}\rightleftarrows\overrightarrow{v}\\
                     0,  & \textrm{ if } \overrightarrow{v}=\overrightarrow{0}
                    \end{array}\right.$$
                    \eizrek


Tudi naslednji izrek se nanaša na novo definirani pojem.

                     \bizrek
                     \label{izrekEnaDelitevDaljiceVekt}
                     For any line segment $AB$ and any $\lambda\in\mathbb{R}\setminus\{-1\}$ there exists
                     exactly one such point $P$ on the line $AB$, that is
                     $$\frac{\overrightarrow{AP}}{\overrightarrow{PB}}=\lambda.$$
                    \eizrek

\begin{figure}[!htb]
\centering
\input{sl.vek.5.2.8.pic}
\caption{} \label{sl.vek.5.2.8.pic}
\end{figure}

 \textbf{\textit{Proof.}}  (Figure \ref{sl.vek.5.2.8.pic})

 Naj bo $P$ takšna točka, da velja: $$\overrightarrow{AP}=\frac{\lambda}{1+\lambda}\cdot\overrightarrow{AB}.$$
Ker je $\lambda\neq-1$, takšna točka vedno obstaja. Ker je še:
$$\overrightarrow{PB}=\overrightarrow{PA}+\overrightarrow{AB}=
-\frac{\lambda}{1+\lambda}\cdot\overrightarrow{AB}+\overrightarrow{AB}=
\frac{1}{1+\lambda}\overrightarrow{AB},$$
velja tudi $\frac{\overrightarrow{AP}}{\overrightarrow{PB}}=\lambda$.

Če za neko drugo točko $P'$ velja $\frac{\overrightarrow{AP'}}{\overrightarrow{P'B}}=\lambda$, iz
$\overrightarrow{AP'}=\lambda\cdot\overrightarrow{P'B}$ in
$\overrightarrow{P'B}=\overrightarrow{P'A}+\overrightarrow{AB}$ z enostavnim računanjem dobimo $\overrightarrow{AP'}=\frac{\lambda}{1+\lambda}\cdot\overrightarrow{AB}$. Torej
 $\overrightarrow{AP'}=\overrightarrow{AP}$ oz.
 $\overrightarrow{P'P}=\overrightarrow{P'A}+\overrightarrow{AP}=
 -\overrightarrow{AP}+\overrightarrow{AP}=\overrightarrow{0}$. Zato je $P'=P$, kar pomeni, da obstaja ena sama točka $P$, za katero velja $\frac{\overrightarrow{AP}}{\overrightarrow{PB}}=\lambda$.
 \kdokaz

 Pravimo, da točka $P$ iz prejšnjega izreka deli daljico $AB$ v \pojem{razmerju} $\lambda$.



                    \bzgled
                    Let $ABCD$ be a trapezium with the base $AB$. Calculate the ratio in which the line
                    segment $PD$ divides the diagonal $AC$, if $|AB| = 3\cdot |CD|$ and $P$ is the midpoint of the line segment $AB$.
                    \ezgled

\begin{figure}[!htb]
\centering
\input{sl.vek.5.2.9.pic}
\caption{} \label{sl.vek.5.2.9.pic}
\end{figure}

 \textbf{\textit{Solution.}}  Presečišče daljice $DP$ in diagonale $AC$ označimo s $S$ (Figure \ref{sl.vek.5.2.9.pic})
in naj bo
 $\overrightarrow{u}=\overrightarrow{AP}$ in $\overrightarrow{v}=\overrightarrow{AD}$.
Vektorja $\overrightarrow{AS}$ in $\overrightarrow{AC}$ sta kolinearna, zato je po izreku \ref{vektKriterijKolin} $\overrightarrow{AS}=\lambda\overrightarrow{AC}$, za nek $\lambda\in \mathbb{R}$. Prav tako sta kolinearna tudi vektorja $\overrightarrow{PS}$ in $\overrightarrow{PD}$,  oz. za nek $\mu\in \mathbb{R}$ velja $\overrightarrow{PS}=\mu\overrightarrow{PD}$. Vektorja $\overrightarrow{AC}$ in $\overrightarrow{AS}$ zapišemo kot linearno kombinacijo vektorjev $\overrightarrow{u}$ in $\overrightarrow{v}$:
 \begin{eqnarray*}
\hspace*{-1.8mm} \overrightarrow{AC}&=&\overrightarrow{AD}+\overrightarrow{DC}
 =\overrightarrow{v}+\frac{2}{3}\overrightarrow{u}
 =\frac{2}{3}\overrightarrow{u}+\overrightarrow{v},\\
 \hspace*{-1.8mm}\overrightarrow{AS}&=&\overrightarrow{AP}+\overrightarrow{PS}
 =\overrightarrow{u}+\mu\overrightarrow{PD}
 =\overrightarrow{u}+\mu(-\overrightarrow{u}+\overrightarrow{v})
 =(1-\mu)\overrightarrow{u}+\mu\overrightarrow{v}.
 \end{eqnarray*}
Ker je še $\overrightarrow{AS}=\lambda\overrightarrow{AC}$, dobimo:
\begin{eqnarray*}
 \overrightarrow{AS}&=&
 \frac{2}{3}\lambda\overrightarrow{u}+\lambda\overrightarrow{v};\\
 \overrightarrow{AS}&=&
 (1-\mu)\overrightarrow{u}+\mu\overrightarrow{v}.
 \end{eqnarray*}
 Ker sta $\overrightarrow{u}$ in $\overrightarrow{v}$ kolinearna vektorja, je po izreku \ref{vektLinKomb1Razcep}  $\frac{2}{3}\lambda=1-\mu$ in $\lambda=\mu$. Če rešimo ta enostaven sistem enačb, dobimo $\lambda=\mu=\frac{3}{5}$. Torej je $\overrightarrow{AS}=\frac{3}{5}\overrightarrow{AC}$, oz. $AS:SC=3:2$.
 \kdokaz

%________________________________________________________________________________
\poglavje{Vector Length}  \label{odd5DolzVekt}

Dolžino vektorja smo definirali že v razdelku \ref{odd5DefVekt}. Iz definicije množenja vektorja s skalarjem v prejšnjem razdelku \ref{odd5LinKombVekt} smo videli, da za vsak vektor $\overrightarrow{v}$ in vsako realno število $\lambda$ velja $|\lambda \cdot \overrightarrow{v}|=|\lambda| \cdot |\overrightarrow{v}|$. V tem razdelku bomo obravnavali še nekatere lastnosti dolžine vektorja.

Najprej iz definicije seštevanja vektorjev in trikotniške neenakosti dobimo naslednjo trditev.

            \bizrek \label{neenakTrikVekt}
            For any two vectors $\overrightarrow{v}$ and $\overrightarrow{u}$ is
            $$|\overrightarrow{v}+\overrightarrow{u}|\leq |\overrightarrow{v}|+|\overrightarrow{u}|.$$
            \eizrek

\begin{figure}[!htb]
\centering
\input{sl.vek.5.1.7.pic}
\caption{} \label{sl.vek.5.1.7.pic}
\end{figure}

 \textbf{\textit{Proof.}}  (Figure \ref{sl.vek.5.1.7.pic})

  Naj bo $A$ poljubna točka ter $B$ in $C$ takšni točki, da velja $\overrightarrow{AB}=\overrightarrow{v}$ in $\overrightarrow{BC}=\overrightarrow{u}$ (izrek \ref{vektAvObst1TockaB}). Iz definicij seštevanja vektorjev in dolžine vektorja ter trikotniške neenakosti (izrek \ref{neenaktrik}) dobimo:
  \begin{eqnarray*}
  |\overrightarrow{v}+\overrightarrow{u}|&=&|\overrightarrow{AB}+\overrightarrow{BC}|=\\
  &=&|\overrightarrow{AC}|=|AC|\leq
  |AB|+|AC|=
  |\overrightarrow{AB}|+|\overrightarrow{AC}|=\\
  &=&|\overrightarrow{v}|+|\overrightarrow{u}|,
   \end{eqnarray*}
 kar je bilo treba dokazati.  \kdokaz

Na podoben način kot količnik lahko definiramo tudi produkt kolinearnih vektorjev.
Če sta $\overrightarrow{v}$ in $\overrightarrow{u}$ dva kolinearna vektorja, potem definiramo operacijo \pojem{množenja dveh kolinearnih vektorjev}\footnote{Gre za poseben primer t. i. \textit{skalarnega produkta}, ki se v linearni algebri definira za poljubna dva vektorja. V evklidskem prostoru je $\overrightarrow{v}\cdot\overrightarrow{u}=
|\overrightarrow{v}|\cdot|\overrightarrow{u}|\cdot\cos
\angle \overrightarrow{v},\overrightarrow{u}$.}:
 \begin{eqnarray*}
 \overrightarrow{v}\cdot \overrightarrow{u}=
\left\{
  \begin{array}{ll}
    |\overrightarrow{v}|\cdot|\overrightarrow{u}|, &
\overrightarrow{v},\overrightarrow{u}\rightrightarrows ; \\
    -|\overrightarrow{v}|\cdot|\overrightarrow{u}|, & \overrightarrow{v},\overrightarrow{u}\rightleftarrows.
  \end{array}
\right.
\end{eqnarray*}
 Rezultat te operacije, ki je realno število, imenujemo  \index{produkt kolinearnih vektorjev} \pojem{produkt kolinearnih vektorjev}.
Že iz definicije sledi:
     \begin{eqnarray} \label{eqnMnozVektDolzina}
     \overrightarrow{v}\cdot \overrightarrow{v}=|\overrightarrow{v}|^2
     \end{eqnarray}
 Jasno je tudi, da za tri kolinearne točke $A$, $B$ in $L$ velja ekvivalenca (Figure \ref{sl.vek.5.3.1.pic}):

 \begin{eqnarray} \label{eqnMnozVektRelacijaB}
     \overrightarrow{LA}\cdot \overrightarrow{LB}<0
\hspace*{1mm}\Leftrightarrow\hspace*{1mm} \mathcal{B}(A,L,B)
\hspace*{3mm} \textrm{(}A,B,L \textrm{ so kolinearne)}.
     \end{eqnarray}


\begin{figure}[!htb]
\centering
\input{sl.vek.5.3.1.pic}
\caption{} \label{sl.vek.5.3.1.pic}
\end{figure}




        \bnaloga\footnote{15. IMO USSR - 1973, Problem 1.}
        Point $O$ lies on line $g$;
         $\overrightarrow{OP_1},\overrightarrow{OP_2},
         \cdots,\overrightarrow{OP_n}$
        are unit vectors such that points $P_1, P_2, \cdots, P_n$
        all lie in a plane containing $g$ and on one side of $g$. Prove that
        if $n$ is odd,
         $$|\overrightarrow{OP_1}+\overrightarrow{OP_2}+
         \cdots+\overrightarrow{OP_n}|\geq 1.$$ \label{OlimpVekt15}
         \enaloga

\begin{figure}[!htb]
\centering
\input{sl.vek.5.3.IMO1.pic}
\caption{} \label{sl.vek.5.3.IMO1.pic}
\end{figure}

 \textbf{\textit{Solution.}} Naj bo $k$ enotska krožnica s središčem
 v točki $O$, ki seka premico $g$ v točkah $A$ in $B$. Vse
 točke $P_1, P_2, \cdots, P_n$ ležijo na pripadajoči
 polkrožnici, ki jo na krožnici $k$ določata točki $A$ in $B$.
 Brez škode za splošnost lahko
 predpostavimo, da so vektorji označeni tako, da velja:
 $\angle BOP_1\leq \angle BOP_2\leq\cdots\leq \angle BOP_n$
 (Figure \ref{sl.vek.5.3.IMO1.pic}).
 Naj bo $s$ simetrala kota $\angle P_1OP_n$ in $g'$
 pravokotnica premice $s$ v točki $O$. Ker je $\angle P_1OP_n$ konveksni
 kot, cel leži na istem bregu premice $g'$. Potem so tudi  vse
 točke $P_i$ ($i\in \{1,2,\cdots,n\}$ na istem bregu premice $g'$,
 ker ležijo v tem kotu. Tako lahko našo predpostavko iz naloge, da so
 vse točke $P_1, P_2, \cdots, P_n$ na istem bregu neke premice
 $g$ oz. $g'$,
 zaostrimo z zahtevo (ki sledi iz te predpostavke),
  da sta prvi in zadnji vektor zaporedja
 simetrična glede na pravokotnico $s$  premice $g'$ skozi točko $O$. To
 dejstvo bomo uporabljali v dokazu.

 Po predpostavki je $n$ liho število. Naj bo $n=2k-1$, $k\in \mathbb{N}$.
 Dokaz bomo izpeljali z indukcijo po $k$.

 (\textit{i}) Če je $k=1$ oz. $n=1$, je že $|\overrightarrow{OP_1}|=1$
 oz. velja tudi $|\overrightarrow{OP_1}|\geq1$ in je trditev
 izpolnjena.

 (\textit{ii}) Predpostavimo, da trditev velja za $k=l$ oz. za vsako zaporedje
 $n=2l-1$ vektorjev. Dokažimo, da potem trditev velja tudi za $k=l+1$ oz. v
 primeru zaporedja $n=2l+1$ vektorjev. Naj bodo
 $\overrightarrow{OP_1},\overrightarrow{OP_2},
 \cdots,\overrightarrow{OP_{2l+1}}$
  enotski vektorji, kjer so $P_1, P_2, \cdots, P_{2l+1}$
 točke, ki
 ležijo v isti polravnini z robom  $g'$ (v polravnini $\alpha$)
  in sta vektorja $\overrightarrow{OP_1}$
 in $\overrightarrow{OP_{2l+1}}$ simetrična glede
  na premico $s$. Iz tega sledi, da vektor
  $\overrightarrow{OU}=\overrightarrow{OP_1}+\overrightarrow{OP_{2l+1}}$
  leži na premici $s$ oz. $U\in s$. Naj bo
  $\overrightarrow{OV}=\overrightarrow{OP_2}+\overrightarrow{OP_3}+
         \cdots+\overrightarrow{OP_{2l}}$. Po indukcijski predpostavki
  je $|\overrightarrow{OV}|=|\overrightarrow{OP_2}+\overrightarrow{OP_3}+
         \cdots+\overrightarrow{OP_{2l}}|\geq 1$. Torej:
 \begin{eqnarray*}
  |\overrightarrow{OP_1}+\overbrace{\overrightarrow{OP_2}+
         \cdots+\overrightarrow{OP_{2l}}}+\overrightarrow{OP_{2l+1}}|=
         |\overrightarrow{OU}+\overrightarrow{OV}|.
 \end{eqnarray*}
 Naj bo
 $\overrightarrow{OW}=\overrightarrow{OU}+\overrightarrow{OV}$.  Štirikotnik $VOUW$ je paralelogram (ali pa so točke $O$, $U$, $V$ in $W$
 kolinearne).
 Po zgledu \ref{vektOPi} sta točki $U$ in $V$ v polravnini $\alpha$. Točka
 $U$ leži na premici $s$, ki je simetrala iztegnjenega kota, ki
 ga določata premica $g'$ in vrh $O$. Zato je $0\leq\angle
 UOV\leq 90^0$. Iz paralelograma $VOUW$ je potem $\angle OVW >
 90^0$. Iz trikotniške neenakosti \ref{neenaktrik} (za trikotnik
 $OVW$) sledi $|\overrightarrow{OW}|>|\overrightarrow{OV}|$.
 Neenakost velja tudi v primeru, kadar so točke $O$, $U$, $V$ in $W$
 kolinearne. Sedaj imamo:
 \begin{eqnarray*}
  |\overrightarrow{OP_1}+\overrightarrow{OP_2}+
         \cdots+\overrightarrow{OP_{2l}}+\overrightarrow{OP_{2l+1}}|=
         |\overrightarrow{OU}+\overrightarrow{OV}|=
         |\overrightarrow{OW}|>|\overrightarrow{OV}|\geq 1,
 \end{eqnarray*}
 kar je bilo treba dokazati. \kdokaz

%________________________________________________________________________________
 \poglavje{Further Use of Vectors}  \label{odd5UporabVekt}

 Najprej si oglejmo trditev, ki je
 direktna posledica definicije vektorjev (razdelek \ref{odd5DefVekt}).

              \bizrek \label{vektParalelogram}
              A quadrilateral $ABCD$ is a parallelogram if and only if
               $\overrightarrow{AB}=\overrightarrow{DC}$.
               \eizrek


 Omenimo tudi, da lahko v terminih vektorjev že
dokazano lastnost srednjice trikotnika - izrek \ref{srednjicaTrik} -
sedaj izrazimo v naslednji obliki.

                \bizrek
                \label{srednjicaTrikVekt}
                 Let $PQ$ be the midsegment of a triangle  $ABC$, corresponding
                to the side
                $BC$. Then
                $$ \overrightarrow{PQ} = \frac{1}{2} \overrightarrow{AB}.$$
                 \eizrek

\begin{figure}[!htb]
\centering
\input{sl.vek.5.4.1c.pic}
\caption{} \label{sl.vek.5.4.1c.pic}
\end{figure}

 \textbf{\textit{Proof.}}  (Figure \ref{sl.vek.5.4.1c.pic})
 Trditev je direktna posledica  izreka \ref{srednjicaTrik}
 \kdokaz

Videli smo, da lahko vektorja seštevamo po paralelogramskem pravilu
(kadar imata skupni začetek) ali po trikotniškem pravilu (kadar je
začetek drugega na koncu prvega). Sedaj bomo vpeljali še eno
pravilo za seštevanje dveh vektorjev, ki sta v poljubni legi. Gre za t. i. \index{pravilo!splošno za seštevanje vektorjev}\pojem{splošno pravilo za seštevanje vektorjev}.

              \bizrek \label{vektSestSplosno}
              If $S_1$ and $S_2$
               are  the midpoints of line segments $A_1B_1$ and $A_2B_2$, then
              $$\overrightarrow{A_1A_2}+\overrightarrow{B_1B_2}=
              2\overrightarrow{S_1S_2}.$$
              \eizrek

\begin{figure}[!htb]
\centering
\input{sl.vek.5.4.1b.pic}
\caption{} \label{sl.vek.5.4.1b.pic}
\end{figure}

 \textbf{\textit{Proof.}}  (Figure \ref{sl.vek.5.4.1b.pic})

Ker sta $S_1$ in $S_2$
    središči daljic $A_1B_1$ in $A_2B_2$, je
    $\overrightarrow{S_1A_1}=-\overrightarrow{S_1B_1}$ in $\overrightarrow{S_2A_2}=-\overrightarrow{S_2B_2}$ oz. $\overrightarrow{S_1A_1}+\overrightarrow{S_1B_1}=\overrightarrow{0}$ in $\overrightarrow{S_2A_2}+\overrightarrow{S_2B_2}=\overrightarrow{0}$.

    Če vektor $\overrightarrow{S_1S_2}$ razstavimo na dva načina po poligonskem pravilu za seštevanje vektorjev, dobimo najprej:
    \begin{eqnarray*}
     \overrightarrow{S_1S_2}&=&\overrightarrow{S_1A_1}+\overrightarrow{A_1A_2}+
     \overrightarrow{A_2S_2}\\
     \overrightarrow{S_1S_2}&=&\overrightarrow{S_1B_1}+\overrightarrow{B_1B_2}+
     \overrightarrow{B_2S_2},
    \end{eqnarray*}
    nato pa s seštevanjem teh še:
  \begin{eqnarray*}
  2\cdot\overrightarrow{S_1S_2}&=&\overrightarrow{S_1A_1}+\overrightarrow{A_1A_2}+
     \overrightarrow{A_2S_2}+\\
     &+&\overrightarrow{S_1B_1}+\overrightarrow{B_1B_2}+
     \overrightarrow{B_2S_2}=\\
     &=& \overrightarrow{A_1A_2}+\overrightarrow{B_1B_2},
    \end{eqnarray*}
  kar je bilo treba dokazati. \kdokaz

Pogosto bomo relacijo iz prejšnjega izreka uporabljali tudi v
obliki:
 $$\overrightarrow{S_1S_2}=
 \frac{1}{2}(\overrightarrow{A_1A_2}+\overrightarrow{B_1B_2}),$$
    kar je posplošitev lastnosti srednjice trapeza in trikotnika.



    Na tem mestu bomo še enkrat ponovili vsa tri pravila za seštevanje dveh vektorjev. V prvem primeru torej predstavnika vektorjev izberemo tako, da je začetek drugega na koncu prvega, v drugem primeru imata vektorja skupno začetno točko, v tretjem sta predstavnika v splošni legi (Figure \ref{sl.vek.5.4.1a.pic}):


  \begin{itemize}
  \item \textit{za vsake tri točke $A$, $B$ in $C$ velja $\overrightarrow{AB}+\overrightarrow{BC}=\overrightarrow{AC}$ (trikotniško pravilo),}
    \item \textit{za vsake tri nekolinearne točke $A$, $B$ in $C$ velja $\overrightarrow{AB}+\overrightarrow{AC}=\overrightarrow{AD}$ natanko tedaj, ko je $ABDC$ paralelogram (paralelogramsko pravilo),}
    \item \textit{če sta $S_1$ in $S_2$
    središči daljic $A_1B_1$ in $A_2B_2$, je
    $\overrightarrow{A_1A_2}+\overrightarrow{B_1B_2}=
    2\overrightarrow{S_1S_2}$ (splošno pravilo).}
  \end{itemize}


\begin{figure}[!htb]
\centering
\input{sl.vek.5.4.1a.pic}
\caption{} \label{sl.vek.5.4.1a.pic}
\end{figure}


 Razen teh treh imamo še poligonsko pravilo, ki se nanaša na seštevanje več vektorjev (Figure \ref{sl.vek.5.1.12.pic}):
  \begin{itemize}
    \item \textit{$\overrightarrow{A_1A_2}+\overrightarrow{A_2A_3}+\cdots +\overrightarrow{A_{n-1}A_n}=\overrightarrow{A_1A_n}$ (poligonsko pravilo).}
  \end{itemize}


\begin{figure}[!htb]
\centering
\input{sl.vek.5.1.12.pic}
\caption{} \label{sl.vek.5.1.12.pic}
\end{figure}


                    \bzgled \label{vektPetkoinikZgled}
                    Points $M$, $N$, $P$ and $Q$ are the midpoints of the sides $AB$, $BC$, $CD$ and $DE$ of a pentagon $ABCDE$. Prove
                    that the line segment $XY$,  determined by the midpoints of the line segments $MP$ and $NQ$, is parallel to the line $AE$
                    and calculate $\frac{\overrightarrow{XY}}{\overrightarrow{AE}}$.
                    \ezgled


\begin{figure}[!htb]
\centering
\input{sl.vek.5.4.2.pic}
\caption{} \label{sl.vek.5.4.2.pic}
\end{figure}

 \textbf{\textit{Solution.}} (Figure \ref{sl.vek.5.4.2.pic})
 Če uporabimo izreka \ref{vektSestSplosno} in \ref{srednjicaTrikVekt} dobimo:
 \begin{eqnarray*}
 \overrightarrow{XY}&=&\frac{1}{2}\left(\overrightarrow{MQ}+\overrightarrow{PN} \right)=\\
 &=&\frac{1}{2}\left(\frac{1}{2}\left(\overrightarrow{AE}+\overrightarrow{BD}\right)
 +\frac{1}{2}\overrightarrow{DB} \right)=\\
 &=&\frac{1}{4}\overrightarrow{AE}
 \end{eqnarray*}
 Torej sta vektorja $\overrightarrow{XY}$ in $\overrightarrow{AE}$  kolinearna in $\frac{\overrightarrow{XY}}{\overrightarrow{AE}}=\frac{1}{4}$.
 \kdokaz


                    \bzgled
                    Let $O$ be an arbitrary point in the plane of a triangle $ABC$ and
                    $D$ and $E$ points of the sides $AB$ and $BC$ such that
                     $$\frac{\overrightarrow{AD}}{\overrightarrow{DB}}=
                    \frac{\overrightarrow{BE}}{\overrightarrow{EC}}=\frac{m}{n}.$$
                    Let $F$ be the intersection of the line segments $AE$ and $CD$. Express $\overrightarrow{OF}$ as
                    a function of  $\overrightarrow{OA}$, $\overrightarrow{OB}$, $\overrightarrow{OC}$, $m$ and $n$.
                    \ezgled

\begin{figure}[!htb]
\centering
\input{sl.vek.5.4.3.pic}
\caption{} \label{sl.vek.5.4.3.pic}
\end{figure}

 \textbf{\textit{Solution.}} (Figure \ref{sl.vek.5.4.3.pic})

Dovolj je izraziti vektor $\overrightarrow{BF}$ kot linearno kombinacijo vektorjev $\overrightarrow{BA}$ in $\overrightarrow{BC}$, kajti:
\begin{eqnarray*}
 \overrightarrow{OF}&=&\overrightarrow{OB}+\overrightarrow{BF}\\
 \overrightarrow{BA}&=&\overrightarrow{OA}-\overrightarrow{OB}\\
 \overrightarrow{BC}&=&\overrightarrow{OC}-\overrightarrow{OB}.
 \end{eqnarray*}
Po \ref{vektParamPremica} je:
\begin{eqnarray*}
 \overrightarrow{BF}&=&\lambda\overrightarrow{BD}+(1-\lambda)\overrightarrow{BC}=
 \lambda\frac{n}{n+m}\overrightarrow{BA}+(1-\lambda)\overrightarrow{BC};\\
 \overrightarrow{BF}&=&\mu\overrightarrow{BA}+(1-\mu)\overrightarrow{BE}=
 \mu\overrightarrow{BA}+(1-\mu)\frac{m}{n+m}\overrightarrow{BC}
 \end{eqnarray*}
za neki števili $\lambda,\mu\in \mathbb{R}$. Ker sta vektorja $\overrightarrow{BA}$ in $\overrightarrow{BC}$ nekolinearna, po izreku \ref{vektLinKomb1Razcep} dobimo sistem:
\begin{eqnarray*}
 & & \lambda\frac{n}{n+m}=\mu\\
 & & 1-\lambda=(1-\mu)\frac{m}{n+m},
 \end{eqnarray*}
 ki ga rešimo po $\lambda$ in $\mu$ v funkciji $m$ in $n$. Dovolj je, če izračunamo le $\lambda$ in uvrstimo v  $\overrightarrow{BF}=
 \lambda\frac{n}{n+m}\overrightarrow{BA}+(1-\lambda)\overrightarrow{BC}$.
 \kdokaz


                    \bzgled
                    Let $A_1$, $A_2$, ..., $A_n$ be points on a line $p$ and $B_1$, $B_2$, ..., $B_n$  points on a line $q$ ($n\geq 3$), such that
                    $$\overrightarrow{A_1A_2}:
                    \overrightarrow{A_2A_3}:\cdots:
                    \overrightarrow{A_{n-1}A_n}=
                    \overrightarrow{B_1B_2}:
                    \overrightarrow{B_2B_3}:\cdots:
                    \overrightarrow{B_{n-1}B_n}.$$
                    Prove that the midpoints of line segments
                    $A_1B_1$, $A_2B_2$, ..., $A_nB_n$ lie on one line.
                    \ezgled


\begin{figure}[!htb]
\centering
\input{sl.vek.5.4.4.pic}
\caption{} \label{sl.vek.5.4.4.pic}
\end{figure}


 \textbf{\textit{Proof.}} (Figure \ref{sl.vek.5.4.4.pic})

 Dovolj je dokazati, da so tri poljubna središča po vrsti kolinearne točke.
Brez škode za splošnost dokažimo le, da so točke $S_1$, $S_2$, $S_3$ kolinearne. Naj bo:
$$\frac{\overrightarrow{A_1A_2}}{\overrightarrow{A_2A_3}}=
\frac{\overrightarrow{B_1B_2}}{\overrightarrow{B_2B_3}}=\lambda.$$
 Po izreku  \ref{vektSestSplosno} je:
  \begin{eqnarray*}
 \overrightarrow{S_1S_2}&=&\frac{1}{2}\left(\overrightarrow{A_1A_2}+
 \overrightarrow{B_1B_2}\right)=\\
 &=&\frac{1}{2}\left(\lambda\overrightarrow{A_2A_3}+
 \lambda\overrightarrow{B_2B_3}\right)=\\
 &=&\frac{\lambda}{2}\overrightarrow{S_2S_3},
 \end{eqnarray*}
 kar pomeni, da sta $\overrightarrow{S_1S_2}$ in $\overrightarrow{S_2S_3}$ kolinearna vektorja (izrek \ref{vektKriterijKolin}), zato so kolinearne tudi točke $S_1$, $S_2$ in $S_3$.
 \kdokaz



%________________________________________________________________________________
 \poglavje{Centroid of a Polygon With Respect to Its Vertices}  \label{odd5TezVeck}

Sedaj bomo posplošili pojem težišča trikotnika na poljubne večkotnike. V razdelku \ref{odd3ZnamTock} smo definirali pojem težišča trikotnika. Naslednja trditev se nanaša na eno dodatno lastnost tega pojma, ki je povezana s pojmom vektorja.


            \bizrek \label{tezTrikVekt}
            If $T$ is the centroid of a triangle $ABC$, then
            $$\overrightarrow{TA}+\overrightarrow{TB}+
            \overrightarrow{TC}=\overrightarrow{0}.$$
            \eizrek


\begin{figure}[!htb]
\centering
\input{sl.vek.5.5.1.pic}
\caption{} \label{sl.vek.5.5.1.pic}
\end{figure}

 \textbf{\textit{Proof.}} Označimo z $A_1$ središče stranice $BC$ (Figure \ref{sl.vek.5.5.1.pic}). Po izreku \ref{vektSredOSOAOB} je $\overrightarrow{TA_1}=
 \frac{1}{2}\left(\overrightarrow{TB}+\overrightarrow{TC}\right)$, po izreku \ref{tezisce} pa za težišče $T$ trikotnika $ABC$ velja  $|AT|:|TA_1|=2:1$ oz. $\overrightarrow{TA}=-2\cdot\overrightarrow{TA_1}$. Torej:
$$\overrightarrow{TA}+\overrightarrow{TB}+
            \overrightarrow{TC}=\overrightarrow{TA}+2\cdot\overrightarrow{TA_1}=
            \overrightarrow{TA}-\overrightarrow{TA}=
            \overrightarrow{0},$$ kar je bilo treba dokazati. \kdokaz


Velja tudi obratna trditev:



            \bizrek \label{tezTrikVektObr}
            Let $A$, $B$ and $C$ be three non-collinear points. If $X$ is a point such that
            $$\overrightarrow{XA}+\overrightarrow{XB}+
            \overrightarrow{XC}=\overrightarrow{0},$$
            then
            $X$ is the centroid of the triangle $ABC$.
            \eizrek

\textbf{\textit{Proof.}} Naj bo $T$ težišče trikotnika. Po
prejšnjem izreku \ref{tezTrikVekt} je
$\overrightarrow{TA}+\overrightarrow{TB}+
\overrightarrow{TC}=\overrightarrow{0}$. Po predpostavki
je tudi $\overrightarrow{XA}+\overrightarrow{XB}+
\overrightarrow{XC}=\overrightarrow{0}$. Če odštejemo
dve enakosti, dobimo
$3 \cdot \overrightarrow{TX}=\overrightarrow{0}$ oz. $X=T$.
 \kdokaz

Dokazana lastnost težišča trikotnika nam da idejo za definicijo težišča poljubnega večkotnika (Figure \ref{sl.vek.5.5.2.pic}).

\begin{figure}[!htb]
\centering
\input{sl.vek.5.5.2.pic}
\caption{} \label{sl.vek.5.5.2.pic}
\end{figure}

Točka $T$ je \index{težišče!večkotnika}\pojem{težišče večkotnika $A_1A_2\ldots A_n$ glede na njegova oglišča}, če velja:
$$\overrightarrow{TA_1}+\overrightarrow{TA_2}+\cdots +\overrightarrow{TA_n}=\overrightarrow{0}.$$
Prejšnjo relacijo lahko zapišemo tudi v obliki:
$$\sum_{k=1}^n\overrightarrow{TA_k}=\overrightarrow{0}.$$

Omenimo še, da težišče poljubnega lika $\Phi$ v ravnini definiramo kot točko $T$, za katero velja:
$$\sum_{X\in \Phi}\overrightarrow{TX}=\overrightarrow{0}.$$

U primeru našega večkotnika $A_1A_2\ldots A_n$ smo pravzaprav dobili težišče lika, ki predstavlja unijo vseh oglišč tega večkotnika $\{A_1,A_2,\ldots, A_n\}$. Zato smo poudarili, da gre za težišče večkotnika glede na njegova oglišča. Razen tega bi lahko govorili o \pojem{težišču večkotnika glede na vse njegove točke}, kar bi bolj ustrezalo splošni definiciji težišča poljubnega lika.

Če obravnavamo težišče lika v fizikalnem smislu kot - središče mase - prva varianta težišča predstavlja središče mase, pri čemer je vsa masa v ogliščih in ima vsako oglišče enako maso. V drugem primeru pa gre za središče mase večkotnika, kjer je masa homogeno razporejena po njegovi celotni notranjosti\footnote{\index{Arhimed}
        \textit{Arhimed
       iz Sirakuze} (3. st. pr. n. š.) starogrški matematik
      je prvi ustvaril koncept težišča,
ki ga je uporabljal v mnogih svojih zapisih o mehaniki, ampak
lahko le ugibamo, kaj točno je imel v mislih, ko je
obravnaval težišče, ker nobeden od njegovih ohranjenih zapisov ne vsebuje
jasne opredelitve pojma. Težišče kot središče mase v fizikalnem smislu je igralo pomembno vlogo v Newtonovi (\index{Newton,
I.}\textit{I. Newton} (1643-1727), angleški fizik in matematik) mehaniki, kjer večja telesa pogosto obravnavamo kot točke z določeno maso.}.

Če govorimo o splošnem večkotniku, sta omenjeni  težišči  enaki le pri poljubnem trikotniku, že pri poljubnem štirikotniku se težišči razlikujeta.
V nadaljevanju bomo obravnavali le težišče večkotnika glede na njegova oglišča, zato ga bomo  imenovali kar \pojem{težišče večkotnika}.

Najprej bomo obravnavali  težišče štirikotnika.

                \bizrek
                The centroid of a parallelogram $ABCD$ is its circumcentre $S$, i.e.
                the intersection of its diagonals.
                \eizrek

\begin{figure}[!htb]
\centering
\input{sl.vek.5.5.3.pic}
\caption{} \label{sl.vek.5.5.3.pic}
\end{figure}

 \textbf{\textit{Proof.}} (Figure \ref{sl.vek.5.5.3.pic})

 Po izreku \ref{paralelogram} je točka $S$ skupno središče njegovih diagonal $AC$ in $BD$, zato iz izreka \ref{vektSredDalj} sledi $\overrightarrow{SA}+\overrightarrow{SC}=\overrightarrow{0}$ in $\overrightarrow{SB}+\overrightarrow{SD}=\overrightarrow{0}$.
 Torej velja:
 $$\overrightarrow{SA}+\overrightarrow{SB}+
 \overrightarrow{SC}+\overrightarrow{SD}=\overrightarrow{0},$$
kar pomeni, da je $S$ težišče paralelograma $ABCD$.
\kdokaz


                \bizrek \label{vektVarignon}
                The centroid of a quadrilateral $ABCD$ is the centroid  of its Varignon
                parallelogram (see theorem \ref{Varignon}).
                \eizrek


\begin{figure}[!htb]
\centering
\input{sl.vek.5.5.1a.pic}
\caption{} \label{sl.vek.5.5.1a.pic}
\end{figure}

 \textbf{\textit{Proof.}}  (Figure \ref{sl.vek.5.5.1a.pic})

Naj bodo $P$, $K$, $Q$ in $L$ središča stranic $AB$, $BC$, $CD$ in $DA$ oz. $PKQL$ Varignonov paralelogram štirikotnika $ABCD$ (štirikotnik $PKQL$ je paralelogram po izreku \ref{Varignon}). Po prejšnjem izreku \ref{vektVarignon} je presečišče diagonal $PQ$ in $LK$ (označimo ga s $T$) hkrati težišče paralelograma $PKQL$. Potem je (izreka \ref{vektSredOSOAOB}) in \ref{paralelogram}):
 \begin{eqnarray*}
 \overrightarrow{TA}+\overrightarrow{TB}+
 \overrightarrow{TC}+\overrightarrow{TD}&=&
 2\cdot\overrightarrow{TP}+
 2\cdot\overrightarrow{TQ}=\\
 &=&
 2\cdot\left(\overrightarrow{TP}+
\overrightarrow{TQ}\right)=\\
 &=&\overrightarrow{0},
 \end{eqnarray*}
kar pomeni, da je $T$ težišče štirikotnika $ABCD$.
\kdokaz



                \bizrek \label{vektTezVeckXT}
                If $X$  is an arbitrary point and $T$ the centroid  of a polygon $A_1A_2\ldots A_n$, then
                $$\overrightarrow{XT}=\frac{1}{n}\left(\overrightarrow{XA_1}
                +\overrightarrow{XA_2}+\cdots +\overrightarrow{XA_n}\right).$$
                \eizrek


\begin{figure}[!htb]
\centering
\input{sl.vek.5.5.4.pic}
\caption{} \label{sl.vek.5.5.4.pic}
\end{figure}

 \textbf{\textit{Proof.}}  (Figure \ref{sl.vek.5.5.4.pic})

Najprej je $\overrightarrow{XT}=\overrightarrow{XA_k}+\overrightarrow{A_kT}$ (za vsak $k\in\{1,2,\ldots,n\}$). Če seštejemo vseh $n$ relacij, dobimo:

\begin{eqnarray*}
 n\cdot\overrightarrow{XT}
 &=&\sum_{k=1}^{n}\left(\overrightarrow{XA_k}+\overrightarrow{A_kT}\right)=\\
 &=&\sum_{k=1}^{n}\overrightarrow{XA_k}+\sum_{k=1}^{n}\overrightarrow{A_kT}=\\
 &=&
 \sum_{k=1}^{n}\overrightarrow{XA_k}-\sum_{k=1}^{n}\overrightarrow{TA_k}=\\
 &=&
 \sum_{k=1}^{n}\overrightarrow{XA_k}-\overrightarrow{0}=\\
 &=&\sum_{k=1}^{n}\overrightarrow{XA_k}.
 \end{eqnarray*}

Zato je:
$$\overrightarrow{XT}=\frac{1}{n}\cdot\sum_{k=1}^{n}\overrightarrow{XA_k},$$ kar je bilo treba dokazati. \kdokaz


Kot posledica prejšnjega izreka je posebej koristna relacija, ki velja za težišče poljubnega trikotnika.



                \bizrek \label{vektTezTrikXT}
                 If $X$  is an arbitrary point and $T$ the centroid  of a triangle $ABC$, then
                $$\overrightarrow{XT}=\frac{1}{3}\left(\overrightarrow{XA}
                +\overrightarrow{XB}+\overrightarrow{XC}\right).$$
                \eizrek


\begin{figure}[!htb]
\centering
\input{sl.vek.5.5.5.pic}
\caption{} \label{sl.vek.5.5.5.pic}
\end{figure}

 \textbf{\textit{Proof.}}  (Figure \ref{sl.vek.5.5.5.pic})

Direktna posledica prejšnjega izreka za $n=3$
\kdokaz



% Postopek za n-kotnik (n-1)-kotnik ...

 Zaenkrat smo imeli efektiven postopek za določanje težišča trikotnikov in štirikotnikov. Za poljubni štirikotnik pravzaprav še ne vemo, ali težišče sploh obstaja. To vprašanje bomo obravnavali v nadaljevanju.

 Idejo bomo iskali v naslednjih dejstvih. Če daljico $AB$ obravnavamo kot degenerirani $2$-kotnik, je njegovo težišče središče daljica $AB$; označimo ga s $T_2$. Za težišče $T_3$ trikotnika $ABC$ potem velja
 $\overrightarrow{CT_3}=\frac{2}{3}\cdot \overrightarrow{CT_2}$ (izrek \ref{tezisce}). To idejo bomo posplošili v naslednjem izreku.


                \bizrek \label{vektTezVeck}
                Every polygon $A_1A_2\ldots A_n$ has exactly one centroid. If
                $T_{n-1}$ is the centroid of the polygon $A_1A_2\ldots A_{n-1}$ and $T_n$ a point such that $$\overrightarrow{A_nT_n}=\frac{n-1}{n}\cdot\overrightarrow{A_nT_{n-1}},$$
                then $T_n$ is the centroid of the polygon $A_1A_2\ldots A_n$.
                \eizrek



\begin{figure}[!htb]
\centering
\input{sl.vek.5.5.2a.pic}
\caption{} \label{sl.vek.5.5.2a.pic}
\end{figure}

 \textbf{\textit{Proof.}}  (Figure \ref{sl.vek.5.5.2a.pic})

 Najprej iz relacije $\overrightarrow{A_nT_n}=\frac{n-1}{n}\cdot\overrightarrow{A_nT_{n-1}}$ sledi
 $\overrightarrow{T_nA_n}=-\frac{n-1}{n}\cdot\overrightarrow{A_nT_{n-1}}$ in $\overrightarrow{T_nT_{n-1}}=\frac{1}{n}\cdot\overrightarrow{A_nT_{n-1}}$.
  Ker je $T_{n-1}$ težišče večkotnika $A_1A_2\ldots A_{n-1}$, po izreku \ref{vektTezVeckXT} velja $\overrightarrow{T_nA_1}+\overrightarrow{T_nA_2}+\cdots +\overrightarrow{T_nA_{n-1}}=\left(n-1\right)\cdot \overrightarrow{T_nT_{n-1}}$.

 Torej:
 \begin{eqnarray*}
 & & \overrightarrow{T_nA_1}+\overrightarrow{T_nA_2}+\cdots
 +\overrightarrow{T_nA_{n-1}}+\overrightarrow{T_nA_n}=\\
 &=&\left(n-1\right)\cdot \overrightarrow{T_nT_{n-1}} +\overrightarrow{T_nA_n}=\\
 &=&\frac{n-1}{n}\cdot\overrightarrow{A_nT_{n-1}} -\frac{n-1}{n}\cdot\overrightarrow{A_nT_{n-1}}=\\
 &=&\overrightarrow{0},
 \end{eqnarray*}

 kar pomeni, da je $T_n$ težišče večkotnika $A_1A_2\ldots A_n$.

 Predpostavimo, da ima večkotnik $A_1A_2\ldots A_n$ še eno težišče $T'$. Toda v tem primeru je (izrek \ref{vektTezVeckXT}):
 \begin{eqnarray*}
 \overrightarrow{0}=
 \overrightarrow{T'A_1}+\overrightarrow{T'A_2}+\cdots
 \overrightarrow{T'A_n}=
 n\cdot\overrightarrow{T'T_n}.
 \end{eqnarray*}
 Torej $\overrightarrow{T'T_n}=\overrightarrow{0}$ oz. $T'=T_n$, kar pomeni, da ima večkotnik $A_1A_2\ldots A_n$ eno samo težišče.
 \kdokaz

 Prejšnji izrek \ref{vektTezVeck} nam omogoča efektivno konstrukcijo težišča večkotnika $A_1A_2\ldots A_n$, tako da po vrsti najprej konstruiramo težišča večkotnikov $A_1A_2$, $A_1A_2A_3$,$A_1A_2A_3A_4$, ... in na koncu $A_1A_2\ldots A_n$ (Figure \ref{sl.vek.5.5.2a.pic}):
 \begin{itemize}
   \item točka $T_2$ je središče daljice $A_1A_2$,
   \item $T_3$ je takšna točka, da velja: $\overrightarrow{A_3T_3}=\frac{2}{3}\cdot \overrightarrow{A_3T_2}$,
   \item $T_4$ je takšna točka, da velja: $\overrightarrow{A_4T_4}=\frac{3}{4}\cdot \overrightarrow{A_4T_3}$,\\
    $\vdots$
   \item $T_n$ je takšna točka, da velja $\overrightarrow{A_nT_n}=\frac{n-1}{n}\cdot \overrightarrow{A_nT_{n-1}}$.
 \end{itemize}


 Še lažji postopek za določanje težišča večkotnika dobimo, če uporabimo relacijo
 $$\overrightarrow{XT}=\frac{1}{n}\left(\overrightarrow{XA_1}
                +\overrightarrow{XA_2}+\cdots +\overrightarrow{XA_n}\right)$$
     iz izreka \ref{vektTezVeckXT}. Torej za poljubno točko $X$ enostavno načrtamo vektor $\overrightarrow{XT}$ in dobimo točko $T$.

     Poudarimo še, da v obeh primerih konstrukcije težišča večkotnika potrebujemo postopek načrtovanja točke, ki dano daljico deli v določenem razmerju. Ta postopek bomo obravnavali v razdelku \ref{odd5TalesVekt} (zgleda \ref{izrekEnaDelitevDaljice} in \ref{izrekEnaDelitevDaljiceNan}).



 V dokazu trditve \ref{vektTezVeck} nismo uporabljali
dejstva, da so točke $A_1$, $A_2$, ..., $A_n$ v isti ravnini.
Trditev velja tudi v primeru, če je $ABCD$ ($n=4$) t. i. \pojem{tetraeder}
\index{tetraeder}. Omenjeno točko tedaj imenujemo
\index{težišče!tetraedra} \pojem{težišče tetraedra}. Mogoče je
namreč dokazati analogni izrek za tetraeder; daljice, ki so
določene z oglišči tetraedra in težišči nasprotnih ploskev,
potekajo skozi težišče tega tetraedra, ki jih deli v razmerju $3 :1$.

 Tudi v splošnem primeru, če je $n\in \mathbb{N}$, lahko govorimo o t. i. \index{simpleks}\pojem{simpleksu} (posplošitev: točka, daljica, trikotnik, tetraeder, ...), ki leži v $(n-1)$-razsežnem evklidskem prostoru.

                    \bzgled
                    Let $A$, $B$ and $C$ be the centroids of a triangles $OMN$,
                    $ONP$ and $OPM$, then the centroid  $T$ of the triangle $MNP$,
                    the centroid $T_1$ of the triangle $ABC$ and the point $O$ are
                    three collinear points. Furthermore, it is $OT_1:T_1T=2:1$.
                    \ezgled

\begin{figure}[!htb]
\centering
\input{sl.vek.5.5.6.pic}
\caption{} \label{sl.vek.5.5.6.pic}
\end{figure}

 \textbf{\textit{Proof.}} Označimo z $P_1$, $M_1$ in $N_1$ središča stranic $MN$, $NP$ in $PM$ trikotnika $PMN$ (Figure \ref{sl.vek.5.5.6.pic}).
  Če uporabimo izreke \ref{vektTezTrikXT}, \ref{tezisce} in \ref{vektSredOSOAOB}, dobimo:

  \begin{eqnarray*}
  \overrightarrow{OT_1}&=&\frac{1}{3}\left(
  \overrightarrow{OA}+\overrightarrow{OB}+\overrightarrow{OC} \right)=\\
  &=&\frac{1}{3}\left(
  \frac{2}{3}\overrightarrow{OP_1}+\frac{2}{3}\overrightarrow{OM_1}+
  \frac{2}{3}\overrightarrow{ON_1} \right)=\\
  &=&\frac{1}{3}\left(
  \frac{1}{3}\left(\overrightarrow{OM}+\overrightarrow{ON}\right)
  +\frac{1}{3}\left(\overrightarrow{ON}+\overrightarrow{OP}\right)+
  \frac{1}{3}\left(\overrightarrow{OP}+\overrightarrow{OM}\right) \right)=\\
  &=&\frac{2}{9}\left(
  \overrightarrow{OM}+\overrightarrow{ON}+
  \overrightarrow{OP} \right)=\\
  &=&\frac{2}{3}\overrightarrow{OT}
  \end{eqnarray*}

Iz $\overrightarrow{OT_1}=\frac{2}{3}\overrightarrow{OT}$ pa sledi $\overrightarrow{OT_1}=2\overrightarrow{T_1T}$, kar pomeni, da sta vektorja  $\overrightarrow{OT_1}$ in $\overrightarrow{T_1T}$ kolinearna (tudi točke $O$, $T_1$ in $T$) in velja $OT_1:T_1T=2:1$.
\kdokaz

                    \bzgled
                    The centroid of a regular $n$-gon $A_1A_2...A_n$ is its centre (i.e. incentre and circumcentre).
                    \ezgled

\begin{figure}[!htb]
\centering
\input{sl.vek.5.5.7.pic}
\caption{} \label{sl.vek.5.5.7.pic}
\end{figure}

 \textbf{\textit{Proof.}} Označimo s $S$ središče pravilnega $n$-kotnika $A_1A_2...A_n$ (Figure \ref{sl.vek.5.5.7.pic}).
Dovolj je dokazati, da velja:
$$\overrightarrow{SA_1}+\overrightarrow{SA_2}+
\cdots+\overrightarrow{SA_n}=\overrightarrow{0}.$$
Čeprav je v primeru, ko je $n$ sodo število, trditev trivialna, bomo dokaz izpeljali za splošno vrednost $n$ (sodo in liho).
Predpostavimo, da je $\overrightarrow{SA_1}+\overrightarrow{SA_2}+\cdots+\overrightarrow{SA_n}=\overrightarrow{SX}$, kjer je $X\neq S$.
Toda če zavrtimo večkotnik okoli središča $S$ za kot $\theta=\frac{360}{n}$ (glej razdelek o rotaciji \ref{odd6Rotac}), se vsota vektorjev na levi strani enakosti ne spremeni, rezultat na desni pa postane vektor $\overrightarrow{SX'}$, kjer je $X'$ točka, ki jo dobimo iz $X$ z isto rotacijo. Ker mora ostati tudi desna stran enakosti nespremenjena, dobimo $\overrightarrow{SX'}=\overrightarrow{SX}$ oz. $X'=X$. To je možno edino v primeru, ko je $X=S$ oz. $\overrightarrow{SX}=\overrightarrow{0}$.
\kdokaz






%________________________________________________________________________________
 \poglavje{Hamilton's Theorem}  \label{odd5Hamilton}

Sedaj bomo nadaljevali z lastnostmi, ki se nanašajo na značilne točke trikotnika (razdelek \ref{odd3ZnamTock}).

        \bizrek \label{HamiltonLema}
        If $O$,  $V$ and $A_1$ are the circumcentre, the orthocentre and the midpoint
        of the side $BC$ of a triangle $ABC$, respectively, then
        $$\overrightarrow{AV}=2\cdot \overrightarrow{OA_1}.$$
        \eizrek


 \textbf{\textit{Proof.}}
Označimo še z $B_1$ središče stranice $AC$  (Figure \ref{sl.vek.5.6.2.pic}).

Vektorja $\overrightarrow{OA_1}$ in $\overrightarrow{AV}$ sta kolinearna, zato je $\overrightarrow{OA_1}=\alpha \cdot \overrightarrow{AV}$ za nek $\alpha \in \mathbb{R}$ (izrek \ref{vektKriterijKolin}). Prav tako je iz istih razlogov $\overrightarrow{OB_1}=\beta \cdot \overrightarrow{BV}$ (oz. $\overrightarrow{B_1O}=\beta \cdot \overrightarrow{VB}$) za nek $\beta \in \mathbb{R}$. Po izreku \ref{srednjicaTrikVekt} (srednjica trikotnika) je:
 $$\overrightarrow{B_1A_1}=\frac{1}{2}\overrightarrow{AB}=
 \frac{1}{2}\left(\overrightarrow{AV}+\overrightarrow{VB}\right)=
 \frac{1}{2}\overrightarrow{AV}+\frac{1}{2}\overrightarrow{VB}.$$
 Hkrati velja tudi:
 $$\overrightarrow{B_1A_1}=\overrightarrow{B_1O}+\overrightarrow{OA_1}=
\beta\overrightarrow{VB}+\alpha\overrightarrow{AV}=
 \alpha\overrightarrow{AV}+\beta\overrightarrow{VB}.$$

Ker sta vektorja $\overrightarrow{AV}$ in $\overrightarrow{VB}$ nekolinearna, iz izreka \ref{vektLinKomb1Razcep} sledi $\alpha=\frac{1}{2}$ in $\beta=\frac{1}{2}$. Zato je $\overrightarrow{OA_1}=\alpha \overrightarrow{AV}=\frac{1}{2}\overrightarrow{AV}$ oz. $\overrightarrow{AV}=2\cdot \overrightarrow{OA_1}$.
\kdokaz

\begin{figure}[!htb]
\centering
\input{sl.vek.5.6.2.pic}
\caption{} \label{sl.vek.5.6.2.pic}
\end{figure}

 Zelo uporaben je naslednji izrek.

             \bizrek \label{Hamilton}\index{izrek!Hamiltonov}
             (Hamilton's\footnote{\index{Hamilton, W. R.}\textit{W. R. Hamilton} (1805--1865), angleški matematik.} theorem)
              If $O$ and $V$ are circumcentre and orthocentre of a triangle $ABC$, respectively then
             $$\overrightarrow{OA}+\overrightarrow{OB}
             +\overrightarrow{OC}=\overrightarrow{OV}.$$

             \eizrek



 \textbf{\textit{Proof.}} Označimo z $A_1$ središče stranice $BC$  (Figure \ref{sl.vek.5.6.2.pic}). Če uporabimo izreka \ref{vektSredOSOAOB} in \ref{HamiltonLema}, dobimo:

$$\overrightarrow{OA}+\overrightarrow{OB}
        +\overrightarrow{OC}
        =\overrightarrow{OA}+2\cdot \overrightarrow{OA_1}=
        \overrightarrow{OA}+\overrightarrow{AV}=
        \overrightarrow{OV},$$ kar je bilo treba dokazati. \kdokaz

 Nadaljevali bomo s posledicami prejšnjih dveh izrekov.


             \bzgled \label{HamiltonPoslTetiv}
            A quadrilateral $ABCD$ is inscribed in a circle with a centre $O$.
            The diagonals $AC$ and $BD$ are perpendicular.
            If $M$ is the foot of the perpendicular from the centre $O$  on the  line $CD$, then
             $$|OM|=\frac{1}{2}|AB|.$$
               \ezgled

\begin{figure}[!htb]
\centering
\input{sl.vek.5.6.3a.pic}
\caption{} \label{sl.vek.5.6.3a.pic}
\end{figure}

 \textbf{\textit{Proof.}} Naj bo $V$ višinska točka trikotnika $BCD$
 (Figure \ref{sl.vek.5.6.3a.pic}).
 Ker je $AC\perp BD$, točka $V$ leži na diagonali $AC$. Po izreku
 \ref{HamiltonLema} je $\overrightarrow{OM}=\frac{1}{2} \overrightarrow{BV}$,
 zato je tudi $|OM|=\frac{1}{2}|BV|$. Ker je še (izreka \ref{ObodObodKot} in
 \ref{KotaPravokKraki}):
  $$\angle BAV=\angle BAC\cong\angle BDC\cong\angle AVB,$$
sledi $BV\cong AB$ (izrek \ref{enakokraki}) oz.
$|OM|=\frac{1}{2}|AB|$.
 \kdokaz


        \bzgled Let $V$ be the orthocentre and $O$ the circumcentre of a triangle $ABC$ and $AV\cong AO$.
            Prove that $\angle BAC=60^0$.
        \ezgled

\begin{figure}[!htb]
\centering
\input{sl.vek.5.6.1a.pic}
\caption{} \label{sl.vek.5.6.1a.pic}
\end{figure}

 \textbf{\textit{Proof.}}
 Naj bo $A_1$ središče stranice $BC$ trikotnika $ABC$ (Figure \ref{sl.vek.5.6.1a.pic}). Po izreku \ref{HamiltonLema} je
$|AV|=2\cdot|OA_1|$. Ker je še $OA\cong OC$, sledi, da v pravokotnemu trikotniku $OA_1C$ velja $|OC|=2\cdot|OA_1|$. Z $O'$ označimo točko, ki je simetrična točki $O$ glede na točko $A_1$. Iz $\triangle OA_1C\cong \triangle O'A_1C$ (izrek \textit{SAS} \ref{SKS}) sledi $OC\cong O'C\cong OO'$, kar pomeni, da je $\triangle OO'C$ enakostranični trikotnik oz. $\angle A_1OC=60^0$. Iz izreka \ref{SredObodKot} in skladnosti trikotnikov $BOA_1$ in $COA_1$ (izrek \textit{SSS} \ref{SSS}) na koncu sledi
$\angle BAC=\frac{1}{2}\angle BOC=\angle A_1OC=60^0$.
  \kdokaz

        \bzgled \label{TetivniVisinska}
        Let $ABCD$ be a cyclic quadrilateral and:
         $V_A$ the orthocentre of the triangle $BCD$,
          $V_B$  the orthocentre of the triangle $ACD$,
           $V_C$  the orthocentre of the triangle $ABD$ and
           $V_D$  the orthocentre of the triangle  $ABC$.
           Prove that:\\
         a) the line segments $AV_A$,
         $BV_B$, $CV_C$ and $DV_D$
         has a common midpoint,\\
         b) the quadrilateral $V_AV_BV_CV_D$ is congruent
          to the quadrilateral $ABCD$.
        \ezgled


\begin{figure}[htp]
\centering
\input{sl.vek.5.6.3b.pic}
\caption{} \label{sl.vek.5.6.3b.pic}
\end{figure}


\begin{figure}[htp]
\centering
\input{sl.vek.5.6.3.pic}
\caption{} \label{sl.vek.5.6.3.pic}
\end{figure}


 \textbf{\textit{Solution.}} Naj bo $O$ središče očrtane krožnice tetivnega štirikotnika $ABCD$ (Figure \ref{sl.vek.5.6.3.pic}). Jasno je, da je točka $O$ hkrati središče skupne očrtane krožnice trikotnikov $BCD$, $ACD$, $ABD$ in $ABC$.
 Po Hamiltonovem izreku \ref{Hamilton} sledi:

\begin{eqnarray*}
 \overrightarrow{OV_A}&=&\overrightarrow{OB}+\overrightarrow{OC}+\overrightarrow{OD}\\
\overrightarrow{OV_B}&=&\overrightarrow{OA}+\overrightarrow{OC}+\overrightarrow{OD}
 \end{eqnarray*}
 Nato dobimo:
\begin{eqnarray*}
 \overrightarrow{V_BV_A}&=&\overrightarrow{V_BO}+\overrightarrow{OV_A}=\\
&=&\overrightarrow{OV_A}-\overrightarrow{OV_B}=\\
 &=&\overrightarrow{OB}+\overrightarrow{OC}+\overrightarrow{OD}
-(\overrightarrow{OA}+\overrightarrow{OC}+\overrightarrow{OD})=\\
&=&\overrightarrow{OB}-\overrightarrow{OA}=\\
&=&\overrightarrow{AB}.
 \end{eqnarray*}


Torej velja
 $\overrightarrow{V_BV_A}=\overrightarrow{AB}$. Po izreku \ref{vektParalelogram} je štirikotnik $ABV_AV_B$ paralelogram, po izreku \ref{paralelogram} pa imata njegovi diagonali $AV_A$ in $BV_B$ skupno središče - označimo ga s $S$
 (Figure \ref{sl.vek.5.6.3b.pic}). Na podoben način imata tudi vsak od parov daljic $AV_A$ in $CV_C$ oz. $AV_A$ in $DV_D$ skupno središče. Ker gre za središče daljice $AV_A$, sledi, da imajo vse štiri daljice $AV_A$,
         $BV_B$, $CV_C$ in $DV_D$
         skupno središče - točko $S$.

 Na podoben način kot $\overrightarrow{V_BV_A}=\overrightarrow{AB}$ sledi tudi $\overrightarrow{V_CV_B}=\overrightarrow{BC}$, $\overrightarrow{V_DV_C}=\overrightarrow{CD}$ in $\overrightarrow{V_DV_A}=\overrightarrow{DA}$.
  To pomeni, da imata štirikotnika $V_AV_BV_CV_D$ in $ABCD$ (izreka \ref{vektVzpSkl} in \ref{KotaVzporKraki}) vse istoležne stranice in notranje kote skladne. Torej velja $V_AV_BV_CV_D\cong ABCD$. Za formalni dokaz tega lahko uporabimo izometrijo $\mathcal{I}:A,B,C\mapsto V_A,V_B,V_C$ in dokažemo $\mathcal{I}(D)=V_D$.
 \kdokaz


          \bzgled \label{HamiltonSimson}\index{premica!Simsonova}
          Let $ABCD$ be a cyclic quadrilateral and:  $a$ is the Simson
          line with respect to the triangle $BCD$ and the point $A$, $b$ is the Simson
         line with respect to the triangle  $ACD$  and the point  $B$, $c$  is the Simson
           line with respect to the triangle  $ABD$  and the point  $C$ ter $d$ S is the Simson
            line with respect to the triangle  $ABC$  and the point  $D$.
           Prove that the lines $a$, $b$, $c$ and
          $d$  intersect at a single point.
          \ezgled


\begin{figure}[htp]
\centering
\input{sl.vek.5.6.4.pic}
\caption{} \label{sl.vek.5.6.4.pic}
\end{figure}

 \textbf{\textit{Solution.}} (Figure \ref{sl.vek.5.6.4.pic}).

 Trditev je direktna posledica izrekov
 \ref{SimsZgled3} in \ref{TetivniVisinska}  - premice $a$, $b$, $c$ in
        $d$ se sekajo v točki $S$ (iz izreka \ref{TetivniVisinska}).
 \kdokaz



%________________________________________________________________________________
 \poglavje{Euler Line}  \label{odd5EulPrem}

Sedaj bomo dokazali pomembno lastnost, ki se nanaša na tri značilne točke trikotnika.

                \bizrek \label{EulerjevaPremica}
                The circumcentre $O$, the centroid $T$ and the orthocentre $V$
                of an arbitrary triangle lies on the same line. Besides that it is
                $$|OT|:|TV|=1:2.$$
                \eizrek

\begin{figure}[!htb]
\centering
\input{sl.vek.5.7.1.pic}
\caption{} \label{sl.vek.5.7.1.pic}
\end{figure}

 \textbf{\textit{Proof.}}  (Figure \ref{sl.vek.5.7.1.pic})

 Če uporabimo izrek \ref{vektTezTrikXT} in Hamiltonov izrek \ref{Hamilton}, dobimo:
 $$\overrightarrow{OT}=\frac{1}{3}\left(\overrightarrow{OA}
                +\overrightarrow{OB}+\overrightarrow{OC}\right)=
                \frac{1}{3}\overrightarrow{OV}.$$

Vektorja $\overrightarrow{OT}$ in $\overrightarrow{OV}$ sta torej kolinearna in velja $\overrightarrow{OT}:\overrightarrow{OV}=1:3$. To pomeni, da so točke $O$, $T$ in $V$ kolinearne in velja $|OT|:|TV|=1:2$.
 \kdokaz

 Premico iz prejšnjega izreka, na kateri ležijo tri značilne točke, imenujemo \index{premica!Eulerjeva} \pojem{Eulerjeva\footnote{Lastnost, ki jo je leta 1765  dokazal švicarski matematik \index{Euler, L.}\textit{L. Euler} (1707--1783).} premica}.

 V naslednjem izreku bomo videli povezavo med Eulerjevo premico in Eulerjevo krožnico, ki smo jo obravnavali v razdelku \ref{odd3EulKroz}.




                    \bizrek \label{EulerKrozPrem1}\index{krožnica!Eulerjeva}
                    The centre of Euler of an arbitrary triangle lies on
                    The Euler line of this triangle.
                    Furthermore, it is the midpoint of the line segment determined by
                     the orthocentre and the circumcentre of this triangle.
                    \eizrek

\begin{figure}[!htb]
\centering
\input{sl.vek.5.7.2.pic}
\caption{} \label{sl.vek.5.7.2.pic}
\end{figure}

 \textbf{\textit{Proof.}}
 Naj bodo $AA'$, $BB'$ in $CC'$ višine ter $A_1$, $B_1$ in $C_1$ središča stranic $BC$, $AC$ in $AB$ trikotnika $ABC$. Označimo z $O$ središče očrtane krožnice, z $V$ višinsko točko tega trikotnika ter z $V_A$, $V_B$ in $V_C$ središča daljic $VA$, $VB$ in $VC$ (Figure \ref{sl.vek.5.7.2.pic}).

 Po izreku \ref{EulerKroznica} ležijo točke $A'$, $B'$, $C'$, $A_1$, $B_1$, $C_1$, $V_A$, $V_B$ in $V_C$ na eni krožnici - t. i. Eulerjevi krožnici. Označimo središče te krožnice z $E$. Ker je $\angle V_AA'A_1\cong \angle AA'C=90^0$, je po izreku \ref{TalesovIzrKroz2} daljica $V_AA_1$ premer te krožnice oz. je točka $E$ središče daljice $V_AA_1$.

 Po izreku \ref{HamiltonLema} velja:
 $$\overrightarrow{OA_1}=\frac{1}{2}\cdot \overrightarrow{AV}=\overrightarrow{V_AV}.$$
 Torej $\overrightarrow{OA_1}=\overrightarrow{V_AV}$, kar pomeni, da je štirikotnik $A_1OV_AV$ paralelogram (izrek \ref{vektParalelogram}). Njegovi diagonali $VO$ in $V_AA_1$ se razpolavljata (\ref{paralelogram}), zato je točka $E$ središče daljice $OV$ in leži na Eulerjevi premici trikotnika $ABC$ (izrek \ref{EulerjevaPremica}).
 \kdokaz

V razdelku \ref{odd7SredRazteg} (izrek \ref{EulerKroznicaHomot}) bomo videli še nadaljevanje prejšnje trditve, ki se nanaša na Eulerjevo krožnico.


%________________________________________________________________________________
 \poglavje{Thales' Theorem - Basic Proportionality Theorem}  \label{odd5TalesVekt}

 Že v razdelku \ref{odd5LinKombVekt} smo ugotovili, da lahko pri dveh kolinearnih vektorjih govorimo o njunem razmerju ali količniku.
 Za kolinearne vektorje $\overrightarrow{v}$ in $\overrightarrow{u}$ ($\overrightarrow{u}\neq \overrightarrow{0}$) smo definirali njuno razmerje oz. količnik
 $$\overrightarrow{v}:\overrightarrow{u}
=\frac{\overrightarrow{v}}{\overrightarrow{u}}=\lambda,$$
 če za nek $\lambda\in\mathbb{R}$ velja $\overrightarrow{v}=\lambda \overrightarrow{u}$.

Na podoben način kot pri številih lahko definiramo tudi sorazmerje dveh parov kolinearnih vektorjev. Če sta $\overrightarrow{a}$ in $\overrightarrow{b}$ ($\overrightarrow{b}\neq \overrightarrow{0}$) par kolinearnih vektorjev oz. $\overrightarrow{c}$ in $\overrightarrow{d}$ ($\overrightarrow{d}\neq \overrightarrow{0}$) par kolinearnih vektorjev, pravimo, da sta para vektorjev \index{sorazmerje kolinearnih vektorjev}\pojem{sorazmerna}, če velja:

        $$\frac{\overrightarrow{a}}{\overrightarrow{b}}
        =\frac{\overrightarrow{c}}{\overrightarrow{d}}.$$

 Naslednji zelo pomemben izrek se nanaša na definirani pojem sorazmerja.

            \bizrek
            \label{TalesovIzrek}(Thales'\footnote{Starogrški filozof in matematik \textit{Tales}
            \index{Tales} iz Mileta (640--546 pr. n. š.) je obravnaval sorazmerje ustreznih daljic,
            ki jih dobimo, če dve premici presekamo z dvema vzporednicama, pri tem pa ni omenjal
            vektorske oblike.} theorem - Basic Proportionality Theorem)\\
            Let $a$, $b$, $p$ and $p'$ be lines in the same plane, and $O=a\cap b$, $A=a\cap p$, $A'=a\cap p'$, $B=b\cap p$ and $B'=b\cap p'$.\\
             If $p\parallel p'$, then
            $$\frac{\overrightarrow{OA'}}{\overrightarrow{OA}}=
            \frac{\overrightarrow{OB'}}{\overrightarrow{OB}}=
            \frac{\overrightarrow{A'B'}}{\overrightarrow{AB}}.$$
            \index{izrek!Talesov o sorazmerju}
            \eizrek

\begin{figure}[!htb]
\centering
\input{sl.vek.5.8.1.pic}
\caption{} \label{sl.vek.5.8.1.pic}
\end{figure}

 \textbf{\textit{Proof.}}  (Figure \ref{sl.vek.5.8.1.pic})

Ker je po predpostavki $p\parallel p'$, sta vektorja $\overrightarrow{A'B'}$ in $\overrightarrow{AB}$  kolinearna. Po izreku \ref{vektKriterijKolin} je $\overrightarrow{A'B'}=\lambda\overrightarrow{AB}$ za nek $\lambda\in \mathbb{R}$. Na podoben način iz kolinearnosti vektorjev $\overrightarrow{OA'}$ in $\overrightarrow{OA}$ oz.  $\overrightarrow{OB'}$ in $\overrightarrow{OB}$ sledi
 $\overrightarrow{OA'}=\alpha\overrightarrow{OA}$ za nek $\alpha\in \mathbb{R}$ oz.
  $\overrightarrow{OB'}=\beta\overrightarrow{OB}$ za nek $\beta\in \mathbb{R}$.
  Iz tega dobimo:
  $$\frac{\overrightarrow{A'B'}}{\overrightarrow{AB}}=\lambda,\hspace*{2mm}
  \frac{\overrightarrow{OA'}}{\overrightarrow{OA}}=\alpha,\hspace*{2mm}
  \frac{\overrightarrow{OB'}}{\overrightarrow{OB}}=\beta.$$
 Dovolj je še dokazati, da velja $\alpha=\beta=\lambda$.
 Če uporabimo pravilo za odštevanje vektorjev \ref{vektOdsev} in izrek \ref{vektVektorskiProstor} (točka $\textit{6}$), dobimo:
 \begin{eqnarray*}
 \overrightarrow{A'B'}&=&\overrightarrow{OB'}-\overrightarrow{OA'}
 =\beta\overrightarrow{OB}-\alpha \overrightarrow{OA};\\
  \overrightarrow{A'B'}&=&\lambda\overrightarrow{AB}=
  \lambda\left(\overrightarrow{OB}-\overrightarrow{OA}\right)
 =\lambda\overrightarrow{OB}-\lambda\overrightarrow{OA}.
 \end{eqnarray*}
 Ker sta $\overrightarrow{OA}$ in $\overrightarrow{OB}$ nekolinearna vektorja je po izreku \ref{vektLinKomb1Razcep}
 $\alpha=\beta=\lambda$ oz. $\frac{\overrightarrow{OA'}}{\overrightarrow{OA}}=
            \frac{\overrightarrow{OB'}}{\overrightarrow{OB}}=
            \frac{\overrightarrow{A'B'}}{\overrightarrow{AB}}$.
\kdokaz


  Direktna posledica je Talesov izrek v obliki sorazmerja daljic, ki ni v vektorski obliki (Figure \ref{sl.vek.5.8.2.pic}).


                         \bizrek \label{TalesovIzrekDolzine}
                        Let $a$, $b$, $p$ and $p'$ be lines in the same plane, and
                         $O=a\cap b$, $A=a\cap p$, $A'=a\cap p'$, $B=b\cap p$ and $B'=b\cap p'$.\\
                         If $p\parallel p'$, then
                         $$\frac{OA'}{OA}=
                         \frac{OB'}{OB}=
                         \frac{A'B'}{AB}$$
                         and also
                         $$\frac{OA'}{OB'}=
                         \frac{OA}{OB}.$$
                         \eizrek

\begin{figure}[!htb]
\centering
\input{sl.vek.5.8.2.pic}
\caption{} \label{sl.vek.5.8.2.pic}
\end{figure}

  \textbf{\textit{Proof.}} Trditev je direktna posledica izrekov \ref{TalesovIzrek} in \ref{vektKolicnDolz}.
 \kdokaz

 Dokazali bomo, da velja tudi obratno, oz. da iz ustreznega sorazmerja sledi vzporednost pripadajočih premic.


                      \bizrek \label{TalesovIzrekObr}(Converse Thales' proportionality theorem)\\
                      Let $a$, $b$, $p$ and $p'$ be lines in the same plane, and $O=a\cap b$, $A=a\cap p$, $A'=a\cap p'$, $B=b\cap p$ and $B'=b\cap p'$.\\ If
                      $$\frac{\overrightarrow{OA'}}{\overrightarrow{OA}}=
            \frac{\overrightarrow{OB'}}{\overrightarrow{OB}},$$
                     then $p\parallel p'$ and also
                    $$\frac{\overrightarrow{A'B'}}{\overrightarrow{AB}}=
                    \frac{\overrightarrow{OA'}}{\overrightarrow{OA}}=
                    \frac{\overrightarrow{OB'}}{\overrightarrow{OB}}.$$
                    \index{izrek!Talesov obratni o sorazmerju}
                    \eizrek


  \textbf{\textit{Proof.}} Označimo $\frac{\overrightarrow{OA'}}{\overrightarrow{OA}}=
            \frac{\overrightarrow{OB'}}{\overrightarrow{OB}}=\lambda$. V tem primeru je najprej $\overrightarrow{OA'}=\lambda\overrightarrow{OA}$ in $\overrightarrow{OB'}=\lambda\overrightarrow{OB}$. Zato je (izreka \ref{vektOdsev} in \ref{vektVektorskiProstor}):
  $$\overrightarrow{A'B'}=\overrightarrow{OB'}-\overrightarrow{OA'}
  =\lambda\overrightarrow{OB}-\lambda\overrightarrow{OA}
  =\lambda\left(\overrightarrow{OB}-\overrightarrow{OA}\right)
  =\lambda\overrightarrow{AB}.$$
  Ker je torej $\overrightarrow{A'B'}=\lambda\overrightarrow{AB}$, sta po izreku \ref{vektKriterijKolin} vektorja $\overrightarrow{A'B'}$ in $\overrightarrow{AB}$ kolinearna. To pomeni, da velja $AB\parallel A'B'$ oz. $p\parallel p'$.

 Na koncu iz izreka \ref{TalesovIzrek} sledi relacija
                   $\frac{\overrightarrow{A'B'}}{\overrightarrow{AB}}=
                    \frac{\overrightarrow{OA'}}{\overrightarrow{OA}}=
                    \frac{\overrightarrow{OB'}}{\overrightarrow{OB}}$.
 \kdokaz

Navedimo še nekaj posledic Talesovega izreka \ref{TalesovIzrek}.

                    \bizrek \label{TalesPosl1}
                    If parallel lines $p_1$, $p_2$, $p_3$ intersect a line $a$ at points $A_1$, $A_2$,
                    $A_3$ and a line $b$ at points $B_1$, $B_2$,
                    $B_3$, then
                     $$\frac{A_1A_2}{B_1B_2}=\frac{A_2A_3}{B_2B_3}
                    \hspace*{1mm}
                    \textrm{ and } \hspace*{1mm}
                    \frac{A_1A_2}{A_2A_3}=\frac{B_1B_2}{B_2B_3}.$$
                    \eizrek

\begin{figure}[!htb]
\centering
\input{sl.vek.5.8.4.pic}
\caption{} \label{sl.vek.5.8.4.pic}
\end{figure}

 \textbf{\textit{Proof.}}  (Figure \ref{sl.vek.5.8.4.pic})

 Brez škode za splošnost naj bo $\mathcal{B}(A_1,A_2,A_3)$ in $\mathcal{B}(B_1,B_2,B_3)$.

Označimo s $c$ vzporednico premice $b$ skozi točko $A_1$ ter s $C_2$ in $C_3$ presečišči premice $c$ s premicama $p_2$ in $p_3$. Štirikotnika $B_1B_3C_3A_1$ in $B_2B_3C_3C_2$ sta paralelograma, zato je $\overrightarrow{B_1B_3}=\overrightarrow{A_1C_3}$ in $\overrightarrow{B_2B_3}=\overrightarrow{C_2C_3}$.

Če uporabimo še izrek \ref{TalesovIzrekDolzine}, dobimo:
 \begin{eqnarray*}
 \frac{|A_1A_2|}{|A_2A_3|}&=&
 \frac{|A_1A_3|-|A_2A_3|}{|A_2A_3|}=
\frac{|A_1A_3|}{|A_2A_3|}-1=\\
 &=&\frac{|A_1C_3|}{|C_2C_3|}-1=
 \frac{|B_1B_3|}{|B_2B_3|}-1=\\
 &=&\frac{|B_1B_3|-|B_2B_3|}{|B_2A_3|}=
\frac{|B_1B_2|}{|B_2B_3|},
 \end{eqnarray*}
oz. $\frac{A_1A_2}{A_2A_3}=\frac{B_1B_2}{B_2B_3}$ in
$\frac{A_1A_2}{B_1B_2}=\frac{A_2A_3}{B_2B_3}$.
 \kdokaz


                    \bizrek \label{TalesPosl2}
                    If parallel lines $p_1$, $p_2$,..., $p_n$ intersect a line $a$ at points $A_1$, $A_2$,...,
                    $A_n$ and a line $b$ at points $B_1$, $B_2$,...,
                    $B_n$, then
                    $$\frac{A_1A_2}{B_1B_2}=\frac{A_2A_3}{B_2B_3}=\cdots=
                    \frac{A_{n-1}A_n}{B_{n-1}B_n}\hspace*{1mm}
                    \textrm{ and } \hspace*{1mm}$$
                    $$A_1A_2:A_2A_3:\cdots :A_{n-1}A_n= B_1B_2:B_2B_3:\cdots :B_{n-1}B_n.$$
                    \eizrek


\begin{figure}[!htb]
\centering
\input{sl.vek.5.8.5.pic}
\caption{} \label{sl.vek.5.8.5.pic}
\end{figure}

 \textbf{\textit{Proof.}}  (Figure \ref{sl.vek.5.8.5.pic})

Trditev je direktna posledica izreka \ref{TalesPosl1}.
 \kdokaz


                    \bizrek \label{TalesPosl3}
                    Let $p_1$, $p_2$ and $p_3$ be lines that intersect at a point $O$. If $a$ and $b$ are
                    parallel lines that intersect the line $p_1$ at points $A_1$ and $B_1$,
                    the line $p_2$ at points $A_2$ and $B_2$, and
                    the line $p_3$ at points $A_3$ and $B_3$, then
                    $$\frac{A_1A_2}{B_1B_2}=\frac{A_2A_3}{B_2B_3}
                    \hspace*{1mm}
                    \textrm{ and } \hspace*{1mm}
                    \frac{A_1A_2}{A_2A_3}=\frac{B_1B_2}{B_2B_3}.$$
                    \eizrek


\begin{figure}[!htb]
\centering
\input{sl.vek.5.8.6.pic}
\caption{} \label{sl.vek.5.8.6.pic}
\end{figure}

 \textbf{\textit{Proof.}}  (Figure \ref{sl.vek.5.8.6.pic})

 Po izreku \ref{TalesovIzrekDolzine} dobimo:
  $$\frac{A_1A_2}{B_1B_2}=
  \frac{OA_2}{OB_2}=\frac{A_2A_3}{B_2B_3},$$ kar je bilo treba dokazati. \kdokaz



                    \bizrek \label{TalesPosl4}
                    Let $p_1$, $p_2$,..., $p_n$  be  lines that intersect at a point $O$. If $a$ and $b$ are
                    parallel lines that intersect the line $p_1$ at points $A_1$ and $B_1$,
                    the line $p_2$ at points $A_2$ and $B_2$,...,
                    the line $p_n$ at points $A_n$ and $B_n$, then
                     $$\frac{A_1A_2}{B_1B_2}=\frac{A_2A_3}{B_2B_3}=\cdots=
                    \frac{A_{n-1}A_n}{B_{n-1}B_n}\hspace*{1mm}
                    \textrm{ and } \hspace*{1mm}$$
                    $$A_1A_2:A_2A_3:\cdots :A_{n-1}A_n= B_1B_2:B_2B_3:\cdots :B_{n-1}B_n.$$
                    \eizrek



\begin{figure}[!htb]
\centering
\input{sl.vek.5.8.8.pic}
\caption{} \label{sl.vek.5.8.8.pic}
\end{figure}

 \textbf{\textit{Proof.}}  (Figure \ref{sl.vek.5.8.8.pic})

Trditev je direktna posledica izreka \ref{TalesPosl3}.
 \kdokaz

 Zelo znani in koristni sta naslednji načrtovalni nalogi.



                     \bzgled
                     \label{izrekEnaDelitevDaljiceNan}
                     \index{delitev daljice!na enake dele}
                      Construct points that divide a line segment $AB$
                      into $n$ congruent line segments.
                    \ezgled

\begin{figure}[!htb]
\centering
\input{sl.vek.5.8.9.pic}
\caption{} \label{sl.vek.5.8.9.pic}
\end{figure}


\textbf{\textit{Solution.}}  (Figure \ref{sl.vek.5.8.9.pic})

Naj bo $X$ poljubna točka, ki ne leži na premici $AB$ in $Q_1$, $Q_2$, ..., $Q_n$ takšne točke na poltraku $AX$, da velja
 $\overrightarrow{AQ_1}=\overrightarrow{Q_1Q_2}=\cdots =\overrightarrow{Q_{n-1}Q_n}$.
 S $P_1$, $P_2$, ..., $P_{n-1}$ označimo presečišče premice $AB$ z vzporednicama premice $BQ_n$ skozi točke $Q_1$, $Q_2$, ..., $Q_{n-1}$.

 Dokažimo, da so $P_1$, $P_2$, ..., $P_{n-1}$ iskane točke. Po izreku \ref{TalesPosl2} velja:
 $$\frac{AP_1}{AQ_1}=\frac{P_1P_2}{Q_1Q_2}=\dots=\frac{P_{n-1}B}{Q_{n-1}Q_n}.$$
 Ker je po predpostavki $|\overrightarrow{AQ_1}|=|\overrightarrow{Q_1Q_2}|=\cdots =|\overrightarrow{Q_{n-1}Q_n}|$, je tudi
$|AP_1|=|P_1P_2|=\cdots =|P_{n-1}B|$.
 \kdokaz


                     \bzgled
                     \label{izrekEnaDelitevDaljice}
                     \index{delitev daljice!v razmerju}
                     Divide a given line segment $AB$ in the ratio $n:m$ ($n,m\in \mathbb{N}$),
                    i.e. determine such point $T$ on the line $AB$ that
                     $$\frac{\overrightarrow{AT}}{\overrightarrow{TB}}=\frac{n}{m}.$$
                     Prove that there is the only one solution for such point $T$.
                    \ezgled


\begin{figure}[!htb]
\centering
\input{sl.vek.5.8.10.pic}
\caption{} \label{sl.vek.5.8.10.pic}
\end{figure}


\textbf{\textit{Solution.}}  (Figure \ref{sl.vek.5.8.10.pic})


  Po izreku \ref{izrekEnaDelitevDaljiceVekt} obstaja ena sama točka $T$, za katero velja $\frac{\overrightarrow{AT}}{\overrightarrow{TB}}=\frac{n}{m}$.

  Sedaj bomo opisali postopek konstrukcije točke $T$. Naj bo $\overrightarrow{v}$ poljubni vektor, ki ni kolinearen z vektorjem $\overrightarrow{AB}$, ter $P$ in $Q$ takšni točki, da velja $\overrightarrow{AP}=n\overrightarrow{v}$ in $\overrightarrow{BQ}=-m\overrightarrow{v}$. S $T$ označimo presečišče premic $AB$ in $PQ$ (presečišče obstaja, ker je $P,Q\div AB$). Potem velja:
   \begin{eqnarray*}
   \frac{\overrightarrow{AT}}{\overrightarrow{TB}}=
   -\frac{\overrightarrow{TA}}{\overrightarrow{TB}}=
   -\frac{\overrightarrow{AP}}{\overrightarrow{BQ}}=
   -\frac{n\overrightarrow{v}}{-m\overrightarrow{v}}=\frac{n}{m},
   \end{eqnarray*}
 kar je bilo treba dokazati.  \kdokaz

 V razdelku \ref{odd7Harm} bomo naprej raziskovali vprašanje delitve daljice v danem razmerju.



                    \bzgled \label{vektTrapezZgled}
                    A line parallel to bases of a trapezium intersects its legs and diagonals in four points
                    and determine three line segments. Prove that two of them are congruent. \\
                    After that, construct a line parallel to the bases of that trapezium  such that all
                    three mentioned line segments are congruent.
                    \ezgled

\begin{figure}[!htb]
\centering
\input{sl.vek.5.8.11.pic}
\caption{} \label{sl.vek.5.8.11.pic}
\end{figure}

\textbf{\textit{Proof.}}
Naj bo $l$ vzporednica nosilke osnovnice
$AB$ trapeza $ABCD$, ki seka
stranici $AD$ in $BC$ ter diagonali $AC$ in $BD$
po vrsti v točkah $M$, $Q$, $N$ in $P$ (Figure \ref{sl.vek.5.8.11.pic}). Po Talesovem izreku \ref{TalesovIzrekDolzine} velja:
$MN:DC=MA:DA$ oz.
$PQ:DC=BQ:BC$. Ker je po
posledici \ref{TalesPosl1} Talesovega izreka $MA:AD=BQ:BC$, je tudi
$MN:DC=PQ:DC$ oz. $MN\cong PQ$.

Če je $E$ središče osnovnice $AB$ in $N_0$ presečišče premic $DE$ in $AC$, iskana
premica $l_0$  poteka skozi točko $N_0$ in je vzporedna s premico $AB$. Če so $M_0$, $P_0$ in $Q_0$ presečišča premice $l_0$ s premicami $AD$,
$BD$ in $CB$, potem po posledici \ref{TalesPosl3} Talesovega
izreka velja $M_0N_0\cong N_0P_0$.
\kdokaz


                    \bzgled If $r$ is the inradius and $r_a$, $r_b$ and $r_c$ exradii
                    of an arbitrary triangle, then $$\frac{1}{r_a}
                    +\frac{1}{r_b} +\frac{1}{r_c}= \frac{1}{r}.$$
                    \ezgled


\begin{figure}[!htb]
\centering
\input{sl.vek.5.8.12.pic}
\caption{} \label{sl.vek.5.8.12.pic}
\end{figure}

\textbf{\textit{Proof.}} Če uporabimo oznake iz velike naloge
\ref{velikaNaloga} (Figure \ref{sl.vek.5.8.12.pic}), iz Talesovega izreka \ref{TalesovIzrek} sledi:
$$ \frac{r}{r_a} = \frac{SQ}{S_aQ_a} =  \frac{AQ}{AQ_a}  =  \frac{s-a}{s}.$$
 Analogno je tudi:
 $$ \frac{r}{r_b} =  \frac{s-b}{s}\hspace*{2mm} \textrm{ in }\hspace*{2mm}
 \frac{r}{r_c} =  \frac{s-c}{s}.$$
 Po seštevanju treh enakosti dobimo iskano relacijo.
  \kdokaz



                \bizrek \label{velNalTockP'}
                Suppose that the incircle and the excircle of a triangle $ABC$ touch its side $BC$ in  points $P$ and $P_a$.
                If $PP'$ is a diameter of the incircle $k(S,r)$ ($P'\in k$), then $Pa$, $P'$ and $A$ are collinear
                points.
                \eizrek



\begin{figure}[!htb]
\centering
\input{sl.vek.5.8.13.pic}
\caption{} \label{sl.vek.5.8.13.pic}
\end{figure}


 \textbf{\textit{Proof.}} Uporabimo oznake iz velike naloge
\ref{velikaNaloga} (Figure \ref{sl.vek.5.8.13.pic}). Ker je daljica $PP'$
  premer včrtane krožnice, je točka $S$ njeno središče.

 Naj bo $\widehat{P'}$ presečišče poltraka $PS$ s premico $AP_a$.
 Dokažimo, da
  je $\widehat{P'}=P'$, oziroma da
velja $S\widehat{P'}=r$. Iz Talesovega izreka sledi:
 $$\frac{S\widehat{P'}}{S_aP_a} = \frac{AS}{AS_a} =\frac{SQ}{S_aQ_a}.$$
 Ker je $S_a P_a  = S_aQ_a  = r_a$,
 je tudi $S\widehat{P'}=SQ=r$.
 \kdokaz

Na podoben način kot \ref{velNalTockP'} se dokažeta tudi
naslednja izreka.




                \bizrek \label{velNalTockP'1}
                 Suppose that the incircle and the excircle of a triangle $ABC$ touch its side $BC$ in  points $P$ and $P_a$.
                If $P_aP_a'$ is a diameter of the excircle $k_a(S_a,r_a)$ ($P_a'\in k_a$), then $P_a'$, $P$ and $A$ are collinear
                points
                  (Figure \ref{sl.vek.5.8.14.pic}).
                \eizrek




\begin{figure}[htp]
\centering
\input{sl.vek.5.8.14.pic}
\caption{} \label{sl.vek.5.8.14.pic}
\end{figure}

%\vspace*{10mm}



             \bizrek \label{velNalTockP'2}
              Suppose that the excircles $k_b(S_b,r_b)$ and $k_c(S_c,r_c)$ of a triangle $ABC$
                touch the line $BC$ in  points $P_b$ and $P_c$.
                If $P_bP_b'$ and $P_cP_c'$ are diameters of the excircles
                 $k_b$ ($P_b'\in k_b$) and $k_c$ ($P_c'\in k_c$), then $P_c'$, $P_b$ and $A$ are collinear
                points and also $P_b'$, $P_c$ and $A$ are collinear
                points (Figure \ref{sl.vek.5.8.15.pic}).
             \eizrek


\begin{figure}[htp]
\centering
\input{sl.vek.5.8.15.pic}
\caption{} \label{sl.vek.5.8.15.pic}
\end{figure}






Zadnje tri trditve lahko uporabimo pri konstrukcijah trikotnikov,
kar bomo ilustrirali v naslednji  nalogi.

                \bzgled
                Construct a triangle $ABC$, with given $r$, $b-c$, $t_a$.
                 \ezgled


\begin{figure}[htp]
\centering
\input{sl.vek.5.8.16.pic}
\caption{} \label{sl.vek.5.8.16.pic}
\end{figure}

 \textbf{\textit{Solution.}}
 Naj bo $ABC$ iskani trikotnik oziroma trikotnik, pri katerem je
polmer včrtane krožnice $r$, ustrezna težiščnica $AA_1$ skladna s
$t_a$ in razlika stranic $AC$ in $AB$ enaka $b-c$. Uporabimo
oznake iz velike naloge \ref{velikaNaloga} in izreka
\ref{velNalTockP'}  (Figure \ref{sl.vek.5.8.16.pic}). Vemo, da velja  $PP_a=b-c$, točka $A_1$ pa je
skupno središče stranice $BC$ in daljice $PP_a$. Iz izreka
\ref{velNalTockP'} sledi, da so točke $P_a$, $P'$ in  $A$
kolinearne.

Torej najprej  konstruiramo pravokotni trikotnik $P'PP_a$, nato pa
 krožnico $k$ s premerom $PP'$ oz. včrtano krožnico
 trikotnika $ABC$ in točko $A_1$ kot
središče daljice $PP_a$. Oglišče $A$ je presečišče poltraka
$P_aP'$ s krožnico $k_1(A_1,t_a)$,  oglišči $B$ in $C$ pa sta
presečišči premice $BC$ s tangentama krožnice $k$ iz točke
$A$.
 \kdokaz

            \bzgled
             Let $A_1$ be the midpoint of the line segment $BC$, $k(S,r)$
            the incircle and $AA'$ the altitude of a triangle $ABC$.
            Suppose that $L$ is the intersection of the lines $A_1S$ and $AA'$.
            Prove that $AL\cong
             r$.
            \ezgled


\begin{figure}[htp]
\centering
\input{sl.vek.5.8.3.pic}
\caption{} \label{sl.vek.5.8.3.pic}
\end{figure}


\textbf{\textit{Solution.}} Označimo s $P$, $Q$ in $R$
dotikališča krožnice $k$ s
 stranicami $BC$, $AC$ in $AB$.
 Naj bo $k(S_a,r_a)$ pričrtana krožnica, ki se dotika stranice
 $BC$ v točki $P_a$, nosilk $AB$ in $AC$ pa v točkah $R_a$ in
 $Q_a$. Označimo še s $P'$ presečišče premic $AP_a$ in $PS$
 (Figure \ref{sl.vek.5.8.3.pic}).
 Iz trditve \ref{velNalTockP'} in velike naloge \ref{velikaNaloga}
 sledi:
\begin{itemize}
  \item točka $P'$ leži na krožnici $k$,
  \item točka $A_1$ je središče daljice $PP_a$.
\end{itemize}
Dokažimo, da velja $AL\cong
 r$. Daljica $SA_1$ je srednjica trikotnika
$PP'P_a$, zato je $SA_1\parallel P'P_a$ oz. $LS\parallel AP'$.
Ker je še $AL\parallel P'S$, je štirikotnik $AP'SL$ paralelogram
in velja $AL\cong P'S\cong r$.
 \kdokaz



%________________________________________________________________________________
\naloge{Exercises}

\begin{enumerate}

  %Vsota  in razlika vektorjev
    %_____________________________________

  \item Načrtaj poljubne vektorje $\overrightarrow{a}$, $\overrightarrow{b}$, $\overrightarrow{c}$ in $\overrightarrow{d}$ tako, da je njihova vsota enaka:

  (\textit{a}) enem od teh štirih vektorjev,\\
  (\textit{b}) razliki dveh od teh štirih vektorjev.

   \item Naj bo $ABCDE$ petkotnik v neki ravnini. Dokaži, da v tej ravnini obstaja
petkotnik s stranicami, ki določajo enake vektorje, kot jih določajo
 diagonale petkotnika $ABCDE$.

  \item Naj bodo $A$, $B$, $C$ in $D$ poljubne točke v ravnini. Ali splošno velja:

  (\textit{a}) $\overrightarrow{AB}+\overrightarrow{BD}=\overrightarrow{AD}+\overrightarrow{BC}$?\\
  (\textit{b}) $\overrightarrow{AB}=\overrightarrow{DC}\hspace*{1mm}\Rightarrow \hspace*{1mm} \overrightarrow{AC}+\overrightarrow{BD}=2\overrightarrow{BC}$?

  \item Dana je daljica $AB$. Samo z ravnilom z možnostjo risanja vzporednic (konstrukcije v afini geometriji) načrtaj točko $C$, tako da bo:

    (\textit{a}) $\overrightarrow{AC}=-\overrightarrow{AB}$, \hspace*{3mm}
   (\textit{b}) $\overrightarrow{AC}=5\overrightarrow{AB}$, \hspace*{3mm}
   (\textit{c}) $\overrightarrow{AC}=-3\overrightarrow{AB}$.

   \item Naj bo $ABCD$ štirikotnik in $O$ poljubna točka v ravnini tega štirikotnika. Izrazi vektorje stranic in
diagonal tega štirikotnika z vektorji $\overrightarrow{a}=\overrightarrow{OA}$, $\overrightarrow{b}=\overrightarrow{OB}$, $\overrightarrow{c}=\overrightarrow{OC}$ in $\overrightarrow{d}=\overrightarrow{OD}$.

   \item Naj bo $ABCD$ štirikotnik in $O$ poljubna točka v ravnini tega štirikotnika. Ali velja ekvivalenca, da je štirikotnik $ABCD$ paralelogram natanko tedaj, ko je $\overrightarrow{OA}+\overrightarrow{OC}=
    \overrightarrow{OB}+\overrightarrow{OD}$?

 \item Naj bo $ABCD$ paralelogram, $S$
presečišče njegovih diagonal in $M$ poljubna točka v ravnini tega paralelograma. Dokaži, da velja:
        $$\overrightarrow{MS} = \frac{1}{4}\cdot
        \left( \overrightarrow{MA}+\overrightarrow{MB}
         +\overrightarrow{MC} + \overrightarrow{MD} \right).$$

 \item Naj bodo $ABB_1A_2$,
$BCC_1B_2$ in $CAA_1C_2$ paralelogrami, ki so načrtani nad stranicami trikotnika $ABC$. Dokaži, da velja:
$$\overrightarrow{A_1A_2}+\overrightarrow{B_1B_2}+
\overrightarrow{C_1C_2}=\overrightarrow{0}.$$

  \item Pravokotni premici $p$ in $q$, ki se sekata v točki $M$, sekata  krožnico $k$ s središčem $O$ v
točkah $A$, $B$, $C$ in $D$. Dokaži, da velja:
$$\overrightarrow{OA}+ \overrightarrow{OB} + \overrightarrow{OC} + \overrightarrow{OD} = 2\overrightarrow{OM}.$$

\item Naj bodo $A$, $B$, $C$ in $D$ poljubne točke v neki ravnini. Ali lahko vseh šest daljic, ki jih določajo te točke, orientiramo tako, da je vsota ustreznih šestih vektorjev enaka vektorju nič?

\item Naj bodo $A_1$, $B_1$ in $C_1$ središča stranic $BC$, $AC$ in $AB$ trikotnika $ABC$ ter $M$ poljubna točka. Dokaži:

    (\textit{a}) $\overrightarrow{AA_1}+\overrightarrow{BB_1}+
\overrightarrow{CC_1}=\overrightarrow{0}$,\\
   (\textit{b}) $\overrightarrow{MA}+\overrightarrow{MB}+\overrightarrow{MC}=
   \overrightarrow{MA_1}+\overrightarrow{MB_1}+\overrightarrow{MC_1}$,\\
   (\textit{c}) Obstaja tak trikotnik $PQR$, da za njegova oglišča velja:\\ \hspace*{7mm} $\overrightarrow{PQ}=\overrightarrow{CC_1}$, $\overrightarrow{PR}=\overrightarrow{BB_1}$ in $\overrightarrow{RQ}=\overrightarrow{AA_1}$.

 \item Naj bodo $M$, $N$, $P$, $Q$, $R$ in  $S$ po vrsti središča stranic poljubnega  šestkotnika.
Dokaži, da velja:
$$\overrightarrow{MN}+\overrightarrow{PQ}+
\overrightarrow{RS}=\overrightarrow{0}.$$


  %Linearna kombinacija vektorjev
    %_____________________________________

  \item Naj bo $ABCDEF$ konveksni šestkotnik, pri katerem je $AB\parallel DE$, točki $K$ in $L$ pa sta središči daljic, ki jih
določajo središča preostalih parov nasprotnih stranic. Dokaži, da je $K=L$ natanko tedaj, ko je $AB\cong DE$.

 \item Naj bosta $P$ in $Q$ takšni točki stranic $BC$ in $CD$ paralelograma $ABCD$, da je $BP:PC=2:3$ in
$CQ:QD=2:5$. Točka $X$ je presečišče daljic $AP$ in $BQ$. Izračunaj razmerji, v katerih točka $X$ deli
daljici $AP$ in $BQ$.

\item Naj bodo $A$, $B$, $C$ in $D$ poljubne točke neke ravnine. Točka $E$ je središče daljice $AB$, $F$ in
$G$ takšni točki, da velja $\overrightarrow{EF} = \overrightarrow{BC}$ in $\overrightarrow{EG} = \overrightarrow{AD}$,  ter $S$ središče daljice $CD$. Dokaži, da so $G$, $S$ in $F$
kolinearne točke.

\item Naj bosta $K$ in $L$  takšni točki stranice $AD$ in diagonale $AC$ paralelograma $ABCD$, da velja $\frac{\overrightarrow{AK}}{\overrightarrow{KD}}=\frac{1}{3}$ in
    $\frac{\overrightarrow{AL}}{\overrightarrow{LC}}=\frac{1}{4}$. Dokaži, da so $K$, $L$ in $B$ kolinearne točke.

\item Naj bosta $X_n$ in $Y_n$ ($n\in \mathbb{N}$) takšni točki stranic $AB$ in $AC$ trikotnika $ABC$, da velja $\overrightarrow{AX_n}=\frac{1}{n+1}\cdot \overrightarrow{AB}$ in
    $\overrightarrow{AY_n}=\frac{1}{n}\cdot \overrightarrow{AC}$. Dokaži, da obstaja točka, ki leži na vseh premicah
$X_nY_n$  ($n\in \mathbb{N}$).

 \item Naj bodo $M$, $N$, $P$ in $Q$ središča stranic $AB$, $BC$, $CD$ in $DA$ štirikotnika
     $ABCD$. Ali velja ekvivalenca, da je štirikotnik $ABCD$ paralelogram natanko tedaj, ko je:

    (\textit{a}) $2\overrightarrow{MP}=\overrightarrow{BC}+\overrightarrow{AD}$ in
    $2\overrightarrow{NQ}=\overrightarrow{BA}+\overrightarrow{CD}$?\\
   (\textit{b}) $2\overrightarrow{MP}+2\overrightarrow{NQ}=
   \overrightarrow{AB}+\overrightarrow{BC}+\overrightarrow{CD}+\overrightarrow{DA}$?

 \item Naj bodo $E$, $F$ in $G$ središča stranic $AB$, $BC$ in $CD$ paralelograma $ABCD $, premici $BG$ in $DE$ pa sekata premico $AF$ v točkah $N$ in $M$. Izrazi $\overrightarrow{AF}$, $\overrightarrow{AM}$ in $\overrightarrow{AN}$ kot linearno kombinacijo vektorjev $\overrightarrow{AB}$ in $\overrightarrow{AD}$. Dokaži, da točki $M$ in $N$ delita daljico $AF$ v razmerju $2:2:1$.

 \item Točke $K$, $L$, $M$ in $N$ ležijo na stranicah $AB$, $BC$, $CD$ in $DA$
štirikotnika $ABCD$. Če je štirikotnik $KLMN$ paralelogram in velja
$$\frac{\overrightarrow{AK}}{\overrightarrow{KB}}=
\frac{\overrightarrow{BL}}{\overrightarrow{LC}}
=\frac{\overrightarrow{CM}}{\overrightarrow{MD}}=
\frac{\overrightarrow{DN}}{\overrightarrow{NA}}=\lambda$$ za nek $\lambda\neq\pm 1$,
 je tudi štirikotnik $ABCD$
paralelogram. Dokaži.

\item Naj bo $M$ središče stranice $DE$ pravilnega šestkotnika $ABCDEF$. Točka
$N$ je središče daljice $AM$, točka $P$ pa središče stranice $BC$. Izrazi $\overrightarrow{NP}$ kot linearno kombinacijo vektorjev
$\overrightarrow{AB}$ in $\overrightarrow{AF}$.


  %Dolžina vektorja
    %_____________________________________

  \item Dokaži, da za poljubne točke $A$, $B$ in $C$ velja:

    (\textit{a}) $|\overrightarrow{AC}|\leq|\overrightarrow{AB}|+|\overrightarrow{BC}|$ \hspace*{6mm}
   (\textit{b}) $|\overrightarrow{AC}|\geq|\overrightarrow{AB}|-|\overrightarrow{BC}|$\\
  Pod katerimi pogoji velja enakost?

  \item Naj bosta $M$ in $N$ takšni točki, ki ležita na daljicah $AD$ oz. $BC$, tako da velja $\frac{\overrightarrow{AM}}{\overrightarrow{MD}}\cdot \frac{\overrightarrow{CN}}{\overrightarrow{NB}}=1$. Dokaži, da je:
      $$|MN|\leq\max\{|AB|, |CD|\}.$$

  %Težišče
    %_____________________________________

  \item Točki $T$ in $T'$ sta težišči $n$-kotnikov $A_1A_2...A_n$ in $A'_1A'_2...A'_n$. Izračunaj:
$$\overrightarrow{A_1A'_1}+\overrightarrow{A_2A'_2}+\cdots+\overrightarrow{A_nA'_n}.$$

    \item Dokaži, da imata štirikotnika $ABCD$ in  $A'B'C'D'$ skupno težišče natanko tedaj, ko je:
$$\overrightarrow{AA'}+\overrightarrow{BB'}+\overrightarrow{CC'}+
\overrightarrow{DD'}=\overrightarrow{0}.$$

    \item Naj bodo $P$, $Q$, $R$ in $S$ težišča trikotnikov $ABD$, $BCA$, $CDB$ in $DAC$. Dokaži, da imata
štirikotnika $PQRS$ in $ABCD$ skupno težišče.

  \item Naj bo $A_1A_2A_3A_4A_5A_6$ poljubni šestkotnik in $B_1$, $B_2$, $B_3$, $B_4$, $B_5$ in $B_6$ po vrsti težišča trikotnikov $A_1A_2A_3$, $A_2A_3A_4$, $A_3A_4A_5$, $A_4A_5A_6$, $A_5A_6A_1$ in $A_6A_1A_2$.
Dokaži, da ta težišča določajo šestkotnik s tremi pari vzporednih stranic.

 \item Naj bodo $A$, $B$, $C$ in $D$ štiri različne točke. Točke $T_A$, $T_B$, $T_C$ in $T_D$
        so težišča trikotnikov $BCD$, $ACD$, $ABD$ in $ABC$. Dokaži, da se daljice $AT_A$, $BT_B$, $CT_C$ in $DT_D$ sekajo v eni točki $T$. V katerem razmerju točka $T$ deli te daljice?

 \item Naj bo $CC_1$ težiščnica trikotnika $ABC$ in $P$ poljubna točka na stranici
$AB$ tega trikotnika. Vzporednica $l$ premice $CC_1$ skozi točko $P$ seka premici $AC$ in $BC$ v točkah $M$ in $N$. Dokaži, da velja:
$$\overrightarrow{PM} + \overrightarrow{PN}= \overrightarrow{AC} + \overrightarrow{BC}.$$


 %Hamilton in Euler
 %______________________________________________________


 \item Naj bodo $A$, $B$, $C$ točke neke ravnine, ki ležijo na isti strani premice $p$, in $O$ točka na
premici $p$, za katero velja  $|\overrightarrow{OA}| = |\overrightarrow{OB}| = |\overrightarrow{OC}| =1$. Dokaži, da potem velja tudi: $$|\overrightarrow{OA} + \overrightarrow{OB} + \overrightarrow{OC}| \geq 1.$$

\item Izračunaj kote, ki jih določajo vektorji $\overrightarrow{OA}$, $\overrightarrow{OB}$ in $\overrightarrow{OC}$, če točke $A$, $B$ in $C$ ležijo na krožnici s središčem
$O$ in dodatno velja še:
$$\overrightarrow{OA} + \overrightarrow{OB} + \overrightarrow{OC} = \overrightarrow{0}.$$

 \item Naj bodo $A$, $B$, $C$ in $D$ točke, ki ležijo na krožnici s središčem
$O$, in velja
$$\overrightarrow{OA} + \overrightarrow{OB} + \overrightarrow{OC} + \overrightarrow{OD} = \overrightarrow{0}.$$
Dokaži, da je $ABCD$ pravokotnik.

\item Naj bodo $\overrightarrow{a}$, $\overrightarrow{b}$ in $\overrightarrow{c}$ vektorji neke ravnine, za katere velja $|\overrightarrow{a}| = |\overrightarrow{b}| = |\overrightarrow{c}| =x$. Razišči, v katerem primeru velja tudi $|\overrightarrow{a} + \overrightarrow{b} + \overrightarrow{c}| = x$.


 %Tales
 %______________________________________________________


  \item Dano daljico $AB$ razdeli:

  (\textit{a}) na pet enakih daljic,\\
  (\textit{b}) v razmerju $2:5$,\\
  (\textit{c})na tri daljice, ki so v razmerju $2:\frac{1}{2}:1$.


  \item Dana je daljica $AB$. Samo z uporabo ravnila z možnostjo risanja vzporednic (konstrukcije v afini geometriji) načrtaj točko $C$, če je:

(\textit{a}) $\overrightarrow{AC}=\frac{1}{3}\overrightarrow{AB}$  \hspace*{3mm}
   (\textit{b}) $\overrightarrow{AC}=\frac{3}{5}\overrightarrow{AB}$  \hspace*{3mm}
   (\textit{c}) $\overrightarrow{AC}=-\frac{4}{7}\overrightarrow{AB}$

  \item Dane so daljice $a$, $b$ in $c$. Načrtaj daljico $x$, tako da bo:

     (\textit{a}) $a:b=c:x$ \hspace*{3mm}
    (\textit{b}) $x=\frac{a\cdot b}{c}$ \hspace*{3mm}
   (\textit{c}) $x=\frac{a^2}{c}$\hspace*{3mm}\\
   (\textit{č}) $x=\frac{2ab}{3c}$\hspace*{3mm}
   (\textit{č}) $(x+c):(x-c)=7:2$

 \item Naj bosta $M$ in $N$ točki na kraku $OX$,  $P$ točka na kraku $OY$ kota
$XOY$ ter $NQ\parallel MP$ in $PN\parallel QS$ ($Q\in OY$, $S\in OX$). Dokaži, da je
$|ON|^2=|OM|\cdot |OS|$ (za daljico $ON$ v tem primeru pravimo, da je \index{geometrijska sredina daljic}\pojem{geometrijska sredina} daljic $OM$
in $OS$).


\item Naj bo  $ABC$ trikotnik in $Q$, $K$, $L$, $M$, $N$ in $P$ takšne točke poltrakov $AB$, $AC$, $BC$,
$BA$, $CA$ in $CB$, da velja $AQ\cong CP\cong AC$, $AK\cong BL\cong AB$ in $BM\cong CN\cong BC$.
Dokaži, da so $MN$, $PQ$ in $LK$ tri vzporednice.

\item Naj bo $P$ središče težiščnice $AA_1$ trikotnika $ABC$. Točka $Q$ je presečišče premice $BP$
s stranico $AC$. Izračunaj razmerji $AQ:QC$ in $BP:PQ$.

\item Točki $P$ in $Q$ ležita na stranicah $AB$ in $AC$ trikotnika $ABC$, pri tem pa velja $\frac{|\overrightarrow{PB}|}{|\overrightarrow{AP}|}
    +\frac{|\overrightarrow{QC}|}{|\overrightarrow{AQ}|}=1$. Dokaži, da težišče tega trikotnika leži na daljici $PQ$.

 \item Naj bodo $a$, $b$ in $c$ trije poltraki s skupnim izhodiščem $S$ ter $M$
točka na poltraku $a$. Če se točka $M$ “giblje” po poltraku $a$,
 je razmerje razdalj te točke od premic $b$ in $c$ konstantno. Dokaži.

 \item Naj bo $D$ točka, ki leži na stranici $BC$ trikotnika $ABC$ ter
 $F$ in $G$ točki, v katerih premica, ki poteka skozi točko $D$ in je vzporedna  s težiščnico $AA_1$, seka premici $AB$ in $AC$.
Dokaži, da je vsota $|DF|+|DG|$ konstantna, če se točka $D$ “giblje” po stranici
$BC$.


   \item
   Načrtaj trikotnik s podatki:

   (\textit{a}) $v_a$, $r$, $b-c$  \hspace*{3mm}
   (\textit{b}) $\beta$, $r$, $b-c$


\end{enumerate}






% DEL 6 - - - - - - - - - - - - - - - - - - - - - - - - - - - - - - - - - - - - - - -
%________________________________________________________________________________
% IZOMETRIJE
%________________________________________________________________________________

  \del{Isometries} \label{pogIZO}


%________________________________________________________________________________
\poglavje{Isometries. Identity Map}  \label{odd6Ident}

Izometrije oz. izometrijske transformacije ravnine
$\mathcal{I}:\mathbb{E}^2\rightarrow \mathbb{E}^2$ smo formalno že
definirali v razdelku \ref{odd2AKSSKL} kot transformacije ravnine, ki
ohranjajo relacijo skladnosti parov točk. Kasneje smo jih uporabili
za definiranje relacije skladnosti likov. Intuitivno jih zaznamujemo
kot gibanja ravnine. Nekatere so nam znane že od prej (v tej
knjigi jih še nismo formalno vpeljali) – rotacija in
translacija (Figure \ref{sl.izo.6.1.1.pic}). Tudi zrcaljenje čez
premico je izometrija. Toda ta se
 razlikuje od omenjenih dveh izometrij, ker ni pravo gibanje (v
ravnini). Da bi jo videli kot gibanje, je potrebno
preiti v prostor, kjer lahko ravnino zavrtimo za $180^0$. S tem
spremenimo orientacijo ravnine.

\begin{figure}[!htb]
\centering
\input{sl.izo.6.1.1.pic}
\caption{} \label{sl.izo.6.1.1.pic}
\end{figure}

Za takšne izometrije, ki spremenijo orientacijo ravnine, pravimo,
da so \index{izometrija!indirektna} \pojem{indirektne}.
Zrcaljenje čez premico je torej indirektna izometrija. Za tiste
izometrije, ki ohranjajo orientacijo ravnine, pravimo, da so
\index{izometrija!direktna}\pojem{direktne}. Rotacija in
translacija sta primera direktnih izometrij. Na tem mestu ne bomo
dokazovali dejstva, da je vsaka izometrija ravnine bodisi direktna
bodisi indirektna, oz. da če neka izometrija ohranja orientacijo
enega lika, ohranja orientacijo tudi vseh drugih likov.

 Jasno je, da  kompozitum dveh direknih ali dveh indirektnih
 izometrij predstavlja direktno izometrijo. Prav tako je
 kompozitum ene direktne in ene idirektne izometrije indirektna
 izometrija.

Intuitivno lahko pri direktnih izometrijah lik in njegovo sliko z
gibanjem v ravnini pripeljemo do prekrivanja. Za indirektne
izometrije to ni mogoče narediti s prostim gibanjem v ravnini -
potrebno je uporabiti gibanje v trirazsežnemu prostoru.

Razen pogoja direktnosti in indirektnosti izometrij je njihova zelo
pomembna lastnost tudi število fiksnih točk. Ponovimo, da za točko
pravimo, da je fiksna točka pri neki izometriji, če se s to
izometrijo preslika sama vase (razdelek \ref{odd2AKSSKL}). Intuitivno ima
rotacija natanko eno fiksno točko – svoje središče. Translacija
nima fiksnih točk. Zrcaljenje čez premico jih ima neskončno mnogo, vendar
so vse fiksne točke na osi tega zrcaljenja.
 Ali je možno, da ima izometrija tri fiksne nekolinearne točke? Intuitivno je
 jasno dejstvo, ki smo ga tudi formalno dokazali (izrek  \ref{IizrekABCident}), da
 so za takšno izometrijo tudi vse ostale točke v ravnini fiksne.
 Takšno izometrijo smo
 imenovali \index{izometrija!identična} identična izometrija ali
 identiteta in jo označili z $\mathcal{E}$. Očitno je tudi identiteta
  direktna izometrija, saj ohranja vse like.
  Zaradi pomembnosti
  bomo že dokazani izrek \ref{IizrekABCident} še enkrat zapisali v nekoliko spremenjeni
  obliki.





                \bizrek \label{IizrekABC2}
                The identity map $\mathcal{E}$ is the only
                isometry of a plane having three non-collinear fixed points.
                 \eizrek

 Z naslednjim izrekom  bomo podali še en kriterij za identitete.



               \bizrek \label{izo2ftIdent}
              A direct isometry of a plane that has at least
              two fixed points is the identity map.
              \eizrek

\begin{figure}[!htb]
\centering
\input{sl.izo.6.1.2.pic}
\caption{} \label{sl.izo.6.1.2.pic}
\end{figure}

\textbf{\textit{Proof.}}
  Predpostavimo, da ima direktna izometrija
ravnine $\mathcal{I}$ vsaj dve fiksni točki $A$ in $B$ oz.
$A'=\mathcal{I}(A)=A$ in $B'=\mathcal{I}(B)=B$ (Figure
\ref{sl.izo.6.1.2.pic}). Naj bo $X$ poljubna točka, ki ne leži na
premici $AB$. Naj bo $X'=\mathcal{I}(X)$. Ker je $\mathcal{I}$
direktna izometrija, sta trikotnika $AXB$ in $AX'B$ enako
orientirana, zato točka $X'$ leži v polravnini z robom $AB$ in
s točko $X$. Iz $\mathcal{I}:A,B,X\mapsto A,B,X'$ sledi
 $XA\cong X'A$, $XB\cong X'B$ in $AB\cong AB$. Ker je
tudi $XA\cong XA$ in $XB\cong XB$,  je po izreku
 \ref{izomEnaC'} $X=X'$. Torej je tudi  $X$ fiksna točka. Ker so $A$, $B$ in $X$ tri
 nekolinearne točke izometrije  $\mathcal{I}$ je $\mathcal{I}=\mathcal{E}$
 (izrek \ref{IizrekABC2}).
 \kdokaz

V nadaljevanju bomo formalno vpeljali in obravnavali različne
vrste izometrij.

%________________________________________________________________________________
 \poglavje{Reflections} \label{odd6OsnZrc}

Čeprav nam je osno zrcaljenje  intuitivno že znana preslikava, bomo
najprej podali formalno definicijo.

Naj bo $s$ premica neke ravnine. Transformacija te ravnine, pri
kateri je vsaka točka premice $s$ fiksna in pri kateri se vsaka
točka $X$, ki ne leži na premici $s$, slika v takšno točko $X'$,
 da je  $s$ simetrala daljice $XX'$, imenujemo
  \index{zrcaljenje!čez premico} \pojem{osno zrcaljenje}
  ali \index{zrcaljenje!osno}\pojem{zrcaljenje čez premico}
  $s$ in jo označimo s $\mathcal{S}_s$ (Figure \ref{sl.izo.6.2.1.pic}).
   Premica $s$ pa je
 \index{os!zrcaljenja} \pojem{os zrcaljenja}.

 Ker je iz oznake $\mathcal{S}_s$
 že jasno, da gre za zrcaljenje čez premico $s$, ga bomo imenovali
 tudi krajše: zrcaljenje $\mathcal{S}_s$.

\begin{figure}[!htb]
\centering
\input{sl.izo.6.2.1.pic}
\caption{} \label{sl.izo.6.2.1.pic}
\end{figure}

 Če se lik $\phi$ z zrcaljenjem $\mathcal{S}_s$ preslika v lik
 $\phi'$ oz.  $\mathcal{S}_s: \phi \rightarrow \phi'$, bomo rekli,
 da sta lika $\phi$ in $\phi'$
 \pojem{simetrična} glede na os $s$ (Figure
 \ref{sl.izo.6.2.2.pic}). Os $s$ pa je \pojem{os simetrije} likov $\phi$ in
 $\phi'$.

  Če je še $\phi=\phi'$ oz. $\mathcal{S}_s: \phi \rightarrow
 \phi$, pravimo, da je lik $\phi$ \index{lik!osnosimetričen}
 \pojem{osno simetričen} oz. \pojem{osno someren}. Premica $s$ je
 \index{os!simetrije lika}\pojem{os simetrije} ali \index{somernica}
 \pojem{somernica} tega lika (Figure \ref{sl.izo.6.2.2.pic}).

\begin{figure}[!htb]
\centering
\input{sl.izo.6.2.2.pic}
\caption{} \label{sl.izo.6.2.2.pic}
\end{figure}

 Iz definicije je že jasno, da vse fiksne točke zrcaljenja $\mathcal{S}_s$
 ležijo na osi $s$ tega zrcaljenja.

 Čeprav je intuitivno jasna, je potrebno dokazati naslednjo  lastnost
 definirane preslikave.



    \bizrek \label{izozrIndIzo}
    A reflection is an opposite isometry.
    \eizrek

\begin{figure}[!htb]
\centering
\input{sl.izo.6.2.3.pic}
\caption{} \label{sl.izo.6.2.3.pic}
\end{figure}


\textbf{\textit{Proof.}}
  Naj bo $\mathcal{S}_s$ zrcaljenje čez premico $s$ (Figure
\ref{sl.izo.6.2.3.pic}). Iz definicije je jasno, da predstavlja bijektivno preslikavo. Potrebno je še
  dokazati, da za poljubni točki
   $X$ in $Y$, ki ju $\mathcal{S}_s$ preslika v točki  $X'$ in $Y'$, velja
   $XY\cong X'Y'$. Obravnavali bomo več primerov.

   Če je $X=X'$ in $Y=Y'$, je ta relacija
    avtomatično izpolnjena (izrek \ref{sklRelEkv}).

   Naj bo $X\neq
   X'$. Označimo z $X_s$ središče daljice $XX'$. Ker je $s$ simetrala
   daljice $XX'$, je po definiciji $X_s\in s$ in $XX'\perp s$.
   Če je $Y=Y'$, sta trikotnika $XX_sY$ in $X'X_sY$ (oz. $X'X_sY'$)
    skladna (izrek
   \textit{SAS} \ref{SKS}), zato je tudi $XY\cong X'Y'$.

   Ostane nam še primer $X\neq X'$ in $Y\neq Y'$. Podobno za
   središče
   $Y_s$ daljice $YY'$ velja $Y_s\in s$ in $YY'\perp s$.
   Trikotnika $XX_sY_s$ in $X'X_sY_s$ sta
    skladna (izrek
   \textit{SAS}, zato obstaja takšna izometrija $\mathcal{I}$, da
   velja  $\mathcal{I}: X, X_s,Y_s\mapsto X', X_s,Y_s$. Le-ta
   preslika polravnino $X_sY_sX$ v polravnino $X_sY_sX'$
   (aksiom \ref{aksIII2}). Ker  izometrija $\mathcal{I}$ pravi kot
   preslika v pravi kot, se z njo poltrak $Y_sY$ preslika $Y_sY'$,
   točka $Y$ pa v točko $Y'$. Iz
   $\mathcal{I}: X, Y\mapsto X', Y'$ na koncu sledi $XY\cong X'Y'$.

    Naj bo $A,B\in s$ in $C\notin s$. Potem je $\mathcal{S}_s(A)=A$, $\mathcal{S}_s(B)=A$ in $\mathcal{S}_s(C)=C'\neq C$. To pomeni, da se trikotnik $ABC$ z osnim zrcaljenjem $\mathcal{S}_s$ preslika v trikotnik $ABC'$. Ker je $C,C' \div s$ oz. $C,C' \div AB$, sta omenjena trikotnika različno orientirana, zato je osno zrcaljenje $\mathcal{S}_s$ indirektna izometrija.
    \kdokaz

Jasno je, da v dokazu prejšnjega izreka velja tudi
$\mathcal{I}=\mathcal{S}_s$. To lahko ugotovimo šele na koncu, ko
dokažemo, da je $\mathcal{S}_s$ izometrija in uporabimo izrek
\ref{IizrekABC}.

Dokažimo še nekaj enostavnih lastnosti zrcaljenja čez premico.


    \bizrek \label{izoZrcPrInvol}
     For an arbitrary line $p$ is $$\mathcal{S}_p^2=\mathcal{E}\hspace*{1mm}\textrm{ i.e. }
    \hspace*{1mm}\mathcal{S}_p^{-1}=\mathcal{S}_p.$$
    \eizrek


\textbf{\textit{Proof.}} Dovolj je dokazati, da velja
$\mathcal{S}_p^2(X)=X$ za vsako točko $X$ ravnine. Naj bo
$\mathcal{S}_p(X)=X'$. Če je $X\in p$ oz. $X=X'$, je relacija
$\mathcal{S}_p^2(X)=X$ avtomatično izpolnjena. Če $X\notin p$, je
po definiciji premica $p$ simetrala daljice $XX'$ (tudi $X'X$),
zato je $\mathcal{S}_p(X')=X$ oz. $\mathcal{S}_p^2(X)=X$.
 \kdokaz

    Vsako izometrijo (tudi vsako preslikavo) $f:\mathbb{E}^2\rightarrow \mathbb{E}^2$,
    za katero velja $f^2=\mathcal{E}$, imenujemo
     \pojem{involucija}\index{involucija}.
     Osno zrcaljenje je torej involucija.



        \bizrek \label{zrcFiksKroz}
         Let $l$ be  an arbitrary circle
        with the centre $S$ and $p$ a line in the plane.
         If $S\in p$, then $\mathcal{S}_p(l)=l$.
        \eizrek


\begin{figure}[!htb]
\centering
\input{sl.izo.6.2.4.pic}
\caption{} \label{sl.izo.6.2.4.pic}
\end{figure}


\textbf{\textit{Proof.}} Naj bo $X\in l$ poljubna točka krožnice
$l$ in $\mathcal{S}_p(X)=X'$ (Figure \ref{sl.izo.6.2.4.pic}). Ker
$\mathcal{S}_p:S,X\mapsto S, X'$, je $SX\cong SX'$ oz. $X'\in l$.
Torej je $\mathcal{S}_p(l)\subseteq l$. Podobno je poljubna točka
$Y$ krožnice $l$ slika točke
$Y'=\mathcal{S}_p^{-1}(Y)=\mathcal{S}_p(Y)$, ki leži na tej
krožnici. Zato je tudi $\mathcal{S}_p(l)\supseteq l$. Torej velja
$\mathcal{S}_p(l)=l$.
 \kdokaz

        \bzgled \label{izoSimVekt}
        If
        $\mathcal{S}_s:A, B\mapsto A', B'$, the vector
        $\overrightarrow{v}=\overrightarrow{AB}+\overrightarrow{A'B'}$
        is parallel to the line $s$.
        \ezgled

\begin{figure}[!htb]
\centering
\input{sl.izo.6.2.5.pic}
\caption{} \label{sl.izo.6.2.5.pic}
\end{figure}


\textbf{\textit{Proof.}} Naj bosta $A_s$ in $B_s$ središči daljic
$AA'$ in $BB'$ (Figure \ref{sl.izo.6.2.5.pic}). Ker je $p$
simetrala teh daljic, je $A_s, B_s\in p$. Zato je vektor
 \begin{eqnarray*}
 \overrightarrow{v}&=&\overrightarrow{AB}+\overrightarrow{A'B'}=\\
 &=&(\overrightarrow{AA_s}+\overrightarrow{A_sB_s}+\overrightarrow{B_sB})+
 (\overrightarrow{A'A_s}+\overrightarrow{A_sB_s}+\overrightarrow{B_sB'})=\\
 &=& 2\overrightarrow{A_sB_s}
 \end{eqnarray*}
vzporeden s premico $s$.
 \kdokaz

  Z lastnostmi osnega zrcaljenja bomo nadaljevali v naslednjem razdelku,
  sedaj pa si oglejmo uporabo te izometrije.



             \bzgled \label{HeronProbl}
             (Heron's\footnote{Ta problem je
            zastavil \index{Heron}\textit{Heron iz Aleksandrije} (20--100). V
            svojem delu ‘‘Catoprica’’ je formuliral zakon, ki pravi, da žarek, ki gre iz
            točke $A$ in se odbije od premice $p$ skozi točko $B$, prehaja
            najkrajšo možno pot.} problem) \index{problem!Heronov}
             Two points $A$ and $B$ are given on the same side of a line $p$.
             Find a point $X$ on the line $p$ such that the sum $|AX|+|XB|$ is minimal.
            \ezgled

\begin{figure}[!htb]
\centering
\input{sl.izo.6.2.6.pic}
\caption{} \label{sl.izo.6.2.6.pic}
\end{figure}


\textbf{\textit{Solution.}}
 Naj bo $A'$ slika točke $A$ pri
zrcaljenju $\mathcal{S}_p$ (Figure \ref{sl.izo.6.2.6.pic}). Z $X$
označimo presečišče premice $p$ in $A'B$ (točki $A'$ in $B$ sta na
različnih bregovih premice $p$). Dokažimo da je $X$ iskana točka.

Naj bo $Y\neq X$ poljubna točka premice $p$. Ker je zrcaljenje
$\mathcal{S}_p$ izometrija, ki daljici $AX$ in $AY$ slika v
daljici $A'X$ in $A'Y$, velja $AX\cong A'X$ in $AY\cong A'Y$. Če
uporabimo še trikotniško neenakost - izrek \ref{neenaktrik} (za
trikotnik $A'YB$), dobimo: $$|AX| + |XB| = |A'X| + |XB| = |A'B| <
|A'Y| + |YB| = |AY| + |YB|,$$ kar je bilo treba dokazati. \kdokaz



          \bzgled
           Let $k$ and $l$ be circles on the same side of a line $p$ in the same plane.
             Construct a point $S$ on the line $p$ such that
            the tangents from this point to the circles $k$ and $l$
            determine the congruent angles with the line $p$.
              \ezgled

\begin{figure}[!htb]
\centering
\input{sl.izo.6.2.7.pic}
\caption{} \label{sl.izo.6.2.7.pic}
\end{figure}


\textbf{\textit{Solution.}} Naj bosta $q$ in $r$ po vrsti tangenti
krožnic $k$ in $l$, ki se sekata na premici $p$ v točki $S$ in z
njo določata skladna kota (Figure \ref{sl.izo.6.2.7.pic}). Premica
$p$ je simetrala kota, ki ga tangenti $q$ in $r$ določata, zato je
$\mathcal{S}_p(r)=q$. Premica $q$ je torej tangenta tudi krožnice
$l'=\mathcal{S}_p(l)$. To pomeni, da lahko premico $q$ načrtamo
kot skupno tangento krožnic $k$ in $l'$ (glej zgled
\ref{tang2ehkroz}). Nato pa je še $S=q\cap p$ in $r=\mathcal{S}_p(q)$.
 \kdokaz


            \bzgled \label{FagnanLema}
            Let $AP$, $BQ$ and $CR$ be the altitudes of a triangle $ABC$.
            If $P'=\mathcal{S}_{AB}(P)$ and
             $P''=\mathcal{S}_{AC}(P)$, prove that $P'$, $R$, $Q$ and $P''$
             are collinear points.
             \ezgled

\begin{figure}[!htb]
\centering
\input{sl.izo.6.2.9a.pic}
\caption{} \label{sl.izo.6.2.9a.pic}
\end{figure}


\textbf{\textit{Solution.}} (Figure \ref{sl.izo.6.2.9a.pic}) Najprej
iz $\angle BRC=90^0$ in $\angle BQC=90^0$ po Talesovem izreku
\ref{TalesovIzrKroz} sledi, da točki $R$ in $Q$ ležita na krožnici
s premerom $BC$. Torej je $BRQC$ tetivni štirikotnik in  po
izreku \ref{TetivniPogojZunanji} velja $\angle ARQ\cong \angle BCQ =
\gamma$. Analogno je (iz tetivnosti štirikotnika $CARP$) tudi
$\angle BRP\cong\gamma$, zato je $\angle ARQ\cong \angle BRP$. Ker
velja $\mathcal{S}_{AB}:P, R, B\mapsto P', R, B$, je $\angle BRP\cong
\angle BRP'$. Iz prejšnjih relacij sledi $\angle ARQ\cong \angle
BRP'$. Trikotnik $ABC$ je ostrokotni, kar pomeni, da sta točki $P$
in $Q$ na istem bregu premice $AB$. Iz tega sledi, da sta točki
$Q$ in $P'$ na različnih straneh premice $AB$. Iz dokazane
relacije $\angle ARQ\cong \angle BRP'$ sedaj sledi, da so $P'$,
$R$ in $Q$ kolinearne točke. Analogno so tudi točke $P''$, $R$ in
$Q$ kolinearne, kar pomeni, da vse štiri točke $P'$, $R$, $Q$ in
$P''$ ležijo na isti premici.
 \kdokaz



             \bizrek \label{FagnanLema1}
            Let $A$ be a point not lying on lines $p$ and $q$ in the same plane.
             Construct points $B\in p$ and $C\in q$ such that
             the perimeter of the triangle $ABC$ is minimal.
             \eizrek

\begin{figure}[!htb]
\centering
\input{sl.izo.6.2.8.pic}
\caption{} \label{sl.izo.6.2.8.pic}
\end{figure}


\textbf{\textit{Solution.}}
 Naj bo $A'=\mathcal{S}_p(A)$, $A''=\mathcal{S}_q(A)$ ter $B$ in $C$
  presecišči premice $A'A''$ s premicama $p$ in $q$
   (Figure \ref{sl.izo.6.2.8.pic}).

Dokažimo, da je $ABC$ iskani trikotnik. Če sta $B_1$ in $C_1$
poljubni točki (kjer je $B_1\neq B$ ali $C_1\neq C$), ki ležita na
premicah $p$ in $q$, je obseg trikotnika $AB_1C_1$  enak dolžini
lomljenke $A'B_1C_1A''$ ($AB_1\cong A'B_1$ in $AC_1\cong
A'' C_1$), obseg trikotnika $ABC$ pa dolžini daljice $A'A''$
($AB\cong A'B$ in $AC\cong A'' C$). Iz izreka \ref{neenakIzlLin}
sledi, da je prva dolžina večja od druge, zato je obseg trikotnika
$ABC$ manjši od obsega trikotnika $AB_1C_1$.
 \kdokaz


            \bzgled
            For a given acute triangle $ABC$ determine the inscribed triangle of minimal perimeter
            \index{problem!Fagnano}(Fagnano's problem\footnote{\textit{G. F. Fagnano}
            \index{Fagnano, G. F.} (1717--1797), italijanski matematik,
             je postavil ta problem leta 1775 in ga  rešil
            z metodami
            diferencialnega računa. Bolj
            elementarno rešitev tega problema je kasneje podal  madžarski
            matematik \index{Fejer, L.} \textit{L. Fejer} (1880--1959).}).
            \ezgled

\begin{figure}[!htb]
\centering
\input{sl.izo.6.2.9.pic}
\caption{} \label{sl.izo.6.2.9.pic}
\end{figure}


\textbf{\textit{Solution.}} Označimo za $AP$, $BQ$ in $CR$ višine
trikotnika $ABC$ (Figure \ref{sl.izo.6.2.9.pic}).
 Če je $X$
poljubna točka stranice $BC$, tedaj  minimalen obseg trikotnika
$XYZ$, kjer oglišči $Y$ in $Z$ ležita na stranicah $AC$ in $AB$,
dobimo na način, ki je opisen v prejšnjem izreku
\ref{FagnanLema1}. Točki $Y$ in $Z$ sta torej presečišči stranic
$AC$ in $AB$ s premico $X'X''$ ($X'=\mathcal{S}_{AB}(X)$,
$X''=\mathcal{S}_{AC}(X)$). Pri tem je obseg trikotnika $XYZ$ enak
dolžini daljice $X'X''$.

Torej nam ostaja, da  ugotovimo, za katero točko $X$ stranice
$BC$ je obseg trikotnika $XYZ$ najmanjši oz.  daljica $X'X''$
najkrajša. Iz lastnosti zrcaljenja sledi $AX'\cong AX\cong AX''$,
$\angle X'AB\cong \angle BAX$ in $\angle X''AC\cong \angle CAX$
oz. $\angle X'AX''=2\angle BAC$. Kot $X'AX''$ enakokrakega
trikotnika $X'AX''$ je torej konstanten (glede na točko $X$), zato
je njegova osnovnica $X'X''$ najkrajša v primeru, kadar je
najkrajši njen krak $AX'$ (izrek \ref{SkladTrikLema}) oz. daljica
$AX\cong AX'\cong AX''$. To pa je izpolnjeno, ko je $X$ nožišče
višine trikotnika $ABC$ iz oglišča $A$ oz. $X=P$. Iz zgleda
\ref{FagnanLema} sledi $Y=Q$ in $Z=R$. Torej je včrtani trikotnik z
najmanjšim obsegom pravzaprav pedalni trikotnik trikotnika
$ABC$.
 \kdokaz



            \bzgled
            Let $P$ and $Q$ be interior points of an angle $aOb$.
            Construct a point $X$ on the side $a$ of this angle such that
            the rays $XP$ and $XQ$ intersect the side $b$ at points $Y$ and $Z$ such that $XY\cong XZ$.
            \ezgled

\begin{figure}[!htb]
\centering
\input{sl.izo.6.2.10.pic}
\caption{} \label{sl.izo.6.2.10.pic}
\end{figure}


\textbf{\textit{Solution.}}
 Naj bodo $X$, $Y$ in $Z$ iskane točke (Figure \ref{sl.izo.6.2.10.pic}).
 Naj bosta še: $S$ središče daljice $YZ$ in $P'=\mathcal{S}_a(P)$.
 Mero kota
$aOb$ označimo z $\omega$. Trikotnik $XYZ$ je enakokrak, zato je
(zaradi skladnosti trikotnikov $XSY$ in $XSZ$ - izrek
\textit{SSS} \ref{SSS})
 težiščnica $XS$ hkrati višina in simetrala notranjega kota $XYZ$
 tega
trikotnika. Torej velja $\angle OSX=90^0$ in $\angle YXS\cong
\angle ZXS$. Iz $\mathcal{S}_a:P, X, O\mapsto P', X, O$ pa sledi
$\angle PXO\cong \angle P'XO$. Sedaj lahko računamo mero kota
$P'XQ$:
 \begin{eqnarray*}
\angle P' XQ &=& \angle P' XP + \angle PXQ = 2\angle OXP + 2\angle
PXS\\ &=& 2\angle OXS = 2(90° - \omega) =180° - 2\omega.
 \end{eqnarray*}
Ker sta nam točki $P'$ in $Q$ znani, točko $X$ dobimo kot
presečišče kraka $a$ in loka s tetivo $P'Q$ z obodnim kotom $180°
- 2\omega$ (izrek \ref{ObodKotGMT}).
 \kdokaz



             \bzgled
             Let $ABCD$ be a rectangle such that $|AB|=3|BC|$. Suppose that $E$ and $F$ are
              points on the side $AB$ such that
               $AE\cong EF\cong FB$.
               Prove $$\angle AED+\angle AFD+\angle
            ABD=90^0.$$
            \ezgled

\begin{figure}[!htb]
\centering
\input{sl.izo.6.2.11.pic}
\caption{} \label{sl.izo.6.2.11.pic}
\end{figure}


\textbf{\textit{Solution.}} Naj bo $F'=\mathcal{S}_{CD}(F)$ in
$B'=\mathcal{S}_{CD}(B)$
  (Figure \ref{sl.izo.6.2.11.pic}).
 Tedaj je $DF'\cong DF$ in $\angle DF'F\cong \angle DFF'$. Iz skladnosti
trikotnikov $DAF$ in $F'B'B$ (izrek \textit{SAS} \ref{SKS}) sledi
$DF\cong F'B$ in $\angle DFA\cong \angle B'BF'$. Velja tudi
$\angle FF'B\cong \angle B'BF'$ (izrek \ref{KotiTransverzala}).
Torej velja:
 $$\angle DF'B=\angle DF'F+\angle FF'B=\angle DFF'+\angle DFA=\angle AFF'=90^0.$$
  Ker je $DF'\cong F'B$, je
$DF'B$ enakokraki pravokotni trikotnik z osnovnico $BD$, zato je
(izrek \ref{enakokraki}) $\angle DBF'=\angle BDF'=45^0$. Ker je
tudi $DAE$  enakokraki pravokotni trikotnik z osnovnico $DE$, je
$\angle AED=45^0$ oz. $\angle AED=\angle DBF'$. Zatorej je:
$$\angle AED+\angle AFD+\angle ABD=\angle DBF'+\angle F'BB'+
\angle ABD=\angle ABC=90^0,$$ kar je bilo treba dokazati. \kdokaz

 Ob naslednjem zgledu bomo opisali standardni postopek uporabe
 izometrij pri načrtovalnih nalogah. Predpostavimo, da je potrebno konstruirati
 točki $X\in k$ in $Y\in l$, kjer sta $k$ in $l$ dani krožnici
 (lahko tudi premici ali krožnica in premica), vemo pa, da za neko
 izometrijo $\mathcal{I}$ velja  $\mathcal{I}(X)=Y$. V tem primeru
 iz $X\in k$ sledi $Y=\mathcal{I}(X)\in \mathcal{I}(k)$. Ker je
 tudi $Y\in l$, lahko točko $Y$ konstruiramo iz pogoja $Y\in
 l\cap \mathcal{I}(k)$.



            \bzgled
            Two circles $k$ and $l$ on different sides of a line $p$ are given.
            Construct a square $ABCD$ such that $A\in k$, $C\in l$ and $B,D\in p$.
            \ezgled

\begin{figure}[!htb]
\centering
\input{sl.izo.6.2.12.pic}
\caption{} \label{sl.izo.6.2.12.pic}
\end{figure}


\textbf{\textit{Solution.}}
  (Figure \ref{sl.izo.6.2.12.pic})

  Naj bo $ABCD$ iskani kvadrat. Potem je $\mathcal{S}_p(C)=A$.
Točka $C$ leži na krožnici $l$, zato njena slika -- točka $A$ -- pri
zrcaljenju $\mathcal{S}_p$ leži na sliki krožnice $l$ -- krožnici
$l'$, ki jo lahko načrtamo. Toda točka $A$ leži tudi na krožnici
$k$, zatorej je $A\in l'\cap k$. Na ta način dobimo najprej
oglišče $A$, nato pa $C=\mathcal{S}_p(A)$. Središče kvadrata $S$
dobimo kot središče diagonale $AC$. Na koncu sta oglišči $B$ in
$D$ presečišči premice $p$ in krožnice s središčem $S$ in polmerom
$SA$.

Število rešitev je odvisno od števila presečišč krožnic $l'$ in
$k$. \kdokaz



        \bnaloga\footnote{38. IMO Argentina - 1997, Problem 2.}
        The angle at $A$ is the smallest angle of triangle $ABC$. The points $B$ and $C$
        divide the circumcircle of the triangle into two arcs. Let $U$ be an interior point
        of the arc between $B$ and $C$ which does not contain $A$. The perpendicular
        bisectors of $AB$ and $AC$ meet the line $AU$ at $V$ and $W$, respectively. The lines
        $BV$ and $CW$ meet at $T$. Show that $$|AU| =|TB| + |TC|.$$
        \enaloga

\begin{figure}[!htb]
\centering
\input{sl.izo.6.2.IMO1.pic}
\caption{} \label{sl.izo.6.2.IMO1.pic}
\end{figure}

 \textbf{\textit{Solution.}} Označimo s $q$ in $p$ simetrali
 stranic $AB$ in $AC$ trikotnika $ABC$
 (Figure \ref{sl.izo.6.2.IMO1.pic}). Njuno presečišče -- točka
 $O$ -- je središče očrtane krožnice $l$ tega trikotnika. Po
 predpostavki je $V=AU\cap q$ in $W=AU\cap p$.  Naj bo $D$ drugo
 presečišče poltraka $CW$ s krožnico $l$. S $\mathcal{S}_p$
 označimo zrcaljenje čez
 premico $p$. $\mathcal{S}_p$ preslika točke $A$, $C$, $W$ in $O$
 po vrsti v točke $C$, $A$, $W$ in $O$ ter poltrak $AW$
 v poltrak $CW$. Ker $O\in p$, je po izreku \ref{zrcFiksKroz}
 $\mathcal{S}_p(l) =l$. Iz tega sledi:
  $$\mathcal{S}_p(U) =\mathcal{S}_p(AW\cap l)=
  \mathcal{S}_p(AW)\cap \mathcal{S}_p(l)=
  CW\cap l=D.$$
   Torej $\mathcal{S}_p:\hspace*{1mm} A,U\mapsto C,D$, zato je
   $AU\cong CD$ in $AD\cong CU$. Iz  skladnosti tetiv $AD$
   in $CU$ sledi skladnost pripadajočih obodnih kotov
   (izrek \ref{SklTetSklObKot}) oz. velja
   $\angle ABD\cong\angle UAC$. Nad tetivo $BC$ pa sta skladna
   tudi kota $BDC$ in $BAC$ (izrek\ref{ObodObodKot}).

   Ker točka $V$ leži na simetrali $q$ daljice $AB$, je
   $\mathcal{S}_q:\hspace*{1mm} A,B,V\mapsto B,A,V$. To pomeni, da je
   $\angle BAV\cong\angle ABV$.

   Iz dokazane skladnosti ustreznih kotov sledi:
    \begin{eqnarray*} \angle BDT&=&\angle BDC\cong\angle BAC=\\
    &=&\angle BAU+
   \angle UAC=\\
   &=&\angle ABV+\angle ABD=\\
    &=&\angle DBT.
   \end{eqnarray*}
 Torej je trikotnik $DTB$ enakokrak in po izreku \ref{enakokraki}
 velja $TD\cong TB$. Na koncu dobimo:
  $$|AU|= |CD|=|CT|+|TD|=|CT|+|TB|,$$ kar je bilo treba dokazati. \kdokaz


                 \bnaloga  \footnote{50. IMO Germany - 2009, Problem 4.}
                Let $ABC$ be a triangle with $|AB|=|AC|$. The angle bisectors of $\angle CAB$ and $\angle ABC$
                meet the sides $BC$ and $CA$ at $D$ and $E$, respectively. Let $K$ be the incentre of triangle $ADC$.
                Suppose that $\angle BEK=45^0$. Find all possible values of $\angle CAB$.
                \enaloga



 \textbf{\textit{Solution.}} Označimo s $S$ središče včrtane krožnice
  trikotnika $ABC$.
  To pomeni, da je $S$  presečišče simetral $AD$ in $BE$
  kotov $\alpha=\angle CAB$ in $\beta=\angle ABC$. Točka $S$ leži tudi na
  simetrali
$CF$ ($F\in AB$) kota $\gamma=\angle ACB$. Iz $|AB|=|AC|$ sledi
$\beta\cong\gamma$, zato je tudi $\angle SBC\cong\angle SCB$.

Po predpostavki je točka $K$ središče včrtane krožnice
  trikotnika $ADC$, zato točka $K$ leži na simetrali $CS$ kota
  $\angle ACD=\angle ACB$. Poltrak $DK$ je simetrala kota
  $ADC$. Iz skladnosti trikotnikov $ADB$ in $ADC$ (izrek \textit{SAS} \ref{SKS})
  je $\angle
  ADC\cong\angle ADB=90^0$. Iz tega sledi $\angle SDK=45^0$.

Naj bo $E'=\mathcal{S}_{SC}(E)$. Ker je $CS$ simetrala kota $ACB$,
se krak $CA$ z zrcaljenjem $\mathcal{S}_{SC}$ preslika v poltrak
$CB$. Zato je $E'\in CB$. Ker $S,K,C\in SC$, je
$\mathcal{S}_{SC}:\hspace*{1mm}S,K,C\mapsto S,K,C$.


\begin{figure}[!htb]
\centering
\input{sl.izo.6.2.IMO2a.pic}
\caption{} \label{sl.izo.6.2.IMO2a.pic}
\end{figure}

Obravnavali bomo dva primera. Če je $E'=D$ (Figure
\ref{sl.izo.6.2.IMO2a.pic}), je $\angle SEC\cong \angle SDC=90^0$.
Trikotnika $ABE$ in $CBE$ sta v tem primeru skladna (izrek \textit{ASA}
\ref{KSK}), zato je $AB\cong BC$. Trikotnik $ABC$ je
enakostraničen oz. velja $\angle BAC=60^0$.


\begin{figure}[!htb]
\centering
\input{sl.izo.6.2.IMO2.pic}
\caption{} \label{sl.izo.6.2.IMO2.pic}
\end{figure}

Naj bo $E'\neq D$ (Figure \ref{sl.izo.6.2.IMO2.pic}). Iz
$\mathcal{S}_{SC}:\hspace*{1mm}S,K\mapsto S,K$ sledi $\angle SE'K
\cong \angle SEK=45^0$. Ker je torej $\angle SDK \cong \angle
SE'K=45^0$, so po izreku \ref{ObodKotGMT} točke $S$, $K$, $D$ in
$E'$ konciklične. Iz izreka \ref{TalesovIzrKroz} potem sledi
$\angle SKE'\cong \angle SDE'=90^0$. Ker zrcaljenje
$\mathcal{S}_{SC}$ preslika kot $SKE$ v kot $SKE'$, je $\angle
SKE\cong \angle SKE'=90^0$ (točke $E$, $K$ in $E'$ so zato
kolinearne). Iz tega in izreka \ref{VsotKotTrik} za trikotnik
$SKE$ sledi $\angle KSE=45^0$. Le-ta je zunanji kot trikotnika
$SBC$, zato je po izreku \ref{zunanjiNotrNotr} $\angle SBC +\angle
SCB=45^0$ oz. $\beta+\gamma=90^0$. Iz slednje relacije pa sledi
$\angle BAC=90^0$.

 Možni velikosti kota
                $\angle CAB$ sta torej $60^0$ in $90^0$.
\kdokaz


%________________________________________________________________________________
 \poglavje{More on Reflections} \label{odd6Sopi}

V začetni raziskavi o izometrijah smo ugotovili, da vsaka direktna
izometrija, ki ima vsaj dve fiksni točki, predstavlja identiteto.
Intuitivno je jasno, da  indirektna izometrija z vsaj dvema fiksnima
točkama predstavlja osno zrcaljenje. Dokazali bomo še močnejšo
trditev.



            \bizrek \label{izo1ftIndZrc}
            An opposite isometry of the plane with at least one fixed point is a reflection.
            The axis of this reflection passes through this point.
            \eizrek

\begin{figure}[!htb]
\centering
\input{sl.izo.6.3.1.pic}
\caption{} \label{sl.izo.6.3.1.pic}
\end{figure}

 \textbf{\textit{Proof.}}
 (Figure \ref{sl.izo.6.3.1.pic})

Naj bo $A$ fiksna točka indirektne izometrije $\mathcal{I}$ neke
ravnine. Ker je indirektna, izometrija $\mathcal{I}$ ni
identiteta, zato obstaja takšna točka $B$ te ravnine, da velja
$\mathcal{I}(B)=B'\neq B$. Naj bo $p$ simetrala daljice $BB'$.
Torej je $\mathcal{I}$ izometrija, ki točki $A$ in $B$ preslika v
točki $A$ in $B'$, zato je $AB\cong AB'$. To pomeni, da točka $A$
leži na simetrali $p$ daljice $BB'$. Dokažimo, da izometrija
$\mathcal{I}$ predstavlja zrcaljenje čez premico $p$ oz.
$\mathcal{I} = \mathcal{S}_p$. Kompozitum $\mathcal{S}_p \circ
\mathcal{I}$ (dveh indirektnih izometrij) je direktna izometrija,
z dvema fiksnima točkama $A$ in $B$, zato na osnovi že omenjenega
izreka \ref{izo2ftIdent} predstavlja identiteto. Torej velja
$\mathcal{S}_p \circ \mathcal{I}=\mathcal{E}$. Iz tega sledi  $
\mathcal{I}=\mathcal{S}_p^{-1} \circ\mathcal{E}$ oz. $\mathcal{I}
= \mathcal{S}_p$.
 \kdokaz

 Zelo koristen je naslednji izrek.


             \bizrek \label{izoTransmutacija}
            Let $\mathcal{I}$ be an isometry and $p$ an arbitrary line in the plane.
            Suppose that this isometry maps the line $p$ to a line $p'$. If $\mathcal{S}_p$ is
             a reflection with axis $p$, then
            $$\mathcal{I}\circ \mathcal{S}_p\circ \mathcal{I}^{-1} = \mathcal{S}_{p'}.$$
             \eizrek

\begin{figure}[!htb]
\centering
\input{sl.izo.6.3.2.pic}
\caption{} \label{sl.izo.6.3.2.pic}
\end{figure}

 \textbf{\textit{Proof.}}
 Naj bo $Y$ poljubna točka premice $p'$ in $X = \mathcal{I}^{-1}(Y)$
 (Figure \ref{sl.izo.6.3.2.pic}).

 Ker je $\mathcal{I}(p)=p'$ oz. $\mathcal{I}^{-1}(p')=p$,
  točka $X$ leži na premici $p$ in je
 $\mathcal{S}_p(X)=X$.
 Zatorej velja:
 $$\mathcal{I}\circ\mathcal{S}_p\circ\mathcal{I}^{-1}(Y)=
 \mathcal{I}\circ\mathcal{S}_p(X)=
 \mathcal{I}(X)=Y.$$
 Torej je kompozitum $\mathcal{I}\circ\mathcal{S}_p\circ\mathcal{I}^{-1}$
 indirektna izometrija s fiksno točko $Y$, zato po prejšnjemu izreku
 \ref{izo1ftIndZrc} predstavlja osno zrcaljenje. Toda $Y$ je poljubna
  točka premice $p'$, kar pomeni, da je os tega zrcaljenja ravno premica
  $p'$ oz. $\mathcal{I}\circ \mathcal{S}_p\circ \mathcal{I}^{-1}
  = \mathcal{S}_{p'}$.
 \kdokaz

Transformacijo
$\mathcal{I}\circ\mathcal{S}_p\circ\mathcal{I}^{-1}$ iz prejšnjega
izreka  imenujemo \index{transmutacija!zrcaljenja čez premico}
\pojem{transmutacija} zrcaljenja $\mathcal{S}_p$ z izometrijo
$\mathcal{I}$.

 Dokažimo enostavno posledico prejšnjega
izreka.



             \bzgled \label{izoZrcKomut}
             Two  reflections in the plane commute if and only if
              their axis are perpendicular or coincident, i.e.
             $$\mathcal{S}_p \circ \mathcal{S}_q
            = \mathcal{S}_q \circ \mathcal{S}_p \Leftrightarrow
             (p\perp q \vee p=q).$$
                \ezgled

\begin{figure}[!htb]
\centering
\input{sl.izo.6.3.3.pic}
\caption{} \label{sl.izo.6.3.3.pic}
\end{figure}

 \textbf{\textit{Proof.}}
 (Figure \ref{sl.izo.6.3.3.pic})

 Najprej je $\mathcal{S}_p \circ \mathcal{S}_q
 = \mathcal{S}_q \circ \mathcal{S}_p
 \Leftrightarrow  \mathcal{S}_p=  \mathcal{S}_q \circ
  \mathcal{S}_p \circ \mathcal{S}_q^{-1}$. Če  uporabimo prejšnji izrek
   \ref{izoTransmutacija} za $\mathcal{I} = \mathcal{S}_q$,
dobimo, da je zadnja enakost ekvivalentna z $\mathcal{S}_p =
\mathcal{S}_{p'}$, kjer je $p' = \mathcal{S}_q(p)$. To pa velja
natanko tedaj, ko je $p=p'=\mathcal{S}_q(p)$ oz. natanko tedaj,
ko je $p\perp q$ ali $p=q$.
 \kdokaz

Kompozitum dveh osnih zrcaljenj $\mathcal{I}=\mathcal{S}_p \circ
\mathcal{S}_q$ bo zelo pomemben v nadaljnjem raziskovanju
izometrij. Takoj  ugotovimo, da kompozitum dveh
indirektnih izometrij predstavlja  direktno izometrijo.
Če osi teh zrcaljenj sovpadata, smo že
ugotovili, da gre za identiteto (izrek \ref{izoZrcPrInvol}).
Zanimivo je vprašanje, kaj $\mathcal{I}$  predstavlja v splošnem
primeru, kadar sta osi $p$ in $q$ teh zrcaljenj različni in
komplanarni; še posebej v primeru, ko se sekata ali sta vzporedni.
Odgovor na to vprašanje bomo izvedeli v naslednjih razdelkih. V
naslednjem izreku pa bomo izvedeli nekaj o fiksnih točkah kompozituma
dveh osnih zrcaljenj.

        \bizrek \label{izoKomppqX}
        Let $p$ and $q$ be two distinct coplanar lines.
        Then
         $$\mathcal{S}_q\circ\mathcal{S}_p(X)=X \Leftrightarrow p\cap q=\{X\}.$$
         \eizrek

\textbf{\textit{Proof.}}

($\Leftarrow$) Trivialno, kajti iz $X\in p$ in $X\in q$ sledi
$\mathcal{S}_p(X)= \mathcal{S}_q(X)=X$ oz.
$\mathcal{S}_q\circ\mathcal{S}_p(X)=X$.

($\Rightarrow$) Naj bo $\mathcal{S}_q\circ\mathcal{S}_p(X)=X$ in
$\mathcal{S}_p(X)=X'$. Iz tega sledi $\mathcal{S}_q(X')=X$.
Predpostavimo, da je $X\neq X'$. V tem primeru sta po definiciji
osnega zrcaljenja $p$ in $q$ simetrali daljice $XX'$. To pa ni
možno, saj ima daljica eno samo simetralo, $p$ in $q$ pa sta po
predpostavki različni. Zatorej je $X=X'$. Iz tega sledi
$\mathcal{S}_p(X)=\mathcal{S}_q(X)=X$, kar pomeni, da velja $X\in
p$ in $X\in q$ oz. $X\in p\cap q$. Ker sta $p$ in $q$ različni
premici, je po aksiomu \ref{AksI1} $p\cap q=\{X\}$.
 \kdokaz

 Če se premici $p$ in $q$ sekata, iz prejšnjega izreka sledi, da ima kompozitum
  $\mathcal{S}_q\circ\mathcal{S}_p$
 eno samo  fiksno točko, ki je njuno presečišče. Če sta  $p$ in
$q$ vzporednici, kompozitum $\mathcal{S}_q\circ\mathcal{S}_p$ nima
fiksnih točk.

Za nadaljevanje  raziskave izometrij je zelo značilen naslednji
pojem.

Množico vseh premic neke ravnine, ki potekajo skozi eno točko $S$ te
ravnine,
 imenujemo \index{šop!konkurentnih premic}
 \pojem{šop konkurentnih premic} (ali \index{šop!eliptični}\pojem{eliptični šop}) s središčem $S$ in jo označimo z
 $\mathcal{X}_S$.

Množico vseh premic neke ravnine, ki so vzporedne z neko premico
$s$ te ravnine, imenujemo \index{šop!vzporednic} \pojem{šop
vzporednic} (ali \index{šop!parabolični}\pojem{parabolični šop} ali \index{snop premic}\pojem{snop premic}) in jo označimo z
 $\mathcal{X}_s$.

\begin{figure}[!htb]
\centering
\input{sl.izo.6.3.4.pic}
\caption{} \label{sl.izo.6.3.4.pic}
\end{figure}



 V obeh primerih bomo govorili o \pojem{šopu premic}
  (Figure \ref{sl.izo.6.3.4.pic}). Če ne bomo posebej poudarili, ali gre za šop
  konkurentnih premic oz. šop vzporednic, bomo ta šop označevali
  le z $\mathcal{X}$.

V zvezi z vpeljanimi pojmi govori naslednji izrek.


         \bizrek \label{izoSop}
        Let $p$, $q$ and $r$ be lines in the plane.
        The product $\mathcal{S}_r \circ \mathcal{S}_q \circ \mathcal{S}_p$
        is a reflection if and only if
        the lines $p$, $q$ and $r$ belong to the same family of lines.
        The axis of the new reflection also belongs to this family.
        \eizrek

\begin{figure}[!htb]
\centering
\input{sl.izo.6.3.5.pic}
\caption{} \label{sl.izo.6.3.5.pic}
\end{figure}

 \textbf{\textit{Proof.}}
 (Figure \ref{sl.izo.6.3.5.pic})

 ($\Leftarrow$) Predpostavimo najprej, da premice $p$, $q$ in $r$
 pripadajo istemu šopu  $\mathcal{X}$. Obravnavali bomo dva
 primera.

\textit{1)} Naj bo $\mathcal{X}$ šop konkurentnih premic s
središčem $S$ oz.  $p, q, r\in \mathcal{X}_S$. V tem primeru
je kompozitum $\mathcal{I}=\mathcal{S}_r \circ \mathcal{S}_q \circ
\mathcal{S}_p$ indirektna izometrija s  fiksno točko $S$, zato po
izreku \ref{izo1ftIndZrc} predstavlja neko zrcaljenje
$\mathcal{S}_t$ z osjo, ki poteka skozi točko $S$ oz. $t\in
\mathcal{X}_S$.

 \textit{2)} Naj bo $\mathcal{X}$ šop vzporednic in $n$ poljubna skupna
 pravokotnica premic $p$, $q$ in $r$. V tem primeru je kompozitum
  $\mathcal{I}=\mathcal{S}_r \circ \mathcal{S}_q \circ
\mathcal{S}_p$ indirektna izometrija  in velja $\mathcal{I}(n)=n$.
Dokažimo, da na premici $n$ obstaja
 neka fiksna točka te izometrije. Naj bo $A$ poljubna točka
 premice
 $n$ in $\mathcal{I}(A)=A'$. Če $A$ ni fiksna točka izometrije
 $\mathcal{I}$, z $O$
 označimo  središče daljice $AA'$ in z $O'=\mathcal{I}(O)$.
  $\mathcal{I}$ je indirektna izometrija, zato sta
   $AO$ in $A'O'$ skladni in različno orientirani daljici  na premici $n$.
 Iz tega in iz dejstva, da je $O$ središče daljice $AA'$, sledi
 $\overrightarrow{A'O}=\overrightarrow{OA}=\overrightarrow{A'O'}$.
   Torej $O=O'$ oz. $O$ je fiksna
 točka izometrije $\mathcal{I}$, zato predstavlja neko osno zrcaljenje
 $\mathcal{S}_t$ (izrek \ref{izo1ftIndZrc}). Pri tem je
  $\mathcal{S}_t(n)=\mathcal{I}(n)=n$. To pomeni, da je $t=n$ ali
  $t\perp n$. Prva možnost odpade, ker iz $\mathcal{S}_r
  \circ \mathcal{S}_q \circ
\mathcal{S}_p=\mathcal{S}_n$ sledi $\mathcal{S}_q \circ
\mathcal{S}_p=\mathcal{S}_r\circ \mathcal{S}_n$. Zadnja relacija
ni možna, ker sta premici $p$ in $q$ vzporedni, premici $r$ in $n$
sta pravokotni in se sekata (izrek \ref{izoKomppqX}). Zatorej je
$t\perp n$, kar pomeni, da je tudi $t$ premica iz šopa vzporednic
$\mathcal{X}$.


 ($\Rightarrow$) Predpostavimo sedaj, da je
 $\mathcal{S}_r \circ \mathcal{S}_q \circ \mathcal{S}_p=\mathcal{S}_t$
  za neko premico $t$. V tem primeru velja
 $\mathcal{S}_q \circ \mathcal{S}_p=\mathcal{S}_r
 \circ\mathcal{S}_t$. Spet bomo obravnavali dva primera.

 \textit{1)} Predpostavimo, da se $p$ in $q$ sekata v neki točki $S$.
 V tem primeru je je $\mathcal{S}_r
 \circ\mathcal{S}_t(S) = \mathcal{S}_q \circ \mathcal{S}_p(S)=S$. Po
 izreku \ref{izoKomppqX} se tudi premici $r$ in $t$ sekata v točki $S$, zato je
  $p, q, r, t\in \mathcal{X}_S$.

  \textit{2)} Predpostavimo, da sta premici $p$ in $q$ vzporedni.
  Iz že omenjenega izreka \ref{izoKomppqX} sledi, da
  kompozitum  $\mathcal{S}_q \circ \mathcal{S}_p$ nima fiksnih točk.
  Tedaj niti kompozitum $\mathcal{S}_r \circ \mathcal{S}_t$
  nima fiksnih točk, zato je tudi $r\parallel t$. Naj bo  $n$
  skupna pravokotnica premic $p$ in $q$. Tedaj je $\mathcal{S}_r
 \circ\mathcal{S}_t(n) = \mathcal{S}_q \circ \mathcal{S}_p(n)=n$.
  Iz tega sledi $\mathcal{S}_t(n)=\mathcal{S}_r(n)=n'$.
  Če je $n\neq n'$, sta obe vzporednici $p$ in $q$ osi
  simetrij premic $n$ in $n'$, kar ni možno. Zato je $n=n'$ in
   $\mathcal{S}_t(n)=\mathcal{S}_r(n)=n$.
  Torej velja $r,t\perp n$, kar pomeni, da premice $p$, $q$, $r$ in
  $t$
  pripadajo šopu vzporednic,
  ki so vse pravokotne s premico $n$.
 \kdokaz

Posledica dokazane trditve je naslednji izrek.



        \bizrek \label{izoSop2n+1}
        The product of an odd number of reflections with the axes
        from the same family of lines is also a reflection.
        \eizrek

 \textbf{\textit{Proof.}} Dokaz bomo izpeljali z indukcijo po
 številu $n\geq 1$, kjer je $m=2n+1$ (liho) število premic.

 Za $n=1$ oz. $m=3$, je trditev direktna posledica prejšnjega izreka
 \ref{izoSop}.

 Predpostavimo, da za vsak $k\in \mathbb{N}$ ($k\geq 1$)
  in vsako $(2 k+1)$-terico
 premic
 $p_1,p_2,\ldots ,p_{2k+1}$, ki so vse iz istega šopa,  kompozitum
 $\mathcal{S}_{p_{2k+1}}\circ
 \cdots \circ\mathcal{S}_{p_2}\circ\mathcal{S}_{p_1}$ predstavlja
 neko osno zrcaljenje $\mathcal{S}_p$ (indukcijska predpostavka).

 Potrebno je dokazati, da potem tudi za $k+1$
 in vsako $(2(k+1)+1)$-terico
 premic
 $p_1,p_2,\ldots ,p_{2(k+1)+1}$, ki so vse iz istega šopa, kompozitum
 $\mathcal{I}=\mathcal{S}_{p_{2(k+1)+1}}\circ
 \cdots \circ\mathcal{S}_{p_2}\circ\mathcal{S}_{p_1}$ predstavlja
 neko osno zrcaljenje $\mathcal{S}_{p'}$. Toda:
   \begin{eqnarray*}
   \mathcal{I}&=&\mathcal{S}_{p_{2(k+1)+1}}\circ
 \cdots \circ\mathcal{S}_{p_2}\circ\mathcal{S}_{p_1}=\\
   &=&\mathcal{S}_{p_{2k+3}}\circ\mathcal{S}_{p_{2k+2}}\circ
   \mathcal{S}_{p_{2k+1}}\circ
 \cdots \circ\mathcal{S}_{p_2}\circ\mathcal{S}_{p_1}=\\
 &=& \mathcal{S}_{p_{2k+3}}\circ\mathcal{S}_{p_{2k+2}}\circ
   \mathcal{S}_p= \hspace*{14mm} \textrm{ (po indukcijski predpostavki)}\\
   &=&  \mathcal{S}_{p'} \hspace*{48mm} \textrm{ (po izreku \ref{izoSop}),}
   \end{eqnarray*}
  kar je bilo treba dokazati.  \kdokaz


 Dokažimo še nekaj direktnih posledic izreka \ref{izoSop}.

        \bzgled \label{iropqrrqp}
        If $p$, $q$ and $r$ are lines from the same family of lines $\mathcal{X}$, then
        $$\mathcal{S}_r \circ \mathcal{S}_q \circ \mathcal{S}_p =
        \mathcal{S}_p \circ \mathcal{S}_q \circ \mathcal{S}_r.$$
        \ezgled

\textbf{\textit{Proof.}} Ker so $p$, $q$ in $r$ premice istega
šopa $\mathcal{X}$, po prejšnjem izreku \ref{izoSop} sledi
$\mathcal{S}_r \circ \mathcal{S}_q \circ \mathcal{S}_p
=\mathcal{S}_t$ za neko os $t\in\mathcal{X}$. Zaradi tega je:
 $$(\mathcal{S}_r \circ
\mathcal{S}_q \circ \mathcal{S}_p)^2
=\mathcal{S}_t^2=\mathcal{E},\hspace*{3mm}\textrm{oz.}$$
  $$\mathcal{S}_r \circ
\mathcal{S}_q \circ \mathcal{S}_p \circ\mathcal{S}_r \circ
\mathcal{S}_q \circ \mathcal{S}_p=\mathcal{E}.$$
 Če zadnjo
relacijo množimo po vrsti z leve s $\mathcal{S}_r$,
$\mathcal{S}_q$ in $\mathcal{S}_p$, dobimo iskano enakost.
 \kdokaz

        \bzgled \label{izoSopabc}
         If $a$, $b$ and $c$ are lines such that $\mathcal{S}_a(b)=c$,
         then those lines are from the same family of lines.
         \ezgled

\begin{figure}[!htb]
\centering
\input{sl.izo.6.3.6a.pic}
\caption{} \label{sl.izo.6.3.6a.pic}
\end{figure}

 \textbf{\textit{Proof.}}
 (Figure \ref{sl.izo.6.3.6a.pic})

Iz izreka o transmutaciji \ref{izoTransmutacija} sledi
$\mathcal{S}_a\circ\mathcal{S}_b\circ\mathcal{S}_a=
\mathcal{S}_{\mathcal{S}_a(b)}= \mathcal{S}_c$. Po množenju
dobljene relacije
$\mathcal{S}_a\circ\mathcal{S}_b\circ\mathcal{S}_a= \mathcal{S}_c$
najprej z desne s $\mathcal{S}_a$, nato pa še z leve s
$\mathcal{S}_c$ dobimo
$\mathcal{S}_c\circ\mathcal{S}_a\circ\mathcal{S}_b=\mathcal{S}_a$.
Po izreku \ref{izoSop} premice $a$, $b$ in $c$ pripadajo istemu
šopu.
 \kdokaz

 Jasno je, da je premica $a$ iz prejšnjega izreka os simetrije
 premic $b$ in $c$.

 Z naslednjim izrekom bomo določili premico $t$ iz relacije
 $\mathcal{S}_r \circ
\mathcal{S}_q \circ \mathcal{S}_p =\mathcal{S}_t$ za
 premice $p$, $q$ in $r$ iz istega šopa.


        \bizrek \label{izoSoppqrt}
        If $p$, $q$, $r$ and $t$ are lines such that
        $\mathcal{S}_r \circ
        \mathcal{S}_q \circ \mathcal{S}_p =\mathcal{S}_t$, then the axis of symmetry of the lines
         $p$ and $r$ is also
        the axis of symmetry of the lines $q$ and $t$.
        \eizrek


\begin{figure}[!htb]
\centering
\input{sl.izo.6.3.6.pic}
\caption{} \label{sl.izo.6.3.6.pic}
\end{figure}

 \textbf{\textit{Proof.}}
 Iz  $\mathcal{S}_r \circ
\mathcal{S}_q \circ \mathcal{S}_p =\mathcal{S}_t$ sledi, da
premice $p$, $q$, $r$ in $t$
 pripadajo istemu šopu $\mathcal{X}$ (izrek \ref{izoSop}).
 Naj bo $s$ os simetrije premic $p$ in $r$ (Figure \ref{sl.izo.6.3.6.pic}).
 To pomeni, da velja $\mathcal{S}_s(p)=r$,
 zato je po izreku o transmutaciji \ref{izoTransmutacija} tudi
  $\mathcal{S}_s\circ\mathcal{S}_p\circ\mathcal{S}_s
  =\mathcal{S}_{\mathcal{S}_s(p)}
  =\mathcal{S}_r$. Iz tega naprej sledi:
 $$\mathcal{S}_t = \mathcal{S}_r \circ\mathcal{S}_q \circ \mathcal{S}_p
 = (\mathcal{S}_s\circ\mathcal{S}_p\circ\mathcal{S}_s)\circ
 \mathcal{S}_q\circ\mathcal{S}_p =
\mathcal{S}_s\circ(\mathcal{S}_p\circ \mathcal{S}_s\circ
\mathcal{S}_q)\circ \mathcal{S}_p.$$
 Ker je $\mathcal{S}_s(p)=r$, iz prejšnjega zgleda \ref{izoSopabc} sledi,
 da premice $s$, $p$ in $r$
pripadajo istemu šopu, zato je $s\in \mathcal{X}$. Iz izreka
\ref{iropqrrqp} sledi $\mathcal{S}_p\circ \mathcal{S}_s\circ
\mathcal{S}_q = \mathcal{S}_q\circ \mathcal{S}_s\circ
\mathcal{S}_p$. Če uporabimo še izrek o transmutaciji
\ref{izoTransmutacija}, dobimo:
$$\mathcal{S}_t =
\mathcal{S}_s\circ\mathcal{S}_p\circ \mathcal{S}_s\circ
\mathcal{S}_q\circ \mathcal{S}_p=
\mathcal{S}_s\circ\mathcal{S}_q\circ \mathcal{S}_s\circ
\mathcal{S}_p\circ \mathcal{S}_p=
\mathcal{S}_s\circ\mathcal{S}_q\circ \mathcal{S}_s=
\mathcal{S}_{\mathcal{S}_s(q)}.$$
  Iz $\mathcal{S}_t =\mathcal{S}_{\mathcal{S}_s(q)}$ sledi
  $\mathcal{S}_s(q)=t$, kar pomeni, da je premica $s$ os simetrije
premic $q$ in $t$.
 \kdokaz

 Ob naslednji načrtovalni nalogi bomo ilustrirali dve metodi
 reševanja.


            \bizrek
              Let $p$, $q$ and $r$ be lines in the plane.
            Construct a triangle $ABC$ such that its vertices $B$ and $C$
             lie on the line $p$ and the lines $q$ and $r$ are perpendicular
             bisectors of the sides $AB$ and $AC$.
            \eizrek


\begin{figure}[!htb]
\centering
\input{sl.izo.6.3.7.pic}
\caption{} \label{sl.izo.6.3.7.pic}
\end{figure}

 \textbf{\textit{Solution.}} (prvi način)

 Naj bosta $q$ in $r$ simetrali stranic $AB$ in $AC$
 (Figure \ref{sl.izo.6.3.7.pic}\textit{a}). Potem je
 $\mathcal{S}_q(B)=A$ in $\mathcal{S}_r(C)=A$.
Točki $B$ in $C$ ležita na premici $p$, zato točka $A$ leži na
slikah  te premice glede na zrcaljenji $\mathcal{S}_q$ in
$\mathcal{S}_r$. Torej lahko oglišče $A$ konstruiramo kot
presečišče premic $p'_q=\mathcal{S}_q(p)$ in
$p'_r=\mathcal{S}_r(p)$.

 \textbf{\textit{Solution.}} (drugi način)


 Presečišče premic $q$ in $r$ je središče očrtane krožnice tega trikotnika
 (izrek \ref{SredOcrtaneKrozn});
označimo jo z $O$ (Figure \ref{sl.izo.6.3.7.pic}\textit{b}). Torej
lahko konstruiramo tudi tretjo simetralo $s$ stranice $BC$, ki je
pravokotnica premice $p$ v točki $O$. Kompozitum $\mathcal{S}_q
\circ \mathcal{S}_r \circ \mathcal{S}_s$ po izreku \ref{izoSop}
predstavlja neko osno zrcaljenje $\mathcal{S}_l$ (ker $q, r, s\in
\mathcal{X}_O$). Fiksni točki tega kompozituma sta $O$ in $B$,
zato je $\mathcal{S}_q \circ \mathcal{S}_r \circ \mathcal{S}_s =
\mathcal{S}_{OB}$. Premico $OB=l$ lahko torej konstruiramo kot
simetralo daljice $XX'$, kjer je $X$ poljubna točka in
$X'=\mathcal{S}_q \circ \mathcal{S}_r \circ \mathcal{S}_s(X)$. V
primeru, če je $X=X'$, je $l=OX$. Na koncu dobimo točko $B$ kot
presečišče premic $p$ in $l$.
 \kdokaz



             \bzgled
               Let $ABCDE$ be a pentagon with a right angle
               at the vertex  $A$.
            The perpendicular bisectors of the sides $AE$, $BC$,
            and $CD$ (lines $p$, $q$, and $r$) intersect
            at a point $O$ and the perpendicular bisectors of the sides $AB$ and $DE$
            (lines $x$ and $y$) intersect at a point $S$
            ($S\neq O$). Prove that one of the vertices of a triangle $KLM$,
            such that $p$, $q$
            and $r$ are the perpendicular bisectors of its sides,
            belongs to the line $OS$.
             \ezgled

\begin{figure}[!htb]
\centering
\input{sl.izo.6.3.8.pic}
\caption{} \label{sl.izo.6.3.8.pic}
\end{figure}

 \textbf{\textit{Solution.}}
 (Figure \ref{sl.izo.6.3.8.pic})

Naj bo $\mathcal{I} = \mathcal{S}_y \circ\mathcal{S}_r
\circ\mathcal{S}_q \circ\mathcal{S}_x \circ\mathcal{S}_p$.
$\mathcal{I}$ je indirektna izometrija, ki ima fiksno točko $E$,
zato po izreku \ref{izo1ftIndZrc} predstavlja neko zrcaljenje
$S_l$. Ker je $\angle EAB$ pravi kot, sta simetrali $p$ in $x$
pravokotni, zato zrcaljenji $\mathcal{S}_p$ in $\mathcal{S}_x$
komutirata (zgled \ref{izoZrcKomut}). Torej velja:
 $$\mathcal{I} =\mathcal{S}_l=
  \mathcal{S}_y \circ\mathcal{S}_r
\circ\mathcal{S}_q \circ\mathcal{S}_x \circ\mathcal{S}_p=
  \mathcal{S}_y \circ (\mathcal{S}_r
\circ\mathcal{S}_q \circ\mathcal{S}_p) \circ\mathcal{S}_x.$$
 Premice $p$, $q$ in $r$ pripadajo istemu šopu
$\mathcal{X}_O$, zato kompozitum $\mathcal{S}_r \circ\mathcal{S}_q
\circ\mathcal{S}_p$ predstavlja neko zrcaljenje $\mathcal{S}_t$,
kjer velja tudi $t\in\mathcal{X}_O$ oz. os $t$ poteka skozi točko
$O$ (izrek \ref{izoSop}).
 Torej velja $\mathcal{S}_l=
  \mathcal{S}_y
\circ\mathcal{S}_t  \circ\mathcal{S}_x$, zato premice $x$, $t$ in
$y$ pripadajo istem šopu, oz. premica $t$ poteka skozi točko $S=x\cap
y$. Ker  premica $t$ poteka skozi različni točki $O$ in $S$,  je
$t=OS$. Zato je $\mathcal{S}_{OS}=\mathcal{S}_t=\mathcal{S}_r
\circ\mathcal{S}_q \circ\mathcal{S}_p$. Naj bodo $p$, $q$ in $r$
simetrale stranic $KL$, $LM$ in $MK$ trikotnika $KLM$. V tem
primeru je $\mathcal{S}_{OS}(K)=\mathcal{S}_t(K)=\mathcal{S}_r
\circ\mathcal{S}_q \circ\mathcal{S}_p(K)=K$, torej točka $K$
leži na premici $OS$.
 \kdokaz



%________________________________________________________________________________
 \poglavje{Rotations} \label{odd6Rotac}


  Do sedaj smo spoznali izometrije, ki imajo več kot eno fiksno točko.
   Dokazali smo, da obstajata dve takšni vrsti izometrij, in sicer
   identiteta kot direktna izometrija z vsaj dvema fiksnima
   točkama (izrek \ref{izo2ftIdent}) in zrcaljenje kot indirektna izometrija z vsaj eno
   fiksno točko (izrek \ref{izo1ftIndZrc}). V tem razdelku bomo
   spoznali vrsto direktnih izometrij, ki imajo natanko eno
   fiksno točko.

Naj bo $S$ poljubna točka in $\omega\neq 0$ orientirani kot
evklidske ravnine. Transformacija te ravnine, v kateri je točka
$S$ fiksna, vsako drugo točko $X\neq S$ te ravnine pa preslikava v
takšno točko $X'$, da velja $\measuredangle XSX'\cong \omega$ in
$SX'\cong SX$, imenujemo rotacija s središčem $S$ za kot $\omega$;
označimo jo z $\mathcal{R}_{S,\omega}$ (Figure
\ref{sl.izo.6.4.1.pic}).

\begin{figure}[!htb]
\centering
\input{sl.izo.6.4.1.pic}
\caption{} \label{sl.izo.6.4.1.pic}
\end{figure}

Iz same definicije direktno sledijo naslednji izreki.



        \bizrek \label{RotacFiksT}
        The only fixed point of a rotation $\mathcal{R}_{S,\omega}$
        is its centre $S$ i.e.
        $$\mathcal{R}_{S,\omega}(X)=X
        \hspace*{1mm} \Leftrightarrow  \hspace*{1mm} X=S.$$
        \eizrek


             \bizrek \label{rotacEnaki}
         Rotations $\mathcal{R}_{S,\alpha}$ and $\mathcal{R}_{B,\beta}$
          are equal if and only if $A=B$ and $\alpha=\beta$ i.e.
          $$\mathcal{R}_{S,\alpha}=\mathcal{R}_{B,\beta}
          \hspace*{1mm} \Leftrightarrow  \hspace*{1mm} A=B \hspace*{1mm}
         \wedge\hspace*{1mm} \alpha=\beta.$$
            \eizrek



         \bizrek
         The inverse transformation of a rotation is a rotation with the same centre
        and congruent angle with opposite orientation, i.e.
        $$\mathcal{R}_{S,\omega}^{-1}=\mathcal{R}_{S,-\omega}.$$
         \eizrek



 Čeprav je intuitivno jasno, je potrebno dokazati, da je rotacija
 izometrija.

             \bizrek
             Rotations are isometries of the plane.
              \eizrek

\begin{figure}[!htb]
\centering
\input{sl.izo.6.4.2.pic}
\caption{} \label{sl.izo.6.4.2.pic}
\end{figure}

 \textbf{\textit{Proof.}} Naj bo $\mathcal{R}_{S,\omega}$ poljubna
 rotacija. Iz definicije je jasno, da predstavlja bijektivno preslikavo. Potrebno je še dokazati, da za poljubni točki $X$ in $Y$
 ter njuni sliki $X'$ in $Y'$ ($\mathcal{R}_{S,\omega}:X,Y\mapsto X', Y'$) velja
 $X'Y'\cong XY$ (Figure \ref{sl.izo.6.4.2.pic}).

Če je ena od točk $X$ ali $Y$ enaka središču rotacije $S$ (npr.
$X=S$), je relacija $X'Y'\cong XY$ avtomatično izpolnjena, saj  v
tem primeru po definiciji rotacije velja $SY'\cong SY$.

Predpostavimo, da nobena od točk $X$ in $Y$ ni središče rotacije
$S$. Najprej je po definiciji rotacije $SX'\cong SX$ in $SY'\cong
SY$. Nato je še (iz relacije \ref{orientKotVsota}):
 \begin{eqnarray*}
 \measuredangle Y'SX'&=& \measuredangle Y'SX+\measuredangle XSX'=
 \measuredangle Y'SX + \omega=\\
 &=& \measuredangle Y'SX + \measuredangle YSY'=
 \measuredangle YSY'+ \measuredangle Y'SX=\\ &=&\measuredangle YSX
 \end{eqnarray*}
 To pomeni, da sta trikotnika $X'SY'$ in $XSY$ skladna (izrek \textit{SAS} \ref{SKS}),
  zato je
 $X'Y'\cong XY$.
 \kdokaz

   Kot vsaka izometrija, tudi rotacija preslika premico v premico.
   V naslednjem izreku bomo videli, kakšen je odnos premice in
   njene slike pri rotaciji.



        \bizrek \label{rotacPremPremKot}
        A line and its rotated image determine an angle congruent
         to the angle of this rotation.
        If the angle of the rotation measures $180^0$,
        the line and its image are parallel.
        \eizrek

\begin{figure}[!htb]
\centering
\input{sl.izo.6.4.3.pic}
\caption{} \label{sl.izo.6.4.3.pic}
\end{figure}

 \textbf{\textit{Proof.}} Naj bo $p'$ slika premice $p$ pri rotaciji
  $\mathcal{R}_{S,\omega}$ (Figure \ref{sl.izo.6.4.3.pic}).

  Če je $S\in p$, je dokaz trivialen. Predpostavimo, da $S\notin p$.
  Označimo s $P$ pravokotno projekcijo središča $S$ na
   premici $p$ in  njeno sliko $P'=\mathcal{R}_{S,\omega}(P)$. Ker $S\notin
   p$,
   je $P\neq S$ oz. $P'\neq P$. Iz
   $P\in p$ sledi $P'\in p'$. Ker izometrija ohranja kote, iz
   $SP\perp p$ sledi $SP'\perp p'$.

   Če je $\omega=180^0$, so točke $P$, $S$ in $P'$ kolinearne in
   imata premici $p$ in $p'$ skupno pravokotnico
   $PP'$, kar pomeni $p\parallel p'$.

   V primeru $\omega\neq 180^0$ se premici sekata (v nasprotnem
    bi iz $p\parallel p'$ sledilo $SP\parallel SP'$, kar pa ni možno).
    Naj bo njuno presečišče točka $V$. Kot, ki ga določata premici
    $p$ in $p'$, je enak kotu, ki ga določata premici $SP$ in
    $SP'$ (kota s pravokotnimi kraki - izrek \ref{KotaPravokKraki}),
     ki pa je ravno kot rotacije $\omega$.
    \kdokaz


   V dokazu prejšnjega izreka je opisan tudi postopek načrtovanja slike
   $p'$
   premice $p$ pri rotaciji $\mathcal{R}_{S,\omega}$.
   Najprej narišemo pravokotno projekcijo $P$ središča $S$ na
   premici $p$, nato pa njeno sliko $P=\mathcal{R}_{S,\omega}(P')$.
   Premico $p'$ dobimo kot pravokotnico premice $SP'$ v točki
   $P'$.
   Drugi način bi seveda bil, da rotiramo dve poljubni točki
   premice $p$.

  V naslednjem pomembnem izreku bomo videli, kako lahko izrazimo
  rotacijo s pomočjo osnih zrcaljenj.




           \bizrek \label{rotacKom2Zrc}
            Any rotation
           $\mathcal{R}_{S,\omega}$  can be expressed as the product of two reflections
            $\mathcal{S}_p$ and $\mathcal{S}_q$ where
               $p$ and $q$  are arbitrary lines, such that $S=p\cap q$
             and $\measuredangle pq=\frac{1}{2}\omega$.\\ The reverse is also true -
             the product of two reflections $\mathcal{S}_p$ and $\mathcal{S}_q$ ($S=p\cap q$)
             is a rotation  with the centre $S$ and the angle
             $\omega=2\cdot\measuredangle pq$, i.e.\\
             $$\mathcal{R}_{S,\omega}=\mathcal{S}_q\circ\mathcal{S}_p
            \hspace*{1mm} \Leftrightarrow  \hspace*{1mm} S=p\cap q \hspace*{1mm}
            \wedge\hspace*{1mm} \measuredangle pq=\frac{1}{2}\omega.$$
            \eizrek

\begin{figure}[!htb]
\centering
\input{sl.izo.6.4.4.pic}
\caption{} \label{sl.izo.6.4.4.pic}
\end{figure}

 \textbf{\textit{Proof.}}

 ($\Leftarrow$) Predpostavimo, da sta $p$
 in $q$ premici, ki se sekata v točki $S$. Naj bo $\omega
  =2\cdot\measuredangle pq$
 (Figure \ref{sl.izo.6.4.4.pic}). Dokažimo, da velja
  $\mathcal{R}_{S,\omega}=\mathcal{S}_q\circ\mathcal{S}_p$,
  oz. $\mathcal{R}_{S,\omega}(X)=\mathcal{S}_q\circ\mathcal{S}_p(X)$ za vsako
točko $X$ ravnine. Obravnavali bomo dva primera.

    \textit{1)} Če je $X=S$, je $\mathcal{S}_p(X)=\mathcal{S}_q(X)=X$ (ker
se premici $p$ in $q$ sekata v točki $S$). Torej  je
$\mathcal{R}_{S,\omega}(S)=S=\mathcal{S}_q\circ\mathcal{S}_p(S)$.

    \textit{2)} Naj bo $X\neq S$ in $\mathcal{S}_q\circ\mathcal{S}_p(X)=X'$.
     Dokažimo, da  je tudi $\mathcal{R}_{S,\omega}(X)=X'$. Naj bo
$\mathcal{S}_p(X)=X_1$. Potem je  $\mathcal{S}_q(X_1)=X'$. Zaradi
tega je najprej $SX\cong SX_1\cong SX'$, nato pa tudi:
 \begin{eqnarray*}
  \measuredangle XSX'&=& \measuredangle XSX_1+ \measuredangle
  X_1SX'=\\
   &=& 2\cdot\measuredangle p,SX_1+2\cdot\measuredangle SX_1,q=\\
   &=& 2\cdot(\measuredangle p,SX_1+ \measuredangle SX_1,q) =\\
   &=& 2\cdot\measuredangle pq=\omega.
 \end{eqnarray*}
 Po definiciji rotacije je $\mathcal{R}_{S,\omega}(X)=X'$.

  ($\Rightarrow$) Naj bo
  $\mathcal{R}_{S,\omega}=\mathcal{S}_q\circ\mathcal{S}_p$. Potem
  je
  $\mathcal{S}_q\circ\mathcal{S}_p(S)=\mathcal{R}_{S,\omega}(S)=S$.
  Po izreku \ref{izoKomppqX} je $S=p\cap q$. Potrebno je dokazati še
  $\measuredangle pq=\frac{1}{2}\omega$ oz. $\omega
  =2\cdot\measuredangle pq$. Označimo $\widehat{\omega}
  =2\cdot\measuredangle pq$. Po prvem delu dokaza ($\Leftarrow$)
  je
  $\mathcal{S}_q\circ\mathcal{S}_p=\mathcal{R}_{S,\widehat{\omega}}$.
  Torej velja $\mathcal{R}_{S,\omega}=\mathcal{S}_q\circ\mathcal{S}_p=
  \mathcal{R}_{S,\widehat{\omega}}$. Iz izreka \ref{rotacEnaki}
  sledi $\omega =\widehat{\omega}
  =2\cdot\measuredangle pq$.
   \kdokaz



         \bizrek \label{RotacDirekt}
        A rotation is a direct isometry.
        \eizrek

 \textbf{\textit{Proof.}} Direktna posledica prejšnjega izreka
 \ref{rotacKom2Zrc}, ker je kompozitum dveh osnih zrcaljenj, ki
 sta indirektni izometriji.
 \kdokaz

Podobno, kot pri osnem zrcaljenju, tudi za rotacijo velja izrek o
transmutaciji\index{transmutacija!rotacije}.


        \bizrek \label{izoTransmRotac}
        For an arbitrary rotation $\mathcal{R}_{O,\alpha}$
        and an arbitrary isometry $\mathcal{I}$ is
        $$\mathcal{I}\circ
        \mathcal{R}_{O,\alpha}\circ\mathcal{I}^{-1}=
        \mathcal{R}_{\mathcal{I}(O),\alpha'},$$
         where: $\alpha'=\alpha$, if $\mathcal{I}$ a direct isometry, or
        $\alpha'=-\alpha$, if $\mathcal{I}$ is an opposite isometry.
        \eizrek

  \textbf{\textit{Proof.}}  Po izreku \ref{rotacKom2Zrc} lahko rotacijo
   $\mathcal{R}_{O,\alpha}$
  zapišemo kot kompozitum dveh osnih
zrcaljenj, in sicer
$\mathcal{R}_{O,\omega}=\mathcal{S}_q\circ\mathcal{S}_p$, kjer je
 $O=p\cap q$ in $\measuredangle pq=\frac{1}{2}\alpha$.
Če uporabimo izrek o transmutaciji za osna zrcaljenja, dobimo:
 \begin{eqnarray*}
  \mathcal{I}\circ
        \mathcal{R}_{O,\alpha}\circ\mathcal{I}^{-1}&=&
        \mathcal{I}\circ\mathcal{S}_q\circ\mathcal{S}_p\circ\mathcal{I}=\\
        &=&
        \mathcal{I}\circ\mathcal{S}_q\circ\mathcal{I}^{-1}
        \circ\mathcal{I}\circ\mathcal{S}_p\circ\mathcal{I}^{-1}=\\
        &=&\mathcal{S}_{q'}\circ\mathcal{S}_{p'}=\\
        &=&\mathcal{R}_{O',\alpha'},
 \end{eqnarray*}
kjer sta $p'$ in $q'$  sliki premic $p$ in $q$ pri izometriji
$\mathcal{I}$, $O'=p'\cap q'$ in $\alpha'=2\cdot\measuredangle
p'q'$. Ker je $O=p\cap q$, velja $\mathcal{I}(O)=
\mathcal{I}(p\cap q)=\mathcal{I}(p)\cap \mathcal{I}(q)=p'\cap
q'=O'$. Ker izometrije ohranjajo relacijo skladnosti kotov, je
 $\alpha'=2\cdot\measuredangle p'q'=2\cdot\measuredangle pq=\alpha$, če je
 $\mathcal{I}$ direktna, oz.
 $\alpha'=2\cdot\measuredangle p'q'=-2\cdot\measuredangle pq=-\alpha$, če je
 $\mathcal{I}$ indirektna.
 \kdokaz

 V naslednjem izreku bomo dokazali pomembno dejstvo, da za skladni daljici, ki nista vzporedni, obstaja rotacija, ki preslika prvo daljico v drugo.



        \bizrek
       Let $AB$ and $A'B'$ be congruent line segments that are not parallel.
        There is exactly one rotation that maps the line segment $AB$ to the line segment $A'B'$,
        such that the point $A$ maps to the point $A'$ and the point $B$ to the point $B'$.
        \eizrek

\begin{figure}[!htb]
\centering
\input{sl.skl.3.1.10Rotac.pic}
\caption{} \label{sl.skl.3.1.10Rotac.pic}
\end{figure}

 \textbf{\textit{Proof.}} Naj bo $S$ presečišče simetral daljic $AA'$ in $BB'$ ter $\omega=\measuredangle AB, A'B'$. Dokazali bomo, da je $\mathcal{R}_{S,\omega}$ iskana rotacija.
 Ker točka $S$ leži na simetralah daljic $AA'$ in $BB'$, je $SA\cong SA'$ in $SB\cong SB'$ (Figure \ref{sl.skl.3.1.10Rotac.pic}). Potem iz $AB\cong A'B'$ po izreku \textit{SSS} \ref{SSS} sledi $\triangle SAB\cong \triangle SA'B'$ (glej tudi zgled \ref{načrt1odd3}). Torej velja  $\measuredangle ASB\cong \measuredangle A'SB'$, oziroma:
 $$\measuredangle ASA'=\measuredangle ASB+\measuredangle BSA'=
 \measuredangle A'SB'+\measuredangle BSA'= \measuredangle BSB'.$$
 Po definiciji rotacije je $\mathcal{R}_{S,\measuredangle ASA'}:A,B\mapsto A',B'$. Po izreku \ref{rotacPremPremKot} je $\measuredangle ASA'=\measuredangle AB, A'B'=\omega$, zato je $\mathcal{R}_{S,\omega}:A,B\mapsto A',B'$.
 Ker rotacija kot izometrija ohranja relacijo $\mathcal{B}$, se daljica $AB$ s to rotacijo preslika v daljico $A'B'$.

Naj bo $\mathcal{R}_{\widehat{S},\widehat{\omega}}$ še ena rotacija, za katero velja $\mathcal{R}_{\widehat{S},\widehat{\omega}}:A,B\mapsto A',B'$. Potem je
$\mathcal{R}^{-1}_{S,\omega} \circ \mathcal{R}_{\widehat{S},\widehat{\omega}}$ direktna izometrija z dvema fiksnima točkama $A$ in $B$, zato po izreku \ref{izo2ftIdent} predstavlja identiteto $\mathcal{E}$  oz. velja
$\mathcal{R}_{\widehat{S},\widehat{\omega}} = \mathcal{R}_{S,\omega}$.
 \kdokaz


 V naslednjih primerih bomo uporabljali dejstvo, da je pozitivno orientiran
 trikotnik
 $ABC$ enakostraničen natanko tedaj, ko je
 $\mathcal{R}_{A,60^0}(B)=C$.



            \bzgled
            Let $P$ be an interior point of an equilateral triangle
            $ABC$.\\
          a) Prove that
          $|PA|+|PB|\geq |PC|.$\\
          b) Suppose that $\angle BPA=\mu$,
            $\angle CPA= \nu$ and $\angle BPC= \xi$. Calculate the interior angles of a triangle,
            with sides that are congruent to the line segments $PA$, $PB$ in $PC$.
           \ezgled

\begin{figure}[!htb]
\centering
\input{sl.izo.6.4.5.pic}
\caption{} \label{sl.izo.6.4.5.pic}
\end{figure}

 \textbf{\textit{Proof.}} Če je $P'=\mathcal{R}_{A,60^0}(P)$,
 je trikotnik $APP'$ pravilen (Figure \ref{sl.izo.6.4.5.pic}).
 Ker je še $\mathcal{R}_{A,60^0}(B)=C$, se
 daljica $BP$ pri tej rotaciji preslika v
skladno daljico $CP'$. Torej ima trikotnik $PP'C$ stranice, ki so
skladne z daljicami $PA$, $PB$ in $PC$. Po trikotniški neenakosti
- izrek \ref{neenaktrik} - je $|PA|+|PB|=|PP'|+|P'C|\geq PC.$

Izračunajmo še kote trikotnika $PP'C$. Rotacija
$\mathcal{R}_{A,60^0}$ preslika trikotnik $ABP$ v trikotnik
$ACP'$, zato sta trikotnika skladna in je $\angle BPA=\angle CP'A$. Iz
tega in iz izreka \ref{VsotKotTrik} sledi:
 \begin{eqnarray*}
  \angle CP'P &=& \angle CP'A-\angle PP'A=\angle
  BPA-60^0=\mu-60^0,\\
  \angle CPP' &=& \angle CPA-60^0=\nu-60^0,\\
  \angle PCP' &=& 180^0
  -(\mu-60^0)-(\nu-60^0)=300^0-(\mu+\nu)=\\
  &=& 300^0-(360^0-\xi)=\xi-60^0,
 \end{eqnarray*}
  kar je bilo treba izračunati.  \kdokaz



        \bzgled
        Let $ABCD$ be a rhombus, such that the interior angle at the vertex $A$
         is equal to $60^0$.
         If a line  $l$ intersects the sides $AB$ and $BC$  of this rhombus at  points $P$ and
        $Q$ such that  $|BP|+|BQ|=|AB|$, then $PQD$ is a regular triangle.
        \ezgled

\begin{figure}[!htb]
\centering
\input{sl.izo.6.4.6.pic}
\caption{} \label{sl.izo.6.4.6.pic}
\end{figure}

 \textbf{\textit{Proof.}} Iz $AB\cong AD$ in $\angle DAB=60^0$
 sledi, da je trikotnik $ABD$ pravilen (Figure \ref{sl.izo.6.4.6.pic}).
 Podobno je pravilen tudi
 trikotnik $BCD$. To pomeni, da $\mathcal{R}_{D,60^0}:A,B\mapsto
 B,C$. Ker je $|BP|+|BQ|=|AB|$ in $|BP|+|AP|=|AB|$, je $AP\cong BQ$.
 Toda rotacija $\mathcal{R}_{D,60^0}$ preslika poltrak $AB$
v poltrak $BC$, zato zaradi pogoja $AP\cong BQ$ preslika tudi
točko $P$ v točko $Q$. Torej je $\mathcal{R}_{D,60^0}(P)=Q$, kar
pomeni, da je trikotnik $DPQ$ pravilen.
 \kdokaz

Ob naslednjem zgledu bomo ilustrirali uporabo rotacije pri
načrtovalnih nalogah.



        \bzgled
        Let $A$ be an interior point of an angle $pOq$. Construct points
        $B$ and $C$ on the sides $p$ and $q$
        such that $ABC$ is a regular triangle.
        \ezgled

\begin{figure}[!htb]
\centering
\input{sl.izo.6.4.7.pic}
\caption{} \label{sl.izo.6.4.7.pic}
\end{figure}

 \textbf{\textit{Solution.}}  (Figure \ref{sl.izo.6.4.7.pic}).
Naj bo $ABC$ takšen pravilni trikotnik, da njegovi oglišči $B$ in
$C$ ležita na krakih $p$ in $q$. Potem je
$\mathcal{R}_{A,60^0}(B)=C$. Točka $B$ leži na poltraku $p$, zato
njena slika - točka $C$ - leži na poltraku
$p'=\mathcal{R}_{A,60^0}(p)$. Oglišče $C$ lahko dobimo kot
presečišče poltrakov $p'$ in $q$, nato pa oglišče $B$ kot
$B=\mathcal{R}_{A,-60^0}(C)$. Če se poltraka $p'$ in $q$ ne
sekata, ni rešitve. Če pa ležita na isti premici, ima naloga
 neskončno mnogo rešitev.
  \kdokaz

Če bi bil v prejšnji nalogi pogoj, da je trikotnik $ABC$
enakokraki in pravokoten s pravim kotom pri oglišču $A$,  bi
uporabili rotacijo s središčem $A$ za kot $45^0$. Podobno bi bilo,
če bi bilo potrebno načrtati kvadrat $PQRS$ s središčem v točki
$A$ pri pogoju $P\in p$, $Q\in q$ (Figure \ref{sl.izo.6.4.8.pic}).

\begin{figure}[!htb]
\centering
\input{sl.izo.6.4.8.pic}
\caption{} \label{sl.izo.6.4.8.pic}
\end{figure}



        \bnaloga\footnote{1. IMO Romania - 1959, Problem 5.}
        An arbitrary point $M$ is selected in the interior of the segment $AB$. The
        squares $AMCD$ and $MBEF$ are constructed on the same side of $AB$, with
        the segments $AM$ and $MB$ as their respective bases. The circles circumscribed
        about these squares, with centres $P$ and $Q$, intersect at $M$ and also
        at another point $N$. Let $N'$ denote the point of intersection of the straight
        lines $AF$ and $BC$.

        (a) Prove that the points $N$ and $N'$ coincide.

        (b) Prove that the straight lines $MN$ pass through a fixed point $S$ independent
            of the choice of $M$.

        (c) Find the locus of the midpoints of the segments $PQ$ as $M$ varies between
            $A$ and $B$
        \enaloga



\begin{figure}[!htb]
\centering
\input{sl.izo.6.4.IMO1.pic}
\caption{} \label{sl.izo.6.4.IMO1.pic}
\end{figure}

 \textbf{\textit{Solution.}} Označimo s $k$ in $l$
 očrtani krožnici kvadratov
  $AMCD$ in $MBEF$
 (Figure \ref{sl.izo.6.4.IMO1.pic}).

 (\textit{i}) Rotacijo $\mathcal{R}_{M,-90^0}$ preslika točki
 $A$ in $F$ po vrsti v točki $C$ in $B$ oz. premico $AF$ v
 premico $CB$. Po izreku \ref{rotacPremPremKot}
   ti dve premici določata kot rotacije,
 zato je $AF\perp CB$. Torej $\angle AN'C\cong\angle BN'F =90^0$, kar pomeni
 (izrek
  \ref{TalesovIzrKroz}), da točka $N'$ leži na obeh krožnicah
  $k$ in $l$
  s premeroma $AC$ in $BF$. Zaradi tega je
  $N'=N$.

 (\textit{ii}) Naj bo $j$ krožnica nad premerom $AB$ in $S$ središče
 tistega loka (polkrožnice) $AB$, ki je na nasprotnem bregu premice
 $AB$ glede na kvadrata $AMCD$ in $MBEF$. Točka $S$ ni odvisna od
 izbire točke $M$. Dokažimo, da gredo vse premice $MN$ skozi točko $S$.
  Ker je $\angle ANB=90^0$, po izreku \ref{TalesovIzrKroz} tudi točka $N$ leži
  na krožnici $j$. Iz izreka \ref{ObodObodKot}
  sledi $\angle ANM =\angle ADM=45^0$ in $\angle MNB =\angle
  MEB=45^0$, zato je $NM$ simetrala kota $ANB$. Po izreku
  \ref{TockaN} simetrala $NM$ poteka skozi točko $S$.


\begin{figure}[!htb]
\centering
\input{sl.izo.6.4.IMO1a.pic}
\caption{} \label{sl.izo.6.4.IMO1a.pic}
\end{figure}

 (\textit{iii}) Naj bo $O$ središče daljice $PQ$.
 Označimo s $P'$, $Q'$ in $O'$ pravokotne projekcije točk
 $P$, $Q$ in $O$ na premici $AB$ (Figure
 \ref{sl.izo.6.4.IMO1a.pic}). Daljica $OO'$ je srednjica pravokotnega trapeza
 $P'Q'QP$, zato je po izreku \ref{srednjTrapez}:
  $$|OO'|=\frac{1}{2}\left(|PP'|+|QQ'| \right)=
   \frac{1}{2}\left(\frac{1}{2}|AM|+\frac{1}{2}|MB| \right)=\frac{1}{4}|AB|.$$
 Torej je razdalja točke $O$ od premice $AB$ konstantna oziroma
 neodvisna od izbire točke $M$.

  Naj bo $L$ središče kvadrata $ABGH$, ki je na istem bregu
  premice $AB$ kot kvadrata $AMCD$ in $MBEF$. Z $O_A$ in $O_B$
  označimo središča daljic $LA$ in $LB$. Ker je $O_AO_B$
  srednjica trikotnika $ALB$ z osnovnico $AB$, je $O_AO_B \parallel
  AB$.
  Iz $|AO_A|=\frac{1}{2}|AL|=\frac{1}{4}|AG|$ pa sledi, da je
  razdalja vzporednic $O_AO_B$ in $AB$ enaka $\frac{1}{4}|AB|$. Iz
  že dokazanega dejstva $d(O,AB)=\frac{1}{4}|AB|$ sledi, da točka
  $O$ leži na premici $O_AO_B$. Toda, ko se točka $M$ giblje po
  notranjosti daljice $AB$, se točki $P$ in $Q$ gibljeta po
  notranjosti daljic $AL$ oz. $BL$. To pomeni, da je točka $O$ v
  notranjosti trikotnika $ALB$ oz. $O$ leži na odprti daljici
  $O_AO_B$.

  Dokažimo še, da je poljubna točka $O$ daljice $O_AO_B$
  središče neke daljice $PQ$ za določeno izbiro  kvadratov
   $AMCD$ in $MBEF$ oz. točke  $M$. Točko $M$ v tem primeru
   dobimo iz pogoja $|AO'|=\frac{1}{2}|AM|+\frac{1}{4}|AB|$ oz.
   $|AM|=2|AO'|-\frac{1}{2}|AB|$. Takšna točka $M$ vedno
   obstaja, če je $\frac{1}{4}|AB|<|AO'|<\frac{3}{4}|AB|$, oz.
   kadar točka $O$ leži na odprti daljici $O_AO_B$.

 Iskano geometrijsko mesto točk $M$ je torej daljica $O_AO_B$.
   \kdokaz


%________________________________________________________________________________
  \poglavje{Half-Turn} \label{odd6SredZrc}

Rotacijo za kot $180^0$ bomo obravnavali kot
posebno vrsto izometrij.

Rotacijo $\mathcal{R}_{S,\omega}$ ravnine za kot $\omega=180^0$
imenujmo \index{zrcaljenje!središčno}\pojem{središčno zrcaljenje}
ali \index{zrcaljenje!čez točko}\pojem{zrcaljenje čez točko} s
\index{središče!središčnega zrcaljenja}\pojem{središčem} $S$ in ga
označimo $\mathcal{S}_S$ (Figure \ref{sl.izo.6.5.1.pic}). Torej
$\mathcal{S}_S=\mathcal{R}_{S,180^0}$.


\begin{figure}[!htb]
\centering
\input{sl.izo.6.5.1.pic}
\caption{} \label{sl.izo.6.5.1.pic}
\end{figure}

Iz definicije je jasno, da za poljubno točko $X\neq S$ velja
$\mathcal{S}_S(X)=X'$ natanko tedaj, ko je $S$ središče daljice
$XX'$.

Direktno iz definicije sledi tudi naslednja trditev.

           \bizrek \label{izoSredZrcInv}
           A half-turn is an involution, i.e.
           $\mathcal{S}^2_S=\mathcal{E}$. \eizrek

  Ker je središčno zrcaljenje vrsta rotacije, ima tudi vse
  lastnosti rotacije. Edina fiksna točka
  središčnega zrcaljenja je torej središče tega zrcaljenja.

 Enako kot rotacijo lahko tudi središčno zrcaljenje
 $\mathcal{S}_S$
 predstavimo kot kompozitum dveh osnih zrcaljenj. Kot, ki ga osi
 določata, je enak polovici kota rotacije (izrek \ref{rotacKom2Zrc}) -
 v našem primeru je
 to $\frac{180^0}{2}=90^0$. To pomeni, da sta osi dveh
 zrcaljenj v primeru središčnega zrcaljenja pravokotni. V tem primeru po
 izreku \ref{izoZrcKomut} osni zrcaljenji komutirata
  (Figure \ref{sl.izo.6.5.2.pic}).
 Torej velja naslednji izrek.

\begin{figure}[!htb]
\centering
\input{sl.izo.6.5.2.pic}
\caption{} \label{sl.izo.6.5.2.pic}
\end{figure}



        \bizrek \label{izoSrZrcKom2Zrc}
        Any half-turn around a point  $S$
         can be expressed as the product of two reflections
          $\mathcal{S}_p$ and $\mathcal{S}_q$ where
          $p$ and $q$  are arbitrary perpendicular lines intersecting at the point $S$.\\
          The reverse is also true -
         the product of two reflections
          $\mathcal{S}_p$ and $\mathcal{S}_q$ ($S=p\cap q$ and
         $p\perp q$)
         is the half-turn around the point $S$, i.e.\\
         $$\mathcal{S}_S=\mathcal{S}_q\circ\mathcal{S}_p=
         \mathcal{S}_p\circ\mathcal{S}_q
         \hspace*{1mm} \Leftrightarrow  \hspace*{1mm} S=p\cap q \hspace*{1mm}
         \wedge\hspace*{1mm} p\perp q.$$
        \eizrek



        \bizrek \label{izoKomp3SredZrc}
        The product of three half-turns is a half-turn.
       If the centres of these half-turns are three non-collinear points,
        then the centre of the new half-turn is the fourth vertex of a parallelogram.
        \eizrek


\begin{figure}[!htb]
\centering
\input{sl.izo.6.5.3.pic}
\caption{} \label{sl.izo.6.5.3.pic}
\end{figure}

 \textbf{\textit{Proof.}}
 Naj bodo $\mathcal{S}_A$, $\mathcal{S}_B$ in $\mathcal{S}_C$ tri središčne
simetrije s središči $A$, $B$ in $C$ (Figure
\ref{sl.izo.6.5.3.pic}). Označimo s $p$ premico $AB$, z $a$, $b$
in $c$ po vrsti pravokotnice premice $p$ skozi točke $A$, $B$ in
$C$ ter s $c'$ pravokotnico premice $c$ skozi točko $C$. Potem je
po izreku \ref{izoSrZrcKom2Zrc}:
 $$\mathcal{S}_C \circ \mathcal{S}_B \circ \mathcal{S}_A=
  \mathcal{S}_{c'} \circ \mathcal{S}_c \circ \mathcal{S}_b
  \circ \mathcal{S}_p \circ \mathcal{S}_p \circ \mathcal{S}_a
  = \mathcal{S}_{c'} \circ \mathcal{S}_c \circ \mathcal{S}_b \circ \mathcal{S}_a.$$
   Osi $a$, $b$ in $c$ pripadajo  istem šopu vzporednic $\mathcal{X}_a$,
ker so vse pravokotne na premico $p$. Kompozitum
$\mathcal{S}_c \circ \mathcal{S}_b \circ \mathcal{S}_a$ torej
predstavlja osno zrcaljenje $S_d$, kjer os $d$ pripada istem šopu
$\mathcal{X}_a$ (izrek \ref{izoSop}). Potem sta tudi premici $c'$
in $d$ pravokotni, zato je:
 $$\mathcal{S}_C \circ \mathcal{S}_B \circ \mathcal{S}_A=
 \mathcal{S}_{c'} \circ \mathcal{S}_d=\mathcal{S}_D,$$
kjer je $D=d\cap c'$. Po izreku \ref{izoSoppqrt} imata para premic $a$,
$c$ in $b$, $d$ skupno somernico. To pomeni, da če
so $A$, $B$ in $C$ tri nekolinearne točke, je štirikotnik $ABCD$
paralelogram.
 \kdokaz

Posledica dokazane trditve je naslednji izrek.




        \bizrek \label{izoKomp2n+1SredZrc}
        The product of an odd number of half-turns is a half-turn.
        \eizrek

 \textbf{\textit{Proof.}} Dokaz bomo izpeljali z indukcijo po
 številu $n\geq 1$, kjer je $m=2n+1$ (liho) število točk.

 Za $n=1$, oz. $m=3$, je trditev direktna posledica prejšnjega izreka
 \ref{izoKomp3SredZrc}.

 Predpostavimo, da za vsak $k\in \mathbb{N}$ ($k\geq 1$)
 in vsako $(2k+1)$-terico
 komplanarnih točk
 $A_1,A_2,\ldots ,A_{2k+1}$  kompozitum
 $\mathcal{S}_{A_{2k+1}}\circ
 \cdots \circ\mathcal{S}_{A_2}\circ\mathcal{S}_{A_1}$ predstavlja
 neko središčno zrcaljenje $\mathcal{S}_A$ (indukcijska predpostavka).

 Potrebno je dokazati, da  potem tudi za $k+1$
 in vsako $(2(k+1)+1)$-terico
 komplanarnih točk
 $A_1,A_2,\ldots ,A_{(2(k+1)+1)}$ kompozitum
 $\mathcal{I}=\mathcal{S}_{A_{2(k+1)+1}}\circ
 \cdots \circ\mathcal{S}_{A_2}\circ\mathcal{S}_{A_1}$ predstavlja
 neko središčno zrcaljenje $\mathcal{S}_{A'}$. Torej:

   \begin{eqnarray*}
   \mathcal{I}&=&\mathcal{S}_{A_{2(k+1)+1}}\circ
 \cdots \circ\mathcal{S}_{A_2}\circ\mathcal{S}_{A_1}=\\
   &=&\mathcal{S}_{A_{2k+3}}\circ\mathcal{S}_{A_{2k+2}}\circ
   \mathcal{S}_{A_{2k+1}}\circ
 \cdots \circ\mathcal{S}_{A_2}\circ\mathcal{S}_{A_1}=\\
 &=& \mathcal{S}_{A_{2k+3}}\circ\mathcal{S}_{A_{2k+2}}\circ
   \mathcal{S}_A= \hspace*{14mm} \textrm{ (po indukcijski predpostavki)}\\
   &=&  \mathcal{S}_{A'} \hspace*{48mm} \textrm{ (po izreku \ref{izoKomp3SredZrc}),}
   \end{eqnarray*}
  kar je bilo treba dokazati.  \kdokaz

Direktna posledica izreka \ref{izoTransmRotac}, ki se nanaša na rotacijo, je naslednja trditev ozirma izrek o transmutaciji središčnega zrcaljenja\index{transmutacija!središčnega zrcaljenja}.


        \bizrek \label{izoTransmSredZrc}
        For an arbitrary half-turn
        $\mathcal{S}_{O}$
         and an arbitrary isometry $\mathcal{I}$ is
        $$\mathcal{I}\circ
        \mathcal{S}_{O}\circ\mathcal{I}^{-1}=
        \mathcal{R}_{\mathcal{I}(O)}.$$
        \eizrek


 Podobno kot osno zrcaljenje tudi središčno
 zrcaljenje pogosto uporabljamo pri načrtovalnih
 nalogah.

        \bzgled
        Let $S$ be an interior point of an angle $pOq$. Construct a square
         $ABCD$ with the centre $S$ such that the vertices
        $A$ and $C$ lie on the sides $p$ and $q$.
        \ezgled

\begin{figure}[!htb]
\centering
\input{sl.izo.6.5.4.pic}
\caption{} \label{sl.izo.6.5.4.pic}
\end{figure}

 \textbf{\textit{Proof.}} Naj bo $ABCD$ kvadrat, ki izpolnjuje dane pogoje -
 $S$ je njegovo središče, $A\in p$ in $C\in q$ (Figure \ref{sl.izo.6.5.4.pic}).
   Ker je točka $S$ središče diagonale
  $AC$, je $\mathcal{S}_S(A)=C$. Iz $A\in p$ sledi $C\in
  \mathcal{S}_S(p)$.
  Točko $C$ torej dobimo iz pogoja $C\in q\cap\mathcal{S}_S(p)$.
  Na koncu je še $A=\mathcal{S}_S(C)$, $D=\mathcal{R}_{S,90^0}(C)$ in
  $B=\mathcal{R}_{S,-90^0}(C)$.
 \kdokaz

            \bzgled
              Let $A$ be one of the two intersections of circles $k$ and $l$. Construct
            a common secant of these two circles passing through the point $A$ and determine
            a two congruent chord with these circles.
           \ezgled


\begin{figure}[!htb]
\centering
\input{sl.izo.6.5.5.pic}
\caption{} \label{sl.izo.6.5.5.pic}
\end{figure}

 \textbf{\textit{Solution.}}
   Označimo z $B$ drugo presečišče krožnic $k$ in $l$
    (Figure \ref{sl.izo.6.5.5.pic}). Potem je ena rešitev naloge
    premica $AB$, saj je daljica $AB$
   skupna tetiva obeh krožnic.

  Naj bo $s\ni A$ skupna sekanta krožnic $k$ in $l$, ki ju seka še
  v točkah $K$ in $L$, tako da velja $AK\cong AL$. V primeru $K=L$
  je $K=L=B$, kar je že omenjena rešitev. Če je $K\neq L$, iz
  pogoja $A, K, L\in s$ sledi, da je $A$ središče daljice $KL$.
  Torej je $\mathcal{S}_A(K)=L$, kar pomeni, da točko $L$ dobimo
  iz pogoja $L\in l\cap \mathcal{S}_A(k)$. Na koncu je
  $K=\mathcal{S}_A(L)$.
   \kdokaz


            \bzgled
           Construct a quadrilateral $ABCD$ with the sides that are congruent to
             the four given line segments $a$, $b$, $c$, and $d$ and
            the line segment defined by the midpoints of
            the two opposite sides $AD$ and $BC$ that is congruent to the given line segment $l$.
           \ezgled

\begin{figure}[!htb]
\centering
\input{sl.izo.6.5.6.pic}
\caption{} \label{sl.izo.6.5.6.pic}
\end{figure}

 \textbf{\textit{Solution.}}
Naj bosta $P$ in $Q$ središči stranic $AD\cong d$ in $BC\cong b$
štirikotnika $ABCD$, tako da velja še $PQ\cong l$, $AB\cong a$ in
$CD\cong c$ (Figure \ref{sl.izo.6.5.6.pic}). Naj bo $S$
središče diagonale $AC$. Trikotnik $PQS$ lahko konstruiramo, saj
je $PQ\cong l$, $QS=\frac{1}{2}a$ in $PS =\frac{1}{2}c$ (izrek
\ref{srednjicaTrik}). Točki $C$ in $A$ ležita po vrsti na
krožnicah $k_d=k(P,\frac{1}{2}d)$ in $k_b=k(Q,\frac{1}{2}b)$. Ker
je $C=\mathcal{S}_S(A)$, točka $C$ leži tudi na sliki $k'_d$
krožnice $k_d$ pri središčnem zrcaljenju $\mathcal{S}_S$.
Točka $C$ je torej eno od presečišč krožnic $k'_d$ in $k_b$. Nato je še
$A=\mathcal{S}_S(C)$, $D=\mathcal{S}_P(A)$ in
$B=\mathcal{S}_Q(C)$.
 \kdokaz


        \bzgled
        Let $P$, $Q$, $R$, $S$ and $T$ be points in the plane. Construct
        a pentagon $ABCDE$ such that the points $P$, $Q$, $R$, $S$ and $T$ are
        the midpoints of its sides $AB$, $BC$, $CD$, $DE$ and $EA$, respectively.
        \ezgled


\begin{figure}[!htb]
\centering
\input{sl.izo.6.5.7.pic}
\caption{} \label{sl.izo.6.5.7.pic}
\end{figure}

 \textbf{\textit{Solution.}}
Naj bo $ABCDE$ iskani petkotnik
    (Figure \ref{sl.izo.6.5.7.pic}) in:
 $$\mathcal{I} = \mathcal{S}_T \circ \mathcal{S}_S
  \circ \mathcal{S}_R \circ \mathcal{S}_Q \circ \mathcal{S}_P.$$
  Iz izreka izreka \ref{izoKomp2n+1SredZrc} sledi, da je izometrija
  $\mathcal{I}$ neko središčno
zrcaljenje $\mathcal{S}_O$. Ker je
$\mathcal{S}_O(A)=\mathcal{I}(A)=A$, je $O=A$  oz. $\mathcal{I}=\mathcal{S}_A$. Točko
$A$ torej lahko konstruiramo kot središče daljice $XX'$, kjer je
$X$ poljubna točka dane ravnine in $X'= \mathcal{I}(X)$.
  \kdokaz


        \bzgled
        Let $A_iB_i$ ($i\in \{1,2,3\}$)
         be parallel chords of a circle $k$. Suppose that $S_1$, $S_2$ and $S_3$
          are the midpoints of the line segments $A_2A_3$, $A_1A_3$ and $A_1A_2$,
           respectively. Prove that
         $C_i=\mathcal{S}_{S_i}(B_i)$ are three collinear points.
        \ezgled

\begin{figure}[!htb]
\centering
\input{sl.izo.6.5.4a.pic}
\caption{} \label{sl.izo.6.5.4a.pic}
\end{figure}

\textbf{\textit{Proof.}} Naj bo $S$ središče krožnice $k$
(Figure \ref{sl.izo.6.5.4a.pic}). Ker je $SA_i\cong SB_i$ ($i\in
\{1,2,3\}$), potekajo vse simetrale $s_i$ tetiv $A_iB_i$ ($i\in
\{1,2,3\}$) skozi točko $S$. Iz $s_i\perp A_iB_i$ in dejstva, da
so tetive medseboj vzporedne, sledi, da so vse simetrale $s_i$
med seboj enake - označimo jih s $s$. Torej je $s$ skupna simetrala
tetiv $A_iB_i$, zato velja $\mathcal{S}_s(A_i)=B_i$  ($i\in
\{1,2,3\}$).

 Ker so $S_1$, $S_2$ in $S_3$
  središča daljic $C_1B_1$, $C_2B_2$ in $C_3B_3$ ter stranic $A_2A_3$, $A_1A_3$ in $A_1A_2$
   trikotnika $A_1A_2A_3$, je po izreku \ref{vektSestSplosno}
$\overrightarrow{S_1S_2}=
\frac{1}{2}(\overrightarrow{C_1C_2}+\overrightarrow{B_1B_2})$ oz.
$\overrightarrow{C_1C_2}=
2\overrightarrow{S_1S_2}-\overrightarrow{B_1B_2}
=\overrightarrow{A_2A_1}-\overrightarrow{B_1B_2}=\overrightarrow{A_2A_1}
+\overrightarrow{B_2B_1}$.
 Ker se točki $A_2$ in $A_1$ s premico $s$
prezrcalita v točki $B_2$ in $B_1$, je vektor
$\overrightarrow{C_1C_2}=\overrightarrow{A_2A_1}
+\overrightarrow{B_2B_1}$ kolinearen s premico $s$ (zgled
\ref{izoSimVekt}). Analogno je tudi vektor $\overrightarrow{C_2C_3}$
kolinearen s premico $s$, zatorej točke $C_1$, $C_2$ in $C_3$
ležijo na isti premici, ki je vzporedna s premico $s$.
 \kdokaz



%________________________________________________________________________________
 \poglavje{Translations} \label{odd6Transl}

 Do sedaj smo obravnavali izometrije, ki imajo vsaj eno fiksno točko. Od direktnih smo imeli identiteto in rotacijo, od indirektnih pa osno zrcaljenje. Sedaj bomo vpeljali novo vrsto izometrij, ki nimajo fiksnih točk.

  Naj bo $\overrightarrow{v}$
 poljubni vektor evklidske ravnine ($\overrightarrow{v}\neq \overrightarrow{0}$). Transformacija te ravnine, pri kateri se
točka $X$  preslika v takšno točko $X'$, da je $\overrightarrow{XX'}=\overrightarrow{v}$, imenujemo \index{translacija} \pojem{translacija} ali \index{vzporedni premik} \pojem{vzporedni premik} za
vektor $\overrightarrow{v}$ in jo označimo s $\mathcal{T}_{\overrightarrow{v}}$ (Figure \ref{sl.izo.6.6.1.pic}). Vektor  $\overrightarrow{v}$ je\index{vektor!translacije} \pojem{vektor translacije}.

\begin{figure}[!htb]
\centering
\input{sl.izo.6.6.1.pic}
\caption{} \label{sl.izo.6.6.1.pic}
\end{figure}

Dokažimo prve osnovne lastnosti translacije.



        \bizrek \label{translEnaki}
        Two translations are equal if and only
        the vectors of these translations are equal, i.e.
        $$\mathcal{T}_{\overrightarrow{v}}=
        \mathcal{T}_{\overrightarrow{u}}\Leftrightarrow \overrightarrow{v}=\overrightarrow{u}.$$
        \eizrek

 \textbf{\textit{Proof.}} Del ($\Leftarrow$) je trivialen. Dokažimo del ($\Rightarrow$). Za poljubno točko $X$ je torej
 $\mathcal{T}_{\overrightarrow{v}}(X)=
 \mathcal{T}_{\overrightarrow{u}}(X)=X'$. V tem primeru je
 $\overrightarrow{v}=\overrightarrow{u}=\overrightarrow{XX'}$.
  \kdokaz


        \bizrek
        A translation has no fixed points.
        \eizrek

\textbf{\textit{Proof.}} Če je $X$ fiksna točka translacije  $\mathcal{T}_{\overrightarrow{v}}$ oz. $\mathcal{T}_{\overrightarrow{v}}(X)=X$, je $\overrightarrow{v}=\overrightarrow{XX}=\overrightarrow{0}$, kar po definiciji translacije ni možno.
\kdokaz


        \bizrek
        The inverse transformation of a translation is a translation
        with the opposite vector, i.e.
         $\mathcal{T}^{-1}_{\overrightarrow{v}}=
         \mathcal{T}_{-\overrightarrow{v}}$
        \eizrek

\textbf{\textit{Proof.}} Naj bo $Y$ poljubna točka in $X$ takšna točka, da velja $\overrightarrow{XY}=\overrightarrow{v}$. Iz tega sledi $\mathcal{T}_{\overrightarrow{v}}(X)=Y$. Iz $\overrightarrow{YX}=-\overrightarrow{v}$ pa sledi $\mathcal{T}_{-\overrightarrow{v}}(Y)=X$. Ker je še $\mathcal{T}^{-1}_{\overrightarrow{v}}(Y)=X$, sledi $\mathcal{T}^{-1}_{\overrightarrow{v}}(Y)=
 \mathcal{T}_{-\overrightarrow{v}}(Y)$. Ker to velja za poljubno točko $Y$, je $\mathcal{T}^{-1}_{\overrightarrow{v}}=
 \mathcal{T}_{-\overrightarrow{v}}$.
  \kdokaz

        \bizrek
         Translations are isometries of the plane.
         \eizrek


\begin{figure}[!htb]
\centering
\input{sl.izo.6.6.2.pic}
\caption{} \label{sl.izo.6.6.2.pic}
\end{figure}

 \textbf{\textit{Proof.}} Naj bo $\mathcal{T}_{\overrightarrow{v}}$ poljubna translacija. Iz definicije je jasno, da predstavlja bijektivno preslikavo. Potrebno je še dokazati, da za poljubni točki $X$ in $Y$
 ter njuni sliki $\mathcal{T}_{\overrightarrow{v}}:X,Y\mapsto X', Y'$ velja
 $X'Y'\cong XY$ (Figure \ref{sl.izo.6.6.2.pic}).
 Iz $\overrightarrow{XX'}=\overrightarrow{v}$ in $\overrightarrow{YY'}=\overrightarrow{v}$ sledi $\overrightarrow{XX'}=\overrightarrow{YY'}$. Po izreku \ref{vektABCD_ACBD} je potem tudi $\overrightarrow{XY}=\overrightarrow{X'Y'}$ oz. $X'Y'\cong XY$.
  \kdokaz

Sedaj bomo pokazali, da lahko vsako translacijo, podobno kot rotacijo, predstavimo kot
kompozitum dveh osnih zrcaljenj.



        \bizrek \label{translKom2Zrc}
          Any translation $\mathcal{T}_{\overrightarrow{v}}$
         can be expressed as the product of two reflections
          $\mathcal{S}_p$ and $\mathcal{S}_q$ where
          $p$ and $q$  are arbitrary parallel lines with a common
          perpendicular line ($P\in p$ and $Q\in q$)
          such that $\overrightarrow{v}
         =2\overrightarrow{PQ}$, i.e.
         $$\mathcal{T}_{\overrightarrow{v}}=\mathcal{S}_q\circ\mathcal{S}_p
         \hspace*{1mm} \Leftrightarrow  \hspace*{1mm} p\parallel q \hspace*{1mm}
         \wedge\hspace*{1mm} \overrightarrow{v}
         =2\overrightarrow{PQ}\hspace*{1mm} (PQ\perp p,\hspace*{1mm}
          P\in p,\hspace*{1mm} Q\in q).$$
        \eizrek


\begin{figure}[!htb]
\centering
\input{sl.izo.6.6.3.pic}
\caption{} \label{sl.izo.6.6.3.pic}
\end{figure}

 \textbf{\textit{Proof.}}  (Figure \ref{sl.izo.6.6.3.pic})

 ($\Leftarrow$) Predpostavimo, da velja $p\parallel q$, $\overrightarrow{v}=2\overrightarrow{PQ}$, $PQ\perp p$,
         $P\in p$ in $Q\in q$.
 Potrebno je dokazati, da velja
     $\mathcal{T}_{\overrightarrow{v}}(X)=
     \mathcal{S}_q\circ\mathcal{S}_p(X)$ za poljubno točko $X$ te ravnine.
 Naj bo $\mathcal{S}_q\circ\mathcal{S}_p(X)=X'$. Dokažimo, da je tudi $\mathcal{T}_{\overrightarrow{v}}(X)=X'$ oz. $\overrightarrow{XX'}= \overrightarrow{v}$.
 Označimo
$\mathcal{S}_p(X)=X_1$. V tem primeru je jasno  $\mathcal{S}_q(X_1)=X'$. Zaradi tega je najprej premica $XX'$ skupna pravokotnica
vzporednic $p$ in $q$, kar pomeni, da je vektor $\overrightarrow{XX'}$ vzporeden z vektorjem $\overrightarrow{v}$. Ne glede na lego točke $X$ je v
vsakem primeru  dolžina daljice $XX'$ dvakrat večja od razdalje med
vzporednicama $p$ in $q$, ta pa je enaka dolžini daljice $PQ$. Torej je $\overrightarrow{XX'} =2\overrightarrow{PQ}= \overrightarrow{v}$. S tem smo dokazali, da je
$\mathcal{T}_{\overrightarrow{v}}(X)=X'$.

($\Rightarrow$) Naj bo sedaj $\mathcal{T}_{\overrightarrow{v}}=\mathcal{S}_q\circ\mathcal{S}_p$.
Premici $p$ in $q$ sta v tem primeru vzporedni, ker bi bila sicer skupna točka teh dveh premic fiksna tačka kompozituma $\mathcal{S}_q\circ\mathcal{S}_p$, translacija $\mathcal{T}_{\overrightarrow{v}}$ pa nima nobene fiksne točke.
Iz prvega dela dokaza ($\Leftarrow$) je kompozitum $\mathcal{S}_q\circ\mathcal{S}_p$ enak translaciji $\mathcal{T}_{\overrightarrow{v_1}}$ za vektor $\overrightarrow{v_1}$, kjer velja $\overrightarrow{v_1}=2\overrightarrow{PQ}$, $PQ\perp p$,
         $P\in p$ in $Q\in q$. Toda iz
$\mathcal{T}_{\overrightarrow{v}}=
\mathcal{S}_q\circ\mathcal{S}_p=\mathcal{T}_{\overrightarrow{v_1}}$
 sledi $\overrightarrow{v}=\overrightarrow{v_1}$ (izrek \ref{translEnaki}) oz. $\overrightarrow{v}=2\overrightarrow{PQ}$.
  \kdokaz

Iz prejšnjega izreka sledi, da je translacija kot kompozitum
dveh indirektnih izometrij direktna izometrija.


        \bizrek
         A translation is a direct isometry.
        \eizrek

Dokazali bomo, da lahko translacijo predstavimo tudi kot
kompozicijo dveh središčnih zrcaljenj.

        \bizrek \label{transl2sred}
        The product of two half-turns is a translation, such that
        $$\mathcal{S}_Q\circ\mathcal{S}_P=
        \mathcal{T}_{2\overrightarrow{PQ}}.$$
        \eizrek

\begin{figure}[!htb]
\centering
\input{sl.izo.6.6.4.pic}
\caption{} \label{sl.izo.6.6.4.pic}
\end{figure}

 \textbf{\textit{Proof.}} Naj bo $X$ poljubna točka, $\mathcal{S}_Q\circ\mathcal{S}_P(X)=X'$ in $\mathcal{S}_P(X)=X_1$ (Figure \ref{sl.izo.6.6.4.pic}). Potem je tudi $\mathcal{S}_Q(X_1)=X'$. Po definiciji središčnih zrcaljenj sta točki $P$ in $Q$ središči daljic $XX_1$ in $X_1X'$, kar pomeni, da je daljica $PQ$ srednjica trikotnika $XX_1X'$ za osnovnico $XX'$. Po izreku \ref{srednjicaTrikVekt} je
  $\overrightarrow{XX'}=2\overrightarrow{PQ}$, zato je $\mathcal{T}_{2\overrightarrow{PQ}}(X)=X'=
  \mathcal{S}_Q\circ\mathcal{S}_P(X)$. Ker to velja za vsako točko $X$, je $\mathcal{S}_Q\circ\mathcal{S}_P=
        \mathcal{T}_{2\overrightarrow{PQ}}$.
        \kdokaz



        \bizrek \label{translKomp}
        The product of two translations is a translation for the vector
         that is the sum of the vectors of these two translations, i.e.
        $$\mathcal{T}_{\overrightarrow{u}}\circ
        \mathcal{T}_{\overrightarrow{v}}=
        \mathcal{T}_{\overrightarrow{v}+\overrightarrow{u}}.$$
        \eizrek

\begin{figure}[!htb]
\centering
\input{sl.izo.6.6.5.pic}
\caption{} \label{sl.izo.6.6.5.pic}
\end{figure}

 \textbf{\textit{Proof.}}  (Figure \ref{sl.izo.6.6.5.pic}).
 Naj bo $P$ poljubna točka, $Q$ takšna točka, da velja
 $\overrightarrow{PQ} = \frac{1}{2}\overrightarrow{v}$,
in $R$ točka, za katero je
$\overrightarrow{QR} = \frac{1}{2}\overrightarrow{u}$.
 Ker velja $\overrightarrow{v} + \overrightarrow{u}
 = 2\overrightarrow{PQ} + 2\overrightarrow{QR} = 2\overrightarrow{PR}$,
 po prejšnjem izreku \ref{transl2sred} sledi:
 $$\mathcal{T}_{\overrightarrow{u}}\circ
        \mathcal{T}_{\overrightarrow{v}}=
        \mathcal{T}_{2\overrightarrow{QR}}\circ
        \mathcal{T}_{2\overrightarrow{PQ}}=
        \mathcal{S}_R\circ\mathcal{S}_Q\circ
        \mathcal{S}_Q\circ\mathcal{S}_P=
        \mathcal{S}_R\circ\mathcal{S}_P=
        \mathcal{T}_{2\overrightarrow{PR}}=
        \mathcal{T}_{\overrightarrow{v}+\overrightarrow{u}},$$ kar je bilo treba dokazati. \kdokaz

 Posledica dokazanega je naslednji izrek.

        \bizrek
       The product of translations is commutative, i.e.
        $$\mathcal{T}_{\overrightarrow{u}}\circ
        \mathcal{T}_{\overrightarrow{v}}=
        \mathcal{T}_{\overrightarrow{v}}\circ
        \mathcal{T}_{\overrightarrow{u}}$$
        \eizrek

  \textbf{\textit{Proof.}} Po prejšnjem izreku \ref{translKomp} je:
  $$\mathcal{T}_{\overrightarrow{u}}\circ
        \mathcal{T}_{\overrightarrow{v}}=
        \mathcal{T}_{\overrightarrow{v}+\overrightarrow{u}}=
        \mathcal{T}_{\overrightarrow{u}+\overrightarrow{v}}=
        \mathcal{T}_{\overrightarrow{v}}\circ
        \mathcal{T}_{\overrightarrow{u}},$$ kar je bilo treba dokazati. \kdokaz


Podobno kot pri osnem zrcaljenju, rotaciji in središčnem zrcaljenju tudi za translacijo velja izrek o
transmutaciji\index{transmutacija!translacije}.


        \bizrek \label{izoTransmTrans}
        For an arbitrary  translation
        $\mathcal{T}_{2\overrightarrow{PQ}}$
         and an arbitrary isometry $\mathcal{I}$ is
        $$\mathcal{I}\circ
        \mathcal{T}_{2\overrightarrow{PQ}}\circ\mathcal{I}^{-1}=
        \mathcal{T}_{2\overrightarrow{\mathcal{I}(P)\mathcal{I}(Q)}}.$$
        \eizrek

  \textbf{\textit{Proof.}}  Po izreku \ref{transl2sred} lahko translacijo
   $\mathcal{T}_{2\overrightarrow{PQ}}$
  zapišemo kot kompozitum dveh središčnih
zrcaljenj, in sicer
$\mathcal{T}_{2\overrightarrow{PQ}}=\mathcal{S}_Q\circ\mathcal{S}_P$.
Če uporabimo izrek  o transmutaciji za središčne simetrije \ref{izoTransmSredZrc}, dobimo:
 \begin{eqnarray*}
  \mathcal{I}\circ
        \mathcal{T}_{2\overrightarrow{PQ}}\circ\mathcal{I}^{-1}&=&
        \mathcal{I}\circ\mathcal{S}_Q\circ\mathcal{S}_P\circ\mathcal{I}=\\
        &=&
        \mathcal{I}\circ\mathcal{S}_Q\circ\mathcal{I}^{-1}
        \circ\mathcal{I}\circ\mathcal{S}_P\circ\mathcal{I}^{-1}=\\
        &=&\mathcal{S}_{\mathcal{I}(Q)}\circ\mathcal{S}_{\mathcal{I}(P)}=\\
        &=&\mathcal{T}_{2\overrightarrow{\mathcal{I}(P)\mathcal{I}(Q)}},
 \end{eqnarray*}
  kar je bilo treba dokazati. \kdokaz



        \bizrek \label{izoKompTranslRot}
        The product of a translation and a rotation (also a rotation and a translation)
        is a rotation with the same angle.
        \eizrek


\begin{figure}[!htb]
\centering
\input{sl.izo.6.6.5a.pic}
\caption{} \label{sl.izo.6.6.5a.pic}
\end{figure}

 \textbf{\textit{Proof.}} Naj bo $\mathcal{I}=\mathcal{R}_{S,\omega}\circ
 \mathcal{T}_{\overrightarrow{v}}$ kompozitum translacije
 $\mathcal{T}_{\overrightarrow{v}}$ in rotacije
 $\mathcal{R}_{S,\omega}$. Naj bo $b$ premica, ki poteka skozi točko
 $S$ in je pravokotna z vektorjem $\overrightarrow{v}$
 (Figure \ref{sl.izo.6.6.5a.pic}). Po izreku
 \ref{translKom2Zrc} za določeno premico $a$, $a\parallel b$, velja
 $\mathcal{T}_{\overrightarrow{v}}=\mathcal{S}_b\circ
 \mathcal{S}_a$. Naj bo $c$ premica skozi točko $S$, za katero
 je $\measuredangle b,c=\frac{1}{2}\omega$. Po izreku
 \ref{rotacKom2Zrc} je $\mathcal{R}_{S,\omega}=\mathcal{S}_c\circ
 \mathcal{S}_b$. Torej:
  $$\mathcal{I}=\mathcal{R}_{S,\omega}\circ
 \mathcal{T}_{\overrightarrow{v}}=\mathcal{S}_c\circ
 \mathcal{S}_b\circ\mathcal{S}_b\circ
 \mathcal{S}_a=\mathcal{S}_c\circ
 \mathcal{S}_a.$$
 Po Playfairovem aksiomu se premici $a$ in $c$ sekata v neki točki
 $S_1$, po izreku \ref{KotiTransverzala} pa je
 $\measuredangle a,c=\measuredangle b,c=\frac{1}{2}\omega$.
 Če še enkrat uporabimo izrek \ref{rotacKom2Zrc}, dobimo:
$$\mathcal{R}_{S,\omega}\circ
 \mathcal{T}_{\overrightarrow{v}}=\mathcal{S}_c\circ
 \mathcal{S}_a=\mathcal{R}_{S_1,\omega}.$$
 Na podoben način se dokaže, da je tudi kompozitum
  rotacije in
        translacije $\mathcal{T}_{\overrightarrow{v}}
        \circ\mathcal{R}_{S,\omega}$ rotacija za isti kot.
 \kdokaz

V nadaljevanju bomo obravnavali uporabo translacije. Najprej bomo
videli, kako lahko tudi translacijo uporabimo pri načrtovalnih
nalogah.


             \bzgled
            Let $A$ and $B$ be interior points of an angle $pOq$.
            Construct points $C$ and $D$ on the sides $p$ and $q$
            such that the quadrilateral $ABCD$ is a parallelogram.
           \ezgled

\begin{figure}[!htb]
\centering
\input{sl.izo.6.6.6.pic}
\caption{} \label{sl.izo.6.6.6.pic}
\end{figure}

 \textbf{\textit{Solution.}}  (Figure \ref{sl.izo.6.6.6.pic})
 Ker je $ABCD$ paralelogram, je $\overrightarrow{AB}= \overrightarrow{DC} = \overrightarrow{v}$.
  Torej je $\mathcal{T}_{\overrightarrow{v}}(D)= C$. Iz danih
pogojev točka $D$ leži na poltraku $p$, zato njena slika $C$ pri
translaciji $\mathcal{T}_{\overrightarrow{v}}$ leži na sliki $p'$
poltraka $p$ pri tej translaciji (sliko $p'$ lahko dobimo, če
narišemo sliko $O'$ vrha $O$, nato pa vzporedni poltrak poltraka $p$
iz točke $O'$). Ker leži točka $C$ tudi na poltraku $q$, jo  lahko
narišemo iz pogoja $C\in p'\cap q$. Število rešitev naloge je
odvisna od tega, ali imata poltraka $p'$ in $q$ kaj skupnih točk. Ker
poltraka $p$ in $q$ nista vzporedna (gre za kraka kota), poltraka
$p'$ in $q$ nimata več kot eno skupno točko. Torej ima naloga
eno ali nobene rešitve.
 \kdokaz



        \bzgled
          Let $AB$ be a chord of a circle $k$, $P$ and $Q$ points of this circle lying on the
        same side of the line $AB$ and $d$ a line segment in the plane.
        Construct a point $L$ on the circle
         $k$ such that $XY\cong d$, where $X$ and $Y$
         are intersections of the lines $LP$ and $LQ$ with the chord $AB$.
        \ezgled


\begin{figure}[!htb]
\centering
\input{sl.izo.6.6.7.pic}
\caption{} \label{sl.izo.6.6.7.pic}
\end{figure}

 \textbf{\textit{Solution.}}  (Figure \ref{sl.izo.6.6.7.pic})
Čeprav točki $X$ in $Y$ nista znani, je znan vektor
$\overrightarrow{v}=\overrightarrow{XY}$, ki ima enako dolžino
kot daljica $d$ ter je vzporeden in enako usmerjen kot vektor
$\overrightarrow{AB}$. Tudi kot $\omega=\angle PLQ$ je znani kot,
ker je obodni kot za tetivo $PQ$ (izrek \ref{ObodKotGMT}). Naj bo
$P'=\mathcal{T}_{\overrightarrow{v}}(P)$. Ker je
$\overrightarrow{PP'}=\overrightarrow{v}=\overrightarrow{XY}$, je
štirikotnik $PP'YX$ paralelogram, zato sta kota $P'YQ$ in $PLQ$
skladna (izrek \ref{KotaVzporKraki}). Točko $Y$ lahko torej
konstruiramo kot presečišče tetive $AB$ z ustreznim lokom za tetivo
$P'Q$ in obodnimm kotom $\omega$. Nato je
$X=\mathcal{T}^{-1}_{\overrightarrow{v}}(Y)$.
 \kdokaz



        \bzgled \footnote{Predlog za MMO 1996. (SL 10.)}
        Let $H$ be the orthocentre of a triangle $ABC$ and $P$ the point lying on
        the circumcircle of this triangle different from its vertices.
        $E$ is the foot of the altitude from the vertex $B$ of the triangle $ABC$.
        Suppose that the quadrilaterals $PAQB$ and $PARC$ are parallelograms. The lines
        $AQ$ and $HR$ intersect at a point $X$. Prove that $EX\parallel AP$.
        \ezgled

\begin{figure}[!htb]
\centering
\input{sl.izo.6.6.8.pic}
\caption{} \label{sl.izo.6.6.8.pic}
\end{figure}

 \textbf{\textit{Solution.}} Označimo z $O$ središče očrtane
 krožnice trikotnika $ABC$ in $\mathcal{T}_{\overrightarrow{PA}}$
 translacijo za vektor $\overrightarrow{PA}$ (Figure \ref{sl.izo.6.6.8.pic}).

 Ker sta $PAQB$ in $PARC$  paralelograma, je $\overrightarrow{BQ}=\overrightarrow{PA}=\overrightarrow{CR}$.
 To pomeni, da translacija $\mathcal{T}_{\overrightarrow{PA}}$ preslika trikotnik $BPC$
 v trikotnik $QAR$, višinsko točko $H_1$ trikotnika $BPC$ pa v višinsko točko $H'_1$
  trikotnika $QAR$. Dokažimo, da je $H'_1=H$ oz. $H=\mathcal{T}_{\overrightarrow{PA}}(H_1)$
   oz. $\overrightarrow{H_1H}=\overrightarrow{PA}$. Če uporabimo Hamiltonov izrek \ref{Hamilton}
    za trikotnika $ABC$ in $PBC$, ki imata skupno središče očrtane krožnice, dobimo
 $\overrightarrow{OA}+\overrightarrow{OB}+\overrightarrow{OC}
 =\overrightarrow{OH}$ in $\overrightarrow{OP}+\overrightarrow{OB}+\overrightarrow{OC}
 =\overrightarrow{OH_1}$. Iz tega sledi najprej $\overrightarrow{OH}-\overrightarrow{OA}=
 \overrightarrow{OH_1}-\overrightarrow{OP}$, nato pa $\overrightarrow{OH}-\overrightarrow{OH_1}=
 \overrightarrow{OA}-\overrightarrow{OP}$ oz. $\overrightarrow{H_1H}=\overrightarrow{PA}$.

 Torej je $H$ višinska točka trikotnika $ARQ$, kar pomeni, da je $RH\perp AQ$ oz. $\angle AXH=90^0$.
 Ker je še $\angle AEH=90^0$, točki $X$ in $E$ ležita na krožnici
 s premerom $AH$ (izrek \ref{TalesovIzrKrozObrat}), zato je $AXEH$ tetivni štirikotnik. Torej:
 \begin{eqnarray*}
 \angle AXE&=&180^0-\angle AHE \hspace*{4mm} \textrm{(izrek \ref{TetivniPogoj})}\\
   &=&180^0-\angle ACB \hspace*{4mm} \textrm{(izrek \ref{KotaPravokKraki})}\\
   &=&180^0-\angle APB \hspace*{4mm} \textrm{(izrek \ref{ObodObodKot})}\\
   &=&\angle QAP \hspace*{4mm} \textrm{(po  izreku \ref{paralelogram}, ker je }PAQB\textrm{ paralelogram)}
 \end{eqnarray*}

    Po izreku \ref{KotiTransverzala} je $EX\parallel AP$.
 \kdokaz

%________________________________________________________________________________
 \poglavje{Composition of Two Rotations} \label{odd6KompRotac}

V prejšnjih razdelkih smo že obravnavali  kompozitume nekaterih preslikav.
Ugotovili smo, da kompozitum dveh osnih zrcaljenj predstavlja direktno
izometrijo, in sicer identiteto ali rotacijo ali translacijo, odvisno od
tega, ali sta premici enaki ali se sekata ali pa sta vzporedni. V prejšnjem
razdelku smo dokazali, da  kompozitum dveh središčnih zrcaljenj in prav
tako kompozitum dveh translacij predstavlja translacijo.

V tem razdelku bomo raziskovali kompozitum dveh rotacij. Začnimo z naslednjim izrekom.


        \bizrek \label{rotacKomp2rotac}
        Let $R_{A,\alpha}$ and $R_{B,\beta}$ be rotations of the plane
         and $\mathcal{I}=R_{B,\beta}\circ R_{A,\alpha}$.
        Then:
        \begin{enumerate}
            \item if $\alpha+\beta\notin\{k\cdot 360^0;k\in \mathbb{Z}\}$ and $A=B$,
          the product $\mathcal{I}$ is a rotation for the angle $\alpha+\beta$ with the centre $A$,
            \item if $\alpha+\beta\notin\{k\cdot 360^0;k\in \mathbb{Z}\}$ and $A\neq B$,
          the product $\mathcal{I}$ is a rotation for the angle $\alpha+\beta$
          with the centre $C$,\\ where $\measuredangle CAB=\frac{1}{2}\alpha$
          and $\measuredangle ABC=\frac{1}{2}\beta$,
            \item if $\alpha+\beta\in\{k\cdot 360^0;k\in \mathbb{Z}\}$ and $A=B$,
          the product $\mathcal{I}$ is the identity map,
            \item if $\alpha+\beta\in\{k\cdot 360^0;k\in \mathbb{Z}\}$ and $A\neq B$,
          the product $\mathcal{I}$ is a translation.
        \end{enumerate}
        \eizrek

\begin{figure}[!htb]
\centering
\input{sl.izo.6.7.1.pic}
\caption{} \label{sl.izo.6.7.1.pic}
\end{figure}

 \textbf{\textit{Proof.}}  (Figure \ref{sl.izo.6.7.1.pic})

Predpostavimo, da sta središči $A$ in $B$ različni (primer
$A=B$ je trivialen).

Naj bosta $p$ in $q$ takšni premici te ravnine, da je
$\measuredangle p,AB=\frac{1}{2}\alpha$ in $\measuredangle
AB,q=\frac{1}{2}\beta$. Če vsako od rotacij razstavimo s pomočjo
osnih zrcaljenj (izrek \ref{rotacKom2Zrc}), dobimo:
$$\mathcal{I}=R_{B,\beta}\circ R_{A,\alpha}=
\mathcal{S}_q\circ\mathcal{S}_{AB}\circ\mathcal{S}_{AB}\circ
\mathcal{S}_p=\mathcal{S}_q\circ
\mathcal{S}_p.$$

Če je $\alpha+\beta\in\{0^0,360^0,-360^0\}$,
je $\frac{1}{2}\left(\alpha+\beta\right)\in\{0^0,180^0,-180^0\}$,
kar pomeni, da sta premici $p$ in $q$ vzporedni (izrek \ref{KotiTransverzala}).
 V tem primeru je kompozitum $\mathcal{I}=\mathcal{S}_q\circ\mathcal{S}_p$
 translacija (izrek \ref{translKom2Zrc}).

Če je $\alpha+\beta\notin\{0^0,360^0,-360^0\}$,
je $\frac{1}{2}\left(\alpha+\beta\right)\notin\{0^0,180^0,-180^0\}$
ter se premici $p$ in $q$ sekata v neki točki $C$ (izrek \ref{KotiTransverzala}).
Kot $\measuredangle pCq$ je zunanji kot trikotnika $ABC$, zato je enak vsoti
kotov $\measuredangle p,AB$ in $\measuredangle AB,q$ (izrek \ref{zunanjiNotrNotr}).
Torej $\measuredangle pCq=\frac{1}{2}\left(\alpha+\beta\right)$.
Kompozitum
$\mathcal{I}=\mathcal{S}_q\circ\mathcal{S}_p$ v tem primeru predstavlja
rotacijo (izrek \ref{rotacKom2Zrc}) $\mathcal{R}_{C,2\measuredangle pCq}=\mathcal{R}_{C,\alpha+\beta}$.
 \kdokaz

 V nadaljevanju bomo videli uporabo prejšnjega izreka v različnih nalogah.


        \bzgled
        Let $P$, $Q$ and $R$ be non-collinear points.
        Construct a triangle $ABC$ such that $ARB$, $BPC$ and $CQA$ are regular triangles
         with the same orientation.
        \ezgled

\begin{figure}[!htb]
\centering
\input{sl.izo.6.7.2.pic}
\caption{} \label{sl.izo.6.7.2.pic}
\end{figure}

 \textbf{\textit{Solution.}}  (Figure \ref{sl.izo.6.7.2.pic})

 Če so $A$, $B$ in $C$ točke, ki izpolnjujejo dane
pogoje, velja:
$$\mathcal{R}_{Q,60^0} \circ \mathcal{R}_{R,60^0} \circ \mathcal{R}_{P,60^0}(C)=C.$$
Toda po prejšnjem izreku \ref{rotacKomp2rotac} je
kompozitum $\mathcal{R}_{Q,60^0} \circ \mathcal{R}_{R,60^0} \circ \mathcal{R}_{P,60^0}$
središčna simetrija (ker je $60^0+60^0+60^0=180^0$) s fiksno
točko $C$, zato je:
$$\mathcal{R}_{Q,60^0} \circ \mathcal{R}_{R,60^0} \circ \mathcal{R}_{P,60^0}=\mathcal{S}_C.$$
Oglišče $C$ lahko torej konstruiramo kot središče daljice $XX'$,
kjer je $X'$ slika poljubne točke $X$ pri kompozitumu
$\mathcal{R}_{Q,60^0} \circ \mathcal{R}_{R,60^0} \circ
\mathcal{R}_{P,60^0}$. Nato sta še $B=\mathcal{R}_{P,60^0}(C)$ in
$A=\mathcal{R}_{R,60^0}(B)$.
 \kdokaz



        \bzgled \label{rotKompZlato}
        Let $BALK$ and $ACPQ$ be squares with the same orientation and
        $Z$ the midpoint of the line segment $PK$.
        Prove that $BZC$ is an isosceles right triangle with the hypotenuse $BC$.
        \ezgled

\begin{figure}[!htb]
\centering
\input{sl.izo.6.7.3.pic}
\caption{} \label{sl.izo.6.7.3.pic}
\end{figure}

 \textbf{\textit{Solution.}}  (Figure \ref{sl.izo.6.7.3.pic})

 Naj bo $\mathcal{I}=\mathcal{R}_{B,90^0}\circ \mathcal{R}_{C,90^0}$. Ker je
$90°+90°= 180^0$, po izreku \ref{rotacKomp2rotac}
izometrija $\mathcal{I}$ predstavlja središčno zrcaljenje $\mathcal{S}_Y$,
kjer velja $\measuredangle YCB=\frac{1}{2}\cdot 90^0=45^0$ in $\measuredangle CBY=\frac{1}{2}\cdot 90^0=45^0$.
Torej je točka $Y$ oglišče enakokrakega pravokotnega
trikotnika $BYC$ s hipotenuzo $BC$. Ker pa velja še
$\mathcal{S}_Y(P)=\mathcal{I}(P)=K$, je $Y=Z$.
 \kdokaz




        \bzgled \label{rotZgl1}
        On each side of a quadrilateral $ABCD$ squares
         $BALK$, $CBMN$, $DCSR$ and $ADPQ$
        are externally erected.
        Let $E$, $F$, $G$ and $H$ be the midpoints of the line segments
        $PK$, $MR$, $LN$ and $SQ$.
          Prove that the quadrilaterals $BFDE$ and $AGCH$ are also squares.
        \ezgled

\begin{figure}[!htb]
\centering
\input{sl.izo.6.7.4.pic}
\caption{} \label{sl.izo.6.7.4.pic}
\end{figure}

 \textbf{\textit{Solution.}} Po prejšnji trditvi \ref{rotKompZlato}
 sta trikotnika $DEB$ in $BFD$ enakokraka in pravokotna s skupno
 hipotenuzo $BD$ (Figure \ref{sl.izo.6.7.4.pic}). Iz tega sledi, da je štirikotnik
        $BFDE$  kvadrat. Na enak način dobimo, da je tudi štirikotnik $AGCH$ kvadrat.
 \kdokaz



        \bzgled \label{rotZgl2}
        Let $A_1$, $B_1$ and $C_1$ be the centres of squares
        externally erected on the sides $BC$, $AC$ and $AB$ of an arbitrary
         triangle $ABC$
         and $P$ the midpoint of the side $AC$. Prove that:
        \begin{enumerate}
          \item  $C_1PA_1$ is an isosceles right triangle,
          \item $C_1B_1$ and $A_1A$ are perpendicular and congruent line segments,
          \item the line segments $AA_1$, $BB_1$ and $CC_1$  intersect at
          a single point\footnote{To je drugi od treh planimetričnih
          izrekov, ki jih je objavil italijanski matematik
          \index{Bellavitis, G.} \textit{G.
            Bellavitis} (1803--1880) na kongresu v Milanu 1844.}.
        \end{enumerate}
        \ezgled


\textbf{\textit{Solution.}}

\textit{1)} Naj bo $\mathcal{I}= \mathcal{R}_{C_1,90^0}
\mathcal{R}_{A_1,90^0}$ (Figure \ref{sl.izo.6.7.5.pic}). Ker je
$90^0+90^0= 180^0$, izometrija $\mathcal{I}$ predstavlja središčno
zrcaljenje $\mathcal{S}_Y$, kjer je točka $Y$  oglišče enakokrakega
pravokotnega trikotnika $C_1YA_1$ s hipotenuzo $C_1A_1$ (izrek
\ref{rotacKomp2rotac}). Ker je še
$\mathcal{S}_Y(C)=\mathcal{I}(C)=A$, sledi $Y=P$.


\begin{figure}[!htb]
\centering
\input{sl.izo.6.7.5.pic}
\caption{} \label{sl.izo.6.7.5.pic}
\end{figure}



\textit{2)} Z rotacijo $\mathcal{R}_{P,90^0}$ se daljica $B_1C_1$
preslika v daljico $AA_1$ (Figure \ref{sl.izo.6.7.5a.pic}). Zato  sta
daljici skladni in pravokotni (izrek \ref{rotacPremPremKot}).

\begin{figure}[!htb]
\centering
\input{sl.izo.6.7.5a.pic}
\caption{} \label{sl.izo.6.7.5a.pic}
\end{figure}

\textit{3)} Iz dokazanega dela \textit{(2)} sledi, da so premice $AA_1$, $BB_1$, $CC_1$
nosilke višin trikotnika $A_1B_1C_1$ (Figure \ref{sl.izo.6.7.5a.pic}), zato se
sekajo v njegovi višinski točki (izrek \ref{VisinskaTocka}).
 \kdokaz



        \bzgled \label{rotZgl3}
        Let $ABCD$ and $AB_1C_1D_1$ be squares with the same orientation,
         $P$ and $Q$  the midpoints of the line segments $BD_1$ and $B_1D$.
        Suppose that $O$ and $S$ are the centres of these squares.
        Prove that the quadrilateral $POQS$ is also a square.
        \ezgled



\begin{figure}[!htb]
\centering
\input{sl.izo.6.7.6.pic}
\caption{} \label{sl.izo.6.7.6.pic}
\end{figure}

\textbf{\textit{Solution.}}
 Trditev je očitna, če dvakrat uporabimo del \textit{(1)}
 iz prejšnjega zgleda \ref{rotZgl2} (Figure \ref{sl.izo.6.7.6.pic}).
 \kdokaz



 Če povežemo dejstva iz zgledov \ref{rotZgl1} in \ref{rotZgl3},
 dobimo naslednjo trditev (Figure \ref{sl.izo.6.7.4a.pic}).

\begin{figure}[!htb]
\centering
\input{sl.izo.6.7.4a.pic}
\caption{} \label{sl.izo.6.7.4a.pic}
\end{figure}



        \bzgled \label{rotZgl4}
        Let $BALK$, $CBMN$, $DCSR$ and $ADPQ$ be the
        squares externally erected on the four sides of an arbitrary quadrilateral $ABCD$.
         Then  all quadrilaterals defined by the following vertices are also squares:
        \begin{enumerate}
          \item the point $B$, the midpoint of the line segment $MR$, the point $D$ and the midpoint of the line segment $PK$,
          \item the point $A$, the midpoint of the line segment $LN$, the point $C$ and the midpoint of the line segment $SQ$,
          \item the midpoints of the line segments $QL$, $LB$, $BD$ and $DQ$,
          \item the midpoints of the line segments $KM$, $MC$, $CA$ and $AK$,
          \item the midpoints of the line segments $NS$, $SD$, $DB$ and $BN$,
          \item the midpoints of the line segments $RP$, $PA$, $AC$ and $CR$.
        \end{enumerate}

        \ezgled

      V naslednjem primeru bomo uporabljali kompozitum rotacij v situaciji, ko je vsota kotov teh rotacij enaka $360^0$.


        \bzgled \label{rotZgl5}
        Let $ABC$ and $A'B'C'$ be isosceles triangles of the same orientation with the bases $BC$ and
        $B'C'$ and $\angle BAC\cong\angle B'A'C'=\alpha$.
        Suppose that $A_0$, $B_0$ and $C_0$ are the midpoints of the line segments $AA'$, $BB'$ and $CC'$.
        Prove that $A_0B_0C_0$ is also an isosceles triangle
        with the base $B_0C_0$ and $\angle B_0A_0C_0\cong\alpha$.
        \ezgled

\begin{figure}[!htb]
\centering
\input{sl.izo.6.7.7.pic}
\caption{} \label{sl.izo.6.7.7.pic}
\end{figure}

 \textbf{\textit{Solution.}}  (Figure \ref{sl.izo.6.7.7.pic}).

Naj bo    $\mathcal{I}= \mathcal{S}_{C_0}
\circ \mathcal{R}_{A',\alpha} \circ \mathcal{S}_{B_0} \circ
\mathcal{R}_{A,-\alpha}$. Kompozitum $\mathcal{I}$ je direktna
izometrija. Ker je vsota ustreznih
kotov rotacij enak $360^0$ in $\mathcal{I}(C)=C$, mora biti
$\mathcal{I}=\mathcal{E}$ (izrek \ref{rotacKomp2rotac}). Če uporabimo  izrek o
transmutaciji rotacije \ref{izoTransmRotac}, dobimo
 $\mathcal{S}_{A_0} \circ\mathcal{R}_{A,\alpha} \circ \mathcal{S}_{A_0} =\mathcal{R}_{A',\alpha}$.
 Iz dokazanega in iz izreka \ref{transl2sred} sledi:
 \begin{eqnarray*}
 \mathcal{E}&=&\mathcal{I}= \mathcal{S}_{C_0}
 \circ \mathcal{R}_{A',\alpha} \circ \mathcal{S}_{B_0}
 \circ \mathcal{R}_{A,-\alpha}=\\
 &=& \mathcal{S}_{C_0} \circ (\mathcal{S}_{A_0} \circ\mathcal{R}_{A,\alpha}
 \circ \mathcal{S}_{A_0}) \circ \mathcal{S}_{B_0} \circ \mathcal{R}_{A,-\alpha}=\\
 &=& (\mathcal{S}_{C_0} \circ \mathcal{S}_{A_0}) \circ\mathcal{R}_{A,\alpha}
 \circ (\mathcal{S}_{A_0} \circ \mathcal{S}_{B_0}) \circ \mathcal{R}_{A,-\alpha}=\\
 &=& \mathcal{T}_{2\overrightarrow{A_0C_0}} \circ\mathcal{R}_{A,\alpha}
 \circ \mathcal{T}_{2\overrightarrow{B_0A_0}} \circ \mathcal{R}_{A,-\alpha}
 \end{eqnarray*}
 Torej velja $\mathcal{E}=\mathcal{T}_{2\overrightarrow{A_0C_0}}
 \circ\mathcal{R}_{A,\alpha} \circ \mathcal{T}_{2\overrightarrow{B_0A_0}}
 \circ \mathcal{R}_{A,-\alpha}$, oz. $\mathcal{T}_{2\overrightarrow{C_0A_0}}=
 \mathcal{R}_{A,\alpha} \circ \mathcal{T}_{2\overrightarrow{B_0A_0}} \circ
 \mathcal{R}_{A,-\alpha}$. Če uporabimo izrek o transmutaciji translacije
 \ref{izoTransmTrans}, dobimo:
 $$\mathcal{T}_{2\overrightarrow{C_0A_0}}= \mathcal{R}_{A,\alpha}
 \circ \mathcal{T}_{2\overrightarrow{B_0A_0}} \circ
 \mathcal{R}_{A,-\alpha}=\mathcal{T}_{2\overrightarrow{B'_0A'_0}},$$
  kjer je $\mathcal{R}_{A,\alpha}:A_0, B_0\mapsto A'_0, B'_0$. Torej
  $\overrightarrow{C_0A_0}=\overrightarrow{B'_0A_0'}$ oz.
  $\overrightarrow{A_0C_0}=\overrightarrow{A'_0B'_0}$.
  Vektor $\overrightarrow{A_0B_0}$ se z rotacijo $\mathcal{R}_{A,\alpha}$
  preslika v vektor $\overrightarrow{A'_0B'_0}=\overrightarrow{A_0C_0}$,
  zato je $|A_0B_0|=|A_0C_0|$ in $\measuredangle B_0A_0C_0=\measuredangle\overrightarrow{A_0B_0},
  \overrightarrow{A_0C_0}=\alpha$.
  \kdokaz


        \bzgled \label{RotacZglVeck}
        Let $A_1A_2...A_n$ and $B_1B_2...B_n$ be regular
         $n$-gons with the same orientation. Suppose that
         $S_1$, $S_2$, ..., $S_n$ are the midpoints of the line segments
        $A_1B_1$, $A_2B_2$, ..., $A_nB_n$. Prove that $S_1S_2...S_n$ is also a regular $n$-gon.
        \ezgled

\begin{figure}[!htb]
\centering
\input{sl.izo.6.7.8.pic}
\caption{} \label{sl.izo.6.7.8.pic}
\end{figure}

 \textbf{\textit{Solution.}}
Ker sta $n$-kotnika $A_1A_2...A_n$ in $B_1B_2...B_n$
pravilna in enako orientirana, sta $A_1A_2A_3$ in $B_1B_2B_3$
enakokraka in enako orientirana
trikotnika z osnovnicama $A_1A_3$ in $B_1B_3$ (Figure \ref{sl.izo.6.7.8.pic}).
Pri tem je še $\angle A_1A_2A_3\cong \angle B_1B_2B_3=\frac{(n - 2)\cdot 180^0}{n}$ (izrek \ref{pravVeckNotrKot}).

Iz prejšnjega zgleda \ref{rotZgl5} je
tudi  $S_1S_2S_3$ enakokraki trikotnik in velja
$\angle S_1S_2S_3=\frac{(n - 2)\cdot 180^0}{n}$.
Torej sta stranici $S_1S_2$ in $S_2S_3$ skladni, notranji kot pri
oglišču $S_2$ pa skladen s kotom  pravilnega $n$-kotnika.
Analogno dokažemo, da so vse
stranice večkotnika $S_1S_2...S_n$ skladne in vsi njegovi
notranji koti skladni, kar pomeni, da je ta večkotnik pravilen.
 \kdokaz

Eno zanimivo posledico izreka o kompozitumu rotacij pa bomo
obravnavali še v razdelku \ref{odd7Napoleon}.


%________________________________________________________________________________
 \poglavje{Glide Reflections} \label{odd6ZrcDrs}

Do sedaj smo spoznali nekaj vrst izometrij - tri direktne izometrije
(identiteta, rotacija in translacija) ter eno indirektno izometrijo (osno zrcaljenje).
Sedaj bomo definirali še eno indirektno izometrijo, ki nima fiksnih točk.

\begin{figure}[!htb]
\centering
\input{sl.izo.6.8.1.pic}
\caption{} \label{sl.izo.6.8.1.pic}
\end{figure}

Naj bo $\mathcal{T}_{\overrightarrow{v}}$ translacija
 za vektor $\overrightarrow{v} = 2\overrightarrow{PQ}$
in $\mathcal{S}_{PQ}$ zrcaljenje čez premico $PQ$. Kompozitum
$\mathcal{S}_{PQ}\circ \mathcal{T}_{\overrightarrow{v}}$ se imenuje
\index{zrcalni zdrs}\pojem{zrcalni zdrs} z osjo $PQ$ za vektor $\overrightarrow{v} = 2\overrightarrow{PQ}$
 (Figure \ref{sl.izo.6.8.1.pic}). Označimo ga z
$\mathcal{G}_{2\overrightarrow{PQ}}$.


 Že iz definicije je jasno, da je zrcalni zdrs določen s svojo osjo in vektorjem.

 Dokažimo osnovne lastnosti zrcalnega zdrsa.



        \bizrek \label{IzoZrcDrs1}
        Any glide reflection can be expressed as the product of three reflections.\\
        The reflection and translation in the product, as the presentation of glide reflection, commute, i.e.
         $$\mathcal{G}_{2\overrightarrow{PQ}}=
         \mathcal{S}_{PQ}\circ \mathcal{T}_{2\overrightarrow{PQ}}=
         \mathcal{T}_{2\overrightarrow{PQ}}\circ \mathcal{S}_{PQ}.$$
        \eizrek

\begin{figure}[!htb]
\centering
\input{sl.izo.6.8.2.pic}
\caption{} \label{sl.izo.6.8.2.pic}
\end{figure}

 \textbf{\textit{Proof.}} Naj bo $\mathcal{G}_{2\overrightarrow{PQ}}=
 \mathcal{S}_{PQ}\circ \mathcal{T}_{2\overrightarrow{PQ}}$ z osjo $PQ$
 in vektorjem $2\overrightarrow{PQ}$. Naj bosta $p$ in $q$ pravokotnici
  premice $PQ$ v točkah $P$ in $Q$ (Figure \ref{sl.izo.6.8.2.pic}).
  Ker je $p,q\perp PQ$ in $p\parallel q$, lahko translacijo
  $\mathcal{T}_{2\overrightarrow{PQ}}$  predstavimo kot
  kompozitum $\mathcal{S}_q\circ\mathcal{S}_p$. Pri tem zrcaljenje $\mathcal{S}_{PQ}$
  komutira z zrcaljenji $\mathcal{S}_p$ in $\mathcal{S}_q$ (izrek \ref{izoZrcKomut}). Torej:
 $$\mathcal{G}_{2\overrightarrow{PQ}}=
 \mathcal{S}_{PQ}\circ \mathcal{T}_{2\overrightarrow{PQ}}=
 \mathcal{S}_{PQ}\circ\mathcal{S}_q\circ\mathcal{S}_p=
 \mathcal{S}_q\circ\mathcal{S}_p\circ\mathcal{S}_{PQ}=
 \mathcal{T}_{2\overrightarrow{PQ}}\circ\mathcal{S}_{PQ},$$ kar je bilo treba dokazati. \kdokaz


        \bizrek
        \label{izoZrcdrsZrcdrs}
        Product of a glide reflection with itself is a translation (Figure \ref{sl.izo.6.8.3.pic}).
        \eizrek

\begin{figure}[!htb]
\centering
\input{sl.izo.6.8.3.pic}
\caption{} \label{sl.izo.6.8.3.pic}
\end{figure}

 \textbf{\textit{Proof.}} Uporabimo prejšnji izrek \ref{IzoZrcDrs1}
 in izrek \ref{translKomp}:
 $$\mathcal{G}^2_{2\overrightarrow{PQ}}=
 \mathcal{T}_{2\overrightarrow{PQ}}\circ \mathcal{S}_{PQ}\circ
 \mathcal{S}_{PQ}\circ \mathcal{T}_{2\overrightarrow{PQ}}=
 \mathcal{T}^2_{2\overrightarrow{PQ}}=
 \mathcal{T}_{4\overrightarrow{PQ}},$$ kar je bilo treba dokazati. \kdokaz

  Iz dokaza izreka \ref{IzoZrcDrs1} sledi, da lahko vsak zrcalni zdrs predstavimo
   kot kompozitum treh osnih
zrcaljenj, kjer je os enega zrcaljenja pravokotna na osi drugih dveh.
Zrcalni zdrs je torej indirektna
izometrija kot kompozitum  treh
osnih zrcaljenj - indirektnih izometrij. Isto dejstvo sledi tudi iz
same definicije zdrsa, ker je ta kompozitum ene direktne in ene indirektne izometrije.

        \bizrek
        A glide reflection has no fixed points.
        \eizrek

\textbf{\textit{Proof.}} Po izreku \ref{IzoZrcDrs1} lahko zrcalni zdrs
 predstavimo kot kompozitum treh osnih zrcaljenj:
$$\mathcal{G}_{2\overrightarrow{PQ}}=
 \mathcal{S}_{PQ}\circ\mathcal{S}_q\circ\mathcal{S}_p,$$
 kjer sta $p$ in $q$ pravokotnici premice $PQ$ v točkah $P$ in $Q$.
 Če bi zrcalni zdrs $\mathcal{G}_{2\overrightarrow{PQ}}$ oz.
 kompozitum $\mathcal{S}_{PQ}\circ\mathcal{S}_q\circ\mathcal{S}_p$
 imel fiksno točko, bi $\mathcal{S}_{PQ}\circ\mathcal{S}_q\circ\mathcal{S}_p$
 kot indirektna izometrija predstavljal osno zrcaljenje (izrek \ref{izo1ftIndZrc}).
 Po izreku \ref{izoSop} bi v tem primeru premice $p$, $q$ in $PQ$ pripadale istemu
 šopu, kar pa ni mogoče. Zrcalni zdrs $\mathcal{G}_{2\overrightarrow{PQ}}$ torej nima fiksnih točk.
 \kdokaz

Uporabili smo že dejstvo iz izreka \ref{IzoZrcDrs1},
da lahko zrcalni zdrs vedno predstavimo kot kompozitum treh osnih
zrcaljenj, kjer osi teh zrcaljenj niso v istem šopu. Ali  velja obratno
- da je kompozitum treh osnih zrcaljenj, katerih osi niso iz istega šopa, vedno zrcalni zdrs?
S tem v zvezi bo  naslednji izrek, ki je zelo pomemben tudi za klasifikacijo
 izometrij, ki jo bomo izpeljali v naslednjem razdelku.


        \bizrek \label{izoZrcdrsprq}
        If lines $p$, $q$ and $r$ in the plane
        are not from the same family of lines, then product
        $\mathcal{S}_p \circ
        \mathcal{S}_q\circ \mathcal{S}_r$ is a glide reflection.
        \eizrek

\begin{figure}[!htb]
\centering
\input{sl.izo.6.8.4.pic}
\caption{} \label{sl.izo.6.8.4.pic}
\end{figure}

 \textbf{\textit{Proof.}}  (Figure \ref{sl.izo.6.8.4.pic})
 Premica $q$ seka bodisi premico $p$ bodisi premico
$r$, ker bi sicer vse tri premice pripadale enemu šopu vzporednih
premic. Brez škode za splošnost naj bo $q\cap r=\{A\}$. Premice $p$,
$q$ in $r$ niso iz istega šopa, zato $A\notin p$. Naj bo $s$
premica, ki poteka skozi točko $A$ in je pravokotna na premico $p$ v
točki $B$. Premice $s$, $q$ in $r$ pripadajo istemu šopu
$\mathcal{X}_A$, zato je po izreku \ref{izoSop} $\mathcal{S}_r \circ
\mathcal{S}_q \circ \mathcal{S}_ s = \mathcal{S}_t$, kjer je tudi
$t\in \mathcal{X}_A$. Iz tega sledi $\mathcal{S}_r \circ
\mathcal{S}_q = \mathcal{S}_t \circ \mathcal{S}_s$ oziroma (če
uporabimo še izrek \ref{izoSrZrcKom2Zrc}):
$$\mathcal{S}_r \circ \mathcal{S}_q \circ \mathcal{S}_p = \mathcal{S}_t
\circ \mathcal{S}_s \circ \mathcal{S}_p = \mathcal{S}_t \circ \mathcal{S}_B.$$
Označimo s $C$ pravokotno projekcijo točke $B$ na premici $t$
(pri tem je $B\neq C$, ker bi se v nasprotnem primeru premici $s$
in $t$, posledično pa tudi $r$ in $q$ prekrivali) in z $b$ premico, ki je v točki $B$ pravokotna na premico $BC$.
Potem je po izrekih \ref{izoSrZrcKom2Zrc} in \ref{translKom2Zrc}:
$$\mathcal{S}_r \circ \mathcal{S}_q \circ \mathcal{S}_p = \mathcal{S}_t \circ \mathcal{S}_B=
\mathcal{S}_t \circ \mathcal{S}_b \circ \mathcal{S}_{BC}=
\mathcal{T}_{2\overrightarrow{BC}} \circ \mathcal{S}_{BC}=
\mathcal{G}_{2\overrightarrow{BC}},$$ kar je bilo treba dokazati. \kdokaz

        \bzgled \label{izoZrcDrsKompSrOsn}
        Any glide reflection can be expressed as the product of a reflection and a half-turn, specifically
        $$\mathcal{G}_{2\overrightarrow{PQ}}=\mathcal{S}_q\circ \mathcal{S}_P,$$
        where $q$ perpendicular to the line  $PQ$ in the point $Q$.
        \ezgled

\begin{figure}[!htb]
\centering
\input{sl.izo.6.8.5.pic}
\caption{} \label{sl.izo.6.8.5.pic}
\end{figure}

 \textbf{\textit{Proof.}}  (Figure \ref{sl.izo.6.8.5.pic})

    Po definiciji je
    $\mathcal{G}_{2\overrightarrow{PQ}}=\mathcal{S}_{PQ}
    \circ \mathcal{T}_{2\overrightarrow{PQ}}$. Če
    uporabimo izreka \ref{transl2sred} in \ref{izoSrZrcKom2Zrc}, je:
 $$\mathcal{G}_{2\overrightarrow{PQ}}=\mathcal{S}_{PQ}\circ \mathcal{T}_{2\overrightarrow{PQ}}=
 \mathcal{S}_{PQ}\circ\mathcal{S}_Q\circ\mathcal{S}_P=
 \mathcal{S}_{PQ}\circ\mathcal{S}_{PQ}\circ
 \mathcal{S}_q\circ\mathcal{S}_P=
 \mathcal{S}_q\circ\mathcal{S}_P,$$ kar je bilo treba dokazati. \kdokaz


%________________________________________________________________________________
 \poglavje{Classification of Plane Isometries. Chasles' Theorem} \label{odd6KlasifIzo}

V prejšnjih razdelkih smo raziskovali določene vrste izometrij.
Postavi se vprašanje, ali so to edine izometrije ali
pa obstaja še kakšna druga vrsta izometrij, ki jo še nismo
obravnavali. V tem razdelku bomo dokazali, da je odgovor na to
vprašanje negativen in naredili končno klasifikacijo vseh
izometrij ravnine.

Najprej bomo obravnavali direktne izometrije. Doslej smo omenili
identiteto, rotacijo in translacijo. Dokazali bomo, da so to edine
vrste direktnih izometrij. Spomnimo se najprej, da lahko vsako od
teh izometrij predstavimo kot kompozitum dveh zrcaljenj čez
premico. Odvisno od tega, ali se osi prekrivata, sekata ali sta
vzporedni, smo dobili identiteto, rotacijo in translacijo. To
so hkrati vse možnosti medsebojne lege dveh premic v
ravnini. Ali se lahko morda vse direktne izometrije predstavijo
kot kompozitum dveh zrcaljenj čez premico? Potemtakem bi bile tri
omenjene direktne izometrije res edine. To idejo bomo
uporabili v naslednjem izreku.


       \bizrek \label{Chaslesov+}
      Any direct isometry can be expressed as the product of two reflections.
        The only direct isometries are
      identity map, rotations and
       translations.
      \eizrek


\textbf{\textit{Proof.}} (Figure \ref{sl.izo.6.10.1.pic})

 Naj bo $\mathcal{I} : \mathbb{E}^2\rightarrow \mathbb{E}^2$
 direktna izometrija ravnine. Dokaz bomo izpeljali po številu
fiksnih točk.

\textit{1)} Če ima direktna izometrija $\mathcal{I}$ vsaj dve fiksni točki,
je po izreku \ref{izo2ftIdent} identiteta. Lahko jo
predstavimo kot kompozitum $\mathcal{I} = \mathcal{S}_p \circ
\mathcal{S}_p$ (izrek \ref{izoZrcPrInvol}) za poljubno premico
$p$.

\begin{figure}[!htb]
\centering
\input{sl.izo.6.10.1.pic}
\caption{} \label{sl.izo.6.10.1.pic}
\end{figure}

 \textit{2)} Predpostavimo, da ima izometrija
$\mathcal{I}$ natanko eno fiksno točko $S$. Naj bo $p$
poljubna premica, ki poteka skozi
 točko $S$. Kompozitum $\mathcal{S}_p\circ \mathcal{I}$
 je indirektna izometrija s fiksno
točko $S$, zato po izreku \ref{izo1ftIndZrc} predstavlja osno
zrcaljenje - denimo $\mathcal{S}_q$, kjer os $q$ poteka skozi točko
$S$. Torej velja $\mathcal{S}_p\circ \mathcal{I} = \mathcal{S}_q$
oz. $\mathcal{I}
=\mathcal{S}_p^{-1}\circ\mathcal{S}_q=\mathcal{S}_p\circ\mathcal{S}_q$
(izrek \ref{izoZrcPrInvol}). Premici $p$ in $q$ se sekata v točki
$S$ (iz $p=q$ sledi $\mathcal{I}=\mathcal{S}_p \circ
\mathcal{S}_p=\mathcal{E}$), zato po izreku \ref{rotacKom2Zrc}
$\mathcal{I}$ predstavlja  rotacijo s središčem v točki $S$.

\textit{3)} Naj bo $\mathcal{I}$ izometrija brez fiksnih točk.
Potem je za poljubno točko $A$ te ravnine $\mathcal{I}(A) = A'\neq
A$. Naj bo $p$ simetrala daljice $AA'$ in $S_p$ zrcaljenje čez
premico $p$. V tem primeru je $\mathcal{S}_p\circ \mathcal{I}(A)=
\mathcal{S}_p(A')=A$. Kompozitum $\mathcal{S} \circ
\mathcal{I}_p$ je torej indirektna izometrija s fiksno točko $A$, zato
po izreku \ref{izo1ftIndZrc} predstavlja zrcaljenje čez neko
premico $q$ - $\mathcal{S}_q$, kjer os $q$ poteka skozi točko $A$.
Enako kot v prejšnjem primeru je
$\mathcal{I}=\mathcal{S}_p\circ\mathcal{S}_q$. Premici $p$ in $q$
se niti ne sekata niti nista enaki, ker bi  sicer iz že dokazanega
$\mathcal{I}$ predstavljala identiteto ali rotacijo in bi imela
vsaj eno fiksno točko. Torej ostane le še možnost, da sta premici
$p$ in $q$ vzporedni. V tem primeru je $\mathcal{I}$ translacija
(izrek \ref{translKom2Zrc}).
 \kdokaz

Ostanejo nam še indirektne izometrije. Edini doslej omenjeni vrsti
sta osno zrcaljenje  in zrcalni zdrs.
Ali sta tudi sicer edini? Odgovor bomo podali v naslednjem
    izreku.



             \bizrek \label{Chaslesov-}
             Any opposite isometry is either a reflection
             either it can be represented as the product of three reflections.
               The only opposite isometries are
            reflections and
            glide reflections.
             \eizrek

\textbf{\textit{Proof.}} (Figure \ref{sl.izo.6.10.2.pic})

Naj bo $\mathcal{I} : \mathbb{E}^2\rightarrow \mathbb{E}^2$
indirektna izometrija ravnine. Dokaz bomo spet izpeljali glede
na število fiksnih točk.

\textit{1)} Če ima  izometrija $\mathcal{I}$ vsaj eno fiksno
točko, je $\mathcal{I}$ po izreku \ref{izo1ftIndZrc} zrcaljenje čez
premico.

\begin{figure}[!htb]
\centering
\input{sl.izo.6.10.2.pic}
\caption{} \label{sl.izo.6.10.2.pic}
\end{figure}

     \textit{2)} Predpostavimo, da je $\mathcal{I}$ izometrija brez fiksnih točk.
     V tem primeru  za poljubno točko $A$ te ravnine velja
$\mathcal{I}(A) = A'\neq A$. Naj bo $p$ simetrala daljice $AA'$ in
$\mathcal{S}_p$ zrcaljenje čez premico $p$. Potem je
$\mathcal{S}_p\circ \mathcal{I}(A)=  \mathcal{S}_p(A')=A$. Zato je
kompozitum $\mathcal{S}_p \circ \mathcal{I}$ direktna izometrija s
fiksno točko $A$ pa po prejšnjem  izreku predstavlja  rotacijo ali
identiteto (translacija nima fiksnih točk). Drugi primer odpade,
saj bi sicer bilo $\mathcal{I} = \mathcal{S}_p$ in bi
izometrija $\mathcal{I}$ imela fiksne točke, kar pa nasprotuje
začetni predpostavki. Torej je kompozitum $\mathcal{S} \circ
\mathcal{I}_p$ rotacija s središčem $A$, ki jo lahko predstavimo
kot kompozitum dveh zrcaljenj čez premici $q$ in $r$, ki se sekata
v točki $A$. Zato je $\mathcal{S}_p\circ \mathcal{I}=\mathcal{S}_q\circ
\mathcal{S}_r$ oz. $\mathcal{I}=\mathcal{S}_p\circ \mathcal{S}_q\circ
\mathcal{S}_r$. Ker je  $A'\neq A$ in $p$ simetrala daljice $AA'$,
točka $A$ ne leži na premici $p$. To pomeni, da premice $p$, $q$
in $r$ niso iz istega šopa. Po izreku \ref{izoZrcdrsprq} je zato
$\mathcal{I}=\mathcal{S}_p\circ \mathcal{S}_q\circ \mathcal{S}_r$
zrcalni zdrs.
 \kdokaz

     Zaradi prejšnjih dveh izrekov lahko rečemo,
      da so omenjene izometrije edine vrste izometrij.
Lahko rečemo tudi, da lahko vsako izometrijo ravnine predstavimo
kot kompozitum osnih simetrij, pri čemer lahko osi izberemo tako, da
 v kompozitumu niso več kot tri. Iz dokaza omenjenih izrekov je
jasno, da lahko vsako izometrijo določimo le na osnovi števila
fiksnih točk in tega, ali je izometrija direktna oz. indirektna.
Ta dejstva bomo formulirali v naslednjih dveh izrekih.



            \bizrek \label{IzoKompZrc}
            Any isometry of the plane can be expressed as the product of
            one, two or three reflections.
            \eizrek



            \bizrek \label{Chaslesov} \index{izrek!Chaslesov}
            (Chasles’\footnote{\index{Chasles, M.}
            \textit{M. Chasles} (1793--1880),
            francoski geometer, ki je to klasifikacijo izpeljal leta 1831.})
            The only isometries of the plane
             $\mathcal{I} : E^2 \rightarrow E^2$ are:
              identity map, reflections, rotations,
            translations and glide reflections. Specifically:\\
                \hspace*{3mm}(i) if $\mathcal{I}$ is
            a direct isometry and has at least two fixed points, then $\mathcal{I}$
            is the identity map,\\
                 \hspace*{3mm}(ii) if $\mathcal{I}$ is a direct isometry and has exactly one fixed point,
             then $\mathcal{I}$ is a rotation (or specially a half-turn),\\
                 \hspace*{3mm}(iii) if $\mathcal{I}$ is
            a direct isometry and has no fixed points,
              then $\mathcal{I}$ is a translation,\\
                 \hspace*{3mm}(iv) if $\mathcal{I}$ is
            an opposite isometry and has at least one fixed point,
             then $\mathcal{I}$ is a reflection,\\
                \hspace*{3mm}(v)
             if $\mathcal{I}$ is
            an opposite isometry and has no fixed points,
              then $\mathcal{I}$ is a glide reflection.
             \eizrek

  Vse, kar smo  v tem razdelku povedali o klasifikaciji izometrij, je
ilustrirano v naslednji tabeli (Figure \ref{IzoKlas.eps}):

\vspace*{-2mm}

%\begin{figure}[!htb]
%\centering
%\input{sl.izo.6.10.3.pic}
%\caption{} \label{sl.izo.6.10.3.pic}
%\end{figure}

\begin{figure}[!htb]
\centering
 \includegraphics[width=0.85\textwidth]{IzoKlas.eps}
\caption{} \label{IzoKlas.eps}
\end{figure}

%________________________________________________________________________________
 \poglavje{Hjelmslev's Theorem} \label{odd6Hjelmslev}

The following theorem, which refers to opposite isometries, is very useful.


            \bizrek \label{Chasles-Hjelmsleva}
            (Hjelmslev's\footnote{
            \index{Hjelmslev, J. T.}
            \textit{J. T. Hjelmslev} (1873--1950), danski
             matematik.}) \index{theorem!Hjelmslev's}
             The midpoints of all line segments defined by corresponding pairs
            of points of an arbitrary indirect isometry lie on the same line.
            \eizrek

\begin{figure}[!htb]
\centering
\input{sl.izo.6.11.1.pic}
\caption{} \label{sl.izo.6.11.1.pic}
\end{figure}

\textbf{\textit{Proof.}} (Figure \ref{sl.izo.6.11.1.pic})

Naj bo $\mathcal{I}$ indirektna izometrija in $X'=\mathcal{I}(X)$ za
poljubno točko $X$ ter $X_s$ središče daljice $XX'$. Iz Chaslesovega
izreka \ref{Chaslesov}sledi, da sta edini vrsti indirektnih izometrij ravnine
osno zrcaljenje in zrcalni zdrs. Dokažimo, da je v obeh primerih
iskana premica ravno os zrcaljenja čez premico oz. zrcalnega zdrsa. V
prvem primeru, kjer je $\mathcal{I}=\mathcal{S}_s$, je trivialno, da je
$X_s\in s$. Naj bo $\mathcal{I}=\mathcal{G}_{2\overrightarrow{PQ}}=
\mathcal{S}_s\circ \mathcal{T}_{2\overrightarrow{PQ}}$, kjer je
$s=PQ$. Če označimo z $X_1=\mathcal{T}_{2\overrightarrow{PQ}}(X)$ in
$X_2$ središče daljice $X_1X'$, je daljica $X_sX_2$ srednjica
trikotnika $XX_1X'$ za osnovnico $XX_1$. Torej je (izrek
\ref{srednjicaTrik}) $X_sX_2\parallel XX_1\parallel s$ oz. $X_s\in
s$ (Playfairjev aksiom \ref{Playfair}).
 \kdokaz

Dokazani izrek lahko uporabimo, če imamo indirektno podobna lika
ali pa če najdemo indirektno izometrijo, ki preslika eno množico
točk v drugo. To bomo ilustrirali z naslednjima primeroma.

         \bzgled
         Let $ABC$ and $A'B'C'$ be congruent triangles with the opposite orientation.
         Prove that the midpoints of the line segments
             $AA$', $BB'$ and $CC'$ lie on the same line.
          \ezgled

\begin{figure}[!htb]
\centering
\input{sl.izo.6.11.2.pic}
\caption{} \label{sl.izo.6.11.2.pic}
\end{figure}

\textbf{\textit{Proof.}} (Figure \ref{sl.izo.6.11.2.pic})
 Ker sta $ABC$ in $A'B'C'$ skladna različno orientirana
    trikotnika, obstaja indirektna izometrija $\mathcal{I}$, ki trikotnik $ABC$
preslika v trikotnik $A'B'C'$. V tem primeru so $A$ in $A'$, $B$ in
$B'$ ter $C$ in $C'$ pari te izometrije, zato po izreku
 \ref{Chasles-Hjelmsleva} središča daljic $AA'$,
$BB'$ in $CC'$ ležijo na eni premici.
 \kdokaz

        \bzgled
          Let $A$ be a point and $p$ a line in the plane. Suppose that points $X_i$
         lie on the line $p$ and $AX_iY_i$ are regular triangles with the same orientation.
          Prove that the midpoints of the line segments
          $X_iY_i$ lie on the same line.
         \ezgled


\begin{figure}[!htb]
\centering
\input{sl.izo.6.11.3.pic}
\caption{} \label{sl.izo.6.11.3.pic}
\end{figure}

\textbf{\textit{Proof.}} (Figure \ref{sl.izo.6.11.3.pic})
Izometrija $\mathcal{R}_{A,60^0}\circ\mathcal{S}_p$ je indirektna izometrija, ki preslika točke $X_i$ v točke $Y_i$,
 zato po izreku  \ref{Chasles-Hjelmsleva} središča
 daljic $X_iY_i$ ležijo
na isti premici.
  \kdokaz


%________________________________________________________________________________
 \poglavje{Isometry Groups. Symmetries of Figures} \label{odd6Grupe}

V  razdelku \ref{odd2AKSSKL} smo ugotovili, da množica
 $\mathfrak{I}$ vseh izometrij neke ravnine skupaj z operacijo kompozituma
preslikav predstavlja t. i.
strukturo \index{grupa}\pojem{grupe}\footnote{Teorijo grup je
odkril genialni mlad francoski matematik \index{Galois, E.}
\textit{E. Galois} (1811--1832).}.
 To pomeni, da so izpolnjene naslednje lastnosti:
\begin{enumerate}
  \item $(\forall f\in \mathfrak{I})(\forall g\in \mathfrak{I})
  \hspace*{1mm}f\circ g\in \mathfrak{I}$,
  \item $(\forall f\in \mathfrak{I})(\forall g\in \mathfrak{I})
  (\forall h\in \mathfrak{I})
  \hspace*{1mm}(f\circ g)\circ h=f\circ (g\circ h)$,
  \item $(\exists e\in \mathfrak{I})(\forall f\in \mathfrak{I})
  \hspace*{1mm}f\circ e=e\circ f=f$,
  \item $(\forall f\in \mathfrak{I})(\exists g\in \mathfrak{I})
  \hspace*{1mm}f\circ g=g\circ f=e$.
\end{enumerate}

  Lastnost (2) velja splošno za
kompozitum preslikav. Lastnosti (1), (3) in (4) smo vpeljali z
aksiomom \ref{aksIII4}. Lastnost (1) pomeni, da je kompozitum dveh
izometrij spet izometrija, (3) in (4) pa se nanašata na identiteto
in inverzno izometrijo. Omenjeno grupo, ki jo določa množica
$\mathfrak{I}$ vseh izometrij neke ravnine glede na operacijo
kompozituma preslikav, imenujemo \index{grupa!izometrij}
\pojem{grupa vseh izometrij ravnine}. Tudi njo bomo označili z  $\mathfrak{I}$.

Obstajajo tudi druge grupe izometrij, ki jih dobimo, če
vzamemo ustrezno podmnožico vseh izometrij ravnine. Lastnost (1)
nam pove, da ta množica ne more biti poljubna. Npr. množica vseh
rotacij ravnine ni grupa, saj kompozitum dveh rotacij ni vedno
rotacija (lahko je tudi translacija).

Iz lastnosti translacij sledi, da  množica vseh translacij
skupaj z identiteto $\mathcal{E}$ predstavlja grupo - t. i.
\index{grupa!translacij}\pojem{grupo translacij} z oznako
$\mathfrak{T}$. Za njo pravimo, da je \index{podgrupa}
\pojem{podgrupa} grupe $\mathfrak{I}$ vseh izometrij ravnine (to
dejstvo označimo s $\mathfrak{T}<\mathfrak{I}$). Pravzaprav je vsaka
grupa izometrij ravnine podgrupa grupe $\mathfrak{I}$ vseh izometrij te ravnine.
Vendar grupa translacij ni edina podgrupa grupe
$\mathfrak{I}$. Vse direktne izometrije ravnine namreč tvorijo eno
takšno podgrupo; označimo jo z $\mathfrak{I}^+$. To pomeni, da je
$\mathfrak{I^+}<\mathfrak{I}$. Ker so translacije direktne
izometrije, je tudi $\mathfrak{T}<\mathfrak{I^+}$. Jasno je, da
množica vseh indirektnih izometrij ne določa grupe, saj je kompozitum
dveh indirektnih izometriji direktna izometrija (tudi identiteta
ni indirektna izometrija).

Obstajajo tudi končne podgrupe grupe $\mathfrak{I}$ (ki imajo
končno število izometrij). Primer takšne podgrupe je t. i.
Kleinova\footnote{\index{Klein, F. C.} \textit{F. C. Klein}
(1849--1925), nemški matematik, ki je na svojem predavanju leta 1872 na
univerzi v Erlangenu predstavil idejo, po kateri je neka
geometrija določena z grupo ustreznih transformacij na množici
(točk) in z invariantami te grupe na tej množici.} grupa
$\mathfrak{K}$ ($\mathfrak{K}<\mathfrak{I}$) (ali \textit{Kleinov četverec}), ki je določena z
množico izometrij $\{\mathcal{E}, \mathcal{S}_p, \mathcal{S}_q,
\mathcal{S}_O\}$, kjer sta $p$ in $q$ pravokotnici, ki se sekata v
točki $O$. To grupo lahko predstavimo tudi s tabelo:

\vspace*{5mm}

\hspace*{22mm}\begin{tabular}{|c||c|c|c|c|} \hline
  % after \\ : \hline or \cline{col1-col2} \cline{col3-col4} ...
  $\circ$ & $\mathcal{E}$ & $\mathcal{S}_p$ & $\mathcal{S}_q$& $\mathcal{S}_O$
   \\ \hline \hline
  $\mathcal{E}$ & $\mathcal{E}$& $\mathcal{S}_p$& $\mathcal{S}_q$& $\mathcal{S}_O$
  \\ \hline
  $\mathcal{S}_p$ & $\mathcal{S}_p$ & $\mathcal{E}$ & $\mathcal{S}_O$& $\mathcal{S}_q$
  \\ \hline
  $\mathcal{S}_q$ & $\mathcal{S}_q$ & $\mathcal{S}_O$ & $\mathcal{E}$& $\mathcal{S}_p$
  \\ \hline
  $\mathcal{S}_O$ & $\mathcal{S}_O$ & $\mathcal{S}_q$ &$\mathcal{S}_p$ & $\mathcal{E}$
  \\ \hline
\end{tabular}


\vspace*{5mm}

Ker Kleinova grupa $\mathfrak{K}$ vsebuje osno zrcaljenje, ki je
indirektna izometrija, $\mathfrak{K}$ ni podgrupa grupe
$\mathfrak{I}^+$.

 Grupo izometrij, ki je sestavljena le iz
 identitete $\{\mathcal{E}\}$, bomo imenovali \index{grupa!trivialna}
  \pojem{trivialna grupa} z oznako $\mathfrak{E}$. Ta grupa je očitno podgrupa
   vsake grupe
  izometrij; npr. $\mathfrak{E}<\mathfrak{T}<\mathfrak{I^+}<\mathfrak{I}$ ali
  $\mathfrak{E}<\mathfrak{K}<\mathfrak{I}$.

Če vzamemo le identiteto $\{\mathcal{E}$ in
eno osno zrcaljenje $\mathcal{S}_p\}$ dobimo še eno končno grupo izometrij.
Isto strukturo dobimo tudi, če namesto zrcaljenja čez premico vzamemo
zrcaljenje čez točko, torej $\{\mathcal{E}$ in $\mathcal{S}_O\}$. Čeprav sta
množici različni, je struktura grupe ista, kar ponazorimo s tabelama.
Pravimo, da sta v tem primeru grupi
\index{grupa!izomorfna}\pojem{izomorfni}.

\vspace*{5mm}

\hspace*{12mm}\begin{tabular}{|c||c|c|} \hline
  % after \\ : \hline or \cline{col1-col2} \cline{col3-col4} ...
  $\circ$ & $\mathcal{E}$ & $\mathcal{S}_p$
   \\ \hline \hline
  $\mathcal{E}$ & $\mathcal{E}$& $\mathcal{S}_p$
  \\ \hline
  $\mathcal{S}_p$ & $\mathcal{S}_p$ & $\mathcal{E}$
  \\ \hline
\end{tabular}
\hspace*{22mm}
\begin{tabular}{|c||c|c|} \hline
  % after \\ : \hline or \cline{col1-col2} \cline{col3-col4} ...
  $\circ$ & $\mathcal{E}$ & $\mathcal{S}_O$
   \\ \hline \hline
  $\mathcal{E}$ & $\mathcal{E}$& $\mathcal{S}_O$
  \\ \hline
  $\mathcal{S}_O$ & $\mathcal{S}_O$ & $\mathcal{E}$
  \\ \hline
\end{tabular}

\vspace*{5mm}

Dokazali smo že, da lahko vsako izometrijo ravnine predstavimo
kot kompozitum končnega števila zrcaljenj čez premico (vedno
lahko izberemo največ tri zrcaljenja) - izrek
\ref{IzoKompZrc}. Pravimo, da zrcaljenja čez premico
\pojem{generirajo} grupo $\mathfrak{I}$ vseh izometrij neke
ravnine oz. da so  \pojem{generatorji} grupe  $\mathcal{I}$.
 Spomnimo
se, da kompozitum sodega števila zrcaljenj čez premico predstavlja direktno izometrijo,
pri indirektni izometriji pa je število zrcaljenj čez premico liho.

 Obstaja še ena pomembna vrsta grup izometrij, ki jo bomo
 predstavili v naslednjem izreku.




        \bizrek
        Let $\phi$ be a figure in the plane. The set of all isometries
        of that plane that map the figure $\phi$ to itself forms a group.
          \eizrek

\textbf{\textit{Proof.}}
 Naj bo torej $\mathfrak{G}$ množica vseh izometrij
    ravnine, ki lik  $\phi$ preslikajo vase. Dokažimo,
    da ta množica določa grupo. Za poljubni izometriji
     $f,g\in \mathfrak{G}$  velja $f(\phi)=\phi$ in
$g(\phi)=\phi$. Toda v tem primeru velja tudi $g \circ f (\phi) =
\phi$  in $f^{-1}(\phi) = \phi$. Torej sta izpolnjena pogoja
(1) in (4). Glede pogoja (2) smo že
povedali, da je za operacijo kompozituma preslikav vedno izpolnjen.
Tudi pogoj (3) je v našem primeru izpolnjen, saj
velja $\mathcal{E}(\phi)=\phi$.
 \kdokaz


    Grupo iz prejšnjega izreka - množico
    vseh izometrij
    te ravnine, ki lik  $\phi$ preslikajo vase,
    imenujemo \index{grupa!simetrij}\pojem{grupa simetrij} lika
      $\phi$ z oznako $\mathfrak{G}(\phi)$. Jasno je, da za vsak lik $\phi$
      velja $\mathfrak{G}(\phi)<\mathfrak{I}$.
    Omenimo, da je pri dokazovanju, da podmnožica neke grupe
    (v našem primeru grupe $\mathfrak{I}$) predstavlja grupo oz.
    njeno podgrupo, dovolj preveriti le pogoja (\textit{i}) in
    (\textit{iv}). Pogoj (\textit{ii}) je namreč vedno izpolnjen,
    pogoj (\textit{iii}) pa sledi direktno iz pogojev (\textit{i}) in
    (\textit{iv}).

    Obstaja še ena grupa, ki je podgrupa  grupe simetrij $\mathfrak{G}(\phi)$.
     To grupo tvori
    množica vseh direktnih izometrij iz $\mathfrak{G}(\phi)$. Imenujemo jo
      \index{grupa!rotacij}\pojem{grupa rotacij} lika
      $\phi$ in jo označimo $\mathfrak{G}^+(\phi)$. Jasno je, da je za
      vsak lik njegova grupa rotacij podgrupa njegove grupe
      simetrij oz. $\mathfrak{G}^+(\phi)<\mathfrak{G}(\phi)$.

  V naslednjem zgledu bomo določili grupe simetrij različnih
  likov.



          \bzgled \label{grupeSimPrimeri}
          Determine the symmetry group  and the rotation group of:
           (i) a square, (ii) a rectangle, (iii) a trapezium,
           (iv) a line, (v) a ray, (vi) a circle.
          \ezgled


\textbf{\textit{Solution.}} (Figure \ref{sl.izo.6.12.1.pic})

(\textit{i}) Kvadrat s središčem $O$ ima štiri somernice, ki se
sekajo v točki $O$, in štiri rotacije (vključno z identiteto),
zato je:

 $\mathfrak{G}(\phi)=\{\mathcal{S}_p, \mathcal{S}_q,
\mathcal{S}_r, \mathcal{S}_s, \mathcal{E}, \mathcal{R}_{O,90^0} ,
\mathcal{R}_{O,180^0} , \mathcal{R}_{O,270^0} \}$ in

$\mathfrak{G}^+(\phi)=\{\mathcal{E}, \mathcal{R}_{O,90^0} ,
\mathcal{R}_{O,180^0} , \mathcal{R}_{O,270^0} \}$.

\begin{figure}[!htb]
\centering
\input{sl.izo.6.12.1.pic}
\caption{} \label{sl.izo.6.12.1.pic}
\end{figure}

(\textit{ii}) Pravokotnik s središčem $O$ ima le dve somernici,
zato je $\mathfrak{G}(\phi)=\{\mathcal{S}_p, \mathcal{S}_q,
\mathcal{E},  \mathcal{S}_O \}$ in
$\mathfrak{G}^+(\phi)=\{\mathcal{E},  \mathcal{S}_O \}$. Zanimivo
je, da je prva grupa pravokotnika $\mathfrak{G}(\phi)$
pravzaprav Kleinova grupa $\mathfrak{K}$.

 (\textit{iii}) Če je $\phi$ poljubni trapez, je identiteta
  edina, ki ga preslikava vase. To pomeni, da je
   $\mathfrak{G}(\phi)=\mathfrak{G}^+(\phi)
  =\{\mathcal{E}\}$  grupa, ki smo jo že poimenovali trivialna grupa.

(\textit{iv}) Grupi simetrij in rotacij premice $p$ sta neskončni
grupi in sicer:
 \begin{eqnarray*}
 \mathfrak{G}(p)&=&\{\mathcal{S}_l;l\perp p\}\cup
 \{\mathcal{S}_p\}\cup
 \{\mathcal{S}_O;O\in p\}\cup
 \{ \mathcal{T}_{\overrightarrow{v}};\overrightarrow{v} \parallel p\}\cup
 \{\mathcal{E}\}\\
 \mathfrak{G}^+(p)&=&
 \{\mathcal{S}_O;O\in p\}\cup
 \{ \mathcal{T}_{\overrightarrow{v}};\overrightarrow{v} \parallel p\}\cup
 \{\mathcal{E}\}
  \end{eqnarray*}

(\textit{v}) Za poltrak $h=OA$ je $\mathfrak{G}(h)=\{\mathcal{E},
\mathcal{S}_{h}\}$ in $\mathfrak{G}^+(h)=\{\mathcal{E}\}$.

(\textit{vi}) Pri krožnici $k(O,r)$ so v grupi simetrij vsa
zrcaljenja z osmi skozi točko $O$ in vse rotacije s središčem
$O$ (vključno z identiteto). Grupa rotacij pa je sestavljena samo
iz omenjenih rotacij. Torej:
 \begin{eqnarray*}
 \mathfrak{G}(k)&=&\{\mathcal{S}_l;l\ni O\}\cup
 \{\mathcal{R}_{O,\alpha};\alpha \textrm{ poljuben kot}\}\cup
 \{\mathcal{E}\}\\
 \mathfrak{G}^+(k)&=&\{\mathcal{R}_{O,\alpha};\alpha \textrm{ poljuben kot}\}\cup
 \{\mathcal{E}\},
  \end{eqnarray*}
 kar je bilo treba dokazati. \kdokaz

Videli smo, da ima pravokotnik natanko dve somernici in je tudi
središčno simetričen. Sedaj bomo dokazali splošno trditev.



        \bzgled
         If a figure in the plane has exactly two axes of symmetry,
          then it also has a centre of symmetry.
        \ezgled

\begin{figure}[!htb]
\centering
\input{sl.izo.6.12.2.pic}
\caption{} \label{sl.izo.6.12.2.pic}
\end{figure}

\textbf{\textit{Proof.}} Naj bosta $p$ in $q$ edini somernici lika
$\phi$. Dokažimo, da je $p\perp q$ (Figure \ref{sl.izo.6.12.2.pic}).
Ker je $\mathcal{S}_p(\phi)=\mathcal{S}_q(\phi)=\phi$, je po
izreku \ref{izoTransmutacija} o transmutaciji tudi
$$\mathcal{S}_{\mathcal{S}_p(q)}(\phi)=
\mathcal{S}_p\circ\mathcal{S}_q\circ\mathcal{S}_p^{-1}(\phi)=\phi.$$
Premici $p$ in $q$ sta edini somernici lika $\phi$, zato je bodisi
$\mathcal{S}_p(q)=p$ bodisi $\mathcal{S}_p(q)=q$. V obeh primerih
sledi  $p=q$ oz. $p\perp q$. Toda prva možnost odpade, saj sta $p$
in $q$ po predpostavki različni. Torej  sta premici $p$ in $q$
pravokotni in se sekata v neki točki $O$, zato je:
$$\mathcal{S}_O(\phi)=
\mathcal{S}_q\circ\mathcal{S}_p(\phi)=\mathcal{S}_q(\mathcal{S}_p(\phi))=
\phi,$$ kar pomeni, da je lik središčno simetričen.
 \kdokaz

V terminih grup lahko prejšnjo trditev zapišemo še v naslednji
obliki.



            \bzgled If the symmetry group of some figure contains exactly two reflections
          then it also contains a half-turn.
           \ezgled

Najprej ugotovimo, da imata lahko različna lika isto grupo
simetrij. Npr. grupa simetrij pravokotnika in daljice je Kleinova
grupa $\mathfrak{K}$. Tudi poltrak, enakokraki trikotnik in enakokraki trapez
imajo enako grupo simetrij $\{\mathcal{E}, \mathcal{S}_{p}\}$.

Tudi prve tri in peta grupa simetrij iz primera
\ref{grupeSimPrimeri} so končne grupe, četrta in šesta pa sta
neskončni. Iz  tega primera vidimo tudi, da neskončnost grupe
simetrije nekega lika ni v povezavi z omejenostjo tega lika.
Grupa simetrij omejenega lika je lahko končna, lahko pa
tudi neskončna. Enako velja za neomejene like. Zaradi nadaljnje
uporabe bomo pojem omejenosti natančneje definirali.

Lik $\phi$ je \index{lik!omejen} \pojem{omejen}, če obstaja takšna
daljica $AB$,  da za poljubni točki $X$ in $Y$ tega lika velja
$XY<AB$.

Kakšne so torej grupe simetrij omejenih likov? Ugotovili smo že,
da so lahko končne (kvadrat) ali  neskončne (krožnica). V
 nekaj naslednjih izrekih bomo raziskali grupe simetrij
omejenih likov.



       \bizrek \label{GrupaSomer} If $\phi$ is a bounded figure, then $\mathfrak{G}(\phi)$ does not contain any
        translations.
        \eizrek

\begin{figure}[!htb]
\centering
\input{sl.izo.6.12.3.pic}
\caption{} \label{sl.izo.6.12.3.pic}
\end{figure}

\textbf{\textit{Proof.}} Naj bo $X$ poljubna točka lika $\phi$
(Figure \ref{sl.izo.6.12.3.pic}). Ker je $\phi$ omejen, obstaja
daljica $AB$, ki je daljša od vsake daljice $XY$, kjer je tudi
$Y\in\phi$. Predpostavimo nasprotno, da neka translacija
$\mathcal{T}_{\overrightarrow{v}}$ pripada grupi
$\mathcal{G}(\phi)$. Ker je v tem primeru
$\mathcal{T}_{\overrightarrow{v}}(\phi) = \phi$, je potem tudi
$\mathcal{T}_{\overrightarrow{v}}^n(\phi) = \phi$ oz.
$\mathcal{T}_{n\overrightarrow{v}}(\phi) = \phi$ za poljubno
naravno število $n\in \mathbb{N}$. Torej je
$\mathcal{T}_{n\overrightarrow{v}}(X) = X_n\in \phi$. Iz pogoja
omejenosti lika $\phi$ sledi $|AB|>|XX_n|=|n\overrightarrow{v}|$.
To naj bi veljalo za vsak $n\in \mathbb{N}$, kar seveda ne
drži. Zadnja relacija nas torej pripelje do protislovja, kar
pomeni, da v grupi $\mathfrak{G}(\phi)$  ni translacij.
 \kdokaz

   Iz prejšnjega izreka sledi, da grupa simetrij
   omejenega lika tudi zrcalnega zdrsa ne vsebuje, saj
je njegov kvadrat translacija (zgled \ref{izoZrcdrsZrcdrs}). Kako
pa je z zrcaljenji in rotacijami?



          \bizrek \label{GrupaSomer1} If a bounded figure has more axes of symmetry,
           then they all intersect at one point.
          \eizrek

\textbf{\textit{Proof.}}  Naj bosta $p$ in $q$ somernici omejenega
lika $\phi$. Premici $p$ in $q$ nista vzporedni, saj bi bil sicer
kompozitum ustreznih zrcaljenj $\mathcal{S}_p \circ \mathcal{S}_q$
translacija. Torej se $p$ in $q$ sekata v neki točki $S$. Če je
$r$ tretja poljubna somernica, vsebuje točko $S$, ker bi bil
v nasprotnem primeru (izrek \ref{izoZrcdrsprq}) kompozitum $\mathcal{S}_p \circ
\mathcal{S}_q\circ \mathcal{S}_r$ zrcalni zdrs.
 \kdokaz



          \bizrek \label{GrupaSomerRot} If a bounded figure has at least one axe of symmetry
           and at least one centre of rotation, then
          this centre lies on the axe of symmetry.
         \eizrek

    \textbf{\textit{Proof.}}
    Predpostavimo nasprotno - naj bo $p$ somernica omejenega lika $\phi$ in
    $S$
    središče rotacije
$\mathcal{R}_{S,\alpha}$, ki ta lik slika vase, vendar
$S\notin p$. Rotacijo $\mathcal{R}_{S,\alpha}$ lahko
predstavimo kot kompozitum dveh zrcaljenj čez premico z osema, ki
gresta skozi točko $S$. Torej velja $\mathcal{S}_p \circ
\mathcal{R}_{S,\alpha} = \mathcal{S}_p \circ \mathcal{S}_q\circ
\mathcal{S}_r$. Premice $p$, $q$ in $r$ niso v istem šopu, zato
(izrek \ref{izoZrcdrsprq}) kompozitum $\mathcal{S}_p
\circ \mathcal{R}_{S,\alpha}$ predstavlja zrcalni zdrs, ki pa ne more
biti v grupi $\mathfrak{G}(\phi)$, kar pomeni, da središče $S$ leži
na somernici $p$.
 \kdokaz



        \bizrek \label{GrupaRot} All rotations of a bounded figure
        have the same centre.
         \eizrek

\begin{figure}[!htb]
\centering
\input{sl.izo.6.12.4.pic}
\caption{} \label{sl.izo.6.12.4.pic}
\end{figure}

\textbf{\textit{Proof.}} (Figure \ref{sl.izo.6.12.4.pic}).
 Predpostavimo nasprotno - naj bosta $\mathcal{R}_{O,\alpha}$ in
 $\mathcal{R}_{S,\beta}$
rotaciji, ki omejen lik $\phi$ preslikata vase in velja $O\neq
S$. Tedaj je tudi $O'=\mathcal{R}_{S,\beta}(O)\neq S$. Po izreku
\ref{izoTransmRotac} je
$\mathcal{R}_{O',\alpha}=\mathcal{R}_{S,\beta}\circ
\mathcal{R}_{O,\alpha}\circ\mathcal{R}_{S,\beta}^{-1}$, zato je
  $\mathcal{R}_{O',\alpha}\in \mathfrak{G}(\phi)$ in nato tudi
  $\mathcal{I}=\mathcal{R}_{O',\alpha}^{-1}\circ
  \mathcal{R}_{O,\alpha}\in \mathfrak{G}(\phi)$.
  Naj bo $p$ premica, ki s premico $OO'$ v točki
$O$ določa kot $\frac{1}{2}\alpha$, in $q$ njena vzporednica skozi
točko $O'$. Po izreku \ref{rotacKom2Zrc} je $\mathcal{I} =
\mathcal{S}_q \circ \mathcal{S}_{OO'} \circ \mathcal{S}_{OO'}
\circ \mathcal{S}_p = \mathcal{S}_q \circ \mathcal{S}_p$, zato
je $\mathcal{I}$ translacija, kar pa ni mogoče. To pomeni, da
$O\neq S$ ne velja, torej imajo vse rotacije omejenega lika
       isto središče.
      \kdokaz

    Iz prejšnjih izrekov dobimo naslednjo trditev.




         \bizrek \label{GrupaOmejenLik}  If symmetry group of a bounded figure $\phi$ is not trivial,
        then there is a point $S$ such that the only possible isometries in this group are:
          \\
          (i) the identity map,\\
            (ii) rotations with the centre $S$,\\
         (iii) reflections with axes,
             which all contains the point $S$.
         \eizrek

    Na ta način smo ugotovili, kakšne so vse možne grupe omejenih likov.
    Te grupe niso nujno
končne. Kakšne so potem končne grupe? Pa raziščimo
nekaj primerov grup simetrij.


           \bzgled \label{GrupaDiederska} Determine the symmetry group and the rotation group
            of a regular $n$-gon.
          \ezgled

\begin{figure}[!htb]
\centering
\input{sl.izo.6.12.5.pic}
\caption{} \label{sl.izo.6.12.5.pic}
\end{figure}

\textbf{\textit{Solution.}} (Figure \ref{sl.izo.6.12.5.pic})
 Ne glede na to, ali je $n$ sodo ali liho število, ima pravilni $n$-kotnik
 natanko $n$
somernic (glej razdelek \ref{odd3PravilniVeck}). Osnovni kot
rotacije, ki $n$-kotnik preslika vase je $\theta = \frac{360^0}{n}$.
Če torej z $\mathcal{D}_n$ in $\mathcal{C}_n$ označimo grupo
simetrij in grupo rotacij
    pravilnega $n$-kotnika, z $O$ pa njegovo središče, dobimo:
 \begin{eqnarray*}
  \mathfrak{D}_n&=&\{ \mathcal{S}_{p_1}, \mathcal{S}_{p_2},\ldots,
  \mathcal{S}_{p_n},
   \mathcal{E}, \mathcal{R}_{O,\theta}, \mathcal{R}_{O,2\theta},\ldots
   \mathcal{R}_{O,(n-1)\theta}\}\\
   \mathfrak{C}_n&=&\{ \mathcal{E}, \mathcal{R}_{O,\theta},
   \mathcal{R}_{O,2\theta},\ldots \mathcal{R}_{O,(n-1)\theta}\},
 \end{eqnarray*}
  kar je bilo treba določiti. \kdokaz

    Grupo simetrij $\mathfrak{D}_n$ pravilnega $n$-kotnika iz
    prejšnjega zgleda
    imenujemo
    \index{grupa!diedrska} \pojem{diedrska grupa}, grupo rotacij
    pravilnega $n$-kotnika $\mathfrak{C}_n$ pa
     \index{grupa!ciklična}  \pojem{ciklična grupa}. Jasno je, da
     velja $\mathfrak{C}_n<\mathfrak{D}_n$.

  Grupe $\mathfrak{D}_n$ in $\mathfrak{C}_n$ ($n\geq 3$)
   lahko posplošimo tudi za primere $n < 3$, čeprav v tem primeru ne govorimo več
   o $n$-kotniku. $\mathfrak{D}_2$ je tako
pravzaprav Kleinova grupa $\mathfrak{K}$ (grupa simetrije daljice
oziroma nekakšnega ‘‘$2$-kotnika’’). Grupo $\mathfrak{C}_2$
sestavljata identiteta in ena središčna simetrija. To grupo si lahko
predstavimo kot grupo rotacij daljice. Grupa $\mathfrak{D}_1$
vsebuje identiteto in osno zrcaljenje, je pa izomorfna grupi
$\mathfrak{C}_2$ (primer, ki smo ga že omenili na začetku tega
razdelka). Grupa $\mathfrak{C}_1$ je trivialna grupa $\mathfrak{E}$.

\begin{figure}[!htb]
\centering
\input{sl.izo.6.12.5a.pic}
\caption{} \label{sl.izo.6.12.5a.pic}
\end{figure}

Omenimo, da grupe $\mathfrak{D}_n$ in $\mathfrak{C}_n$ ($n\in
\mathbb{N}$) niso grupe simetrij le za pravilne $n$-kotnike (Figure
\ref{sl.izo.6.12.5a.pic}). Lahko so grupe simetrij tudi neomejenih
likov. Npr. Kleinova grupa $\mathfrak{D}_2$ je tudi grupa simetrij
hiperbole z enačbo $x\cdot y= 1$ v kartezičnem pravokotnem sistemu
$O_{xy}$. Somernici sta tedaj določeni z enačbami $y = x$ in $y =
-x$, središče simetrije pa je izhodišče $O$. $\mathfrak{D}_2$ je
tudi grupa simetrij elipse, ki je dana z enačbo
$\frac{x^2}{a^2}+\frac{y^2}{b^2}=1$.

Jasno je, da so vse grupe $\mathfrak{D}_n$ in $\mathfrak{C}_n$
($n\in \mathbb{N}$)  končne grupe izometrij. Ugotovili bomo, da so
to edine končne grupe izometrij!



           \bizrek \label{GrupaLeonardo} \index{izrek!Leonarda da Vincija}
           (Leonardo da Vinci\footnote{Slavni italijanski slikar, arhitekt in
           izumitelj \index{Leonardo da Vinci}
              \textit{Leonardo da Vinci} (1452--1519) je proučeval vse možne
            simetrije središčne stavbe in možne razporeditve kapel okoli
            nje, ki ohranjajo osnovno simetrijo.}) The only finite groups of isometries are $\mathfrak{D}_n$ and $\mathfrak{C}_n$.
              \eizrek



\textbf{\textit{Proof.}} Naj bo $\mathfrak{G}$ končna grupa
izometrij. Če bi vsebovala translacijo
$\mathcal{T}_{\overrightarrow{v}}$, bi imela neskončno
podgrupo $\{ \mathcal{E}, \mathcal{T}_{\overrightarrow{v}},
\mathcal{T}_{\overrightarrow{v}}^2,
\mathcal{T}_{\overrightarrow{v}}^3,\ldots \}$, kar ni mogoče. Torej
je $\mathfrak{G}$ grupa brez translacij. Na podoben način kot prej
(izrek \ref{GrupaOmejenLik}) lahko dokažemo, da so edine možne
izometrije v tej grupi (razen identične preslikave) rotacije z
istim središčem $S$ in zrcaljenja čez premico z osmi, ki gredo
skozi točko $S$. Če grupa nima nobene rotacije, je možno kvečjemu
eno osno zrcaljenje (že dve bi generirali rotacijo). V tem primeru
sta edini možni grupi $\mathfrak{D}_1$ in $\mathfrak{C}_1$.

Brez škode za splošnost predpostavimo, da so vsi koti rotacij
pozitivni. Ker je grupa $\mathfrak{G}$ končna, obstaja rotacija
$\mathcal{R}_{O,\theta}$ z najmanjšim kotom $\theta$. Iz istega
razloga obstaja naravno število $n$, za katero je
$\mathcal{R}_{O,\theta}^n = \mathcal{E}$ (sicer bi končna
grupa $\mathfrak{G}$ imela neskončno podgrupo $\{ \mathcal{E},
\mathcal{R}_{O,\theta}, \mathcal{R}_{O,\theta}^2,
\mathcal{R}_{O,\theta}^3,\ldots \}$).
 Torej je  $\theta =
\frac{360^0}{n}$.

 Naj bo $\mathcal{R}_{O,\delta}$ poljubna rotacija iz
 $\mathfrak{G}$.
Dokažimo, da je kot $\delta$ večkratnik kota $\theta$ oz.
$\delta=k\cdot \theta$ za nek $k\in \mathbb{N}$. Predpostavimo
nasprotno, da tak $k$ ne obstaja. Ker je $\delta>\theta$,  za
nek $l\in \mathbb{N}$ velja $l\theta<\delta<(l+1)\theta$. Če
preoblikujemo, dobimo $0<\delta-l\theta<\theta$. Toda
$\mathcal{R}_{O,\delta-l\theta}=\mathcal{R}_{O,\delta}\circ
\left(\mathcal{R}_{O,\theta}^{-1}\right)^l\in \mathfrak{G}$, kar je
protislovno s tem, kako smo definirali kot $\theta$. Torej so
vse rotacije $\mathfrak{G}$ grupe oblike
$\mathcal{R}_{O,\theta}^k$, $k\in \{ 1,2,\ldots, n\}$.

Če v grupi ni osnih zrcaljenj, je torej $\mathfrak{G}$ ciklična
grupa oz. $\mathfrak{G}=\mathfrak{C}_n$ za nek $n\in \mathbb{N}$. Če so v grupi
$\mathfrak{G}$ še zrcaljenja čez premico, so njihovi kompozitumi
(vsakih dveh zrcaljenj) rotacije, zato jim vsaki par osi  določa kot, ki je
polovica nekega kota rotacij. V tem primeru je
$\mathfrak{G}$ diederska grupa oz. $\mathfrak{G}=\mathfrak{D}_n$ za nek $n\in \mathbb{N}$.
 \kdokaz

    Še enkrat poudarimo, da končna grupa simetrij in grupa simetrij
    omejenega lika nista isti pojem.
    Grupa simetrij omejenih likov namreč ni nujno končna
    (npr. grupa simetrij krožnice). Prav tako imajo lahko neomejeni liki
končno grupo simetrij (npr. poltrak). Seveda obstajajo omejeni
liki s končno grupo simetrij (npr. pravilni večkotnik) in
neomejeni liki z neskončno grupo simetrij (npr. premica). Ta
dejstva bomo ponazorili z naslednjim diagramom (Figure
\ref{sl.izo.6.12.6.pic}).


\begin{figure}[!htb]
\centering
\input{sl.izo.6.12.6.pic}
\caption{} \label{sl.izo.6.12.6.pic}
\end{figure}


    Omenili bomo še eno vrsto neskončnih grup izometrij, ki ohranjajo
     določena tlakovanja ravnine (glej razdelek
     \ref{odd3Tlakovanja}). Torej ne gre le za grupe simetrij
     teh tlakovanj, ampak tudi za vse podgrupe teh grup.

     Če npr. izberemo
      tlakovanja
     ravnine s poljubnim paralelogramom  $ABCD$ (Figure
    \ref{sl.izo.6.12.7.pic}), je
     neskončna grupa simetrij tega tlakovanja generirana z
    zrcaljenji glede na oglišče $A$ ter središči daljic $AB$ in $AD$. V tej
    grupi so potem tudi zrcaljenja glede na oglišča, središča stranic in
    presečišča
    diagonal vseh
    paralelogramov tega tlakovanja ter vse translacije, ki so kompozitumi
     po dveh od
    teh zrcaljenj.
     Toda ena njena podgrupa, ki tudi ohranja isto tlakovanje, je grupa
     vseh omenjenih translacij. Generirana je z
     dvema
     translacijama $\mathcal{T}_{\overrightarrow{AB}}$
     in $\mathcal{T}_{\overrightarrow{AD}}$.

\begin{figure}[!htb]
\centering
\input{sl.izo.6.12.7aa.pic}
\input{sl.izo.6.12.7bb.pic}
\caption{} \label{sl.izo.6.12.7.pic}
\end{figure}

     Grupa simetrij v
     primeru tlakovanja s pravilnim trikotnikom $ABC$ (Figure
    \ref{sl.izo.6.12.7.pic})
     oz. tlakovanja $(3,6)$  vsebuje
      rotacije v ogliščih mreže za kote $k\cdot 60^0$,
      zrcaljenja čez premice, ki so
     določene s stranicami in višinami teh trikotnikov, ter translacije,
      ki jih generirata translaciji $\mathcal{T}_{\overrightarrow{AB}}$
     in $\mathcal{T}_{\overrightarrow{AC}}$. Ta grupa ima več
     različnih podgrup, ki vse ohranjajo tlakovanje $(3,6)$.

     Vse takšne grupe, ki ohranjajo določena tlakovanja, imenujemo
    \index{grupa!diskretna} \pojem{diskretne grupe izometrij}.
     Formalno jih definiramo na
    naslednji način: Grupa izometrij $\mathfrak{G}$ je diskretna grupa izometrij,
     če obstaja tak
    $\varepsilon>0$, da so dolžine vektorjev vseh translacij in
    mere kotov vseh rotacij iz te grupe večji od $\varepsilon$.



 Razlikujemo dve vrsti diskretnih grup izometrij: \index{grupa!frizna}
 \pojem{frizne
 grupe}\footnote{Termin \textit{friz}, ki
  so ga uporabljali že Stari Grki,
 je pomenil ponavljajoč se vzorec bordur.} in \index{grupa!tapetna}\pojem{tapetne grupe}. Pri friznih
 grupah je podgrupa
 vseh translacij generirana z eno samo translacijo, pri tapetnih
 grupah pa z dvema translacijama, ki ju določata nekolinearna
 vektorja. Dokazano je, da obstaja natanko 7 friznih grup in 17
 tapetnih grup\footnote{Vseh 17 grup oz. ustreznih vrst ornamentov je
 bilo znanih  že Egipčanom,
 pogosto pa so jih uporabljali tudi  muslimanski umetniki. V dvorcu Alhambra (Španija) so
 Mavri v 14. stoletju
 naslikali vseh 17 vrst ornamentov. Formalni dokaz, da obstaja
 natanko
 17 tapetnih grup, je prvi podal ruski matematik,
 kristalograf in minerolog \index{Fedorov, E.}
 \textit{E. Fedorov}
  (1853–-1919) leta 1891,
 nato je leta 1924 to delo dopolnil in nadaljeval madžarski matematik
  \index{Pólya, G.}\textit{G. Pólya} (1887–-1985), ki je bolj znan po svoji
  znameniti knjigi  ‘‘Kako rešujemo matematične probleme’’.}.
  Frizne grupe določajo 7 različnih vrst \index{bordura}
 \pojem{bordur} -
 trak s ponavljajočimi se vzorci (Figure \ref{sl.izo.6.12.7}a\footnote{http://mathworld.wolfram.com/WallpaperGroups.html}), tapetne
 pa 17 različnih vrst \pojem{ornamentov} - prekrivanje ravnine
 z enakimi vzorci (Figure \ref{sl.izo.6.12.7}b\footnote{http://www.quadibloc.com/math/tilint.htm}).

\begin{figure}[!h]
\centering
 \includegraphics[bb=0 0 7cm 7cm]{bands.eps}\\
\vspace*{15mm}
\includegraphics[width=1\textwidth]{wall17_phpbtUJbf.eps}
\caption{} \label{sl.izo.6.12.7}
\end{figure}

%v bmp
%\begin{figure}[!h]
%\centering
 %\includegraphics[bb=0 0 7cm 7cm]{bands.bmp}\\
%\vspace*{8mm}
%\includegraphics[bb=0 0 12cm 9.85cm]{wall17_phpbtUJbf.bmp}
%\caption{} \label{sl.izo.6.12.7}
%\end{figure}



            \bnaloga\footnote{40. IMO Romania - 1999, Problem 1.}
           Determine all finite sets $\mathbf {S}$ of at least three points in the plane which satisfy the
           following condition:\\
            for any two distinct points $A$ and $B$ in $\mathbf {S}$, the perpendicular bisector
            of the line segment $AB$ is an axis of symmetry for $\mathbf {S}$.
            \enaloga


\textbf{\textit{Solution.}}
  Naj bo $\mathbf {S}=\{A_1,A_2,\ldots
A_n \}$ in $\mathcal{M}=\{s_{A_iA_j};\hspace*{1mm}A_i,A_j\in
\mathcal{S}\}$ množica vseh simetral. Ker je  $\mathbf {S}$
omejena množica (oz. lik), po izreku \ref{GrupaSomer1} vse
somernice te množice oz. simetrale iz množice $\mathcal{M}$ potekajo skozi
eno točko - označimo jo z $O$. Jasno je, da je pri tem $O$
presečišče poljubnih dveh simetral iz $\mathcal{M}$, zato je tudi
$O=s_{A_1A_2}\cap s_{A_2A_3}$.

Naj bo $k$ krožnica s središčem $O$, ki poteka skozi točko $A_1$.
Dokažimo najprej, da vse točke množice $\mathbf {S}$ ležijo na tej
krožnici. Naj bo $A_i\in \mathbf {S}$ poljubna točka. Iz
dokazanega $O\in s_{A_1A_i}$ in $A_1 \in k$ sledi
$A_i=\mathcal{S}_{s_{A_1A_i}}(A_1)\in
\mathcal{S}_{s_{A_1A_i}}(k)=k$ (Figure \ref{sl.izo.6.12.IMO1.pic}).



\begin{figure}[!htb]
\centering
\input{sl.izo.6.12.IMO1.pic}
\caption{} \label{sl.izo.6.12.IMO1.pic}
\end{figure}

Brez škode za splošnost lahko predpostavimo, da so točke $\mathbf
{S}$ urejene po vrsti na krožnici $k$, tako da velja $\angle
A_1OA_2<\angle A_1OA_3<\cdots \angle A_1OA_n$ (drugače lahko
naredimo novo označevanje teh točk). To pomeni, da $A_1A_2\ldots
A_n$ predstavlja večkotnik, ki je včrtan krožnici $k$. Dokažimo
še, da je ta večkotnik pravilen. Ker je ta večkotnik koncikličen, je dovolj
dokazati še, da ima vse stranice skladne. Ker je $s_{A_1A_3}$
somernica množice $\mathbf {S}$ (zato tudi večkotnika $A_1A_2\ldots
A_n$), je $A_1A_2\cong A_2A_3$. Podobno je tudi $s_{A_iA_{i+2}}$
($i\in \{2,3,\ldots n-1\}$) somernica  večkotnika $A_1A_2\ldots A_n$
in velja $A_iA_{i+1}\cong A_{i+1}A_{i+2}$, kar pomeni, da je
$A_1A_2\ldots A_n$ pravilni večkotnik.

Množica $\mathbf {S}$ torej predstavlja oglišča pravilnega
večkotnika.
 \kdokaz

%\vspace*{31mm}

\newpage

\naloge{Exercises}

\begin{enumerate}

  \item Dana je premica $p$ ter točki $A$ in $B$, ki ležita na
  nasprotnih straneh premice $p$. Konstruiraj točko
  $X$, ki leži na
premici $p$, tako da bo razlika $|AX|-|XB|$ maksimalna.

  \item V ravnini so dane premice $p$, $q$ in $r$. Konstruiraj
  enakostranični trikotnik $ABC$, tako da
   oglišče $B$ leži na premici $p$, $C$ na $q$,
višina iz oglišča $A$ pa na premici $r$.

\item Dan je štirikotnik $ABCD$ in točka $S$. Načrtaj paralelogram
s središčem v točki $S$, tako
da njegova oglišča ležijo na nosilkah
stranic danega štirikotnika.

\item Naj bo $\mathcal{I}$ indirektna izometrija ravnine, ki
 preslika  točko $A$ v točko $B$,
$B$ pa v $A$. Dokaži, da je $\mathcal{I}$ osno zrcaljenje.

\item  Naj bosta $K$ in $L$ točki, ki sta simetrični z ogliščem
$A$ trikotnika $ABC$ glede na
simetrali notranjih kotov ob ogliščih  $B$ in $C$. Točka $P$ naj bo
dotikališče včrtane krožnice tega trikotnika in stranice $BC$.
Dokaži, da je $P$ središče daljice $KL$.

\item  Naj bosta $k$ in $l$ krožnici na različnih bregovih premice
$p$. Načrtaj enakostranični trikotnik $ABC$, tako da njegova višina
$AA'$ leži na premici $p$, oglišče $B$ na krožnici $k$, oglišče
$C$ pa na krožnici $l$.

\item Naj bo $k$ krožnica ter $a$, $b$ in $c$ premice v isti ravnini.
Načrtaj trikotnik $ABC$, ki je
včrtan krožnici $k$, tako da bodo njegove stranice $BC$, $AC$ in
 $AB$ vzporedne po vrsti s premicami $a$, $b$ in $c$.

\item  Naj bo $ABCDE$ tetivni petkotnik, v katerem je $BC\parallel
DE$ in $CD\parallel EA$.
Dokaži, da oglišče $D$ leži na simetrali daljice $AB$.

\item  Premice $p$, $q$ in $r$ ležijo v isti ravnini. Dokaži
ekvivalenco $\mathcal{S}_r\circ\mathcal{S}_q\circ\mathcal{S}_p
 =\mathcal{S}_p\circ\mathcal{S}_q\circ \mathcal{S}_r$ natanko
 tedaj, ko premice $p$, $q$ in $r$ pripadajo istemu šopu.

\item Naj bodo $O$, $P$ in $Q$ tri nekolinearne točke. Konstruiraj
kvadrat $ABCD$ (v ravnini $OPQ$) s središčem v  točki $O$, tako
da točki $P$ in $Q$ po vrsti
ležita na premicah $AB$ in $BC$.

\item Naj bosta $\mathcal{R}_{S,\alpha}$ rotacija in $\mathcal{S}_p$
osno zrcaljenje v isti ravnini in $S\in p$.
Dokaži, da kompozituma $\mathcal{R}_{S,\alpha}\circ\mathcal{S}_p$
in $\mathcal{S}_p\circ\mathcal{R}_{S,\alpha}$ predstavljata osno
zrcaljenje.

\item Dani sta točka $A$ in krožnica $k$ v isti ravnini. Načrtaj
kvadrat $ABCD$, tako da krajišči diagonale $BD$ ležita na krožnici $k$.

\item Naj bo $ABC$ poljubni trikotnik. Dokaži:
 $$\mathcal{R}_{C,2\measuredangle BCA}\circ
 \mathcal{R}_{B,2\measuredangle ABC}\circ
 \mathcal{R}_{A,2\measuredangle CAB}=\mathcal{E}.$$

\item Dokaži, da kompozitum osnega zrcaljenja $\mathcal{S}_p$
in središčnega zrcaljenja $\mathcal{S}_S$ ($S\in p$)
predstavlja osno zrcaljenje.

\item Naj bodo $O$, $P$ in $Q$ tri nekolinearne točke. Konstruiraj
kvadrat $ABCD$ (v ravnini $OPQ$) s središčem v točki $O$, tako
 da točki $P$ in $Q$
ležita po vrsti na premicah $AB$ in $CD$.

\item Kaj predstavlja kompozitum translacije in središčnega zrcaljenja?

\item Dani so premica $p$ ter krožnici  $k$ in $l$, ki
ležijo v isti ravnini. Načrtaj premico, ki je vzporedna s
premico $p$, tako da na krožnicah $k$ in $l$ določa skladni tetivi.

\item Naj bo $c$ premica, ki seka vzporednici $a$ in $b$, ter $l$
 daljica. Načrtaj enakostranični trikotnik $ABC$, tako da velja
  $A\in a$, $B\in b$, $C\in c$ in $AB\cong l$.

\item Dokaži, da kompozitum rotacije in osnega zrcaljenja
neke ravnine
predstavlja zrcalni zdrs natanko tedaj, ko središče
rotacije ne leži na osi osnega zrcaljenja.

\item Naj bo $ABC$ enakostranični trikotnik. Dokaži,
da kompozitum $\mathcal{S}_{AB}
    \circ\mathcal{S}_{CA}
    \circ\mathcal{S}_{BC}$
predstavlja zrcalni zdrs. Določi še vektor in os tega zdrsa.

\item  Dani sta  točki $A$ in $B$ na istem bregu premice
$p$.
Načrtaj daljico  $XY$, ki leži na premici $p$ in je skladna
z dano daljico $l$, tako da bo vsota
$|AX|+|XY|+|YB|$ minimalna.

\item  Naj bo $ABC$ enakokraki pravokotni trikotnik s pravim
kotom pri oglišču $A$. Kaj predstavlja kompozitum:
$\mathcal{G}_{\overrightarrow{AB}}\circ \mathcal{G}_{\overrightarrow{CA}}$?

\item V isti ravnini so dane  premice  $a$, $b$ in $c$.
Načrtaj točki $A\in a$ in $B\in b$
tako, da bo $\mathcal{S}_c(A)=B$.

\item  Dani sta premici $p$ in $q$ ter točka $A$ v isti ravnini.
Načrtaj točki $B$ in $C$ tako,
da bosta premici $p$ in $q$ simetrali notranjih kotov pri
ogliščih $B$ in $C$ trikotnika $ABC$.

\item  Naj bodo $p$, $q$ in $r$ premice ter $K$ in $L$ točki v
isti ravnini. Načrtaj
premici $s$ in $s'$, ki gresta po vrsti skozi točki $K$ in $L$,
tako da velja $\mathcal{S}_r\circ\mathcal{S}_q\circ\mathcal{S}_p(s)=s'$.

\item  Naj bo $s$ simetrala enega od kotov, ki jih določata premici $p$
in $q$. Dokaži, da je $\mathcal{S}_s\circ\mathcal{S}_p =
\mathcal{S}_q\circ\mathcal{S}_s$.

\item Naj bo $S$ središče trikotniku $ABC$ včrtane krožnice in $P$ točka,
 v kateri se ta krožnica dotika stranice $BC$. Dokaži: $$\mathcal{S}_{SC}
 \circ\mathcal{S}_{SA}\circ\mathcal{S}_{SB} =\mathcal{S}_{SP}.$$

\item  Premice $p$, $q$ in $r$ neke ravnine potekajo skozi središče
$S$ krožnice $k$.
Načrtaj trikotnik $ABC$, ki je očrtan  tej krožnici, tako da
bodo premice $p$, $q$ in $r$ simetrale notranjih kotov pri
ogliščih $A$, $B$ in $C$ tega trikotnika.

\item  Premice $p$, $q$, $r$, $s$ in $t$ neke ravnine se sekajo
v točki $O$, točka $M$ pa leži na premici $p$.
Načrtaj tak petkotnik, da je $M$ središče ene njegove stranice,
premice $p$, $q$, $r$, $s$ in $t$ pa simetrale stranic.

\item  Točka $P$ leži v ravnini trikotnika $ABC$. Dokaži, da
premice, ki so simetrične s
premicami $AP$, $BP$ in $CP$ glede na simetrale notranjih kotov
 ob ogliščih $A$, $B$ in $C$ tega trikotnika, pripadajo istemu šopu.

\item  Izračunaj kot, ki ga določata premici $p$ in $q$, če velja:
$\mathcal{S}_p\circ\mathcal{S}_q\circ\mathcal{S}_p =
\mathcal{S}_q\circ\mathcal{S}_p\circ\mathcal{S}_q$.

\item  Naj bosta $\mathcal{R}_{A,\alpha}$ in $\mathcal{R}_{B,\beta}$
rotaciji v isti ravnini. Določi vse točke $X$ v tej ravnini, za
katere velja
$\mathcal{R}_{A,\alpha}(X)=\mathcal{R}_{B,\beta}(X)$.

\item  Premici $p$ in $q$ se pod kotom $60^0$ sekata v središču $O$ enakostraničnega
 trikotnika $ABC$. Dokaži, da
 sta odseka, ki jih na  premicah določata stranici
 trikotnika $ABC$, skladni daljici.

\item   Točka $S$ naj bo središče pravilnega petkotnika $ABCDE$.
Dokaži, da velja:
 $$\overrightarrow{SA} + \overrightarrow{SB} + \overrightarrow{SC}
  + \overrightarrow{SD} + \overrightarrow{SE} = \overrightarrow{0}.$$

\item   Dokaži, da se diagonale pravilnega petkotnika
sekajo v točkah, ki so tudi
oglišča pravilnega petkotnika.

\item   Naj bosta $ABP$ in $BCQ$  pravilna trikotnika z
isto orientacijo in $\mathcal{B}(A,B,C)$. Točki $K$ in $L$ sta
središči daljic $AQ$ in $PC$. Dokaži, da je tudi $BLK$ pravilni
trikotnik.

\item   Dane so tri koncentrične krožnice in premica v isti ravnini.
Načrtaj pravilni trikotnik tako, da njegova oglišča
po vrsti ležijo na  teh krožnicah, ena stranica pa bo vzporedna
dani premici.

\item  Točka $P$ je notranja točka pravilnega trikotnika
$ABC$, tako da velja
$\angle APB=113^0$ in $\angle BPC=123^0$. Izračunaj velikosti
kotov trikotnika, čigar stranice so skladne z daljicami
$PA$, $PB$ in $PC$.

\item   Dane so nekolinearne točke $P$, $Q$ in $R$.
Načrtaj trikotnik $ABC$,
da bodo $P$, $Q$ in $R$  središča kvadratov, ki so konstruirani
nad stranicami $BC$, $CA$ in $AB$ tega trikotnika.

\item   Naj bosta  $A$ in $B$  točki ter $p$ premica v isti
ravnini. Dokaži, da je
kompozitum $\mathcal{S}_B\circ\mathcal{S}_p\circ\mathcal{S}_A$
 osno zrcaljenje natanko tedaj, ko je $AB\perp p$.

\item   Naj bodo $p$, $q$ in $r$ tangente trikotniku $ABC$ včrtane krožnice,
ki so vzporedne z njegovimi stranicami
$BC$, $AC$ in $AB$. Dokaži,
da premice $p$, $q$, $r$, $BC$, $AC$ in $AB$ določajo tak
 šestkotnik, v katerem so pari nasprotnih stranic skladne daljice.

\item   Načrtaj trikotnik s podatki:  $\alpha$, $t_b$, $t_c$.

\item Naj bosta $ALKB$ in $ACPQ$ kvadrata, ki sta zunaj trikotnika $ABC$ načrtana
nad stranicama $AB$ in
$AC$ ter $X$ središče stranice $BC$. Dokaži, da
je $AX\perp LQ$ in
$|AX|=\frac{1}{2}|QL|$.

\item Naj bo $O$ središče pravilnega trikotnika $ABC$ ter $D$ in
$E$ točki stranic $CA$ in $CB$, tako
 da velja $CD\cong CE$. Točka $F$ je četrto oglišče paralelograma
 $BODF$. Dokaži,
da je trikotnik $OEF$ pravilen.

\item Naj bo $L$ točka, v kateri se trikotniku
$ABC$ včrtana krožnica dotika njegove stranice $BC$.
Dokaži: $$\mathcal{R}_{C,\measuredangle ACB}\circ\mathcal{R}_{A,\measuredangle BAC}
\circ\mathcal{R}_{B,\measuredangle CBA} =\mathcal{S}_L.$$

\item Točke $P$ in $Q$ ter $M$ in $N$ so središča po dveh kvadratov,
ki so zunaj načrtani nad
nasprotnimi stranicami poljubnega štirikotnika. Dokaži, da je
$PQ\perp MN$ in $PQ\cong MN$.

\item  Naj bosta $APB$ in $ACQ$ pravilna trikotnika, ki sta zunaj trikotnika $ABC$
načrtana  nad
stranicama $AB$ in $AC$. Točka $S$ je središče
stranice $BC$ in $O$ središče trikotnika $ACQ$. Dokaži, da je
$|OP|=2|OS|$.

\item Dokaži, da osno zrcaljenje in translacija neke ravnine
komutirata natanko tedaj, ko je os tega zrcaljenja vzporedna z
vektorjem translacije.

\item   V isti ravnini so dani premica $p$, krožnici $k$ in $l$ ter daljica $d$.
Načrtaj romb $ABCD$ s stranico, ki je skladna daljici $d$, stranica $AB$ leži na
premici $p$, oglišči $C$ in $D$ pa po vrsti ležita na krožnicah $k$ in $l$.

\item  Naj bo $p$ premica, $A$ in $B$ pa točki, ki ležita na istem bregu
premice $p$, ter $d$ daljica v isti ravnini.
Načrtaj točki $X$ in $Y$ na premici $p$ tako, da bo $AX\cong BY$ in $XY\cong d$.

\item  Naj bo $H$ višinska točka trikotnika $ABC$ in $R$ polmer očrtane krožnice tega
trikotnika. Dokaži, da je $|AB|^2+|CH|^2=4R^2$.

\item  Naj bo $EAB$ trikotnik, ki je načrtan nad stranico $AB$ kvadrata
$ABCD$. Naj bo tudi $M=pr_{\perp AE}(C)$ in $N=pr_{\perp BE}(D)$ ter točka $P$
presečišče premic $CM$ in $DN$. Dokaži, da je $PE\perp AB$.

\item  Načrtaj enakostranični trikotnik $ABC$ tako, da njegova oglišča po vrsti
ležijo na treh vzporednicah $a$, $b$ in $c$ v isti ravnini,
središče tega trikotnika pa leži na premici $s$, ki seka
premice $a$, $b$ in $c$.

\item Če ima petkotnik vsaj dve osi simetrije,  je pravilen. Dokaži.

\item Naj bodo $A$, $B$ in $C$ tri kolinearne točke. Kaj predstavlja
kompozitum $\mathcal{G}_{\overrightarrow{BC}}\circ \mathcal{S}_A$?

\item Naj bodo $p$, $q$ in $r$ premice, ki niso iz istega šopa, in $A$ točka v isti
ravnini. Načrtaj premico $s$, ki poteka skozi točko $A$, tako da velja
$\mathcal{S}_r\circ \mathcal{S}_q\circ \mathcal{S}_p(s)=s'$ in $s\parallel s'$.

%nove naloge
%___________________________________

\item Naj bosta $Z$ in $K$ notranji točki pravokotnika $ABCD$.
Načrtaj točke $A_1$, $B_1$, $C_1$ in $D_1$, ki po vrsti ležijo na
stranicah $AB$, $BC$, $CD$ in $DA$ tega pravokotnika, tako da velja
$\angle ZA_1A\cong\angle B_1A_1B$, $\angle A_1B_1B\cong\angle C_1B_1C$,
$\angle B_1C_1C\cong\angle D_1C_1D$ in  $\angle C_1D_1D\cong\angle KD_1A$.

\item Točka $A$ leži na premici $a$, točka $B$ pa na premici $b$.
Določi rotacijo, ki preslika premico $a$ v premico $b$ in točko $A$ v točko $B$.

\item V središču kvadrata se sekata dve pravokotnici.
Dokaži, da ti pravokotnici sekata stranice kvadrata v točkah, ki so
oglišča novega kvadrata.

\item Dana je krožnica $k$ in premice $a$, $b$, $c$, $d$ in $e$, ki ležijo
v isti ravnini. Krožnici $k$ včrtaj petkotnik s stranicami, ki
so po vrsti vzporedne s premicami  $a$, $b$, $c$, $d$ in $e$.

\item Točka $P$ leži v notranjosti kota $aOb$. Načrtaj premico $p$ skozi točko $P$,
 ki s krakoma $a$ in $b$ določa trikotnik z najmanjšo ploščino.

\item Paralelogram $PQKL$ naj bo včrtan v paralelogram $ABCD$ (oglišča prvega ležijo na stranicah
drugega). Dokaži, da imata paralelograma skupno središče.

\item Loki $l_1, l_2,\cdots , l_n$ ležijo na krožnici $k$ in je vsota njihovih
dolžin manjša od polobsega te krožnice. Dokaži, da obstaja tak premer
$PQ$ krožnice $k$, da nobeno od njegovih krajišč ne leži na nobenem od
lokov $l_1, l_2,\cdots , l_n$.

\item Dana je krožnica $k(S,20)$. Igralca $\mathcal{A}$ in $\mathcal{B}$
izmenično rišeta krožnici s polmeri $x_i$ ($1<x_i<2$), ki ležita v
notranjosti krožnice $k$ tako, da nobena od krožnic nima skupnih točk
z nobeno od prejšnjih narisanih krožnic. Zmaga igralec, ki
nariše zadnjo krožnico. Ali obstaja zmagovalna strategija za nekega od
igralcev $\mathcal{A}$ in $\mathcal{B}$?

\item Naj bosta $AB$ in $CD$ tetivi krožnice $k$, ki nimata skupnih
točk, in $P$ poljubna točka, ki leži na tetivi $CD$. Načrtaj takšno
točko $X$ na krožnici $k$, da tetivi $XA$ in $XB$ sekata tetivo $CD$
v točkah $Y$ in $Z$ tako, da je točka $P$ središče daljice $ZY$.

\item Zunaj paralelograma $ABCD$ so nad njegovimi stranicami  konstruirani enakostranični
 trikotniki. Dokaži, da so središča teh tri\-kot\-ni\-kov oglišča novega paralelograma.

\item Načrtaj trapez, tako da bosta osnovnici skladni z danima daljicama $a$ in
$c$, diagonali pa skladni z danima daljicama $e$ in $f$.

\item Mesti (točki) $A$ in $B$ sta na različnih bregovih reke
(pasa, ki ga določata vzporednici $p$ in $q$). Potrebno je narediti
most (daljico $PQ\perp p$, $P\in P$ in $Q\in q$) čez reko, ki bo povezal
mesti $A$ in $B$, tako da bo pot med mestoma najkrajša ($|AP|+|PQ|+|QB|$ minimalno).

\item V ravnini so dane premice $a$, $b$ in $p$ ter daljica $d$.
Načrtaj vzporednico $q$ premice $p$, ki s premicama $a$ in $b$ določa
daljico, ki je skladna z daljico $d$.

\item Trikotnik $ABE$ je zunaj pravokotnika $ABCD$ načrtan nad stranico $AB$.
Pravokotnici premic $AE$ in $BE$ iz točk $C$ in $D$ se sekata v točki
$P$.  Dokaži, da je $PE\parallel BC$.

\item Točka $M$ leži v notranjosti kvadrata $ABCD$. Dokaži, da obstaja
štirikotnik s pravokotnima diagonalama in s stranicami, ki so skladne z
daljicami $MA$, $MB$, $MC$ in $MD$.

\item Skladni krožnici se sekata v točkah $P$ in $Q$. Premica $l$ je
vzporedna s premico $m$, ki poteka skozi središči dveh krožnic, in $l$ seka
krožnici po vrsti v točkah $A$ in $B$ ter $C$ in $D$. Dokaži, da mera
kota $APC$ ni odvisna od izbire premice $l$.

\item Kaj predstavlja kompozitum $\mathcal{S}_A\circ \mathcal{S}_p$?

\item Naj bo $t$ tangenta očrtane krožnice trikotnika $ABC$ v oglišču $A$.
Dokaži, da velja:
$$\mathcal{G}_{\overrightarrow{CA}} \circ \mathcal{G}_{\overrightarrow{BC}}
\circ \mathcal{G}_{\overrightarrow{AB}} =\mathcal{S}_t .$$

\item Kaj predstavlja kompozitum $\mathcal{S}_A\circ
\mathcal{S}_B\circ \mathcal{S}_{AB}$?

\item Naj bo $ABC$ enakostranični trikotnik. Določi os in vektor
zrcalnega zdrsa, ki je določen s kompozitumom
$\mathcal{S}_{BC}\circ \mathcal{S}_{AB}\circ \mathcal{S}_{CA}$.


\item  V isti ravnini sta dana sedemkotnik $PQRSTUV$ in krožnica $k$.
Načrtaj sedemkotnik $ABCDEFG$, ki je včrtan dani krožnici,
njegove stranice pa so vzporedne s stranicami danega sedemkotnika.


\item Dokaži, da velja:
$\mathcal{S}_A\circ\mathcal{S}_B\circ\mathcal{S}_C=
\mathcal{S}_C\circ\mathcal{S}_B\circ\mathcal{S}_A$.

\item Naj bodo $A$, $B$ in $C$ tri nekolinearne točke. Določi točko
$S$, za katero velja: $$\mathcal{S}_S\circ\mathcal{S}_A\circ
\mathcal{S}_S\circ\mathcal{S}_B\circ
\mathcal{S}_S\circ\mathcal{S}_C=\mathcal{E}.$$

\item Če je $S$ središče daljice $AB$, je
$\mathcal{S}_S\circ\mathcal{S}_A\circ\mathcal{S}_S=\mathcal{S}_B$. Dokaži.

%olimp
%____________________________________________________

\item \footnote{Predlog za MMO 1971. (SL 12.)} Naj bosta $ABC$ in $A'B'C'$ skladna
enakostranična trikotnika. Dokaži, da so središča daljic $AA'$, $BB'$ in $CC'$
bodisi kolinearne točke bodisi oglišča novega enakostraničnega trikotnika.

\item \footnote{Predlog za MMO 1967. (LL 41.)} Premica $l$ poteka skozi višinsko
točko trikotnika $ABC$. Označimo z $l_a$, $l_b$ in $l_c$ zrcalne slike
premice $l$ glede na premice $BC$, $AC$ in $BC$. Dokaži, da se premice
  $l_a$, $l_b$ in $l_c$ sekajo v skupni točki, ki leži na očrtani krožnici trikotnika $ABC$.

\item \footnote{Predlog za MMO 1982. (SL 20.)} Enakostranični trikotniki
 $BAM$, $DCP$, $BCN$ in $DAQ$ so konstruirani nad stranicami konveksnega
 štirikotnika $ABCD$. Prva dva trikotnika sta konstruirana zunaj, druga
 dva pa znotraj tega štirikotnika. Kaj lahko rečemo o štirikotniku $MNPQ$?

 \item Naj bo $\mathcal{R}_{D,90^0}\circ
 \mathcal{R}_{C,90^0}\circ\mathcal{R}_{B,90^0}\circ\mathcal{R}_{A,90^0}
 =\mathcal{E}$. Dokaži, da je $AC\perp BD$ in $AC\cong BD$.


\end{enumerate}







% DEL 7 - - - - - - - - - - - - - - - - - - - - - - - - - - - - - - - - - - - - - - -
%________________________________________________________________________________
% PODOBNOST
%________________________________________________________________________________


  \del{Similarity} \label{pogPOD}


V poglavjih \ref{pogSKL} in \ref{pogSKK} smo se ukvarjali z relacijo skladnosti likov. Za definicijo te relacije smo uporabili izometrije, ki smo jih vpeljali v razdelku \ref{odd2AKSSKL} in še bolj podrobno obravnavali v poglavju \ref{pogIZO}.
Analogno bomo sedaj  vpeljali najprej transformacije podobnosti, da bomo z njimi definirali relacijo podobnosti likov (Figure \ref{sl.pod.7.0.1.pic}).

\begin{figure}[!htb]
\centering
\input{sl.pod.7.0.1.pic}
\caption{} \label{sl.pod.7.0.1.pic}
\end{figure}

Zelo pomemben izrek, ki je na določen način že povezan s pojmom podobnosti, je Talesov izrek \ref{TalesovIzrek}, ki smo ga posebej obravnavali in dokazali v razdelku \ref{odd5TalesVekt}.





%________________________________________________________________________________
\poglavje{Similarity Transformations}
\label{odd7TransfPodob}

Idejo za definicijo transformacij podobnosti dobimo iz pojma izometrij. Intuitivno izometrije predstavljajo gibanja. V razdelku \ref{odd2AKSSKL} smo jih formalno definirali kot bijektivne preslikave, ki ohranjajo relacijo skladnosti parov točk.

 Bijektivna preslikava ravnine v ravnino $f:\hspace*{1mm}\mathbb{E}^2\rightarrow \mathbb{E}^2$ je \index{transformacije podobnosti}\pojem{transformacija podobnosti} s \pojem{koeficientom} \index{koeficient!podobnosti} $k\in \mathbb{R}^+$, če za vsaki dve točki $A,B\in \mathbb{E}^2$ in njuni sliki $A'=f(A)$ in $B'=f(B)$ velja: $A'B':AB=k$ oz.  $|A'B'|=k\cdot |AB|$ (Figure \ref{sl.pod.7.0.2.pic}).

\begin{figure}[!htb]
\centering
\input{sl.pod.7.0.2.pic}
\caption{} \label{sl.pod.7.0.2.pic}
\end{figure}


Jasno je, da so izometrije ravnine poseben primer transformacij podobnosti s koeficientom $k=1$.



            \bizrek \label{TransPodB}
            Similarity transformations preserve the relation $\mathcal{B}$, i.e.
            for every three points $A$, $B$ and $C$ of the plane and their images$A'$, $B'$ and $C'$ it is:
             $$\mathcal{B}(A,B,C)\hspace*{1mm}\Rightarrow\hspace*{1mm}\mathcal{B}(A',B',C').$$
            \eizrek

\begin{figure}[!htb]
\centering
\input{sl.pod.7.0.3.pic}
\caption{} \label{sl.pod.7.0.3.pic}
\end{figure}


\textbf{\textit{Proof.}} Označimo s $f$ transformacijo podobnosti s koeficientom $k$.
Predpostavimo, da za točke $A$, $B$ in $C$ velja $\mathcal{B}(A,B,C)$ (Figure \ref{sl.pod.7.0.3.pic}). To pomeni, da je $AB+BC=AC$.
Po definiciji transformacij podobnosti je potem:
$$A'B' + B'C' = k\cdot AB + k\cdot BC = k\cdot (AB + BC) = k\cdot AC = A'C'.$$
Iz tega pa sledi $\mathcal{B}(A',B',C')$.
\kdokaz

Direktna posledica prejšnjega izreka \ref{TransPodB} je naslednja trditev.


            \bizrek \label{TransPodKol}
            A similarity transformation maps a line to a line, a line segment to a
            line segment, a ray to a ray, a half-plane to a half-plane, an angle to an
            angle and an $n$-gon to an $n$-gon.
            \eizrek

Tudi dokaz naslednje trditve je neposreden.



            \bizrek \label{TransPodOhranjajoRazm}
            Similarity transformations preserve the relation of congruence and the ratio of line segments.
            It means that for every four points $A$, $B$, $C$ and $D$ of the plane and their images $A'$, $B'$, $C'$ and $D'$ it is:
            \begin{itemize}
              \item $AB\cong CD\hspace*{1mm}\Rightarrow\hspace*{1mm}A'B'\cong C'D'$;
              \item $AB:CD=A'B':C'D'$.
            \end{itemize}
            \eizrek

\begin{figure}[!htb]
\centering
\input{sl.pod.7.0.4.pic}
\caption{} \label{sl.pod.7.0.4.pic}
\end{figure}

\textbf{\textit{Proof.}} Označimo s $f$ transformacijo podobnosti s koeficientom $k$, za katero je $f:\hspace*{1mm}A,B,C,D\mapsto A',B',C',D'$ (Figure \ref{sl.pod.7.0.4.pic}). Potem je $A'B'=k\cdot AB$ in $C'D'=k\cdot CD$. Torej:
$$\frac{A'B'}{C'D'}=\frac{k\cdot AB}{k\cdot CD}=\frac{AB}{CD},$$
iz tega pa dobimo obe zahtevani lastnosti.
\kdokaz

A direct consequence is the following theorem.



            \bizrek \label{TransPodKroznKrozn}
            A similarity transformation maps a circle to a circle; the centre, the radius and the diameter
             of the first circle maps to the centre, the radius, and the diameter of
             the second circle (Figure \ref{sl.pod.7.0.5.pic}).
           \eizrek

\begin{figure}[!htb]
\centering
\input{sl.pod.7.0.5.pic}
\caption{} \label{sl.pod.7.0.5.pic}
\end{figure}


Podobno kot pri izometrijah lahko definiramo tudi dve vrsti
transformacij podobnosti.
Za takšne transformacije podobnosti, ki ohranjajo orientacijo ravnine, pravimo da so
\index{transformacije podobnosti!direktna}\pojem{direktne}.
Za tiste
transformacije podobnosti, ki orientacijo ravnine obrnejo, pa pravimo,
da so \index{transformacije podobnosti!indirektna} \pojem{indirektne} (Figure \ref{sl.pod.7.0.1a.pic}).
 Ne bomo formalno
dokazovali dejstva, da je vsaka transformacija podobnosti ravnine bodisi direktna
bodisi indirektna, oz. če neka transformacija podobnosti ohranja orientacijo
enega lika, ohranja tudi orientacijo vseh drugih likov.

\begin{figure}[!htb]
\centering
\input{sl.pod.7.0.1a.pic}
\caption{} \label{sl.pod.7.0.1a.pic}
\end{figure}

 Jasno je, da kompozitum dveh direknih ali dveh indirektnih
 transformacij podobnosti predstavlja direktno transformacijo podobnosti. Prav tako je
 kompozitum ene direktne in ene indirektne transformacije podobnosti indirektna
 transformacija podobnosti.


Množico vseh transformacij podobnosti neke ravnine označimo s $\mathfrak{P}$. Jasno je, da velja $\mathfrak{I}\subset\mathfrak{P}$, kjer je $\mathfrak{I}$  množica vseh izometrij neke ravnine.
V  razdelku \ref{odd2AKSSKL} smo ugotovili, da
 $\mathfrak{I}$ predstavlja
 grupo glede na operacijo kompozituma
preslikav. Za to grupo smo uporabili isto oznako kot za samo množico oz. $\mathfrak{I}=(\mathfrak{I},\circ)$. Lastnosti te grupe smo bolj podrobno obravnavali v razdelku \ref{odd6Grupe}.
Na tem mestu bomo dokazali, da tudi $\mathfrak{P}=(\mathfrak{P},\circ)$ predstavlja strukturo grupe - t. i. \index{grupa!transformacij podobnosti}\pojem{grupa transformacij podobnosti}.



            \bizrek \label{TransPodGrupa}
            The set of all similarity transformations $\mathfrak{P}$ with respect to the composition
            of mappings $\circ$ form a group, i.e.:
            \begin{enumerate}
             \item $(\forall f\in \mathfrak{P})(\forall g\in \mathfrak{P})
            \hspace*{1mm}f\circ g\in \mathfrak{P}$;
            \item $(\forall f\in \mathfrak{P})(\forall g\in \mathfrak{P})
            (\forall h\in \mathfrak{P})
            \hspace*{1mm}(f\circ g)\circ h=f\circ (g\circ h)$;
             \item $(\exists e\in \mathfrak{P})(\forall f\in \mathfrak{P})
            \hspace*{1mm}f\circ e=e\circ f=f$;
            \item $(\forall f\in \mathfrak{P})(\exists g\in \mathfrak{P})
            \hspace*{1mm}f\circ g=g\circ f=e$.
            \end{enumerate}
            \eizrek


\textbf{\textit{Proof.}}  Po vrsti dokažimo vse lastnosti.

(\textit{1}) Naj bosta $f$ in $g$ transformaciji podobnosti s koeficientoma $k_2$ in $k_1$. Označimo z $X$ in $Y$ poljubni točki ravnine ter $g:\hspace*{1mm}X,Y\mapsto X_1,Y_1$ in $f:\hspace*{1mm}X_1,Y_1\mapsto X',Y'$. Potem je $f\circ g:\hspace*{1mm}X,Y\mapsto X',Y'$. Ker je $X_1Y_1=k_1\cdot XY$ in $X'Y'=k_2\cdot X_1Y_1$, je tudi $X'Y'=k_1k_2\cdot XY$, kar pomeni, da je $f\circ g$ transformacija podobnosti s koeficientom $k=k_1\cdot k_2$.

(\textit{2})  Lastnost velja splošno za operacijo kompozituma preslikav.

 (\textit{3}) Identiteta $e=\mathcal{E}$ je transformacija podobnosti s koeficientom $k=1$.

  (\textit{4}) Naj bo $f$  transformacija podobnosti s koeficientom $k$. Ker je $f$ bijektivna transformacija, obstaja njena inverzna preslikava $f^{-1}$. Naj bosta  $X$ in $Y$ poljubni točki ravnine ter $f:\hspace*{1mm}X,Y\mapsto X',Y'$. Potem je  $f^{-1}:\hspace*{1mm}X',Y'\mapsto X,Y$. Iz $X'Y'=k\cdot XY$ pa sledi $XY=\frac{1}{k}\cdot X'Y'$ ($\frac{1}{k}$ obstaja, ker $k\in \mathbb{R}^+$), kar pomeni, da je $f^{-1}$ transformacija podobnosti s koeficientom $\frac{1}{k}$.
 \kdokaz

 Jasno je, da je grupa vseh izometrij ravnine podgrupa grupe vseh transformacij podobnosti te ravnine. To dejstvo smo zapisali na naslednji način: $\mathfrak{I}<\mathfrak{P}$ (glej razdelek \ref{odd6Grupe}). Tudi vse podgrupe grupe  $\mathfrak{I}$ so hkrati podgrupe grupe  $\mathfrak{P}$. Še eno podgrupo dobimo, če vzamemo le direktne transformacije podobnosti $\mathfrak{P}^+<\mathfrak{P}$.




%________________________________________________________________________________
\poglavje{Homothety}
\label{odd7SredRazteg}

V prejšnjem razdelku smo imeli za primere transformacij podobnosti vse izometrije, če za koeficient podobnosti izberemo $k=1$. Sedaj pa bomo definirali novo vrsto transformacij podobnosti.

Naj bo $S$ poljubna točka v ravnini in $k\in \mathbb{R}\setminus \{0\}$. Transformacijo te ravnine, s katero se
poljubna točka $X$ te ravnine preslika v takšno točko $X'$, da velja
 $$\overrightarrow{SX'} = k\cdot \overrightarrow{SX},$$
 imenujemo \index{središčni razteg}\pojem{središčni razteg} ali \index{homotetija}\pojem{homotetija} $h_{S,k}$ s \index{središče!središčnega raztega}\pojem{središčem} $S$ in \index{koeficient!središčnega raztega}\pojem{koeficientom} $k$. Vsi poltraki iz točke $S$ so \index{žarek središčnega raztega}\pojem{žarki} središčnega raztega. Lika sta \index{lika!homotetična}\pojem{homotetična}, če
obstaja središčni razteg, ki en lik preslika v drugega (Figure \ref{sl.pod.7.1.1.pic}).


\begin{figure}[!htb]
\centering
\input{sl.pod.7.1.1.pic}
\caption{} \label{sl.pod.7.1.1.pic}
\end{figure}

 Iz definicije sledi, da je središčni razteg bijektivna transformacija ravnine, ki je
enolično določena s svojim središčem in koeficientom.

Iz same definicije je jasno tudi, da v primerih $k=1$ ali $k=-1$ dobimo identiteto oz. središčno zrcaljenje. Drugače zapisano: $h_{S,1}=\mathcal{E}$ oz. $h_{S,-1}=\mathcal{S}_S$ (Figure \ref{sl.pod.7.1.2.pic}).


\begin{figure}[!htb]
\centering
\input{sl.pod.7.1.2.pic}
\caption{} \label{sl.pod.7.1.2.pic}
\end{figure}



            \bizrek \label{raztFiksTock}
            The only fixed point of a homothety $h_{S,k}$ (for $k\neq 1$) is its centre is $S$.
            \eizrek


\textbf{\textit{Proof.}} Dokažimo najprej $h_{S,k}(S)=S$. Naj bo $h_{S,k}(S)=S'$. Po definiciji središčnega raztega je $\overrightarrow{SS'}=k\cdot \overrightarrow{SS}=k\cdot \overrightarrow{0}=\overrightarrow{0}$, iz tega pa sledi $S'=S$.

Predpostavimo, da za neko točko $X\neq S$ velja $h_{S,k}(X)=X$. V tem primeru je $\overrightarrow{SX}=k\cdot \overrightarrow{SX}$. Ker velja $\overrightarrow{SX}\neq \overrightarrow{0}$, sledi $k=\frac{\overrightarrow{SX}}{\overrightarrow{SX}}=1$, kar po predpostavki ni možno. Torej je $S$ edina fiksna točka središčnega raztega $h_{S,k}$.
\kdokaz

Napovedali smo že novo vrsto transformacij podobnosti. V zvezi s tem bomo najprej dokazali pomožni izrek.



            \bizrek \label{RaztTales}
            Suppose that $h_{S,k}:\hspace*{1mm}X,Y\mapsto X',Y'$. Then:
            $$\overrightarrow{X'Y'}=k\cdot \overrightarrow{XY}.$$
            \eizrek

\begin{figure}[!htb]
\centering
\input{sl.pod.7.1.3.pic}
\caption{} \label{sl.pod.7.1.3.pic}
\end{figure}

\textbf{\textit{Proof.}} Iz $X'=h_{S,k}(X)$ in $Y'=h_{S,k}(Y)$ po definiciji središčnega raztega sledi $\overrightarrow{SX'} = k\cdot \overrightarrow{SX}$ in $\overrightarrow{SY'} = k\cdot \overrightarrow{SY}$ (Figure \ref{sl.pod.7.1.3.pic}). Po izrekih \ref{vektOdsev} in \ref{vektVektorskiProstor} je:

$$\overrightarrow{X'Y'}=
\overrightarrow{SY'}-\overrightarrow{SX'}
=k\cdot \overrightarrow{SY}-k\cdot \overrightarrow{SX}=
k\cdot\left(\overrightarrow{SY}-\overrightarrow{SX} \right)
k\cdot \overrightarrow{XY},$$ kar je bilo treba dokazati. \kdokaz

            \bizrek \label{RaztTransPod}
            A homothety $h_{S,k}$ is a similarity transformation with the coefficient $|k|$.
            \eizrek


\textbf{\textit{Proof.}} Naj bo $X'=h_{S,k}(X)$ in $Y'=h_{S,k}(Y)$ za poljubni točki $X$ in $Y$. Po prejšnjem izreku \ref{RaztTales} je $\overrightarrow{X'Y'}=k\cdot \overrightarrow{XY}$ oz. $\frac{\overrightarrow{X'Y'}}{\overrightarrow{XY}}=k$. Zato je $\frac{X'Y'}{XY}=\frac{|\overrightarrow{X'Y'}|}{|\overrightarrow{XY}|}=|k|$, kar pomeni, da je $h_{S,k}$ je transformacija podobnosti s koeficientom $|k|$.
 \kdokaz


Središčni razteg ima torej vse splošne lastnosti, ki smo jih že dokazali za transformacije podobnosti. Posebej bomo poudarili naslednji trditvi.

            \bizrek \label{RaztKol}
            A homothety maps a line to a line, a line segment to a
            line segment, a ray to a ray, a half-plane to a half-plane, an angle to an
            angle and an $n$-gon to an $n$-gon.
            \eizrek

\textbf{\textit{Proof.}}
 Trditev je direktna posledica izrekov \ref{RaztTransPod} in \ref{TransPodKol}.
 \kdokaz

            \bzgled \label{RaztKroznKrozn}
            A homothety
            maps a circle to a circle; the centre, the radius and the diameter
             of the first circle maps to the centre, the radius, and the diameter of
             the second circle.
            \ezgled


\textbf{\textit{Proof.}}
 Trditev je direktna posledica izrekov \ref{RaztTransPod} in \ref{TransPodKroznKrozn}.
 \kdokaz

 Omenimo še dve posledici izreka \ref{RaztTales}.



                \bizrek  \label{RaztPremica}
                A homothety maps each line to its parallel line.
                The only lines that map to itself under a  homothety $h_{S,k}$ ($k\neq 1$), are those that contain the centre $S$.
                \eizrek

\begin{figure}[!htb]
\centering
\input{sl.pod.7.1.4.pic}
\caption{} \label{sl.pod.7.1.4.pic}
\end{figure}

\textbf{\textit{Proof.}} Naj bo $XY$ poljubna premica. Po izreku \ref{RaztKol} je njena slika s središčnim raztegom $h_{S,k}$ premica $X'Y'$, kjer je $X'=h_{S,k}(X)$ in $Y'=h_{S,k}(Y)$ (Figure \ref{sl.pod.7.1.4.pic}). Iz izreka \ref{RaztTales} pa sledi $\overrightarrow{X'Y'}=k\cdot \overrightarrow{XY}$, kar pomeni, da sta vektorja $\overrightarrow{X'Y'}$ in $\overrightarrow{XY}$ kolinearna (\ref{vektKriterijKolin}) oz. premici $X'Y'\parallel XY$.

Če premica $p$ poteka skozi središče $S$ središčnega raztega $h_{S,k}$, je po izreku \ref{raztFiksTock} $S\in p'=h_{S,k}(p)$. Toda iz dokazanega dela te trditve je $p'\parallel p$. Po posledici Playfairjevega aksioma \ref{Playfair1} je $p'=p$.

Predpostavimo, da za neko premico $p$ velja $p=p'=h_{S,k}(p)$. Naj bo $X\in p$ poljubna točka premice $p$ in $X'=h_{S,k}(X)$.
Jasno je potem tudi $X'\in p'=p$. Če je $X=X'$, je po izreku \ref{raztFiksTock} $S=X\in p$. Če pa velja $X'\neq X$, je $S\in XX'=p$.
 \kdokaz

                \bizrek \label{homotOhranjaKote}
                A homothety maps an angle to the congruent angle.
                \eizrek

\begin{figure}[!htb]
\centering
\input{sl.pod.7.1.5.pic}
\caption{} \label{sl.pod.7.1.5.pic}
\end{figure}

\textbf{\textit{Proof.}} (Figure \ref{sl.pod.7.1.5.pic})

 Trditev je direktna posledica izrekov \ref{RaztKol}, \ref{RaztPremica} in \ref{KotaVzporKraki}.
 \kdokaz

 Trditev iz prejšnjega izreka lahko zapišemo tudi v naslednji obliki:
 Središčni razteg ohranja kote (njihovo mero). V tem smislu se bosta dve krivulji sekali pod enakim kotom kot njuni sliki pri središčnem raztegu. Preslikave, ki imajo to lastnost, so t. i. \index{konformna preslikava}\pojem{konformne preslikave}.

 Dokazali smo, da središčni razteg preslika krožnico v krožnico. Velja tudi obratna trditev.



            \bizrek \label{RaztKroznKrozn1}
            For any two circles of a plane, there is a homothety,
            which maps one circle to another.
            \eizrek


\begin{figure}[!htb]
\centering
\input{sl.pod.7.1.12.pic}
\caption{} \label{sl.pod.7.1.12.pic}
\end{figure}



\textbf{\textit{Proof.}} Naj bosta $k_1(S_1,r_1)$ in $k_2(S_2,r_2)$ poljubni krožnici v isti ravnini (Figure \ref{sl.pod.7.1.12.pic}).

Po izreku \ref{RaztKroznKrozn} je dovolj izbrati središčni razteg $h_{S,k}$, ki preslika točko $S_1$ v točko $S_2$ in polmer $r_1$ v polmer $r_2$. Ker je $h_{S,k}$ transformacija podobnosti s koeficientom $|k|$, iz zadnjega pogoja sledi $r_2=|k|\cdot r_1$ oz. $|k|=\frac{r_2}{r_1}$. V primeru $r_1\neq r_2$ oz. $|k|\neq 1$ za  $k$ lahko izberemo kar negativno vrednost $k=-\frac{r_2}{r_1}$. Iz $h_{S,k}(S_1)=S_2$ pa dobimo $\overrightarrow{SS_2}=k\overrightarrow{SS_1}$, oziroma: $$\frac{\overrightarrow{S_2S}}{\overrightarrow{SS_1}}=
-\frac{\overrightarrow{SS_2}}{\overrightarrow{SS_1}}=-k=\frac{r_2}{r_1}.$$
Po izreku \ref{izrekEnaDelitevDaljice} obstaja ena takšna točka $S$.

V primeru, ko je $r_1=r_2$, lahko izberemo središčno zrcaljenje $\mathcal{S}_S$, ki preslika točko $S_1$ v točko $S_2$ ($S$ je središče daljice $S_1S_2$), saj je središčno zrcaljenej hkrati središčni razteg oz. $\mathcal{S}_S=h_{S,-1}$.
 \kdokaz

 V dokazu prejšnjega izreka smo v primeru $r_1\neq r_2$ oz. $|k|\neq 1$ za $k$ uporabili negativno vrednost $k=-\frac{r_2}{r_1}$, posledično je bilo razmerje $\frac{\overrightarrow{S_2S}}{\overrightarrow{SS_1}}$ pozitivno. Točka $S$  v tem primeru leži na daljici $S_1S_2$ - gre za t. i. \pojem{notranjo delitev daljice} $S_1S_2$ v razmerju  $\frac{r_2}{r_1}$. Če bi za $k$ vzeli pozitivno vrednost $k=\frac{r_2}{r_1}$, bi dobili $\frac{\overrightarrow{S_2S}}{\overrightarrow{SS_1}}=-\frac{r_2}{r_1}$ oz. še eno rešitev za točko $S$ in središčni razteg (Figure \ref{sl.pod.7.1.12.pic}). V tem primeru govorimo o t. i. \pojem{zunanji delitvi daljice} - več o tem bomo povedali v razdelku \ref{odd7Harm}.


Naslednji izrek je intuitivno jasen, zato ga bomo podali brez formalnega dokaza (Figure \ref{sl.pod.7.1.6.pic}).


            \bizrek \label{homotDirekt}
            A homothety $h_{S,k}$ is a direct similarity transformation.
            \eizrek

\begin{figure}[!htb]
\centering
\input{sl.pod.7.1.6.pic}
\caption{} \label{sl.pod.7.1.6.pic}
\end{figure}


                \bizrek \label{homotGrupa}
                The set of all homotheties with the same centre $S$  with respect to the composition
            of mappings $\circ$ form a commutative group. Furthermore:
                \begin{itemize}
                  \item $h_{S,k_2}\circ h_{S,k_1}=h_{S,k_1\cdot k_2}$;
                  \item $h^{-1}_{S,k}=h_{S,\frac{1}{k}}$.
                \end{itemize}
                \eizrek

\textbf{\textit{Proof.}} Dokažimo da so izpolnjene vse lastnosti strukture komutativne grupe.

(\textit{1}) Naj bosta $h_{S,k_1}$ in $h_{S,k_2}$ središčna raztega neke ravnine z istim središčem $S$ in $X$ poljubna točka te ravnine. Dokažimo, da je tudi njun kompozitum središčni razteg.
Označimo $h_{S,k_1}(X)=X_1$ in $h_{S,k_2}(X_1)=X'$. V tem primeru je $h_{S,k_2}\circ h_{S,k_1}(X)=X'$. Toda iz $\overrightarrow{SX_1}=k_1\cdot \overrightarrow{SX}$ in $\overrightarrow{SX'}=k_2\cdot \overrightarrow{SX_1}$ sledi $\overrightarrow{SX'}=k_1k_2\cdot \overrightarrow{SX}$ oz. $X'=h_{S,k_1\cdot k_2}(X)$.
Ker to velja za vsako točko $X$, je: $$h_{S,k_2}\circ h_{S,k_1}=h_{S,k_1\cdot k_2}.$$


(\textit{2})  Asociativnost velja splošno za operacijo kompozituma preslikav.

 (\textit{3}) Identiteta $e=\mathcal{E}$ je središčni razteg s koeficientom $k=1$ oz. $\mathcal{E}=h_{S,1}$.

 (\textit{4}) Naj bo $h_{S,k}$ poljubni središčni razteg s središčem $S$. Označimo $h^{-1}_{S,k}=h_{S,\frac{1}{k}}$. Iz dokazanega v (\textit{1}) dobimo:
  $$h_{S,k}\circ h_{S,\frac{1}{k}}=h_{S,k\cdot \frac{1}{k}}=h_{S,1}=\mathcal{E}.$$

(\textit{5}) Dokažimo še, da velja komutativnost, torej da za poljubna središčna raztega $h_{S,k_1}$ in $h_{S,k_2}$ velja
 $h_{S,k_2}\circ h_{S,k_1}=h_{S,k_1}\circ h_{S,k_2}$. Toda iz dokazanega v (\textit{1}) in komutativnosti množenja realnih števil dobimo:
 $$h_{S,k_2}\circ h_{S,k_1}=h_{S,k_1\cdot k_2}
 =h_{S,k_2\cdot k_1}=h_{S,k_1}\circ h_{S,k_2},$$ kar je bilo treba dokazati. \kdokaz

Grupo iz prejšnjega izreka imenujemo \index{grupa!homotetij}\pojem{grupa homotetij} in jo označimo z $\mathfrak{H}_S$. Ta grupa je podgrupa grupe
transformacij podobnosti $\mathfrak{P}$ (izrek \ref{RaztTransPod}) in je pravzaprav izomorfna z grupo $(\mathbb{R}\setminus \{0\},\cdot)$.

 Naslednja primera sta povezana z direktno konstrukcijo slike točke oz. lika pri danem središčnem raztegu.



            \bzgled
           Let $S$ and $A$ be two distinct points in the plane. Construct
            the point $A'=h_{S,k}(A)$ if:

            (i) $k=2$, \hspace*{4mm}   (ii) $k=\frac{1}{3}$, \hspace*{4mm}
               (iii) $k=-3$, \hspace*{4mm}    (iv) $k=-\frac{2}{5}$.
            \ezgled

\begin{figure}[!htb]
\centering
\input{sl.pod.7.1.8.pic}
\caption{} \label{sl.pod.7.1.8.pic}
\end{figure}

\textbf{\textit{Solution.}} (Figure \ref{sl.pod.7.1.8.pic})

Primera (\textit{i}) in (\textit{iii}) sta enostavna. V primerih (\textit{ii}) in (\textit{iv}) pa uporabimo posledico
 Talesovega izreka \ref{izrekEnaDelitevDaljiceNan}.
 \kdokaz


            \bzgled
            Construct the image of a pentagon $ABCDE$ under a homothety $h_{S,k}$, if $k=1,5$.
            \ezgled

\begin{figure}[!htb]
\centering
\input{sl.pod.7.1.7.pic}
\caption{} \label{sl.pod.7.1.7.pic}
\end{figure}

\textbf{\textit{Solution.}} (Figure \ref{sl.pod.7.1.7.pic})

Najprej iz pogoja $\overrightarrow{SX'}=1,5\cdot \overrightarrow{SX}$ načrtamo $A'=h_{S,k}$ na isti način kot v prejšnjem primeru. Nato pa
z uporabo izreka \ref{RaztPremica} načrtamo slike preostalih oglišč. Npr. $B'=SB\cap l$, kjer je $l$ vzporednica premice $AB$
skozi točko $A'$.
 \kdokaz

 V naslednjih dveh primerih bomo videli uporabo središčnega raztega pri različnih konstrukcijah.


                \bzgled
                Let $A$ be one of the two intersections of a circles $k$ and $l$.
               Construct a line $s$ through the point $A$ that intersects the circles $k$ and $l$
               and defines chords that are in the ratio $3:2$.
                \ezgled

\begin{figure}[!htb]
\centering
\input{sl.pod.7.1.9.pic}
\caption{} \label{sl.pod.7.1.9.pic}
\end{figure}

\textbf{\textit{Solution.}} (Figure \ref{sl.pod.7.1.9.pic})

Naj bosta $X$ in $Y$ presečišči iskane premice $s$ s
krožnicama $k$ in $l$, tako da velja $XA:AY=3:2$. Tedaj je
$Y=h_{A,-\frac{2}{3}}$. Torej lahko točko $Y$
načrtamo kot drugo presečišče krožnic $l$ in $k'=h_{A,-\frac{2}{3}}(k)$. Premica $s$ je potem določena s točkama $A$ in $Y$,
točka $X$ pa je drugo presečišče premice $s$ s krožnico $k$.
 \kdokaz



                \bzgled \label{sredRaztegZgledKvadrat}
                Construct a square $PQRS$ into the given acute triangle $ABC$,
                so that its side $PQ$ lies on the side $BC$ of the triangle and the vertices $R$ and
                $S$ lie on the pages $AB$ and $AC$.
                \ezgled


\begin{figure}[!htb]
\centering
\input{sl.pod.7.1.10.pic}
\caption{} \label{sl.pod.7.1.10.pic}
\end{figure}

\textbf{\textit{Solution.}} (Figure \ref{sl.pod.7.1.10.pic})

 Če ‘‘pozabimo’’ na pogoj, da oglišče $R$ leži na stranici $AC$, je takšnih
kvadratov neskončno mnogo. Enega $P_1Q_1R_1S_1$  lahko načrtamo. Izberimo za oglišče $S_1$
poljubno točko stranice $AB$, potem oglišče $P_1$ dobimo kot pravokotno projekcijo točke $S_1$ na
 stranico $BC$ in na koncu $Q_1$ in $R_1$ kot oglišča kvadrata. Če je $PQRS$ iskani kvadrat, sta
kvadrata $PQRS$ in $P_1Q_1R_1S_1$ homotetična s središčem središčnega raztega (homotetije) v točki $B$. Res, po
Talesovem izreku \ref{TalesovIzrek} je najprej:
$$\frac{\overrightarrow{BP_1}}{\overrightarrow{BP}}=
\frac{\overrightarrow{BS_1}}{\overrightarrow{BS}}$$
 oz. za nek $k\in \mathbb{R}\setminus \{0\}$ velja:
 $$\overrightarrow{BP_1}=k\cdot\overrightarrow{BP}\hspace*{3mm}\textrm{ in }
\hspace*{3mm}\overrightarrow{BS_1}=k\cdot\overrightarrow{BS},$$
kar pomeni, da središčni razteg $h_{B,k}$ preslika točki $P$ in $Q$ v točki $P_1$ in $Q_1$, prav tako kvadrat $PQRS$ v kvadrat $P_1Q_1R_1S_1$ (ker se kvadrat s središčnim raztegom preslika v kvadrat, kar ni težko dokazati).

Torej lahko točko $R$ dobimo kot presek poltraka $BR_1$ s stranico $AC$ (na žarku središčnega raztega), nato pa ostala oglišča kvadrata $PQRS$
kot ustrezne pravokotne projekcije.
Naloga ima vedno le eno rešitev.
\kdokaz



            \bzgled
            Let $Pp$ and $Qq$ be rays in the plane. Construct points $A\in Pp$ and $B\in Qq$ such that
            $$PA:AB:BQ=1:2:1.$$
            \ezgled

\begin{figure}[!htb]
\centering
\input{sl.pod.7.1.11.pic}
\caption{} \label{sl.pod.7.1.11.pic}
\end{figure}

\textbf{\textit{Solution.}}
Z $A'$ in $B''$ označimo poljubni točki poltrakov $Pp$ in $Qq$,
 da velja $PA'\cong QB''\cong x$, kjer je $x$ poljubna daljica (Figure \ref{sl.pod.7.1.11.pic}). Naj
bo $B'$ eno od presečišč krožnice $k(A',2x)$ in vzporednice premice $PQ$, ki
gre skozi točko $B''$, ter $Q'$
četrto oglišče paralelograma $QB''B'Q'$. Točka $Q'$ torej
leži na premici $PQ$ in velja $Q'B'\cong QB''\cong x$ in $A'B'\cong 2x$. Naj bo $h_{P,k}$ središčni razteg s središčem $P$ in koeficientom $k=\frac{\overrightarrow{PQ}}{\overrightarrow{PQ'}}$.
Potem je $h_{P,k}(Q')=Q$. Če označimo $h_{P,k}(A')=A$ in $h_{P,k}(B')=B$, točki $A$ in $B$ ležita na poltrakih $Pp$ in $Qq$ ($B\in Qq$ po izreku \ref{RaztPremica}, ker je $Q'B'\parallel Qq$). Po izreku \ref{RaztPremica} je še $AB\parallel A'B'$.
Po Talesovem izreku \ref{TalesovIzrekDolzine} je:
$\frac{PA}{PA'}=\frac{AB}{A'B'}=\frac{PB}{PB'}=\frac{BQ}{B'Q'}$,
zato je tudi $PA:AB:BQ= PA':A'B':B'Q'=x:2x:x=1:2:1$.
\kdokaz



Na tem mestu bomo razširili trditev iz izreka \ref{EulerKrozPrem1}. Tudi že dokazani del trditve bomo še enkrat dokazali na drugačen način - s pomočjo središčnega raztega.


            \bizrek \label{EulerKroznicaHomot} \index{krožnica!Eulerjeva}
           The centre of the Euler circle of a triangle lies on its Euler line.
        It is the midpoint of the segment joining the orthocentre and the circumcentre of that triangle.
         The radius of the Euler circle is half of the radius of the circumcircle.
            \eizrek

\begin{figure}[!htb]
\centering
\input{sl.pod.7.1.0e.pic}
\caption{} \label{sl.pod.7.1.0e.pic}
\end{figure}

\textbf{\textit{Proof.}}
Naj bodo $AA'$, $BB'$ in $CC'$ višine ter $A_1$, $B_1$ in $C_1$ središča stranic $BC$, $AC$ in $AB$ trikotnika $ABC$. Označimo z $O$ središče očrtane krožnice $k$, z $V$ višinsko točko tega trikotnika ter z $V_A$, $V_B$ in $V_C$ središča daljic $VA$, $VB$ in $VC$ (Figure \ref{sl.pod.7.1.0e.pic}).

 Po izreku \ref{EulerKroznica} ležijo točke $A'$, $B'$, $C'$, $A_1$, $B_1$, $C_1$, $V_A$, $V_B$ in $V_C$ na eni krožnici $e$ - t. i. Eulerjevi krožnici. Središče te krožnice označimo z $E$.

 Naj bosta $V_a$ oz. $V_{A_1}$ točki, ki sta simetrični višinski točki $V$ glede na nosilko $BC$ oz. točko $A_1$. Na podoben način definiramo tudi točke $V_b$, $V_{B_1}$, $V_c$ in $V_{C_1}$. Po definiciji središčnega raztega je:
 $$\hspace*{-1.8mm} h_{V,\frac{1}{2}}:\hspace*{1mm} A, V_a, V_{A_1}, B, V_b, V_{B_1}, C, V_c, V_{C_1}
 \mapsto V_A, A', A_1, V_B, B', B_1, V_C, C', C_1.$$

 Naj bo še $h_{V,\frac{1}{2}}(O)=\widehat{E}$. Točka $\widehat{E}$ je torej središče daljice $VO$.

Po izrekih \ref{TockaV'} in \ref{TockaV1} ležijo točke $V_a$, $V_{A_1}$, $V_b$, $V_{B_1}$, $V_c$ in $V_{C_1}$ na krožnici $k$. To velja tudi za točke $A$, $B$ in $C$. Slike teh točk pri središčnem raztegu $h_{V,\frac{1}{2}}$ ležijo na sliki $k'=h_{V,\frac{1}{2}}(k)$ krožnice $k$ oz. iz $A$, $V_a$, $V_{A_1}$, $B$, $V_b$, $V_{B_1}$, $C$, $V_c$, $V_{C_1}$ $\in k$ sledi $V_A$, $A'$, $A_1$, $V_B$, $B'$, $B_1$, $V_C$, $C'$, $C_1$ $\in k'=h_{V,\frac{1}{2}}(k)$. Ker je tudi $V_A, A', A_1, V_B, B', B_1, V_C, C', C_1\in e$, je $k'=e$ in $E=\widehat{E}$ (zgled \ref{RaztKroznKrozn}). Iz tega sledi, da je polmer krožnice $e$ enak polovici polmera krožnice $k$, njeno središče pa hkrati središče daljice $VO$, ki je nosilka Eulerjeve premice (razdelek \ref{odd5EulPrem}).
 \kdokaz



            \bzgled \label{SimsEuler} \index{premica!Simsonova}
            Let $PQ$ be an arbitrary diameter of the circumcircle of a triangle
            $ABC$ and $p$ and $q$ Simson lines at the points $P$ and $Q$. Prove that $p$ and $q$ are
            perpendicular lines intersecting on the Euler circle of this triangle.
            \ezgled

\begin{figure}[!htb]
\centering
\input{sl.skk.4.7.1e.pic}
\caption{} \label{sl.skk.4.7.1e.pic}
\end{figure}

 \textbf{\textit{Proof.}}  (Figure \ref{sl.skk.4.7.1e.pic}).
Premici $p$ in $q$ sta pravokotni, kar je posledica trditve iz
zgleda \ref{SimsZgled2}. Označimo z $L$ presečišče premic $p$ in
$q$. Naj bo $k(O,R)$ očrtana krožnica trikotnika $ABC$, $V$ višinska
točka tega trikotnika ter $P_1$ in $Q_1$ središči daljic $VP$ in
$VQ$. Točki $P_1$ in $Q_1$ ležita po vrsti na premicah $p$ in $q$
(zgled \ref{SimsZgled3}). Potrebno je še dokazati, da točka $L$ leži
na Eulerjevi krožnici trikotnika $ABC$.

Po izreku \ref{EulerKroznicaHomot} središčni razteg
$h_{V,\frac{1}{2}}$ preslika očrtano krožnico $k$ trikotnika $ABC$ v
Eulerjevo krožnico $e(E,\frac{R}{2})$ tega trikotnika; pri tem je
tudi $h_{V,\frac{1}{2}}(O)=E$.
 Toda isti razteg preslika točke $P$ in $Q$ v točke $P_1$ in $Q_1$,
 oz. premer $PQ$ krožnice $k$ v premer $P_1Q_1$ krožnice $e$
 (zgled \ref{RaztKroznKrozn}).
Ker je še $\angle P_1LQ_1=90^0$, po izreku
\ref{TalesovIzrKroz2} točka $L$ leži na Eulerjevi krožnici $e$.
 \kdokaz



            \bnaloga\footnote{3. IMO Hungary - 1961, Problem 5.}
            Construct triangle $ABC$ if $AC = b$, $AB = c$ and $\angle AMB =\omega$, where $M$ is
            the midpoint of segment $BC$ and $\omega<90^0$.
            \enaloga

\begin{figure}[!htb]
\centering
\input{sl.pod.7.1.IMO1.pic}
\caption{} \label{sl.pod.7.1.IMO1.pic}
\end{figure}

\textbf{\textit{Solution.}} Naj bo $ABC$ takšen trikotnik, da
izpolnjuje pogoje  $AC = b$, $AB = c$ in $\angle AMB =\omega$,
kjer je $M$ središče daljice $BC$ in $\omega<90^0$ (Figure
\ref{sl.pod.7.1.IMO1.pic}). Iz neenakosti $\omega<90^0$  sledi $AC>AB$,
oz. $b>c$. Ker je $\angle AMB =\omega$, po izreku
\ref{ObodKotGMT} točka $M$ leži na loku $l$ s tetivo $AB$ in obodnim kotom
$\omega$. Razteg $h_{B,2}$  preslika točke $B$ in $M$ po vrsti v
točke $B$ in $C$, lok $l$ pa v lok $l'$. Iz $M\in l$ sledi $C\in
l'$. Ker je še $AC=b$, sledi da točka $C$ leži tudi na
krožnici $k(A,b)$. Torej $C\in l'\cap k(A,b)$. Dokazana dejstva
omogočajo konstrukcijo.

 Načrtajmo najprej daljico $AB=c$ ter lok $l$ s to tetivo in obodnim kotom
$\omega$, nato pa lok $l'=h_{B,2}(l)$ ter na koncu točko $C$ kot
eno od presečišč loka $l'$ s krožnico $k(A,b)$.

Dokažimo, da konstruirani trikotnik $ABC$ izpolnjuje pogoje iz
naloge. Po konstrukciji je takoj $AB=c$. Iz $C\in k(A,b)$ sledi
$AC=b$. Naj bo $M=h^{-1}_{B,2}(C)$. Ker je $C\in l'=h_{B,2}(l)$, je
$M \in l$. Ker je po konstrukciji $l$ lok s to tetivo in obodnim
kotom $\omega$, je $\angle AMB=\omega$. Potrebno je še dokazati,
da je točka $M$ središče daljice $BC$, kar sledi direktno iz
relacije $M=h^{-1}_{B,2}(C)$.

Poiščimo pogoje za število rešitev naloge. Omenili smo že, da
mora zaradi pogoja $\omega<90^0$ biti $b>c$. V primeru $b\leq c$
 rešitve ni. Število rešitev naloge je naprej odvisno od
 števila presečišč loka $l'$ s krožnico $k(A,b)$.
 \kdokaz





%________________________________________________________________________________
 \poglavje{Classification of Similarity Transformations} \label{odd7PrezentTransPod}

V razdelku \ref{odd6KlasifIzo} smo klasificirali izometrije, tu bomo pa na podoben način naredili klasifikacijo transformacij podobnosti. Tudi pri tej klasifikaciji
bo pomembno število fiksnih točk in dejstvo, ali je transformacija podobnosti direktna ali indirektna.

V prejšnjih dveh razdelkih smo ugotovili, da vse izometrije in središčni raztegi predstavljajo transformacije podobnosti. Prav tako je njun kompozitum transformacija podobnosti (izrek \ref{TransPodGrupa}). Sedaj bomo dokazali, da velja tudi obratno.



                \bizrek \label{TransPodKompHomIzo}
                Each similarity transformations $f$ with coefficient $k$ can be expressed as the product
                 of one isometry and one homothety with an arbitrary centre:
                $$f=h_{S,k}\circ\mathcal{I}_1=\mathcal{I}_2\circ h_{S,k}.$$
                \eizrek

\textbf{\textit{Proof.}}

Naj bo $f$ poljubna transformacija podobnosti s koeficientom $k$.
Če s $h_{S,\frac{1}{k}}$ označimo središčni razteg s poljubnim središčem $S$ in
koeficientom $\frac{1}{k}$, potem kompozitum $h_{S,\frac{1}{k}}\circ f$ predstavlja
transformacijo podobnosti s koeficientom $k\cdot \frac{1}{k}=1$ (izrek \ref{TransPodGrupa}) oz. izometrijo. Torej $h_{S,\frac{1}{k}}\circ f=\mathcal{I}_1$, kjer je $\mathcal{I}_1$ neka izometrija. Po izreku \ref{homotGrupa} je $f=h_{S,\frac{1}{k}}^{-1}\circ\mathcal{I}_1=h_{S,k}\circ\mathcal{I}_1$.
Prav tako je $f\circ h_{S,\frac{1}{k}}=\mathcal{I}_2$, kjer je $\mathcal{I}_2$ neka izometrija, zato je $f=\mathcal{I}_2\circ h_{S,k}$.
 \kdokaz


                \bizrek \label{TransPodOhranjaKote}
                Similarity transformations preserve the measure of angles, i.e. there map an angle to the congruent angle.
                \eizrek


\textbf{\textit{Proof.}}
 Trditev je direktna posledica izrekov \ref{TransPodKompHomIzo} in \ref{homotOhranjaKote}.
\kdokaz

            \bizrek \label{homotTransm}
            For each isometry  $\mathcal{I}$ and each homothety $h_{S,k}$
             it is:
            $$\mathcal{I}\circ h_{S,k}\circ \mathcal{I}^{-1}=h_{\mathcal{I}(S),k}$$
            \eizrek


\begin{figure}[!htb]
\centering
\input{sl.pod.7.1p.2.pic}
\caption{} \label{sl.pod.7.1p.2.pic}
\end{figure}

\textbf{\textit{Proof.}}  (Figure
\ref{sl.pod.7.1p.2.pic})

Naj bo $\mathcal{I}(S)=S_1$. Označimo z $f=\mathcal{I}\circ h_{S,k}\circ \mathcal{I}^{-1}$. Potrebno je dokazati, da je $f=h_{S_1,k}$ oz. $f(X_1)=h_{S_1,k}(X_1)$ za poljubno točko $X_1$ te ravnine.


Označimo še $X=\mathcal{I}^{-1}(X_1)$, $X'=h_{S,k}(X)$ in
$X'_1=\mathcal{I}(X_1)$. Potem je:
\begin{eqnarray*}
f(X_1)&=& \mathcal{I}\circ h_{S,k}\circ \mathcal{I}^{-1}(X_1)\\
 &=& \mathcal{I}\circ h_{S,k}(X)\\
 &=& \mathcal{I}(X')\\
 &=& X'_1
\end{eqnarray*}

 Torej $f(X_1)=X'_1$. Iz $X'=h_{S,k}(X)$ sledi $\overrightarrow{SX'}=k\cdot\overrightarrow{SX}$. Ker je $\mathcal{I}$ izometrija in $\mathcal{I}:\hspace*{1mm}S,X,X'\mapsto S_1,X_1,X'_1$, je tudi
$\overrightarrow{S_1X'_1}=k\cdot\overrightarrow{S_1X_1}$ oz. $h_{S_1,k}(X_1)=X'_1$.
To pomeni, da za poljubno točko $X_1$ velja $f(X_1)=X'_1=h_{S_1,k}(X_1)$, zato je
$f=h_{S_1,k}$.
\kdokaz



            \bizrek \label{homotIzomKom}
             An Isometry $\mathcal{I}$ and a homothety $h_{S,k}$ commute
             if and only if
             the centre of this homothety is a fixed point of the isometry $\mathcal{I}$, i.e.:
            $$\mathcal{I}\circ h_{S,k}=h_{S,k}\circ\mathcal{I}\hspace*{1mm}
            \Leftrightarrow\hspace*{1mm}\mathcal{I}(S)=S.$$
            \eizrek


\textbf{\textit{Proof.}}
 Po prejšnjem izreku (\ref{homotTransm}) je:
 \begin{eqnarray*}
\mathcal{I}\circ h_{S,k}=h_{S,k}\circ\mathcal{I}
\hspace*{1mm}
            &\Leftrightarrow& \hspace*{1mm} \mathcal{I}\circ h_{S,k}\circ\mathcal{I}^{-1}=h_{S,k}\\
\hspace*{1mm}
            &\Leftrightarrow& \hspace*{1mm} h_{\mathcal{I}(S),k}=h_{S,k}\\
\hspace*{1mm}
            &\Leftrightarrow& \hspace*{1mm} \mathcal{I}(S)=S,
\end{eqnarray*}
 kar je bilo treba dokazati. \kdokaz

Če za izometrijo iz izreka \ref{TransPodKompHomIzo} izberemo rotacijo z istim središčem kot središčni razteg, dobimo zelo koristno vrsto transformacij podobnosti.

Kompozitum rotacije in središčnega raztega z istim središčem imenujemo \index{rotacijski razteg} \pojem{rotacijski razteg} (Figure
\ref{sl.pod.7.1p.1.pic}):
$$\rho_{S,k,\omega}=h_{S,k}\circ \mathcal{R}_{S,\omega}$$
s \index{središče!rotacijskega raztega}\pojem{središčem} $S$, \index{koeficient!rotacijskega raztega}\pojem{koeficientom} $k$ in \index{kot!rotacijskega raztega}\pojem{kotom} $\omega$.


\begin{figure}[!htb]
\centering
\input{sl.pod.7.1p.1.pic}
\caption{} \label{sl.pod.7.1p.1.pic}
\end{figure}

Po izreku \ref{homotIzomKom} je:
$$\rho_{S,k,\omega}=h_{S,k}\circ \mathcal{R}_{S,\omega}=
\mathcal{R}_{S,\omega}\circ h_{S,k}.$$


Ker sta središčni razteg in rotacija direktni transformaciji (izreka \ref{homotDirekt} in \ref{RotacDirekt}), je tudi rotacijski razteg direktna transformacija podobnosti.

Jasno je, da tudi središčni razteg lahko obravnavamo kot vrsto rotacijskega raztega, če privzamemo, da je identiteta rotacija za kot $0^0$:
$$h_{S,k}=\rho_{S,k,0^0}.$$

Prav tako lahko tudi rotacijo vidimo kot vrsto rotacijskega raztega:
$$\mathcal{R}_{S,\omega}=\rho_{S,1,\omega}.$$


                \bizrek \label{rotacRaztKot}
                An arbitrary line and its image under a stretch rotation
                determine an oriented angle which is congruent to the angle of this stretch rotation:
                    $$\rho_{S,k,\omega}(p)=p'\hspace*{1mm} \Rightarrow
                    \hspace*{1mm} \angle p,p'=\omega.$$
                 \eizrek


\begin{figure}[!htb]
\centering
\input{sl.pod.7.1p.3.pic}
\caption{} \label{sl.pod.7.1p.3.pic}
\end{figure}

\textbf{\textit{Proof.}}
Naj bo $p'=\rho_{S,k,\omega}(p)$ slika premice $p$ pri rotacijskem raztegu $\rho_{S,k,\omega}=h_{S,k}\circ \mathcal{R}_{S,\omega}$ (Figure \ref{sl.pod.7.1p.3.pic}) ter $p_1=\mathcal{R}_{S,\omega}(p)$. Potem je $h_{S,k}(p_1)=p'$. Po izreku \ref{rotacPremPremKot} je $\measuredangle p,p_1=\omega$. Po izreku \ref{RaztPremica} je $p_1\parallel p'$. Torej $\measuredangle p,p'=\angle p,p_1=\omega$ (izrek \ref{KotiTransverzala1}).
\kdokaz



                \bizrek \label{rotacRaztKompSredZrc}
                The product of a half-turn and a stretch rotation
                with the same centre is a stretch rotation. Furthermore:
                $$\rho_{S,k,\omega}\circ \mathcal{S}_S=
                \mathcal{S}_S\circ\rho_{S,k,\omega}=\rho_{S,-k,\omega}.$$
                \eizrek

\textbf{\textit{Proof.}}  (Figure \ref{sl.pod.7.1p.3a.pic})

Kot smo že omenili v razdelku \ref{odd7SredRazteg}, je  $\mathcal{S}_S=h_{S,-1}$.
Če uporabimo izrek \ref{homotGrupa}, dobimo:
\begin{eqnarray*}
\rho_{S,k,\omega}\circ \mathcal{S}_S=
\mathcal{R}_{S,\omega}\circ h_{S,k}\circ h_{S,-1}=
\mathcal{R}_{S,\omega}\circ h_{S,-k}=\rho_{S,-k,\omega}.
\end{eqnarray*}
Na enak način dobimo tudi:
\begin{eqnarray*}
\mathcal{S}_S\circ\rho_{S,k,\omega}=
h_{S,-1}\circ h_{S,k}\circ \mathcal{R}_{S,\omega}=
h_{S,-k}\circ\mathcal{R}_{S,\omega}=\rho_{S,-k,\omega},
\end{eqnarray*}
 kar je bilo treba dokazati. \kdokaz


\begin{figure}[!htb]
\centering
\input{sl.pod.7.1p.3a.pic}
\caption{} \label{sl.pod.7.1p.3a.pic}
\end{figure}

                A direct consequence is the following theorem.

                \bizrek \label{rotacRaztNegKoefk}
                For each stretch rotation is:
                $$\rho_{S,-k,\omega}=\rho_{S,k,180^0+\omega}.$$
                \eizrek

\textbf{\textit{Proof.}} (Figure \ref{sl.pod.7.1p.3a.pic})

Po prejšnjem izreku \ref{rotacRaztKompSredZrc} in izreku \ref{rotacKomp2rotac} je:
\begin{eqnarray*}
\rho_{S,-k,\omega}=
\mathcal{S}_S\circ\rho_{S,k,\omega}=
\mathcal{S}_S\circ\mathcal{R}_{S,\omega}\circ h_{S,k}
=\mathcal{R}_{S,180^0+\omega}\circ h_{S,k}=
\rho_{S,k,180^0+\omega},
\end{eqnarray*}
 kar je bilo treba dokazati. \kdokaz

Če za izometrijo iz izreka \ref{TransPodKompHomIzo} izberemo zrcaljenje čez premico, ki poteka skozi središče središčnega raztega, dobimo še eno vrsto transformacij podobnosti.

Kompozitum osnega zrcaljenja $s$ in središčnega raztega $h_{S,k}$ s središčem $S\in s$ imenujemo \index{osni razteg} \pojem{osni razteg} (Figure
\ref{sl.pod.7.1p.1a.pic}):
$$\sigma_{S,k,s}=h_{S,k}\circ \mathcal{S}_s;\hspace*{2mm} (S\in s)$$
s \index{središče!osnega raztega}\pojem{središčem} $S$, \index{koeficient!rotacijskega raztega}\pojem{koeficientom} $k$ in \index{os!osnega raztega}\pojem{osjo} $s$.


\begin{figure}[!htb]
\centering
\input{sl.pod.7.1p.1a.pic}
\caption{} \label{sl.pod.7.1p.1a.pic}
\end{figure}

Po izreku \ref{homotIzomKom} je:
$$\sigma_{S,k,s}=h_{S,k}\circ \mathcal{S}_s=
\mathcal{S}_s\circ h_{S,k}.$$

Ker je središčni razteg direktna in rotacija indirektna transformacija (izreka \ref{homotDirekt} in \ref{izozrIndIzo}), je  osni razteg indirektna transformacija podobnosti.




                \bizrek \label{transPod1FixTocLema}
                Let $f$ be a similarity transformation that is not an isometry.
                If $f$ maps each line to its parallel line, then $f$ is a homothety.
                \eizrek



\begin{figure}[!htb]
\centering
\input{sl.pod.7.1p.1bb.pic}
\caption{} \label{sl.pod.7.1p.1bb.pic}
\end{figure}

\textbf{\textit{Proof.}}  (Figure \ref{sl.pod.7.1p.1bb.pic})

Označimo s $k$ koeficient podobnosti transformacije $f$. Ker po predpostavki $f$ ni izometrija, je $k\neq 1$.

Po predpostavki $f$ preslika vsako premico v vzporedno premico. Dokažimo najprej, da obstajata vsaj dve premici, ki se sekata in se ne preslikata vase.
Naj bodo $X$, $Y$ in $Z$ poljubne nekolinearne točke ravnine ter  $p$, $q$ in $r$ premice, določene s točkami  $X$, $Y$ in $Z$: $p=XY$, $q=YZ$ in $r=XZ$. Označimo s $p'=f(p)$, $q'=f(q)$ in $r'=f(r)$. Po predpostavki je $p\parallel p'$, $q\parallel q'$ in $r\parallel r'$. Dokažimo, da se vsaj ena od treh premic ne preslika vase. Predpostavimo nasprotno, da je $p=p'$, $q=q'$ in $r=r'$. Toda v tem primeru je $f(X)=f(p\cap q)=f(p)\cap f(q)=p\cap q=X$ in podobno npr. $f(Y)=Y$. Daljica $XY$ bi se v tem primeru preslikala vase, kar ni možno, saj je $k\neq 1$. Brez škode za splošnost naj bo $p\neq p'$. Na isti način z uporabo nekega trikotnika, pri katerem nobena nosilka stranic ni vzporedna s premico $p$, lahko dokažemo, da obstaja še ena premica, ki seka premico $p$ in se s preslikavo $f$ ne preslika vase.

Torej obstajata premici $b$ in $c$, ki se sekata v točki $A$ ter za $b'=f(b)$ in $c'=f(c)$ velja $b\parallel b'$, $c\parallel c'$, $b\neq b'$ in $c\neq c'$. Naj bosta $B\in b$ ter $C\in c$ poljubni točki, ki sta različni od točke $A$.
Označimo $A'=b\cap c$, $B'=f(B)$ in $C'=f(C)$. Najprej je $f(A)=f(b\cap c)=f(b)\cap f(c)=b'\cap c'=A'$, $B'\in b'$ in $C'\in c'$.
Ker je $b\parallel b'$, je tudi $AB\parallel A'B'$. Daljica $AB$ se s transformacijo $f$ slika v daljico $A'B'$, zato je $A'B'=k\cdot AB$.
Premici $AA'$ in $BB'$ nista vzporedni. V nasprotnem bi bil štirikotnik $ABB'A'$ paralelogram oz. $AB\cong A'B'$ (izrek \ref{paralelogram}), kar ni možno, ker je $k\neq 1$. Označimo s $S$ presečišče premic $AA'$ in $BB'$. Ker je $b\parallel b'$, je po Talesovem izreku:
$$\frac{SA'}{SA}=\frac{SB'}{SB}=\frac{A'B'}{AB}=k.$$
 To pomeni, da središčni razteg $h_{S,k_1}$ s središčem $S$ in koeficientom $k_1=k$ (ali $k_1=-k$) preslika točki $A$ in $B$ v točki $A'$ in $B'$.
 Naj bo $\widehat{C'}=h_{S,k_1}(C)$. Ker je po predpostavki $C'=f(C)$ oz. $A'C'=k\cdot AC$, je po izreku \ref{RaztPremica} $\widehat{C'}=C'$ oz. $h_{S,k_1}(C)=C'$.

 Preslikava
$$g=f^{-1}\circ h_{S,k_1}$$
 je po izreku \ref{TransPodGrupa} transformacija podobnosti s koeficientom podobnosti $\frac{1}{k}\cdot |k_1|=\frac{1}{k}\cdot k=1$, zato predstavlja izometrijo.
Toda $g(A)=f^{-1}\circ h_{S,k_1}(A)=f^{-1}(A')=A$, oz. $A$ je fiksna točka izometrije $g$. Na podoben način dokažemo, da sta tudi $B$ in $C$ fiksni točki izometrije $g$, kar pomeni, da je $g=\mathcal{E}$ identiteta (izrek \ref{IizrekABC2}). Torej $f^{-1}\circ h_{S,k_1}=g=\mathcal{E}$ oz. $f=h_{S,k_1}$.
\kdokaz

 Sedaj smo pripravljeni na naslednji pomemben izrek.

                \bizrek \label{transPod1FixToc}
                Each similarity transformation other than isometry has exactly one fixed point.
                \eizrek


\begin{figure}[!htb]
\centering
\input{sl.pod.7.1p.1c.pic}
\caption{} \label{sl.pod.7.1p.1c.pic}
\end{figure}

\textbf{\textit{Proof.}}  (Figure \ref{sl.pod.7.1p.1c.pic})

Naj bo $f$ transformacija podobnosti s koeficientom $k$. Po predpostavki je $k\neq 1$.
Podobno kot v dokazu prejšnjega izreka \ref{transPod1FixTocLema} se hitro prepričamo, da $f$ ne more imeti dveh fiksnih točk, ker bi bil $k=1$.

Dokažimo, da ima $f$ fiksno točko.
Po prejšnjem izreku  (\ref{transPod1FixTocLema}) lahko predpostavimo, da obstaja vsaj ena premica $p$, ki se ne preslika v svojo vzporednico. Res, če predpostavimo, da se vsaka premica ravnine slika v svojo vzporednico, je po omenjenem izreku  \ref{transPod1FixTocLema} preslikava $f$ središčni razteg, ki ima fiksno točko (njegovo središče).
Naj bo torej $p'=f(p)$ in $ p'\nparallel p$ ($\neg p'\parallel p$). Označimo z $A$ presečišče premic $p$ in $p'$. Naj bo $A'=f(A)$. Če je $A'=A$, je $A$ fiksna točka transformacije $f$ in je dokaz končan. Naj bo torej $A'\neq A$. Iz $A\in p$ sledi $A'\in p'$. Označimo z $q$ vzporednico premice $p$ skozi točko $A'$. Ker je $A'\neq A$, je $q\neq p$. Naj bo $q'=f(q)$. Ker je $p\parallel q$ in $f:\hspace*{1mm}p,q\rightarrow p',q'$, je tudi $p'\parallel q'$ (če bi se $p'$ in $q'$ sekali v neki točki $T$, bi se tudi premici $p$ in $q$ sekali v točki $f^{-1}(T)$). Na enak način iz $q\neq p$ sledi $q'\neq p'$. To pomeni, da premice $p$, $q$, $p'$ in $q'$ določajo paralelogram $AA'BC$, kjer je
 $B=q\cap q'$ in $C=p\cap q'$. Naj bo $B'=f(B)$. Iz $B\in q$ sledi $B'\in q'$. Če je $B'=B$, je $B$ fiksna točka in je dokaz končan. Tako lahko naprej predpostavimo $B'\neq B$. Iz $p'\parallel q'$ sledi $AA'\parallel BB'$. V primeru $AB\parallel A'B'$ bi bil štirikotnik $AA'B'B$ paralelogram oz. $AB\cong A'B'$, kar ni možno, ker je $f:\hspace*{1mm}A,B\rightarrow A',B'$ in $k\neq 1$. Torej se premici $AB$ in $A'B'$ sekata v neki točki $S$. Označimo $S'=f(S)$. Dokažimo, da je $S$ fiksna točka oz. $S'=S$.
 Po Talesovem izreku je:
\begin{eqnarray} \label{eqnTransfPod1Ft1}
 \frac{AS}{SB}=\frac{A'S}{SB'}
\end{eqnarray}
 Ker $f:\hspace*{1mm}A,B,S\rightarrow A',B',S'$, je tudi (izrek \ref{TransPodOhranjajoRazm}):
\begin{eqnarray}  \label{eqnTransfPod1Ft2}
 \frac{AS}{SB}=\frac{A'S'}{S'B'}
\end{eqnarray}
Prav tako iz $S\in AB$ sledi $S'\in A'B'$.
Če povežemo zadnji dve enakosti  \ref{eqnTransfPod1Ft1} in
 \ref{eqnTransfPod1Ft2}, je:
\begin{eqnarray}  \label{eqnTransfPod1Ft3}
 \frac{A'S}{SB'}=\frac{A'S'}{S'B'},
\end{eqnarray}
kjer sta $S$ in $S'$ točki premice $A'B'$. Toda transformacije podobnosti ohranjajo relacijo $\mathcal{B}$ (izrek \ref{TransPodB}), oziroma:
\begin{eqnarray*}
 \mathcal{B}(A,S,B)\hspace*{1mm} &\Leftrightarrow& \hspace*{1mm} \mathcal{B}(A',S',B');\\
 \mathcal{B}(S,A,B)\hspace*{1mm} &\Leftrightarrow& \hspace*{1mm} \mathcal{B}(S',A',B');\\
\mathcal{B}(A,B,S)\hspace*{1mm} &\Leftrightarrow& \hspace*{1mm} \mathcal{B}(A',B',S').
\end{eqnarray*}
 Iz tega in iz relacije \ref{eqnTransfPod1Ft3} sledi:
\begin{eqnarray*}
 \frac{\overrightarrow{A'S}}{\overrightarrow{SB'}}=
\frac{\overrightarrow{A'S'}}{\overrightarrow{S'B'}},
\end{eqnarray*}
 zato je po izreku \ref{izrekEnaDelitevDaljiceVekt} $S'=S$ oz. $S$ je fiksna točka transformacije podobnosti $f$.
\kdokaz

Sedaj lahko naredimo napovedano klasifikacijo transformacij podobnosti.



                \bizrek \label{transPodKlasif}
                The only similarity transformations of the plane are:
                    \begin{itemize}
                      \item isometries,
                      \item homotheties,
                      \item stretch rotations,
                      \item stretch reflections.
                    \end{itemize}
                \eizrek

\textbf{\textit{Proof.}}
Naj bo $f$ poljubna transformacija podobnosti s koeficientom $k$.

Če je $k=1$, je $f$ izometrija.

Predpostavimo, da je $k\neq 1$ oz. $f$ ni izometrija. Po izreku \ref{transPod1FixToc} ima $f$ natanko eno fiksno točko - označimo jo s $S$. Torej $f(S)=S$. Po izreku  \ref{TransPodKompHomIzo} lahko transformacijo podobnosti $f$ predstavimo kod kompozitum središčnega raztega s poljubnim središčem (izberimo za središče ravno točko $S$) in koeficientom $k$ ter ene izometrije $\mathcal{I}$:
$$f=\mathcal{I}\circ h_{S,k}.$$
 Ker je $S$ fiksna točka transformacije podobnosti $f$ in središčnega raztega $h_{S,k}$,
je:
$$S=f(S)=\mathcal{I}\circ h_{S,k}(S)=\mathcal{I}(S).$$
Torej $\mathcal{I}(S)=S$ oz.  $\mathcal{I}$ je izometrija s fiksno točko $S$. Po izreku \ref{Chaslesov} je  $\mathcal{I}$ lahko: identiteta, rotacija s središčem $S$ ali zrcaljenje čez premico, ki poteka skozi točko $S$:
\begin{eqnarray*}
\mathcal{I}=\left\{
              \begin{array}{l}
                \mathcal{E}, \\
                \mathcal{R}_{S,\omega}, \\
                \mathcal{S}_s (S\in s)
              \end{array}
            \right.
\end{eqnarray*}
Zato je:
\begin{eqnarray*}
f=\mathcal{I}\circ h_{S,k}=\left\{
              \begin{array}{l}
                h_{S,k}, \\
                \rho_{S,k,\omega}, \\
                \sigma_{S,k,s}
              \end{array}
            \right.
\end{eqnarray*}
 kar je bilo treba dokazati. \kdokaz

 A direct consequence is the following theorem.


                \bizrek \label{transPodKlasifDirInd}
                The only direct similarity transformations of the plane other than isometries are:
                    \begin{itemize}
                      \item homotheties,
                      \item stretch rotations.
                    \end{itemize}
                The only opposite similarity transformations of the plane other than isometries are:
                    \begin{itemize}
                      \item stretch reflections.
                    \end{itemize}
                \eizrek


Zelo koristna je lastnost kompozituma dveh rotacijskih raztegov.



            \bizrek \label{RotRazKomoz}
            The product of two stretch rotations $\rho_{S_1,k_1,\omega_1}$ and $\rho_{S_2,k_2,\omega_2}$
            is a direct isometry, a homothety or a stretch rotation. Furthermore (Figure \ref{sl.pod.7.1p.5a.pic}):
            \begin{eqnarray*}
            \rho_{S_2,k_2,\omega_2}\circ \rho_{S_1,k_1,\omega_1}=
            \left\{
              \begin{array}{ll}
                \mathcal{R}_{S,\omega}, & k_1k_2=1, \hspace*{2mm}
                \omega=\omega_1+\omega_2\neq n\cdot 180^0; \\
                \mathcal{T}_{\overrightarrow{v}}, & k_1k_2=1, \hspace*{2mm}
                \omega=\omega_1+\omega_2=n\cdot 180^0; \\
                h_{S,k}, & k=k_1k_2\neq 1, \hspace*{2mm}
                \omega=\omega_1+\omega_2=n\cdot 180^0; \\
                \rho_{S,k,\omega}, &  k=k_1k_2\neq 1, \hspace*{2mm}
                \omega=\omega_1+\omega_2\neq n\cdot 180^0.
              \end{array}
            \right.
            \end{eqnarray*}
            for $n\in \mathbb{Z}$.
            \eizrek


\begin{figure}[!htb]
\centering
\input{sl.pod.7.1p.5a.pic}
\caption{} \label{sl.pod.7.1p.5a.pic}
\end{figure}


\textbf{\textit{Proof.}} (Figure \ref{sl.pod.7.1p.5.pic})

Brez škode za splošnost predpostavimo, da velja $k_1>0$ in $k_2>0$. Če je npr. $k_1<0$, lahko po izreku \ref{rotacRaztNegKoefk} zapišemo
 $\rho_{S,k_1,\omega}=\rho_{S,-k_1,180^0+\omega}$, kjer je $-k_1>0$.

Označimo $f=\rho_{S_2,k_2,\omega_2}\circ \rho_{S_1,k_1,\omega_1}$. Po izrekih \ref{RaztTransPod} in \ref{TransPodGrupa} je $f$ transformacija podobnosti
s koeficientom $k=k_1\cdot k_2$. Le-ta je direktna transformacija kot kompozitum dveh direktnih transformacij. Po izreku  \ref{transPodKlasifDirInd} je $f$ lahko izometrija, središčni razteg ali rotacijski razteg.

\begin{figure}[!htb]
\centering
\input{sl.pod.7.1p.5.pic}
\caption{} \label{sl.pod.7.1p.5.pic}
\end{figure}


Naj bo $p$ poljubna premica in $f(p)=p'$. Izračunajmo mero orientiranega kota $\measuredangle p,p'$. Naj bo $\rho_{S_1,k_1,\omega_1}(p)=p_1$ in posledično $\rho_{S_2,k_2,\omega_2}(p_1)=p'$. Po izreku \ref{rotacRaztKot} je $\angle p,p_1=\omega_1$ in $\angle p_1,p'=\omega_2$.
Če je $\omega=\omega_1+\omega_2\neq n\cdot 180^0$ za vsak $n\in \mathbb{Z}$, je
$\angle p,p'=\omega_1+\omega_2$ (izrek \ref{zunanjiNotrNotr}).
Če je $\omega=\omega_1+\omega_2=n\cdot 180^0$ za nek $n\in \mathbb{Z}$, je $p\parallel p'$ (izrek \ref{KotiTransverzala}).


Če je $k_1\cdot k_2=1$, je kompozitum $f$ transformacija podobnosti s koeficientom $k=1$ in predstavlja direktno izometrijo $f=\mathcal{I}\in \mathfrak{I}^+$. Le-ta je lahko translacija ali rotacija (v posebnem primeru identiteta) (izrek \ref{Chaslesov+}).
 Če je $\omega=\omega_1+\omega_2\neq n\cdot 180^0$ oz.
$\angle p,p'=\omega_1+\omega_2$, gre za rotacijo za kot $\omega=\omega_1+\omega_2$ (izrek \ref{rotacPremPremKot}).
 Če je $\omega=\omega_1+\omega_2=n\cdot 180^0$ oz. $p\parallel p'$, je $\mathcal{I}$ translacija (ali identiteta).

Predpostavimo sedaj, da velja $k_1\cdot k_2\neq 1$. Kot smo že omenili, je v tem primeru $f$ lahko središčni razteg ali rotacijski razteg s koeficientom $k=k_1\cdot k_2$.
 Če je $\omega=\omega_1+\omega_2\neq n\cdot 180^0$ oz.
$\angle p,p'=\omega_1+\omega_2$, gre za rotacijski razteg za kot $\omega=\omega_1+\omega_2$  (izrek \ref{rotacRaztKot}).
 Če je $\omega=\omega_1+\omega_2=n\cdot 180^0$ oz. $p\parallel p'$, je $f$ po izreku \ref{RaztPremica} središčni razteg (ali  središčno zrcaljenje, če je $k_1\cdot k_2=-1$).
\kdokaz

A direct consequence is the following theorem.


            \bizrek

             The product of two homotheties
            is a direct isometry or a homothety. Furthermore:
            \begin{eqnarray*}
            h_{S_2,k_2}\circ h_{S_1,k_1}=
            \left\{
              \begin{array}{ll}
                \mathcal{T}_{\overrightarrow{v}}, & k_1k_2=1; \\
                h_{S,k}, & k=k_1k_2\neq \pm 1.
              \end{array}
            \right.
            \end{eqnarray*}
            \eizrek

\textbf{\textit{Proof.}}
Trditev sledi direktno iz prejšnjega izreka, če zapišemo
$h_{S_1,k_1}=\rho_{S_1,k_1,0^0}$ in $h_{S_2,k_2}=\rho_{S_2,k_2,0^0}$.
\kdokaz



            \bnaloga\footnote{17. IMO Bulgaria - 1975, Problem 3.}
            On the sides of an arbitrary triangle $ABC$, triangles $ABR$, $BCP$, $CAQ$ are
            constructed externally with $\angle CBP\cong\angle CAQ=45^0$, $\angle BCP\cong\angle ACQ=35^0$,
            $\angle ABR\cong\angle BAR=15^0$. Prove that $\angle QRP=90^0$ and $QR\cong RP$.
            \enaloga


\begin{figure}[!htb]
\centering
\input{sl.pod.7.1.IMO2.pic}
\caption{} \label{sl.pod.7.1.IMO2.pic}
\end{figure}

\textbf{\textit{Solution.}} Naj bo $S$ tretje oglišče
enakostraničnega trikotnika $BAS$, ki je načrtan nad stranico $BA$
trikotnika $ABC$ (Figure \ref{sl.pod.7.1.IMO2.pic}). Ker je $\angle
RBS\cong\angle SAR=45^0$ in $\angle BSR\cong\angle ASR=30^0$, so
si trikotniki $BPC$, $AQC$, $BRS$ in $ARS$ med seboj podobni.

Naj bosta $f_1$ in $f_2$ rotacijska raztega:
 \begin{eqnarray*}
  &&f_1=\rho_{B,k,45^0}=h_{B,k}\circ \mathcal{R}_{B,45^0}\\
  &&f_2=\rho_{A,\frac{1}{k},45^0}=h_{A,\frac{1}{k}}\circ
  \mathcal{R}_{A,45^0},
 \end{eqnarray*}
kjer je:
$$k=\frac{|BC|}{|BP|}=\frac{|AC|}{|AQ|}=\frac{|BS|}{|BR|}
=\frac{|AS|}{|AR|}.$$
 Naj bo $f=f_2\circ f_1$. Po izreku \ref{RotRazKomoz} je:
 $$f=f_2\circ f_1=\rho_{A,\frac{1}{k},45^0}\circ
 \rho_{B,k,45^0}=\rho_{T,1,90^0}=\mathcal{R}_{T,90^0}.$$
  Naprej velja:
 \begin{eqnarray*}
  &&\mathcal{R}_{T,90^0}(P)=f(P)=f_2\circ f_1(P)=f_2(C)=Q\\
  &&\mathcal{R}_{T,90^0}(R)=f(R)=f_2\circ f_1(R)=f_2(S)=R.
 \end{eqnarray*}
 Iz prejšnje (druge) relacije vidimo, da je $\mathcal{R}_{T,90^0}(R)=R$, kar pomeni, da je
 $R=T$ (izrek \ref{RotacFiksT}) oz.
 $\mathcal{R}_{T,90^0}=\mathcal{R}_{R,90^0}$.
 Iz tega in iz prve relacije sedaj sledi:
  $$\mathcal{R}_{R,90^0}(P)=Q,$$
  kar pomeni, da velja $\angle QRP=90^0$
             in $QR\cong RP$.
 \kdokaz


%________________________________________________________________________________
 \poglavje{Similar Figures. Similarity of Triangles} \label{odd7PodobTrik}


Kot smo že napovedali v uvodu tega poglavja, nam transformacije podobnosti omogočajo definicijo pojma podobnosti likov.

Pravimo, da je lik $\Phi$ \index{lika!podobna}\pojem{podoben} liku $\Phi'$ iz iste ravnine  (oznaka $\Phi\sim \Phi'$),
 če obstaja transformacija podobnosti $f$ te ravnine, ki lik $\Phi$ preslika v lik $\Phi'$ oz. $f(\Phi)=\Phi'$ (Figure \ref{sl.pod.7.2.1.pic}).
 Koeficient podobnosti transformacije $f$ je hkrati \index{koeficient!podobnosti likov}\pojem{koeficient podobnosti likov} $\Phi$ in $\Phi'$.

\begin{figure}[!htb]
\centering
\input{sl.pod.7.2.1.pic}
\caption{} \label{sl.pod.7.2.1.pic}
\end{figure}

Jasno je, da za $k=1$ dobimo skladnost likov kot poseben primer podobnosti. Skladna lika sta si torej tudi podobna, obratno pa ne velja, torej:
$$\Phi\cong \Phi' \hspace*{1mm}\Rightarrow \hspace*{1mm}\Phi\sim \Phi'.$$

Dokažimo najpomembnejšo lastnost relacije podobnosti likov.

                \bizrek
                 The similarity of figures is an equivalence relation.
                \eizrek

 \textbf{\textit{Proof.}} Potrebno (in dovolj) je dokazati, da je relacija podobnosti likov refleksivna, simetrična in tranzitivna.


 (\textit{R}) Za vsak lik $\Phi$ velja $\Phi\sim \Phi$, ker je identiteta $\mathcal{E}$, ki preslika lik $\Phi$ vase, transformacija podobnosti s koeficientom $k=1$  (Figure \ref{sl.pod.7.2.1r.pic}).


\begin{figure}[!htb]
\centering
\input{sl.pod.7.2.1r.pic}
\caption{} \label{sl.pod.7.2.1r.pic}
\end{figure}

 (\textit{S}) Predpostavimo, da za lika $\Phi_1$ in $\Phi_2$ velja $\Phi_1\sim \Phi_2$ (Figure \ref{sl.pod.7.2.1s.pic}). Po definiciji  obstaja transformacija podobnosti $f$, ki lik $\Phi_1$ preslika v lik $\Phi_2$, oz. $f:\hspace*{1mm}\Phi_1\rightarrow \Phi_2$.
 Po izreku \ref{TransPodGrupa} je inverzna preslikava $f^{-1}$, za katero je $f^{-1}:\hspace*{1mm}\Phi_2\rightarrow \Phi_1$,  prav tako transformacija podobnosti. Torej velja $\Phi_2\sim \Phi_1$.

\begin{figure}[!htb]
\centering
\input{sl.pod.7.2.1s.pic}
\caption{} \label{sl.pod.7.2.1s.pic}
\end{figure}

 (\textit{T}) Predpostavimo, da za like $\Phi_1$, $\Phi_2$ in $\Phi_3$ velja $\Phi_1\sim \Phi_2$ in $\Phi_2\sim \Phi_3$ (Figure \ref{sl.pod.7.2.1t.pic}). Dokažimo, da je potem tudi $\Phi_1\sim \Phi_3$. Po definiciji obstajata takšni transformaciji podobnosti $f$ in $g$, da velja $f:\hspace*{1mm}\Phi_1\rightarrow \Phi_2$ in $g:\hspace*{1mm}\Phi_2\rightarrow \Phi_3$. Toda po izreku \ref{TransPodGrupa} je kompozitum preslikav $g\circ f:\hspace*{1mm}\Phi_1\rightarrow \Phi_3$
 tudi transformacija podobnosti, zato velja $\Phi_1\sim \Phi_3$.
\kdokaz

\begin{figure}[!htb]
\centering
\input{sl.pod.7.2.1t.pic}
\caption{} \label{sl.pod.7.2.1t.pic}
\end{figure}



Ker je iz prejšnjega izreka relacija podobnosti likov simetrična, bomo v primeru $\Phi\sim \Phi'$ rekli, da sta si lika $\Phi$ in $\Phi'$ podobna.

V nadaljevanju bomo posebej obravnavali podobnost trikotnikov. Predpostavimo, da sta si trikotnika $ABC$ in $A'B'C'$ podobna oz. $\triangle ABC\sim\triangle A'B'C'$. To pomeni, da obstaja transformacija podobnosti, ki preslika trikotnik $ABC$ v trikotnik $A'B'C'$. V primeru trikotnikov (tudi večkotnikov) bomo dodatno zahtevali, da se oglišča s transformacijo podobnosti preslikajo po vrsti, torej $f:\hspace*{1mm}A,B,C\mapsto A',B',C'$.
Ker transformacija podobnosti  preslika daljice v daljice in kote v kote (izrek \ref{TransPodKol}), se s transformacijo $f$ stranice $AB$, $BC$ in $CA$ preslikajo v stranice $A'B'$, $B'C'$ in $C'A'$, notranji koti $BAC$, $ABC$ in $ACB$ trikotnika $ABC$ pa v notranje kote $B'A'C'$, $A'B'C'$ in $A'C'B'$ trikotnika $A'B'C'$. Za pare elementov v tej preslikavi bomo rekli, da so \pojem{ustrezni} ali \pojem{istoležni}.

Po izreku \ref{TransPodOhranjajoRazm} transformacije podobnosti ohranjajo razmerje daljic, kar pomeni, da so ustrezne stranice sorazmerne,torej\footnote{Z uporabo podobnosti enakokrakih pravokotnih trikotnikov s pomočjo ustreznega sorazmerja je \index{Tales}\textit{Tales iz Mileta} (7.--6. st. pr. n. š.) izračunal višino Keopsove piramide.} (Figure \ref{sl.pod.7.2.2.pic}):
 \begin{eqnarray} \label{eqnPodTrik1}
 \triangle ABC\sim\triangle A'B'C'\hspace*{1mm} \Rightarrow \hspace*{1mm} \frac{A'B'}{AB}=\frac{A'C'}{AC}=\frac{B'C'}{BC}=k,
 \end{eqnarray}
kjer je $k$ koeficient podobnosti.

\begin{figure}[!htb]
\centering
\input{sl.pod.7.2.2.pic}
\caption{} \label{sl.pod.7.2.2.pic}
\end{figure}

Po izreku \ref{TransPodOhranjaKote} transformacije podobnosti ohranjajo kote, torej (Figure \ref{sl.pod.7.2.2.pic}):
 \begin{eqnarray} \label{eqnPodTrik2}
 \triangle ABC\sim\triangle A'B'C'\hspace*{1mm} \Rightarrow \hspace*{1mm}
 \left\{
   \begin{array}{l}
    \angle B'A'C'\cong\angle BAC; \\
    \angle A'B'C'\cong\angle ABC; \\
     \angle A'C'B'\cong\angle ACB.
   \end{array}
 \right.
 \end{eqnarray}



 Iz dejstva, da transformacija podobnosti ohranja kote (izrek \ref{TransPodOhranjaKote}) in razmerje daljic (izrek \ref{TransPodOhranjajoRazm}), sledi tudi, da se višine, težiščnice,... enega trikotnika preslikajo v višine, težiščnice,... drugega trikotnika, pri tem pa se ohranja razmerje ustreznih elementov.

Pri podobnosti trikotnikov imamo enak problem kot pri skladnosti - zelo pogosto ni enostavno dokazati podobnosti trikotnikov direktno po definiciji. Tako, analogno kot pri skladnosti, dobimo tudi \index{izrek!o podobnosti trikotnikov}\pojem{izreke o podobnosti trikotnikov}.
Ugotovili smo že, da iz podobnosti dveh trikotnikov $ABC$ in $A'B'C'$ dobimo sorazmerje ustreznih stranic \ref{eqnPodTrik1} oz. skladnost ustreznih kotov \ref{eqnPodTrik2}. Naslednji izreki govorijo o tem, kateri pogoji zadoščajo, da sta trikotnika podobna.




         \bizrek \label{PodTrikSKS}
         Triangles $ABC$ and $A'B'C'$ are similar
         if two pairs of the sides of the triangles are proportional and the pair of included angles between the sides is congruent, i.e.:
          \begin{eqnarray*}
            \left.
             \begin{array}{l}
              \angle B'A'C'\cong \angle BAC\\
             \frac{A'B'}{AB}=\frac{A'C'}{AC}
            \end{array}
            \right\}\hspace*{1mm}\hspace*{1mm}\Rightarrow\triangle ABC\sim\triangle A'B'C'
            \end{eqnarray*}
            \eizrek

\begin{figure}[!htb]
\centering
\input{sl.pod.7.2.3.pic}
\caption{} \label{sl.pod.7.2.3.pic}
\end{figure}



 \textbf{\textit{Proof.}} (Figure \ref{sl.pod.7.2.3.pic})
Naj bo:
 $$k=\frac{A'B'}{AB}=\frac{A'C'}{AC}.$$
Naj bosta še $B''$ i $C''$ takšni točki poltrakov
$A'B'$ in $A'C'$, da velja $A'B''\cong AB$ in
$A'C''\cong AC$. Trikotnika
 $A'B''C''$ in $ABC$ sta skladna (izrek \textit{SAS} \ref{SKS}), zato obstaja izometrija $\mathcal{I}$, ki trikotnik $ABC$ preslika v trikotnik
$A'B''C''$. Ker je
 $$\frac{A'B'}{A'B''}=\frac{A'B'}{AB}=k\hspace*{1mm}\textrm{ in }\hspace*{1mm}\frac{A'C'}{A'C''}=\frac{A'C'}{AC}=k,$$ oz. (ker točki $B''$ in $C''$ ležita na poltrakih
$A'B'$ i $A'C'$) $\overrightarrow{A'B'}=k\cdot \overrightarrow{A'B''}$ in $\overrightarrow{A'C'}=k\cdot \overrightarrow{A'C''}$, središčni razteg $h_{A',k}$ preslika točke $A'$, $B''$ in $C''$ v točke  $A'$, $B'$ in $C'$ oz. trikotnik $A'B''C''$ v trikotnik $A'B'C'$. Torej kompozitum $f=h_{A',k}\circ \mathcal{I}$, ki je transformacija podobnosti, preslika trikotnik $ABC$ v trikotnik $A'B'C'$, kar pomeni, da velja $\triangle ABC\sim\triangle A'B'C'$.
 \kdokaz



            \bizrek \label{PodTrikKKK}
            Triangles $ABC$ and $A'B'C'$ are similar
         if two pairs of the angles of the triangles  are congruent, i.e.:
            \begin{eqnarray*}
            \left.
             \begin{array}{l}
             \angle B'A'C'\cong\angle BAC\\
             \angle A'B'C'\cong\angle ABC
            \end{array}
            \right\}\hspace*{1mm}\hspace*{1mm}\Rightarrow\triangle ABC\sim\triangle A'B'C'
            \end{eqnarray*}
        \eizrek

\begin{figure}[!htb]
\centering
\input{sl.pod.7.2.4.pic}
\caption{} \label{sl.pod.7.2.4.pic}
\end{figure}

 \textbf{\textit{Proof.}} (Figure \ref{sl.pod.7.2.4.pic})
Označimo:
 $$k=\frac{A'B'}{AB}.$$
Naj bosta $B''$ i $C''$ takšni točki poltrakov
$A'B'$ in $A'C'$, da velja $A'B''\cong AB$ in
$A'C''\cong AC$. Potem je tudi $\frac{A'B'}{A'B''}=k$. Trikotnika
 $A'B''C''$ in $ABC$ sta skladna (izrek \textit{SAS} \ref{SKS}), zato obstaja izometrija $\mathcal{I}$, ki preslika trikotnik $ABC$ v trikotnik
$A'B''C''$. Iz te skladnosti sledi še $\angle A'B''C''\cong\angle ABC$. Ker je po predpostavki $\angle A'B'C'\cong\angle ABC$, je tudi $\angle A'B''C''\cong\angle A'B'C'$. Po izreku \ref{KotiTransverzala} je $B''C''\parallel B'C'$. Po Talesovem \ref{TalesovIzrek} izreku je
$$\frac{A'C'}{A'C''}=\frac{A'B'}{A'B''}=k.$$ Ker sta $B''$ in $C''$ točki poltrakov
$A'B'$ in $A'C'$, je tudi $\overrightarrow{A'B'}=k\cdot \overrightarrow{A'B''}$ in $\overrightarrow{A'C'}=k\cdot \overrightarrow{A'C''}$. To pomeni, da središčni razteg $h_{A',k}$ preslika točke $A'$, $B''$ in $C''$ v točke  $A'$, $B'$ in $C'$ oz. trikotnik $A'B''C''$ v trikotnik $A'B'C'$. Kompozitum $f=h_{A',k}\circ \mathcal{I}$, ki je transformacija podobnosti, torej preslika trikotnik $ABC$ v trikotnik $A'B'C'$, zato je $\triangle ABC\sim\triangle A'B'C'$.
 \kdokaz




        \bizrek \label{PodTrikSSS}
        Triangles $ABC$ and $A'B'C'$ are similar
         if three pairs of the sides of the triangles are proportional, i.e.
        \begin{eqnarray*}
             \frac{A'B'}{AB}=\frac{A'C'}{AC}=\frac{B'C'}{BC}
            \hspace*{1mm}\hspace*{1mm}\Rightarrow\triangle ABC\sim\triangle A'B'C'
            \end{eqnarray*}
        \eizrek


\begin{figure}[!htb]
\centering
\input{sl.pod.7.2.5.pic}
\caption{} \label{sl.pod.7.2.5.pic}
\end{figure}

 \textbf{\textit{Proof.}} (Figure \ref{sl.pod.7.2.5.pic})
Označimo:
 $$k=\frac{A'B'}{AB}=\frac{A'C'}{AC}=\frac{B'C'}{BC}.$$
Naj bosta $B''$ in $C''$ takšni točki poltrakov
$A'B'$ in $A'C'$, da velja $A'B''\cong AB$ in
$A'C''\cong AC$. Potem je tudi $$\frac{A'B'}{A'B''}=\frac{A'C'}{A'C''}=k.$$
Po obratnem Talesovem izreku \ref{TalesovIzrekObr} je $B''C''\parallel B'C'$ in $$\frac{B'C'}{B''C''}=\frac{A'B'}{A'B''}.$$
Torej $$\frac{B'C'}{B''C''}=\frac{A'B'}{A'B''}=\frac{A'C'}{A'C''}=k=\frac{B'C'}{BC},$$
kar pomeni, da velja $B''C''\cong BC$. Iz tega sledi, da sta
trikotnika
 $A'B''C''$ in $ABC$ skladna (izrek \textit{SSS} \ref{SSS}), zato obstaja izometrija $\mathcal{I}$, ki preslika trikotnik $ABC$ v trikotnik
$A'B''C''$. Prav tako (podobno kot v dokazu prejšnjih dveh izrekov)  središčni razteg $h_{A',k}$ preslika točke $A'$, $B''$ in $C''$ v točke  $A'$, $B'$ in $C'$ oz. trikotnik $A'B''C''$ v trikotnik $A'B'C'$. Na koncu kompozitum $f=h_{A',k}\circ \mathcal{I}$, ki je transformacija podobnosti, preslika trikotnik $ABC$ v trikotnik $A'B'C'$, zato je $\triangle ABC\sim\triangle A'B'C'$.
 \kdokaz

Naslednji izrek bomo podali brez dokaza  (Figure \ref{sl.pod.7.2.6.pic}).




        \bizrek \label{PodTrikSSK}
        Triangles $ABC$ and $A'B'C'$ are similar
         if two pairs of the sides of the triangles are proportional and the pair of the angles opposite to the longer sides is congruent.
        \eizrek


\begin{figure}[!htb]
\centering
\input{sl.pod.7.2.6.pic}
\caption{} \label{sl.pod.7.2.6.pic}
\end{figure}

Nadaljevali bomo z uporabo izrekov o podobnosti trikotnikov.





                \bzgled
                 Let $k$ be the circumcircle of a triangle $ABC$ and $B'$ and $C'$
                the foots of the perpendiculars  from the vertices $B$ and $C$ on the tangent of the circle $k$ in the vertex $A$.
                Prove that the altitude $AD$ of this triangle is the geometric mean of the line segments $BB'$ and $CC'$, i.e.:
                $$|AD|=\sqrt{|BB'|\cdot |CC'|}.$$
                \ezgled

\begin{figure}[!htb]
\centering
\input{sl.pod.7.2.7.pic}
\caption{} \label{sl.pod.7.2.7.pic}
\end{figure}

 \textbf{\textit{Proof.}} (Figure \ref{sl.pod.7.2.7.pic})

Kot $BAB'$ je skladen obodnemu kotu s
tetivo $AB$ oz. $\angle BAB'\cong\angle ACD$ (izrek \ref{ObodKotTang}). Ker je še
$\angle AB'B\cong\angle CDA=90^0$, je
$\triangle AB'B\sim\triangle CDA$ (izrek\ref{PodTrikKKK}). Iz tega sledi:
\begin{eqnarray} \label{eqnPodTrikZgl1a}
AD:BB'=AC:BA.
\end{eqnarray}

Na enak način dokažemo tudi $\triangle AC'C\sim\triangle BDA$ oz.
\begin{eqnarray} \label{eqnPodTrikZgl1b}
 CC':AD=AC:BA.
\end{eqnarray}
 Iz relacij \ref{eqnPodTrikZgl1a} in \ref{eqnPodTrikZgl1b}
sledi $AD:BB'=CC':AD$ oz. $|AD|=\sqrt{|BB'|\cdot |CC'|}$.
 \kdokaz


            \bzgled \label{izrekSinusni}
             Let $v_a$ be the length of the altitude $AD$, $k(O,R)$ the circumcircle of a triangle $ABC$
             and $b=|AC|$ and $c=|AB|$. Prove that\footnote{Trditev iz tega zgleda v trigonometriji predstavlja t.
            i. \index{izrek!Sinusni}\textit{sinusni izrek}, ki ga je dokazal arabski
            matematik \index{al-Biruni, A. R.}\textit{A. R. al-Biruni} (973--1048). Če
            namreč v dano enakost vstavimo $\frac{v_a}{c}=\sin \beta$, dobimo
            $\frac{b}{\sin\beta}=2R$ in analogno
             $\frac{a}{\sin\alpha}=\frac{b}{\sin\beta}=\frac{c}{\sin\gamma}=2R$.}:
            $$bc=2R\cdot v_a.$$
            \ezgled


\begin{figure}[!htb]
\centering
\input{sl.pod.7.2.8a.pic}
\caption{} \label{sl.pod.7.2.8a.pic}
\end{figure}

 \textbf{\textit{Proof.}} Označimo še z $A_0=\mathcal{S}_O(A)$ (Figure \ref{sl.pod.7.2.8a.pic}). Daljica $AA_0$ je premer krožnice $k$, zato je po izreku \ref{TalesovIzrKroz2} $\angle ACA_0=90^0=\angle ADB$. Obodna kota $CBA$ in $CA_0A$ nad tetivo $AC$ krožnice $k$ sta skladna (izrek \ref{ObodObodKot}), oz. $\angle DBA=\angle CBA\cong\angle CA_0A$. Po izreku \ref{PodTrikKKK} je $\triangle ABD\sim\triangle AA_0C$, zato je:
$$\frac{AB}{AA_0}=\frac{AD}{AC},$$
oz.
$bc=2R\cdot v_a$.
\kdokaz



            \bzgled
            Suppose that the bisector of the interior angle $BA$C of a triangle $ABC$ intersects
            its side $BC$ at the point $E$ and the circumcircle of this triangle at the point $N$
            ($N\neq A$). Let's denote $b=|AC|$, $c=|AB|$ and $l_a=|AE|$. Prove that:
            $$|AN|=\frac{bc}{l_a}.$$
            \ezgled

\begin{figure}[!htb]
\centering
\input{sl.pod.7.2.8.pic}
\caption{} \label{sl.pod.7.2.8.pic}
\end{figure}

 \textbf{\textit{Proof.}} Označimo še z $AD$ višino ($v_a=|AD|$), s $k(O,R)$ očrtano krožnico trikotnika $ABC$ in $M=\mathcal{S}_O(N)$ (Figure \ref{sl.pod.7.2.8.pic}). Po izreku \ref{TockaN} leži točka $N$ na simetrali stranice $BC$, kar pomeni, da je $NM$ premer krožnice $k$ in $MN\perp BC$. Po izreku \ref{TalesovIzrKroz2} je $\angle NAM=90^0=\angle EDA$. Iz $AD,MN\perp BC$ sledi $AD\parallel MN$, zato je po izreku \ref{KotiTransverzala} $\angle DAE\cong\angle MNA$. Trikotnika $ADE$ in $NAM$ sta si torej podobna (izrek \ref{PodTrikKKK}), zato je:
$$\frac{AD}{AN}=\frac{AE}{NM}.$$
Če uporabimo prejšnji zgled \ref{izrekSinusni}, dobimo:
 $$|AN|=\frac{2R\cdot v_a}{l_a}=\frac{bc}{l_a},$$ kar je bilo treba dokazati. \kdokaz



%________________________________________________________________________________
 \poglavje{The Theorems of Ceva and Menelaus} \label{odd7MenelCeva}


 Na tem mestu bomo obravnavali t. i. \index{izrek!izreka dvojčka}\pojem{izreka dvojčka}.




            \bizrek \label{izrekCeva}\index{izrek!Cevov}
         (Ceva\footnote{\index{Ceva, G.} \textit{G. Ceva} (1648--1734), italijanski matematik, ki
         je ta izrek dokazal leta 1678.})
         Let $P$, $Q$ and $R$ be points lying on the lines containing the sides $BC$, $CA$ and $AB$
          of a triangle $ABC$. Then the lines $AP$, $BQ$ and $CR$ belong
        to the same family of lines if and only if:
        \begin{eqnarray}\label{formulaCeva}
         \frac{\overrightarrow{BP}}{\overrightarrow{PC}}\cdot
        \frac{\overrightarrow{CQ}}{\overrightarrow{QA}}\cdot
         \frac{\overrightarrow{AR}}{\overrightarrow{RB}}=1.
         \end{eqnarray}
        \eizrek

\begin{figure}[!htb]
\centering
\input{sl.pod.7.5.1.pic}
\caption{} \label{sl.pod.7.5.1.pic}
\end{figure}

%slikaNova1-3-1
%\includegraphics[width=100mm]{slikaNova1-3-1.pdf}

 \textbf{\textit{Proof.}}
($\Rightarrow$) Predpostavimo, da premice $AP$, $BQ$ in $CR$
pripadajo istemu eliptičnemu šopu, torej da se sekajo v neki
točki $S$ (bralcu bomo pustili, da dokaže izrek v primeru, če
so premice vzporedne). Premico, ki vsebuje točko $A$ in je
vzporedna s premico $BC$, označimo s $p$, njena presečišča s
premicama $BQ$ in $CR$ označimo s $K$ in $L$ (Figure
\ref{sl.pod.7.5.1.pic}). Sedaj iz Talesovega izreka \ref{TalesovIzrek} sledi
 $\frac{\overrightarrow{BP}}{\overrightarrow{KA}}=
  \frac{\overrightarrow{SP}}{\overrightarrow{SA}}=
  \frac{\overrightarrow{PC}}{\overrightarrow{AL}}.$
Iz tega dobimo:
 \begin{eqnarray*}
  \frac{\overrightarrow{BP}}{\overrightarrow{PC}}=
  \frac{\overrightarrow{KA}}{\overrightarrow{AL}}.
 \end{eqnarray*}
 Analogno je tudi:
 \begin{eqnarray*}
  \frac{\overrightarrow{CQ}}{\overrightarrow{QA}}=
  -\frac{\overrightarrow{QC}}{\overrightarrow{QA}}=
  -\frac{\overrightarrow{CB}}{\overrightarrow{AK}}=
  \frac{\overrightarrow{CB}}{\overrightarrow{KA}},\\
 \frac{\overrightarrow{AR}}{\overrightarrow{RB}}=
  -\frac{\overrightarrow{RA}}{\overrightarrow{RB}}=
  -\frac{\overrightarrow{AL}}{\overrightarrow{BC}}=
  \frac{\overrightarrow{AL}}{\overrightarrow{CB}},
 \end{eqnarray*}
 kar pomeni, da velja:
 $$\frac{\overrightarrow{BP}}{\overrightarrow{PC}}\cdot
  \frac{\overrightarrow{CQ}}{\overrightarrow{QA}}\cdot
  \frac{\overrightarrow{AR}}{\overrightarrow{RB}}=
 \frac{\overrightarrow{KA}}{\overrightarrow{AL}}\cdot
  \frac{\overrightarrow{CB}}{\overrightarrow{KA}}\cdot
  \frac{\overrightarrow{AL}}{\overrightarrow{CB}}=1.$$
  ($\Leftarrow$) Predpostavimo sedaj, da velja relacija
  (\ref{formulaCeva}) in da je
   $S$ presečišče premic $BQ$ in $CR$ (primer, ko sta $BQ$ in $CR$ vzporedni,
    spet prepuščamo bralcu). Če sedaj s $P'$ označimo presečišče premice $AS$
     s premico $BC$, iz prvega dela dokaza sledi (Figure \ref{sl.pod.7.5.1.pic}):
     $$\frac{\overrightarrow{BP'}}{\overrightarrow{P'C}}\cdot
  \frac{\overrightarrow{CQ}}{\overrightarrow{QA}}\cdot
  \frac{\overrightarrow{AR}}{\overrightarrow{RB}}=1.$$
  Po predpostavki (\ref{formulaCeva}) je potem
  $\frac{\overrightarrow{BP}}{\overrightarrow{PC}}=
  \frac{\overrightarrow{BP'}}{\overrightarrow{P'C}}$   oz. $P=P'$ (izrek \ref{izrekEnaDelitevDaljiceVekt}).
   Torej se premice $AP$, $BQ$, $CR$ sekajo v točki $S$.
  \kdokaz

 Opazimo, da izrek velja v primeru, ko se premice $AP$, $BQ$ in $CR$
  sekajo v eni točki (eliptični šop), kot tudi v primeru, ko so te premice
  vzporedne (parabolični šop). Zdi se, da bi izrek laže dokazali,
  če bi vpeljali
   točke v neskončnosti in s tem pojem (eliptičnega) šopa posplošili
  tudi na primer, ko gre za vzporedne premice. To so ideje, ki so pripeljale do razvoja t. i. \index{geometrija!projektivna}\pojem{projektivne geometrije}.




            \bizrek \index{izrek!Menelajev}\label{izrekMenelaj}
            (Menelaj\footnote{\index{Menelaj} \textit{Starogrški matematik Menelaj iz Aleksandrije} (1. st.)
             je ta izrek dokazal v svojem delu \textit{Sferika}, in sicer
             v primeru sferičnih trikotnikov. Trditev v primeru trikotnikov v ravnini
            omenja kot že znano. Ker prejšni dokumenti o tem niso ohranjeni, izrek
            po njem imenujemo Menelajev izrek. Zaradi svoje podobnosti se Menelajev
            in Cevov izrek imenujeta tudi \textit{izreka dvojčka}. Obdobje 1500 let, ki
            loči ti
             dve odkritji, priča o tem, kako se je v tem
             obdobju geometrija počasi razvijala.})
             Let $P$, $Q$ and $R$ be points lying on the lines containing the sides $BC$, $CA$ and $AB$
             of a triangle $ABC$. Then the points $P$, $Q$ and $R$ are collinear if and only if:
            \begin{eqnarray}\label{formulaMenelaj}
             \frac{\overrightarrow{BP}}{\overrightarrow{PC}}\cdot
            \frac{\overrightarrow{CQ}}{\overrightarrow{QA}}\cdot
             \frac{\overrightarrow{AR}}{\overrightarrow{RB}}=-1.
             \end{eqnarray}
            \eizrek

\begin{figure}[!htb]
\centering
\input{sl.pod.7.5.2.pic}
\caption{} \label{sl.pod.7.5.2.pic}
\end{figure}


 \textbf{\textit{Proof.}}
 ($\Rightarrow$) Naj bodo $P$, $Q$ in $R$ točke neke premice $l$.
 Z $A'$, $B'$ in $C'$ označimo nožišča pravokotnic iz oglišč
  $A$, $B$ in $C$ trikotnika na premico $l$ (Figure \ref{sl.pod.7.5.2.pic}).
  Z uporabo Talesovega izreka (\ref{TalesovIzrek}) dobimo:
 \begin{eqnarray*}
 \frac{\overrightarrow{BP}}{\overrightarrow{PC}}\cdot
  \frac{\overrightarrow{CQ}}{\overrightarrow{QA}}\cdot
  \frac{\overrightarrow{AR}}{\overrightarrow{RB}}=
  -\frac{\overrightarrow{PB}}{\overrightarrow{PC}}\cdot
  \frac{\overrightarrow{QC}}{\overrightarrow{QA}}\cdot
  \frac{\overrightarrow{RA}}{\overrightarrow{RB}}=
  -\frac{\overrightarrow{BB'}}{\overrightarrow{CC'}}\cdot
  \frac{\overrightarrow{CC'}}{\overrightarrow{AA'}}\cdot
  \frac{\overrightarrow{AA'}}{\overrightarrow{BB'}}=
  -1.
   \end{eqnarray*}
 ($\Leftarrow$) Naj bo sedaj izpolnjena relacija (\ref{formulaMenelaj}). S $P'$
  označimo presečišče premic $QR$ in $BC$. Če bi bili premici $QR$ in $BC$
   vzporedni, bi iz tega po Talesovem izreku sledilo
  $\frac{\overrightarrow{CQ}}{\overrightarrow{QA}}=
  \frac{\overrightarrow{RB}}{\overrightarrow{AR}}$
   in nato iz (\ref{formulaMenelaj}) še
   $\frac{\overrightarrow{BP}}{\overrightarrow{PC}}=-1$ oz.
   $\overrightarrow{BP}=\overrightarrow{CP}$, kar
   ni mogoče. Naj bo torej $P'=QR\cap BC$. Tedaj so točke
   $P'$, $Q$, in $R$ kolinearne. Po dokazanem v prvem delu
   izreka
  imamo:
 \begin{eqnarray*}
  \frac{\overrightarrow{BP'}}{\overrightarrow{P'C}}\cdot
  \frac{\overrightarrow{CQ}}{\overrightarrow{QA}}\cdot
  \frac{\overrightarrow{AR}}{\overrightarrow{RB}}=-1.
  \end{eqnarray*}
  Iz tega in predpostavljene relacije (\ref{formulaMenelaj}) sledi
  $\frac{\overrightarrow{BP}}{\overrightarrow{PC}}=
  \frac{\overrightarrow{BP'}}{\overrightarrow{P'C}}$   oz. $P=P'$ (izrek \ref{izrekEnaDelitevDaljiceVekt}),
   kar pomeni, da so točke $P$, $Q$ in $R$ kolinearne.
 \kdokaz


Nadaljujmo z uporabo dveh dokazanih izrekov.



                \bizrek
                The lines, joining the vertices of a triangle to the tangent points of the incircle, intersect at one point (so-called \index{točka!Gergonova}\pojem{Gergonne\footnote{To trditev je dokazal \index{Gergonne, J. D.}\textit{J. D. Gergonne} (1771--1859), francoski matematik. V 19. stoletju je bila posvečena posebna
                pozornost metričnim lastnostim trikotnika, tako so bile odkrite tudi druge karakteristične točke trikotnika (glej naslednja zgleda).} point} \color{blue} of this triangle).
                \eizrek


\begin{figure}[!htb]
\centering
\input{sl.pod.7.5.3.pic}
\caption{} \label{sl.pod.7.5.3.pic}
\end{figure}


 \textbf{\textit{Proof.}}
Naj bodo $P$, $Q$ in $R$ točke, v katerih se včrtana krožnica trikotnika $ABC$ (glej izrek \ref{SredVcrtaneKrozn}) dotika njegovih
stranic $BC$, $CA$ in $AB$ (Figure \ref{sl.pod.7.5.3.pic}). Dokažimo, da se premice $AP$, $BQ$ in $CR$ sekajo v eni točki. Po izreku \ref{TangOdsek} so skladne
 ustrezne tangentne daljice: $BP\cong BR$, $CP\cong CQ$ in
$AQ\cong AR$. Ker je $\mathcal{B}(B,P,C)$, $\mathcal{B}(C,Q,A)$ in $\mathcal{B}(A,R,B)$, je $\frac{\overrightarrow{BP}}{\overrightarrow{PC}}
\cdot \frac{\overrightarrow{CQ}}{\overrightarrow{QA}}
\cdot \frac{\overrightarrow{AR}}{\overrightarrow{RB}} >0$. Torej:
$$\frac{\overrightarrow{BP}}{\overrightarrow{PC}}
\cdot \frac{\overrightarrow{CQ}}{\overrightarrow{QA}}
\cdot \frac{\overrightarrow{AR}}{\overrightarrow{RB}}=
\frac{|BP|}{|PC|}\cdot\frac{|CQ|}{|QA|}\cdot\frac{|AR|}{|RB|}=1.$$
 Po Cevovem izreku \ref{izrekCeva} premice $AP$, $BQ$ in $CR$ pripadajo enem šopu. Ker se po posledici Pashovega aksioma \ref{PaschIzrek} premici $AP$ in $BQ$ sekata, gre za eliptični šop, kar pomeni, da  se premice $AP$, $BQ$ in $CR$ sekajo v eni točki.
 \kdokaz



                \bzgled
                Prove that the lines, joining the vertices of a triangle to the tangent points of the excircles to the opposite sides, intersect at one point (so-called \index{točka!Nagelova}\pojem{Nagel\footnote{\index{Nagel, C. H.}\textit{C. H. Nagel} (1803--1882), nemški matematik, ki je objavil to trditev leta 1836. Vsaka od premic v tej trditvi deli obseg trikotnika na dva enaka dela, zato to trditev imenujemo tudi \index{izrek!o polobsegu trikotnika}\pojem{izrek o polobsegu trikotnika}.}
                point} \color{green1} of this triangle).
                \ezgled

\begin{figure}[!htb]
\centering
\input{sl.pod.7.5.6.pic}
\caption{} \label{sl.pod.7.5.6.pic}
\end{figure}


 \textbf{\textit{Proof.}}
 Uporabimo oznake iz velike naloge (\ref{velikaNaloga}). Dokažimo, da se premice $AP_a$, $BQ_b$ in $CR_c$ sekajo v eni točki (Figure \ref{sl.pod.7.5.6.pic}). Če uporabimo dejstvo $\frac{\overrightarrow{BP_a}}{\overrightarrow{P_aC}}
\cdot \frac{\overrightarrow{CQ_b}}{\overrightarrow{Q_bA}}
\cdot \frac{\overrightarrow{AR_c}}{\overrightarrow{R_cB}} >0$ in relacije iz velike naloge (\ref{velikaNaloga}), dobimo:
\begin{eqnarray*}
\frac{\overrightarrow{BP_a}}{\overrightarrow{P_aC}}
\cdot \frac{\overrightarrow{CQ_b}}{\overrightarrow{Q_bA}}
\cdot \frac{\overrightarrow{AR_c}}{\overrightarrow{R_cB}}&=&
\frac{|BP_a|}{|P_aC|}\cdot\frac{|CQ_b|}{|Q_bA|}\cdot\frac{|AR_c|}{|R_cB|}=\\&=&
\frac{s-c}{s-b}\cdot\frac{s-a}{s-c}\cdot\frac{s-b}{s-a}=
1
\end{eqnarray*}
Po Cevovem izreku \ref{izrekCeva} premice $AP_a$, $BQ_b$ in $CR_c$ pripadajo enemu šopu. Ker se po posledici Pashovega aksioma \ref{PaschIzrek} premici $AP_a$ in $BQ_b$ sekata, gre za eliptični šop, kar pomeni, da  se premice $AP_a$, $BQ_b$ in $CR_c$ sekajo v eni točki.
 \kdokaz



                \bzgled
                Prove that the lines, joining the vertices of a triangle to the
                 points dividing opposite sides in the ratio of squares of adjacent sides,
                  intersect at one point (so-called \index{točka!Lemoinova}
                \pojem{Lemoine\footnote{\index{Lemoine, E. M. H.}\textit{E. M. H. Lemoine} (1751--1816), francoski matematik.} point}  \color{green1}  of this triangle).
                \ezgled



\begin{figure}[!htb]
\centering
\input{sl.pod.7.5.5.pic}
\caption{} \label{sl.pod.7.5.5.pic}
\end{figure}


 \textbf{\textit{Proof.}} Naj bodo $X$, $Y$ in $Z$ takšne točke stranic $BC$, $AC$ in $AB$ trikotnika $ABC$, da velja: $\frac{|BX|}{|XC|}=\frac{|BA|^2}{|AC|^2}$, $\frac{|CY|}{|YA|}=\frac{|CB|^2}{|BA|^2}$ in $\frac{|AZ|}{|ZB|}=\frac{|AC|^2}{|CB|^2}$
 (Figure \ref{sl.pod.7.5.5.pic}).
 Ker je $\frac{\overrightarrow{BX}}{\overrightarrow{XC}}
\cdot \frac{\overrightarrow{CY}}{\overrightarrow{YA}}
\cdot \frac{\overrightarrow{AZ}}{\overrightarrow{ZB}} >0$, je:
\begin{eqnarray*}
\frac{\overrightarrow{BX}}{\overrightarrow{XC}}
\cdot \frac{\overrightarrow{CY}}{\overrightarrow{YA}}
\cdot \frac{\overrightarrow{AZ}}{\overrightarrow{ZB}}&=&
\frac{|BX|}{|XC|}\cdot\frac{|CY|}{|YA|}\cdot\frac{|AZ|}{|ZB|}=\\
&=&
\frac{|BA|^2}{|AC|^2}\cdot\frac{|CB|^2}{|BA|^2}\cdot\frac{|AC|^2}{|CB|^2}=
1.
\end{eqnarray*}
 Po Cevovem izreku \ref{izrekCeva} premice $AX$, $BY$ in $CZ$ pripadajo enem šopu. Ker se po posledici Pashovega aksioma \ref{PaschIzrek} premici $AX$ in $BY$ sekata, gre za eliptični šop, kar pomeni, da  se premice $AX$, $BY$ in $CZ$ sekajo v eni točki.
 \kdokaz



                \bzgled
                Let $Z$ and $Y$ be points of the sides $AB$ and $AC$ of a triangle
                $ABC$ such that $AZ:ZB=1:3$ and $AY:YC=1:2$, and $X$ the point,
                in which the line $YZ$ intersects the line containing the side $BC$ of this triangle. Calculate
                 $\overrightarrow{BX}:\overrightarrow{XC}$.
                \ezgled


\begin{figure}[!htb]
\centering
\input{sl.pod.7.5.4.pic}
\caption{} \label{sl.pod.7.5.4.pic}
\end{figure}


 \textbf{\textit{Proof.}}
 (Figure \ref{sl.pod.7.5.4.pic})

Ker so točke $X$, $Y$ in $Z$ kolinearne, po Menelajevem izreku \ref{izrekMenelaj} sledi:
$$\frac{\overrightarrow{BX}}{\overrightarrow{XC}}
\cdot\frac{\overrightarrow{CY}}{\overrightarrow{YA}}
\cdot\frac{\overrightarrow{AZ}}{\overrightarrow{ZB}}=-1.$$
Če uporabimo dane pogoje, dobimo:
$$\frac{\overrightarrow{BX}}{\overrightarrow{XC}}
\cdot\frac{2}{1}
\cdot\frac{1}{3}=-1$$
oziroma $\overrightarrow{BX}:\overrightarrow{XC}=-3:2$.
\kdokaz



                \bzgled
                Prove that the tangents of the circumcircle of an arbitrary scalene triangle $ABC$ at its
                vertices $A$, $B$ and $C$ intersect the lines containing the opposite sides of this triangle
                at three collinear points\footnote{Izrek je poseben primer Pascalovega izreka \ref{izrekPascalEvkl} (razdelek \ref{odd7PappusPascal}). \index{Pascal, B.} \textit{B. Pascal} (1623--1662), francoski matematik in filozof.}.
                \ezgled


\begin{figure}[!htb]
\centering
\input{sl.pod.7.12.2a.pic}
\caption{} \label{sl.pod.7.12.2a.pic}
\end{figure}

\textbf{\textit{Solution.}}
 Naj bodo $X$, $Y$ in $Z$ presečišča
 tangent očrtane krožnice $k$ trikotnika $ABC$ v ogliščih $A$, $B$ in $C$
 z nosilkami stranic $BC$, $AC$ in $AB$ tega
 trikotnika (Figure \ref{sl.pod.7.12.2a.pic}). Dokažimo, da so $X$, $Y$ in $Z$ kolinearne točke. Po izreku \ref{ObodKotTang} je $\angle XAB\cong\angle ACB$. Ker imata trikotnika $AXB$ in $CXA$ še skupni notranji kot ob oglišču $X$, je $\triangle AXB\sim \triangle CXA$ (izrek \ref{PodTrikKKK}), zato je:
 $$\frac{|AX|}{|CX|}=\frac{|AB|}{|AC|}=\frac{|BX|}{|AX|}.$$
Iz $\frac{|AX|}{|CX|}=\frac{|BX|}{|AX|}$ sledi $|XB|\cdot |XC|=|XA|^2$. Če slednjo relacijo delimo z $|XC|^2$, dobimo $\frac{|BX|}{|CX|}=\frac{|AX|^2}{|CX|^2}$. Ker je še $\frac{|AX|}{|CX|}=\frac{|AB|}{|AC|}$, na koncu dobimo:
 \begin{eqnarray*}
&& \frac{|BX|}{|XC|}=\frac{|BA|^2}{|AC|^2}
 \end{eqnarray*}
 in podobno
 \begin{eqnarray*}
&& \frac{|CY|}{|YA|}=\frac{|CB|^2}{|BA|^2},\\
&& \frac{|AZ|}{|ZB|}=\frac{|AC|^2}{|CB|^2}.
 \end{eqnarray*}
 Ker točke $X$, $Y$ in $Z$ ne ležijo na stranicah $BC$, $AC$ in $AB$, je $\frac{\overrightarrow{BX}}{\overrightarrow{XC}}
\cdot \frac{\overrightarrow{CY}}{\overrightarrow{YA}}
\cdot \frac{\overrightarrow{AZ}}{\overrightarrow{ZB}} <0$, zato je:
\begin{eqnarray*}
\frac{\overrightarrow{BX}}{\overrightarrow{XC}}
\cdot \frac{\overrightarrow{CY}}{\overrightarrow{YA}}
\cdot \frac{\overrightarrow{AZ}}{\overrightarrow{ZB}}&=&
-\frac{|BX|}{|XC|}\cdot\frac{|CY|}{|YA|}\cdot\frac{|AZ|}{|ZB|}=\\
&=&
-\frac{|BA|^2}{|AC|^2}\cdot\frac{|CB|^2}{|BA|^2}\cdot\frac{|AC|^2}{|CB|^2}=
-1.
  \end{eqnarray*}
  Po Menelajevem izreku \ref{izrekMenelaj} so točke $X$, $Y$ in $Z$ kolinearne.
  \kdokaz

      Naslednja lastnost štirikotnikov je nadaljevanje
      zgledov
       \ref{TetivniVcrtana} in \ref{TetivniVisinska}. Gre namreč za sledeči problem.
       Dan je tetivni štirikotnik $ABCD$. Kaj predstavlja štirikotnik, ki ima za  oglišča določeno značilno točko trikotnikov $BCD$, $ACD$, $ABD$ in $ABC$? Odgovor je trivialen, če gre za središče očrtane krožnice, saj je $ABCD$ tetivni štirikotnik, zato so v tem primeru vsa središča očrtanih krožnic ena točka in iskanega štirikotnika ni. Primera središč včrtanih krožnice in višinskih točk sta obravnavana v omenjenih zgledih \ref{TetivniVcrtana} in \ref{TetivniVisinska}. Naslednji zgled pa nam da odgovor za primer težišč in sicer v primeru poljubnih štirikotnikov.




            \bzgled \label{TetivniTezisce}
              Let $ABCD$ be an arbitrary quadrilateral and
            $T_A$ the centroid of the triangle $BCD$,
            $T_B$ the centroid of the triangle $ACD$,
            $T_C$ the centroid of the triangle $ABD$ and
            $T_D$ the centroid of the triangle $ABC$.
            Prove that:\\
              a) a) the lines $AT_A$, $BT_B$, $CT_C$ and $DT_D$ intersect at one point which is the common centroid
             of the quadrilaterals $ABCD$ and $T_AT_BT_CT_D$,\\
                b) the quadrilateral $T_AT_BT_CT_D$ is similar to the quadrilateral $ABCD$ with the coefficient
            of similarity equal to $\frac{1}{3}$.
            \ezgled

\begin{figure}[!htb]
\centering
\input{sl.pod.7.2.9.pic}
\caption{} \label{sl.pod.7.2.9.pic}
\end{figure}

 \textbf{\textit{Proof.}} (Figure \ref{sl.pod.7.2.9.pic})

\textit{a)}
Označimo s $P$, $K$, $Q$, $L$, $M$ in $N$ središča daljic $AB$, $BC$, $CD$, $DA$, $AC$ in $BD$.
Ker je točka $T_A$ težišče trikotnika $BCD$, je presečišče njegovih težiščnic $BQ$, $CN$ in $DK$, pri tem še velja $QT_A:T_AB=1:2$ (izrek \ref{tezisce}) oz. $\overrightarrow{QT_A}=\frac{1}{3}\overrightarrow{QB}$. Analogno je iz trikotnika $ACD$ $\overrightarrow{QT_B}=\frac{1}{3}\overrightarrow{QA}$. Iz slednjih dveh relacij in izreka \ref{vektVektorskiProstor} dobimo: $$\overrightarrow{T_BT_A}=\overrightarrow{T_BQ}+\overrightarrow{QT_A}
=\frac{1}{3}\overrightarrow{AQ}+\frac{1}{3}\overrightarrow{QB}=
\frac{1}{3}\left(\overrightarrow{AQ}+\overrightarrow{QB} \right)=\frac{1}{3}\overrightarrow{AB}.$$
Torej:
\begin{eqnarray} \label{eqnCevaTez1}
\overrightarrow{T_BT_A}=\frac{1}{3}\overrightarrow{AB},
\end{eqnarray}
zato je tudi $T_BT_A\parallel AB$. Označimo s $T$ presečišče daljic $AT_A$ in $BT_B$. Po Talesovem izreku in \ref{eqnCevaTez1} je:
\begin{eqnarray} \label{eqnCevaTez2}
\frac{\overrightarrow{TT_A}}{\overrightarrow{TA}}
=\frac{\overrightarrow{TT_B}}{\overrightarrow{TB}}=
\frac{\overrightarrow{T_AT_B}}{\overrightarrow{AB}}=-\frac{1}{3}.
\end{eqnarray}
Daljici $AT_A$ in $BT_B$ se torej sekata v točki $T$, ki ju deli v razmerju $2:1$. Analogno se tudi daljici $AT_A$ in $CT_C$ oz.  daljici $AT_A$ in $DT_D$ sekata v točki, ki ju deli v razmerju $2:1$, to pa je (zaradi daljice $AT_A$) ravno točka $T$. To pomeni, da se  premice $AT_A$, $BT_B$, $CT_C$ in $DT_D$ sekajo v točki $T$.

Dokažimo še, da je točka $T$ skupno težišče štirikotnikov $ABCD$ in
          $T_AT_BT_CT_D$.
Ker je
$$\frac{\overrightarrow{QT_B}}{\overrightarrow{T_BA}}\cdot
\frac{\overrightarrow{AP}}{\overrightarrow{PB}}\cdot
\frac{\overrightarrow{BT_A}}{\overrightarrow{T_AQ}}=
\frac{1}{2}\cdot\frac{1}{1}\cdot\frac{2}{1}=1,
$$
se po Cevovem izreku \ref{izrekCeva} za trikotnik $QAB$ premice $AT_A$, $BT_B$ in $PQ$ sekajo v eni točki - točki $T$. točka $T$ torej leži na premici $PQ$. Analogno točka $T$ leži na premici $KL$, kar pomeni, da je točka $T$ presečišče diagonal $PQ$ in $KL$ štirikotnika $PKQL$. Ker gre za Varignonov paralelogram štirikotnika $ABCD$, je točka $T$ po izreku \ref{vektVarignon} težišče štirikotnika $ABCD$. Ker je iz \ref{eqnCevaTez2} še:
$$\overrightarrow{TT_A}+\overrightarrow{TT_B}+\overrightarrow{TT_C}+\overrightarrow{TT_D}=
-\frac{1}{3}\cdot\left(
\overrightarrow{TA}+\overrightarrow{TB}+\overrightarrow{TC}+\overrightarrow{TD}
\right)=-\frac{1}{3}\cdot \overrightarrow{0}=\overrightarrow{0},$$
je točka $T$ težišče tudi štirikotnika $T_AT_BT_CT_D$.

\textit{b)} Iz \ref{eqnCevaTez2} sledi $\overrightarrow{TT_A}=-\frac{1}{3}\overrightarrow{TA}$,
$\overrightarrow{TT_B}=-\frac{1}{3}\overrightarrow{TB}$,
$\overrightarrow{TT_C}=-\frac{1}{3}\overrightarrow{TC}$ in
$\overrightarrow{TT_D}=-\frac{1}{3}\overrightarrow{TD}$ oz.
$h_{T,-\frac{1}{3}}:\hspace*{1mm}A,B,C,D\mapsto T_A,T_B,T_C,T_D$, kar pomeni, da je štirikotnik $T_AT_BT_CT_D$ podoben štirikotniku
         $ABCD$ s koeficientom podobnosti $\frac{1}{3}$.
\kdokaz

%________________________________________________________________________________
 \poglavje{Harmonic Conjugate Points. Apollonius Circle}
  \label{odd7Harm}


Ker je zelo pomemben za nadaljevanje, bomo izrek \ref{izrekEnaDelitevDaljiceVekt} iz razdelka \ref{odd5LinKombVekt} zapisali še v drugi obliki.



        \bizrek \label{HarmCetEnaSamaDelitev}
        If $A$ and $B$ are different points on the line $p$ and $\lambda\neq -1$
        an arbitrary real number, then  there is exactly one point $L$ on the line $p$ such that:
        $$ \frac{\overrightarrow{AL}}{\overrightarrow{LB}}=\lambda.$$
        \eizrek


V razdelku \ref{odd5TalesVekt} smo ugotovili, kako lahko daljico $AB$ razdelimo v razmerju $m:n$ (izrek \ref{izrekEnaDelitevDaljice}). To idejo bomo sedaj uporabili za konstrukcijo točke $L$ iz prejšnjega izreka \ref{HarmCetEnaSamaDelitev}.



            \bzgled
            The points $A$ and $B$ and the real number $\lambda\neq -1$ are given.
            Construct a point $L$ on the line $AB$ such that:
        $$\frac{\overrightarrow{AL}}{\overrightarrow{LB}}=\lambda.$$
            \ezgled


\begin{figure}[!htb]
\centering
\input{sl.pod.7.4.1.pic}
\caption{} \label{sl.pod.7.4.1.pic}
\end{figure}

 \textit{\textbf{Rešitev.}}  (Figure \ref{sl.pod.7.4.1.pic})

V primeru $\lambda=0$ je dokaz neposreden - takrat je $L=A$. Naj bo $\lambda\neq 0$.
Konstruirajmo
poljubni vzporednici $a$ in $b$ skozi točki $A$ in $B$. S $P$ in $Q$ označimo  takšne točke na premicah $a$ in $b$,
da velja $P,Q\div AB$ ter $|AP|=|\lambda|$ in $|BQ|=1$. Označimo še $Q'=\mathcal{S}_B(Q)$.
Obravnavali bomo dva primera.

\textit{1)} Naj bo $\lambda>0$. Jasno je, da mora za iskano točko $L$ veljati $\mathcal{B}(A,L,B)$. Z $L_1$ označimo
presečišče premic $AB$ in $PQ$. Po Talesovem izreku \ref{TalesovIzrek} je:
 $$\frac{\overrightarrow{AL_1}}{\overrightarrow{L_1B}}=
\frac{\overrightarrow{AP}}{\overrightarrow{QB}}=\lambda.$$

 \textit{2)} Če je $\lambda<0$, iskana točka ne leži na daljici $AB$. Z $L_1$ označimo
presečišče premic $AB$ in $PQ'$. Ker je $\lambda\neq -1$, to presečišče obstaja. Po Talesovem izreku \ref{TalesovIzrek} je:
 $$\frac{\overrightarrow{AL_2}}{\overrightarrow{L_2B}}=
\frac{\overrightarrow{AP}}{\overrightarrow{Q'B}}
=-\frac{\overrightarrow{AP}}{\overrightarrow{BQ'}}=-\left(-\lambda\right)=\lambda,$$ kar je bilo treba dokazati. \kdokaz

Iz prejšnjega zgleda je torej jasno, da za vsakega od pogojev $\frac{\overrightarrow{AL}}{\overrightarrow{LB}}=\lambda$ ($\lambda>0$) oz. $\frac{\overrightarrow{AL}}{\overrightarrow{LB}}=\lambda$ ($\lambda<0$ in $\lambda\neq -1$) obstaja natanko ena rešitev za točko $L$ na premici $AB$. V prvem primeru točka $L$ leži na daljici $AB$, zato pravimo, da gre za \index{delitev daljice!notranja}\pojem{notranjo delitev} daljice $AB$ v razmerju $\lambda$ ($\lambda>0$). V drugem primeru točka $L$ ne leži na daljici $AB$, zato pravimo, da gre za \index{delitev daljice!zunanja}\pojem{zunanjo delitev} daljice $AB$ v razmerju $|\lambda|$ ($\lambda<0$ in $\lambda\neq -1$).

Lahko rečemo tudi, da za pogoj $\frac{AL}{LB}=\lambda$ ($\lambda>0$, $\lambda\neq 1$) obstajata dve rešitvi  za točko $L$ na premici - notranja oziroma zunanja delitev.

Če sta $L_1$ in $L_2$ notranja in zunanja delitev daljice (za isto $\lambda$), velja:
$$\frac{\overrightarrow{AL_1}}{\overrightarrow{L_1B}}:
\frac{\overrightarrow{AL_2}}{\overrightarrow{L_2B}}=-1.$$
 Če izraz $\frac{\overrightarrow{AL_1}}{\overrightarrow{L_1B}}:
\frac{\overrightarrow{AL_2}}{\overrightarrow{L_2B}}$ definiramo kot \index{dvorazmerje parov točk}\pojem{dvorazmerje parov točk} $(A,B)$ in $(L_1,L_2)$ ter označimo
$$d(A,B;L_1,L_2)=\frac{\overrightarrow{AL_1}}{\overrightarrow{L_1B}}:
\frac{\overrightarrow{AL_2}}{\overrightarrow{L_2B}},$$
 vidimo, da je za notranjo in zunanjo delitev daljice $AB$ s točkama $L_1$ in $L_2$ ustrezno dvorazmerje enako $-1$, oz. $d(A,B;L_1,L_2)=-1$
Tako dobimo idejo za novo definicijo.

Pravimo, da različne kolinearne točke $A$, $B$, $C$ in $D$ določajo
 \index{harmonična četverica točk}
 \pojem{harmonično četverico točk} oz. da je par $(A, B)$,
 \pojem{harmonično konjugiran} s parom  $(C, D)$, oznaka
 $\mathcal{H}(A,B;C,D)$, če je:
  \begin{eqnarray}\label{formulaHarmEvkl}
  \frac{\overrightarrow{AC}}{\overrightarrow{CB}}=
  -\frac{\overrightarrow{AD}}{\overrightarrow{DB}},
 \end{eqnarray} oziroma:
 \begin{eqnarray*}
d(A,B;C,D)=\frac{\overrightarrow{AC}}{\overrightarrow{CB}}:
\frac{\overrightarrow{AD}}{\overrightarrow{DB}}=-1.
 \end{eqnarray*}

Dokažimo osnovne lastnosti definirane relacije.
 Kot smo že omenili, točki $C$ in $D$, za kateri velja $\mathcal{H}(A,B;C,D)$, predstavljata
notranjo in zunanjo delitev daljice $AB$ v nekemu razmerju $\lambda$ ($\lambda>0$ in $\lambda\neq 1$). Takšnih parov $C$, $D$ je torej
neskončno mnogo. Toda, če je dana ena od točk $C$ ali $D$, je druga enolično določena. To povejmo še z drugimi besedami v naslednji trditvi.



        \bizrek \label{HarmCetEnaSamaTockaD}
        Let $C$ be a point on a line $AB$ different from the points $A$ and $B$
        and also from the midpoint of the line segment $AB$,
        then  there is exactly one point $D$ such that
        $\mathcal{H}(A,B;C,D)$.
        \eizrek


 \textit{\textbf{Proof.}} Naj bo $\frac{\overrightarrow{AC}}{\overrightarrow{CB}}=\lambda$. Ker se točka $C$ razlikuje od točk $A$, $B$ in
         središča daljice $AB$, je $\lambda\neq 0$ in $\lambda\neq 1$. Iščemo točko $D$, za katero je $\mathcal{H}(A,B;C,D)$ oz. $d(A,B;C,D)=-1$ ali ekvivalentno $\frac{\overrightarrow{AD}}{\overrightarrow{DB}}=-\lambda$. Ker je $-\lambda\neq -1$, po izreku \ref{HarmCetEnaSamaDelitev} obstaja ena sama točka $D$, za katero je to izpolnjeno.
\kdokaz



        \bizrek \label{HarmCetEF}
       Let $A$, $B$, $C$ and $D$ be four different collinear points
            on a line $p$ and $O$ a point not lying on this line. Suppose that
            a line that is parallel to the line $OA$ through the point $B$
            intersects the lines $OC$ and $OD$ at the points $E$ and $F$.
            Then:
        $$\mathcal{H}(A,B;C,D) \hspace*{1mm} \Leftrightarrow \hspace*{1mm} \mathcal{S}_B(E)=F.$$
        \eizrek


 \textit{\textbf{Proof.}}
  (Figure \ref{sl.pod.7.4.2.pic})

($\Rightarrow$) Če velja $\mathcal{H}(A,B;C,D)$, je po Talesovemu izreku \ref{TalesovIzrek}:
$$\frac{\overrightarrow{AO}}{\overrightarrow{EB}}=
\frac{\overrightarrow{AC}}{\overrightarrow{CB}}=
-\frac{\overrightarrow{AD}}{\overrightarrow{DB}}=
-\frac{\overrightarrow{AO}}{\overrightarrow{FB}}=
\frac{\overrightarrow{AO}}{\overrightarrow{BF}},$$
 zato je
$\overrightarrow{EB}=\overrightarrow{BF}$ oz. $\mathcal{S}_B(E)=F$.

($\Leftarrow$) Naj bo sedaj $\mathcal{S}_B(E)=F$. Iz tega sledi $\overrightarrow{EB}=\overrightarrow{BF}$, zato je (spet po Talesovemu izreku \ref{TalesovIzrek}):
$$\frac{\overrightarrow{AC}}{\overrightarrow{CB}}=
\frac{\overrightarrow{AO}}{\overrightarrow{EB}}=
\frac{\overrightarrow{AO}}{\overrightarrow{BF}}=
-\frac{\overrightarrow{AO}}{\overrightarrow{FB}}=
-\frac{\overrightarrow{AD}}{\overrightarrow{DB}},$$
torej je po definiciji $\mathcal{H}(A,B;C,D)$.
\kdokaz

\begin{figure}[!htb]
\centering
\input{sl.pod.7.4.2.pic}
\caption{} \label{sl.pod.7.4.2.pic}
\end{figure}

Prejšnji izrek nam omogoča efektivno konstrukcijo četrte točke v harmonični četverici točk.



        \bzgled \label{HarmCetEnaSamaTockaDKonstr}
        Let $C$ be a point that lies on a line $AB$. Suppose that $C$ is different from the points $A$,
        $B$ and the midpoint of the line segment $AB$.  Construct a point $D$ such that
        $\mathcal{H}(A,B;C,D)$.
        \ezgled


 \textit{\textbf{Rešitev.}} Po izreku \ref{HarmCetEnaSamaTockaD} obstaja ena sama takšna točka $D$, za katero je $\mathcal{H}(A,B;C,D)$. Sedaj jo bomo tudi
 konstruirali. Naj bo $O$ poljubna točka, ki ne leži na premici $AB$  (Figure \ref{sl.pod.7.4.2.pic}), in $l$ vzporednica premice $AO$ v točki $B$. Z $E$ označimo presečišče premic $OC$ in $l$ in $F=\mathcal{S}_B(E)$. Točko $D$ dobimo kot presečišče premic $OF$ in $AB$.
 Po prejšnjem izreku \ref{HarmCetEF} je $\mathcal{H}(A,B;C,D)$.
\kdokaz

Na podoben način kot v prejšnjem zgledu lahko za dane kolinearne točke $A$, $B$ in $D$ (točka $D$ ne leži na daljici $AB$) načrtamo takšno točko $C$, da velja $\mathcal{H}(A,B;C,D)$.

 Intuitivno je jasno, da iz $\mathcal{H}(A,B;C,D)$ sledi $\mathcal{H}(A,B;D,C)$, kajti če točki $C$ in $D$ delita daljico $AB$ v istem razmerju, enako velja tudi za točki $D$ in $C$. Zanimivo je, da iz $\mathcal{H}(A,B;C,D)$ sledi tudi $\mathcal{H}(C,D;A,B)$, kar pomeni, da če točki $C$ in $D$ delita daljico $AB$ v istem razmerju, potem tudi točki $A$ in $B$ delita daljico $CD$ v istem razmerju. Dokažimo formalno obe lastnosti.

                \bizrek
                  a) $\mathcal{H}(A,B;C,D) \hspace*{1mm} \Rightarrow
                 \hspace*{1mm} \mathcal{H}(A,B;D,C)$; \\
                    \hspace*{22mm}b) $\mathcal{H}(A,B;C,D)
                 \hspace*{1mm} \Rightarrow
                 \hspace*{1mm} \mathcal{H}(C,D;A,B)$;
                \eizrek


 \textit{\textbf{Proof.}}

 $$a)\hspace*{1mm}\mathcal{H}(A,B;C,D)
\hspace*{1mm} \Rightarrow
                 \hspace*{1mm}
 \frac{\overrightarrow{AC}}{\overrightarrow{CB}}= -\frac{\overrightarrow{AD}}{\overrightarrow{DB}}
\hspace*{1mm} \Rightarrow
                 \hspace*{1mm}
\frac{\overrightarrow{AD}}{\overrightarrow{DB}}= -\frac{\overrightarrow{AC}}{\overrightarrow{CB}}
\hspace*{1mm} \Rightarrow
                 \hspace*{1mm}
\mathcal{H}(A,B;D,C).$$


 $$b)\hspace*{1mm}\mathcal{H}(A,B;C,D)
\hspace*{1mm} \Rightarrow
                 \hspace*{1mm}
 \frac{\overrightarrow{AC}}{\overrightarrow{CB}}= -\frac{\overrightarrow{AD}}{\overrightarrow{DB}}
\hspace*{1mm} \Rightarrow
                 \hspace*{1mm}
\frac{\overrightarrow{CA}}{\overrightarrow{AD}}= -\frac{\overrightarrow{CB}}{\overrightarrow{BD}}
\hspace*{1mm} \Rightarrow
                 \hspace*{1mm}
\mathcal{H}(C,D;A,B),$$ kar je bilo treba dokazati. \kdokaz



             \bizrek \label{izrek 1.2.1}
            Let $A$, $B$, $C$ and $D$ be different collinear points.
            Then $\mathcal{H}(A,B;C,D)$ if and only if there exists such a quadrilateral $PQRS$,
            that:
            $$A=PQ\cap RS, \hspace*{2mm}B=QR\cap PS, \hspace*{2mm} C\in PR \hspace*{2mm}
             \textrm{and}
            \hspace*{2mm} D\in QS.$$
            \eizrek


 \textit{\textbf{Proof.}} ($\Rightarrow$) Naj bo $PQRS$ tak štirikotnik,
  da je $A=PQ\cap RS$, $B=QR\cap PS$, $C\in PR$ in $D\in QS$
  (Figure \ref{sl.pod.7.4.7a.pic}). Po Menelajevem izreku
  \ref{izrekMenelaj}   za trikotnik $ABP$ in premico $QS$ dobimo:
 $$\frac{\overrightarrow{AD}}{\overrightarrow{DB}}\cdot
   \frac{\overrightarrow{BS}}{\overrightarrow{SP}}\cdot
   \frac{\overrightarrow{PQ}}{\overrightarrow{QA}}=-1.$$
 Analogno z uporabo Cevovega izreka \ref{izrekCeva} za isti
 trikotnik in točko $R$ dobimo:
 $$\frac{\overrightarrow{AC}}{\overrightarrow{CB}}\cdot
   \frac{\overrightarrow{BS}}{\overrightarrow{SP}}\cdot
   \frac{\overrightarrow{PQ}}{\overrightarrow{QA}}=1.$$
Iz teh dveh relacij sledi
$\frac{\overrightarrow{AC}}{\overrightarrow{CB}}=
-\frac{\overrightarrow{AD}}{\overrightarrow{DB}}$ oz.
$\mathcal{H}(A,B;C,D)$.

\begin{figure}[!htb]
\centering
\input{sl.pod.7.4.7a.pic}
\caption{} \label{sl.pod.7.4.7a.pic}
\end{figure}

 ($\Leftarrow$) Predpostavimo sedaj, da velja
 $\mathcal{H}(A,B;C,D)$. Brez škode za splošnost naj bo točka
 $C$ med točkama $A$ in $B$; v drugih dveh primerih dokaz poteka
 enako. Naj bo $P$ poljubna točka izven premice $AB$ in $Q$
 poljubna točka med točkama $A$ in $P$. Po Paschovem\footnote{\index{Pasch, M.}
 \textit{M. Pasch}
 (1843--1930), nemški matematik, ki je relacijo urejenosti točk vpeljal v svojem delu
  \textit{Predavanja o novejši geometriji} iz leta 1882.} \ref{PaschIzrek}
  aksiomu se premici $PC$
  in $QB$ sekata v neki točki $R$ ter premici $AR$ in $PB$ v neki
  točki $S$, zato se tudi premici $AB$ in $QS$ sekata v neki točki $D_1$ (iz $AB\parallel QS$
   sledi, da je
  $C$ središče daljice $AB$, kar ni možno zaradi
  $\mathcal{H}(A,B;C,D)$). Sedaj iz prvega dela dokaza ($\Rightarrow$)
   sledi $\mathcal{H}(A,B;C,D_1)$. Ker je po predpostavki še
   $\mathcal{H}(A,B;C,D)$, iz enoličnosti četrte točke harmonične
   četverice točk (izrek \ref{HarmCetEnaSamaTockaD})
    sledi $D=D_1$. Torej je $PQRS$ iskani
   štirikotnik.
   \kdokaz

  Opazimo, da je prejšnji izrek smiseln tudi v primeru, ko je $C$
  središče daljice $AB$ in $D$ točka v neskončnosti (Figure \ref{sl.pod.7.4.7a.pic}).

  Izjavo na desni strani ekvivalence iz prejšnjega izreka (obstoj
  ustreznega štirikotnika) bi lahko privzeli kot definicijo
  harmonične četverice točk v evklidski geometriji. V
  projektivni geometriji je ta definicija še bolj naravna, saj v
  takšni definiciji ne uporabljamo metrike.

  Iz prejšnjega izreka sledi tudi, da se relacija harmonične
  četverice ohranja pri t. i. \index{središčna projekcija}\pojem{središčni projekciji} v prostoru. Res, če so $A$,
  $B$, $C$ in $D$ točke, za katere velja $\mathcal{H}(A,B;C,D)$,
  ter $A'$,  $B'$, $C'$ in $D'$ središčne projekcije teh točk,
  tedaj iz obstoja ustreznega štirikotnika za četverico $A$,
  $B$, $C$, $D$ sledi obstoj ustreznega štirikotnika tudi za
  četverico $A'$,  $B'$, $C'$, $D'$. Drugi je središčna
  projekcija prvega štirikotnika, saj središčna projekcija
  ohranja kolinearnost. Seveda bi v prejšnjo obravnavo morali
  vključiti tudi primer, ko so nekatere od središčnih projekcij
  točke v neskončnosti.

  \index{geometrija!projektivna}V tem smislu lahko projektivno geometrijo opišemo kot
  tisto geometrijo, ki se ukvarja z objekti in lastnostmi, ki se ohranjajo
  pri središčni projekciji. Ker je 'biti harmonička četverica
  točk' ena od takšnih lastnosti, je predmet preučevanja v projektivni
  geometriji. Še več - v tej geometriji je ta relacija ena od
  osnovnih pojmov (glej \cite{Mitrovic}).

 Ugotovili smo že, da na premici $AB$ obstajata natanko dve točki, ki delita daljico $AB$ v
razmerju $\lambda>0$, $\lambda\neq 1$ (v primeru $\lambda= 1$ je to le ena točka - središče daljice).  Omenjeni točki predstavljata
notranjo in zunanjo delitev daljice v tem razmerju in skupaj s točkama $A$ in $B$ določata harmonijsko četverico točk. Zastavlja se vprašanje: Kaj predstavlja množico vseh
takšnih točk $X$ v ravnini, da velja $\frac{AX}{XB}=\lambda$ ($\lambda\neq 1$)? Odgovor nam da naslednji izrek.








             \bizrek \label{ApolonijevaKroznica}
               Suppose that $A$ and $B$ are points in the plane and $\lambda>0$, $\lambda\neq 1$
                an arbitrary real number. If $C$ and $D$ are points of the line $AB$ such that:
                $$\frac{\overrightarrow{AC}}{\overrightarrow{CB}}=
                -\frac{\overrightarrow{AD}}{\overrightarrow{DB}}=\lambda,$$
                i.e.  $\mathcal{H}(A,B;C,D)$, then the set of all points $X$ of this plane such that:
                $$\frac{AX}{XB}=\lambda,$$
                is a circle with the diameter $CD$
               (so-called  Apollonius Circle\footnote{\index{Apolonij}
             \textit{Apolonij iz Perge} (3.-- 2. st. pr. n.
            š.), starogrški matematik.}).
             \index{krožnica!Apolonijeva}
             \eizrek

\begin{figure}[!htb]
\centering
\input{sl.pod.7.4.3.pic}
\caption{} \label{sl.pod.7.4.3.pic}
\end{figure}

 \textit{\textbf{Proof.}}
Naj bo $k$ krožnica nad premerom $CD$. Naj bo nato še
$X$ poljubna točka te ravnine in $p$ vzporednica premice $AX$ skozi točko $B$. Z $E$ in $F$ označimo presečišča premic $XC$ in $XD$
s premico $p$ (Figure \ref{sl.pod.7.4.3.pic}). Po izreku \ref{HarmCetEF}, je točka $B$ središče
daljice $EF$, ker je $\mathcal{H}(A,B;C,D)$. Potrebno je dokazati, da velja:
$$\frac{AX}{XB}=\lambda \hspace*{1mm} \Leftrightarrow \hspace*{1mm} X\in k.$$

($\Leftarrow$) Naj bo $X\in k$. Potem je $\angle EXF=\angle CXD=90^0$ in velja $BE\cong BF\cong BX$  (izrek \ref{TalesovIzrKroz2}). Iz tega in iz Talesovega izreka \ref{TalesovIzrek} sledi:
$$\frac{AX}{XB}=\frac{AX}{EB}=\frac{AC}{CB}=\lambda.$$

($\Rightarrow$) Naj bo $\frac{AX}{XB}=\lambda$. Iz tega in iz Talesovega izreka \ref{TalesovIzrek} sledi:
$$\frac{AX}{XB}=\lambda=\frac{AC}{CB}=\frac{AX}{EB},$$
zato je $XB\cong EB\cong BF$. Torej je $EXF$
pravokotni trikotnik (izrek \ref{TalesovIzrKroz2}). Potem  $\angle CXD=\angle EXF=90^0$, kar pomeni, da točka $X$ leži na krožnici $k$ nad
premerom $CD$ (izrek \ref{TalesovIzrKroz2}).
\kdokaz


Za dani točki $A$ in $B$ za različne vrednosti $\lambda\in \mathbb{R}$ ($\lambda>0$, $\lambda\neq 1$) imamo različne Apolonijeve krožnice (Figure \ref{sl.pod.7.4.4a.pic}). Označili jih bomo z $\mathcal{A}_{AB,\lambda}$, če pa vemo za katero daljico gre, lahko tudi krajše $k_{\lambda}$. V primeru $\lambda=1$  dejansko iščemo množico vseh takšnih točk $X$ te ravnine, za katere je:
                $\frac{AX}{XB}=1$, oz. $AX\cong BX$. Takrat iskana množica ne predstavlja krožnice, ampak simetralo $s_{AB}$ daljice $AB$.

\begin{figure}[!htb]
\centering
\input{sl.pod.7.4.4a.pic}
\caption{} \label{sl.pod.7.4.4a.pic}
\end{figure}



            \bzgled
            Points $A$ and $B$ and a line $l$ in the plane are given.
            Construct the point $L$ on the line $l$ such that $LA:LB=5:2$.
            \ezgled

\begin{figure}[!htb]
\centering
\input{sl.pod.7.4.4.pic}
\caption{} \label{sl.pod.7.4.4.pic}
\end{figure}

 \textit{\textbf{Rešitev.}}
 (Figure \ref{sl.pod.7.4.4.pic})

Načrtajmo najprej takšno točko $C$ na daljici $AB$, da velja $AC:CB=5:2$
 (glej zgled \ref{izrekEnaDelitevDaljice}), nato pa četrto točko harmonične četverice  $\mathcal{H}(A,B;C,D)$ (glej zgled \ref{HarmCetEnaSamaTockaDKonstr}). Konstruirajmo Aplolonijevo krožnico $k_{\frac{5}{2}}$ nad premerom $CD$. Točka $L$ je potem tista, za katero velja $L\in k\cap l$.

Po prejšnjem izreku (\ref{ApolonijevaKroznica}) je:
$$\frac{AL}{LB}=\frac{AC}{CB}=\frac{5}{2}.$$

Naloga ima 0, 1 ali 2 rešitvi, odvisno od števila presečišč krožnice $k$ in premice $l$.
\kdokaz



            \bizrek \label{HarmOhranjaVzporProj}
             A parallel projection preserves the relation of a harmonic conjugate points,
              i.e. if $A$, $B$, $C$, $D$ and $A'$, $B'$, $C'$, $D'$ are two quartets
            of collinear points such that $AA'\parallel BB'\parallel CC'\parallel DD'$, then:
            $$\mathcal{H}(A,B;C,D)\hspace*{1mm}\Rightarrow\hspace*{1mm}
            \mathcal{H}(A',B';C',D').$$
            \eizrek

\begin{figure}[!htb]
\centering
\input{sl.pod.7.4.5.pic}
\caption{} \label{sl.pod.7.4.5.pic}
\end{figure}

 \textit{\textbf{Rešitev.}}
 (Figure \ref{sl.pod.7.4.5.pic})

Naj bo $\mathcal{H}(A,B;C,D)$ oz. $\frac{\overrightarrow{AC}}{\overrightarrow{CB}}:
\frac{\overrightarrow{AD}}{\overrightarrow{DB}}=-1$. Po Talesovem izreku \ref{TalesovIzrek} je:
$$\frac{\overrightarrow{A'C'}}{\overrightarrow{C'B'}}:
\frac{\overrightarrow{A'D'}}{\overrightarrow{D'B'}}=
\frac{\overrightarrow{AC}}{\overrightarrow{CB}}:
\frac{\overrightarrow{AD}}{\overrightarrow{DB}}=-1,$$
kar pomeni, da velja tudi $\mathcal{H}(A',B';C',D')$.
\kdokaz

Zanimivo je, da tudi središčna projekcija ohranja relacijo harmonične četverice točk. Zato je omenjena relacija predmet raziskovanja v projektivni geometriji (glej \cite{Mitrovic}).



        \bizrek \label{HarmCetSimKota}
            Suppose that a line $BC$ intersects the bisectors of the interior and
            the exterior angle  at the vertex $A$ of a triangle $ABC$ at points $E$ and $F$. Then:

        a) $BE:CE=BF:CF=AB:AC$,

        b) $\mathcal{H}(B,C;E,F)$.
        \eizrek

\begin{figure}[!htb]
\centering
\input{sl.pod.7.4.6.pic}
\caption{} \label{sl.pod.7.4.6.pic}
\end{figure}

 \textit{\textbf{Rešitev.}} Naj bo $L$ poljubna točka, za katero velja $\mathcal{B}(C,A,L)$.
Označimo z $M$ in $N$ točki, v katerih vzporednica premice $AC$ skozi točko $B$ seka po vrsti
simetrale $AE$ in $AF$ notranjega in zunanjega kota (Figure \ref{sl.pod.7.4.6.pic}).

\textit{a)} Po izreku \ref{KotiTransverzala} je:
 \begin{eqnarray*}
\angle BMA &\cong& \angle CAM=\angle CAE\cong\angle BAE=\angle BAM\\
\angle BNA &\cong& \angle LAN=\angle LAF\cong\angle BAF=\angle BAN
\end{eqnarray*}

Torej sta $AMB$ in $ANB$ enakokraka trikotnika z osnovnicama $AM$ oz. $AN$ (izrek \ref{enakokraki} in velja:
 $$BM\cong BA\cong BN.$$
Iz tega in iz posledice Talesovega izreka \ref{TalesovIzrekDolzine} (ker je $AC\parallel NM$) dobimo:
\begin{eqnarray*}
\frac{BE}{EC}&=& \frac{BM}{AC}=\frac{BA}{AC},\\
\frac{BF}{FC}&=& \frac{BN}{AC}=\frac{BA}{AC}.
\end{eqnarray*}

\textit{b)} Ker je $\mathcal{B}(B,E,C)$ in $\neg\mathcal{B}(B,F,C)$, je
\begin{eqnarray*}
\frac{\overrightarrow{BE}}{\overrightarrow{EC}}&=& \frac{BA}{AC},\\
\frac{\overrightarrow{BF}}{\overrightarrow{FC}}&=& -\frac{BA}{AC}.
\end{eqnarray*}
Torej:
\begin{eqnarray*}
\frac{\overrightarrow{BE}}{\overrightarrow{EC}}=
-\frac{\overrightarrow{BF}}{\overrightarrow{FC}},
\end{eqnarray*}
kar pomeni, da velja $\mathcal{H}(B,C;E,F)$.
\kdokaz

Nadaljevali bomo z uporabo prejšnjega izreka.

            \bzgled \label{HarmTrikZgl1}
            Construct a triangle $ABC$, with given: $a$, $t_a$, $b:c=2:3$.
            \ezgled

\begin{figure}[!htb]
\centering
\input{sl.pod.7.4.8.pic}
\caption{} \label{sl.pod.7.4.8.pic}
\end{figure}

 \textit{\textbf{Rešitev.}}
Naj bo $ABC$ trikotnik, v katerem je stranica $BC\cong a$, težiščnica $AA_1\cong t_a$ in $AC:AB=2:3$ (Figure \ref{sl.pod.7.4.8.pic}). Torej lahko
konstruiramo stranico $BC$ in njeno središče $A_1$.
Iz danih pogojev oglišče $A$ leži na krožnici
$k(A_1,t_a)$ in Apolonijevi krožnici $\mathcal{A}_{BC,\frac{3}{2}}$, ker je $AB:AC=3:2$ (izrek \ref{ApolonijevaKroznica}). Torej  oglišče $A$ dobimo kot eno od presečišč krožnic
$k$ in $\mathcal{A}_{BC,\frac{3}{2}}$.

Čeprav tega dejstva ne potrebujemo pri konstrukciji, omenimo, da
Apolonijeva krožnica $\mathcal{A}_{BC,\frac{3}{2}}$ predstavlja krožnico s premerom $EF$, kjer sta $E$ in $F$ definirani kot v izreku \ref{HarmCetSimKota}.
\kdokaz



            \bzgled
            Construct a parallelogram, with the sides congruent to
            given line segments $a$ and $b$, and the diagonals in the ratio $3:7$.
            \ezgled

\begin{figure}[!htb]
\centering
\input{sl.pod.7.4.9.pic}
\caption{} \label{sl.pod.7.4.9.pic}
\end{figure}

 \textit{\textbf{Rešitev.}}  (Figure \ref{sl.pod.7.4.9.pic})

Naj bo $S$ presečišče diagonal iskanega paralelograma $ABCD$, v katerem je $AB\cong a$, $BC\cong b$ in $AC:BD=3:7$. Označimo še s $P$ središče starnice $AB$.
Ker se pri paralelogramu
diagonali razpolavljata, sta tudi daljici $SA$ in $SB$ v razmerju $2:3$. To pomeni, da lahko najprej konstruiramo
trikotnik $ASB$, podobno kot v prejšnjem primeru \ref{HarmTrikZgl1}:
$AB\cong a$, $SP=\frac{1}{2}b$ in $SA:SB=3:7$.
 \kdokaz





        \bizrek \label{harmVelNal}
                    Suppose that $k(S, r)$ is the incircle and $k_a (S_a, r_a)$ the excircle of a triangle $ABC$
            and $P$ and $P_a$ the touching points of these circles with the aide $BC$.
            Let $A'$ be the foot of the altitude $AA'$ and $E$
            intersection of the bisectors of the
            interior  angle at the vertex $A$ and the side $BC$.
            If $L$ and $L_a$ are foots of the perpendiculars from the points $S$ and $S_a$ on the line $AA'$, then:

        \hspace*{2mm} (i) $\mathcal{H}(A,E;S,Sa)$ \hspace*{2mm} (ii)
        $\mathcal{H}(A,A';L,La)$ \hspace*{2mm} (iii)
        $\mathcal{H}(A',E;P,Pa)$.
         \eizrek


\begin{figure}[!htb]
\centering
\input{sl.pod.7.3.7.pic}
\caption{} \label{sl.pod.7.3.7.pic}
\end{figure}

\textbf{\textit{Proof.}} (Figure \ref{sl.pod.7.3.7.pic})

\textit{(i)}  Premici $CS$ in $CS_a$ sta simetrali
notranjega in zunanjega kota pri oglišču $C$
trikotnika $ACE$, zato po izreku \ref{HarmCetSimKota} velja  $\mathcal{H}(A,E;S,S_a)$.

\textit{(ii)}  Točke $A$, $A'$, $L$ in $L_a$
so pravokotne projekcije točk $A$, $E$, $S$ in $S_a$ na premici $AA'$. Po izreku \ref{HarmOhranjaVzporProj} je $\mathcal{H}(A,A';L,La)$.

\textit{(iii)} Analogno prejšnji trditvi, saj so
točke $A'$, $E$, $P$ in $P_a$
 pravokotne projekcije točk $A$, $E$, $S$ in $S_a$ na premici $BC$.
\kdokaz


            \bzgled Construct a triangle $ABC$, with given:

            \hspace*{4mm} (i) $r$, $a$, $v_a$ \hspace*{5mm} (ii) $v_a$,
             $r$, $b-c$
             \ezgled

\begin{figure}[!htb]
\centering
\input{sl.pod.7.4.10.pic}
\caption{} \label{sl.pod.7.4.10.pic}
\end{figure}

\textbf{\textit{Solution.}}
Uporabimo oznake kot v veliki nalogi \ref{velikaNaloga} in izreku \ref{harmVelNal} (Figure \ref{sl.pod.7.4.10.pic}). V obeh primerih \textit{(i)} in \textit{(ii)} lahko uporabimo dejstvo $\mathcal{H}(A,A';L,La)$ iz izreka \ref{harmVelNal}, ker iz $AA'\cong v_a$ in $LA'\cong r_a$ lahko načrtamo četrto točko v harmonični četverici $\mathcal{H}(A,A';L,La)$. Tako dobimo $r_a\cong A'L_a$.

\textit{(i)} Uporabimo relacijo $RR_a\cong a$ (velika naloga \ref{velikaNaloga}), narišemo krožnici $k(S,SR)$ in $k_a(S_a,R_a)$, nato pa še njune ustrezne tri skupne tangente (zgled \ref{tang2ehkroz}).


\textit{(ii)} Uporabimo relacijo $PP_a\cong b-c$ (velika naloga \ref{velikaNaloga}), narišemo krožnici $k(S,SP)$ in $k_a(S_a,P_a)$, nato pa še njune ustrezne tri skupne tangente.
\kdokaz

V prejšnjem primeru smo videli, kako lahko s pomočjo dokazane relacije $\mathcal{H}(A,A';L,La)$ iz elementov trikotnika $v_a$ in $r$ dobimo $r_a$. Na podoben način iz vsakega znanega para  trojice $(v_a,r,r_a)$ dobimo tretji element. To dejstvo bomo zapisali v obliki $\langle v_a,r,r_a\rangle$.





            \bzgled
            Construct a cyclic quadrilateral $ABCD$, with the sides congruent to
            given line segments $a$, $b$, $c$ in $d$.
            \ezgled


\begin{figure}[!htb]
\centering
\input{sl.pod.7.4.11.pic}
\caption{} \label{sl.pod.7.4.11.pic}
\end{figure}

\textbf{\textit{Solution.}} (Figure \ref{sl.pod.7.4.11.pic})

Naj bo $ABCD$ tetivni štirikotnik s
stranicami, ki so skladne z danimi
daljicami, oz. $AB\cong a$, $BC\cong b$, $CD\cong c$ in $DA\cong d$, $h_{A,k}$ središčni razteg s središčem $A$
in koeficientom $k=\frac{AD}{AB}=\frac{d}{a}$, $\mathcal{R}_{A,\alpha}$ rotacija s
središčem $A$ za orientirani kot $\alpha=\angle BAD$ ter kompozitum:
 $$\rho_{A,k,\alpha}=\mathcal{R}_{A,\alpha}\circ h_{A,k}$$
rotacijski razteg s središčem $A$, koeficientom $k$ in kotom $\alpha$.

Naj bo $B'=h_{A,k}(B)$ in $C'=h_{A,k}(C)$. Ker je $|AB'|=k\cdot |AB|=\frac{|AD|}{|AB|}\cdot |AB|= |AD|$ in $\angle B'AD=\angle BAD= \alpha$, velja $R_{A,\alpha}(B')=D$, zato je tudi $\rho_{A,k,\alpha}(B)=D$. Naj bo $E=R_{A,\alpha}(C')$
oz.  $E=\rho_{A,k,\alpha}(C)$. Potem je trikotnik $ADE$ slika trikotnika $ABC$ pri rotacijskem raztegu $\rho_{A,k,\alpha}$ (ki je transformacija podobnosti), zato sta ta
dva trikotnika podobna s koeficientom podobnosti $k$. Iz tega in iz izreka \ref{TetivniPogoj} sledi:
$$\angle EDC=\angle EDA+\angle ADC=\angle CBA+\angle ADC=180^0,$$
kar pomeni, da so točke $E$, $D$ in $C$ kolinearne. Daljico $ED$ lahko konstruiramo (z uporabo Talesovega izreka \ref{TalesovIzrek}), ker je:
$$ED\cong B'C'=k\cdot BC=\frac{AD}{AB}\cdot BC=\frac{b\cdot d}{a}.$$
Po konstrukciji točk $C$, $D$ in $E$ lahko načrtamo tudi točko $A$, kajti:
$$\frac{AE}{AC}=\frac{AC'}{AC}=k,$$
zato točka $A$ leži na Apolonijevi krožnici $\mathcal{A}_{EC,k}$, prav tako tudi na krožnici $k(D,d)$.
\kdokaz


        \bnaloga\footnote{44. IMO Japan - 2003, Problem 4.}
         $ABCD$ is cyclic. The foot of the perpendicular from $D$ to the
        lines $AB$, $BC$, $CA$ are $P$, $Q$, $R$, respectively. Show that the angle bisectors of
        $\angle ABC$ and $\angle CDA$ meet on the line $AC$ if and only if $RP \cong RQ$.
        \enaloga

\begin{figure}[!htb]
\centering
\input{sl.pod.7.3.IMO1.pic}
\caption{} \label{sl.pod.7.3.IMO1.pic}
\end{figure}

\textbf{\textit{Solution.}} Brez škode za splošnost predpostavimo, da
velja $\mathcal{B}(B,C,Q)$ in $\mathcal{B}(B,P,A)$. Po izreku
\ref{SimpsPrem} so točke $P$, $Q$ in $R$ kolinearne in ležijo na
\index{premica!Simsonova} Simsonovi premici (Figure
\ref{sl.pod.7.3.IMO1.pic}). Iz dokaza tega izreka sledi tudi
 $\angle CDQ\cong \angle ADP$,
 kar pomeni, da sta $CDQ$ in $ADP$ podobna pravokotna trikotnika,
 zato je: $$CD:CQ=AD:AP \textrm{ oz. } CD:AD=CQ:AP.$$
 Če označimo  $P_1=\mathcal{S}_A(P)$, iz prejšnje relacije dobimo:
  \begin{eqnarray} \label{eqn.pod.7.3.IMO1}
  CD:AD=CQ:AP_1.
   \end{eqnarray}
($\Rightarrow$) Predpostavimo najprej, da se simetrali kotov $ABC$
in $CDA$ sekata v točki $E$, ki leži na premici
 $AC$. Po izreku \ref{HarmCetSimKota} je potem:
 $BC:BA=CE:EA=DC:DA$. Iz tega in iz dokazane relacije
 \ref{eqn.pod.7.3.IMO1} sledi $BC:BA=CQ:AP_1$ oz. $BC:CQ=BA:AP_1$.
 Ker velja $\mathcal{B}(B,C,Q)$
 in $\mathcal{B}(B,A,P_1)$, iz prejšnje relacije dobimo
 $BC:BQ=BA:BP_1$. Po obratnem Talesovem izreku
 \ref{TalesovIzrekObr} je $CA\parallel QP_1$ oz. $AR\parallel QP_1$,
  zato je po Talesovem
 izreku \ref{TalesovIzrek} $PR:RQ=PA:AP_1=1:1$ oz. $PR\cong RQ$.

 ($\Leftarrow$) Naj bo sedaj $PR\cong RQ$. Ker je $A$ središče daljice $PP_1$,
  je $RA$ srednjica
 trikotnika $PQP_1$. Iz tega sledi $AR\parallel QP_1$ oz.
 $CA\parallel QP_1$. Po Talesovem izreku je $BC:CQ=BA:AP_1$
 oz. $BC:BA=CQ:AP_1$. Iz dokazane relacije
 \ref{eqn.pod.7.3.IMO1} sedaj dobimo $BC:BA=CD:AD$. Naj bosta $E_1$
 in $E_2$ presečišči simetral
 kotov $ABC$ in $CDA$ s premico $AC$. Če uporabimo
 prejšnjo relacijo in izrek \ref{HarmCetSimKota}, dobimo:\\
 $CE_1:AE_1=BC:BA=CD:AD=CE_2:AE_2$. Ker točki $E_1$ in $E_2$ obe ležita
  na daljici $AC$, je  $E_1=E_2$ (izrek \ref{HarmCetEnaSamaDelitev}), zato se simetrali
 kotov $ABC$ in $CDA$ sekata na premici $AC$.
 \kdokaz


%________________________________________________________________________________
 \poglavje{The Right Triangle Altitude Theorem. Euclid's Theorems}  \label{odd7VisinEvkl}

V algebri in matematični analizi sta znana pojma aritmetične oz. geometrijske sredine dveh (ali več) števil.
Na tem mestu bomo najprej definirali t. i. aritmetično in geometrijsko sredino dveh daljic.
Pravimo, da je $x$ \index{aritmetična sredina daljic} \pojem{aritmetična sredina} daljic $a$ in $b$, če velja:

\begin{eqnarray} \label{eqnVisEvkl1}
|x|=\frac{1}{2}\left( |a|+|b| \right).
\end{eqnarray}
Prav tako je $y$ \index{geometrijska sredina daljic}\pojem{geometrijska (tudi geometrična) sredina} daljic $a$ in $b$, če velja:

\begin{eqnarray} \label{eqnVisEvkl2}
|y|= \sqrt{|a|\cdot |b|}.
\end{eqnarray}

Torej aritmetična oz. geometrijska sredina dveh daljic je daljica z dolžino, ki je enaka aritmetični oz. geometrijski sredini dolžin teh dveh daljic. Relaciji \ref{eqnVisEvkl1} in \ref{eqnVisEvkl2} bomo pogosto pisali v krajši obliki:
\begin{eqnarray*}
x=\frac{1}{2}\left( a+b \right)\hspace*{1mm}  \textrm{ oz. }\hspace*{1mm}
y=\sqrt{ab}.
\end{eqnarray*}

Jasno je, da na zelo enostaven način lahko konstruiramo aritmetično sredino dveh danih daljic. V tem razdelku bomo izpeljali konstrukcijo geometrijske sredine  dveh danih daljic. Dokažimo najprej glavna izreka, ki se nanašata na pravokotne trikotnike.




                \bizrek \index{izrek!višinski}\label{izrekVisinski}
                The altitude on the hypotenuse of a right-angled triangle
                 is the geometric mean
                 of the line segments into which it divides the hypotenuse.\\
                (The right triangle altitude theorem)
                \eizrek

\begin{figure}[!htb]
\centering
\input{sl.pod.7.6E.1.pic}
\caption{} \label{sl.pod.7.6E.1.pic}
\end{figure}

\textbf{\textit{Proof.}}
Naj bo $ABC$ pravokotni trikotnik s
hipotenuzo $AB$ in z višino $CC'$. Označimo $v_c=|CC'|$, $b_1=|AC'|$ in $a_1=|BC'|$ (Figure \ref{sl.pod.7.6E.1.pic}).

Najprej je $\angle ACC'=90^0-\angle CAC'=\angle CBC'$ in
 $\angle BCC'=90^0-\angle CBC'=\angle CAC'$.
Iz podobnosti pravokotnih trikotnikov $CC'A$ in $BC'C$ (izrek \ref{PodTrikKKK}) je:
$$\frac{CC'}{BC'}=\frac{C'A}{C'C},$$
torej:
\begin{eqnarray} \label{eqnVisinski}
v_c^2=a_1\cdot b_1.
\end{eqnarray}
\kdokaz



                \bizrek \index{izrek!Evklidov}\label{izrekEvklidov}
                Each leg of a right-angled triangle
                 is the geometric mean
                  of the hypotenuse and the line segment of the hypotenuse adjacent to the leg.\\
                (Euclid's\footnote{Starogrški filozof in matematik \index{Evklid}\textit{Evklid iz Aleksandrije} (3. st. pr. n. š.).} theorems)
                \eizrek

\begin{figure}[!htb]
\centering
\input{sl.pod.7.6E.2.pic}
\caption{} \label{sl.pod.7.6E.2.pic}
\end{figure}

\textbf{\textit{Proof.}}
Naj bo $ABC$ pravokotni trikotnik s
hipotenuzo $AB$ in višino $CC'$. Označimo $a=|BC|$,  $b=|AC|$, $c=|AB|$,  $b_1=|AC'|$ in $a_1=|BC'|$ (Figure \ref{sl.pod.7.6E.2.pic}).

Enako kot v dokazu izreka \ref{izrekVisinski} je $\angle ACC'=90^0-\angle CAC'=\angle CBA$ in
 $\angle CAC'=\angle CAB$.
Iz podobnosti pravokotnih trikotnikov $CC'A$ in $BCA$ (izrek \ref{PodTrikKKK}) je:
$$\frac{CA}{BA}=\frac{C'A}{CA},$$
oziroma:
\begin{eqnarray} \label{eqnEvklidov1}
b^2=b_1\cdot c.
\end{eqnarray}
Analogno je tudi:
\begin{eqnarray} \label{eqnEvklidov2}
a^2=a_1\cdot c.
\end{eqnarray}
\kdokaz



                \bzgled \label{EvklVisPosl}
                If $c$ is the length of the hypotenuse, $a$ and $b$ the lengths of the legs and $v_c$
                the length of the altitude on the hypotenuse of a right-angled triangle $ABC$, then
                $$c\cdot v_c=a\cdot b.$$
                \ezgled


\textbf{\textit{Proof.}}
 Po množenju relacij \ref{eqnEvklidov1} in \ref{eqnEvklidov2} iz evklidovega izreka \ref{izrekEvklidov} ter vstavljanju relacije \ref{eqnVisinski} iz višinskega izreka \ref{izrekVisinski}  dobimo:
 $$a^2b^2=a_1b_1 c^2=v^2c^2,$$
 iz tega pa sledi $ab=cv_c$.
 \kdokaz

 Sedaj bomo izpeljali napovedano konstrukcijo.



                \bzgled
                Construct a line segment $x$ that is the geometric mean of given line segments
                $a$ and $b$, i.e. $x=\sqrt{ab}$.
                \ezgled


\textbf{\textit{Proof.}} Konstrukcijo bomo izpeljali na dva načina.

\begin{figure}[!htb]
\centering
\input{sl.pod.7.6E.3.pic}
\caption{} \label{sl.pod.7.6E.3.pic}
\end{figure}

\textit{1)}
 Načrtajmo najprej takšne točke $P$, $Q$ in $R$, da velja $PQ\cong a$, $QR\cong b$ in $\mathcal{B}(P,Q,R)$ (Figure \ref{sl.pod.7.6E.3.pic}), nato polkrožnico $k$ nad premerom $PR$ in na koncu točko $X$ kot presečišče polkrožnice $k$ s pravokotnico premice $PR$ v točki $Q$.

 Dokažimo, da je $x=QX$ iskana daljica. Po izreku \ref{TalesovIzrKroz2} je $PXR$ pravokotni trikotnik s hipotenuzo $PR$, $XQ$ je po konstrukciji njegova višina na tej hipotenuzi. Po višinskem izreku \ref{izrekVisinski} je $x$ geometrijska sredina daljic $a$ in $b$.

\textit{2)} Brez škode za splošnost naj bo $a>b$ (če je $a=b$ je tudi $x=a$).


\begin{figure}[!htb]
\centering
\input{sl.pod.7.6E.4.pic}
\caption{} \label{sl.pod.7.6E.4.pic}
\end{figure}

 Načrtajmo najprej takšne točke $P$, $Q$ in $R$, da velja $PQ\cong a$, $QR\cong b$ in $\mathcal{B}(P,R,Q)$ (Figure \ref{sl.pod.7.6E.4.pic}), nato polkrožnico $k$ nad premerom $PQ$ in na koncu točko $X$ kot presečišče polkrožnice $k$ s pravokotnico premice $PQ$ v točki $R$.


 Dokažimo, da je $x=QX$ iskana daljica. Po izreku \ref{TalesovIzrKroz2} je $PXQ$ pravokotni trikotnik s hipotenuzo $PQ$, $XR$ je po konstrukciji njegova višina na to hipotenuzo. Po Evklidovem izreku \ref{izrekEvklidov} je $x$ geometrijska sredina daljic $a$ in $b$.
\kdokaz

V naslednjem zgledu bomo obravnavali še eno pomembno konstrukcijo.

                \bzgled \label{konstrKoren}
                Construct a line segment $x=e\sqrt{6}$, for a given line segment $e$.
                \ezgled

\textbf{\textit{Proof.}} Podobno kot v prejšnjem zgledu bomo konstrukcijo izpeljali na dva načina.

\begin{figure}[!htb]
\centering
\input{sl.pod.7.6E.5.pic}
\caption{} \label{sl.pod.7.6E.5.pic}
\end{figure}

\textit{1)}
 Načrtajmo najprej takšne točke $P$, $Q$ in $R$, da velja: $PQ=3e$, $QR=2e$ in $\mathcal{B}(P,Q,R)$ (Figure \ref{sl.pod.7.6E.5.pic}), nato polkrožnico $k$ nad premerom $PR$ in na koncu točko $X$ kot presečišče polkrožnice $k$ s pravokotnico premice $PR$ v točki $Q$.

 Dokažimo, da je $x=QX$ iskana daljica. Po izreku \ref{TalesovIzrKroz2} je $PXR$ pravokotni trikotnik s hipotenuzo $PR$, $XQ$ je po konstrukciji njegova višina na tej hipotenuzi. Po višinskem izreku \ref{izrekVisinski} je $x$ geometrijska sredina daljic $PQ=3e$ in $QR=2e$, oz. $x=\sqrt{3e\cdot 2e}=e\sqrt{6}$.


\begin{figure}[!htb]
\centering
\input{sl.pod.7.6E.6.pic}
\caption{} \label{sl.pod.7.6E.6.pic}
\end{figure}

\textit{2)}
 Načrtajmo najprej takšne točke $P$, $Q$ in $R$, da velja: $PQ=3e$, $QR=2e$ in $\mathcal{B}(P,R,Q)$ (Figure \ref{sl.pod.7.6E.6.pic}), nato polkrožnico $k$ nad premerom $PQ$ in na koncu točko $X$ kot presečišče polkrožnice $k$ s pravokotnico premice $PQ$ v točki $R$.


 Dokažimo, da je $x=QX$ iskana daljica. Po izreku \ref{TalesovIzrKroz2} je $PXQ$ pravokotni trikotnik s hipotenuzo $PQ$, $XR$ je po konstrukciji njegova višina na to hipotenuzo. Po Evklidovem izreku \ref{izrekEvklidov} je $x$ geometrijska sredina daljic $PQ=3e$ in $QR=2e$, oz. $x=\sqrt{3e\cdot 2e}=e\sqrt{6}$.
\kdokaz

 V prejšnjem zgledu smo torej opisali postopek za konstrukcijo daljice $x=e\sqrt{n}$, kjer je $e$ dana daljica, $n\in \mathbb{N}$ pa dano naravno število. Vedno imamo dve možnosti za konstrukcijo - z uporabo višinskega izreka oz. Evklidovega izreka. Toda če je $n$ sestavljeno število, imamo več možnosti za izbiro daljic $PQ$ in $QR$. Npr. za $n=12$, lahko izberemo $PQ=3e$ in $QR=4e$, lahko pa tudi $PQ=6e$ in $QR=2e$. Če je $n$ praštevilo, npr. $n=7$, je najbolj ugodno, če izberemo $PQ=7e$ in $QR=e$.


%________________________________________________________________________________
  \poglavje{Pythagoras' Theorem} \label{odd7Pitagora}

Dokažimo znameniti Pitagorov izrek.



        \bizrek \label{PitagorovIzrek}
         \index{izrek!Pitagorov}
         The square of the length of the hypotenuse of a right-angled triangle is
            equal to the sum of the squares of the lengths of both legs, i.e. for a right-angled triangle
            $ABC$ with the hypotenuse of length $c$ and the legs of lengths $a$ and $b$ is
         $$a^2+b^2=c^2$$
        (Pythagoras'\footnote{Predpostavlja se, da je bil ta izrek znan že Egipčanom (pribl. 3000 let pr. n. š.) in Babiloncem (pribl. 2000
        let pr. n. š.), toda starogrški filozof in matematik \index{Pitagora}\textit{Pitagora z otoka Samosa} (582--497 pr. n. š.) ga je verjetno prvi dokazal. Prvi pisni dokument dokaza Pitagorovega izreka je dal \index{Evklid}\textit{Evklid iz Aleksandrije} (3. st. pr. n. š.) v svojem delu ‘‘Elementi’’.
        Pri Starih Grkih se je Pitagorov izrek običajno nanašal na  zvezo med ploščinami kvadratov nad stranicami pravokotnega trikotnika.} theorem)
        \eizrek


\begin{figure}[!htb]
\centering
\input{sl.pod.7.6.0.pic}
\caption{} \label{sl.pod.7.6.0.pic}
\end{figure}

\textbf{\textit{Proof.}} (Figure \ref{sl.pod.7.6.0.pic})

Če uporabimo izrek \ref{izrekEvklidov} in oznake iz dokaza tega izreka, dobimo:
$$a^2 + b^2 = c\cdot a_1 + c\cdot b_1 = c\cdot (a_1 + b_1) = c^2,$$ kar je bilo treba dokazati. \kdokaz

V prejšnji obliki se Pitagorov izrek nanaša na kvadrate dolžin stranic pravokotnega trikotnika. Drugo obliko, ki se nanaša na zvezo med ploščinami kvadratov nad stranicami pravokotnega trikotnika, bomo obravnavali v razdelku \ref{odd8PloTrik}.

Pitagorov izrek nam omogoča, da izračunamo tretjo stranico, če sta dani dve stranici pravokotnega trikotnika. Če s $c$ označimo dolžino hipotenuze ter z $a$ in $b$ dolžini katet pravokotnega trikotnika, iz Pitagorovega izreka dobimo:
\begin{eqnarray*}
c&=&\sqrt{a^2+b^2}\\
a&=&\sqrt{c^2-b^2}\\
b&=&\sqrt{c^2-a^2}
\end{eqnarray*}



        \bizrek \label{PitagorovIzrekObrat}
         \index{izrek!Obratni Pitagorov}
         Let $ABC$ be an arbitrary triangle. If
            $$|AC|^2+|BC|^2=|AB|^2,$$
            then $ABC$ is a right-angled triangle with the right angle at the vertex $C$.\\
        (Converse of Pythagorean Theorem)
        \eizrek

\begin{figure}[!htb]
\centering
\input{sl.pod.7.6.0a.pic}
\caption{} \label{sl.pod.7.6.0a.pic}
\end{figure}

\textbf{\textit{Proof.}} Označimo: $c=|AB|$,  $b=|AC|$ in $a=|BC|$ (Figure \ref{sl.pod.7.6.0a.pic}). Torej velja $a^2+b^2=c^2$ oz.:
\begin{eqnarray} \label{eqnPitagObrat1}
|AB|=c=\sqrt{a^2+b^2}.
\end{eqnarray}
Naj bo $A'B'C'$ takšen pravokotni trikotnik s pravim kotom pri oglišču $C'$, da velja
$A'C'\cong AC$ in $B'C'\cong BC$. Po Pitagorovem izreku je
$|A'B'|^2= |A'C'|^2+|B'C'|^2=b^2+a^2$
oziroma:
\begin{eqnarray} \label{eqnPitagObrat2}
|A'B'|=\sqrt{a^2+b^2},
\end{eqnarray}
 zato je iz \ref{eqnPitagObrat1} in  \ref{eqnPitagObrat2}  $A'B'\cong AB$.
Trikotnika $ABC$ in $A'B'C'$ sta torej skladna (po izreku \textit{SSS} \ref{SSS}), iz tega pa sledi $\angle ACB\cong \angle A'C'B'=90^0$, kar pomeni, da je $ABC$ pravokotni trikotnik s pravim kotom pri oglišču $C$.
\kdokaz



Če so dolžine stranic $a$, $b$ in $c$ pravokotnega trikotnika $ABC$ (kjer je $c$ dolžina hipotenuze) naravna števila, pravimo, da trojica $(a,b,c)$ predstavlja \index{pitagorejska trojica}\pojem{pitagorejsko trojico}. Po Pitagorovem \ref{PitagorovIzrek} izreku za pitagorejsko trojico $(a, b, c)$ velja $a^2+b^2=c^2$. Po obratnem Pitagorovem izreku pa, če za naravna števila $a$, $b$ in $c$ velja $a^2+b^2=c^2$, je $(a,b,c)$ pitagorejska trojica. Najbolj znana pitagorejska trojica\footnote{Ta trojica je bila znana že Starim Egipčanom  in Babiloncem. Imenuje se tudi \index{trikotnik!egipčanski}\pojem{egipčanski trikotnik}, ker so s pomočjo njega Stari Egipčani določali pravi kot na terenu.} je $(3, 4, 5)$ (Figure \ref{sl.pod.7.6.0b.pic}), ker velja $3^2+4^2=5^2$. Pogosto srečamo tudi pitagorejski trojici $(5, 12, 13)$ in $(7, 24, 25)$.


\begin{figure}[!htb]
\centering
\input{sl.pod.7.6.0b.pic}
\caption{} \label{sl.pod.7.6.0b.pic}
\end{figure}

Pravzaprav obstaja neskončno mnogo pitagorejskih trojic. Že iz ene takšne trojice $(a,b,c)$ jih lahko dobimo neskončno mnogo: $(ka, kb, kc)$ (za poljubno $k\in \mathbb{N}$). Očitno gre v tem primeru za podobne pravokotne trikotnike.
Za pitagorejsko trojico pravimo, da je \index{pitagorejska trojica!primitivna}\pojem{primitivna}, če $a$, $b$ in $c$ nimajo skupnega delitelja. Izkaže se, da obstaja tudi neskončno mnogo primitivnih pitagorejskih trojic.

En postopek računanja novih pitagorejskih trojic dobimo na naslednji način.
Če sta $m$ in $n$ poljubni naravni števili ($m > n$), potem je:
 \begin{eqnarray*}
    a &=& m^2 - n^2,\\
    b &=& 2mn,\\
    c &=& m^2 + n^2.
 \end{eqnarray*}
 Z enostavnim računom se lahko prepričamo, da $(a, b, c)$ res predstavlja pitagorejsko trojico.

 V nadaljevanju bomo uporabili Pitagorov izrek še pri drugih likih.




        \bzgled \label{PitagorovPravokotnik}
        If $a$ and $b$ are the lengths of the sides and $d$ the lengths of the diagonal of
            a rectangle, then:
        $$d=\sqrt{a^2+b^2}.$$
        \ezgled

\begin{figure}[!htb]
\centering
\input{sl.pod.7.6.0c.pic}
\caption{} \label{sl.pod.7.6.0c.pic}
\end{figure}

\textbf{\textit{Proof.}} Naj bo $ABCD$ poljubni pravokotnik. Označimo: $a=|AB|$,  $b=|BC|$ in $d=|AC|$ (Figure \ref{sl.pod.7.6.0c.pic}). Ker je $ABC$ pravokotni trikotnik s hipotenuze $AC$, je po Pitagorovem izreku \ref{PitagorovIzrek} $d^2=a^2+b^2$ oz.  $d=\sqrt{a^2+b^2}$.
\kdokaz

  Direktna posledica (za $a=b$) je naslednja trditev.




        \bzgled \label{PitagorovKvadrat}
        If $a$ is the length of the side and $d$ the length of the diagonal of a square,
            then (Figure \ref{sl.pod.7.6.0d.pic}):
        $$d=a\sqrt{2}.$$
        \ezgled


\begin{figure}[!htb]
\centering
\input{sl.pod.7.6.0d.pic}
\caption{} \label{sl.pod.7.6.0d.pic}
\end{figure}

Dejstvo, da je $\sqrt{2}$ iracionalno število ($\sqrt{2}\notin \mathbb{Q}$) oz. da ga ni možno zapisati v obliki ulomka, ima za posledico, da  diagonala kvadrata $d$ in stranica $a$ nista \index{sorazmerni daljici}\pojem{sorazmerni} oz. \index{primerljivi daljici}\pojem{primerljivi} daljici\footnote{Stari Grki niso še poznali iracionalnih števil. Pitagorejci - filozofska šola, ki jo je utemeljil starogrški matematik  \index{Pitagora}
        \textit{Pitagora z otoka Samosa}
         (582--497 pr. n. š.) - so bili prepričanja, da je vse število. Pri tem so mislili na racionalna števila in verjeli, da sta vsaki dve daljici sorazmerni. Prvi, ki je ugotovil nasprotno, je bil Starogrški matematik in filozof iz pitagorejske šole \index{Hipas}\textit{Hipas iz Metaponta} (5. st. pr. n. š.). Za to odkritje, ki je bilo torej v popolnem nasprotju s pitagorejsko filozofijo,  je vezano več legend - od njegovega samomora do tega,  da so Hipasa utopili drugi pitagorejci ali pa so ga le izključili iz svojega kroga.}. To pomeni, da ne obstaja nobena daljica $e$ kot enota, da bi za neka naravna števila $n,m\in \mathbb{N}$ veljalo $d=n\cdot e$ in $a=m\cdot e$.



        \bzgled \label{PitagorovEnakostr}
        Suppose that $a$ is the length of the side of an equilateral triangle $ABC$.
        If $v$ is the length of the altitude, $R$ the circumradius and $r$ the inradius
         of that triangle, then
        $$v=\frac{a\sqrt{3}}{2},\hspace{2mm} R=\frac{a\sqrt{3}}{3},
        \hspace{2mm} r=\frac{a\sqrt{3}}{6}.$$
        \ezgled

\begin{figure}[!htb]
\centering
\input{sl.pod.7.6.0e.pic}
\caption{} \label{sl.pod.7.6.0e.pic}
\end{figure}

\textbf{\textit{Proof.}} Naj bo $AA'$ višina trikotnika $ABC$ (Figure \ref{sl.pod.7.6.0e.pic}).

V razdelku \ref{odd3ZnamTock} smo dokazali, da so pri enakostraničnem trikotniku vse štiri značilne točke enake ($O=S=T=V$). To pomeni, da je težišče $T$ trikotnika $ABC$ hkrati središče očrtane in včrtane krožnice tega trikotnika. Prav tako je $T$ hkrati višinska točka tega trikotnika, kar pomeni, da je $AA'$ tudi težiščnica, zato je točka $A'$ središče stranice $BC$, oz. $|BA'|=\frac{a}{2}$. Po izreku \ref{tezisce} je $AT:TA'=2:1$. Torej:
 \begin{eqnarray} \label{eqnPitagEnakostr1}
 R=|TA|=\frac{2}{3}|AA'|=\frac{2}{3}v \hspace*{1mm} \textrm{ in } \hspace*{1mm}
 r=|TA'|=\frac{1}{3}|AA'|=\frac{1}{3}v.
 \end{eqnarray}

Ker je $ABA'$ pravokotni trikotnik s hipotenuzo $AB$, je po Pitagorovem izreku \ref{PitagorovIzrek}
\begin{eqnarray*}
 v=\sqrt{a^2-\left(\frac{a}{2} \right)^2}=\sqrt{\frac{3a^2}{4}}=\frac{a\sqrt{3}}{2}.
 \end{eqnarray*}
 Če uporabimo še relaciji iz \ref{eqnPitagEnakostr1}, dobimo $R=\frac{a\sqrt{3}}{3}$
        in $r=\frac{a\sqrt{3}}{6}$.
\kdokaz



        \bzgled \label{PitagorovRomb}
        If $a$ is the length of the side and $e$ and $f$ the lengths of the diagonals of a rhombus, then
        $$\left(\frac{e}{2}\right)^2+\left(\frac{f}{2}\right)^2=a^2.$$
        \ezgled

\begin{figure}[!htb]
\centering
\input{sl.pod.7.6.0f.pic}
\caption{} \label{sl.pod.7.6.0f.pic}
\end{figure}

\textbf{\textit{Proof.}} Naj bo $S$ presečišče diagonal romba $ABCD$ (Figure \ref{sl.pod.7.6.0f.pic}). Po izreku \ref{paralelogram} je $S$ središče diagonal $AC$ in $BD$. Torej $|SA|=\frac{e}{2}$ in $|SB|=\frac{f}{2}$. Po izreku \ref{RombPravKvadr} sta diagonali $AC$ in $BD$ pravokotni, zato je $ASB$ pravokotni trikotnik s hipotenuzo dolžine $|AB|=c$ in katetama dolžin $|SA|=\frac{e}{2}$ in $|SB|=\frac{f}{2}$. Iskana relacija je sedaj direktna posledica Pitagorovega izreka \ref{PitagorovIzrek}.
\kdokaz



                \bzgled
                If $c$ is the length of the hypotenuse, $a$ and $b$ the lengths of the legs
                 and $v_c$ the length of the altitude on the hypotenuse of a right-angled
                triangle, then
                $$\frac{1}{a^2}+\frac{1}{b^2}=\frac{1}{v_c^2}.$$
                \ezgled


\textbf{\textit{Proof.}}
 Če uporabimo Pitagorov izrek \ref{PitagorovIzrek} in trditev iz zgleda \ref{EvklVisPosl}, dobimo:
  $$\frac{1}{a^2}+\frac{1}{b^2}=\frac{a^2+b^2}{a^2b^2}
  =\frac{c^2}{a^2b^2}=\frac{c^2}{c^2v_c^2}=\frac{1}{v_c^2},$$ kar je bilo treba dokazati. \kdokaz

V zgledu \ref{konstrKoren} smo obravnavali dva načina konstrukcije daljice $x=e\sqrt{n}$ ($n\in \mathbb{N}$) z uporabo višinskega in Evklidovega izreka. Na tem mestu bomo to nalogo rešili s pomočjo Pitagorovega izreka - ponovno na dva načina.


                \bzgled \label{konstrKorenPit}
                Construct a line segment $e\sqrt{7}$, for a given line segment $e$.
                \ezgled

\textbf{\textit{Solution.}} Konstrukcijo bomo izpeljali na dva načina.


\textit{1)} (Figure \ref{sl.pod.7.6.3.pic}).
Ideja je, da načrtamo pravokotni trikotnik $ABC$ s kateto  $a=e\sqrt{7}$, kjer sta druga kateta $b=n\cdot e$ in hipotenuza $c=m\cdot e$ ($n$ in $m$ sta naravni ali vsaj racionalni števili).
Po Pitagorovem izreku \ref{PitagorovIzrek} je $\left(e\sqrt{7}\right)^2+b^2=c^2$ oz. $m^2 e^2-n^2e^2=7e^2$ ali ekvivalentno $(m+n)(m-n)=7$. Eno rešitev te enačbe po $m$ in $n$ dobimo, če rešimo sistem:
\begin{eqnarray*}
&& m+n=7\\
&& m-n=1
\end{eqnarray*}
S seštevanjem in odštevanjem enačb iz tega sistema dobimo torej eno možnost $m=4$ in $n=3$; iz tega je $c=4e$ in $b=3e$. Ta ideja nam omogoča konstrukcijo.

Načrtajmo najprej pravokotni trikotnik $ABC$ s hipotenuzo $AB=4e$ in kateto $AC=3e$. Po Pitagorovem izreku \ref{PitagorovIzrek} je $BC^2=\left(4e\right)^2-\left(3e\right)^2=7e^2$ oz. $BC=e\sqrt{7}$.

\begin{figure}[!htb]
\centering
\input{sl.pod.7.6.3.pic}
\caption{} \label{sl.pod.7.6.3.pic}
\end{figure}

\textit{2)} (Figure \ref{sl.pod.7.6.3.pic}).
Konstruirajmo najprej pravokotni trikotnik $A_0A_1A_2$ s katetama $A_0A_1=A_1A_2=e$, nato pravokotni trikotnik $A_0A_2A_3$ s katetama $A_0A_2$ in $A_2A_3=e$. Postopek nadaljujemo in konstruiramo zaporedje pravokotnih trikotnikov $A_0A_{n-1}A_n$ s katetama $A_0A_{n-1}$ in $A_{n-1}A_n=e$. Po Pitagorovem izreku je:
 \begin{eqnarray*}
A_0A_2&=&\sqrt{A_0A_1^2+A_1A_2^2}=\sqrt{e^2+e^2}=e\sqrt{2}\\
A_0A_3&=&\sqrt{A_0A_2^2+A_2A_3^2}=\sqrt{2e^2+e^2}=e\sqrt{3}\\
&\vdots&\\
A_0A_n&=&\sqrt{A_0A_{n-1}^2+A_{n-1}A_n^2}=\sqrt{(n-1)e^2+e^2}=e\sqrt{n}
\end{eqnarray*}
 Iskano daljico torej dobimo iz šestega pravokotnega trikotnika: $A_0A_7=e\sqrt{7}$.
\kdokaz

Nekaj naslednjih primerov se nanaša na uporabo Pitagorovega izreka pri krožnici.



            \bzgled \label{PitagoraCofman}
            (Example from the book \cite{Cofman})
            Let $a$ and $b$ be two circles of the same
            radius $r=1$, touching each other externally at the point $P$,
             and $t$ their common external tangent.
             If $c_1$, $c_2$, $\ldots$, $c_n$,... is a sequence of circles touching
            circles $a$ and $b$, the first of them touching the line $t$ and each circle from the sequence
            touches the previous one. Calculate the diameter of the circle $c_n$.\\

            \ezgled

\begin{figure}[!htb]
\centering
\input{sl.pod.7.6.4.pic}
\caption{} \label{sl.pod.7.6.4.pic}
\end{figure}

\textbf{\textit{Proof.}}
 Označimo: z $A$ in $B$ središči krožnic $a$ in $b$, z $A'$ in $B'$ dotikališči teh  krožnic s
tangento $t$, s $S_n$ središča krožnic $c_n$ ($n =1,2,\ldots$), z $r_n$ njihove polmere, z $d_n=2r_n$ njihove
premere, s $P_n$ dotikališča krožnic $c_n$ in $c_{n-1}$ (oz. krožnice $c_1$ in premice $t$) ter s $S_n'$ in $X_n$ pravokotne projekcije točk $S_n$
oz. $P_n$ na premici $AA'$ (Figure \ref{sl.pod.7.6.4.pic}). Naj bo še $x_n=|AX_n|$. Če uporabimo Pitagorov izrek \ref{PitagorovIzrek} za trikotnik
$AS_nS_n'$, dobimo $|AS_n|^2=|S_nS_n'|^2+|S_n'A|^2$ oz.
$\left(1+r_n \right)^2=1+\left(x_n-r_n \right)^2$. Iz tega reševanjem po $r_n$ dobimo:
 \begin{eqnarray} \label{eqnPitagoraCofman1}
 r_n=\frac{x_n^2}{2\left(1+x_n \right)}.
 \end{eqnarray}
 Po drugi strani je jasno $x_n-x_{n+1}=d_n=2r_n$. Če to povežemo z relacijo \ref{eqnPitagoraCofman1}, dobimo:
 \begin{eqnarray*}
 x_{n+1}=x_n-2r_n=x_n-\frac{x_n^2}{1+x_n}=\frac{x_n}{1+x_n}.
 \end{eqnarray*}
 Ker pri tem velja $x_1=1$, z direktnim računanjem dobimo $x_2=\frac{1}{2}$, $x_3=\frac{1}{3}$, $\ldots$ Iz tega intuitivno sklepamo, da velja:
 \begin{eqnarray} \label{eqnPitagoraCofman2}
 x_n=\frac{1}{n}.
 \end{eqnarray}
Dokažimo relacijo \ref{eqnPitagoraCofman2} formalno - z uporabo matematične indukcije. Relacija \ref{eqnPitagoraCofman2} jasno velja za $n=1$. Predpostavimo, da relacija \ref{eqnPitagoraCofman2} velja za $n=k$ ($x_k=\frac{1}{k}$) in dokažimo, da iz tega sledi, da velja za $n=k+1$ ($x_{k+1}=\frac{1}{k+1}$):
 \begin{eqnarray*}
 x_{k+1}=\frac{x_k}{1+x_k}=\frac{\frac{1}{k}}{1+\frac{1}{k}}=\frac{1}{k+1}.
 \end{eqnarray*}
S tem smo dokazali, da relacija \ref{eqnPitagoraCofman2} velja za vsak $n\in \mathbb{N}$.

Na koncu iz relacij \ref{eqnPitagoraCofman1} in \ref{eqnPitagoraCofman2} sledi:
\begin{eqnarray*}
 d_n=2r_n=\frac{x_n^2}{1+x_n}=
 \frac{\left(\frac{1}{n}\right)^2}{1+\frac{1}{n}}=\frac{1}{n(n+1)}
 \end{eqnarray*}
oz.
\begin{eqnarray} \label{eqnPitagoraCofman3}
 d_n=\frac{1}{n(n+1)},
 \end{eqnarray}
 kar je bilo potrebno izračunati.
\kdokaz

S pomočjo trditve iz prejšnjega zgleda oz. relacije \ref{eqnPitagoraCofman3} lahko pridemo do zanimive neskončne vrste. Namreč $c_1, c_2, \ldots
c_n,\ldots$ je neskončno zaporedje krožnic in pri tem velja:
\begin{eqnarray*}
 d_1+d_2+\cdots + d_n+\cdots=|PP_1|=1.
 \end{eqnarray*}
Sedaj iz relacije \ref{eqnPitagoraCofman3} dobimo:
\begin{eqnarray*}
 \frac{1}{1\cdot 2}+\frac{1}{2\cdot 3}+\cdots +\frac{1}{n\cdot (n+1)}+\cdots=1,
 \end{eqnarray*}
 oz.
\begin{eqnarray*}
 \sum_{n=1}^{\infty}\frac{1}{n (n+1)}=1.
 \end{eqnarray*}



        \bizrek \label{Jung}
       Let $\mathcal{P}=\{A_1,A_2,\ldots A_n\}$ be a finite set of
            points in the plane and $d=\max \{|A_iA_j|;\hspace*{1mm}1\leq i,j\leq n\}$.
             Prove that there exists a circle with radius $\frac{d}{\sqrt{3}}$,
             which contains all the points of this set (\index{izrek!Jungov}Jung's theorem\footnote{Nemški matematik
        \index{Jung, H. W. E.}\textit{H. W. E. Jung} (1876--1953)
        je leta 1901 dokazal
        splošno trditev za $n$-razsežni primer.}).
        \eizrek



\begin{figure}[!htb]
\centering
\input{sl.pod.7.6.1a.pic}
\caption{} \label{sl.pod.7.6.1a.pic}
\end{figure}

\textbf{\textit{Proof.}}  (Figure \ref{sl.pod.7.6.1a.pic})

Po trditvi \ref{lemaJung} je dovolj dokazati, da za vsake tri
točke od teh $n$ točk obstaja ustrezen krog. Naj bodo to točke
$A$, $B$ in $C$. Po predpostavki nobena od stranic trikotnika $ABC$
ne presega $d$. Brez škode za splošnost naj bo $BC$ največja
stranica tega trikotnika. Potem je $|AC|, |AB| \leq |BC| \leq d$.
Obravnavali bomo dva primera:

\textit{1)} V primeru, ko je $ABC$ pravokotni ali topokotni trikotnik,
krog   $\mathcal{K}'(S, \frac{|BC|}{2})$ (kjer je $S$ središče stranice $BC$)
vsebuje vse tri točke $A$, $B$ in $C$. Ker je $\frac{|BC|}{2}\leq
\frac{d}{2}<\frac{d}{\sqrt{3}}$, točke $A$, $B$ in $C$ ležijo tudi v
krogu $\mathcal{K}(S, \frac{d}{\sqrt{3}})$.

\textit{2)} Če je trikotnik $ABC$ ostrokotni, mora biti $\angle BAC
\geq 60^0$, ker je $BC$  največja stranica tega trikotnika. Naj bo
$BA'C$ pravilen trikotnik, tako da sta točki $A$ in $A'$ na istem
bregu premice $BC$, in $\mathcal{K}'(S,R)$ krog, ki je določen z očrtano
krožnico tega trikotnika. Zaradi pogoja $\angle BAC \geq 60^0$ je
točka $A$ bodisi na robu kroga $\mathcal{K}'$ bodisi  njegova notranja točka.
Polmer tega kroga je po izreku \ref{PitagorovEnakostr} enak
$R=\frac{|BC|\sqrt{3}}{3}=\frac{|BC|}{\sqrt{3}}$. Ker je $|BC|\leq d$, točki $A$, $B$ ležita tudi v krogu $\mathcal{K}(S, \frac{d}{\sqrt{3}})$.
\kdokaz



        \bizrek
        Let $M$ and $N$ be common points of two congruent circles
            $k$ and $l$ with a radius $r$. Points $P$ and $Q$ are the intersections of these two circles with
            a line defined by their centres such that $P$ and $Q$ are on the same side of the line $MN$. Prove that
             $$|MN|^2 + |PQ|^2 = 4r^2.$$
        \eizrek

\begin{figure}[!htb]
\centering
\input{sl.pod.7.6.1.pic}
\caption{} \label{sl.pod.7.6.1.pic}
\end{figure}


\textbf{\textit{Proof.}}
Naj bosta $O$ in $S$ središči teh dveh krožnic in $\overrightarrow{v}
= \overrightarrow{OS}$ (Figure \ref{sl.pod.7.6.1.pic}). Translacija
$\mathcal{T}_{\overrightarrow{v}}$
preslika krožnico $k$ v krožnico $l$ ter točko $M$ v neko točko $M'$.
Ker točka $M$ leži na krožnici
$k$, njena slika $M'$ leži na krožnici $l$. Iz istih razlogov je tudi
   $\mathcal{T}_{\overrightarrow{v}}(P )=Q$. Zato je
$\overrightarrow{MM'} = \overrightarrow{v} = \overrightarrow{OS} = \overrightarrow{PQ}$, kar pomeni,
  da sta premici $MM'$ in $OS$ vzporedni in $MM'\cong OS\cong PQ$. Ker je premica
$MN$ pravokotna na premici $OS$ ($OS$ je simetrala daljice $MN$),
je premica $MN$ pravokotna tudi na vzporednici $MM'$ premice $OS$.
Torej je  $\angle NMM'$ pravi kot, zato je $NM'$ premer krožnice $l$.
 Če uporabimo Pitagorov izrek \ref{PitagorovIzrek}, dobimo:
$$|MN|^2 + |PQ|^2 = |MN|^2 + |MM'|^2 = |NM'|^2 = 4r^2,$$ kar je bilo treba dokazati. \kdokaz



            \bzgled
            Lines $b$ and $c$ and a point $A$ in the same plane are given. Construct
            a square $ABCD$ such that the vertices $B$ and $C$ lie on the lines $b$
            and $c$, respectively.
            \ezgled

\begin{figure}[!htb]
\centering
\input{sl.pod.7.6.2.pic}
\caption{} \label{sl.pod.7.6.2.pic}
\end{figure}

\textbf{\textit{Solution.}} (Figure \ref{sl.pod.7.6.2.pic})

Uporabimo rotacijski razteg $\rho_{A,\sqrt{2},45^0}$. Ker je $\rho_{A,\sqrt{2},45^0}(B)=C$, lahko točko $C$ načrtamo iz pogoja
$C\in c\cap\rho_{A,\sqrt{2},45^0}(b)$.
\kdokaz



        \bnaloga\footnote{1. IMO, Romania - 1959, Problem 4.}
        Construct a right triangle with given hypotenuse $c$ such that the median
        drawn to the hypotenuse is the geometric mean of the two legs of the triangle.
        \enaloga

\begin{figure}[!htb]
\centering
\input{sl.pod.7.6.IMO1.pic}
\caption{} \label{sl.pod.7.6.IMO1.pic}
\end{figure}

\textbf{\textit{Solution.}}
 Predpostavimo, da je $ABC$ pravokotni trikotnik s hipotenuzo
  $AB\cong c$ in je njegova težiščnica   na hipotenuzo $CM=t_c$
 geometrijska sredina katet $AC=b$ in $BC=a$ oz.
 $t_c=\sqrt{a b}$ (Figure \ref{sl.pod.7.6.IMO1.pic}).

 Iz izreka \ref{TalesovIzrKroz} sledi $t_c=\frac{1}{2}\cdot c$. Če
 uporabimo še Pitagorov izrek (\ref{PitagorovIzrek}), dobimo:
  $$(a+b)^2=a^2+b^2+2ab=c^2+2t^2_c=c^2+2\left( \frac{1}{2}\cdot c\right)^2=
  \frac{6}{4}\cdot c^2$$
  oz.
  $$a+b=\frac{\sqrt{6}}{2}\cdot c.$$
  Slednja relacija omogoča konstrukcijo.

  Opišimo najprej pomožno konstrukcijo daljice, ki ima dolžino
  $\frac{\sqrt{6}}{2}\cdot c$. Načrtajmo najprej daljico $UV=c$
  in njeno središče $W$, nato pa pravokotni trikotnik $TUW$
  ($TU=c$ in $\angle TUW=90^0$) ter pravokotni trikotnik
  $TWZ$ ($WZ=\frac{1}{2}\cdot c$
  in $\angle TWZ=90^0$).

  Sedaj bomo konstruirali trikotnik $ABC$. Načrtajmo daljico $DB\cong
  TZ$ in  $\angle BDX=45^0$. Točka $A$ naj bo eno od
  presečišč krožnice $k(B,c)$ in kraka $DX$. Na koncu načrtajmo točko
  $C$ kot presečišče daljice $BD$ s simetralo
  $s_{DA}$ daljice $DA$.

  Dokažimo, da trikotnik $ABC$ izpolnjuje vse pogoje iz naloge.
  Najprej iz konstrukcije sledi $A\in k(B,c)$, kar pomeni, da velja
  $AB=c$. Točka $C$ po konstrukciji leži na simetrali daljice
  $DA$, zato je $DC\cong AC$ oz. (izrek \ref{enakokraki}) $\angle
  DAC\cong \angle ADC =45^0$. Iz trikotnika $ACD$ je po izreku \ref{VsotKotTrik}
   $\angle ACD=90^0$. Ker je po konstrukciji $\mathcal{B}(D,C,B)$, je
   tudi $\angle ACB=90^0$ in je $ABC$ pravokotni  trikotnik s
   hipotenuzo $AB=c$.

   Dokažimo še, da je njegova
   težiščnica na hipotenuzo $CM=t_c$
   geometrijska sredina njegovih katet $AC=b$ in $BC=a$ oz. $t_c=\sqrt{a b}$.
   Po konstrukciji sta trikotnika $TUW$ in $TWZ$ pravokotna in je
   $UW=WZ=\frac{1}{2}\cdot c$ in $TW=c$,
   zato iz Pitagorovega izreka sledi:
    \begin{eqnarray*}
     |TZ|^2&=&|TW|^2+|WZ|^2=\\
     &=& |TU|^2+|UW|^2+|WZ|^2=\\
     &=&
     c^2+\left(\frac{c}{2}\right)^2+\left(\frac{c}{2}\right)^2=\\
     &=& \frac{6}{4}\cdot c^2
     \end{eqnarray*}
  Po konstrukciji je $BD\cong TZ$, zato je tudi $|BD|^2=\frac{6}{4}\cdot
  c^2$. Torej:
 \begin{eqnarray*}
     \frac{6}{4}\cdot c^2&=&|BD|^2=\left(|BC|+|CD|\right)^2=\\
        &=&\left(|BC|+|CA|\right)^2=(a+b)^2=\\
        &=&a^2+b^2+2ab=\\
        &=&c^2+2ab\\
 \end{eqnarray*}
 Iz tega sledi $\left(\frac{c}{2} \right)^2=ab$. Ker je po izreku
 \ref{TalesovIzrKroz} $t_c=\frac{c}{2}$, na koncu dobimo
 $t_c^2=ab$.

 Število rešitev naloge je
 odvisno od števila presečišč krožnice $k(B,c)$ in poltraka
 $DX$.
  \kdokaz



      \bnaloga\footnote{23. IMO, Hungary - 1982, Problem 5.}
        The diagonals $AC$ and $CE$ of the regular hexagon $ABCDEF$ are
        divided by the inner points $M$ and $N$, respectively, so that
        $$\frac{AM}{AC}=\frac{CN}{CE}=r.$$
        Determine $r$ if $B$, $M$ and $N$ are collinear.
        \enaloga


\begin{figure}[!htb]
\centering
\input{sl.skk.4.2.IMO2.pic}
\caption{} \label{sl.skk.4.2.IMO2.pic}
\end{figure}

\textbf{\textit{Solution.}} Označimo z $O$ središče pravilnega
šestkotnika $ABCDEF$ (Figure \ref{sl.skk.4.2.IMO2.pic}) in z $a$
dolžino njegove stranice.
 Iz predpostavke $\frac{AM}{AC}=\frac{CN}{CE}=r$ in iz dejstva $AC\cong CE$
  sledi $CM\cong EN$ in $AM\cong CN$.
Trikotnika $ACB$ in $CED$ sta skladna (izrek \textit{SSS}
\ref{SSS}), zato je $\angle ACB\cong\angle CED$. To pomeni, da sta
skladna tudi trikotnika $BCM$ in $DEN$ (izrek \textit{ASA}
\ref{KSK} oz. $\angle BMC\cong \angle DNE$.
 Trikotnik $ACE$ je pravilen, zato je $\angle ACE=60^0$. Torej:
 \begin{eqnarray*}
 \angle BND&=&\angle BNC+\angle CND\\
 &=&\angle BMC- \angle ECA+180^0-\angle DNE=120^0
 \end{eqnarray*}
 Ker je tudi $\angle BOD=120^0$ in $CO\cong CD\cong CB$,
 točka $N$  leži na krožnici $k(C,CO)$. Zato je $AM\cong CN\cong
 CB=a$. Če z $v$ označimo dolžino višine pravilnega trikotnika
 $OCD$ in uporabimo relacijo iz zgleda \ref{PitagorovEnakostr}:
  $$r=\frac{CN}{CE}=\frac{a}{2v}=\frac{\sqrt{3}}{3},$$ kar je bilo treba izračunati. \kdokaz

Let $ABC$ be an acute-angled triangle with circumcentre $O$. Let $P$ on $BC$ be the foot of the altitude from $A$.
Suppose that $\angle BCA>\angle ABC+30^0$.
Prove that $\angle CAB+\angle COP<90^0$.

      \bnaloga\footnote{42. IMO, USA - 2001, Problem 1.}
        Let $ABC$ be an acute-angled triangle with circumcentre $O$. Let $P$ on $BC$ be the foot of the altitude from $A$.
        Suppose that $\angle BCA>\angle ABC+30^0$.
        Prove that $\angle CAB+\angle COP<90^0$.
        \enaloga


\begin{figure}[!htb]
\centering
\input{sl.pod.7.6.IMO3.pic}
\caption{} \label{sl.pod.7.6.IMO3.pic}
\end{figure}

\textbf{\textit{Solution.}} Označimo z $A_1$ središče stranice
$BC$, s $P'$ in $A'$ drugi presečišči premic $AP$ in $AO$ z
očrtano krožnico $k(O,R)$ trikotnika $ABC$, ter s $Q$ pravokotno
projekcijo točke $A'$ na premici $BC$ (Figure
\ref{sl.pod.7.6.IMO3.pic}). Označimo še z $\alpha$, $\beta$ in
$\gamma$ notranje kote trikotnika $ABC$ ob ogliščih $A$, $B$ in
$C$.

Ker je $AA'$ premer krožnice $k$, je $\angle AP'A'=90^0$ (izrek
\ref{TalesovIzrKroz}). To pomeni, da je $PP'A'Q$ pravokotnik in
velja:
 \begin{eqnarray}
PQ\cong A'P'. \label{pod.7.6.IMO31}
 \end{eqnarray}
 Točka $O$ je središče premera $AA'$, $A_1$ pa je po Talesovem izreku
\ref{TalesovIzrek}  središče dalice $PQ$. Točka $A_1$ je
torej skupno središče daljic $BC$ in $PQ$.

Po izreku \ref{TockaNbetagama} je še $\angle A'AP'=\angle
OAP=\gamma-\beta$. Iz danega pogoja $\angle BCA>\angle ABC+30^0$
oz. $\gamma-\beta>30^0$ torej sledi:
 \begin{eqnarray}
\angle A'AP>30^0. \label{pod.7.6.IMO32}
\end{eqnarray}
Iz izreka \ref{SredObodKot} je $\angle BOC=2\alpha$. Ker je $BOC$
enakokraki trikotnik z osnovnico $BC$, je $\angle
OCB=\frac{1}{2}\left( 180^0-2\alpha\right)=90^0-\alpha$ (izreka
\ref{enakokraki} in \ref{VsotKotTrik}). Pogoj $\angle CAB+\angle
COP<90^0$ oz. $\angle COP<90^0-\alpha$, ki ga želimo dokazati, je
zato ekvivalenten s pogojem $\angle COP<\angle OCB=\angle OCP$
(velja $\angle OCB=\angle OCP$, ker je $ABC$ ostrokotni trikotnik
in je $\mathcal{B}(B,P,C)$). Slednji pa je po izreku
\ref{vecstrveckot} ekvivalenten s pogojem $CP<OP$. Dovolj je torej
dokazati, da velja:
 \begin{eqnarray}
CP<OP. \label{pod.7.6.IMO3a}
 \end{eqnarray}
Če uporabimo Pitagorov izrek \ref{PitagorovIzrek} za pravokotni
trikotnik $OA_1P$, dobimo:
$|OP|^2=|OA_1|^2+|PA_1|^2=|OA_1|^2+\frac{|PQ|^2}{4}$. Po drugi
strani je

$|CP|^2=\frac{\left(|BC|-|PQ|\right)^2}{4}$. Zato je neenakost
\ref{pod.7.6.IMO3a}, ki jo želimo dokazati, ekvivalentna z
$\frac{|BC|^2}{4}-\frac{|BC|\cdot|PQ|}{2}<|OA_1|^2$ oziroma:
\begin{eqnarray}
4|OA_1|^2+2|BC|\cdot|PQ|>|BC|^2. \label{pod.7.6.IMO3b}
 \end{eqnarray}

Toda iz neenakosti \ref{pod.7.6.IMO32} in iz izreka \ref{SredObodKot}
sledi $\angle A'OP'>60^0$, zato je v enakokrakem trikotniku $OA'P'$
$\angle A'OP'$ največji kot (izreka \ref{enakokraki} in
\ref{VsotKotTrik}). To pomeni, da velja $A'P'>OA'=R$. Iz
\ref{pod.7.6.IMO31} je potem $PQ\cong A'P'>R$. Na koncu je:
\begin{eqnarray*}
\hspace*{-1.5mm} 4|OA_1|^2+2|BC|\cdot|PQ|>2|BC|\cdot|PQ|>2|BC|\cdot R=|BC|\cdot
2R>|BC|^2,
 \end{eqnarray*}
kar je bilo še potrebno dokazati.
 \kdokaz



%_______________________________________________________________________________
 \poglavje{Napoleon Triangles} \label{odd7Napoleon}

Definirajmo najprej dva trikotnika, ki ju priredimo poljubnemu trikotniku $ABC$.
Nad njegovimi stranicami z zunanje strani skonstruirajmo enakostranične trikotnike. Središča teh trikotnikov so oglišča t. i. \index{trikotnik!Napoleonov zunanji}\pojem{zunanjega Napoleonovega\footnote{Ta trditev je pripisana \textit{Napoleónu Bonapartu} (1769--1821), francoskemu cesaarju in vojskovodji, ki se je zanimal
za elementarno geometrijo, čeprav je vprašanje, ali je imel dovolj znanja, da bi to trditev tudi dokazal.} trikotnika} (Figure \ref{sl.pod.7.7n.1.pic}). Če enakostranične trikotnike konstuiramo z notranje strani, na enak način dobimo t. i. \index{trikotnik!Napoleonov notranji}\pojem{notranji Napoleonov trikotnik}.


\begin{figure}[!htb]
\centering
\input{sl.pod.7.7n.1.pic}
\caption{} \label{sl.pod.7.7n.1.pic}
\end{figure}



               \bizrek
                The outer and inner Napoleon triangles are equilateral.
                \eizrek


\begin{figure}[!htb]
\centering
\input{sl.pod.7.7n.2.pic}
\caption{} \label{sl.pod.7.7n.2.pic}
\end{figure}

\textbf{\textit{Proof.}}
Naj bodo $P$,$ Q$ in $R$ središča pravilnih trikotnikov, ki so konstruirani z zunanje strani nad stranicami poljubnega trikotnika $ABC$, torej je $PQR$ zunanji Napoleonov trikotnik trikotnika $ABC$ (Figure \ref{sl.pod.7.7n.2.pic}). Dokažimo, da je $PQR$ enakostranični trikotnik.

Naj bo:
$$\mathcal{I}=\mathcal{R}_{R,120^0}\circ
\mathcal{R}_{P,120^0}\circ\mathcal{R}_{Q,120^0}.$$
Ker je $120^0+120^0+120^0=360^0$, je kompozitum $\mathcal{I}$ bodisi identiteta bodisi translacija (izrek \ref{rotacKomp2rotac}). Vendar druga možnost odpade, saj $\mathcal{I}(A)=A$. Torej $\mathcal{I}=\mathcal{E}$ oz.:
 $$\mathcal{R}_{R,120^0}\circ
\mathcal{R}_{P,120^0}\circ\mathcal{R}_{Q,120^0}=\mathcal{E}.$$
Iz tega sledi:
 $$\mathcal{R}_{R,120^0}\circ\mathcal{R}_{P,120^0}=
 \mathcal{R}^{-1}_{Q,120^0}=\mathcal{R}_{Q,-120^0}=\mathcal{R}_{Q,240^0}.$$
 Naj bo $Q'$ tretje oglišče enakostraničnega trikotnika $PQ'R$. Po izreku \ref{rotacKomp2rotac} je:
 $$\mathcal{R}_{R,120^0}\circ\mathcal{R}_{P,120^0}=\mathcal{R}_{Q',240^0}.$$
 Torej velja $\mathcal{R}_{Q,240^0}=\mathcal{R}_{Q',240^0}$ oz. $Q=Q'$, kar pomeni, da je $PQR$ enakostranični trikotnik.
 Na podoben način dokažemo (če uporabimo rotacije za kot $-120^0$), da je tudi notranji Napoleonov trikotnik enakostraničen.
        \kdokaz


Dokažimo še dodatno lastnost dveh Napoleonovih trikotnikov.



                \bizrek
                The outer and inner Napoleon triangles of an arbitrary
                 triangle $ABC$ have the same centre, which is at the same time the centroid of the triangle $ABC$.
                \eizrek


\begin{figure}[!htb]
\centering
\input{sl.pod.7.7n.3.pic}
\caption{} \label{sl.pod.7.7n.3.pic}
\end{figure}

\textbf{\textit{Proof.}}
Naj bosta $PQR$ in $P'Q'R'$ ustrezni zunanji in notranji Napoleonov trikotnik trikotnika $ABC$ (Figure \ref{sl.pod.7.7n.3.pic}). Označimo z $Y$ in $A_1$ središči daljic $PR$ in $BC$. Dokažimo najprej:

\begin{enumerate}
  \item Trikotnika $QP'C$ in $RBP'$ sta skladna in sta podobna trikotniku $ABC$.
  \item Štirikotnika $ARP'Q$ in $CP'RQ'$ sta paralelograma.
  \item $\overrightarrow{YA_1}=\frac{1}{2}\overrightarrow{AQ}$.
\end{enumerate}

\textit{1}) Najprej iz $\angle BCP'\cong \angle ACQ=30^0$ sledi
$\angle P'CQ\cong\angle BCA$ (z rotacijo $\mathcal{R}_{C.30^0}$ se kot $\angle P'CQ$ preslika v kot $\angle BCA$). Velja tudi $\frac{CQ}{CA}=\frac{CP'}{CB}=\frac{\sqrt{3}}{3}$ (po izreku \ref{PitagorovEnakostr}, ker sta $CQ$ in $CP'$ polmera očrtanih krožnic dveh enakostraničnih trikotnikov s stranicama $CA$ oz. $CB$), zato sta si trikotnika $ABC$ in $QP'C$  podobna (izrek \ref{PodTrikSKS}) s
koeficientom podobnosti $k=\frac{\sqrt{3}}{3}$.
 Analogno sta si tudi trikotnika $ABC$ in$ RBP'$
podobna z enakim koeficientom podobnosti, kar pomeni, da sta trikotnika $QP'C$ in $RBP'$ skladna.

\textit{2}) Iz skladnosti trikotnikov $QP'C$ in $RBP'$ in iz dejstva, da sta trikotnika $RAB$ in $QAC$ enakokraka, sledi:
$QP'\cong RB\cong RA$ in $P'R\cong CQ\cong QA$, zato je štirikotnik $ARP'Q$ paralelogram. Analogno dokažemo, da
je tudi štirikotnik $CP'RQ'$ paralelogram.


\textit{3}) Daljica $YA_1$ je srednjica trikotnika $RPP'$ z osnovnico $RP'$, zato je po izrekih \ref{srednjicaTrikVekt} in \ref{vektParalelogram}:

$$\overrightarrow{YA_1}=\frac{1}{2}\overrightarrow{RP'}=\frac{1}{2}\overrightarrow{AQ}$$

Dokazati moramo le še začetno trditev. Ker je $\overrightarrow{YA_1}=\frac{1}{2}\overrightarrow{AQ}$,
 se težiščnica $AA_1$ trikotnika $ABC$ in
težiščnica $QY$ trikotnika $PQR$ sekata v točki, ki ju deli v razmerju $2:1$. Torej gre za skupno težišče trikotnikov $ABC$ in $PQR$ (izrek \ref{tezisce}). Analogno se dokaže, da imata tudi trikotnika
$ABC$ in $P'Q'R'$ skupno težišče.
 \kdokaz


%________________________________________________________________________________
 \poglavje{Ptolemy's Theorem} \label{odd7Ptolomej}

Dokažimo zelo pomemben izrek, ki se nanaša na tetivne štirikotnike.



                \bizrek \label{izrekPtolomej}
                \index{izrek!Ptolomejev}(Ptolemy’s\footnote{\index{Ptolomej Aleksandrijski}\textit{Ptolomej Aleksandrijski} (2. st.) je dokazal ta izrek v svojemu delu ‘‘Veliki zbornik’’.} theorem)
                    If $ABCD$ is a cyclic quadrilateral then
                     the product of the lengths of its diagonals
                     is equal to the sum of the products of the lengths of the pairs of opposite sides,
                      i.e.
                $$|AC|\cdot |BD|=|AB|\cdot |CD|+|BC|\cdot |AD|.$$

                \eizrek

\begin{figure}[!htb]
\centering
\input{sl.pod.7.7.1.pic}
\caption{} \label{sl.pod.7.7.1.pic}
\end{figure}

\textbf{\textit{Proof.}}
Brez škode za splošnost predpostavimo, da je $\angle CBD\leq\angle ABD$.
Naj bo $L$ takšna točka diagonale $AC$, da velja
$\angle ABL\cong \angle CBD$ (Figure \ref{sl.pod.7.7.2.pic}). Ker je še $\angle BAL=\angle BAC\cong\angle BDC$ (izrek \ref{ObodObodKot}), sta trikotnika
$ABL$ in $DBC$ podobna (izrek \ref{PodTrikKKK}), zato je $AB:DB=AL:DC$ oziroma:
\begin{eqnarray} \label{eqnPtolomej1}
|AB|\cdot |CD|=|AL|\cdot |BD|.
\end{eqnarray}
Ker je $\angle BCL=\angle BCA\cong\angle BDA$ (izrek \ref{ObodObodKot}) in $\angle LBC\cong\angle ABD$ (iz predpostavke $\angle CBD\cong\angle ABL$), velja
tudi $\triangle BCL\sim \triangle BDA$ (izrek \ref{PodTrikKKK}), zato je $BC:BD=CL:DA$ oziroma:
\begin{eqnarray} \label{eqnPtolomej2}
|BC|\cdot |AD|=|CL|\cdot |BD|.
\end{eqnarray}
 Po seštevanju relacij \ref{eqnPtolomej1} in \ref{eqnPtolomej2} dobimo iskano enakost.
\kdokaz


Trditev iz zgleda \ref{zgledTrikABCocrkrozP} bomo na tem mestu dokazali na bolj enostaven način - z uporabo Ptolomejevega izreka.




        \bzgled \label{zgledTrikABCocrkrozPPtol}
        Let  $k$ be the circumcircle of a regular triangle $ABC$.
         If $P$ is an arbitrary point lying
        on the shorter arc $BC$ of the circle $k$, then
         $$|PA|=|PB|+|PC|.$$
        \ezgled

\begin{figure}[!htb]
\centering
\input{sl.pod.7.7.2a.pic}
\caption{} \label{sl.pod.7.7.2a.pic}
\end{figure}

\textbf{\textit{Solution.}}
Štirikotnik
$ABPC$ je tetiven (Figure \ref{sl.pod.7.7.2a.pic}), zato iz Ptolomejevega izreka \ref{izrekPtolomej} sledi
$|PA|\cdot |BC|=|PB|\cdot |AC|+|PC|\cdot |AB|$. Ker je $ABC$ enakostranični trikotnik, oz. $|BC|=|AC|= |AB|$, iz prejšnje relacije sledi $|PA|=|PB|+|PC|$.
 \kdokaz

 Še ena uporaba Ptolomejevega izreka se nanaša na enakokraki trapez.




                \bzgled
                Let $a$ and $c$ be the bases, $b$ the leg and $d$ the diagonal
                isosceles trapezium. Prove that
                $$ac=d^2-b^2.$$
                \ezgled

\begin{figure}[!htb]
\centering
\input{sl.pod.7.7.4.pic}
\caption{} \label{sl.pod.7.7.4.pic}
\end{figure}

\textbf{\textit{Solution.}} (Figure \ref{sl.pod.7.7.4.pic})

Po izreku \ref{trapezTetivEnakokr} je enakokraki trapez tetiven, zato lahko uporabimo Ptolomejev izrek \ref{izrekPtolomej} in dobimo $d^2=ac+b^2$ oz. $ac=d^2-b^2$.
\kdokaz

Zanimivo je, da če v prejšnjem izreku vzamemo poseben primer $a=c$ (Figure \ref{sl.pod.7.7.4.pic}), ko je enakokraki trapez pravokotnik, dobimo formulo za diagonalo pravokotnika $d^2=a^2+b^2$, kar je sicer direktna posledica Pitagorovega izreka \ref{PitagorovIzrek}. V tem smislu lahko tudi rečemo, da je Pitagorov izrek posledica Ptolomejevega izreka, če pravokotni trikotnik dopolnimo do pravokotnika.

\begin{figure}[!htb]
\centering
\input{sl.pod.7.7.5.pic}
\caption{} \label{sl.pod.7.7.5.pic}
\end{figure}

Seveda nas uporaba Ptolomejevega izreka v kvadratu vodi v že znano zvezo (izrek \ref{PitagorovKvadrat}) med diagonalo $d$ in stranico $a$ kvadrata: $d^2=2a^2$  oz. $d=a\sqrt{2}$ (Figure \ref{sl.pod.7.7.5.pic}).

 Vemo, da za vsak pravilen $n$-kotnik obstaja očrtana
krožnica \ref{sredOcrtaneKrozVeck}. To nam omogoča uporabo Ptolomejevega izreka, če v pravilnem $n$-kotniku (v primeru $n>4$) izberemo ustrezna štiri oglišča.
Sedaj bomo to idejo uporabili v primerih $n=5$ in $n=7$.



                    \bzgled \label{PtolomejPetkotnik}
                    Let $a$ be the side and $d$  the diagonal of a regular pentagon ($5$-gon).
                    Prove that
                    $$d=\frac{1+\sqrt{5}}{2}a.$$
                    \ezgled


\begin{figure}[!htb]
\centering
\input{sl.pod.7.7.3.pic}
\caption{} \label{sl.pod.7.7.3.pic}
\end{figure}

\textbf{\textit{Solution.}}
 Naj bo $A_1A_2A_3A_4A_5$ pravilni petkotnik s stranico $a$ in diagonalo $d$
 (Figure \ref{sl.pod.7.7.3.pic}). Če uporabimo Ptolomejev izrek \ref{izrekPtolomej} za štirikotnik $A_1A_2A_3A_5$, dobimo: $d^2=ad+a^2$ oz. $d^2-ad-a^2=0$. Pozitivna rešitev te kvadratne enačbe po $d$ je ravno $d=\frac{1+\sqrt{5}}{2}a$.
 \kdokaz

Prejšnji primer nam omogoča konstrukcijo pravilnega petkotnika s stranico, ki je skladna daljici $a$.
Res, če je $ABCDE$ iskani pravilni petkotnik, lahko najprej
načrtamo trikotnik $ABC$, kjer je $AB\cong BC\cong a$ in $AC =\frac{1+\sqrt{5}}{2}a$ Za konstrukcijo $\frac{1+\sqrt{5}}{2}a$ uporabimo Pitagorov izrek (\ref{PitagorovIzrek}) za pravokotni trikotnik s katetama $a$ in $2a$ - njegova hipotenuza meri $a\sqrt{5}$. Tej hipotenuzi dodamo še daljico $a$ in dobljeno daljico s simetralo razdelimo na dve daljici. Središče očrtane krožnice pravilnega petkotnika $ABCDE$ je hkrati središče očrtane krožnice trikotnika $ABC$  (Figure \ref{sl.pod.7.7.3a.pic}).


\begin{figure}[!htb]
\centering
\input{sl.pod.7.7.3a.pic}
\caption{} \label{sl.pod.7.7.3a.pic}
\end{figure}


Na podoben način lahko konstruiramo pravilni petkotnik, ki je včrtan dani krožnici. To naredimo tako,
da najprej konstruiramo pravilni petkotnik s poljubno stranico, nato pa uporabimo središčni razteg (Figure \ref{sl.pod.7.7.3a.pic}).

Več o konstrukcijah pravilnih $n$-kotnikov bomo povedali v razdelku \ref{odd9LeSestilo}.



                \bzgled
                Let $a$ be the side, $d$ the shorter and $D$ the longer diagonal
                of a regular heptagon ($7$-gon). Prove that
                    $$\frac{1}{a}=\frac{1}{D}+\frac{1}{d}$$
                \ezgled

\begin{figure}[!htb]
\centering
\input{sl.pod.7.7.2.pic}
\caption{} \label{sl.pod.7.7.2.pic}
\end{figure}

\textbf{\textit{Solution.}}

Naj bo $A_1A_2A_3A_4A_5A_6A_7$ pravilni sedemkotnik in $k$ očrtana
krožnica  (Figure \ref{sl.pod.7.7.2.pic}). Štirikotnik
$A_1A_2A_3A_5$ je tetiven, zato iz Ptolomejevega izreka \ref{izrekPtolomej} sledi
$ad+aD=dD$ oz. $\frac{1}{a}=\frac{1}{D}+\frac{1}{d}$.
 \kdokaz

Če večkrat uporabimo Ptolomejev izrek,  dobimo eno posplošitev naloge \ref{zgledTrikABCocrkrozPPtol}:



            \bzgled
            Let $P$ be an arbitrary point of the shorter arc $A_1A_{2n+1}$
                of the circumcircle of a regular polygon $A_1A_2\cdots A_{2n+1}$.
                If we denote $d_i=PA_i$ ($i\in \{1,2,\cdots , 2n+1 \}$),
                prove that (Figure \ref{sl.pod.7.7.6.pic})
            $$d_1+d_3+\cdots +d_{2n+1}=d_2+d_4+\cdots +d_{2n}.$$
            \ezgled


\begin{figure}[!htb]
\centering
\input{sl.pod.7.7.6.pic}
\caption{} \label{sl.pod.7.7.6.pic}
\end{figure}

%________________________________________________________________________________
 \poglavje{Stewart's Theorem} \label{odd7Stewart}

Naslednji izrek se nanaša na zelo zanimivo in pomembno metrično lastnost trikotnika.



            \bizrek \label{StewartIzrek}
            \index{izrek!Stewartov} (Stewart's\footnote{\index{Stewart, M.}\textit{M. Stewart} (1717--1785),
            angleški matematik, ki je leta 1746 dokazal in objavil to trditev. S to trditvijo ga je seznanil
            njegov učitelj - angleški matematik \index{Simson, R.}\textit{R. Simson} (1687--1768) - ki jo je objavil šele leta 1749. Predpostavlja se,
            da je trditev morda bila znana že \index{Arhimed} \textit{Arhimedu iz Sirakuze} (3. st. pr. n. š.).} theorem)
            If $X$ is an arbitrary point of the side
            $BC$ of a triangle $ABC$, then
            $$|AX|^2=\frac{|BX|}{|BC|}|AC|^2+\frac{|CX|}{|BC|}|AB|^2-|BX|\cdot |CX|.$$
            \eizrek


\begin{figure}[!htb]
\centering
\input{sl.pod.7.8.1.pic}
\caption{} \label{sl.pod.7.8.1.pic}
\end{figure}

\textbf{\textit{Proof.}}
Z $a$, $b$ in $c$ označimo dolžine ustreznih stranic $BC$, $AC$ in $AB$, s $p$, $q$ in $x$ pa po vrsti dolžine daljic $BX$, $CX$ in $AX$. Z $A'$ označimo še
nožišče višine $v_a$ iz oglišča $A$ trikotnika $ABC$ (Figure \ref{sl.pod.7.8.1.pic}). Predpostavimo, da je $\mathcal{B}(B,A',C)$.

Dokažimo najprej trditev za primer, ko je $X=A'$. Iz Pitagorovega izreka \ref{PitagorovIzrek} sledi $v_a^2 =b^2- q^2$
in $v_a^2 =c^2- p^2$. Z množenjem prve enakosti s $p$ in druge s $q$ ter po seštevanju dobljenih relacij in upoštevanju $p+q=a$ dobimo  $av_a^2=pb^2+qc^2-apq$  oz.:
$$v_a^2=\frac{p}{a}b^2+\frac{q}{a}c^2-pq,$$
kar pomeni, da v primeru višin trditev velja.

Naj bo sedaj $X\neq A'$. Daljica $v_a$ je višina trikotnikov $ABX$ in $ABC$ iz oglišča $A$. Brez škode za splošnost predpostavimo, da je $\mathcal{B}(B,A',X)$. Če z $y$ označimo dolžino daljice $BA'$ in uporabimo dokazani del trditve za višine,
dobimo:
\begin{eqnarray*}
v_a^2&=&\frac{y}{p}x^2+\frac{p-y}{p}c^2-y(p-y),\\
v_a^2&=&\frac{y}{a}b^2+\frac{a-y}{a}c^2-y(a-y).
\end{eqnarray*}
Če izenačimo desni strani teh dveh enakosti in poenostavimo, dobimo:
$$x^2=\frac{p}{a}b^2+\frac{q}{a}c^2-pq,$$
kar pomeni, da trditev velja tudi v primeru, ko je $X\neq A'$.

Dokaz za višino $v_a$ je podoben tudi v primeru, ko ni $\mathcal{B}(B,A',C)$, le da takrat dobimo $v_a^2=\frac{p}{a}b^2+\frac{q}{a}c^2+pq$.
\kdokaz


Stewartov izrek \ref{StewartIzrek} lahko zapišemo tudi v drugačni obliki:


             \bizrek \label{StewartIzrek2}
             Let $a=|BC|$, $b=|AC|$ and $c=|AB|$ be the length of
            the sides of a triangle $ABC$. If $X$ is the point that divides the side $BC$ of this triangle
            in the ratio $n:m$ and $x=|AX|$ then
            $$x^2=\frac{n}{n+m}b^2+\frac{m}{n+m}c^2-\frac{mn}{(m+n)^2}bc.$$
            \eizrek

Najbolj znana je uporaba Stewartovega izreka za težiščnice trikotnika.



                \bizrek \label{StwartTezisc}
                If $a$, $b$ and $c$ are the sides and $t_a$ the triangle median
                on the side $a$, then
                $$t_a^2=\frac{1}{2}b^2+\frac{1}{2}c^2-\frac{1}{4}a^2.$$
                \eizrek



\begin{figure}[!htb]
\centering
\input{sl.pod.7.8.2.pic}
\caption{} \label{sl.pod.7.8.2.pic}
\end{figure}


\textbf{\textit{Proof.}}  Trditev je direktna posledica Stewartovega izreka \ref{StewartIzrek2}, saj je v primeru težiščnice $n=m=1$ (Figure \ref{sl.pod.7.8.2.pic}).
\kdokaz


Direktna posledica je naslednja trditev.



                \bzgled \label{StwartTezisc2}
              If $a$, $b$ and $c$ are the sides and $t_a$,  $t_b$ and $t_c$ the triangle medians
                on those sides, respectively, then
                $$t_a^2+t_b^2+t_c^2=\frac{3}{4}\left(a^2+ b^2+c^2\right).$$
                \ezgled


\begin{figure}[!htb]
\centering
\input{sl.pod.7.8.3.pic}
\caption{} \label{sl.pod.7.8.3.pic}
\end{figure}


\textbf{\textit{Proof.}}  (Figure \ref{sl.pod.7.8.3.pic})

 Če trikrat uporabimo dokazano relacijo iz zgleda \ref{StwartTezisc}, dobimo:
 \begin{eqnarray*}
 t_a^2&=&\frac{1}{2}b^2+\frac{1}{2}c^2-\frac{1}{4}a^2\\
 t_b^2&=&\frac{1}{2}a^2+\frac{1}{2}c^2-\frac{1}{4}b^2\\
 t_c^2&=&\frac{1}{2}a^2+\frac{1}{2}b^2-\frac{1}{4}c^2
 \end{eqnarray*}
 Po seštevanju vseh treh enakosti dobimo iskano relacijo.
\kdokaz

V primeru enakokrakega trikotnika se enakost na desni strani Stewartovega izreka poenostavi.



                \bzgled \label{StewartEnakokraki}
                If $ABC$ is an isosceles triangle with the base $BC$ and
                $X$  an arbitrary point of this base, then
            $$|AX|^2=|AB|^2-|BX|\cdot |CX|.$$
                \ezgled

\begin{figure}[!htb]
\centering
\input{sl.pod.7.8.8.pic}
\caption{} \label{sl.pod.7.8.8.pic}
\end{figure}

\textbf{\textit{Proof.}}  (Figure \ref{sl.pod.7.8.8.pic})

 Če je torej $AB\cong AC$, direktno iz Stewartovega izreka (\ref{StewartIzrek}) sledi:
\begin{eqnarray*}
|AX|^2 &=& \frac{|BX|}{|BC|}|AC|^2+\frac{|CX|}{|BC|}|AB|^2-|BX|\cdot |CX|=\\
&=&\left(\frac{|BX|}{|BC|}+\frac{|CX|}{|BC|}\right)|AB|^2-|BX|\cdot |CX|=\\
&=&\frac{|BX|+|CX|}{|BC|}|AB|^2-|BX|\cdot |CX|=\\
&=&|AB|^2-|BX|\cdot |CX|,
\end{eqnarray*}
 kar je bilo treba dokazati. \kdokaz


Z uporabo Stewartovega izreka bomo še malo nadaljevali.



                \bzgled
                Let $E$ be the intersection of the side $BC$ with the bisector of the interior angle
                 $BAC$ of a triangle $ABC$. If we denote
                 $a=|BC|$, $b=|AC|$, $c=|AB|$, $l_a=|AE|$ and $s=\frac{a+b+c}{2}$, then
                $$l_a=\frac{2\sqrt{bc}}{b+c}\sqrt{s(s-a)}.$$
                \ezgled


\begin{figure}[!htb]
\centering
\input{sl.pod.7.8.4.pic}
\caption{} \label{sl.pod.7.8.4.pic}
\end{figure}


\textbf{\textit{Proof.}}  (Figure \ref{sl.pod.7.8.4.pic})

 Po izreku \ref{HarmCetSimKota} je: $BE:CE=c:b$. Če uporabimo Stewartov izrek \ref{StewartIzrek2} za trikotnik $ABC$ in daljico $AE$, dobimo:
$$l_a^2=\frac{c}{b+c}b^2+\frac{b}{b+c}c^2-\frac{bc}{(b+c)^2}a^2.$$
Po enostavnem preoblikovanju in poenostavljanju izraza na desni strani enakosti, dobimo iskano relacijo.
\kdokaz


                \bizrek \label{izrekEulerStirik}\index{izrek!Eulerjev za štirikotnike}
                (Euler's\footnote{Švicarski matematik \index{Euler, L.}\textit{L. Euler} (1707--1783).} theorem for quadrilaterals)
                If $P$ and $Q$ are the midpoints of
                the diagonals $e$ and $f$ of an arbitrary quadrilateral with sides $a$, $b$, $c$ and $d$, then
                $$|PQ|^2=\frac{1}{4}\left(a^2+b^2+c^2+d^2-e^2-f^2 \right).$$
                \eizrek


\begin{figure}[!htb]
\centering
\input{sl.pod.7.8.5.pic}
\caption{} \label{sl.pod.7.8.5.pic}
\end{figure}


\textbf{\textit{Proof.}}
Naj bosta $P$ in $Q$ središči diagonal $AC$ in $BD$ poljubnega štirikotnika $ABCD$ (Figure \ref{sl.pod.7.8.5.pic}).
Označimo še $a=|AB|$, $b=|BC|$, $c=|CD|$, $d=|DA|$, $e=|AC|$ in $f=|BD|$. Daljica $PQ$ je težiščnica trikotnika $AQC$, zato po izreku \ref{StwartTezisc} velja:
 \begin{eqnarray} \label{eqnEulStirik}
  |PQ|^2=\frac{1}{2}|AQ|^2+\frac{1}{2}|CQ|^2-\frac{1}{4}e^2.
 \end{eqnarray}
 Prav tako sta daljici $QA$ in $QC$  težiščnici trikotnikov $ABD$ in
$CBD$, zato je (izrek \ref{StwartTezisc}):
 \begin{eqnarray*}
  |AQ|^2&=&\frac{1}{2}d^2+\frac{1}{2}a^2-\frac{1}{4}f^2,\\
  |CQ|^2&=&\frac{1}{2}c^2+\frac{1}{2}b^2-\frac{1}{4}f^2.
 \end{eqnarray*}
 Če uvrstimo zadnji dve enakosti v \ref{eqnEulStirik}, dobimo iskano relacijo.
 \kdokaz

 Direktno posledico prejšnjega izreka dobimo, če za štirikotnik izberemo paralelogram.

                \bizrek
                A quadrilateral is a parallelogram if and only if the sum of
                the squares of all of its sides is equal to the sum of the squares of its diagonals.
                \eizrek



\begin{figure}[!htb]
\centering
\input{sl.pod.7.8.6.pic}
\caption{} \label{sl.pod.7.8.6.pic}
\end{figure}


\textbf{\textit{Proof.}}
Naj bosta $P$ in $Q$ središči diagonal $AC$ in $BD$ poljubnega štirikotnika $ABCD$. Označimo še $a=|AB|$, $b=|BC|$, $c=|CD|$, $d=|DA|$, $e=|AC|$ in $f=|BD|$. Po izreku \ref{paralelogram} je $ABCD$ paralelogram natanko tedaj, ko je $P=Q$ (Figure \ref{sl.pod.7.8.6.pic}) oz.$|PQ|=0$. Slednje pa po prejšnjem izreku \ref{izrekEulerStirik} velja natanko tedaj, ko je
$\frac{1}{4}\left(a^2+b^2+c^2+d^2-e^2-f^2 \right)=0$ oz. $a^2+b^2+c^2+d^2=e^2+f^2$.
\kdokaz



                \bizrek \label{GMTmnl}
                Let $A$ and $B$ be given points in the plane and
                $m, n, l\in R^+\setminus \{0\}$.
                 Determine a set of all points of this plane such that
                $$m|AX|^2 + n|BX|^2 = l^2.$$
                \eizrek


\begin{figure}[!htb]
\centering
\input{sl.pod.7.8.7.pic}
\caption{} \label{sl.pod.7.8.7.pic}
\end{figure}


\textbf{\textit{Proof.}} (Figure \ref{sl.pod.7.8.7.pic})

 Po trditvi iz zgleda \ref{izrekEnaDelitevDaljice} obstaja ena sama takšna točka $S$ na daljici $AB$, da velja
$\overrightarrow{AS}:\overrightarrow{SB}=n:m$. Naj bo $X$ poljubna točka. Če uporabimo
Stewartov izrek \ref{StewartIzrek2} za trikotnik $AXB$ in daljico $XS$, dobimo:
\begin{eqnarray*}
|XS|^2=\frac{m}{n+m}|AX|^2+\frac{n}{n+m}|BX|^2-\frac{nm}{(n+m)^2}|AB|^2, \textrm{ oz.}
\end{eqnarray*}
 \begin{eqnarray} \label{eqnStewMnozTock}
|XS|^2=\frac{1}{n+m}\left(m|AX|^2+n|BX|^2\right)-\frac{nm}{(n+m)^2}|AB|^2.
\end{eqnarray}
Točka $X$ leži na iskani množici točk natanko tedaj, ko je $m|AX|^2 + n|BX|^2 = l^2$. Zaradi \ref{eqnStewMnozTock} to velja natanko tedaj, ko je:
 \begin{eqnarray*}
|XS|^2=\frac{1}{n+m}l^2-\frac{nm}{(n+m)^2}|AB|^2=c,
\end{eqnarray*}
kjer je $c$ konstanta, ki ni odvisna od točke $X$. Če je torej $c>0$,  je iskana
množica točk krožnica $k(S, c)$. Če je $c=0$, je množica točk le $\{S\}$, če pa je $c<0$, je množica
točk prazna množica.
\kdokaz



                 \bnaloga\footnote{50. IMO, Germany - 2009, Problem 2.}
                 Let $ABC$ be a triangle with circumcentre $O$. The points $P$ and $Q$ are interior points
                of the sides $CA$ and $AB$, respectively. Let $K$, $L$ and $M$ be the midpoints of the segments $BP$, $CQ$
                and $PQ$, respectively, and let $l$ be the circle passing through $K$, $L$ and $M$. Suppose that the line
                $PQ$ is tangent to the circle $l$. Prove that $|OP|=|OQ|$.
                \enaloga

\begin{figure}[!htb]
\centering
\input{sl.pod.7.8.IMO2.pic}
\caption{} \label{sl.pod.7.8.IMO2.pic}
\end{figure}

\textbf{\textit{Solution.}} Označimo s $k(O,R)$  očrtano
krožnico trikotnika $ABC$ (Figure \ref{sl.pod.7.8.IMO2.pic}).
 Daljici $MK$ in $ML$ sta srednjici trikotnikov $QPB$ in $PQC$,
 zato je (izrek \ref{srednjicaTrikVekt}):
  \begin{eqnarray} \label{eqn72}
  \overrightarrow{MK}=\frac{1}{2}\overrightarrow{QB} \hspace*{3mm}
  \textrm{in} \hspace*{3mm} \overrightarrow{ML}=\frac{1}{2}\overrightarrow{PC}
  \end{eqnarray}
 Iz izrekov \ref{KotiTransverzala} in \ref{KotaVzporKraki} potem sledi:
 \begin{eqnarray} \label{eqn73}
 \angle AQP \cong \angle QMK,\hspace*{3mm}
 \angle APQ \cong \angle PML\hspace*{3mm} \textrm{in} \hspace*{3mm}
 \angle BAC \cong  \angle KML
  \end{eqnarray}
 Ker je po predpostavki premica $PQ$ tangenta krožnice $l$, je po
 izreku \ref{ObodKotTang}:
  $$\angle MLK\cong\angle QMK \hspace*{3mm} \textrm{in} \hspace*{3mm}
   \angle MKL\cong\angle PML.$$
 Iz tega in \ref{eqn73} sledi:
 \begin{eqnarray*}
 \angle AQP \cong \angle MLK,\hspace*{3mm}
 \angle APQ \cong \angle MKL\hspace*{3mm} \textrm{in} \hspace*{3mm}
  \angle BAC \cong\angle KML,
  \end{eqnarray*}
 kar pomeni, da sta si trikotnika $AQP$ in $MLK$ podobna (po izreku
  \ref{PodTrikKKK} je dovolj dokazati
  skladnost dveh parov pripadajočih kotov). Zato iz definicije podobnosti
  likov sledi:
  $\frac{AQ}{ML}=\frac{AP}{MK}$. Če slednjo relacijo kombiniramo z relacijo
    \ref{eqn72}, dobimo $\frac{AQ}{AP}=\frac{ML}{MK}=
    \frac{\frac{1}{2}\cdot CP}{\frac{1}{2}\cdot
    BQ}=\frac{CP}{BQ}$. Torej velja:
 \begin{eqnarray} \label{eqn74}
 |AQ|\cdot |BQ| = |AP|\cdot |CP|
  \end{eqnarray}
 Iz Stewartovega izreka \ref{StewartIzrek} za enakokraki trikotnik $AOB$
  ($|OA|=|OB|=R$) sledi:
 $$|OQ|^2=|OA|^2\cdot \frac{QB}{AB}+|OB|^2\cdot \frac{QA}{AB}
 -|AQ|\cdot |BQ|=R^2-|AQ|\cdot |BQ|.$$
 Analogno iz trikotnika $AOC$ po istem izreku dobimo:
  $$|OP|^2=R^2-|AP|\cdot |CP|.$$
Iz dokazane relacije \ref{eqn74} na koncu sledi $|OQ|^2=|OP|^2$
oz.  $|OQ|=|OP|$.
 \kdokaz



%________________________________________________________________________________
\poglavje{Desargues' Theorem} \label{odd7Desargues}

Tudi naslednji izrek je zgodovinsko povezan z razvojem projektivne geometrije.



             \bizrek \label{izrekDesarguesEvkl} \index{izrek!Desarguesov}
            (Desargues’\footnote{
             \index{Desargues, G.} \textit{G. Desargues} (1591--1661), francoski arhitekt, ki je bil eden od
             utemeljiteljev projektivne geometrije.} theorem)
            Let $ABC$ and $A'B'C'$ be two triangles in the plane such that the lines $AA'$, $BB'$ and $CC'$ intersect
                at a point $S$
                 (i.e the triangles are \index{perspective triangles}\pojem{perspective with respect to the centre}
                  \color{blue}  $S$).
                   If $P=BC\cap B'C'$, $Q=AC\cap A'C'$ and $R=AB\cap A'B'$,
                 then the points $P$, $Q$ and
                $R$ are collinear
                 (i.e. triangles are \pojem{perspective with respect to the axis} \color{blue}  $PQ$).
             \eizrek


\begin{figure}[!htb]
\centering
\input{sl.pod.7.10D.1.pic}
\caption{} \label{sl.pod.7.10D.1.pic}
\end{figure}


\textit{\textbf{Proof.}} (Figure
\ref{sl.pod.7.10D.1.pic})

 Če uporabimo Menelajev izrek \ref{izrekMenelaj} za
trikotnike $SA'B'$, $SA'C'$ in $SB'C'$, dobimo:
 \begin{eqnarray*}
\hspace*{-2mm} \frac{\overrightarrow{SA}}{\overrightarrow{AA'}}\cdot
 \frac{\overrightarrow{A'R}}{\overrightarrow{RB'}}\cdot
 \frac{\overrightarrow{B'B}}{\overrightarrow{BS}}=-1, \hspace*{1mm}
\frac{\overrightarrow{SA}}{\overrightarrow{AA'}}\cdot
 \frac{\overrightarrow{A'Q}}{\overrightarrow{QC'}}\cdot
 \frac{\overrightarrow{C'C}}{\overrightarrow{CS}}=-1, \hspace*{1mm}
 \frac{\overrightarrow{SC}}{\overrightarrow{CC'}}\cdot
 \frac{\overrightarrow{C'P}}{\overrightarrow{PB'}}\cdot
 \frac{\overrightarrow{B'B}}{\overrightarrow{BS}}=-1.
 \end{eqnarray*}
 Iz teh treh relacij sledi:
  \begin{eqnarray*}
  \frac{\overrightarrow{A'Q}}{\overrightarrow{QC'}}\cdot
   \frac{\overrightarrow{C'P}}{\overrightarrow{PB'}}\cdot
   \frac{\overrightarrow{B'R}}{\overrightarrow{RA'}}=-1.
   \end{eqnarray*}
Zato so po Menelajevem izreku \ref{izrekMenelaj} (obratna smer) za trikotnik $A'B'C'$
točke $P$, $Q$ in $R$ kolinearne.
 \kdokaz

Na podoben način dokažemo tudi obratno trditev.



            \bizrek \label{izrekDesarguesObr} \index{izrek!Desarguesov obratni}
            Let $ABC$ and $A'B'C'$ be two triangles in the
            plane such that the lines $AA'$ and $BB'$ intersect at the point $S$.
            If $P=BC\cap B'C'$,
             $Q=AC\cap A'C'$ and $R=AB\cap A'B'$ are collinear points, then also
            $S\in CC'$.\\
             (Converse of Desargues’ theorem)
            \eizrek


Naslednji trditvi sta na določen način podobni Desarguesovem izreku \ref{izrekDesarguesEvkl}.



            \bizrek \label{izrekDesarguesZarkVzp}
             Let $ABC$ and $A'B'C'$ be two triangles in the plane such that the lines $AA'$, $BB'$ and $CC'$
            are parallel to each other.
                   If $P=BC\cap B'C'$, $Q=AC\cap A'C'$ and $R=AB\cap A'B'$,
                 then the points $P$, $Q$ and
                $R$ are collinear.
            \eizrek

\begin{figure}[!htb]
\centering
\input{sl.pod.7.10D.2.pic}
\caption{} \label{sl.pod.7.10D.2.pic}
\end{figure}

\textbf{\textit{Proof.}}
  (Figure \ref{sl.pod.7.10D.2.pic})

   Po trikratni uporabi Talesovega izreka \ref{TalesovIzrek} dobimo:
 \begin{eqnarray*}
 \frac{\overrightarrow{BP}}{\overrightarrow{PC}}=
 \frac{\overrightarrow{BB'}}{\overrightarrow{C'C}},\hspace*{4mm}
 \frac{\overrightarrow{CQ}}{\overrightarrow{QA}}=
 \frac{\overrightarrow{CC'}}{\overrightarrow{A'A}},\hspace*{4mm}
 \frac{\overrightarrow{AR}}{\overrightarrow{RB}}=
 \frac{\overrightarrow{AA'}}{\overrightarrow{B'B}}.
 \end{eqnarray*}
 Po množenju teh treh relacij je naprej:
  \begin{eqnarray*}
  \frac{\overrightarrow{BP}}{\overrightarrow{PC}}\cdot
   \frac{\overrightarrow{CQ}}{\overrightarrow{QA}}\cdot
   \frac{\overrightarrow{AR}}{\overrightarrow{RB}}=-1,
   \end{eqnarray*}
in zato so po Menelajevem izreku \ref{izrekMenelaj} (obratna smer) za trikotnik $ABC$
točke $P$, $Q$ in $R$ kolinearne.
 \kdokaz


                   If $P=BC\cap B'C'$, $Q=AC\cap A'C'$ and $R=AB\cap A'B'$,
                 then the points $P$, $Q$ and
                $R$ are collinear.

                \bizrek \label{izrekDesarguesOsNesk}
                Let $ABC$ and $A'B'C'$ be two triangles in the plane such that
                the lines $AA'$, $BB'$ and $CC'$ intersect
                at a point $S$.
                If
                $BC\parallel B'C'$ and
                 $AC\parallel A'C'$, then also $AB\parallel A'B'$.
                \eizrek

\begin{figure}[!htb]
\centering
\input{sl.pod.7.10D.3.pic}
\caption{} \label{sl.pod.7.10D.3.pic}
\end{figure}


 \textbf{\textit{Proof.}} (Figure \ref{sl.pod.7.10D.3.pic})

  Ker je $BC\parallel B'C'$ in
  $AC\parallel A'C'$, iz Talesovega izreka \ref{TalesovIzrek} sledi
  $\frac{\overrightarrow{SB}}{\overrightarrow{SB'}}=
  \frac{\overrightarrow{SC}}{\overrightarrow{SC'}}$ in
    $\frac{\overrightarrow{SA}}{\overrightarrow{SA'}}=
  \frac{\overrightarrow{SC}}{\overrightarrow{SC'}}$.
  Iz prejšnjih dveh relacij sledi najprej
    $\frac{\overrightarrow{SB}}{\overrightarrow{SB'}}=
  \frac{\overrightarrow{SA}}{\overrightarrow{SA'}}$, zato po
  Talesovem ireku (obratna smer) \ref{TalesovIzrekObr} velja tudi $AB\parallel A'B'$.
  \kdokaz

 Formulacije predhodnih treh izrekov so zelo podobne, čeprav so dokazi teh
 izrekov bistveno različni. Formulacije se celo ne bi razlikovale, če bi
  privzeli, da se vse vzporednice (v eni smeri) neke ravnine sekajo v
  isti točki v neskončnosti ter da vse točke v neskončnosti neke ravnine
  določajo natanko eno premico v neskončnosti. To je pravzaprav bila glavna motivacija za razvoj projektivne geometrije,
  v kateri se vsaki dve premici v ravnini sekata (glej \cite{Mitrovic}).

  V nadaljevanju bomo videli nekaj posledic Desarguesovega izreka.



              \bzgled \label{zgled 3.2}
             Let $p$, $q$ and $r$ be three lines in the plane which
            intersect at the same point, and points $A$, $B$ and $C$ of this plane which do not belong
            to these lines. Construct a triangle whose vertices belong to the given
            lines, and the sides contain the given points.
             \ezgled



\begin{figure}[!htb]
\centering
\input{sl.pod.7.10D.4.pic}
\caption{} \label{sl.pod.7.10D.4.pic}
\end{figure}


 \textbf{\textit{Solution.}} (Figure \ref{sl.pod.7.10D.4.pic})

  Naj bo $PQR$ tak trikotnik, da
 njegova oglišča $P$, $Q$ in $R$ pripadajo premicam $p$, $q$ in
 $r$, stranice $QR$, $PR$ in $PQ$ pa vsebujejo točke $A$, $B$ in
 $C$. Naj bo $S$  skupna točka premic $p$, $q$ in $r$.

 Če je $P'Q'R'$ poljuben trikotnik, ki je perspektiven trikotniku
 $PQR$ glede na središče $S$, katerega stranici $R'Q'$ in $R'P'$
 vsebujeta točki $A$ in $B$ (izpuščen je torej pogoj glede na točko
 $C$), sta po Desarguesovem \ref{izrekDesarguesEvkl} izreku trikotnika $PQR$ in $P'Q'R'$
 perspektivna glede na neko os $s$. Torej os $s$, vsebuje točke
 $A$, $B$ in $Z=PQ\cap P'Q'$.

 Trikotnik $PQR$ konstruiramo tako, da najprej
 konstruiramo pomožni trikotnik $P'Q'R'$, pri katerem je točka
 $R'\in r$ poljubna. Potem konstruiramo točko $Z$ kot
 presečišče premic $AB$ in $P'Q'$. S točkama $Z$ in $C$ je
 določena stranica $PQ$.

 V dokazu, da je $PQR$ iskani trikotnik, uporabimo obratni
 Desargeusov izrek \ref{izrekDesarguesObr}.
  \kdokaz



%________________________________________________________________________________
\poglavje{Power of a Point} \label{odd7Potenca}

Med najbolj zanimive značilnosti krožnice, ki izpostavlja nekatere njene
metrične lastnosti, je potenca tačke\footnote{Besedo potenca je v tem pomenu prvi uporabil švicarski matematik \index{Steiner, J.}\textit{J. Steiner} (1769--1863).}. Pred prehodom na definicijo dokažimo naslednji izrek.



        \bizrek \label{izrekPotenca}
        Suppose that $P$ is an arbitrary point in the plane of a circle $k(S,r)$.
            For any line of this plane containing the point $P$ and intersecting the circle $k$ at points
            $A$ and $B$, the expression $\overrightarrow{PA}\cdot \overrightarrow{PB}$
             (Figure \ref{sl.pod.7.12.1b.pic})
         is constant, furthermore:
        $$\overrightarrow{PA}\cdot \overrightarrow{PB} = |PS|^2 - r^2.$$
       If $P$ is an exterior point of the circle $k$ and $PT$ its tangent at a point $T$, then:
           $$\overrightarrow{PA}\cdot \overrightarrow{PB} = |PT|^2.$$
        \eizrek



\begin{figure}[!htb]
\centering
\input{sl.pod.7.12.1b.pic}
\caption{} \label{sl.pod.7.12.1b.pic}
\end{figure}


\textbf{\textit{Proof.}} Obravnavali bomo tri možne primere.


\textit{1)} (Figure \ref{sl.pod.7.12.1.pic})

Naj bo $P$  zunanja točka krožnice $k$. V tem primeru ni $\mathcal{B}(A,P,B)$, zato je (ekvivalenca \ref{eqnMnozVektRelacijaB} iz razdelka \ref{odd5DolzVekt}):
\begin{eqnarray} \label{eqnPotenIzr1}
\overrightarrow{PA}\cdot \overrightarrow{PB}>0.
\end{eqnarray}
 Brez škode za splošnost predpostavimo, da velja $\mathcal{B}(P,A,B)$. Ker je $\angle PTA\cong\angle TBA=\angle TBP$ (izrek \ref{ObodKotTang}) in $\angle TPA=\angle BPT$,
sta si trikotnika $PAT$ in $PTB$  podobna  (izrek \ref{PodTrikKKK}), zato je
$PA:PT=PT:PB$. Če uporabimo še Pitagorov izrek, dobimo:
 $$|PA|\cdot |PB| = |PT|^2 = |PS|^2 - r^2.$$
 Iz tega zaradi relacije \ref{eqnPotenIzr1} sledi:
$$\overrightarrow{PA}\cdot \overrightarrow{PB} = |PS|^2 - r^2.$$

\begin{figure}[!htb]
\centering
\input{sl.pod.7.12.1.pic}
\caption{} \label{sl.pod.7.12.1.pic}
\end{figure}

\textit{2)} (Figure \ref{sl.pod.7.12.1a.pic})

Če točka $P$ leži na krožnici $k$, je $P=A$ ali $P=B$, torej:
$$\overrightarrow{PA}\cdot \overrightarrow{PB} = 0 = |PS|^2 - r^2.$$

\begin{figure}[!htb]
\centering
\input{sl.pod.7.12.1a.pic}
\caption{} \label{sl.pod.7.12.1a.pic}
\end{figure}


\textit{3)} (Figure \ref{sl.pod.7.12.1a.pic})

Naj bo $P$  notranja točka krožnice $k$. V tem primeru je $\mathcal{B}(A,P,B)$, zato je (ekvivalenca \ref{eqnMnozVektRelacijaB} iz razdelka \ref{odd5DolzVekt}):
\begin{eqnarray} \label{eqnPotenIzr2}
\overrightarrow{PA}\cdot \overrightarrow{PB}<0.
\end{eqnarray}
Naj bosta
$A_1$ in $B_1$ presecišči premice $SP$ s krožnico $k$ (brez škode za splošnost naj bo $\mathcal{B}(A_1,S,P)$). Zaradi skladnosti ustreznih
obodnih kotov (izrek \ref{ObodObodKot}) je $\triangle APA_1\sim \triangle B_1PB$ (izrek \ref{PodTrikKKK}), zato je $AP:B_1P = PA_1:PB$, torej
$$|PA|\cdot |PB|=|PA_1|\cdot |PB_1|=
\left(r+|PS|\right)\cdot\left(r-|PS|\right)=r^2-|PS|^2.$$
 Iz tega zaradi relacije \ref{eqnPotenIzr1} sledi
$$\overrightarrow{PA}\cdot \overrightarrow{PB} = |PS|^2 - r^2,$$ kar je bilo treba dokazati. \kdokaz

Konstanten produkt $\overrightarrow{PA}\cdot \overrightarrow{PB}$ iz prejšnjega izreka \ref{izrekPotenca} imenujemo
\index{potenca točke}\pojem{potenca točke} $P$ glede na krožnico $k$ in jo označimo $p(P,k)$.

Po prejšnjem izreku (\ref{izrekPotenca}) je torej potenca točke $P$ glede na krožnico $k(S,r)$ število $|PS|^2 - r^2$.
To število je pozitivno, negativno ali nič, odvisno od tega, ali je $P$ zunanja ali notranja ali točka na  kružnici $k$. Torej:
 \begin{eqnarray*}
 p(P,k)\hspace*{1mm}\left\{
                      \begin{array}{ll}
                        >0, & \textrm{če je } OP>r; \\
                        =0, & \textrm{če je } OP=r; \\
                        <0, & \textrm{če je } OP<r.
                      \end{array}
                    \right.
\end{eqnarray*}


V nadaljevanju nas bo zanimalo, kaj predstavlja množica takšnih točk, da sta potenci glede na dve dani krožnici enaki. Dokažimo najprej pomožno trditev.



                \bizrek \label{PotencOsLema}
                Let $A$ and $B$ be points and $d$  a line segment in the plane.
                A set of all points $X$ in this plane such that
                $$|AX|^2-|BX|^2=|d|^2,$$
                is a line perpendicular to the line $AB$.
                \eizrek

\begin{figure}[!htb]
\centering
\input{sl.pod.7.12.3.pic}
\caption{} \label{sl.pod.7.12.3.pic}
\end{figure}

\textbf{\textit{Proof.}}
Naj bo $D$ takšna točka te ravnine, da velja:
$DB\perp BA$ in $DB\cong d$. Z $X_0$ označimo presečišče
premice $AB$ in simetrale daljice $AD$ (Figure \ref{sl.pod.7.12.3.pic}).
Ker je $DBX_0$ pravokotni trikotnik s hipotenuzo $X_0D$, iz Pitagorovega izreka (\ref{PitagorovIzrek}) sledi, da za
točko $X_0$ velja:
\begin{eqnarray} \label{eqnPotencaLema1}
 |AX_0|^2-|BX_0|^2=|DX_0|^2-|BX_0|^2=|DB|^2=|d|^2.
\end{eqnarray}
Dokažimo, da je iskana množica točk pravokotnica $x$ premice
$AB$ v točki $X_0$. Dovolj je dokazati ekvivalenco:
$$X\in x \hspace*{1mm} \Leftrightarrow \hspace*{1mm} |AX|^2-|BX|^2=|d|^2$$

($\Rightarrow$) Če je $X\in x$, je po Pitagorovem izreku (\ref{PitagorovIzrek}) za pravokotna trikotnika $XX_0A$ in $XX_0B$ ter po trditvi \ref{eqnPotencaLema1}:
 \begin{eqnarray*}
 |AX|^2-|BX|^2&=&\left(|AX_0|^2+|XX_0|^2\right) -\left(|BX_0|^2-|XX_0|^2\right)=\\
&=&|AX_0|^2-|BX_0|^2=|d|^2
 \end{eqnarray*}

($\Leftarrow$) Predpostavimo sedaj, da velja $|AX|^2-|BX|^2=|d|^2$.
 Naj bo $X'$ pravokotna projekcija točke $X$ na premici $AB$. Po Pitagorovem izreku (\ref{PitagorovIzrek}) za pravokotna trikotnika $XX'A$ in $XX'B$ velja:
\begin{eqnarray*}
 |d|^2=|AX|^2-|BX|^2&=&\left(|AX'|^2+|XX'|^2\right) -\left(|BX'|^2-|XX'|^2\right)=\\
&=&|AX'|^2-|BX'|^2.
 \end{eqnarray*}
 Če to povežemo z relacijo \ref{eqnPotencaLema1}, dobimo:
\begin{eqnarray*}
 |d|^2=|AX'|^2-|BX'|^2=|AX_0|^2-|BX_0|^2,
 \end{eqnarray*}
oz. (če upoštevamo še lastnost \ref{eqnMnozVektDolzina} množenja kolinearnih vektorejev):
\begin{eqnarray*}
&&|AX'|^2-|X'B|^2=|AX_0|^2-|X_0B|^2\\
&\Rightarrow&
\left(\overrightarrow{AX'}- \overrightarrow{X'B}\right)\left(\overrightarrow{AX'}+ \overrightarrow{X'B}\right)=
\left(\overrightarrow{AX_0}- \overrightarrow{X_0B}\right)\left(\overrightarrow{AX_0}+ \overrightarrow{X_0B}\right)\\
  &\Rightarrow&
\left(\overrightarrow{AX'}- \overrightarrow{X'B}\right)\overrightarrow{AB}=
\left(\overrightarrow{AX_0}- \overrightarrow{X_0B}\right)\overrightarrow{AB}\\
  &\Rightarrow&
\overrightarrow{AX'}- \overrightarrow{X'B}=
\overrightarrow{AX_0}- \overrightarrow{X_0B}\\
  &\Rightarrow&
2\overrightarrow{AX'}- \overrightarrow{AB}=
2\overrightarrow{AX_0}- \overrightarrow{AB}\\
  &\Rightarrow&
\overrightarrow{AX'}=
\overrightarrow{AX_0}\\
  &\Rightarrow&
X'=X_0.
 \end{eqnarray*}
To pomeni, da je $XX_0\perp AB$ oz. $X\in x$.
\kdokaz



        \bizrek \label{PotencnaOs}
        A set of all points that have the same power with respect to
             two non-concentric circles is a line perpendicular
             to the line containing the centres of these two circles
         (Figure \ref{sl.pod.7.12.4.pic}).\\
        If the circles intersect, this line is their common secant.\\
        However, if the circles touch each other, this is their common tangent through their touching point.
        \eizrek

\begin{figure}[!htb]
\centering
\input{sl.pod.7.12.4.pic}
\caption{} \label{sl.pod.7.12.4.pic}
\end{figure}


\textbf{\textit{Proof.}}

Naj bosta $k_1(S_1,r_1)$ in $k_2(S_2,r_2)$ (brez škode za splošnost naj bo $r_1\geq r_2$) poljubni krožnici. Točka $P$ pripada iskani množici
natanko tedaj, ko je $p(P,k_1)=p(P,k_2)$ oz. $|S_1P|^2-r_1^2=|S_2P|^2-r_2^2$; slednja relacija je pa ekvivalentna z: $$|S_1P|^2-|S_2P|^2=r_1^2-r_2^2.$$
Ker je po predpostavki $r_1^2-r_2^2\geq 0$. Če je $r_1^2-r_2^2>0$, po prejšnjem izreku (\ref{PotencOsLema}) iskana množica
predstavlja pravokotnico premice $S_1S_2$. V primeru $r_1=r_2$ je jasno $|S_1P|=|S_2P|$, torej je iskana množica simetrala daljice $S_1S_2$.
Označimo to premico s $p$.

V posebnem primeru, ko se krožnici sekata v točkah $A$ in $B$, je $p(A,k_1)=p(A,k_2)=0$ in $p(B,k_1)=p(B,k_2)=0$. To pomeni, da točki $A$ in $B$ ležita na iskani premici $p$, torej je to ravno premica $AB$.

Če se krožnici dotikata v točki $T$, je $p(T,k_1)=p(T,k_2)=0$. To pomeni, da točka $T$ leži na premici $p$, ki je pravokotnica premice $S_1S_2$. Torej je premica $p$ skupna tangenta dveh danih krožnic skozi njuno dotikališče $T$.
\kdokaz

 Premica $p$ iz prejšnjega izreka se imenuje \index{potenčna!premica}
   \pojem{potenčna premica} dveh
 krožnic. Potenčno premico krožnic $k_1$ in $k_2$ bomo označili z $p(k_1,k_2)$.

Zanimivo je ugotoviti kako efektivno konstruiramo potenčno premico dveh danih krožnic  $k_1$ in $k_2$.
 V primerih, ko se krožnici sekata ali dotikata, smo že dali odgovor v izreku \ref{PotencnaOs} (Figure \ref{sl.pod.7.12.4.pic}).
 Ostane nam še konstrukcija potenčne premice v primeru, ko krožnici nimata skupnih točk. Ena možnost je, da direktno uporabimo izrek \ref{PotencOsLema}. Nekoliko hitrejši postopek je vezan za konstrukcijo
pomožne krožnice $l$, ki seka dani krožnici v točkah $A$ in $B$ oziroma $C$ in $D$. Potem presečišče
premic $AB$ in $CD$ - točka $X$ - leži na iskani potenčni premici $p(k_1,k_2)$. Res, iz $p(k_1,l)=AB$ in $p(k_2,l)=CD$ sledi $X\in p(k_1,l)$ in $X\in p(k_1,l)$ oz. $p(X,p_1)=p(X,l)=p(X,p_2)$.

V izreku \ref{PotencnaOs} ni obravnavan en primer - ko
 sta $k_1(S,r_1)$ in $k_2(s,r_2)$ koncentrični krožnici. V tem primeru je omenjena množica točk prazna množica. To dobimo iz pogoja
$|S_1P|^2-|S_2P|^2=r_1^2-r_2^2$. Če je namreč $S_1=S_2$ in $r_1\neq r_2$, dobimo pogoj $0=r_1^2-r_2^2\neq 0$, ki ni izpolnjen za nobeno točko $P$.


Definirajmo še nekatere pojme, ki so v zvezi s potenčno premico dveh krožnic.

Množico vseh takšnih krožnic neke ravnine, da imata vsaki dve potenčno premico $p$, ki je enaka za vsak par krožnic te množice, imenujmo \index{šop krožnic}\pojem{šop krožnic}. Premica $p$ je \index{potenčna!os}\pojem{potenčna os} tega šopa (Figure \ref{sl.pod.7.12.5.pic}).

\begin{figure}[!htb]
\centering
\input{sl.pod.7.12.5.pic}
\caption{} \label{sl.pod.7.12.5.pic}
\end{figure}

Naj bo $p$ potenčna os nekega šopa krožnic. Iz izreka \ref{PotencnaOs} je jasno, da vsa središča krožnic tega šopa ležijo na isti premici $s$, ki je pravokotna na premico $p$.
Obravnavali bomo tri primere.

\textit{1)} Če se vsaj dve krožnici šopa sekata v točkah $A$ in $B$, je $p=AB$, zato vse krožnice tega šopa potekajo skozi točki $A$ in $B$. V tem primeru pravimo, da gre za \index{šop krožnic!eliptični}\pojem{eliptični šop} krožnic.

\textit{2)} Če se vsaj dve krožnici šopa dotikata v točki $T$, je $p$ pravokotnica premice $s$ v točki $T$, zato imajo vse krožnice skupno tangento $p$ v točki $T$. V tem primeru pravimo, da gre za \index{šop krožnic!parabolični}\pojem{parabolični šop} krožnic.

\textit{3)} Če nobeni dve krožnici šopa nimata skupnih točk, pravimo, da gre za \index{šop krožnic!hiperbolični}\pojem{hiperbolični šop} krožnic. V tem primeru velja naslednja lastnost: če poljubna krožnica seka krožnice tega šopa (ni nujno, da seka vse) v točkah $A_i$ in $B_i$, potem vse premice $A_iB_i$ potekajo skozi eno točko, ki leži na premici $p$.




        \bizrek \label{PotencnoSr}
        Let $k$, $l$ and $j$ be three non-concentric circles with non-collinear centres.
             Then there is exactly one point that has the same power with respect to all three circles.
             This point is the intersection of their three radical axes $p(k,l)$, $p(l,j)$ and $p(k,j)$.
        \eizrek


\textbf{\textit{Proof.}} (Figure \ref{sl.pod.7.12.6.pic}) Ker so središča krožnic $k$, $l$ in $j$  tri nekolinearne točke, nobeni dve od premic $p(k,l)$, $p(l,j)$ in $p(k,j)$ nista vzporedni. Naj bo $P=p(k,l)\cap p(l,j)$. Potem je
$$p(P,k)=p(P,l)=p(P,j),$$
 oz. $P\in p(k,j)$, kar pomeni, da se potenčne premice  $p(k,l)$, $p(l,j)$ in $p(k,j)$ sekajo v točki $P$.

Če za neko drugo točko $\widehat{P}$ te ravnine velja $p(\widehat{P},k)=p(\widehat{P},l)=p(\widehat{P},j)$, je $\widehat{P}\in p(k,l),\hspace*{1mm}p(l,j),\hspace*{1mm}p(k,j)$, zato je $\widehat{P}=P$.
\kdokaz

          Točka iz prejšnjega izreka (\ref{PotencnoSr}) se imenuje \index{središče!potenčno}
          \pojem{potenčno središče} treh krožnic. Potenčno središče krožnic  $k$, $l$ in $j$ bomo označili s $p(k,l,j)$.


\begin{figure}[!htb]
\centering
\input{sl.pod.7.12.6.pic}
\caption{} \label{sl.pod.7.12.6.pic}
\end{figure}

 V primeru, ko so središča treh krožnic tri kolinearne točke, krožnice pa niso iz istega šopa, so vse tri potenčne premice vzporedne, ker so vse pravokotne na skupno centralo teh krožnic (izrek \ref{PotencnaOs}).
 Iz tega in izreka \ref{PotencnoSr} direktno sledi naslednji izrek (Figure \ref{sl.pod.7.12.6.pic}).


Let $k$, $l$ and $j$ be three non-concentric circles with non-collinear centres.
             Then there is exactly one point that has the same power with respect to all three circles.
             This point is the intersection of their three radical axes $p(k,l)$, $p(l,j)$ and $p(k,j)$.

                Radical axes of three circles in the plane that are not
                from the same pencil  and no two of them are concentric,
                belong to the same family of lines.


                \bizrek \label{PotencnoSrSop}
                Radical axes of three circles in the plane that are not
                from the same pencil  and no two of them are concentric,
                belong to the same family of lines.
                \eizrek

Poseben primer izreka \ref{PotencnoSr}, ko se vsaki dve premici sekata, je naslednja trditev.


                \bzgled
                Let $k$, $l$, and $j$ be three circles of some plane with nonlinear
                centres and:
                \begin{itemize}
                  \item  $A$ and $B$ intersections of the circles $k$ and $l$,
                  \item  $C$ and $D$ intersections of the circles $l$ and $j$,
                 \item  $E$ and $F$ intersections of the circles $j$ and $k$.
                \end{itemize}
                Prove that the lines $AB$, $CD$ and $EF$ intersect at a single point.
                \ezgled

\begin{figure}[!htb]
\centering
\input{sl.pod.7.12.7.pic}
\caption{} \label{sl.pod.7.12.7.pic}
\end{figure}


\textbf{\textit{Proof.}} (Figure \ref{sl.pod.7.12.7.pic})

Po izreku \ref{PotencnaOs} so $p(k,l)=AB$, $p(l,j)=CD$ in $p(j,k)=EF$ ustrezne potenčne osi. Po izreku \ref{PotencnoSr} se le-te sekajo v eni točki - potenčnem središču $P=p(k,l,j)$ teh treh krožnic.
\kdokaz

Zelo podobna je tudi naslednja trditev.




                \bzgled
                Let $k$, $l$, and $j$ be three circles of some plane with nonlinear
                centres and:
                \begin{itemize}
                  \item $t_1$ the common tangent of the circles  $k$ in $l$,
                  \item $t_2$ the common tangent of the circles $l$ in $j$,
                  \item $t_3$ the common tangent of the circles $j$ in $k$.
                \end{itemize}
                Prove that the lines $t_1$, $t_2$ and $t_3$ intersect at a single point.
                \ezgled

\begin{figure}[!htb]
\centering
\input{sl.pod.7.12.8.pic}
\caption{} \label{sl.pod.7.12.8.pic}
\end{figure}


\textbf{\textit{Proof.}} (Figure \ref{sl.pod.7.12.8.pic})

Po izreku \ref{PotencnaOs} so $p(k,l)=t_1$, $p(l,j)=t_2$ in $p(j,k)=t_3$ ustrezne potenčne osi. Po izreku \ref{PotencnoSr} se le-te sekajo v eni točki - potenčnem središču $P=p(k,l,j)$ teh treh krožnic.
\kdokaz



                \bzgled
                   a) Suppose that circles $k$ and $l$ are touching each other externally, and a line $t$ is
                    is the common tangent of these circles at their common point. Let $AB$ be a second
                    common tangent of these circles at touching points $A$ and $B$. Prove that the midpoint
                    the line segment $AB$ lies on the tangent $t$.\\
                   b) Suppose that circles $k$ and $l$ intersect at points $P$ and $Q$. Let $AB$ be a common tangent
                    of these circles at touching points  $A$ and $B$. Prove that the midpoint of the line segment $AB$ lies on
                    line $PQ$.
                \ezgled

\begin{figure}[!htb]
\centering
\input{sl.pod.7.12.9.pic}
\caption{} \label{sl.pod.7.12.9.pic}
\end{figure}


\textbf{\textit{Proof.}} (Figure \ref{sl.pod.7.12.9.pic})

\textit{a)} Naj bo $S$ središče daljice $AB$. Potem je:
$p(S,k)= |SA|^2 = |SB|^2 = p(S,l)$,
zato točka $S$ leži na potenčni premici krožnic $k$ in $l$ oz. premici $t$ (izrek \ref{PotencnaOs}).

\textit{b)} Enako kot v prejšnjem primeru, le da je potenčna premica krožnic $k$ in $l$ v tem primeru premica $AB$.
\kdokaz



        \bizrek \label{EulerjevaFormula}
        \index{formula!Eulerjeva}
        (Euler's\footnote{\index{Euler, L.}
        \textit{L. Euler}
        (1707--1783), švicarski matematik.} formula) If $k(S,r)$ is the incircle and $l(O,R)$ the circumcircle
of an arbitrary triangle, then
        $$|OS|^2=R^2- 2Rr.$$
        \eizrek

\begin{figure}[!htb]
\centering
\input{sl.pod.7.12.10.pic}
\caption{} \label{sl.pod.7.12.10.pic}
\end{figure}


\textbf{\textit{Proof.}} (Figure \ref{sl.pod.7.12.10.pic})

Označimo z $A$, $B$ in $C$ oglišča trikotnika, $NM$ premer očrtane krožnice $l$, ki je pravokoten na
stranico $BC$ (in še $A,N\div BC$). Po izreku \ref{TockaN} leži točka $N$ na bisekstrisi notranjega kota pri oglišču $A$ oz. na poltraku
$AS$ (izrek \ref{SredVcrtaneKrozn}). Po izreku \ref{TockaN.NBNC} je $NS\cong NC$. Če uporabimo še potenco točke $S$ glede na krožnico $l$ (izrek \ref{PotencnoSr}) in dejstvo $\mathcal{B}(A,S,N)$, dobimo $p(S,l)= |SO|^2 - R^2 = \overrightarrow{SA}\cdot \overrightarrow{SN}
 = -|SA|\cdot |SN|
= -|SA|\cdot |CN|$, torej:
 \begin{eqnarray} \label{eqnEulFormOS}
  |SO|^2 - R^2 =  -|SA|\cdot |CN|.
\end{eqnarray}
 Označimo s $Q$ dotikališče včrtane krožnice $k$  s stranico $AC$ trikotnika $ABC$.
 Po izrekih \ref{TangPogoj} in \ref{TalesovIzrKroz2} je $\angle AQS\cong\angle MCN=90^0$, iz izreka \ref{ObodObodKot} sledi še $\angle SAQ=\angle NAC\cong\angle NMC$.
Torej sta si trikotnika $AQS$ in $MCN$ podobna (izrek \ref{PodTrikKKK}), zato je:
$$\frac{AS}{MN}=\frac{SQ}{NC}$$ oz. $|AS|\cdot |NC|=|MN|\cdot |SQ|=2Rr$. Če to vstavimo v relacijo \ref{eqnEulFormOS}, dobimo:
$$|SO|^2 = R^2-|SA|\cdot |CN|=R^2-2Rr,$$
kar je bilo treba dokazati. \kdokaz


Naslednja naloga je poseben primer prejšnjega izreka in je torej
njegova posledica (Figure \ref{sl.pod.7.12.10a.pic}).

\begin{figure}[!htb]
\centering
\input{sl.pod.7.12.10a.pic}
\caption{} \label{sl.pod.7.12.10a.pic}
\end{figure}


            \bnaloga\footnote{4. IMO, Czechoslovakia - 1962, Problem 6.}
            Consider an isosceles triangle. Let $r$ be the radius of its circumscribed circle
            and $\rho$ the radius of its inscribed circle. Prove that the distance $d$ between
            the centres of these two circles is $$d = \sqrt{r(r-2\rho)}.$$
          \enaloga

Tudi naslednja trditev je direktna posledica izreka \ref{EulerjevaFormula}.


                \bizrek
                If $k(S,r)$ is the incircle and $l(O,R)$ the circumcircle
                of an arbitrary triangle, then
                $$R\geq 2r.$$
            Equality is achieved for an equilateral triangle.
                \eizrek

 Naslednja načrtovalna naloga je eden od desetih Apolonijevih problemov o dotiku krožnic, ki jih bomo bolj podrobno raziskovali v razdelku \ref{odd9ApolDotik}.



                \bzgled
                Construct a circle through two given points $A$ and $B$ and tangent to a given line $t$.
                \ezgled

\begin{figure}[!htb]
\centering
\input{sl.pod.7.12.11.pic}
\caption{} \label{sl.pod.7.12.11.pic}
\end{figure}


\textbf{\textit{Solution.}} (Figure \ref{sl.pod.7.12.11.pic})

Naj bo $k$ iskana krožnica, ki poteka skozi točki $A$ in $B$ in se dotika premice $t$ v točki $T$.

Če sta premici $AB$ in $t$ vzporedni,
tretjo točko $T$ krožnice $k$ dobimo kot presečišče premice
$t$ s simetralo daljice $AB$.

Naj bo $P$ presečišče premic $AB$ in $t$.

Uporabili bomo potenco točke $P$ glede na
 krožnico $k$. Z $l$ označimo poljubno
krožnico, ki poteka skozi točki $A$ in $B$. Po izreku \ref{PotencnaOs} je $AB$
potenčna premica krožnic $k$ in $l$, zato je $p(P,k)=p(P,l)$. Označimo s $PT$ in
$PT_1$ tangenti krožnic $k$ in $l$ v njunih točkah $P$ in $P_1$. Potem velja (izrek \ref{izrekPotenca}):
 $$|PT|^2=p(P,k)=p(P,l)=|PT_1|^2$$
oz. $|PT|=|PT_1|$.

Zadnja relacija nam omogoča konstrukcijo tretje točke $T$ krožnice $k$.
 \kdokaz



            \bzgled
            Let $E$ be the intersection of the bisector of the interior angle at the vertex
            $A$ with the side $BC$ of a triangle $ABC$ and $A_1$ the midpoint of this side.
            Let $P$ and $Q$
            be intersections of the circumcircle  of the triangle $AEA_1$ with the sides $AB$
            and $AC$ of the triangle $ABC$. Prove that:
             $$BP\cong CQ.$$
            \ezgled

\begin{figure}[!htb]
\centering
\input{sl.pod.7.12.12.pic}
\caption{} \label{sl.pod.7.12.12.pic}
\end{figure}


\textbf{\textit{Proof.}} (Figure \ref{sl.pod.7.12.12.pic})

Označimo s $k$ očrtano krožnico trikotnika $AEA_1$. Če uporabimo potenco točk $B$ in $C$ glede na krožnico $k$, relaciji $\mathcal{B}(B,P,A)$ in $\mathcal{B}(C,Q,A)$ (po predpostavki točki $P$ in $Q$ ležita na stranicah $AB$ oz. $AC$ trikotnika $ABC$) ter
ekvivalenco \ref{eqnMnozVektRelacijaB} iz razdelka \ref{odd5DolzVekt}, dobimo:
 \begin{eqnarray*}
p(B,k)&=&|BP|\cdot |BA|=|BE|\cdot |BA_1|,\\
p(C,k)&=&|CP|\cdot |CA|=|CE|\cdot |CA_1|.
\end{eqnarray*}
 Iz tega in relacije $BA_1\cong CA_1$ ter izreka \ref{HarmCetSimKota} dobimo:
\begin{eqnarray*}
\frac{|BP|\cdot |BA|}{|CP|\cdot |CA|}
=\frac{|BE|\cdot |BA_1|}{|CE|\cdot |CA_1|}
=\frac{|BE|}{|CE|}
=\frac{|BA|}{|CA|}.
\end{eqnarray*}
Torej velja $\frac{|BP|\cdot |BA|}{|CP|\cdot |CA|}=\frac{|BA|}{|CA|}$ oz. $|BP|=|CP|$.
 \kdokaz

            \bzgled
            Construct a circle that is perpendicular to three given circles
            $k$, $l$ in $j$.
            \ezgled

\begin{figure}[!htb]
\centering
\input{sl.pod.7.12.13.pic}
\caption{} \label{sl.pod.7.12.13.pic}
\end{figure}


\textbf{\textit{Solution.}} (Figure \ref{sl.pod.7.12.13.pic})

Predpostavimo najprej, da so središča teh krožnic $k$, $l$ in $j$ nekolinearne
točke. Naj bo $x$ iskana krožnica s središčem $P$, ki je pravokotna na krožnice  $k$, $l$ in $j$ ter $A\in x\cap k$, $B\in x\cap l$ in $C\in x\cap j$. Ker je $x\perp k,j,l$, so $PA$, $PB$ in $PC$ tangente teh krožnic iz točke $P$ (izrek \ref{pravokotniKroznici}). Po izreku \ref{izrekPotenca} je $p(P,k)=|PA|^2$, $p(P,l)=|PB|^2$ in $p(P,j)=|PC|^2$. Ker točke $A$, $B$ in $C$ ležijo na krožnici $x$ s središčem $P$, je $|PA|^2=|PB|^2=|PC|^2$ oz. $p(P,k)=p(P,l)= p(P,j)$. To pomeni, da je $P=p(k,l,j)$ potenčno središče krožnic $k$, $l$ in $j$.

Iskano krožnico $x$ lahko torej načrtamo tako, da najprej narišemo njeno središče $P=p(k,l,j)$, nato polmer $PA$, kjer je premica $PA$ tangenta krožnice $k$ v točki $A$.
Jasno je, da v primeru, ko je $P$ notranja točka neke od krožnic $k$, $l$, $j$, naloga nima rešitev.

Tudi v primeru, ko so središča
krožnic kolinearne točke, iskana krožnica ne
obstaja (le-ta je ‘‘degenerirana
krožnica’’ oz. premica, ki predstavlja njihovo skupno centralo).
 \kdokaz



            \bzgled
            Circles $k(O,r)$ and $l(S,\rho)$ and a point $P$ in the same
             plane are given.
            Construct a line passing through the point $P$, which determine congruent chords
            on the circles $k$ and $l$.
            \ezgled


\begin{figure}[!htb]
\centering
\input{sl.pod.7.12.14.pic}
\caption{} \label{sl.pod.7.12.14.pic}
\end{figure}


\textbf{\textit{Solution.}} (Figure \ref{sl.pod.7.12.14.pic})

Naj bo $p$ premica, ki poteka skozi točko $P$,
krožnici $k$ in $l$ pa seka v takšnih točkah $A$ in $B$ oz. $C$ in $D$, da je
$AB\cong CD$. Naj bo
$\overrightarrow{v}$ vektor, ki je določen s
središčema daljic $AB$ in $CD$.
Potem je
$\mathcal{T}_{\overrightarrow{v}}:\hspace*{1mm}A,B\mapsto C,D$,
krožnica $k$ se s to translacijo  preslika v
krožnico $k'$, ki poteka skozi točki $C$ in $D$. Torej se
krožnici $k'$ in $l$ sekata v točkah $C$ in $D$.
Tako se problem konstrukcije premice $p$ prevede na problem
 konstrukcije vektorja
 $\overrightarrow{v}$ oz. točke
$O'= \mathcal{T}_{\overrightarrow{v}}(O)$, ki predstavlja središče krožnice $k'$.

Premica, ki
je določena s središčema $S$ in $O'$ krožnic $l$ in $k'$, je pravokotna na njuni skupni tetivi $CD$.
Ker je $\overrightarrow{OO'}= \overrightarrow{v}  \parallel p$, sledi
 $$\angle OO'S=90^0.$$
Po izreku \ref{TalesovIzrKroz2} točka $O'$ leži na krožnici nad premerom $OS$.

Točka $P$ leži na potenčni premici $p$ krožnic $k'$ in $l$ (izrek \ref{PotencnaOs}). Iz tega sledi $PL\cong PK$, kjer sta $PL$ in
$PK$ tangenti krožnic $l$ in $k'$ v točkah $L$ in $K$.
To pomeni, da lahko konstruiramo trikotnik, ki je skladen s pravokotnim trikotnikom $PKO'$ ($PK\cong PL$, $\angle PO'K=90^0$ in $KO'\cong r$), s tem pa tudi
 $d$, ki je skladna z daljico $PO'$. Torej točka $O'$ pripada
preseku krožnice s središčem $P$ in polmerom $d$ ter krožnice nad premerom $OS$.
 \kdokaz



        \bnaloga\footnote{36. IMO Canada - 1995, Problem 1.}
        Let $A$, $B$, $C$, $D$ be four distinct points on a line, in that order $\mathcal{B}(A,B,C,D)$. The
        circles with diameters $AC$ and $BD$ intersect at $X$ and $Y$. The line $XY$
        meets $BC$ at $Z$. Let $P$ be a point on the line $XY$ other than $Z$. The
        line $CP$ intersects the circle with diameter $AC$ at $C$ and $M$, and the
        line $BP$ intersects the circle with diameter $BD$ at $B$ and $N$. Prove
        that the lines $AM$, $DN$, $XY$ are concurrent.
        \enaloga

\begin{figure}[!htb]
\centering
\input{sl.pod.7.12.IMO1.pic}
\caption{} \label{sl.pod.7.12.IMO1.pic}
\end{figure}

\textbf{\textit{Solution.}} Označimo s $k_1$ in $k_2$ krožnici
nad polmeroma $AC$ in $BD$ (Figure \ref{sl.pod.7.12.IMO1.pic}). Po
izreku \ref{KroznPresABpravokOS} je premica $XY$ pravokotna na
centrali $BC$ teh dveh krožnic  in je hkrati njuna potenčna os
(izrek \ref{PotencnaOs}). Naj bo $S_1=AM\cap XY$ in $S_2=DN\cap
XY$. Dovolj je dokazati $S_1=S_2$.

  Ker $M\in k_1$, je po izreku \ref{TalesovIzrek} $\angle
  AMC=90^0$ oz. $AS_1\perp MC$. Zaradi tega je
  $\angle MCA\cong\angle AS_1Z$ (kota s pravokotnimi kraki
   - izrek \ref{KotaPravokKraki}). Iz slednje skladnosti kotov
   sledi, da sta si pravokotna trikotnika $AZS_1$ in $PZC$ podobna.
   Zato je $\frac{AZ}{PZ}=\frac{ZS_1}{ZC}$ oz.
   $|ZS_1|=\frac{|ZC|\cdot |ZA|}{|PZ|}$. Toda $|ZC|\cdot
   |ZA|=p(Z,k_1)=|ZX|\cdot
   |ZY|=|ZX|^2$. Iz tega sledi:
   $$|ZS_1|=\frac{|ZX|^2}{|PZ|}.$$
   Na enak način se dokaže, da velja tudi:
 $|ZS_2|=\frac{|ZX|^2}{|PZ|}$. Ker sta $S_1$ in $S_2$ na
 istem poltraku $ZP$, je $S_1=S_2$.
  \kdokaz
     


                \bzgled 
                Let $P$ be an arbitrary point in the plane of a triangle $ABC$ which
                does not lie on any of lines containing altitudes of this triangle. Suppose $A_1$ is a point,
                in which a perpendicular line  of the line $AP$ at the point $P$ intersects the line $BC$. Analogously
                we can also define points $B_1$ and $C_1$.
                Prove that $A_1$, $B_1$ and $C_1$ are three collinear points.
                \ezgled

\textbf{\textit{Solution.}} Označimo z $A_C$ in $B_C$ pravokotni
projekciji oglišč $A$ in $B$ na premici $CP$. Analogno sta $A_B$
in $C_B$ pravokotni projekciji oglišč $A$ in $C$ na premici $BP$
ter $B_A$ in $C_A$ pravokotni projekciji oglišč $B$ in $C$ na
premici $AP$ (Figure \ref{sl.pd.7.4.6.pic}). Po Talesovem izreku \ref{TalesovIzrek}
je:
 \begin{eqnarray} \label{4.1}\frac{AC_1}{C_1B}
 \cdot \frac{BA_1}{A_1C}
 \cdot \frac{CB_1}{B_1A}=
\frac{A_CP}{PB_C}
 \cdot
\frac{B_AP}{PC_A}
 \cdot
\frac{C_BP}{PA_B}
 \end{eqnarray}

\begin{figure}[!htb]
\centering
\input{sl.pd.7.4.6.pic}
\caption{} \label{sl.pd.7.4.6.pic}
\end{figure}


 Iz $\angle AA_CC\cong\angle AC_AC=90^0$ sledi,
 da točki $A_C$ in $C_A$ ležita na krožnici s premerom $AC$. Zato
 je potenca točke $P$ na to krožnico enaka (izrek \ref{izrekPotenca})
 $\overrightarrow{PC}\cdot \overrightarrow{PA_C}=
  \overrightarrow{PA}\cdot \overrightarrow{PC_A}$.
   Analogno je  $\overrightarrow{PB}\cdot \overrightarrow{PC_B}=
  \overrightarrow{PC}\cdot \overrightarrow{PB_C}$ in
   $\overrightarrow{PB}\cdot \overrightarrow{PA_B}=
  \overrightarrow{PA}\cdot \overrightarrow{PB_A}$. Iz teh relacij
  dobimo: $\frac{PA_C}{PC_A}=\frac{PA}{PC}$,
   $\frac{PC_B}{PB_C}=\frac{PC}{PB}$ in
   $\frac{PB_A}{PA_B}=\frac{PB}{PA}$. Če te relacije uvrstimo v
   \ref{4.1}, dobimo  $\frac{AC_1}{C_1B}
 \cdot \frac{BA_1}{A_1C}
 \cdot \frac{CB_1}{B_1A}=1$.
 Ker je $\frac{\overrightarrow{AC_1}}{\overrightarrow{C_1B}}
 \cdot \frac{\overrightarrow{BA_1}}{\overrightarrow{A_1C}}
 \cdot \frac{\overrightarrow{CB_1}}{\overrightarrow{B_1A}}<0$,
 velja
 $\frac{\overrightarrow{AC_1}}{\overrightarrow{C_1B}}
 \cdot \frac{\overrightarrow{BA_1}}{\overrightarrow{A_1C}}
 \cdot \frac{\overrightarrow{CB_1}}{\overrightarrow{B_1A}}=-1$. Po
 Menelajevem izreku so točke $A_1$, $B_1$ in $C_1$ kolinearne.
\kdokaz



        \bnaloga\footnote{41. IMO, S. Korea - 2000, Problem 1.}
        $AB$ is tangent to the circles $CAMN$ and $NMBD$. $M$ lies
        between $C$ and $D$ on the line $CD$, and $CD$ is parallel to $AB$. The chords
        $NA$ and $CM$ meet at $P$; the chords $NB$ and $MD$ meet at $Q$. The rays $CA$
        and $DB$ meet at $E$. Prove that $PE\cong QE$.
        \enaloga

\begin{figure}[!htb]
\centering
\input{sl.pod.7.12.IMO3.pic}
\caption{} \label{sl.pod.7.12.IMO3.pic}
\end{figure}

\textbf{\textit{Solution.}} Označimo z $L$ presečišče premic $MN$
in $AB$ ter s $k$ in $l$ očrtani krožnici štirikotnikov
        $CAMN$ in $NMBD$ (Figure \ref{sl.pod.7.12.IMO3.pic}).

Po izreku \ref{PotencnaOs} je premica $MN$ potenčna os krožnic $k$
in $l$, zato za njeno točko $L\in MN$ velja
$|LA|^2=p(L,k)=p(L,l)=|LB|^2$, kar pomeni, da je $L$ središče
daljice $AB$. Ker je po predpostavki $AB\parallel CD$, oz.
$AB\parallel PQ$, je po Talesovem izreku $MP:MQ=LA:LB=1$. Torej
je točka $M$ središče daljice $PQ$ oz. $MP\cong MQ$.

Če uporabimo izreka \ref{ObodKotTang} in \ref{KotiTransverzala},
dobimo:
 \begin{eqnarray*}
 \angle EAB &\cong& \angle AMC \cong\angle MAB\\
 \angle EBA &\cong& \angle BMQ \cong\angle MBA
 \end{eqnarray*}
To pomeni, da sta trikotnika $AEB$ in $AMB$ skladna (izrek
\textit{ASA} \ref{KSK}), in sicer simetrična glede na os $AB$. To
pomeni, da je $EM\perp AB$. Ker je $AB\parallel PQ$, je tudi
$EM\perp PQ$ oz. $\angle PME\cong\angle QME =90^0$.
Da je $MP\cong MQ$, smo že dokazali, zato sta trikotnika $PME$ in $QME$ skladna
(izrek \textit{SAS} \ref{SKS}), iz tega pa sledi $PE\cong QE$.
 \kdokaz



\bnaloga\footnote{40. IMO, Romania - 1999, Problem 5.}
        Two circles $k_1$ and $k_2$ are contained inside the circle $k$, and are tangent to $k$
        at the distinct points $M$ and $N$, respectively. $k_1$ passes through the center of
        $k_2$. The line passing through the two points of intersection of $k_1$ and $k_2$ meets
        $k$ at $A$ and $B$. The lines $MA$ and $MB$ meet $k_1$ at $C$ and $D$, respectively.
        Prove that $CD$ is tangent to $k_2$.
        \enaloga

\begin{figure}[!htb]
\centering
\input{sl.pod.7.12.IMO4.pic}
\caption{} \label{sl.pod.7.12.IMO4.pic}
\end{figure}

\textbf{\textit{Solution.}} Označimo z $O_1$ in $O_2$ središči
krožnic $k_1$ in $k_2$, z $r_1$ in $r_2$ njuna polmera ter z $E$
drugo presečišče premice $AN$ s krožnico $k_2$ (Figure
\ref{sl.pod.7.12.IMO4.pic}). Brez škode za splošnost predpostavimo
$r_1\geq r_2$.

Dokažimo najprej, da je premica $CE$ skupna tangenta krožnic $k_1$
in $k_2$. Naj bo $\widehat{E}$ drugo presečišče očrtane krožnice
$k'$ trikotnika $CMN$ in krožnice $k_2$. Točka $A$ leži na
potenčnih oseh krožnic $k_1$ in $k_2$ oz. $k_1$ in $k'$, zato je
po izreku \ref{PotencnoSr} točka $A$ potenčno središče krožnic
$k_1$, $k_2$ in $k'$. To pomeni, da točka $A$ leži na potenčni
premici krožnic $k_2$ in $k'$ - premici $N\widehat{E}$. Iz tega
sledi, da je $\widehat{E}\in AN\cap k_2$ oz. $\widehat{E}=E$.
Torej so točke $M$, $C$, $E$ in $N$ konciklične in je po izreku
\ref{ObodKotTang} $\angle ACE \cong\angle ANM$. Označimo z $L$
poljubno točko skupne tangente krožnic $k$ in $k_1$ v točki $M$,
ki leži v polravnini z robom $AC$, v kateri nista točki $B$ in
$D$. Po istem izreku \ref{ObodKotTang} (glede na krožnici $k$ in
$k_1$) je $\angle LMA \cong\angle MBA$ in $\angle LMC \cong\angle
MDC$. Iz izreka \ref{ObodObodKot} (za krožnico $k$ in tetivo $AM$)
dobimo še $\angle ANM \cong\angle ABM$. Če povežemo dokazane
relacije, dobimo:
$$\angle ACE \cong \angle ANM
\cong \angle ABM\cong \angle LMA \cong\ \angle CDM. $$
 Iz $\angle ACE \cong \angle CDM$ pa po izreku \ref{ObodKotTang} sledi,
  da je $EC$ tangenta krožnice $k_1$. Ker v dokazu še nismo
  uporabili dejstva $O_2\in k_1$,
  analogno dokažemo tudi, da je $CE$ tangenta krožnice $k_2$.


  Označimo s $T$ presečišče daljice $O_2O_1$ in krožnice $k_2$. Dovolj je še
  dokazati $T\in CD$ in $\angle CTO_2=90^0$. Naj bo $O'_2$
  pravokotna projekcija točke $O_2$ na premici $O_1C$. Ker je $CE$
  skupna tangenta krožnic $k_1$ in $k_2$, sta polmera $O_1C$ in
  $O_2E$ pravokotna na to tangento. Zatorej je $CEO_2O'_2$
  pravokotnik in velja $O'_2C\cong O_2E=r_2$. Iz tega sledi
  $O_1O'_2=r_1-r_2=O_1T$. To pomeni, da sta trikotnika
  $O_1O'_2O_2$ in $O_1TC$ skladna (izrek \textit{SAS} \ref{SKS}), zato je
  $\angle CTO_1\cong\angle O_2O'_2O_1=90^0$ oz. $\angle
  CTO_2=90^0$. Premici $CT$ in $AB$ sta pravokotni na
  centralo $O_1O_2$ dveh krožnic. Torej je $CT\parallel AB$. Ker
  je (zaradi že dokazane relacije $\angle ABM\cong \angle CDM$), je
  tudi $CD\parallel AB$. Po Playfairovem aksiomu \ref{Playfair}
  sta $CT$ in $CD$ ista premica oz. $T\in CD$. To pomeni, da se
  premica $CD$ dotika krožnice $k_2$ v točki $T$.
   \kdokaz




%________________________________________________________________________________
\poglavje{The Theorems of Pappus and Pascal} \label{odd7PappusPascal}

Izreka v tem razdelku sta zgodovinsko povezana z razvojem \index{geometrija!projektivna}projektivne geometrije.

            

            \bizrek \label{izrek Pappus} \index{izrek!Pappusov}(Pappus'\footnote{\index{Pappus} \textit{Pappus} iz Aleksandrije (3. st.), eden od zadnjih
            velikih starogrških geometrov. Ta izrek je dokazal v evklidskem primeru,
            uporabljajoč pri tem metriko. Toda fundamentalna vloga Pappusovega izreka
                v projektivni geometriji je bila odkrita šele šestnajst stoletij
            kasneje.} theorem)
            Let $A$, $B$ and $C$ be three different
            points of a line $p$ and $A'$, $B'$ and $C'$ three different points of another line
            $p'$ in the same plane. Then the points
            $X=BC'\cap B'C$, $Y=AC'\cap A'C$ and $Z=AB'\cap A'B$ are collinear. 
            \eizrek


\begin{figure}[!htb]
\centering
\input{sl.pod.7.10.1.pic}
\caption{} \label{sl.pod.7.10.1.pic}
\end{figure}


 \textbf{\textit{Proof.}}
Naj bo $L=AB'\cap BC'$, $M=AB'\cap CA'$ in $N=CA'\cap BC'$ (Figure \ref{sl.pod.7.10.1.pic}). Uporabimo petkrat Menelajev izrek (\ref{izrekMenelaj})
glede na trikotnik $LMN$ in premice $BA'$, $AC'$, $CB'$, $AB$, $A'B'$:

\begin{eqnarray*}
& & \frac{\overrightarrow{LZ}}{\overrightarrow{ZM}}\cdot \frac{\overrightarrow{MA'}}{\overrightarrow{A'N}}\cdot \frac{\overrightarrow{NB}}{\overrightarrow{BL}}=-1,\\
& & \frac{\overrightarrow{LA}}{\overrightarrow{AM}}\cdot \frac{\overrightarrow{MY}}{\overrightarrow{YN}}\cdot \frac{\overrightarrow{NC'}}{\overrightarrow{C'L}}=-1,\\
& & \frac{\overrightarrow{LB'}}{\overrightarrow{B'M}}\cdot \frac{\overrightarrow{MC}}{\overrightarrow{CN}}\cdot \frac{\overrightarrow{NX}}{\overrightarrow{XL}}=-1,\\
& & \frac{\overrightarrow{LA}}{\overrightarrow{AM}}\cdot \frac{\overrightarrow{MC}}{\overrightarrow{CN}}\cdot \frac{\overrightarrow{NB}}{\overrightarrow{BL}}=-1,\\
& & \frac{\overrightarrow{LB'}}{\overrightarrow{B'M}}\cdot \frac{\overrightarrow{MA'}}{\overrightarrow{A'N}}\cdot \frac{\overrightarrow{NC'}}{\overrightarrow{C'L}}=-1.
\end{eqnarray*}
Iz teh petih relacij (če pomnožimo prve tri, nato pa v dobljeno relacijo vstavimo četrto in peto) dobimo:
 $$\frac{\overrightarrow{LZ}}{\overrightarrow{ZM}}\cdot \frac{\overrightarrow{MY}}{\overrightarrow{YN}}\cdot \frac{\overrightarrow{NX}}{\overrightarrow{XL}}=-1.$$
 Po obratnem Menelajevem izreku (\ref{izrekMenelaj}) so točke $X$, $Y$ in $Z$ kolinearne.
 \kdokaz

  Dokažimo sedaj še Pascalov\footnote{Ni točno znano, kako je to trditev za krožnico
   dokazal
   francoski matematik in filozof \index{Pascal, B.}
  \textit{B. Pascal} (1623--1662), ker je
  originalen dokaz izgubljen. Lahko pa predvidevamo, da je uprabljal rezultate
   in metode tistega časa, kar pomeni, da je verjetno uporabljal Menelajev izrek.}
    izrek za
  krožnico.

        

        \bizrek \index{izrek!Pascalov} \label{izrekPascalEvkl}
         Let $A$, $B$, $C$, $D$, $E$ and  $F$ be arbitrary points on
        some circle $k$. Then the points $X=AE\cap BD$, $Y=AF\cap CD$ and
        $Z=BF\cap CE$ are collinear. 
        \eizrek

\begin{figure}[!htb]
\centering
\input{sl.pod.7.10.2.pic}
\caption{} \label{sl.pod.7.10.2.pic}
\end{figure}


 \textbf{\textit{Proof.}}
Označimo $L=AE\cap BF$, $M=AE\cap CD$ in $N=CD\cap BF$.
 Če trikrat uporabimo Menelajev izrek (\ref{izrekMenelaj}) za trikotnik $LMN$ in premice $BD$,
$AF$ in $CE$, dobimo (Figure \ref{sl.pod.7.10.2.pic}):
 \begin{eqnarray*}
  \hspace*{-2mm} \frac{\overrightarrow{LX}}{\overrightarrow{XM}}\cdot
   \frac{\overrightarrow{MD}}{\overrightarrow{DN}}\cdot
   \frac{\overrightarrow{NB}}{\overrightarrow{BL}}=-1,\hspace*{1mm}
   \frac{\overrightarrow{LA}}{\overrightarrow{AM}}\cdot
   \frac{\overrightarrow{MY}}{\overrightarrow{YN}}\cdot
   \frac{\overrightarrow{NF}}{\overrightarrow{FL}}=-1,\hspace*{1mm}
   \frac{\overrightarrow{LE}}{\overrightarrow{EM}}\cdot
   \frac{\overrightarrow{MC}}{\overrightarrow{CN}}\cdot
   \frac{\overrightarrow{NZ}}{\overrightarrow{ZL}}=-1.
  \end{eqnarray*}
 Če uporabimo potence točk $M$, $N$ in $L$ glede na krožnico
 $k$  (izrek \ref{izrekPotenca}) ali pa upoštevamo podobnost ustreznih trikotnikov, sledi:
  \begin{eqnarray*}
  \overrightarrow{MC}\cdot\overrightarrow{MD}=
  \overrightarrow{MA}\cdot\overrightarrow{ME},\hspace*{3mm}
  \overrightarrow{NC}\cdot\overrightarrow{ND}=
  \overrightarrow{NF}\cdot\overrightarrow{NB},\hspace*{3mm}
  \overrightarrow{LA}\cdot\overrightarrow{LE}=
  \overrightarrow{LB}\cdot\overrightarrow{LF}.
  \end{eqnarray*}
 Iz prejšnjih šestih relacij sledi:
 \begin{eqnarray*}
  \frac{\overrightarrow{LX}}{\overrightarrow{XM}}\cdot
   \frac{\overrightarrow{MY}}{\overrightarrow{YN}}\cdot
   \frac{\overrightarrow{NZ}}{\overrightarrow{ZL}}=-1.
   \end{eqnarray*}
Torej so po Menelajevem izreku (\ref{izrekMenelaj}, obratna smer) točke $X$, $Y$ in
$Z$ kolinearne.
 \kdokaz

  Prejšnji izrek, ki se nanaša na krožnico,
   lahko posplošimo tudi na poljubno stožnico\footnote{Proučevanje stožnic se je
začelo že pri Starih Grkih. Termine elipsa, parabola, hiperbola
 je prvi uporabil  starogrški matematik \index{Apolonij} \textit{Apolonij} iz Perge (262--200 pr. n. š.) v
 svojem
znanem delu \textit{Razprava o presekih stožca}, ki je
sestavljeno iz
 osmih knjig, kjer stožnico definira kot presek ravnine s
krožnim stožcem. Komentar k temu Apolonijevemu delu je napisala starogrška filozofinja in matematičarka ter zadnja predstavnica antične znanosti \index{Hipatija} \textit{Hipatija} iz Aleksandrije (370--415). Ponovno zanimanje za stožnice sta oživela
 nemški astronom \index{Kepler, J.} \textit{J. Kepler} (1571--1630) in
francoski matematik in filozof \index{Pascal, B.} \textit{B.
Pascal} (1623--1662) v 17. stoletju.}
  oz. krivuljo drugega razreda.
   V evklidskem primeru so to: elipsa, parabola in hiperbola.
  To lahko ugotovimo na ta način, da stožnico definiramo
   kot presek  nosilk vseh
   stranic stožca in neke ravnine.
   Zaradi tega na stožnico $\mathcal{K}$ v splošnem primeru gledamo kot na
    središčno
   projekcijo neke krožnice $k$ na neko ravnino. Središče projiciranja je
    vrh stožca $S$. Omenili smo že, da
    središčna projekcija ohranja kolinearnost. Če so $A'$, $B'$, $C'$,
     $D'$, $E'$ in  $F'$ točke na stožnici $\mathcal{K}$ ter $X'$, $Y'$ in $Z'$
     ustrezne točke, definirane kot v izreku \ref{izrekPascalEvkl},
     so te točke slike nekih točk $A$, $B$, $C$, $D$, $E$, $F$, $X$, $Y$ in
     $Z$.
    Pri tem je prvih šest točk na krožnici $k$, zadnje tri pa so
    po izreku \ref{izrekPascalEvkl} kolinearne. Iz tega sledi, da so tudi
    točke $X'$, $Y'$ in $Z'$ kolinearne (Figure \ref{sl.pod.7.10.3.pic}). To pomeni, da izrek \ref{izrekPascalEvkl}
    res velja v splošnem primeru za poljubne stožnice. Ta splošni izrek je
     znan kot   Pascalov izrek za stožnice\footnote{\index{Pascal, B.} \textit{B. Pascal}
(1623--1662), francoski matematik in filozof, ki je
  že kot šestnajstletnik dokazal ta pomemben izrek o
  stožnicah, objavil pa ga je
leta 1640, vendar se takrat trditev ni neposredno nanašala na
projektivno geometrijo.}.

\begin{figure}[!htb]
\centering
\input{sl.pod.7.10.3.pic}
\caption{} \label{sl.pod.7.10.3.pic}
\end{figure}

 Ideje, ki smo jih obravnavali v tem razdelku, nas napeljujejo na ugotovitev,
  da lahko Pascalov izrek  izpeljemo tudi v projektivni geometriji.
  Še več - v projektivni geometriji lahko definiramo in raziskujemo
    tudi stožnice, vendar
 ni možno razlikovati med elipso, hiperbolo ali
  parabolo (glej \cite{Mitrovic}).


%________________________________________________________________________________
 \poglavje{The Golden Ratio} \label{odd7ZlatiRez}

 Pravimo, da točka $Z$ daljice $AB$ deli to daljico v razmerju \index{zlati!rez}\pojem{zlatega reza}\footnote{Takšno delitev je obravnaval že \index{Pitagora}\textit{Pitagora z otoka Samosa} (582--497 pr. n. š.), starogrški filozof in matematik. Prve znane zapise o zlatem rezu naj bi ustvaril starogrški matematik \index{Evklid}
 \textit{Evklid iz Aleksandrije} (3. st. pr. n. š.). V svojem znamenitem delu \textit{Elementi} je zastavil problem: ‘‘\textit{Dano daljico razdeli na dva neenaka dela tako, da bo ploščina pravokotnika, ki ima dolžino enako celotni daljici, višino pa krajšemu delu daljice, enaka ploščini kvadrata, načrtanega nad daljšim delom daljice.}’’
Termin zlatega reza, ki ga danes uporabljamo, je vpeljal \index{Leonardo da Vinci}\textit{Leonardo da Vinci} (1452--1519),
italijanski slikar, arhitekt in izumitelj. Zlati rez ljudje že tisočletja uporabljajo v slikarstvu in arhitekturi.}, če je razmerje dolžine celotne daljice proti daljšemu delu enako razmerju daljšega dela  proti krajšemu (Figure \ref{sl.pod.7.15.1.pic}) oz:

            \begin{eqnarray} \label{eqnZlatiRez}
            AB:AZ=AZ:ZB.
             \end{eqnarray}

\begin{figure}[!htb]
\centering
\input{sl.pod.7.15.1.pic}
\caption{} \label{sl.pod.7.15.1.pic}
\end{figure}

Če
sedaj krajši del ‘‘vstavimo’’ v daljšega, dobimo enako
razmerje. Res, ker je
$AZ:ZB=AB:AZ=(AZ+ZB):AZ$, velja tudi
$$ZB:(AZ-ZB)=AZ:ZB.$$
Ta postopek lahko nadaljujemo (Figure \ref{sl.pod.7.15.2.pic}).

\begin{figure}[!htb]
\centering
\input{sl.pod.7.15.2.pic}
\caption{} \label{sl.pod.7.15.2.pic}
\end{figure}

Seveda se zastavlja vprašanje, kako konstruiramo takšno točko $Z$. To konstrukcijo bomo opisali v naslednjem zgledu.
                
                 
                
                \bzgled
                For a given line $AB$, construct a point $Z$ that divides the line segment into
                the golden ratio.
                \ezgled


\begin{figure}[!htb]
\centering
\input{sl.pod.7.15.3.pic}
\caption{} \label{sl.pod.7.15.3.pic}
\end{figure}


 \textbf{\textit{Solution.}}
Načrtajmo najprej krožnico $k(S,SB)$ s polmerom
$|SB|=\frac{1}{2}|AB|$, ki se dotika premice $AB$ v točki $B$ (Figure \ref{sl.pod.7.15.3.pic}). Konstruirajmo še presečišči
te krožnice  s premico $AS$ - označimo ju z $X$ in $Y$ (naj bo pri tem $\mathcal{B}(A,X,S)$). Točko $Z$ sedaj dobimo kot
 presečišče daljice $AB$ in krožnice
$l(A,AX)$.

Dokažimo, da je $Z$ iskana točka. Če uporabimo potenco točke $A$ glede na krožnico $k$ (izrek \ref{izrekPotenca}), dobimo:
$$\hspace*{-1.5mm} |AB|^2=p(A,k)=|AX|\cdot |AY|=|AX|\cdot(|AX|+|XY|)=|AZ|\cdot(|AZ|+|AB|).$$
 Torej velja:
\begin{eqnarray} \label{eqnZlatiRez2}
  |AB|^2=|AZ|\cdot(|AZ|+|AB|)
  \end{eqnarray}
 Naprej je:
 \begin{eqnarray*}
 |AB|^2=|AZ|\cdot(|AZ|+|AB|)\hspace*{1mm}&\Leftrightarrow &\hspace*{1mm}
 \frac{|AB|}{|AZ|}=\frac{|AZ|+|AB|}{|AB|}\\
 \hspace*{1mm}&\Leftrightarrow &\hspace*{1mm}
 \frac{|AB|}{|AZ|}=\frac{|AZ|}{|AB|}+1\\
 \hspace*{1mm}&\Leftrightarrow &\hspace*{1mm}
 \frac{|AB|}{|AZ|}-1=\frac{|AZ|}{|AB|}\\
 \hspace*{1mm}&\Leftrightarrow &\hspace*{1mm}
 \frac{|AB|-|AZ|}{|AZ|}=\frac{|AZ|}{|AB|}\\
 \hspace*{1mm}&\Leftrightarrow &\hspace*{1mm}
 \frac{|BZ|}{|AZ|}=\frac{|AZ|}{|AB|}
  \end{eqnarray*}
 kar je ekvivalentno z relacijo \ref{eqnZlatiRez}. To pomeni, da točka $Z$ deli daljico $AB$ v razmerju zlatega reza.
 \kdokaz

 V naslednjem primeru bomo izračunali vrednost razmerja, ki ga določa zlati rez.

                

                 \bizrek \label{zlatiRezStevilo}
                If a point $Z$ divides a line segment $AB$ into the golden ratio
                ($AZ$ is the longer part), then
                \begin{eqnarray*}
                && AZ:ZB=AB:AZ=\frac{\sqrt{5}+1}{2}, \hspace*{1mm}\textrm{ i.e.}\\
                && AZ=\frac{\sqrt{5}-1}{2}AB.
                \end{eqnarray*}
                \eizrek

 \textbf{\textit{Proof.}}
 Relacija \ref{eqnZlatiRez2} iz prejšnjega izreka je ekvivalentna z relacijo:
 $$|AB|^2-|AB|\cdot |AZ|-|AZ|^2=0.$$
 Če to kvadratno enačbo rešimo po $|AZ|$, dobimo:
 $$|AZ|=\frac{\sqrt{5}+1}{2}|AB|,$$
 iz tega pa sledita iskani enakosti.
 \kdokaz

 Število $$\Phi=\frac{\sqrt{5}+1}{2}$$ iz prejšnjega izreka, ki torej predstavlja vrednost razmerja, ki ga določa zlati rez, imenujemo \index{število!zlato}\pojem{zlato število}. Seveda gre za iracionalno število ($\Phi\notin \mathbb{Q}$). Njegova približna vrednost znaša:
 $$\Phi=\frac{\sqrt{5}+1}{2}\doteq 1,62.$$

 Prav tako je:
 $$\frac{\sqrt{5}-1}{2}\doteq 0,62,$$
 kar pomeni, da je daljši del $AZ$ pri zlatem rezu približno $62\%$ celotne daljice $AB$.

                . 

                 \bzgled \label{zlatiRezKonstr}
                Construct a golden rectangle $ABCD$ with two sides
                $a$ and $b$ in the golden ratio -  so-called  \index{golden!rectangle}\pojem{golden rectangle}\color{green1}.
                \ezgled


\begin{figure}[!htb]
\centering
\input{sl.pod.7.15.4.pic}
\caption{} \label{sl.pod.7.15.4.pic}
\end{figure}


 \textbf{\textit{Solution.}} Narišemo poljubno daljico $AB$, nato pa točko $Z$, ki deli daljico $AB$ v zlatem rezu tako, da je $AZ$ daljši del (glej prejšnji zgled \ref{zlatiRezKonstr}). Na koncu dobimo $D=\mathcal{R}_{A,90^0}(Z)$ in $C=\mathcal{T}_{\overrightarrow{AB}}(D)$
 (Figure \ref{sl.pod.7.15.4.pic}).
 \kdokaz

 Če zlati pravokotnik po daljši stranici v razmerju zlatega reza razdelimo na dva pravokotnika, dobimo kvadrat in še en zlati pravokotnik. Novi zlati pravokotnik razdelimo na enak način in postopek nadaljujemo. Če povežemo ustrezna oglišča (krajišč diagonal kvadratov, ki določajo lomljenko), tako da narišemo krožne loke s središčnim kotom $90^0$, dobimo približno konstrukcijo t. i. \index{zlata spirala}\pojem{zlate spirale} ali \index{logaritemska spirala}\pojem{logaritemske spirale} (Figure \ref{sl.pod.7.15.5.pic}).



\begin{figure}[!htb]
\centering
\input{sl.pod.7.15.5.pic}
\caption{} \label{sl.pod.7.15.5.pic}
\end{figure}

Logaritemska spirala\footnote{Predlog za ime logaritemska spirala je podal francoski matematik \index{Varignon, P.}\textit{P. Varignon} (1645--1722). Logaritemska spirala ima lastnost, da jo vsaka ravna premica iz središča spirale seka pod enakim kotom. Pojavlja se v različnih oblikah v naravi: od spiral v cvetu soncnič, polžjih hišic in pajkovih mrež pa vse do oddaljenih galaksij. Imenujemo jo tudi spirala mirabilis (čudežna spirala) - to ime je predlagal švicarski matematik \index{Bernoulli, J.}\textit{J. Bernoulli} (1667–-1748) zaradi njenih čudovitih lastnosti.} je krivulja, ki jo definiramo z enačbami v parametrični obliki:
\begin{eqnarray*}
x&=&a e^{bt}\cdot\cos t \\
y&=&a e^{bt}\cdot\sin t,
\end{eqnarray*}
kjer je $t\in \mathbb{R}$ parameter, $a$ in $b$ poljubni realni konstanti ter $e\doteq 2,72$ t. i. \index{število!Eulerjevo }\pojem{Eulerjevo\footnote{Število $e$ je iracionalno število in predstavlja vrednost limite $\lim_{n\rightarrow \infty}\left(1+\frac{1}{n} \right)^n$ =e. Ime je dobilo po švicarskem matematiku \index{Euler, L.}\textit{L. Eulerju} (1707--1783).} število.}.
Prejšnja konstrukcija (s krožnimi loki) je, kot smo že omenili, približna, toda predstavlja zelo dobro aproksimacijo te krivulje (Figure \ref{sl.pod.7.15.5a.pic}).


\begin{figure}[!htb]
\centering
\input{sl.pod.7.15.5a.pic}
\caption{} \label{sl.pod.7.15.5a.pic}
\end{figure}

Nadaljevali bomo s pravilnim petkotnikom.

                

                \bzgled \label{zlatiRezPravPetk1}
                A diagonal and a side of a regular pentagon are in the golden ratio. 
                \ezgled


 \textbf{\textit{Proof.}}  (Figure \ref{sl.pod.7.15.6.pic})

 Označimo z $a$  stranico in $d$  diagonalo pravilnega petkotnika $ABCDE$. Iz \ref{PtolomejPetkotnik} izreka sledi:
                    $$d=\frac{1+\sqrt{5}}{2}a.$$
Zato sta po izreku \ref{zlatiRezStevilo}  daljici $d$ in $a$ v razmerju zlatega reza.
\kdokaz

\begin{figure}[!htb]
\centering
\input{sl.pod.7.15.6.pic}
\caption{} \label{sl.pod.7.15.6.pic}
\end{figure}

                 

                \bzgled
                Two diagonals of a regular pentagon intersect at a point that
                divides them in the golden ratio\footnote{Za to to lastnost petkotnika so vedeli že pitagorejci. Že omenjeno šolo pitagorejcev je ustanovil starogrški filozof in matematik \index{Pitagora}\textit{Pitagora z otoka Samosa} (582--497 pr. n. š.) v Crotoni v južni
                Italiji. Njegovi učenci so se ukvarjali s filozofijo, z matematiko in naravoslovjem. Za svoj razpoznavni znak so
                izbrali pentagram, ki je sestavljen iz petkotnikovih diagonal. Vpliv pitagorejske šole na matematiko Starih
                Grkov je obstajal še več stoletij po Pitagorovi smrti.}.
                \ezgled


 \textbf{\textit{Proof.}}
S $P$ označimo presečišče diagonal $AC$ in $BD$ pravilnega petkotnika $ABCDE$ ter s $k$
njegovo očrtano krožnico (Figure \ref{sl.pod.7.15.6.pic}). Koti $EAD$, $DAC$, $CAB$ in $DBC$, ki so prirejeni skladnim
tetivam $ED$, $CD$ in $CB$, so skladni (izrek \ref{SklTetSklObKot}). Ker je po izreku \ref{pravVeckNotrKot} notranji kot $EAB$ pravilnega petkotnika enak
$108^0$, so koti $EAD$, $DAC$, $CAB$ (oz. $PAB$) in
$DBC$ enaki $36^0$. Iz tega dobimo $\angle PBA=\angle DBA=\angle CBA-\angle CBD=108^0-36^0= 72^0$.  Po izreku \ref{VsotKotTrik} iz vsote notranjih kotov trikotnika $ABP$ sledi
$\angle APB=180^0-\angle PBA-\angle PAB=72^0$. Torej je trikotnik $PAB$ enakokrak z osnovnico $PB$ (izrek \ref{enakokraki}),
oz. velja $AP\cong AB$. Iz tega sledi:
$$AC:AP=AC:AB.$$
Če uporabimo še prejšnjo trditev \ref{zlatiRezPravPetk1}, ugotovimo, da točka $P$ deli diagonalo $AC$ v razmerju zlatega reza.
\kdokaz


Če v pravilnem petkotniku $ABCDE$ izberemo oglišča $A$, $B$ in $D$, dobimo enakokraki trikotnik $ABD$ z osnovnico, ki je enaka stranici $a$ petkotnika $ABCDE$, kraka pa sta enaka diagonali $d$  tega petkotnika. Krak in osnovnica tega trikotnika sta torej v razmerju zlatega reza (Figure \ref{sl.pod.7.15.7.pic}). Zato trikotnik $ABD$ imenujemo \index{zlati!trikotnik}\pojem{zlati trikotnik}. Koti zlatega trikotnika merijo $72^0$, $72^0$ in $36^0$.

\begin{figure}[!htb]
\centering
\input{sl.pod.7.15.7.pic}
\caption{} \label{sl.pod.7.15.7.pic}
\end{figure}

Podobno kot pri zlatem pravokotniku, če uporabimo simetrale ustreznih notranjih kotov ob osnovnici zlatega trikotnika, dobimo zaporedje zlatih trikotnikov, ki so vsi podobni. S pomočjo ustreznih krožnih lokov lahko izpeljemo še eno približno konstrukcijo zlate (oz. logaritemske) spirale (Figure \ref{sl.pod.7.15.7.pic}).

%________________________________________________________________________________
 \poglavje{Morley's Theorem and Some More Theorems} \label{odd7Morly}

Pravimo, da sta poltraka $SP$ in $SQ$ \index{trisektrisa}\pojem{trisektrisi} kota $ASB$, če točki $P$ in $Q$ ležita v tem kotu in velja $$\angle ASP\cong\angle PSQ\cong\angle QSB,$$
oz. gre za poltraka, ki delita kot na tri skladne kote
(Figure \ref{sl.pod.7.16.0.pic}).



\begin{figure}[!htb]
\centering
\input{sl.pod.7.16.0.pic}
\caption{} \label{sl.pod.7.16.0.pic}
\end{figure}

 Dokažimo najprej pomožno trditev - lemo.

               


                \bizrek \label{izrekMorleyLema}
                 Let $Y'$, $Z$, $Y$ and $Z'$ be points in the plane such that
                    $Y'Y\cong ZY\cong ZZ'$ and
                     $$\angle Z'ZY\cong \angle ZYY'=180^0-2\alpha>60^0.$$
                If $A$ is a point in this plane that is on the different side
                of the line $Z'Y'$ with respect to
                the point $Z$ and also $Y'AZ'=3\alpha$, then the points $A$, $Y'$, $Z$, $Y$ and $Z'$ are
                concyclic and also
                $$\angle Z'AZ\cong \angle ZAY\cong \angle YAY'=\alpha.$$
                \eizrek

\begin{figure}[!htb]
\centering
\input{sl.pod.7.16.1a.pic}
\caption{} \label{sl.pod.7.16.1a.pic}
\end{figure}


 \textbf{\textit{Proof.}} Najprej je dani pogoj $180^0-2\alpha>60^0$ ekvivalenten s pogojem $\alpha<60^0$.

 Naj bo $s$ simetrala daljice $ZY$, $\mathcal{S}_s$ pa zrcaljenje čez premico $s$ (Figure \ref{sl.pod.7.16.1a.pic}). Torej $\mathcal{S}_s(Z)=Y$. Ker je $\mathcal{S}_s$ izometrija, ki ohranja velikost kotov in dolžine daljic, je tudi $\mathcal{S}_s(Z')=Y'$. Iz $ZY,Z'Y'\perp s$ sledi $ZY\parallel Z'Y'$, kar pomeni, da je $ZYY'Z'$ enakokraki trapez.
 Po izreku \ref{trapezTetivEnakokr} je ta trapez tetiven - označimo s $k$ njegovo očrtano krožnico. Po izreku \ref{KotiTransverzala} sta kota $YZZ'$ in $Y'Z'Z$ suplementarna, zato je:
  \begin{eqnarray} \label{eqnMorleyLema1}
  \angle ZZ'Y'=180^0-\angle Z'ZY=2\alpha.
  \end{eqnarray}
  Iz $Z'Z\cong ZY$ po izreku \ref{SklTetSklObKot} sledi $\angle ZZ'Y\cong \angle YZ'Y'$, zato je po relaciji \ref{eqnMorleyLema1} $\angle ZZ'Y\cong \angle YZ'Y'=\alpha$. Trikotnik $Z'ZY$ je enakokrak z osnovnico $Z'Y$, zato je po izreku \ref{enakokraki} tudi $\angle ZYZ'\cong \angle ZZ'Y=\alpha$. Analogno je $\angle YY'Z\cong \angle ZY'Z'=\alpha$ in $\angle YZY'\cong \angle YY'Z=\alpha$. Če vse povežemo, dobimo $\angle ZZ'Y\cong \angle YZ'Y'\cong \angle ZYZ'\cong
  \angle YY'Z\cong \angle ZY'Z'\cong\angle YZY'=\alpha$, torej:
   \begin{eqnarray} \label{eqnMorleyLema2}
  \angle Z'Y'Z \cong \angle Y'Z'Y= \alpha.
  \end{eqnarray}
  Ker je $\angle Z'ZY'=\angle Z'ZY-\angle Y'ZY=180^0-2\alpha-\alpha=180^0-3\alpha$ (ta kot obstaja, ker je $\alpha<60^0$) oz. $\angle Z'ZY'+\angle Z'AY'=180^0-3\alpha+3\alpha=180^0$, je po izreku \ref{TetivniPogoj} $Z'ZY'A$ tetivni štirikotnik, kar pomeni, da točka $A$ leži na krožnici $k$.

 Iz relacije \ref{eqnMorleyLema2} in izreka \ref{ObodObodKot} sledi:
  \begin{eqnarray*}
  &&\angle Z'AZ\cong \angle Z'Y'Z=\alpha,\\
  &&\angle Y'AY\cong \angle Y'Z'Y=\alpha.
  \end{eqnarray*}
  Na koncu je še:
  \begin{eqnarray*}
  \angle ZAY=\angle Z'AY'-\angle Z'AZ-\angle Y'AY =3\alpha-\alpha-\alpha=\alpha,
  \end{eqnarray*}
  kar je bilo treba dokazati.  \kdokaz

  Sedaj smo pripravljeni za osnovni izrek.


               


                \bizrek \label{izrekMorley}\index{izrek!Morleyev}
                 If $X$, $Y$ and $Z$ are three points of intersection of the adjacent angle trisectors
                 of a triangle $ABC$, then $XYZ$ is an equilateral triangle.\\
                (Morley's\footnote{\index{Morley, F.}\textit{F. Morley} (1860--1937), angleški matematik, je odkril to lastnost trikotnika leta 1904, objavil pa jo je šele 20 let
                pozneje. V tem času je bil izrek kot naloga objavljen v časopisu \textit{Educational Times}. Na tem mestu bomo podali eno od takrat predlaganih rešitev.}  theorem)
                \eizrek



\begin{figure}[!htb]
\centering
\input{sl.pod.7.16.1.pic}
\caption{} \label{sl.pod.7.16.1.pic}
\end{figure}


 \textbf{\textit{Proof.}}
  (Figure \ref{sl.pod.7.16.1.pic})

  Naj bosta $X$ in $V$ presečišči ustreznih trisektris notranjih kotov $ABC$ in $ACB$ trikotnika $ABC$, tako da je $X$ notranja točka trikotnika $BVC$. Označimo še z $Z_1$ in $Y_1$ točki trisektris $BV$ in $CV$, za kateri velja $\angle Z_1XV \cong \angle VXY_1=30^0$. Dokažimo najprej, da je $XY_1Z_1$ enakostranični trikotnik.
Ker sta premici $BX$ in $CX$ simetrali kotov $VBC$ in $VCB$, je točka $X$ središče včrtane krožnice trikotnika $BVC$, zato je tudi premica $VX$ simetrala kota $BVC$ (izrek \ref{SredVcrtaneKrozn}). Iz skladnosti trikotnikov $VXZ_1$ in
$VXY_1$ (izrek \textit{ASA} \ref{KSK}) sledi $XZ_1\cong XY_1$. Ker je še $\angle Z_1XY_1=60^0$, je $XY_1Z_1$ enakostranični trikotnik (izrek \ref{enakokraki}). Dovolj je, če še dokažemo, da velja $Y_1=Y$ in $Z_1=Z$ oz. $\angle BAZ_1\cong\angle Z_1AY_1\cong\angle Z_1AC$.

Iz skladnosti trikotnikov $VXZ_1$ in
$VXY_1$ sledi še $Z_1V \cong Y_1V$. Torej je $VZ_1Y_1$ enakokraki trikotnik z osnovnico $Z_1Y_1$, zato je po izreku \ref{enakokraki}:
\begin{eqnarray} \label{eqnMorley1}
\angle VZ_1Y_1\cong\angle VY_1Z_1.
 \end{eqnarray}
Naj bosta $Z'$ in $Y'$ takšni točki na stranicah $AB$ in $AC$ trikotnika $ABC$, da velja $BZ'\cong BX$ in $CY'\cong CX$.
Trikotnika $BZ'Z_1$ in $BXZ_1$ sta skladna (izrek \textit{SAS} \ref{SKS}), zato je $Z'Z_1\cong XZ_1$. Prav tako je $\angle Z'Z_1B\cong\angle XZ_1B$ oz. je premica $BV$ simetrala kota $Z'Z_1X$.
Analogno je tudi $Y'Y_1\cong XY_1$. Iz tega in iz dejstva, da je $XY_1Z_1$ enakostranični trikotnik, sledi:
\begin{eqnarray} \label{eqnMorley2}
Z'Z_1\cong Z_1Y_1\cong Y_1Y'.
 \end{eqnarray}
 Označimo s $3\alpha$, $3\beta$, $3\gamma$ mere
notranjih trikotnika $ABC$, kotov ob ogliščih $A$, $B$ in $C$.
Jasno je $3\alpha+3\beta+3\gamma=180^0$, torej:
\begin{eqnarray} \label{eqnMorley3}
2\alpha+2\beta+2\gamma=120^0.
 \end{eqnarray}
Če uporabimo dejstvo, da je premica $BV$  simetrala konveksnega oz. nekonveksnega kota $Z'Z_1X$ ter relaciji \ref{eqnMorley1} in \ref{eqnMorley3}, po enostavnem računanju dobimo:
\begin{eqnarray*}
 \angle Z'Z_1Y_1&=&\angle Z'Z_1V+\angle VZ_1Y_1=\\
 &=&\angle VZ_1X+\angle VZ_1Y_1=\\
 &=&60^0+2\angle VZ_1Y_1=\\
 &=&60^0+\angle VZ_1Y_1+\angle VY_1Z_1=\\
 &=&60^0+180^0-\angle Z_1VY_1=\\
 &=&60^0+2\beta+2\gamma=\\
 &=&60^0+120^0-2\alpha=\\
 &=&180^0-2\alpha.
 \end{eqnarray*}
 Analogno je tudi $\angle Y'Y_1Z_1=180^0-2\alpha$, zato je (če upoštevamo še relacijo \ref{eqnMorley2}) po lemi \ref{izrekMorleyLema}
  $\angle Z'AZ_1\cong\angle Z_1AY_1\cong\angle Y_1AY'= \alpha$, kar je bilo potrebno dokazati.
 \kdokaz

                 

                \bizrek (Leibniz's\footnote{\index{Leibniz, G. W.}\textit{G. W. Leibniz} (1646--1716), nemški matematik.} theorem)
                \label{izrekLeibniz}\index{theorem!Leibniz's}
                If $T$ is the centroid and $X$ an arbitrary
                point in the plane of a triangle $ABC$, then
                $$|XA|^2 + |XB|^2 + |XC|^2 = \frac{1}{3}\left(|AB|^2 +|BC|^2 +|CA|^2\right) + 3|XT|^2
                ,\textrm{ i.e.}$$
                $$|XA|^2 + |XB|^2 + |XC|^2 = |TA|^2 +|TB|^2 +|TC|^2 + 3|XT|^2.$$
                \eizrek


\begin{figure}[!htb]
\centering
\input{sl.pod.7.16.2.pic}
\caption{} \label{sl.pod.7.16.2.pic}
\end{figure}


 \textbf{\textit{Proof.}} (Figure \ref{sl.pod.7.16.2.pic})
  Naj bo $A_1$ središče
stranice $BC$ oz. $|AA_1|=t_a$ njegova
težiščnica. Ker je $AT:TA_1=2:1$ (izrek \ref{tezisce}), po
Stewartovem izreku \ref{StewartIzrek2}
glede na trikotnik
$AXA_1$ sledi:
\begin{eqnarray} \label{eqnLeibniz1}
|XT|^2=\frac{1}{3}|XA|^2+\frac{2}{3}|XA_1|^2-\frac{2}{9}t_a^2
\end{eqnarray}
Če še enkrat uporabimo Stewartov izrek  \ref{StewartIzrek2} (ali njegovo posledico za težiščnico \ref{StwartTezisc}) glede na trikotnik $BXC$, dobimo:

\begin{eqnarray} \label{eqnLeibniz2}
|XA_1|^2=\frac{1}{2}|XB|^2+\frac{1}{2}|XC|^2-\frac{1}{4}|BC|^2
\end{eqnarray}
in če vstavimo \ref{eqnLeibniz2} v \ref{eqnLeibniz1}:
\begin{eqnarray} \label{eqnLeibniz3}
|XT|^2=\frac{1}{3}\left(|XA|^2+|XB|^2+|XC|^2\right)-\frac{1}{6}|BC|^2-\frac{2}{9}t_a^2
\end{eqnarray}
ter analogno še:
\begin{eqnarray} \label{eqnLeibniz4a}
\hspace*{-4mm} |XT|^2&=&\frac{1}{3}\left(|XA|^2+|XB|^2+|XC|^2\right)-\frac{1}{6}|AC|^2-\frac{2}{9}t_b^2\\
\hspace*{-4mm} |XT|^2&=&\frac{1}{3}\left(|XA|^2+|XB|^2+|XC|^2\right)-\frac{1}{6}|AB|^2-\frac{2}{9}t_c^2\label{eqnLeibniz4}
\end{eqnarray}
S seštevanjem enakosti iz \ref{eqnLeibniz3} - \ref{eqnLeibniz4} dobimo:
\begin{eqnarray*}
3|XT|^2&=&|XA|^2+|XB|^2+|XC|^2-\frac{1}{6}\left(|BC|^2+|AC|^2+|AB|^2\right)-\\
&-&\frac{2}{9}\left(t_a^2+t_b^2+t_c^2\right)
\end{eqnarray*}
Na koncu iz trditve \ref{StwartTezisc2} sledi:
\begin{eqnarray*}
3|XT|^2=|XA|^2+|XB|^2+|XC|^2-\frac{1}{3}\left(|BC|^2+|AC|^2+|AB|^2\right)
\end{eqnarray*}
oz. obe relaciji iz trditve.
\kdokaz

Direktna posledica Leibnizovega izreka je naslednja trditev.




                \bizrek
                A point in the plane of a triangle for which the squared distances
                 from its vertices has a minimum value is its centroid. 
                \eizrek

 \textbf{\textit{Proof.}} Po Leibnitzovem izreku za poljubno točko $X$ v ravnini trikotnika $ABC$ s težiščem $T$ velja:
$$|XA|^2 + |XB|^2 + |XC|^2 = \frac{1}{3}\left(|AB|^2 +|BC|^2 +|CA|^2\right) + 3|XT|^2.$$
Minimum vsote $|XA|^2 + |XB|^2 + |XC|^2$ po $X$ se torej doseže, ko je $|XT|$ najmanjše, to pa je, kadar je $X=T$.
\kdokaz


               
               


                \bizrek (Carnot's\footnote{\index{Carnot, L. N. M.}\textit{L. N. M. Carnot} (1753--1823), 
                francoski matematik.} theorem)\index{theorem!Carnot's}
                Let $P$, $Q$ and $R$ be points on the lines containing the
                sides $BC$, $CA$ and $AB$ of a triangle $ABC$. Perpendicular lines on the lines $BC$, $CA$.
                and $AB$ at the points $P$, $Q$ and $R$ intersect at one point if and only if
                \begin{eqnarray} \label{eqnCarnotIzrek1}
                |BP|^2 - |PC|^2 + |CQ|^2 - |QA|^2 + |AR|^2 - |RB|^2 = 0.
                \end{eqnarray}

                \eizrek


\begin{figure}[!htb]
\centering
\input{sl.pod.7.16.3.pic}
\caption{} \label{sl.pod.7.16.3.pic}
\end{figure}

 \textbf{\textit{Proof.}} Označimo s $p$, $q$ in $r$ pravokotnice premic $BC$, $CA$ in $AB$ skozi točke $P$, $Q$ in $R$ (Figure \ref{sl.pod.7.16.3.pic}).

($\Rightarrow$) Predpostavimo najprej, da se premice $p$, $q$ in $r$  sekajo v neki točki $L$. Če šestkrat uporabimo Pitagorov izrek \ref{PitagorovIzrek}, dobimo:
 \begin{eqnarray*}
\begin{array}{cc}
  |AR|^2+|RL|^2=|AL|^2 & \hspace{6mm} -|AQ|^2-|QL|^2=-|AL|^2 \\
  |BP|^2+|PL|^2=|BL|^2 & \hspace{6mm} -|BR|^2-|RL|^2=-|BL|^2\\
  |CQ|^2+|QL|^2=|CL|^2 & \hspace{6mm} -|CP|^2-|PL|^2=-|CL|^2
\end{array}
\end{eqnarray*}
 Če seštejemo vseh šest enakosti, dobimo relacijo \ref{eqnCarnotIzrek1}.

($\Leftarrow$) Predpostavimo sedaj, da velja
relacija \ref{eqnCarnotIzrek1}.
Pravokotnici $q$ in $r$ stranic $AC$ in $AB$  nista vzporedni (ker bi bile sicer po posledici Playfairjevega aksioma  \ref{Playfair1} točke $A$, $B$ in $C$ kolinearne). Označimo z $\widehat{L}$ presečišče premic $q$ in $r$. Označimo s $\widehat{P}$ pravokotno projekcijo točke  $\widehat{L}$ na premici $BC$. Ker se pravokotnice nosilk stranic trikotnika $ABC$ v točkah $\widehat{P}$, $Q$ in $R$ sekajo v točki $\widehat{L}$, iz prvega dela dokaza ($\Rightarrow$) sledi:
    \begin{eqnarray} \label{eqnCarnotIzrek2}
    |B\widehat{P}|^2 - |\widehat{P}C|^2 + |CQ|^2 - |QA|^2 + |AR|^2 - |RB|^2 = 0.
    \end{eqnarray}
 Iz \ref{eqnCarnotIzrek1} in \ref{eqnCarnotIzrek2} sledi:
$$|B\widehat{P}|^2 - |\widehat{P}C|^2=|BP|^2 - |PC|^2.$$
 Na enak način kot na koncu drugega dela dokaza izreka  \ref{PotencOsLema} dobimo $P=\widehat{P}$, oz. premice $p$, $q$ in $r$ se sekajo v eni točki.
\kdokaz


                

                \bizrek (Butterfly theorem\footnote{En dokaz tega izreka je objavil angleški matematik \index{Horner, W. J.}\textit{W. J. Horner} (1786--1837) leta 1815.})
                \index{theorem!butterfly}
                Let $S$ be the midpoint of chord $PQ$ of a circle $k$. Suppose that $AB$ and $CD$
                are arbitrary chords of this circle passing through the point $S$. If $X$ and $Y$
                are the points of intersection of the chords $AD$ and $BC$ with the chord $PQ$,
                then $S$ is the midpoint of the line segment $XY$.
                \eizrek



\begin{figure}[!htb]
\centering
\input{sl.pod.7.16.4.pic}
\caption{} \label{sl.pod.7.16.4.pic}
\end{figure}


 \textbf{\textit{Proof.}} Označimo $x=|SX|$ in $y=|SY|$
 (Figure \ref{sl.pod.7.16.4.pic}).
  Naj bodo
$X_1$ in $X_2$ oz. $Y_1$ in $Y_2$
pravokotne projekcije točk $X$ oz. $Y$ na tetivah $AB$
in $CD$.
Po Talesovem izreku (\ref{TalesovIzrekDolzine}) je:
\begin{eqnarray*}
 \frac{x}{y}=\frac{XX_1}{YY_1}=\frac{XX_2}{YY_2}
\end{eqnarray*}
 Če uporabimo slednje enakosti, nato
podobnost trikotnikov $AXX_1$ in $CYY_2$ oz.
trikotnikov $DXX_2$ in $BYY_1$ (izrek \ref{PodTrikKKK}) ter na koncu še potenco
točk $X$ in $Y$ glede na krožnico $k$ (izrek \ref{izrekPotenca}), dobimo:
\begin{eqnarray*}
 \frac{x^2}{y^2}&=& \frac{|XX_1|}{|YY_1|}\cdot\frac{|XX_2|}{|YY_2|}=
 \frac{|XX_1|}{|YY_2|}\cdot\frac{|XX_2|}{|YY_1|}=\\
 &=& \frac{|AX|}{|CY|}\cdot\frac{|DX|}{|BY|}=\frac{|AX|\cdot |DX|}{|CY|\cdot |BY|}=\\
 &=& \frac{|XP|\cdot |XQ|}{|YP|\cdot |YQ|}=
 \frac{\left(|PS|-x\right)\cdot\left(|PS|+x\right)}
 {\left(|PS|+y\right)\cdot\left(|PS|-y\right)}=\\
 &=& \frac{|PS|^2-x^2}{|PS|^2-y^2}.
\end{eqnarray*}
 Iz $$\frac{x^2}{y^2}=\frac{|PS|^2-x^2}{|PS|^2-y^2}$$ na koncu sledi $x=y$.
\kdokaz


                \bizrek \index{krožnica!Taylorjeva} \label{izrekTaylor}
                Let $A'$, $B'$ and $C'$ be the foots of the altitudes of a triangle $ABC$.
                The foot of the perpendiculars from the points $A'$,
                $B'$ and $C'$ on the lines containing the  adjacent sides
                of that triangle lie on a circle, so-called \pojem{Taylor\footnote{\index{Taylor, B.} \textit{B. Taylor} (1685--1731), angleški matematik.}  circle} \color{blue} of that triangle.
                \eizrek


\begin{figure}[!htb]
\centering
\input{sl.pod.7.16.5.pic}
\caption{} \label{sl.pod.7.16.5.pic}
\end{figure}


 \textbf{\textit{Proof.}}
 Naj bosta $A_c$ in $A_b$ pravokotni projekciji točk $C'$ in $B'$ na premici $BC$,
 $B_a$ in $B_c$ pravokotni projekciji točk $A'$ in $C'$ na premici $AC$ ter
 $C_a$ in $C_b$ pravokotni projekciji točk $A'$ in $B'$ na premici $AB$ (Figure \ref{sl.pod.7.16.5.pic}).

 Trikotnik $A'B'C'$ je
pedalni trikotnik trikotnika $ABC$, zato iz dokaza izreka \ref{PedalniVS} sledi
$\angle B'A'C\cong\angle BAC=\alpha$ in $\angle C'B'A\cong\angle CBA=\beta$. Ker je $\angle B'B_aA'=\angle B'A_bA'=90^0$, je po izreku \ref{TalesovIzrKroz2} $A'A_bB_aB'$ tetivni štirikotnik, zato je tudi $\angle A_bB_aC\cong B'A'C=\alpha$ (izrek \ref{TetivniPogojZunanji}). Iz tega po izreku \ref{KotiTransverzala} sledi $A_bB_a \parallel BA$.

Trikotnika $C'A_cB_c$ in $VA'B'$ sta perspektivna glede na točko $C$ (to pomeni, da se premice $C'V$, $A_cA'$ in $B_cB'$ sekajo v točki $C$). Ker je $C'A_c\parallel VA'$ in $C'B_c\parallel VB'$, je po posplošitvi Desarguesovega izreka \ref{izrekDesarguesOsNesk} tudi $A_cB_c\parallel A'B'$. Po izreku \ref{KotiTransverzala1} je potem $\angle B_cA_cA'\cong B'A'C=\alpha$. Ker je torej $\angle B_cA_cA'=\alpha=A_bB_aC$, je $A_cA_bB_aB_c$ tetivni štirikotnik (izrek \ref{TetivniPogojZunanji}); označimo njegovo očrtano krožnico s $k$.

Ostane nam še dokaz, da tudi točki $C_a$ in $C_b$ ležita na
krožnici $k$.  Analogno z dokazanim $A_cB_c\parallel A'B'$ je tudi $C_aB_a\parallel C'B'$ in analogno z dokazanim, da je $\angle A_bB_aC\cong B'A'C=\alpha$, je tudi $\angle C_aA_cB\cong\angle A'C'B=\gamma$.
Iz vzporednosti $C_aB_a\parallel C'B'$ sledi (izrek \ref{KotiTransverzala1})
$\angle C_aB_aA \cong\angle C'B'A = \beta$, zato je:
 $$\angle C_aB_aA_b=180^0-\angle A_bB_aC-\angle C_aB_aA= 180^0-\alpha-\beta=\gamma.$$
 Torej $\angle C_aA_cB\cong\angle C_aB_aA_b=\gamma$, kar pomeni (izrek \ref{TetivniPogojZunanji}), da je $C_aA_cA_bB_a$ tetivni štirikotnik, torej točka $C_a$ leži na krožnici $k$. Analogno dokažemo, da tudi točka $C_b$ leži na krožnici $k$.
\kdokaz

 A direct consequence is the following theorem.


            
                \bizrek
             Let $A'$, $B'$ and $C'$ be the foots of the altitudes of a triangle $ABC$.
                The foot of the perpendiculars from the points $A'$,
                $B'$ and $C'$ on the lines containing the  adjacent sides
                of that triangle determine three congruent line segments.
                \eizrek


\begin{figure}[!htb]
\centering
\input{sl.pod.7.16.6.pic}
\caption{} \label{sl.pod.7.16.6.pic}
\end{figure}


 \textbf{\textit{Proof.}} (Figure \ref{sl.pod.7.16.6.pic})

V dokazu prejšnjega izreka \ref{izrekTaylor} smo ugotovili, da velja $A_bB_a\parallel BA$, zato je štirikotnik $C_bC_aA_bB_a$ trapez. Ta
trapez je hkrati tetiven (prejšnji izrek \ref{izrekTaylor}) in je zato tudi enakokrak (\ref{trapezTetivEnakokr}). Njegovi diagonali sta skladni (izrek \ref{trapezEnakokraki}), oz. velja
$A_bC_b\cong C_aB_a$. Analogno je tudi $A_bC_b\cong A_cB_c$.
\kdokaz



 \vspace*{12mm}


%________________________________________________________________________________
\naloge{Exercises}

\begin{enumerate}



%Tales

\item
Naj bo $S$ presečišče diagonal $AC$ in $BD$ trapeza $ABCD$. Naj bosta
 $P$ in $Q$ presečišči vzporednice osnovnic $AB$ in $CD$ skozi točko $S$ s krakoma tega trapeza. Dokaži, da je $S$ središče daljice $PQ$.

\item
Naj bo $ABCD$ trapez z osnovnico $AB$, točka $S$ presečišče njegovih
diagonal in $E$ presečišče nosilk krakov tega trapeza. Dokaži, da
premica $SE$ poteka skozi  središči osnovnic $AB$ in $CD$.

\item
Naj bodo $P$, $Q$ in $R$ točke, v katerih poljubna premica skozi točko $A$ seka
nosilke stranic $BC$ in $CD$ ter diagonalo $BD$ paralelograma
$ABCD$. Dokaži, da je $|AR|^2=|PR|\cdot |QR|$.

\item
 Točki $D$ in $K$ ležita na stranicah $BC$ in $AC$ trikotnika $ABC$ tako, da velja $BD:DC=2:5$ in $AK:KC=3:2$. Izračunaj razmerje, v katerem premica $BK$ deli daljico
$AD$.

\item
Naj bo $P$ takšna točka stranice $AD$ paralelograma $ABCD$, da velja $\overrightarrow{AP}=
\frac{1}{n}\overrightarrow{AD}$, in $Q$ presečišče  premic $AC$ in $BP$. Dokaži, da velja:
 $$\overrightarrow{AQ}=\frac{1}{n + 1}\overrightarrow{AC}.$$



%Homotetija

\item
Dani so: točka $A$, premici $p$ in $q$ ter daljici $m$ in $n$. Načrtaj premico $s$ skozi točko $A$,
ki naj seka premici $p$ in $q$ v takšnih točkah $X$ in $Y$, da velja $XA:AY= m:n$.

\item Dani so: točka $S$, premice $p$, $q$ in $r$ ter daljici $m$ in $n$. Načrtaj premico $s$ skozi točko $S$,
ki naj seka  premice $p$, $q$ in $r$ v takšnih točkah  $X$, $Y$ in $Z$, da velja $XY:YZ=m:n$.

 \item V dani trikotnik $ABC$ včrtaj takšen pravokotnik $PQRS$, da stranica $PQ$ leži na stranici $BC$,
oglišči $R$ in $S$ ležita na stranicah $AB$ in $AC$, pri tem pa še velja $PQ=2QR$.

\item
Načrtaj:\\
(\textit{a}) romb, če sta dani stranica $a$ in razmerje diagonal $e:f$;\\
 (\textit{b}) trapez, če so dani: notranja kota $\alpha$ in $\beta$ ob eni osnovnici, razmerje te osnovnice in višine $a:v$ ter druga osnovnica $c$.

\item
Naj bodo: $A$ točka v notranjosti kota $pSq$, $l$ premica in $\alpha$ kot v neki ravnini. Načrtaj takšen trikotnik
$APQ$, da velja: $P\in p$, $Q\in q$, $\angle PAQ\cong \alpha$ in
$PQ\parallel l$.

\item
Načrtaj krožnico $l$, ki se dotika dane krožnice $k$ in dane premice $p$, če je dano še dotikališče:\\
(\textit{a}) $l$ in $p$;\hspace*{3mm}
(\textit{b}) $l$ in $k$.

\item
Načrtaj krožnico, ki se dotika krakov kota $pSq$ in:\\
(\textit{a}) poteka skozi dano točko,\\
(\textit{b}) se dotika dane krožnice.

\item Naj bodo: $p$ in $q$ premici, $S$ točka ($S\notin p$ in $S\notin q$) ter daljici $m$ in $n$.
Načrtaj krožnici $k$ in $l$, ki se od zunaj dotikata v točki $S$, prva se dotika premice $p$,
druga premice $q$, razmerje polmerov pa je enako $m:n$.


\item
V ravnini so dani: premica $p$ ter  točki $B$ in $C$, ki ležita na krožnici $k$. Načrtaj takšno točko $A$
na krožnici $k$, da težišče trikotnika $ABC$ leži na premici $p$.

\item
Naj bo $k$ krožnica s premerom $PQ$. Načrtaj kvadrat $ABCD$ tako, da velja $A,B\in PQ$ in $C,D\in k$.

\item
V isti ravnini so dani: premici $p$ in $q$, točka $A$ ter daljici $m$ in $n$. Načrtaj takšen
pravokotnik $ABCD$, da velja $B\in p$, $D\in q$
in $AB:AD=m:n$.

\item V dani trikotnik $ABC$ včrtaj trikotnik tako, da bodo njegove stranice vzporedne z danimi premicami $p$, $q$ in $r$.

\item Naj bodo $p$, $q$ in $r$ tri premice neke ravnine. Načrtaj premico $t$, ki je
pravokotna  na premico $p$ in seka premice $p$, $q$ in $r$ po vrsti v takšnih točkah $P$, $Q$ in $R$, da velja $PQ\cong QR$.




%Podobnost trikotnikov

\item Naj bo $P$ notranja točka trikotnika $ABC$ ter $A_1$, $B_1$ in $C_1$ pravokotne projekcije
točke $P$ na trikotnikovih stranicah $BC$, $AC$ in $AB$. Analogno so točke $A_2$, $B_2$ in $C_2$ določene s točko $P$
in trikotnikom $A_1B_1C_1$,..., točke $A_{n+1}$, $B_{n+1}$ in $C_{n+1}$  s točko $P$ in trikotnikom $A_nB_nC_n$... Kateri od trikotnikov $A_1B_1C_1$, $A_2B_2C_2$, ... so podobni trikotniku $ABC$?

\item
Naj bo $k$ očrtana krožnica štirikotnika $ABCD$, $E$ presečišče njegovih diagonal ter $CB\cong CD$. Dokaži, da je $\triangle ABC \sim\triangle BEC$.

\item
Naj bo $ABCD$ paralelogram. V točkah $E$ in $F$ trikotniku $ABC$ očrtana krožnica seka premici $AD$ in $CD$. Dokaži, da je $\triangle EBC\sim\triangle EFD$.

\item
 Naj bosta $AA'$ in $BB'$ višini ostrokotnega trikotnika $ABC$. Dokaži, da je
$\triangle ABC\sim\triangle A'B'C$.

\item
Naj bo nosilka višine $AD$ trikotnika $ABC$ hkrati tangenta očrtane krožnice tega trikotnika. Dokaži, da velja $|AD|^2=|BD|\cdot |CD|$.

\item
V trikotniku $ABC$ naj bo notranji kot ob oglišču $A$ dvakrat  večji
od notranjega kota ob oglišču $B$. Dokaži, da velja $|BC|^2= |AC|^2+|AC|\cdot |AB|$.

\item Dokaži, da sta polmera očrtanih krožnic dveh podobnih trikotnikov sorazmerna z ustreznima stranicama teh dveh trikotnikov.


 \item Krožnici s središčem $S$ je včrtan štirikotnik $ABCD$. Diagonali tega štirikotnika sta pravokotni in se sekata v točki $E$. Premica, ki poteka skozi točko $E$ in je pravokotna na stranici $AD$, seka stranico $BC$ v točki $M$. \\
(\textit{a}) Dokaži, da je točka $M$ središče daljice $BC$.\\
(\textit{b}) Določi množico vseh točk $M$, če se diagonala $BD$ spreminja in je vedno pravokotna na diagonalo $AC$.

\item Naj bo $t$ tangenta očrtane krožnice $l$ trikotnika $ABC$ v oglišču $A$.
Naj bo $D$ takšna točka premice $AC$, da je $BD\parallel t$. Dokaži, da velja
$|AB|^2=|AC|\cdot |AD|$.

 \item Višinska točka ostrokotnega trikotnika naj deli njegovi višini v enakem razmerju (od oglišča do nožišča višine). Dokaži, da gre za enakokraki trikotnik.

\item V trikotniku $ABC$ se višina $BD$ dotika očrtane krožnice tega trikotnika.
Dokaži:\\
(\textit{a}) da je razlika kotov ob osnovnici $AC$ enaka $90^0$,\\
(\textit{b}) da velja $|BD|^2=|AD|\cdot |CD|$.

 \item Krožnica s središčem na osnovnici $BC$ enakokrakega trikotnika $ABC$ se dotika
krakov $AB$ in $AC$. Točki $P$ in $Q$ sta presečišči teh krakov s poljubno
tangento te krožnice. Dokaži, da velja $4\cdot |PB|\cdot |CQ|=|BC|^2$.

\item Naj bo $V$ višinska točka ostrokotnega trikotnika $ABC$, točka $V$ središče višine
$AD$, višino $BE$ pa točka $V$ deli v razmerju $3:2$. Izračunaj razmerje, v katerem $V$ deli višino $CF$.

\item Naj bo $S$ zunanja točka krožnice $k$. $P$ in $Q$
sta točki, v katerih se krožnica $k$ dotika svojih tangent iz točke $S$, $X$ in $Y$ pa presečišči te krožnice s poljubno premico, ki poteka skozi točko $S$. Dokaži da je $XP:YP=XQ:YQ$.

\item Naj bo $D$  točka, ki leži na stranici $BC$ trikotnika $ABC$. Točki $S_1$ in $S_2$ naj bosta središči očrtanih krožnic trikotnikov $ABD$ in $ACD$. Dokaži, da velja
 $\triangle ABC\sim\triangle AS_1S_2$.

\item Točka $P$ leži na hipotenuzi $BC$ trikotnika  $ABC$. Pravokotnica premice $BC$ v točki $P$ seka premici $AC$ in $AB$ v točkah $Q$ in $R$ ter očrtano krožnico trikotnika $ABC$ v točki $S$. Dokaži, da velja $|PS|^2=|PQ|\cdot |PR|$.

\item Točka $A$  leži na kraku $OP$ pravega kota $POQ$. Naj bodo
$B$, $C$ in $D$ takšne točke kraka $OQ$, da velja $\mathcal{B}(O,B,C)$, $\mathcal{B}(B,C,D)$ in
$OA\cong OB\cong BC\cong CD$. Dokaži, da velja tudi $\triangle ABC\sim\triangle DBA$.

\item Načrtaj trikotnik, če so znani naslednji podatki:\\
(\textit{a}) $\alpha$, $\beta$, $R+r$, \hspace*{3mm}
 (\textit{b}) $a$, $b:c$, $t_c-v_c$,\hspace*{3mm}
 (\textit{c}) $v_a$, $v_b$, $v_c$.

\item Naj bosta $AB$ in $CD$ osnovnici enakokrakega tangentnega trapeza $ABCD$, $r$ pa naj bo polmer včrtane krožnice. Dokaži, da je $|AB|\cdot |CD|=4r^2$.




%Harmon cetverica

\item Dane so krožnica $k$ ter točki $A$ in $B$. Načrtaj takšno točko $X$ na krožnici $k$, da bo $AX:XB=2:5$.

\item Načrtaj trikotnik z danimi podatki:\\
(\textit{a}) $a$, $v_a$, $b:c$, \hspace*{3mm}
 (\textit{b}) $a$, $t_a$, $b:c$,\hspace*{3mm}
 (\textit{c}) $a$, $b$, $b:c$,\\
 (\textit{d}) $a$, $\alpha$, $b:c$,\hspace*{3mm}
  (\textit{e}) $a$, $l_a$, $b:c$.

\item Načrtaj trikotnik, če so dani naslednji podatki:\\
(\textit{a}) $v_a$, $r$, $\alpha$, \hspace*{3mm}
(\textit{b}) $v_a$, $r_a$, $a$, \hspace*{3mm}
(\textit{c}) $v_a$, $t_a$, $b-c$.

\item Načrtaj paralelogram, pri katerem sta ena stranica in ustrezna višina skladni z danima daljicama $a$ in $v_a$, diagonali pa sta v razmerju $3:5$.

\item Točka $E$ naj bo presečišče simetrale notranjega kota $BAC$ trikotnika $ABC$ z njegovo stranico $BC$. Dokaži da velja:
    $$\overrightarrow{AE}=\frac{|AC|}{|AB|+|AC|}\cdot\overrightarrow{AB}+
    \frac{|AB|}{|AB|+|AC|}\cdot\overrightarrow{AC}.$$

\item Dane so štiri kolinearne točke, za katere velja $\mathcal{H}(A,B;C,D)$. Načrtaj točko $L$, iz katere se daljice $AC$, $CB$ in $BD$ vidijo pod enakim kotom.

\item Naj bo $AE$ ($E\in BC$) simetrala notranjega kota trikotnika $ABC$ ter $a=|BC|$, $b=|AC|$ in  $c=|AB|$. Dokaži, da velja:
$$|BE|=\frac{ac}{b+c} \hspace*{1mm} \textrm{ in } \hspace*{1mm}  |CE|=\frac{ab}{b+c}.$$

\item Naj bosta $AE$ ($E\in BC$) in $BF$ ($F\in AC$) simetrali notranjih kotov ter $S$ središče včrtane krožnice trikotnika $ABC$. Dokaži, da je $ABC$ enakokraki trikotnik (z osnovnico $AB$) natanko tedaj, ko je $AS:SE=BS:SF$.




%Menelaj Ceva


\item Dokaži, da simetrale zunanjih kotov poljubnega trikotnika sekajo nosilke nasprotnih stranic v treh kolinearnih točkah.



%Ppitagorov iizrek

\item Če so $a$, $b$ in $c$ ($a>b$) dane daljice, načrtaj takšno daljico $x$, da velja:\\
(\textit{a}) $x=\sqrt{a^2+b^2}$, \hspace*{3mm}
(\textit{b}) $x=\sqrt{a^2-b^2}$, \hspace*{3mm}
(\textit{c}) $x=\sqrt{3ab}$,\\
(\textit{d}) $x=\sqrt{a^2+bc}$, \hspace*{3mm}
(\textit{e}) $x=\sqrt{3ab-c^2}$, \hspace*{3mm}
(\textit{f}) $x=\frac{a\sqrt{ab+c^2}}{b+c}$.



%Stewartov iizrek

\item Naj bodo $a$, $b$ in $c$ stranice nekega trikotnika in velja $a^2+b^2=5c^2$. Dokaži, da sta težiščnici,
ki ustrezata stranicama $a$ in $b$, med seboj pravokotni.

\item Naj bodo $a$, $b$, $c$ in $d$ stranice, $e$ in $f$ diagonali ter $x$ daljica, ki je določena s središčema
 stranic $b$ in $d$ nekega štirikotnika. Dokaži:
$$x^2 = \frac{1}{4} \left(a^2 +c^2 -b^2 -d^2 +e^2 +f^2 \right).$$

\item Naj bodo $a$, $b$ in $c$ stranice trikotnika $ABC$. Dokaži, da je razdalja središča $A_1$ stranice $a$ od nožišča $A'$ višine na
to stranico enaka:
$$|A_1A'|=\frac{|b^2-c^2|}{2a}.$$



%Pappus in Ppascal

\item Naj bosta ($A$, $B$, $C$) ter ($A_1$, $B_1$, $C_1$) trojici kolinearnih točk neke ravnine, ki nista na
isti premici. Če je $AB_1\parallel A_1B$ in $AC_1\parallel A_1C$, tedaj je tudi $CB_1\parallel C_1B$. (\textit{Pappusov izrek}\footnote{Pappus iz Aleksandrije\index{Pappus} (3. st.), starogrški matematik. Gre za posplošitev Pappusovega izreka (glej izrek \ref{izrek Pappus}), če za $X$, $Y$ in $Z$ izberemo točke v neskončnosti.})



%Desarguesov iizrek


\item Naj bodo $P$, $Q$ in $R$ takšne točke stranic $BC$, $AC$ in $AB$
trikotnika $ABC$, da so premice $AP$, $BQ$ in
$CR$ iz istega šopa. Dokaži: Če je $X=BC\cap QR$, $Y=AC\cap PR$ in $Z=AB\cap PQ$, so točke
$X$, $Y$ in $Z$
kolinearne.


\item Naj bodo $AA'$, $BB'$ in $CC'$ višine trikotnika $ABC$ ter $X=B'C'\cap BC$,  $Y=A'C'\cap AC$ in $Z=A'B'\cap AB$. Dokaži, da so $X$, $Y$ in $Z$ kolinearne točke.

\item Naj bosta $A$ in $B$ točki izven premice $p$. Načrtaj presečišče premic $p$ in $AB$ brez direktnega risanja premice
$AB$.

\item Naj bosta $p$ in $q$ premici neke ravnine, ki se sekata v točki $S$, ki je ‘‘izven papirja’’, in $A$
točka te ravnine. Načrtaj premico, ki poteka skozi točki $A$ in $S$.

\item Načrtaj trikotnik tako, da njegova oglišča ležijo na treh danih vzporednih premicah, nosilke njegovih stranic pa gredo skozi tri dane
točke.




%Ppotenca

\item
Dana je  krožnica $k(S,r)$.\\
(\textit{a}) Katere vrednosti vse lahko ima potenca točke glede na krožnico $k$?\\
 (\textit{b}) Katera je najmanjša vrednost te potence in za katero točko se ta minimalna vrednost doseže?\\
(\textit{c}) Določi množico vseh točk, za katere je potenca glede na krožnico enaka $\lambda\in \mathbb{R}$.

\item Naj bosta $k_a(S_a,r_a)$ in $l(O,R)$ pričrtana in očrtana krožnica nekega trikotnika. Dokaži enakost\footnote{Trditev je posplošitev Eulerjeve formule za krožnico (glej izrek \ref{EulerjevaFormula}). \index{Euler, L.}
        \textit{L. Euler}
        (1707--1783), švicarski matematik.}:
   $$S_aO^2=R^2+2r_aR.$$


\item Načrtaj krožnico, ki poteka skozi dani točki $A$ in $B$ in se dotika dane krožnice $k$.

\item Dokaži, da so središča daljic, ki so določena s skupnimi tangentami dveh krožnic, kolinearne točke.

\item Načrtaj krožnico, ki je pravokotna na dve dani krožnici, tretjo dano krožnico
pa seka v točkah, ki določata premer te tretje krožnice.



%Raazno

\item Naj bosta $M$ in $N$ presečišči stranic $AB$ in $AC$ trikotnika $ABC$ s premico,
ki poteka skozi središče včrtane krožnice tega trikotnika in je vzporedna z njegovo
stranico $BC$. Izrazi dolžino daljice $MN$ kot funkcijo dolžin stranic trikotnika $ABC$.

\item Naj bo $AA_1$ težiščnica trikotnika $ABC$. Točki $P$ in $Q$ naj bosta presečišči
simetral kotov $AA_1B$ in $AA_1C$ s stranicami $AB$ in $AC$. Dokaži, da je
$PQ\parallel BC$.

\item V trikotniku $ABC$ naj bo vsota (ali razlika) notranjih kotov $ABC$ in $ACB$ enaka pravemu
kotu. Dokaži, da je $|AB|^2+|AC|^2=4r^2$, kjer je $r$ polmer očrtane krožnice tega trikotnika.

\item Naj bo $AD$ višina trikotnika $ABC$. Dokaži, da je
vsota (ali razlika) notranjih kotov $ABC$ in $ACB$ enaka pravemu kotu natanko tedaj, ko je:
$$\frac{1}{|AB|^2}+\frac{1}{|AC|^2}=\frac{1}{|AD|^2}.$$

\item Izrazi razdaljo med težiščem in središčem očrtane krožnice trikotnika kot
funkcijo dolžin njegovih stranic in polmera očrtane krožnice.

\item Dokaži, da pri trikotniku $ABC$ simetrala zunanjega kota ob oglišču $A$ in simetrali
notranjih kotov ob ogliščih $B$ in $C$ sekajo nosilke nasprotnih stranic v treh kolinearnih točkah.

\item Dokaži, da so pri trikotniku $ABC$ središče višine $AD$, središče včrtane krožnice in točka, v kateri se stranica $BC$ dotika pričrtane krožnice tega trikotnika, tri
kolinearne točke.

\item Dokaži Simsonov izrek \ref{SimpsPrem} z uporabo Menelajevega izreka \ref{izrekMenelaj}.

\item Skozi točko $M$ stranice $AB$ trikotnika $ABC$ je konstruirana premica, ki seka
premico $AC$ v točki $K$. Izračunaj razmerje, v katerem premica $MK$ deli stranico $BC$,
če je $AM:MB=1:2$ in $AK:AC=3:2$.

\item Naj bo $A_1$ središče stranice $BC$ trikotnika $ABC$ in naj bosta $P$ in $Q$ takšni točki stranic
$AB$ in $AC$, da velja $BP:PA=2:5$ in $AQ:QC=6:1$. Izračunaj razmerje, v katerem premica  $PQ$ deli težiščnico $AA_1$.

\item Dokaži, da se pri poljubnem trikotniku premice, ki so določene z oglišči in
dotikališči ene pričrtane krožnice z nosilkami nasprotnih stranic, sekajo v skupni točki.

\item Kaj predstavlja množica vseh točk, iz katerih odseka tangent glede na dve dani krožnici predstavljata dve skladni daljici?

\item Naj bosta $PP_1$ in $QQ_1$ zunanji tangenti krožnic $k(O,r)$ in $k_1(O_1,r_1)$ (točke $P$, $P_1$, $Q$ in $Q_1$ so ustrezna dotikališča). Točka $S$ naj bo presečišče teh dveh tangent, $A$ eno od presečišč krožnic $k$ in $k_1$ ter $L$ in $L_1$ presečišči
premice $SO$ s premicama $PQ$ in $P_1Q_1$. Dokaži, da velja $\angle LAO\cong\angle L_1AO_1$.

\item Dokaži, da je stranica pravilnega desetkotnika enaka
večjem delu delitve polmera očrtane krožnice tega  desetkotnika v razmerju zlatega reza.

\item Naj bodo $a_5$, $a_6$ in $a_{10}$ stranice pravilnega petkotnika, šestkotnika in desetkotnika, ki so včrtani isti krožnici. Dokaži, da velja:
 $$a_5^2=a_6^2+a_{10}^2.$$


\item Naj bodo $t_a$, $t_b$ in $t_c$ težiščnice in $s$ polobseg nekega trikotnika. Dokaži, da velja:
    $$t_a^2+t_b^2+t_c^2\geq s^2.$$ % zvezek - dodatni MG


\end{enumerate}




% DEL 8 - - - - - - - - - - - - - - - - - - - - - - - - - - - - - - - - - - - - - - -
%________________________________________________________________________________
% PLOŠČINA
%________________________________________________________________________________

 \del{Area of Figures} \label{pogPLO}


%________________________________________________________________________________
\poglavje{Area of Figures. Definition}  \label{odd8PloLik}

 V tem razdelku bomo definirali pojem ploščine določenega razreda likov. Ploščina temelji na pojmu dolžine daljice v nekem sistemu merjenja in na teoriji infinitezimalnega računa\footnote{Prve korake v tej smeri sta naredila starogrška matematika \index{Evdoks}\textit{Evdoks} (408--355 pr. n. š.) in \index{Arhimed}\textit{Arhimed} (287--212 pr. n. š.) pri  računanju prostornine teles.}

Naj bo $\Omega_0$ kvadrat s stranico dolžine $1$ v določenem sistemu merjenja. Omenjeni kvadrat predstavlja \index{enota merjenja ploščine}\pojem{enoto merjenja ploščine} - imenujemo ga tudi \index{enotski kvadrat}\pojem{enotski kvadrat}. Intuitivno ploščina nekega lika $\Phi$ predstavlja število enotskih kvadratov, ki so skladni kvadratu $\Omega_0$ in s katerimi lahko ''prekrijemo'' lik $\Phi$.

Sedaj bomo bolj formalno pristopili k definiranju ploščine. Naj bo $\Phi$ poljubni lik v ravnini. Enotski kvadrat $\Omega_0$ določa tlakovanje $(4,4)$ (glej razdelek \ref{odd3Tlakovanja}) - označimo ga s $\mathcal{T}_0$. S $\underline{S}_0$ označimo število kvadratov tlakovanja $\mathcal{T}_0$, ki ležijo v liku $\Phi$, oz. so njegova podmnožica.
S $\overline{S}_0$ pa označimo število kvadratov tlakovanja $\mathcal{T}_0$, ki imajo z likom $\Phi$ vsaj eno skupno točko (Figure \ref{sl.plo.8.1.1.pic}). Jasno je, da velja:
 $$0\leq\underline{S}_0\leq\overline{S}_0.$$


\begin{figure}[!htb]
\centering
\input{sl.plo.8.1.1.pic}
\caption{} \label{sl.plo.8.1.1.pic}
\end{figure}


Če vsako stranico kvadrata razdelimo na $10$ daljic, lahko enotski kvadrat $\Omega_0$ razdelimo na $10^2$ skladnih kvadratov, ki so vsi skladni enemu od teh kvadratov  $\Omega_1$, ki ima stranico dolžine $\frac{1}{10}$. Kvadrat $\Omega_1$ določa novo tlakovanje  $\mathcal{T}_1$. Podobno kot v prejšnjem primeru naj bo $\underline{S}_1$ število kvadratov tlakovanja $\mathcal{T}_1$, ki ležijo v liku $\Phi$, in $\overline{S}_1$ število kvadratov tlakovanja $\mathcal{T}_1$, ki imajo z likom $\Phi$ vsaj eno skupno točko. Jasno je, da velja $\underline{S}_1\leq\overline{S}_1$ in še:
 $$0\leq\underline{S}_0\leq\frac{\underline{S}_1}{10^2}
\leq\frac{\overline{S}_1}{10^2}\leq\overline{S}_0.$$

Če postopek nadaljujemo, dobimo zaporedje kvadratov $\Omega_n$, ki imajo stranico dolžine $\frac{1}{10^n}$, tlakovanja  $\mathcal{T}_n$ in pare števil $\underline{S}_n$ in $\overline{S}_n$, za katere velja:
$$0\leq\underline{S}_0\leq\frac{\underline{S}_1}{10^2}\leq \cdots \leq \frac{\underline{S}_n}{10^{2n}}\leq\cdots\leq \frac{\overline{S}_n}{10^{2n}}\leq\cdots
\leq\frac{\overline{S}_1}{10^2}\leq\overline{S}_0.$$

Zaporedje $\frac{\underline{S}_n}{10^{2n}}$ je naraščajoče in navzgor omejeno, zato je po znanem izreku matematične analize konvergentno in ima svojo limito:
$$\underline{S}=\lim_{n\rightarrow\infty}\frac{\underline{S}_n}{10^{2n}}.$$

Podobno je zaporedje $\frac{\overline{S}_n}{10^{2n}}$ padajoče in navzdol omejeno, zato je konvergentno in ima svojo limito:
$$\overline{S}=\lim_{n\rightarrow\infty}\frac{\overline{S}_n}{10^{2n}}.$$

Če je $\underline{S}=\overline{S}$, pravimo, da je lik $\Phi$ merljiv. Število $S=\underline{S}=\overline{S}$ pa je njegova \index{ploščina lika}\pojem{ploščina}. Označimo jo z $p(\Phi)$ oz. $p_{\Phi}$.

Intuitivno je jasno, da je ploščina lika  $\Phi$ ni odvisna od lege enotskega kvadrata - torej je odvisna le od sistema merjenja daljic, v katerem ima stranica enotskega kvadrata dolžino 1. Na tem mestu tega dejstva ne bomo formalno dokazovali.

Dokažimo prvo pomembno lastnost ploščine.



                \bizrek \label{ploscDaljice}
                The area of an arbitrary point is 0.
                The area of an arbitrary line segment is 0.
                \eizrek

\begin{figure}[!htb]
\centering
\input{sl.plo.8.1.2.pic}
\caption{} \label{sl.plo.8.1.2.pic}
\end{figure}

 \textbf{\textit{Proof.}}

\textit{1)} Naj bo $A$ poljubna točka. Jasno je, da za vsak $n\in \mathbb{N}$ velja: $\underline{S}_n=0$ in $0\leq\overline{S}_n\leq4$ (točka leži v največ štirih kvadratih tlakovanja $T_n$). Torej velja:
$$\underline{S}=\lim_{n\rightarrow\infty}\frac{\underline{S}_n}{10^{2n}}
=\lim_{n\rightarrow\infty}\frac{0}{10^{2n}}=0$$
in
$$0\leq\overline{S}=\lim_{n\rightarrow\infty}\frac{\overline{S}_n}{10^{2n}}\leq
\lim_{n\rightarrow\infty}\frac{4}{10^{2n}}=0,$$
zato je $p_A=S=\underline{S}=\overline{S}=0$

\textit{2)} Naj bo $AB$ poljubna daljica (Figure \ref{sl.plo.8.1.2.pic}) in $|AB|=d$. S $k$ označimo celi del števila $d$, oz. $k=[d]$. Izberimo enotski kvadrat $\Omega_0$, tako da je eno njegovo oglišče točka $A$, ena njegova stranica pa leži na daljici $AB$. V tem primeru je $\underline{S}_n=0$ in $\overline{S}_n\leq (k+1)\cdot 10^n$. Tedaj je:
$$\underline{S}=\lim_{n\rightarrow\infty}\frac{\underline{S}_n}{10^{2n}}
=\lim_{n\rightarrow\infty}\frac{0}{10^{2n}}=0$$
in
$$0\leq\overline{S}=\lim_{n\rightarrow\infty}\frac{\overline{S}_n}{10^{2n}}\leq
\lim_{n\rightarrow\infty}\frac{(k+1)10^n}{10^{2n}}=
\lim_{n\rightarrow\infty}\frac{k+1}{10^n}=0,$$
zato je $p_{AB}=S=\underline{S}=\overline{S}=0.$
\kdokaz

Naslednji izrek, ki se nanaša na osnovne lastnosti ploščine, bomo podali brez dokaza (glej \cite{Lucic}).

            \bizrek \label{ploscGlavniIzrek}
            Naj bo $p$ ploščina, definirana na množici $\mu$ merljivih likov
            ravnine.
             Potem velja:
            \begin{enumerate}
              \item $p(\Omega_0)=1$;
              \item $(\forall \Phi\in\mu)\hspace*{1mm}p(\Phi)\geq 0$;
              \item $(\forall \Phi_1,\Phi_2\in\mu)\hspace*{1mm}
            \left(\Phi_1\cong\Phi_2
            \hspace*{1mm}\Rightarrow\hspace*{1mm}p(\Phi_1)=p(\Phi_2)\right)$;
              \item $(\forall \Phi_1,\Phi_2\in\mu)\hspace*{1mm}
            \left(p(\Phi_1\cap\Phi_2)=0
            \hspace*{1mm}\Rightarrow\hspace*{1mm}p(\Phi_1\cup\Phi_2)=
            p(\Phi_1)+p(\Phi_2)\right)$.
            \end{enumerate}
            \eizrek

Zelo koristen je naslednji izrek.


                \bizrek \label{ploscLomljenke}
                Ploščina poljubne lomljenke je enaka 0.
                \eizrek


\begin{figure}[!htb]
\centering
\input{sl.plo.8.1.3.pic}
\caption{} \label{sl.plo.8.1.3.pic}
\end{figure}

 \textbf{\textit{Proof.}} (Figure \ref{sl.plo.8.1.3.pic})

Trditev je direktna posledica izrekov \ref{ploscDaljice} in \ref{ploscGlavniIzrek} (\textit{4}).
\kdokaz



%________________________________________________________________________________
 \poglavje{Area of Rectangles, Parallelogram, and Trapezoids} \label{odd8PloParalel}


 Na tem mestu bomo izpeljali formule za ploščino določenih štirikotnikov.

                \bizrek \label{ploscPravok}
                Če je $p$ ploščina pravokotnika s stranicama dolžin $a$ in $b$, potem
                je:
                $$p=ab.$$
                \eizrek




\begin{figure}[!htb]
\centering
\input{sl.plo.8.2.1a.pic}
\caption{} \label{sl.plo.8.2.1a.pic}
\end{figure}


 \textbf{\textit{Proof.}}
Naj bo $ABCD$ pravokotnik s stranicama $AB$ in $AD$ dolžin $|AB|=a$ in $|AD|=b$ (Figure \ref{sl.plo.8.2.1a.pic}).
Izberimo enotski kvadrat $\Omega_0$, tako da je eno njegovo oglišče točka $A$, ena njegova stranica pa leži na daljici $AB$. Naj bodo $\mathcal{T}_n$ ($n\in \mathbb{N}\cup\{0\}$) ustrezna tlakovanja, ki jih določa enotni kvadrat $\Omega_0$.

Označimo z $a_n=[a\cdot 10^n]$ in $b_n=[b\cdot 10^n]$  (kjer $[x]$ predstavlja celi del števila $x$). Najprej je:
 \begin{eqnarray} \label{eqnPloscPrav1a}
 \frac{a_n}{10^n}\leq a\leq \frac{a_n+1}{10^n} \hspace*{2mm} \textrm{ in }  \hspace*{2mm}
\frac{b_n}{10^n}\leq b\leq \frac{b_n+1}{10^n},
 \end{eqnarray}
nato pa še:
 \begin{eqnarray} \label{eqnPloscPrav2a}
 \underline{S}_n=\frac{a_n\cdot b_n}{10^{2n}} \hspace*{2mm} \textrm{ in }  \hspace*{2mm}
\overline{S}_n=\frac{\left(a_n+1\right)\cdot \left(b_n+1\right)}{10^{2n}}.
 \end{eqnarray}
Iz \ref{eqnPloscPrav2a} in \ref{eqnPloscPrav1a} dobimo:
\begin{eqnarray*}
0&\leq&\lim_{n\rightarrow\infty}\left(\overline{S}_n-\underline{S}_n\right)=\\
&=&\lim_{n\rightarrow\infty}\left(\frac{\left(a_n+1\right)\cdot \left(b_n+1\right)}{10^{2n}}-\frac{a_n\cdot b_n}{10^{2n}}\right)=\\
&=& \lim_{n\rightarrow\infty}\frac{a_n+ b_n+1}{10^{2n}}\leq\\
&\leq& \lim_{n\rightarrow\infty}\frac{a+b+1}{10^n}=\\
&=& 0.
 \end{eqnarray*}
Torej:
\begin{eqnarray*}
\lim_{n\rightarrow\infty}\left(\overline{S}_n-\underline{S}_n\right)=0,
 \end{eqnarray*}
zato je $\underline{S}=\lim_{n\rightarrow\infty}\underline{S}_n
=\lim_{n\rightarrow\infty}\overline{S}_n=\overline{S}$, kar pomeni, da je pravokotnik $ABCD$ merljiv lik s ploščino $p=S=\underline{S}=\overline{S}$.

Dokažimo še $p=ab$. Iz \ref{eqnPloscPrav1a} dobimo:
 \begin{eqnarray*}
 \frac{a_n\cdot b_n}{10^{2n}} \leq ab\leq \frac{\left(a_n+1\right)\cdot \left(b_n+1\right)}{10^{2n}},
 \end{eqnarray*}
Ker to velja za vsak $n\in \mathbb{N}$, sledi:
 \begin{eqnarray*}
 \hspace*{-2mm} \underline{S}=\lim_{n\rightarrow\infty}\underline{S}_n=
\lim_{n\rightarrow\infty}\frac{a_n\cdot b_n}{10^{2n}} \leq ab\leq \lim_{n\rightarrow\infty}\frac{\left(a_n+1\right)\cdot \left(b_n+1\right)}{10^{2n}}=\lim_{n\rightarrow\infty}\overline{S}_n=\overline{S}.
 \end{eqnarray*}
Iz $p=S=\underline{S}=\overline{S}$ na koncu sledi $p=ab$.
\kdokaz

Direktna posledica prejšnjega izreka \ref{ploscPravok} je naslednja trditev (Figure \ref{sl.plo.8.2.2.pic}).

                \bizrek \label{ploscKvadr}
                Če je $p_{\square}$ ploščina kvadrata s stranico dolžine $a$, potem je:
                $$p_{\square}=a^2.$$
                \eizrek



\begin{figure}[!htb]
\centering
\input{sl.plo.8.2.2.pic}
\caption{} \label{sl.plo.8.2.2.pic}
\end{figure}

Izpeljimo formule za ploščino še za nekatere štirikotnike.


                \bizrek \label{ploscParal}
                Če je $p$ ploščina paralelograma s stranicama dolžin $a$ in $b$ ter
                pripadajočima višinama dolžin $v_a$ in $v_b$, potem
                je:
                $$p=av_a=bv_b.$$
                \eizrek


 \textbf{\textit{Proof.}}
 Naj bo $ABCD$ paralelogram s stranico $AB$ dolžine  $|AB|=a$ in pripadajočo višino $v_a$. Označimo z $E$ in $F$ pravokotni projekciji oglišč $D$ in $C$ na premici $AB$. Štirikotnik $EFCD$ je pravokotnik s stranicama $|CD|=a$ in $|FC|=v_a$, zato je po izreku   \ref{ploscPravok}:
\begin{eqnarray}
p_{EFCD}=av_a. \label{eqnPloscPrav2}
\end{eqnarray}
Pravokotna trikotnika $AED$ in $BFC$ sta skladna (izrek \textit{ASA} \ref{KSK}) zato je po izreku \ref{ploscGlavniIzrek} \textit{3)}:
\begin{eqnarray}
p_{AED}=p_{BFC}. \label{eqnPloscPrav1}
\end{eqnarray}
Brez škode za splošnost predpostavimo, da je $\angle BAD\leq 90^0$. V tem primeru sta točki $B$ in $E$  na isti strani točke $A$ (v primeru $E=A$ gre za pravokotnik $ABCD$ in trditev sledi direktno iz izreka \ref{ploscPravok}). Obravnavali bomo več možnih primerov: $\mathcal{B}(A,E,B)$, $E=B$ in $\mathcal{B}(A,B,E)$.

\textit{1)} Naj bo $\mathcal{B}(A,E,B)$ (Figure \ref{sl.plo.8.2.3.pic}).

\begin{figure}[!htb]
\centering
\input{sl.plo.8.2.3.pic}
\caption{} \label{sl.plo.8.2.3.pic}
\end{figure}

Če uporabimo relaciji \ref{eqnPloscPrav2} in \ref{eqnPloscPrav1} ter izreka \ref{ploscGlavniIzrek} \textit{4)} in \ref{ploscDaljice}, dobimo:
 \begin{eqnarray*}
 p_{ABCD}=p_{AED}+p_{EBCD}=p_{BFC}+p_{EBCD}=p_{EFCD}=av_a.
\end{eqnarray*}

\textit{2)} V primeru $E=B$ (Figure \ref{sl.plo.8.2.3a.pic}) podobno iz relacij \ref{eqnPloscPrav1} in \ref{eqnPloscPrav1} ter izrekov \ref{ploscGlavniIzrek} \textit{4)} in \ref{ploscDaljice} dobimo:
 \begin{eqnarray*}
 p_{ABCD}=p_{AED}+p_{BCD}=p_{BFC}+p_{BCD}=p_{EFCD}=av_a.
\end{eqnarray*}


\begin{figure}[!htb]
\centering
\input{sl.plo.8.2.3a.pic}
\caption{} \label{sl.plo.8.2.3a.pic}
\end{figure}



\textit{3)} Predpostavimo, da velja $\mathcal{B}(A,B,E)$ (Figure \ref{sl.plo.8.2.3b.pic}).

\begin{figure}[!htb]
\centering
\input{sl.plo.8.2.3b.pic}
\caption{} \label{sl.plo.8.2.3b.pic}
\end{figure}

Daljici $BC$ in $DE$ se potem sekata v neki točki $L$. Po izrekih \ref{ploscGlavniIzrek} \textit{4)} in \ref{ploscDaljice} je:
$p_{AED}=p_{ABLD}+p_{BEL}$ in $p_{BFC}=p_{EFCL}+p_{BEL}$. Ker je po relaciji \ref{eqnPloscPrav1} $p_{AED}=p_{BFC}$, je tudi:
 \begin{eqnarray*}
 p_{ABLD}=p_{EFCL}.
\end{eqnarray*}
Iz tega in iz izrekov \ref{ploscGlavniIzrek} \textit{4)} in \ref{ploscDaljice} sledi:
 \begin{eqnarray*}
 p_{ABCD}=p_{ABLD}+p_{DLC}=p_{EFCL}+p_{DLC}=p_{EFCD}=av_a,
\end{eqnarray*}
 kar je bilo treba dokazati. \kdokaz

                \bizrek \label{ploscTrapez}
                Če je $p$ ploščina trapeza z osnovnicama dolžin $a$ in $c$ ter
                 višino dolžine $v$, potem
                je:
                $$p=\frac{a+c}{2}\cdot v.$$
                \eizrek


\begin{figure}[!htb]
\centering
\input{sl.plo.8.2.4.pic}
\caption{} \label{sl.plo.8.2.4.pic}
\end{figure}


 \textbf{\textit{Proof.}} Naj bo $ABCD$ trapez z osnovnicama $AB$ in $CD$ dolžin $|AB|=a$ in $|CD|=c$ ter višino dolžine $v$
 (Figure \ref{sl.plo.8.2.4.pic}).
 S $S$ označimo središče kraka $BC$ in $\mathcal{S}_S:\hspace*{1mm}A,D\mapsto A', D'$. Ker je $S$ skupno središče daljic $AA'$ in $DD'$, je po izreku \ref{paralelogram} $AD'A'D$ paralelogram. Ker je še $\mathcal{S}_S(B)=C$ in $\mathcal{B}(A,B,D')$, je tudi $\mathcal{B}(A',C,D)$. Izometrija $\mathcal{S}$ preslika trapez $ABCD$ v skladni trapez $A'CBD'$, zato je $|BD'|=|CD|=c$ oz. $|AD'|=a+c$, in po izreku \ref{ploscGlavniIzrek} \textit{3)} še $p_{ABCD}=p_{A'CBD'}$. Torej je paralelogram $AD'A'D$ z osnovnico $AD'$ dolžine $|AD'|=a+c$ in višino, ki je enaka višini trapeza $ABCD$ dolžine $v$, razdeljen na dva skladna trapeza $ABCD$ in $A'CBD'$ z enakima ploščinama.

Iz tega in izrekov \ref{ploscGlavniIzrek} \textit{4)}, \ref{ploscDaljice} in \ref{ploscParal} sledi:
 \begin{eqnarray*}
 2\cdot p_{ABCD}=p_{ABCD}+p_{A'CBD'}=p_{AD'A'D}=\left(a+c\right)v,
\end{eqnarray*}
oz. iskana relacija.
\kdokaz


Če uporabimo izrek \ref{srednjTrapez}, vidimo, da izraz $\frac{a+c}{2}$ predstavlja dolžino srednjice trapeza, zato za ploščino trapeza velja tudi formula:
 $$p=mv,$$
kjer je $m$ dolžina srednjice trapeza in $v$ njegova višina
  (Figure \ref{sl.plo.8.2.4a.pic}).


\begin{figure}[!htb]
\centering
\input{sl.plo.8.2.4a.pic}
\caption{} \label{sl.plo.8.2.4a.pic}
\end{figure}




                \bzgled \label{ploscStirikPravok}
                Če je $p$ ploščina štirikotnika s pravokotnima diagonalama dolžin $e$ in
                $f$, potem je:
                $$p=\frac{ef}{2}.$$
                \ezgled


\begin{figure}[!htb]
\centering
\input{sl.plo.8.2.5.pic}
\caption{} \label{sl.plo.8.2.5.pic}
\end{figure}


 \textbf{\textit{Proof.}} Naj bo $ABCD$ štirikotnik s pravokotnima diagonalama $AC$ in $BD$ dolžin $|AC|=e$ in $|BD|=f$ (Figure \ref{sl.plo.8.2.5.pic}). Vzporednice teh diagonal skozi oglišča štirikotnika $ABCD$ določajo pravokotnik $A'B'C'D'$ s stranicama dolžin $|A'B'|=f$ in $|B'C'|=e$. Iz skladnosti trikotnikov $ASD$ in $DA'A$ (izrek \textit{SAS} \ref{SKS}) sledi $p_{ASD}=p_{DA'A}$ (izrek \ref{ploscGlavniIzrek} \textit{3)}). Analogno je tudi $p_{ASB}=p_{BB'A}$, $p_{CSB}=p_{BC'C}$ in $p_{CSD}=p_{DD'C}$. Iz tega in iz izrekov \ref{ploscGlavniIzrek} \textit{4)}, \ref{ploscDaljice} in \ref{ploscPravok} sledi:
 \begin{eqnarray*}
 2\cdot p_{ABCD}&=&2\cdot\left(p_{ASD}+p_{ASB}+p_{CSB}+p_{CSD}\right)=\\
&=&2\cdot p_{ASD}+2\cdot p_{ASB}+2\cdot p_{CSB}+2\cdot p_{CSD}=\\
&=& p_{ASD}+ p_{DA'A}+ p_{ASB}+p_{BB'A}+\\
 && + p_{CSB}+p_{BC'C} +p_{CSD}+p_{DD'C}=\\
&=& p_{A'B'C'D'}= ef,
\end{eqnarray*}

oz. iskana relacija.
\kdokaz


Kot posledico imamo naslednji trditvi (Figure \ref{sl.plo.8.2.5a.pic}).


                \bzgled \label{ploscDeltoid}
                Če je $p$ ploščina deltoida z diagonalama dolžin $e$ in
                $f$, potem je:
                $$p=\frac{ef}{2}.$$
                \ezgled

 \textbf{\textit{Proof.}}

     Trditev je direktna posledica izreka \ref{ploscStirikPravok} in definicije deltoida.
\kdokaz


\begin{figure}[!htb]
\centering
\input{sl.plo.8.2.5a.pic}
\caption{} \label{sl.plo.8.2.5a.pic}
\end{figure}


                \bzgled \label{ploscRomb}
                Če je $p$ ploščina romba z diagonalama dolžin $e$ in
                $f$, potem je:
                $$p=\frac{ef}{2}.$$
                \ezgled

 \textbf{\textit{Proof.}}
    Trditev je direktna posledica izrekov \ref{ploscStirikPravok} in \ref{RombPravKvadr}.
\kdokaz



                \bzgled \label{ploscRomb1}
                Če so: $a$ dolžina stranice, $v$ višina ter $e$ in
                $f$ dolžini diagonal romba, potem je:
                $$av=\frac{ef}{2}.$$
                \ezgled

\begin{figure}[!htb]
\centering
\input{sl.plo.8.2.6.pic}
\caption{} \label{sl.plo.8.2.6.pic}
\end{figure}

 \textbf{\textit{Proof.}} (Figure \ref{sl.plo.8.2.5a.pic})

   Trditev je direktna posledica izrekov \ref{ploscRomb1} in \ref{ploscParal}.
\kdokaz


%________________________________________________________________________________
 \poglavje{Area of Triangles} \label{odd8PloTrik}

V tem razdelku bomo izpeljali več formul za ploščino trikotnika.

            \bizrek \label{PloscTrik} Če je $p_\triangle$ ploščina
           trikotnika $ABC$, $v_a$, $v_b$ in $v_c$ dolžine višin, ki ustrezajo stranicam $BC$, $AC$ in $AB$ z dolžinami $a$, $b$ in $c$, potem velja:
           $$p_\triangle=\frac{a\cdot v_a}{2}=\frac{b\cdot v_b}{2}
           =\frac{c\cdot v_c}{2}.$$
           \eizrek

\begin{figure}[!htb]
\centering
\input{sl.plo.8.3.1a.pic}
\caption{} \label{sl.plo.8.3.1a.pic}
\end{figure}

 \textbf{\textit{Proof.}}  Naj bo $D$ četrto oglišče paralelograma $ABCD$ (Figure \ref{sl.plo.8.3.1a.pic}). Trikotnika $ABC$ in $ADC$ sta skladna (izreka \ref{paralelogram} in \ref{SSS}), zato je $p_{ABC}=p_{ADC}$ (izrek \ref{ploscGlavniIzrek} \textit{3)}). Paralelogram $ABCD$ ima stranico $BC$ in pripadajočo višino dolžin $a$ in $v_a$, zato je po izreku \ref{ploscParal} $p_{ABCD}=av_a$. Če uporabimo še izreka \ref{ploscGlavniIzrek} \textit{4)} in \ref{ploscDaljice}, dobimo:
 \begin{eqnarray*}
 2\cdot p_{ABC}=p_{ABC}+p_{ADC}=p_{ABCD}=av_a,
\end{eqnarray*}
oz. iskano relacijo $p_\triangle=\frac{a\cdot v_a}{2}$. Analogno je tudi $p_\triangle=\frac{b\cdot v_b}{2}$ oz. $p_\triangle=\frac{c\cdot v_c}{2}$.
\kdokaz

          \bizrek \label{PloscTrikVcrt} Če je $p_\triangle$ ploščina
           trikotnika $ABC$ s
          polobsegom $s=\frac{a+b+c}{2}$  in polmerom včrtane krožnice
          $r$, potem velja:
          $$p_\triangle=sr.$$
          \eizrek

\begin{figure}[!htb]
\centering
\input{sl.plo.8.3.1.pic}
\caption{} \label{sl.plo.8.3.1.pic}
\end{figure}

\textbf{\textit{Solution.}}  Naj bo $S$ središče včrtane krožnice trikotnika $ABC$ (Figure \ref{sl.plo.8.3.1.pic}).

Po izrekih  \ref{ploscGlavniIzrek} \textit{4)}, \ref{ploscDaljice} in \ref{PloscTrik} je:
\begin{eqnarray*}
 p_{ABC}=p_{SBC}+p_{ASC}+p_{ABS}&=&
 \frac{a\cdot r}{2}+\frac{b\cdot r}{2}+\frac{c\cdot r}{2}=\\
&=&
 \frac{a+b+c}{2}\cdot r=sr,
\end{eqnarray*}
 kar je bilo treba dokazati. \kdokaz


        \bizrek \label{PloscTrikPricrt}  Če je $p_\triangle$
        ploščina trikotnika $ABC$ s
        polobsegom $s=\frac{a+b+c}{2}$ in polmerom pričrtane krožnice
        $r_a$, potem velja:
        $$p_\triangle=(s-a)r_a.$$
        \eizrek



 \textbf{\textit{Proof.}} Uporabimo oznake iz velike naloge \ref{velikaNaloga} (Figure \ref{sl.plo.8.3.2.pic}). Iz iste naloge je: $AR=s-a$ in $AR_a=s$.

Pravokotna trikotnika  $ARS$ in $AR_aS_a$ sta si podobna (izrek \ref{PodTrikKKK}), zato je $\frac{SR}{S_aR_a}=\frac{AR}{AR_a}$, oz. $\frac{r}{r_a}=\frac{s-a}{s}$. Če uporabimo še prejšnji izrek \ref{PloscTrikVcrt}, dobimo:
\begin{eqnarray*}
 p_{ABC}=sr=(s-a)r_a,
\end{eqnarray*}
 kar je bilo treba dokazati. \kdokaz

\begin{figure}[!htb]
\centering
\input{sl.plo.8.3.2.pic}
\caption{} \label{sl.plo.8.3.2.pic}
\end{figure}


        \bizrek \label{PloscTrikHeron}  Če je $p_\triangle$
        ploščina trikotnika $ABC$ s
        polobsegom $s=\frac{a+b+c}{2}$, potem velja
        \index{formula!Heronova}
        (Heronova\footnote{\index{Heron}
        \textit{Heron iz Aleksandije}(20--100),
        starogrški matemetik.} formula):
        $$p_\triangle=\sqrt{s(s-a)(s-b)(s-c)}.$$
        \eizrek


 \textbf{\textit{Proof.}} Tudi v tem primeru uporabimo oznake iz velike naloge \ref{velikaNaloga} (Figure \ref{sl.plo.8.3.2.pic}). Iz iste naloge je: $BP=s-b$ in $BP_a=s-c$.

Kota $SBP$ in $BS_aP_a$ sta skladna, ker imata paroma pravokotna kraka (izrek \ref{KotaPravokKraki}). Torej sta si pravokotna trikotnika $SBP$ in $BS_aP_a$ podobna  (izrek \ref{PodTrikKKK}), zato je $\frac{SP}{BP_a}=\frac{BP}{S_aP_a}$ oz. $\frac{r}{s-c}=\frac{s-b}{r_a}$ oz. $rr_a=(s-b)(s-c)$.
 Če uporabimo še izreka \ref{PloscTrikVcrt} in \ref{PloscTrikPricrt}, dobimo:
\begin{eqnarray*}
p_{ABC}^2=sr(s-a)r_a=s(s-a)rr_a=s(s-a)(s-b)(s-c),
\end{eqnarray*}
 kar je bilo treba dokazati. \kdokaz


        \bizrek \label{PloscTrikOcrt}  Če je $p_\triangle$
        ploščina trikotnika $ABC$ z
        dolžinami stranic $a$, $b$ in $c$ ter $R$ polmer očrtane krožnice,
         potem velja:
        $$p_\triangle=\frac{abc}{4R}.$$
        \eizrek


\begin{figure}[!htb]
\centering
\input{sl.plo.8.3.3.pic}
\caption{} \label{sl.plo.8.3.3.pic}
\end{figure}


 \textbf{\textit{Proof.}} (Figure \ref{sl.plo.8.3.3.pic})

 Trditev je direktna posledica izrekov \ref{PloscTrik} in \ref{izrekSinusni}:
\begin{eqnarray*}
p_{ABC}=\frac{av_a}{2}=\frac{a}{2}\cdot v_a=\frac{a}{2}\cdot \frac{bc}{2R}=\frac{abc}{4R},
\end{eqnarray*}
 kar je bilo treba dokazati. \kdokaz


Prejšnje formule za ploščino trikotnika lahko uporabimo za računanje polmerov očrtane, včrtane in pričrtanih krožnic trikotnika kot funkcije njegovih stranic.

        \bzgled \label{PloscTrikOcrtVcrt}  Če sta $R$ in $r$ polmera
         očrtane in včrtane krožnice, $r_a$, $r_b$ in $r_c$ polmeri pričrtanih krožnic ter
         $s=\frac{a+b+c}{2}$ polobseg
          trikotnika $ABC$,
         potem velja:

 (i) $R=\frac{abc}{4\sqrt{s(s-a)(s-b)(s-c)}}$,\hspace*{7mm}
 (ii) $r=\sqrt{\frac{(s-a)(s-b)(s-c)}{s}}$,\\
 (iii) $r_a=\sqrt{\frac{s(s-b)(s-c)}{s-a}}$,\hspace*{2.7mm}
 (iv) $r_b=\sqrt{\frac{(s-a)s(s-c)}{s-b}}$,\hspace*{2.7mm}
 (v) $r_c=\sqrt{\frac{(s-a)(s-b)s}{s-c}}$.

        \ezgled



 \textbf{\textit{Proof.}}
 Trditev je direktna posledica izrekov \ref{PloscTrikOcrt}, \ref{PloscTrikVcrt}, \ref{PloscTrikPricrt} in \ref{PloscTrikHeron}.
\kdokaz


        \bzgled \label{PloscTrikVcrtPricrt}  Če je $r$ polmer
          včrtane krožnice, $r_a$, $r_b$ in $r_c$ polmeri pričrtanih krožnic ter $p_{\triangle}$ ploščina trikotnika $ABC$,
         potem velja:
        $$p_{\triangle}=\sqrt{rr_ar_br_c}.$$
        \ezgled


 \textbf{\textit{Proof.}}
 Če uporabimo trditve iz prejšnjega zgleda \ref{PloscTrikOcrtVcrt}, dobimo:
 \begin{eqnarray*}
rr_ar_br_c=s(s-a)(s-b)(s-c)=p_{\triangle}^2,
\end{eqnarray*}
kar je bilo treba dokazati. \kdokaz

            \bizrek \label{ploscTrikPedalni}
            Naj bo $A'B'C'$ pedalni trikotnik ostrokotnega trikotnika $ABC$, $s'$ polobseg
             trikotnika $A'B'C'$, $R$ polmer očrtane krožnice in $p_{\triangle}$ ploščina
             trikotnika $ABC$, potem je:
             $$p_{\triangle} = s'R.$$
            \eizrek



\begin{figure}[!htb]
\centering
\input{sl.plo.8.3.3a.pic}
\caption{} \label{sl.plo.8.3.3a.pic}
\end{figure}


 \textbf{\textit{Proof.}} Označimo z $O$ središče očrtane krožnice trikotnika $ABC$ (Figure \ref{sl.plo.8.3.3a.pic}). Po izreku \ref{PedalniLemaOcrtana} je $OA\perp B'C'$, zato iz izreka \ref{ploscStirikPravok} sledi $p_{AC'OB'}=\frac{|AO|\cdot |B'C'|}{2}=\frac{R\cdot |B'C'|}{2}$. Analogno je tudi $p_{BA'OC'}=\frac{R\cdot |A'C'|}{2}$ in $p_{CB'OA'}=\frac{R\cdot |A'B'|}{2}$. Ker gre za ostrokotni trikotnik $ABC$, je točka $O$ v njegovi notranjosti (glej razdelek \ref{odd3ZnamTock}), zato po izrekih \ref{ploscGlavniIzrek} \textit{4)} in \ref{ploscDaljice} velja:
\begin{eqnarray*}
p_{\triangle}&=&p_{AC'OB'}+p_{BA'OC'}+p_{CB'OA'}=\\
&=&\frac{R}{2}\left(|B'C'|+|A'C'|+|A'B'| \right)=s'R,
\end{eqnarray*}
 kar je bilo treba dokazati. \kdokaz

                \bzgled \label{ploscTrikPedalni1}
                    Naj bosta $R$ in $r$ polmera očrtane in včrtane krožnice ostrokotnega
                    trikotnika $ABC$ ter $s$ in $s'$ polobsega tega trikotnika in njegovega
                    pedalniga trikotnika.
                    Dokaži, da velja:
                    $$\frac{R}{r}=\frac{s}{s'}.$$
                \ezgled

\textbf{\textit{Proof.}} Po izrekih \ref{PloscTrikVcrt} in \ref{ploscTrikPedalni} je:
\begin{eqnarray*}
p_{ABC}=sr=s'R.
\end{eqnarray*}
Iz tega pa dobimo iskano relacijo.
\kdokaz

V posebnem primeru dobimo formuli za ploščino pravokotnega in enakostraničnega trikotnika.

            \bzgled \label{PloscTrikPravokotni} Če je $p_\triangle$ ploščina
           pravokotnega trikotnika $ABC$ s katetama  dolžin $a$ in $b$, potem velja:
           $$p_\triangle=\frac{ab}{2}.$$
           \ezgled


\begin{figure}[!htb]
\centering
\input{sl.plo.8.3.4a.pic}
\caption{} \label{sl.plo.8.3.4a.pic}
\end{figure}


 \textbf{\textit{Proof.}} (Figure \ref{sl.plo.8.3.4a.pic})
       Trditev je direktna posledica izreka \ref{PloscTrik}, saj je v tem primeru $v_a=b$.
\kdokaz

            \bizrek \label{PloscTrikEnakostr} Če je $p_\triangle$ ploščina
           enakostraničnega trikotnika $ABC$ s stranico  dolžine $a$, potem velja:
           $$p_\triangle=\frac{a^2\sqrt{3}}{4}.$$
           \eizrek


\begin{figure}[!htb]
\centering
\input{sl.plo.8.3.4.pic}
\caption{} \label{sl.plo.8.3.4.pic}
\end{figure}


 \textbf{\textit{Proof.}} (Figure \ref{sl.plo.8.3.4.pic})
  Trditev je direktna posledica izrekov \ref{PloscTrik} in \ref{PitagorovEnakostr}:
\begin{eqnarray*}
p_{ABC}=\frac{av_a}{2}=\frac{a}{2}\cdot v_a=\frac{a}{2}\cdot \frac{a\sqrt{3}}{2}=\frac{a^2\sqrt{3}}{4},
\end{eqnarray*}
 kar je bilo treba dokazati. \kdokaz

 Nadaljevali bomo z uporabo formul za ploščino trikotnika.





                \bzgled \label{CarnotOcrtLema}
                Naj bo $P$ notranja točka trikotnika $ABC$ ter $x$, $y$ in $z$ razdalje te točke od njegovih stranic $BC$, $AC$ in $AB$ z dolžinami $a$, $b$ in $c$. Če je $r$ polmer včrtane krožnice tega trikotnika, potem je:
                $$xa+yb+zc=2\cdot p_{\triangle ABC}=r(a+b+c).$$
                \ezgled

\begin{figure}[!htb]
\centering
\input{sl.plo.8.3.6.pic}
\caption{} \label{sl.plo.8.3.6.pic}
\end{figure}

\textbf{\textit{Proof.}}   (Figure \ref{sl.plo.8.3.6.pic})

$x$, $y$ in $z$ so dolžine višin trikotnikov $PBC$, $APC$ in $ABP$, zato
po izrekih \ref{PloscTrik} in \ref{PloscTrikVcrt} sledi:
 \begin{eqnarray*}
 xa+yb+zc &=& 2\cdot p_{\triangle PBC}+2\cdot p_{\triangle APC}+2\cdot p_{\triangle ABP}\\
  &=& 2\cdot p_{\triangle ABC}\\
   &=& r(a+b+c),
 \end{eqnarray*}
 kar je bilo treba dokazati. \kdokaz


                \bzgled \index{izrek!Vivianijev} (Vivianijev\footnote{\index{Viviani, V.}\textit{V. Viviani} (1622--1703),
            italijanski matematik in fizik.} izrek)
                Naj bo $P$ notranja točka enakostraničnega trikotnika $ABC$ ter $x$, $y$ in $z$ razdalje te točke od njegovih stranic $BC$, $AC$ in $AB$. Če je $v$ višina tega trikotnika, potem je:
                $$x+y+z=v.$$
                \ezgled

\begin{figure}[!htb]
\centering
\input{sl.plo.8.3.6a.pic}
\caption{} \label{sl.plo.8.3.6a.pic}
\end{figure}

\textbf{\textit{Proof.}}
Po trditvi iz prejšnjega zgleda \ref{CarnotOcrtLema} in izreka \ref{PloscTrik}:
$xa+ya+za=2\cdot p_{\triangle ABC}=v_aa$.
\kdokaz

                \bizrek \label{CarnotOcrt}\index{izrek!Carnotov o očrtani krožnici}(Carnotov\footnote{\index{Carnot, L. N. M.}\textit{L. N. M. Carnot} (1753--1823), francoski matematik.} izrek o očrtani krožnici.)
                Vsota razdalj središča očrtane krožnice trikotnika od njegovih stranic je enaka vsoti polmerov očrtane in včrtane krožnice tega trikotnika. Če so torej $l(O,R)$ očrtana in $k(S,r)$ včrtana krožnica ter $A_1$, $B_1$ in $C_1$ središča stranic trikotnika $ABC$, velja:
                $$|OA_1|+|OB_1|+|OC_1|=R+r.$$

                \eizrek

\begin{figure}[!htb]
\centering
\input{sl.plo.8.3.7.pic}
\caption{} \label{sl.plo.8.3.7.pic}
\end{figure}

\textbf{\textit{Proof.}} Označimo $x=|OA_1|$, $y=|OB_1|$ in $z=|OC_1|$  (Figure \ref{sl.plo.8.3.7.pic}).  Po trditvi iz zgleda \ref{CarnotOcrtLema} je:
\begin{eqnarray} \label{eqnCarnotOcrt1}
xa+yb+zc=r(a+b+c)
\end{eqnarray}
 Iz skladnosti trikotnikov $BA_1O$ in $CA_1O$ (izrek \textit{SAS} \ref{SKS}) in izreka \ref{SredObodKot} sledi: $\angle BAC=\frac{1}{2}\angle BOC=\angle BOA_1$. To pomeni, da velja $\triangle OA_1B\sim \triangle AB'B \sim \triangle AC'C$ (izrek \ref{PodTrikKKK}), zato je:
 \begin{eqnarray*}
 \frac{OA_1}{AB'}=\frac{OB}{AB} \hspace*{2mm} \textrm{ in } \hspace*{2mm}
 \frac{OA_1}{AC'}=\frac{OB}{AC}
\end{eqnarray*}
oziroma:
\begin{eqnarray*}
 cx=R\cdot |AB'| \hspace*{2mm} \textrm{ in } \hspace*{2mm}
 bx=R\cdot |AC'|
\end{eqnarray*}
Po seštevanju zadnjih dveh enakosti dobimo:
\begin{eqnarray*}
 (b+c)x=R(|AB'|+|AC'|)
\end{eqnarray*}
in analogno še:
\begin{eqnarray*}
 (a+c)y=R\cdot (|BA'|+|BC'|)\\
 (a+b)z=R\cdot (|CA'|+|CB'|).
\end{eqnarray*}

Po seštevanju zadnjih treh relacij dobimo:
\begin{eqnarray} \label{eqnCarnotOcrt2}
(b+c)x+(a+c)y+(a+b)z=R(a+b+c).
\end{eqnarray}
Če na koncu seštejemo enakosti \ref{eqnCarnotOcrt1} in \ref{eqnCarnotOcrt2} in dobljeno enakost delimo z $a+b+c$, dobimo:
\begin{eqnarray*}
x+y+z=r+R,
\end{eqnarray*}
 kar je bilo treba dokazati. \kdokaz


            \bnaloga\footnote{12. IMO, Hungary - 1970, Problem 1.}
             Let $M$ be a point on the side $AB$ of triangle $ABC$. Let $r_1$, $r_2$ and $r$ be the radii
            of the inscribed circles of triangles $AMC$, $BMC$ and $ABC$. Let $q_1$, $q_2$ and $q$
            be the radii of the escribed circles of the same triangles that lie in the angle
            $ACB$. Prove that
            $$\frac{r_1}{q_1}\cdot\frac{r_2}{q_2}=\frac{r}{q}.$$
            \enaloga

\begin{figure}[!htb]
\centering
\input{sl.plo.8.3.IMO1.pic}
\caption{} \label{sl.plo.8.3.IMO1.pic}
\end{figure}

\textbf{\textit{Solution.}} Označimo najprej $a=|BC|$,  $b=|AC|$,
$c=|AB|$,  $x=|AM|$, $y=|BM|$ in $m=|CM|$  (Figure
\ref{sl.plo.8.3.IMO1.pic}).

 Če uporabimo izreka \ref{PloscTrikVcrt} in
 \ref{PloscTrikPricrt}, dobimo:
  \begin{eqnarray*}
  && p_{\triangle ABC}=\frac{a+b+x+y}{2}\cdot r=\frac{a+b-x-y}{2}\cdot q\\
  && p_{\triangle AMC}=\frac{b+m+x}{2}\cdot r_1=\frac{b+m-x}{2}\cdot q_1\\
  && p_{\triangle ABC}=\frac{a+m+y}{2}\cdot r_2=\frac{a+m-y}{2}\cdot q_2
  \end{eqnarray*}
Iz tega sledi:
  \begin{eqnarray*}
  \frac{r}{q}=\frac{a+b+x+y}{a+b-x-y},\hspace*{6mm}
   \frac{r_1}{q_1}=\frac{b+m+x}{b+m-x},\hspace*{6mm}
   \frac{r_2}{q_2}=\frac{a+m+y}{a+m-y}
  \end{eqnarray*}
Zato je relacija
$\frac{r_1}{q_1}\cdot\frac{r_2}{q_2}=\frac{r}{q}$, ki jo želimo
dokazati, ekvivalentna z:
$$(a+b+x+y)(a+m-y)(b+m-x)=(a+b-x-y)(a+m+y)(b+m+x),$$
slednja pa je po preoblikovanju ekvivalentna z:
$$(a+b)(a+m)x+(a+b)(b+m)y=c(a+m)(b+m)+(x+y)xy,$$
oziroma, če upoštevamo $x+y=c$ in dodatno preoblikujemo, z:
$$m^2=a^2\frac{x}{c}+b^2\frac{y}{c}-xy.$$
 Zadnja relacija velja, ker predstavlja Stewartov izrek
 \ref{StewartIzrek} za trikotnik $ABC$ in daljico $CM$.
 \kdokaz


%________________________________________________________________________________
 \poglavje{Area of Polygons} \label{odd8PloVeck}

Ploščino poljubnega večkotnika dobimo, če ga z njegovimi diagonalami razdelimo na unijo trikotnikov (Figure \ref{sl.plo.8.4.1.pic}). Po izrekih \ref{ploscGlavniIzrek} in \ref{ploscDaljice} je ploščina tega večkotnika enaka vsoti ploščin teh trikotnikov.



\begin{figure}[!htb]
\centering
\input{sl.plo.8.4.1.pic}
\caption{} \label{sl.plo.8.4.1.pic}
\end{figure}

Posebej za pravilni šestkotnik velja naslednja trditev.

            \bizrek
            Če je $p$ ploščina in $a$ dolžina stranice pravilnega šestkotnika, potem je:
            $$p=\frac{3\sqrt{3}\cdot a^2}{2}.$$
            \eizrek


\begin{figure}[!htb]
\centering
\input{sl.plo.8.4.2.pic}
\caption{} \label{sl.plo.8.4.2.pic}
\end{figure}

 \textbf{\textit{Proof.}} (Figure \ref{sl.plo.8.4.2.pic})

Trditev je direktna posledica izrekov \ref{ploscGlavniIzrek}, \ref{ploscDaljice} \ref{PloscTrikEnakostr} in dejstva, da pravilni šestkotnik lahko razdelimo na šest pravilnih trikotnikov (razdelek \ref{odd3PravilniVeck}).
\kdokaz

Za ploščino tangentnih večkotnikov velja podobna formula kot v primeru trikotnikov (izrek \ref{PloscTrikVcrt}).

            \bizrek \label{ploscTetVec}
            Če so: $s$ polobseg, $k(S,r)$ včrtana krožnica tangentnega večkotnika
             $A_1A_2\ldots A_n$ in $p$
            njegova ploščina, tedaj je $$p = sr.$$
            \eizrek


\begin{figure}[!htb]
\centering
\input{sl.plo.8.4.3.pic}
\caption{} \label{sl.plo.8.4.3.pic}
\end{figure}

 \textbf{\textit{Proof.}} (Figure \ref{sl.plo.8.4.3.pic})

Trikotniki $A_1SA_2$, $A_2SA_3$, ... , $A_{n-1}SA_n$ in $A_nSA_1$ imajo vsi enako dolžino višine iz oglišča $S$, ki je enaka $r$. Če uporabimo izreke \ref{ploscGlavniIzrek} \textit{4)}, \ref{ploscDaljice} in \ref{PloscTrik}, dobimo:
\begin{eqnarray*}
 p &=& p_{A_1SA_2}+p_{A_2SA_3}+\cdots +p_{A_{n-1}SA_n} +p_{A_nSA_1}=\\
 &=& \frac{|A_1A_2|\cdot r}{2}+\frac{|A_2A_3|\cdot r}{2}+\cdots+
\frac{|A_{n-1}A_n|\cdot r}{2}+\frac{|A_n A_1|\cdot r}{2}=\\
&=&sr,
\end{eqnarray*}
kar je bilo treba dokazati. \kdokaz



            \bnaloga\footnote{30. IMO, Germany  - 1989, Problem 2.}
             Let $S$, $S_a$, $S_b$ and $S_c$ be incentre and excentres of an acute-angled triangle $ABC$.
                $N_a$, $N_b$ and $N_c$ are midpoints of line segments $SS_a$, $SS_b$ and $SS_c$, respectively.
            Let $\mathcal{S}_{X_1X_2\ldots X_n}$ be the area of a polygon $X_1X_2\ldots X_n$.
            Prove that:
              $$\mathcal{S}_{S_aS_bS_c} =
                2\mathcal{S}_{AN_cBN_aCN_b} \geq 4\mathcal{S}_{ABC}.$$
            \enaloga

\begin{figure}[!htb]
\centering
\input{sl.plo.8.3.IMO2.pic}
\caption{} \label{sl.plo.8.3.IMO2.pic}
\end{figure}

\textbf{\textit{Solution.}}  Naj bo $l$ očrtana krožnica
trikotnika $ABC$ (Figure \ref{sl.plo.8.3.IMO2.pic}). Po veliki
nalogi (\ref{velikaNaloga}) je točka $N_a$ središče tistega loka
$BC$ krožnice $l$, na katerem točka $A$ ne leži - označimo ta lok
z $l_{BC}$. Podobno sta $N_b$ in $N_c$ središči ustreznih lokov
$AC$ in $AB$ na krožnici $l$.

 Najprej iz $SN_a\cong N_aS_a$ sledi
  $\mathcal{S}_{SCN_a}=\mathcal{S}_{N_aCS_a}$ oz.
  $\mathcal{S}_{SCS_a}=2\cdot\mathcal{S}_{SCN_a}$. Če uporabimo
  analogne enakosti in upoštevamo, da je $S$ notranja točka trikotnika
  $S_aS_bS_c$, dobimo:
 \begin{eqnarray*}
\mathcal{S}_{S_aS_bS_c} &=&
\mathcal{S}_{SCS_a}+\mathcal{S}_{SBS_a}+
\mathcal{S}_{SCS_b}+\mathcal{S}_{SAS_b}+
\mathcal{S}_{SAS_c}+\mathcal{S}_{SBS_c}=\\
 &=&
2\cdot\left(\mathcal{S}_{SCN_a}+\mathcal{S}_{SBN_a}+
\mathcal{S}_{SCN_b}+\mathcal{S}_{SAN_b}+
\mathcal{S}_{SAN_c}+\mathcal{S}_{SBN_c}\right)=\\
&=& 2\mathcal{S}_{AN_cBN_aCN_b}.
 \end{eqnarray*}

 Dokažimo še $2\mathcal{S}_{AN_cBN_aCN_b} \geq 4\mathcal{S}_{ABC}$,
  oz. $\mathcal{S}_{AN_cBN_aCN_b} \geq 2\mathcal{S}_{ABC}$.
  Naj bo $V$ višinska točka trikotnika $ABC$ in $V_a=S_{BC}(V)$.
  Trikotnik $ABC$ je po predpostavki ostrokotni, zato točka
  $V$ leži v njegovi notranjosti.
  Po izreku \ref{TockaV'} točka $V_a$ leži na loku $l_{BC}$. Omenili smo že,
  da je $N_a$ središče tega loka, zato je višina trikotnika
  $BN_aC$ daljša od višine trikotnika $BV_aC$ za isto osnovnico $BC$.
   Iz tega sledi $\mathcal{S}_{BN_aC}\geq \mathcal{S}_{BV_aC}$. Iz
    skladnosti trikotnikov $BV_aC$ in $BVC$ dobimo:
   $$\mathcal{S}_{BN_aC}\geq \mathcal{S}_{BV_aC}=\mathcal{S}_{BVC}.$$
 Podobno je tudi
  $\mathcal{S}_{AN_bC}\geq \mathcal{S}_{AVC}$ in
   $\mathcal{S}_{AN_cB}\geq \mathcal{S}_{AVB}$. Ker je $V$
   notranja točka trikotnika $ABC$, sledi:
 \begin{eqnarray*}
\mathcal{S}_{AN_cBN_aCN_b}&=&\mathcal{S}_{ABC}+
\mathcal{S}_{BN_aC}+\mathcal{S}_{AN_bC}+\mathcal{S}_{AN_cB}\geq\\
 &\geq& \mathcal{S}_{ABC}+
\mathcal{S}_{BVC}+\mathcal{S}_{AVC}+\mathcal{S}_{AVB}=\\
&=& 2\mathcal{S}_{ABC},
 \end{eqnarray*}
 kar je bilo treba dokazati.  \kdokaz



%________________________________________________________________________________
\poglavje{Koch Snowflake} \label{odd8PloKoch}

V tem razdelku bomo obravnavali lik, ki je omejen (podmnožica nekega kroga), ima končno ploščino, njegov obseg pa je neskončen.

Najprej bomo definirali posebno vrsto večkotnikov. Enakostranični trikotnik $ABC$ s stranico dolžine $a=a_0$ označimo s $\mathcal{K}_0(a)$.  Večkotnik $\mathcal{K}_1(a)$ dobimo, če vsako stranico trikotnika $\mathcal{K}_0(a)$ razdelimo na tri enake dele in nad srednjim narišemo enakostranični trikotnik z dolžino stranice $a_1=\frac{1}{3}\cdot a_0$. Postopek nadaljujemo in večkotnik $\mathcal{K}_n(a)$ dobimo, če vsako stranico trikotnika $\mathcal{K}_{n-1}(a)$ razdelimo na tri enake dele in nad srednjim narišemo enakostranični trikotnik z dolžino stranice $a_n=\frac{1}{3}\cdot a_{n-1}$ (Figure \ref{sl.plo.8.5.Koch1.pic}).


\begin{figure}[!htb]
\centering
\input{sl.plo.8.5.Koch1.pic}
\caption{} \label{sl.plo.8.5.Koch1.pic}
\end{figure}

 Če opisani postopek nadaljujemo do neskončnosti, dobimo t. i.
\index{Kochova snežinka} \pojem{Kochovo\footnote{Ta lik je leta 1904 definiral švedski matematik \index{Koch, H.}\textit{Helge von Koch} (1870--1924). Danes Kochovo snežinko uvršcamo
med t. i. \textit{fraktale}.} snežinko} $\mathcal{K}(a)$.

Izračunajmo najprej obseg večkotnika $\mathcal{K}_n(a)$.

Iz postopka, po katerem smo dobili večkotnik $\mathcal{K}_n(a)$, je jasno, da ima večkotnik $\mathcal{K}_n(a)$ $4$-krat več stranic kot  $\mathcal{K}_{n-1}(a)$ (vsaka stranica večkotnika $\mathcal{K}_{n-1}(a)$ je zamenjana s štirimi stranicami večkotnika $\mathcal{K}_n(a)$). Označimo s $s_n$ število stranic večkotnika $\mathcal{K}_n(a)$. Očitno je $s_n$ geometrijsko zaporedje s kvocientom $q=4$ in začetnim členom $s_0=3$, zato je:
\begin{eqnarray} \label{eqnKoch1}
s_n=3\cdot 4^n
\end{eqnarray}
Vsaka stranica $a_n$ večkotnika $\mathcal{K}_n(a)$ je po konstrukciji $3$-krat krajša od stranice $a_{n-1}$ večkotnika $\mathcal{K}_{n-1}(a)$. Tako imamo spet geometrijsko zaporedje $a_n$  s kvocientom $q=\frac{1}{3}$ in začetnim členom $a_0=a$, zato je:
\begin{eqnarray} \label{eqnKoch2}
a_n=\left( \frac{1}{3} \right)^n\cdot a
\end{eqnarray}
Ker za obseg $o_n$ večkotnika $\mathcal{K}_n(a)$ velja $o_n=s_n\cdot a_n$, iz \ref{eqnKoch1} in \ref{eqnKoch2} dobimo:
\begin{eqnarray} \label{eqnKoch3}
o_n=3\cdot \left( \frac{4}{3} \right)^n\cdot a
\end{eqnarray}
Če želimo dobiti obseg $o$ Kochove snežinke $\mathcal{K}(a)$ ga računamo kot $o=\lim_{n\rightarrow\infty}o_n$, toda iz \ref{eqnKoch3}  dobimo (ker je $\frac{4}{3}>1$):
\begin{eqnarray} \label{eqnKoch4}
o=\lim_{n\rightarrow\infty}o_n=\lim_{n\rightarrow\infty}3\cdot \left( \frac{4}{3} \right)^n\cdot a=3a\cdot\lim_{n\rightarrow\infty}\left( \frac{4}{3} \right)^n=\infty
\end{eqnarray}
kar pomeni, da Kochova snežinka $\mathcal{K}(a)$ nima končnega obsega, torej obseg $o_n$ večkotnika $\mathcal{K}_n(a)$ neomejeno raste, ko se število korakov $n$ povečuje.

Izračunajmo sedaj ploščino $p$ Kochove snežinke $\mathcal{K}(a)$. Označimo s $p_n$ ploščino večkotnika $\mathcal{K}_n(a)$.
 Ploščino $p_n$ dobimo, če ploščini $p_{n-1}$ dodamo določeno število ploščin manjših trikotnikov
stranice $a_n$. Koliko je teh trikotnikov? Po konstrukciji večkotnik $\mathcal{K}_n(a)$ dobimo iz večkotnika $\mathcal{K}_{n-1}(a)$, če pri vsaki stranici večkotnika $\mathcal{K}_{n-1}(a)$ pričrtamo manjši trikotnik. Torej je število teh trikotnikov enako $s_{n-1}$. Zaradi tega in izreka \ref{PloscTrikEnakostr} je:
\begin{eqnarray*}
p_n=p_{n-1}+s_{n-1}\cdot \frac{a_n^2\cdot \sqrt{3}}{4}=p_{n-1}
+\frac{3}{16}\cdot \left(\frac{4}{9} \right)^n\cdot a^2\sqrt{3}.
\end{eqnarray*}
Ploščina $p$ Kochove snežinke $\mathcal{K}(a)$ je torej neskončna vsota:
\begin{eqnarray*}
p&=&p_0+ \sum_{n=1}^{\infty}\frac{3}{16}\cdot \left(\frac{4}{9} \right)^n\cdot a^2\sqrt{3}=\\
&=&\frac{a^2\cdot \sqrt{3}}{4}+ \frac{3}{16} a^2\sqrt{3}\cdot\sum_{n=1}^{\infty} \left(\frac{4}{9}\right)^n=\\
&=&\frac{a^2\cdot \sqrt{3}}{4}+ \frac{3}{16} a^2\sqrt{3}\cdot\left(\frac{4}{9}+\left(\frac{4}{9}\right)^2+\cdots + \left(\frac{4}{9}\right)^n+\cdots \right).
\end{eqnarray*}
Če izračunamo vsoto neskončnega geometrijskega zaporedja in poenostavimo, dobimo:
\begin{eqnarray} \label{eqnKoch5}
p=\frac{2a^2\sqrt{3}}{5}.
\end{eqnarray}


%________________________________________________________________________________
\poglavje{Circumference and Area of a Circle} \label{odd8PloKrog}

%OBSEG

V razdelku \ref{odd3NeenTrik} smo že definirali obseg večkotnika kot vsoto vseh njegovih stranic. V tem smislu je obseg predstavljal daljico. V nadaljevanju bomo pod obsegom razumeli tudi dolžino te daljice oz. vsoto dolžin vseh stranic večkotnika. Na tem mestu bomo obravnavali obseg kroga. Čeprav je intuitivno jasen, je ta nov pojem potrebno najprej definirati.

Naj bo $k(S,r)$ poljubna krožnica. Pripadajoči krog označimo s $\mathcal{K}(S,r)$. Obseg tega kroga intuitivno predstavlja dolžino krožnice $k$. Naj bo $A_1A_2\ldots A_n$ ($n\in \mathbb{N}$ in $n\geq 3$) pravilni večkotnik, ki je včrtan krožnici $k$. Tangente te krožnice v ogliščih večkotnika $A_1A_2\ldots A_n$ določajo nosilke stranic pravilnega večkotnika $B_1B_2\ldots B_n$, ki je  krožnici $k$ očrtan (Figure \ref{sl.plo.8.5.1.pic}).


\begin{figure}[!htb]
\centering
\input{sl.plo.8.5.1.pic}
\caption{} \label{sl.plo.8.5.1.pic}
\end{figure}

Označimo z $\underline{o}_n$ obseg večkotnika  $A_1A_2\ldots A_n$ in $\overline{o}_n$ obseg večkotnika $B_1B_2\ldots B_n$. Po trikotniški neenakosti (izrek \ref{neenaktrik}) je $|A_1B_1|+|B_1A_2|>|A_1A_2|$, $|A_2B_2|+|B_2A_3|>|A_2A_3|$,..., $|A_nB_n|+|B_nA_1|>|A_nA_1|$ (Figure \ref{sl.plo.8.5.1a.pic}). Če seštejemo vse te neenakosti dobimo $\overline{o}_n>\underline{o}_n$. Podobno velja $\underline{o}_n<\underline{o}_{n+1}$ in $\overline{o}_n>\overline{o}_{n+1}$. Torej za vsak $n\in \mathbb{N}$, $n\geq 3$ velja:
\begin{eqnarray*}
\underline{o}_1<\underline{o}_2<\cdots<
\underline{o}_n<\overline{o}_n<\cdots<\overline{o}_2<\overline{o}_1.
\end{eqnarray*}
 To pomeni, da je $\underline{o}_n$ naraščajoče zaporedje navzgor omejeno z $\overline{o}_1$, zato je po znanem izreku matematične analize konvergentno in ima svojo limito:
\begin{eqnarray*}
\underline{o}=\lim_{n\rightarrow\infty}\underline{o}_n.
\end{eqnarray*}
 Podobno je $\overline{o}_n$ padajoče zaporedje navzdol omejeno z $\underline{o}_1$, zato  ima svojo limito:
\begin{eqnarray*}
\overline{o}=\lim_{n\rightarrow\infty}\overline{o}_n.
\end{eqnarray*}

\begin{figure}[!htb]
\centering
\input{sl.plo.8.5.1a.pic}
\caption{} \label{sl.plo.8.5.1a.pic}
\end{figure}

Na tem mestu ne bomo formalno dokazovali intuitivno jasnega dejstva, da je za dovolj veliko naravno število $n$ razlika $\overline{o}_n-\underline{o}_n$ poljubno majhna, oz. $\underline{o}=\overline{o}$. \index{obseg!kroga}\pojem{Obseg kroga} $\mathcal{K}$ z oznako $o$ potem predstavlja $o=\underline{o}=\overline{o}$.

Dokažimo naslednji pomemben izrek.

                \bizrek
                Če je $o$ obseg kroga s polmerom $r$, potem je razmerje
                $o:2r$
                 konstantno in tako neodvisno od izbire kroga.
                \eizrek

\begin{figure}[!htb]
\centering
\input{sl.plo.8.5.1b.pic}
\caption{} \label{sl.plo.8.5.1b.pic}
\end{figure}


\textbf{\textit{Proof.}} (Figure \ref{sl.plo.8.5.1b.pic})

Naj bosta $\mathcal{K}(S,r)$ in $\mathcal{K}'(S',r')$ poljubna kroga z obsegoma $o$ in $o'$ ter $k(S,r)$ in $k'(S',r')$ pripadajoči  krožnici. Po izreku \ref{RaztKroznKrozn1} obstaja središčni razteg, ki preslika krožnico $k$ v krožnico $k'$. Po izreku \ref{RaztTransPod} je ta središčni razteg tranformacija podobnosti (označimo jo s $f$) z nekim koeficientom $\lambda$. Najprej je $r':r=\lambda$ oziroma:
\begin{eqnarray}
 r'=\lambda r. \label{eqnPloscKrogrr'oo'}
\end{eqnarray}
Naj bosta $A_1A_2\ldots A_n$ in $B_1B_2\ldots B_n$ ($n\in \mathbb{N}$ in $n\geq 3$)  včrtani in očrtani pravilni večkotnik krožnice $k$ z obsegoma $\underline{o}_n$ in $\overline{o}_n$. Označimo:
$$f:S,A_1,A_2,\ldots,A_n,B_1,B_2,\ldots,B_n\mapsto S',A_1',A_2',\ldots,A_n',B_1',B_2',\ldots,B_n'.$$
Večkotnika $A_1'A_2'\ldots A_n'$ in $B_1'B_2'\ldots B_n'$ ($n\in \mathbb{N}$ in $n\geq 3$) sta včrtani in očrtani pravilni večkotnik krožnice $k$ in za njuna obsega $\underline{o'}_n$ in $\overline{o'}_n$ velja $\underline{o'}_n=\lambda\underline{o}_n$ in $\overline{o'}_n=\lambda\overline{o}_n$. Iz tega in iz definicije obsega kroga sledi:
\begin{eqnarray}\label{eqnPloscKrogrr'oo'1}
 o'=\underline{o'}=\lim_{n\rightarrow\infty}\underline{o'}_n=
\lim_{n\rightarrow\infty}\lambda\underline{o}_n=
\lambda\lim_{n\rightarrow\infty}\underline{o}_n=\lambda \underline{o}=\lambda o.
\end{eqnarray}
Iz \ref{eqnPloscKrogrr'oo'} in \ref{eqnPloscKrogrr'oo'1} sledi:
$$\frac{o'}{2r'}=\frac{\lambda o}{2\lambda r}=\frac{o}{2r},$$ kar je bilo treba dokazati. \kdokaz

Konstanto iz prejšnjega izreka imenujemo
\index{število!$\pi$}\pojem{število $\pi$} (\index{Arhimedova konstanta}\pojem{Arhimedova konstanta} oz. \index{število!Ludolfovo}\pojem{Ludolfovo število}) in ga tudi označimo s $\pi$.

Iz prejšnjega izreka in iz definicije števila $\pi$ dobimo naslednji izrek.


                \bizrek \label{obsegKtoznice}
                Če je $o$ obseg kroga s polmerom $r$, potem velja:
                $$o=2r\pi.$$
                \eizrek

najprej bomo podali prvo grobo oceno števila $\pi$.

                \bizrek \label{stevPiOcena}
                Za število $\pi$ velja:
                $$3<\pi<4.$$
                \eizrek


\begin{figure}[!htb]
\centering
\input{sl.plo.8.5.2.pic}
\caption{} \label{sl.plo.8.5.2.pic}
\end{figure}


\textbf{\textit{Proof.}} (Figure \ref{sl.plo.8.5.2.pic})

Naj bo $\mathcal{K}$ poljuben krog s polmerom $r$ in z obsegom $o$ ter $k$ pripadajoča krožnica. Po izreku \ref{obsegKtoznice} je $o=2r\pi$.
Izberimo za pravilni včrtani večkotnik $n=6$ in pravilni očrtani večkotnik $n=4$. Iz definicije obsega kroga je:
$$\underline{o}_6<o<\overline{o}_4,$$ torej:
$$6r<2r\pi<8r.$$
Iz tega po deljenju neenakosti z $2r$ sledi $3<\pi<4$.
\kdokaz

Z različnimi metodami lahko še bolj natančno določimo približno vrednost števila $\pi$\footnote{Sumerci (okoli leta 2000 pr. n. š.) niso našli boljšega približka za število $\pi$, kot je tudi poznejši biblijski $\pi \doteq 3$.\\
Starogrški matematik in filozof \index{Arhimed}\textit{Arhimed iz Sirakuze} (287--212 pr. n. š.) je okoli leta 230 pr. n. š. v svojem delu \textit{Merjenje kroga} našel približek $\pi \doteq 3,14185110664$ (torej na tri decimalke natančno) tako, da je za obseg kroga uporabil včrtane in očrtane pravilne $n$-kotnike za $n\in\{6, 12, 24, 48, 96\}$, oz. $n=3\cdot 2^k$ ($k\in\{1,2,3,4,5\}$). Po njem število $\pi$ imenujemo tudi \textit{Arhimedova konstanta}.
Francoski matematik  \index{Vi\'{e}te, F.} \textit{F. Vi\'{e}te}
(1540--1603) je izboljšal Arhimedove rezultate in leta 1579 z $n$-kotnikom s številom stranic $n = 393216 = 3\cdot 2^{17}$ izračunal število $\pi$ na 9 decimalk natančno.\\
Razvoj infinitezimalnega računa v 17. stoletju je našel nove postopke za še hitrejše izračunavanje  števila $\pi$. Tako je nizozemsko-nemški matematik \index{Ludolph van Ceulen} \textit{Ludolph van Ceulen} (1540--1610), ki je veliki del svojega življenja posvečal približnemu računanju števila $\pi$, izračunal 20 pravilnih decimalk. Ta rezultat je kasneje izboljšal na 35 decimalk. Po njem  število $\pi$ imenujemo tudi  \textit{Ludolfovo število}.
 Švicarski matematik \index{Euler, L.}\textit{L. Euler} (1707–-1783) je izračunal   128 decimalk.
 Slovenski matematik \index{Vega, J.}\textit{J. Vega} (1754–-1820) je dosegel tedanji svetovni rekord in izračunal število $\pi$ na 140 decimalk.\\
 Angleški matematik \index{Ferguson, D. F.}\textit{D. F. Ferguson} je leta 1946 z računalom izračunal 620 pravilnih decimalk. 2. avgusta 2010 pa je japonski sistemski inženir \index{Kondo, S.}\textit{S. Kondo} (1955-- ) s prirejenim osebnim računalnikom, ki ga je sestavil sam,  izračunal 5.000.000.000.000 decimalk števila $\pi$. Računanje skupaj s preverjanjem je trajalo 90 dni.
}.

Število $\pi$ je iracionalno število\footnote{To lastnost je leta 1761 dokazal francoski matematik \index{Lambert, J. H.}\textit{J.
H. Lambert} (1728--1777). Število $\pi$ je pravzaprav \index{število!transcendentno}transcendentno (ne obstaja polinom z racionalnimi koeficienti, katerega koren je $\pi$), kar je leta 1882 dokazal nemški matematik \index{Lindemann, C. L. F.}\textit{C. L. F. Lindemann} (1852–-1939).} ($\pi\notin \mathbb{Q}$), torej se ne da zapisati v obliki ulomka. V decimalni obliki ima neskončno decimalk brez periode:
$$\pi= 3,14159 26535 89793 23846 26433 83279 50288 41971 69399 37510 58\ldots$$
Njegova približna vrednost je torej:
$$\pi\doteq 3,14.$$

Poleg tega najbolj uporabljanega približka
 obstaja približek $\pi\doteq \frac{22}{7}$ (tudi na dve decimalki). Zelo dober približek (na šest decimalk) nam da ulomek $355/113$ (lahko si ga zapomnimo takole: zapišimo število 113355, števec določajo zadnje tri števke tega števila, imenovalec pa prve tri).

Točno vrednost števila $\pi$ v obliki neskončne vrste nam dasta
\index{formula!Leibnizova}\pojem{Leibnizova\footnote{\index{Leibniz, G. W.}\textit{G. W. Leibniz} (1646--1716), nemški matematik.} formula}:
$$\frac{\pi}{4}=\sum_{k=0}^{\infty}\frac{(-1)^k}{2k+1}$$
oziroma
$$\pi=4\cdot\left(1-\frac{1}{3}+\frac{1}{5}-\frac{1}{7}+\frac{1}{9}-\frac{1}{11}
+\cdots\right),$$
in
 \index{formula!Ramanujanova }\pojem{Ramanujanova\footnote{\index{Ramanujan, S.}\textit{S. Ramanujan}  (1887--1920), indijski matematik.} formula}:
$$\frac{1}{\pi}=
\frac{2\sqrt{2}}{9801}\sum_{k=0}^{\infty}
\frac{(4k)!\cdot(1103+26390k)}{(k!)\cdot396^{4k}}.$$

Na tem mestu ne bomo formalno vpeljali pojma dolžine krožnega loka. Ideja bi bila podobna kot pri definiranju obsega kroga. Brez dokaza bomo tudi sprejeli dejstvo, ki je intuitivno jasno, da sta dolžina krožnega loka $l$ in obseg $o$ pripadajočega kroga sorazmerni ustreznemu središčnemu kotu tega loka z mero $\alpha$ (v kotnih stopinjah) in polnem kotu, ki meri $360^0$ (Figure \ref{sl.plo.8.5.5a.pic}). Po izreku \ref{obsegKtoznice} torej velja:
$l:o=\alpha:360^0$ oz. $l:2r\pi=\alpha:360^0$. Iz tega sledi naslednja trditev.

                \bizrek \label{obsegKrozLok}
                Če je $l$ dolžina krožnega loka z ustreznim središčnim kotom z mero
                 (v kotnih stopinjah) $\alpha$ in
                polmerom $r$, potem je:
                    $$l=\frac{\pi r \alpha}{180^0}.$$
                \eizrek


\begin{figure}[!htb]
\centering
\input{sl.plo.8.5.5a.pic}
\caption{} \label{sl.plo.8.5.5a.pic}
\end{figure}




%PlOSCINA


V nadaljevanju bomo obravnavali ploščino kroga. Dokažimo najprej pomožno trditev.

                \bizrek \label{ploscKrogLema}
                Če sta $\Phi_1$ in $\Phi_2$ dva lika v isti ravnini, potem velja:
                $$\Phi_1\subseteq \Phi_2\hspace*{1mm}\Rightarrow\hspace*{1mm}
                 p_{\Phi_1}\leq  p_{\Phi_2}.$$
                \eizrek


\begin{figure}[!htb]
\centering
\input{sl.plo.8.5.3a.pic}
\caption{} \label{sl.plo.8.5.3a.pic}
\end{figure}


\textbf{\textit{Proof.}} (Figure \ref{sl.plo.8.5.3a.pic})

Ker je $\Phi_1\subseteq \Phi_2$, velja  $\Phi_2 = \Phi_1 \cup \left(\Phi_2\setminus \Phi_1\right)$ in $\Phi_1 \cap \left(\Phi_2\setminus \Phi_1\right)=\emptyset$ oz. $p_{\Phi_1 \cap \left(\Phi_2\setminus \Phi_1\right)}=0$. Po izreku \ref{ploscGlavniIzrek} \textit{4)} in \textit{2)} je:
\begin{eqnarray*}
p_{\Phi_2} =  p_{\Phi_1}+ p_{\Phi_2\setminus \Phi_1}  \geq p_{\Phi_1},
\end{eqnarray*}
 kar je bilo treba dokazati. \kdokaz

                \bizrek \label{ploscKrog}
                Če je $p_0$ ploščina kroga $\mathcal{K}$ s polmerom $r$, potem je
                $$p_0=r^2\pi.$$
                \eizrek


\begin{figure}[!htb]
\centering
\input{sl.plo.8.5.3.pic}
\caption{} \label{sl.plo.8.5.3.pic}
\end{figure}


\textbf{\textit{Proof.}} (Figure \ref{sl.plo.8.5.3.pic})

Označimo z $\underline{\mathcal{V}}_n=A_1A_2\ldots A_n$ pravilni včrtani ter $\overline{\mathcal{V}}_n=B_1B_2\ldots B_n$ pravilni očrtani $n$-kotnik pripadajoče krožnice $k$, $a_n$ in $b_n$ pa dolžine stranic teh dveh $n$-kotnikov.
Po prejšnjem izreku (\ref{ploscKrogLema}) je:
 \begin{eqnarray} \label{eqnPloscKrog1}
p_{\underline{\mathcal{V}}_n} \leq  p_0 \leq p_{\overline{\mathcal{V}}_n}.
\end{eqnarray}
 Naj bo $S$ središče kroga $\mathcal{K}$. Višina trikotnika $B_1SB_2$ iz oglišča $S$ je enaka polmeru $r$ kroga $\mathcal{K}$, z $v_n$ pa označimo višino trikotnika $A_1SA_2$ iz oglišča $S$.
 Po izrekih \ref{ploscGlavniIzrek} 3) in 4) ter \ref{PloscTrik} je:
\begin{eqnarray*}
p_{\underline{\mathcal{V}}_n} &=&n\cdot p_{A_1SA_2}=n\cdot \frac{a_nv_n}{2}=\frac{1}{2}na_nv_n=\frac{1}{2}\underline{o}_nv_n;\\
p_{\overline{\mathcal{V}}_n} &=&n\cdot p_{B_1SB_2}=n\cdot \frac{b_nr}{2}=\frac{1}{2}nb_nr=\frac{1}{2}\overline{o}_nr.
\end{eqnarray*}
Iz \ref{eqnPloscKrog1} in prejšnjih dveh relacij sledi:
 \begin{eqnarray} \label{eqnPloscKrog2}
\frac{1}{2}\underline{o}_nv_n \leq  p_0 \leq \frac{1}{2}\overline{o}_nr.
\end{eqnarray}
 Ker ta relacija velja za vsak $n\in \mathbb{N}$ in
 $$\lim_{n\rightarrow\infty} \underline{o}_n=\lim_{n\rightarrow\infty} \overline{o}_n=o,$$ kjer je $o$ obseg kroga $\mathcal{K}$, ter $$\lim_{n\rightarrow\infty} v_n=r,$$ iz \ref{eqnPloscKrog2} dobimo:
\begin{eqnarray*}
\frac{1}{2}or \leq  p_0 \leq \frac{1}{2}or,
\end{eqnarray*}
oz. (če uporabimo še izrek \ref{obsegKtoznice}):
\begin{eqnarray*}
  p_0 = \frac{1}{2}or=\frac{1}{2}2r\pi r=r^2\pi,
\end{eqnarray*}
 kar je bilo treba dokazati. \kdokaz

Na tem mestu bomo definirali nov pojem. Naj bosta $\mathcal{K}_1(S,r_1)$ in $\mathcal{K}_2(S,r_2)$ ($r_2>r_1$) dva kroga z istim središčem (pripadajoči krožnici $k_1$ in $k_2$ sta koncentrični). Množico $\mathcal{K}_2\setminus \mathcal{K}_1$ imenujemo
\index{krožni!kolobar}\pojem{krožni kolobar}.


                \bizrek \label{ploscKrozKolob}
                Če je $p_{kl}$ ploščina krožnega kolobarja, ki ga določata krožnici $k_1(S,r_1)$
                in  $k_2(S,r_2)$ ($r_2>r_1$), potem je:
                $$p_{kl}=\left( r_2^2-r_1^2 \right)\cdot\pi.$$
                \eizrek


\begin{figure}[!htb]
\centering
\input{sl.plo.8.5.4.pic}
\caption{} \label{sl.plo.8.5.4.pic}
\end{figure}


\textbf{\textit{Proof.}} (Figure \ref{sl.plo.8.5.4.pic})

Označimo s  $\mathcal{K}_1(S,r_1)$ in  $\mathcal{K}_2(S,r_2)$ pripadajoča kroga krožnic  $k_1(S,r_1)$ in  $k_2(S,r_2)$.

Iz $\mathcal{K}_2=\mathcal{K}_2\setminus \mathcal{K}_1\cup \mathcal{K}_1 $  in $\left( \mathcal{K}_2\setminus \mathcal{K}_1 \right) \cap \mathcal{K}_1=\emptyset$ po izrekih \ref{ploscGlavniIzrek} \textit{4)} in \ref{ploscKrog} sledi
 $p_{\mathcal{K}_2}=p_{\mathcal{K}_2\setminus \mathcal{K}_1} + p_{\mathcal{K}_1}$ oz.:
$$p_{kl}=p_{\mathcal{K}_2\setminus \mathcal{K}_1}= p_{\mathcal{K}_2}- p_{\mathcal{K}_1}=r_2^2\pi-r_1^2\pi
=\left( r_2^2-r_1^2 \right)\cdot\pi,$$ kar je bilo treba dokazati. \kdokaz

Na podoben način kot pri dolžini krožnega loka (izrek \ref{obsegKrozLok}) dobimo naslednjo trditev.


                \bizrek \label{ploscKrozIzsek}
                Če je $p_i$  ploščina krožnega izseka z ustreznim središčnim kotom z mero
               (v kotnih stopinjah) $\alpha$ in s
                polmerom $r$, potem je:
                    $$p_i=\frac{r^2\pi\cdot\alpha}{360^0}.$$
                \eizrek


\begin{figure}[!htb]
\centering
\input{sl.plo.8.5.5.pic}
\caption{} \label{sl.plo.8.5.5.pic}
\end{figure}



                \bzgled
                Če v formuli za ploščino kroga število $\pi$ zamenjamo z njegovo približno
                 vrednostjo $3$, dobimo formulo za ploščino v ustrezno krožnico včrtanega
                pravilnega dvanajstkotnika\footnote{To nalogo je rešil kitajski matematik
                \index{Liu, H.}\textit{H. Liu} (3. st.)}.
                \ezgled


\begin{figure}[!htb]
\centering
\input{sl.plo.8.5.6.pic}
\caption{} \label{sl.plo.8.5.6.pic}
\end{figure}


\textbf{\textit{Proof.}} (Figure \ref{sl.plo.8.5.6.pic})

Naj bo $A_1A_2\ldots A_{12}$ pravilni dvanajstkotnik, ki je
včrtan krožnici $k(S,r)$. Potem je $A_1A_3\ldots A_{11}$
pravilni šestkotnik, ki je včrtan isti krožnici. Označimo s $p_{12}$ ploščino
 omenjenega dvanajstkotnika. Po izreku \ref{ploscGlavniIzrek} \textit{4)} je $p_{12}=12\cdot p_{SA_1A_2}$.

Ker je še trikotnik $A_1SA_3$ pravilen, je višina $A_1P$
 trikotnika $SA_1A_2$ enaka polovici stranice $A_1A_3$
šestkotnika $A_1A_3\ldots A_{11}$, torej:
  $$|A_1P|=\frac{1}{2}\cdot |A_1A_3|=\frac{r}{2}.$$
Po izreku \ref{PloscTrik} je potem:
 $$p_{12}=12\cdot p_{SA_1A_2}=12\cdot\frac{|SA_2|\cdot |A_1P|}{2}=12\cdot\frac{r^2}{4}=3\cdot r^2,$$ kar je bilo treba dokazati. \kdokaz



                \bzgled \label{HipokratoviLuni}
                Nad katetama pravokotnega trikotnika konstruirajmo polkroga
                    in od tako povečanega trikotnika odrežimo polkrog nad hipotenuzo
                     (Figure \ref{sl.plo.8.5.7.pic}). Pokaži, da je ploščina ostanka
                    enaka ploščini prvotnega
                    pravokotnega trikotnika\footnote{Specialno trditev (naslednji zgled v tem razdelku), če je dani trikotnik enakokrak, je dokazal \index{Hipokrat}\textit{Hipokrat iz Kiosa} (5. st. pr. n. š.), starogrški
matematik. Omenjeni figuri imenujemo \index{Hipokratovi luni}\pojem{Hipokratovi luni}. Hipokrat je prvič odkril, da lahko nekatere like s krivim robom  z ravnilom in s šestilom  pretvorimo v ploščinsko enak kvadrat.}.
                \ezgled


\begin{figure}[!htb]
\centering
\input{sl.plo.8.5.7.pic}
\caption{} \label{sl.plo.8.5.7.pic}
\end{figure}

\textbf{\textit{Solution.}}
Naj bo $ABC$ omenjeni trikotnik s hipotenuzo $c$ ter katetama $a$ in $b$.
Označimo s $p$ iskano ploščino, $p_{\triangle}$ ploščino trikotnika $ABC$ ter $p_a$, $p_b$ in $p_c$ ploščine ustreznih polkrogov. Po izrekih \ref{ploscGlavniIzrek} \textit{4)}, \ref{ploscKrozIzsek} in \ref{PloscTrik} ter Pitagorovem izreku \ref{PitagorovIzrek} je:
 \begin{eqnarray*}
 p&=& p_a+p_b-p_c+p_{\triangle}=\\
    &=& \left(\frac{a}{2} \right)^2\cdot \pi
+\left(\frac{a}{2} \right)^2\cdot \pi-\left(\frac{a}{2} \right)^2\cdot \pi+p_{\triangle}=\\
       &=& \frac{\pi}{4}\cdot \left(a^2+b^2-c^2 \right)+p_{\triangle}=\\
   &=& \frac{\pi}{4}\cdot 0+p_{\triangle}=\\
   &=& p_{\triangle},
 \end{eqnarray*}
 kar je bilo treba dokazati. \kdokaz

                \bzgled \label{HipokratoviLuni2}
                Dan je kvadrat $PQRS$.
                Načrtaj kvadrat, ki ima ploščino enako ploščini lika, ki nastane, ko
                krogu, ki je očrtan kvadratu $PQRS$, odrežemo
                presek s krogom s
                polmerom $PQ$
                (Figure \ref{sl.plo.8.5.8.pic}).
                \ezgled


\begin{figure}[!htb]
\centering
\input{sl.plo.8.5.8.pic}
\caption{} \label{sl.plo.8.5.8.pic}
\end{figure}

\textbf{\textit{Solution.}}
Če izberemo enakokraki pravokotni trikotnik $QSQ'$ (kjer je $Q'=\mathcal{S}_p(Q)$), je po prejšnjem izreku (\ref{HipokratoviLuni}) ploščina iskanega lika enaka ploščini enakokrakega pravokotnega trikotnika $SPQ$ (Figure \ref{sl.plo.8.5.8.pic}), njegova ploščina
 pa je enaka ploščini kvadrata $PLSN$, kjer je $L$ središče daljice $SQ$ in $N=\mathcal{S}_{SP}(L)$
 (Figure \ref{sl.plo.8.5.8a.pic}).

\begin{figure}[!htb]
\centering
\input{sl.plo.8.5.8a.pic}
\caption{} \label{sl.plo.8.5.8a.pic}
\end{figure}
\kdokaz


        \bzgled
           Kvadrat $ABCD$  povečajmo za štiri polkroge nad stranicami, od
        dobljenega lika pa odrežimo kvadratu očrtan krog  (Figure \ref{sl.plo.8.5.9.pic}). Dokaži, da je
        ploščina ostanka enaka ploščini prvotnega kvadrata.
        \ezgled


\begin{figure}[!htb]
\centering
\input{sl.plo.8.5.9.pic}
\caption{} \label{sl.plo.8.5.9.pic}
\end{figure}

\textbf{\textit{Proof.}} Naj bo $S$ središče kvadrata $ABCD$.
Trditev je direktna posledica trditve iz dokaza prejšnjega izreka (\ref{HipokratoviLuni2}) - uporabimo ustrezno relacijo za trikotnike $ASB$, $BSC$, $CSD$ in $DSA$ ter vse seštejemo.
\kdokaz


            \bzgled \label{arbelos}
            Naj bo $C$ poljubna točka na premeru $AB$ polkrožnice $k$ ter $D$ presečišče te
            polkrožnice s pravokotnico premera $AB$ v točki $C$. Naj bosta $j$ in $l$
            polkrožnici s premeroma $AC$ in $BC$. Dokaži, da je ploščina lika, ki ga
            omejujejo polkrožnice $k$, $l$ in $j$, enaka ploščini kroga s premerom
            $CD$\footnote{To nalogo je v svoji ‘‘Knjigi lem’’ objavil starogrški matematik \index{Arhimed}\textit{Arhimed iz Sirakuze} (3. st. pr. n. š.).
Omenjeno figuro je imenoval \index{arbelos}\textit{arbelos} (kar v grščini pomeni čevljarski nož).}.
            \ezgled


\begin{figure}[!htb]
\centering
\input{sl.plo.8.5.10.pic}
\caption{} \label{sl.plo.8.5.10.pic}
\end{figure}

\textbf{\textit{Proof.}}  (Figure \ref{sl.plo.8.5.10.pic})

Označimo z $a=|AC|$ in $b=|BC|$ ustrezna premera, $p$  ploščino omenjenega lika in $p_0$ ploščino kroga s premerom $|CD|$. Po izreku \ref{izrekVisinski}
 je $|CD|^2 = ab$. Če uporabimo še izreka \ref{ploscKrozIzsek} in \ref{ploscKrog},
dobimo:
\begin{eqnarray*}
p&=&\frac{1}{2}\left(\frac{a+b}{2}\right)^2\cdot\pi-
\frac{1}{2}\left(\frac{a}{2}\right)^2\cdot\pi-
\frac{1}{2}\left(\frac{b}{2}\right)^2\cdot\pi=\\
&=& \frac{ab}{4}\cdot\pi=\\
&=& \left(\frac{|CD|}{2}\right)^2\cdot\pi=p_0,
\end{eqnarray*}
 kar je bilo treba dokazati. \kdokaz


                \bzgled
                Naj bo $k$ včrtana krožnica pravilnega trikotnika in $\mathcal{K}$ pripadajoči krog. Včrtajmo potem tri
                manjše krožnice, ki se dotikajo krožnice $k$ in dveh stranic tega
                trikotnika, nato še tri manjše krožnice, ki se dotikajo prejšnjih treh
                 krožnic in dveh stranic trikotnika, in tako naprej postopek nadaljujemo do
                 neskončnosti. Ali je ploščina vseh pripadajočih krogov, razen kroga $\mathcal{K}$,
                manjša
                od polovice ploščine kroga $\mathcal{K}$?
                \ezgled



\begin{figure}[!htb]
\centering
\input{sl.plo.8.5.11.pic}
\caption{} \label{sl.plo.8.5.11.pic}
\end{figure}

\textbf{\textit{Solution.}}  (Figure \ref{sl.plo.8.5.11.pic})

Naj bo $ABC$ dani trikotnik in $k(S,r)$ včrtana krožnica tega trikotnika. S
$k_i(S_i ,r_i)$ ($i\in\mathbb{N}$) označimo krožnice iz danega zaporedja krožnic pri oglišču $A$ ter s $p_i$ ($i \in \mathbb{N}$)
 ploščine pripadajočih krogov. Krožnico $k$ označimo še s $k_0(S_0 ,r_0)$ in ploščino pripadajočega kroga $\mathcal{K}$ s $p=p_0$. Naj bosta
$k_n(S_n ,r_n)$ in $k_{n+1}(S_{n+1} ,r_{n+1})$ zaporedni krožnici omenjenega zaporedja. Njuni dotikališči s stranico $AB$ trikotnika $ABC$
označimo s $T_n$ in $T_{n+1}$, pravokotno projekcijo središča $S_{n+1}$ na premico $S_nT_n$ pa z $L_n$. V
pravokotnem trikotniku $S_{n+1}L_nS_n$ je dolžina katete $|S_nL_n| = r_n- r_{n+1}$ in hipotenuze $|S_{n+1}S_n| = r_n+ r_{n+1}$. Kot
$L_nS_{n+1}S_n$ je enak polovici notranjega kota pri oglišču $A$, in sicer $30°$. Iz tega sledi $\angle S_{n+1}S_nL_n=60^0$. Naj bo $S_n'=\mathcal{S}_{S_{n+1}L_n}(S_n)$. Trikotnika $S_{n+1}L_nS_n$ in $S_{n+1}L_nS_n'$ sta skladna (izrek \textit{SAS} \ref{SKS}), zato je tudi $\angle S_{n+1}S_n'L_n=60^0$. Torej je $S_{n+1}S_n'S_n$ enakostranični trikotnik, kar pomeni, da velja $|S_{n+1}S_n| =|S_n'S_n| =2\cdot|S_nL_n|$ oz.
$$r_n+ r_{n+1}=2\cdot(r_n-r_{n+1}).$$
Če iz zadnje enakosti izrazimo $r_{n+1}$, dobimo:
 \begin{eqnarray*}
 r_{n+1}=\frac{1}{3}\cdot r_n.
\end{eqnarray*}
Iz tega pa po izreku \ref{ploscKrog} sledi:
 \begin{eqnarray*}
 p_{n+1}=\frac{1}{9}\cdot p_n.
\end{eqnarray*}
Zaporedje $p_n$ je torej geometrijsko zaporedje s koeficientom $q=\frac{1}{9}$ in začetno vrednostjo $p_0=p$, zato je:
 \begin{eqnarray*}
 p_n=\left( \frac{1}{9} \right)^n \cdot p.
\end{eqnarray*}
To pomeni, da skupna ploščina $p_A$ krožnic iz zaporedja pri oglišču $A$ predstavlja vsoto  neskončnega geometrijskega zaporedja $p_n$ ($n\in \mathbb{N}$). Torej:
\begin{eqnarray*}
 p_A&=&p_1+p_2+p_3+\cdots=\\
&=&\frac{1}{9}\cdot p+\left(\frac{1}{9}\right)^2\cdot p+
\left(\frac{1}{9}\right)^3\cdot p+\cdots=\\
&=&\frac{1}{9}p\cdot \frac{1}{1-\frac{1}{9}}=\\
&=& \frac{1}{8}\cdot p.
\end{eqnarray*}
Vsota ploščin vseh krožnic pri ogliščih $A$, $B$ in $C$ je potem enaka $\frac{3}{8}\cdot p$ in ni večja od polovice $p$.
\kdokaz



            \bnaloga\footnote{6. IMO, USSR - 1964, Problem 3.}
             A circle is inscribed in triangle $ABC$ with sides $a$, $b$, $c$. Tangents to the circle
            parallel to the sides of the triangle are constructed. Each of these tangents
            cuts off a triangle from triangle $ABC$. In each of these triangles, a circle is inscribed.
            Find the sum of the areas of all four inscribed circles (in terms of $a$, $b$, $c$).
            \enaloga


\begin{figure}[!htb]
\centering
\input{sl.plo.8.5.IMO1.pic}
\caption{} \label{sl.plo.8.5.IMO1.pic}
\end{figure}

\textbf{\textit{Solution.}} Naj bo $k(S,\rho)$ včrtana krožnica
trikotnika $ABC$ z višinami $v_a$, $v_b$ in $v_c$ in $p$ ploščina
ustreznega kroga. Označimo z $AB_aC_a$, $A_bBC_b$ in $A_cB_cC$
nove trikotnike, $k_a$, $k_b$ in $k_c$ včrtane krožnice teh
trikotnikov s polmeri $\rho_a$, $\rho_b$ in $\rho_c$ ter $p_a$,
$p_b$ in $p_c$ ploščine ustreznih krogov. (Figure
\ref{sl.plo.8.5.IMO1.pic}).

Trikotnika $ABC$ in $AB_aC_a$ sta si podobna s koeficientom
$\frac{v_a-2\rho}{v_a}$. Zato je:
 $$\rho_a=\frac{v_a-2\rho}{v_a}\cdot \rho.$$
 Ker po izreku \ref{PloscTrikVcrt} za ploščino trikotnika $ABC$
 velja $p_\triangle=s\cdot \varrho$
 (kjer je $s=\frac{a+b+c}{2}$ polobseg trikotnika $ABC$), sledi:
$$\rho_a^2=\left(1-\frac{2\rho}{v_a}\right)^2\cdot \rho^2=
 \left(1-\frac{2\frac{P_\triangle}{s}}{2\frac{P_\triangle}{a}}\right)^2\cdot \rho^2=
 \frac{(s-a)^2}{s^2}\cdot \rho^2.$$
Podobno dobimo $\rho_b^2=
 \frac{(s-b)^2}{s^2}\cdot \rho^2$ in $\rho_c^2=
 \frac{(s-c)^2}{s^2}\cdot \rho^2$. Iz trditve \ref{PloscTrikOcrtVcrt}
 sledi
  $\rho^2=\frac{(s-a)(s-b)(s-c)}{s}$, zato je:
\begin{eqnarray*}
 && p+p_a+p_b+p_c= \pi (\rho^2+\rho_a^2+\rho_b^2+\rho_c^2)=\\
 &&=\pi
 \rho^2(1+\frac{(s-a)^2}{s^2}+\frac{(s-b)^2}{s^2}+\frac{(s-c)^2}{s^2})=\\
 &&=
 \frac{(s-a)(s-b)(s-c)(s^2+(s-a)^2+(s-b)^2+(s-c)^2)}{s^3}\cdot \pi,
\end{eqnarray*}
 kar je bilo treba izraziti. \kdokaz


%________________________________________________________________________________
\poglavje{Pythagoras' Theorem and Area} \label{odd8PloPitagora}

 V razdelku \ref{odd7Pitagora} smo dokazali Pitagorov izrek \ref{PitagorovIzrek} in napovedali, da ga bomo izrazili v obliki, ki se nanaša na ploščine.

            \bizrek \index{izrek!Pitagorov za ploščine} \label{PitagorovIzrekPlosc}(Pitagorov izrek v obliki ploščin)\\
            Ploščina kvadrata nad hipotenuzo pravokotnega trikotnika je enaka vsoti ploščin kvadratov nad obema katetama tega trikotnika.

            \eizrek


\begin{figure}[!htb]
\centering
\input{sl.plo.8.6.1.pic}
\caption{} \label{sl.plo.8.6.1.pic}
\end{figure}

\textbf{\textit{Proof.}} Na tem mestu bomo podali dva dokaza in eno idejo dokaza tega izreka.

\textit{1)} Izrek je direktna posledica Pitagorovega izreka \ref{PitagorovIzrek} in formule za ploščino kvadrata \ref{ploscKvadr} (Figure
\ref{sl.plo.8.6.1.pic}).


\textit{2)\footnote{Ta dokaz je objavil \index{Evklid}\textit{Evklid iz Aleksandrije} (3. st. pr. n. š.) v svojem znamenitem delu ‘‘Elementi’’, ki je sestavljeno iz 13 knjig.}}
 Naj bodo $AMNB$, $BPQC$ in $CKLA$ kvadrati, ki so po vrsti konstruirani nad hipotenuzo in
katetama pravokotnega trikotnika $ABC$. Naj bosta $D$ in $E$ projekciji točke $C$ na premicah $AB$ in $MN$ (Figure
\ref{sl.plo.8.6.1.pic}).

Ker je ploščina kvadrata $AMNB$ enaka vsoti ploščin pravokotnikov $NBDE$ in $EDAM$, je dovolj  dokazati, da sta ploščini kvadratov $BPQC$ in
$CKLA$ enaki ploščinam pravokotnikov $NBDE$ in $EDAM$. Dokazali bomo le prvo enakost $p_{BPQC}=p_{NBDE}$ (druga
enakost $p_{CKLA}=p_{EDAM}$ je potem  analogna).

Za to relacijo $p_{BPQC}=p_{NBDE}$ je dovolj dokazati enakost ploščin trikotnikov $BPQ$ in
$NBD$, saj je $p_{BPQC}=2\cdot p_{BPQ}$ in $p_{NBDE}=2\cdot p_{NBD}$.
 Tedaj:
 \begin{eqnarray*}
   p_{BPQ}&=&p_{BPA}=\hspace*{3mm} \textrm{(trikotnika imata enako osnovnico in višino)}\\
   &=&p_{NBC}=\hspace*{3mm} \textrm{(trikotnika sta skladna zaradi } \mathcal{R}_{B,90^0+\angle CBA}\textrm{)}\\
   &=&p_{NBD}\hspace*{4mm} \textrm{(trikotnika imata enako osnovnico in višino)}
 \end{eqnarray*}
Torej imata trikotnika $BPQ$ in $NBD$ enaki ploščini, kar je bilo dovolj dokazati.


\begin{figure}[!htb]
\centering
\input{sl.plo.8.6.2.pic}
\caption{} \label{sl.plo.8.6.2.pic}
\end{figure}

\textit{3)\footnote{Ta dokaz (na podlagi ponazoritve) pripisujejo Starim Indijcem. Lahko rečemo, da mu zadostuje le komentar: ‘‘Poglej!’’ - tako je preprost in eleganten.}} Ideja dokaza je podana s sliko
\ref{sl.plo.8.6.2.pic}. Formalni dokaz bomo prepustili bralcu.
\kdokaz

            \bzgled
            Ploščina krožnega kolobarja, ki ga določata očrtana in včrtana krožnica pravilnega $n$-kotnika s stranico dolžine $a$ ni odvisna od števila $n$.
            \ezgled


\begin{figure}[!htb]
\centering
\input{sl.plo.8.6.3.pic}
\caption{} \label{sl.plo.8.6.3.pic}
\end{figure}

\textbf{\textit{Proof.}} Naj bo $AB$ ena stranica, $S$ središče te stranice in $O$ središče očrtane oz. včrtane krožnice poljubnega pravilnega $n$-kotnika $\mathcal{V}_n(a)$ (Figure \ref{sl.plo.8.6.3.pic}). Po predpostavki je $|AB|=a$, zato je $|AS|=\frac{a}{2}$. Označimo z $R$ in $r$ polmera očrtane in včrtane krožnice $n$-kotnika $\mathcal{V}_n(a)$. Iz skladnosti trikotnikov $OSA$ in $OSB$ (izrek \textit{SSS} \ref{SSS}) sledi $\angle OSA\cong \angle OSB=90^0$. Torej $OSA$ je pravokotni trikotnik s hipotenuzo $OA$, zato je po Pitagorovem izreku \ref{PitagorovIzrek} $|OA|^2-|OS|^2=|AS|^2$ oz. $R^2-r^2=\left(\frac{a}{2}\right)^2$. Iz te relacije in formule za ploščino krožnega kolobarja \ref{ploscKrozKolob} dobimo:
 $$p_k=\left(R^2-r^2\right)\cdot \pi=\left(\frac{a}{2}\right)^2\cdot \pi=\frac{a^2\pi}{4},$$
kar pomeni, da je ploščina kolobarja odvisna le od dolžine stranice pravilnega $n$-kotnika $\mathcal{V}_n(a)$.
\kdokaz

            \bzgled
            S pomočjo trditve iz zgleda \ref{PitagoraCofman} dokaži neenakost:
            \begin{eqnarray*}
            \frac{1}{1^2\cdot 2^2}+\frac{1}{2^2\cdot 3^2}+\cdots +\frac{1}{n^2 \left(n+1\right)^2}+\cdots<\frac{2}{3}
            \end{eqnarray*}
             oziroma
            \begin{eqnarray*}
                \sum_{n=1}^{\infty}\frac{1}{n^2 \left(n+1\right)^2}<\frac{2}{3}.
             \end{eqnarray*}
            \ezgled


\begin{figure}[!htb]
\centering
\input{sl.plo.8.6.4.pic}
\caption{} \label{sl.plo.8.6.4.pic}
\end{figure}

\textbf{\textit{Proof.}}  (Figure \ref{sl.plo.8.6.4.pic})

Uporabimo oznake iz zgleda \ref{PitagoraCofman}. Po relaciji \ref{eqnPitagoraCofman3} iz tega zgleda je $d_n=\frac{1}{n(n+1)}$, zato je $r_n=\frac{1}{2n(n+1)}$. Če s $p_n$ označimo ploščino pripadajočega kroga krožnice $c_n$, je (izrek \ref{ploscKrog}):
\begin{eqnarray*}
p_n=\pi r_n^2=\frac{\pi}{4}\frac{1}{n^2\left(n+1\right)^2}.
\end{eqnarray*}
Torej je vsota ploščin vseh pripadajočih krogov enaka:
\begin{eqnarray} \label{eqnPitagoraCofmanPlo2}
\sum_{n=1}^{\infty} p_n=\frac{\pi}{4}\sum_{n=1}^{\infty}\frac{1}{n^2\left(n+1\right)^2}.
\end{eqnarray}
Toda ta vsota je manjša od ploščine $p$ lika, ki ga omejujejo daljica
$A'B'$ ter krožna loka $A'P$ in $B'P$ krožnic $a$ in $b$ s središčnim kotom $90^0$. Ploščina $p$ je enaka razliki ploščine pravokotnika $A'ABB'$ in ploščine dveh ustreznih krožnih izsekov s središčnim kotom $90^0$. Torej (izreka \ref{ploscPravok} in \ref{ploscKrozIzsek}):
   \begin{eqnarray} \label{eqnPitagoraCofmanPlo3}
p=2-2\cdot\frac{\pi}{4}=2-\frac{\pi}{2}.
\end{eqnarray}
Ker je $\sum_{n=1}^{\infty} p_n<p$, iz \ref{eqnPitagoraCofmanPlo2} in \ref{eqnPitagoraCofmanPlo3} dobimo:
\begin{eqnarray*}
\frac{\pi}{4}\sum_{n=1}^{\infty}\frac{1}{n^2\left(n+1\right)^2}<2-\frac{\pi}{2}.
\end{eqnarray*}
Če uporabimo še dejstvo, da je $\pi>3$ (izrek \ref{stevPiOcena}), dobimo:
\begin{eqnarray*}
\sum_{n=1}^{\infty}\frac{1}{n^2\left(n+1\right)^2}<\frac{8}{\pi}-2<\frac{8}{3}-2=\frac{2}{3},
\end{eqnarray*}
 kar je bilo treba dokazati. \kdokaz



%________________________________________________________________________________
\naloge{Exercises}


\begin{enumerate}

% Ttrikotniki

\item Naj bo $T$ težišče trikotnika $ABC$. Dokaži:
   $$p_{TBC}=p_{ATC}=p_{ABT}.$$

\item Naj bodo $c$ hipotenuza, $v_c$ pripadajoča višina ter $a$ in $b$ kateti pravokotnega trikotnika. Dokaži, da velja $c+v_c>a+b$.
% zvezek - dodatni MG


  \item Načrtaj kvadrat, ki ima enako ploščino kot pravokotni trikotnik s katetama, ki sta skladni z danima daljicama $a$ in $b$. % (Hipokratovi luni)

\item Naj bo $ABC$ enakokraki pravokotni trikotnik z dolžino katet $|CB|=|CA|=a$. Označimo z $A_1$, $B_1$, $C_1$ in $C_2$ točke, za katere velja: $\overrightarrow{CA_1}=\frac{n-1}{n}\cdot \overrightarrow{CA}$, $\overrightarrow{CB_1}=\frac{n-1}{n}\cdot \overrightarrow{CB}$, $\overrightarrow{AC_1}=\frac{n-1}{n}\cdot \overrightarrow{AB}$ in  $\overrightarrow{AC_2}=\frac{n-2}{n}\cdot \overrightarrow{AB}$ za nek $n\in \mathbb{N}$. Izrazi ploščino štirikotnika, ki ga določajo premice $AB$, $A_1B_1$, $CC_1$ in $CC_2$, kot funkcijo $a$ in $n$.

\item Dan je pravokotni trikotnik $ABC$ s hipotenuzo $AB$ in ploščino $p$. Naj bo $C'=\mathcal{S}_{AB}(C)$,  $B'=\mathcal{S}_{AC}(B)$ in  $A'=\mathcal{S}_{BC}(A)$. Izrazi ploščino trikotnika $A'B'C'$ kot funkcijo $p$.

\item Naj bosta $R$ in $Q$ točki, v katerih se trikotniku $ABC$ včrtana krožnica dotika njegovih stranic $AB$ in $AC$. Naj simetrala notranjega kota $ABC$ seka premico $QR$ v točki $L$. Določi razmerje ploščin trikotnikov $ABC$ in $ABL$.
    % pripremni zadaci - naloga 193


\item Določi točko v notranjosti trikotnika $ABC$, za katero je produkt njenih razdalj od stranic tega trikotnika maksimalen. % zvezek - dodatni MG

\item   Trikotniku
             $ABC$ s stranicami dolžin $a$, $b$ in $c$ je včrtana krožnica. Načrtane so
             tangente te krožnice, ki so vzporedne s stranicami
             trikotnika.
             Vsaka od tangent v notranjosti trikotnika določa ustrezne daljice dolžin $a_1$, $b_1$ in $c_1$. Dokaži, da velja:
 $$\frac{a_1}{a}+\frac{b_1}{b}+\frac{b_1}{b}=1.$$ % zvezek - dodatni MG

\item Naj bosta $T$ in $S$ težišče in središče včrtane krožnice trikotnika $ABC$. Naj bo tudi $|AB|+|AC|=2\cdot |BC|$. Dokaži, da je $ST\parallel BC$. % zvezek - dodatni MG

\item Naj bo $p$ ploščina trikotnika $ABC$, $R$ polmer očrtane krožnice in $s'$ obseg pedalnega trikotnika. Dokaži, da velja $p=Rs'$. %Lopandic - nal 918

% Sstirikotniki


\item Naj bo $L$ poljubna točka v notranjosti paralelograma $ABCD$. Dokaži, da velja:
        $$p_{LAB}+p_{LCD}=p_{LBC}+p_{LAD}.$$ %Lopandic - nal 890

\item Naj bo $P$ središče kraka $BC$ trapeza $ABCD$. Dokaži:
 $$p_{APD}=\frac{1}{2}\cdot p_{ABCD}.$$

\item Naj bodo $o$ obseg, $v$ višina in $p$ ploščina tangentnega trapeza. Dokaži, da velja: $p=\frac{o\cdot v}{4}$.

\item Načrtaj premico $p$, ki poteka skozi oglišče $D$ trapeza $ABCD$ ($AB>CD$), tako da razdeli ta trapez na dva ploščinsko enaka lika.


\item Načrtaj premici $p$ in $q$, ki potekata skozi oglišče $D$ kvadrata $ABCD$ in ga razdelita na ploščinsko enake like.

\item Naj bo $ABCD$ kvadrat, $E$ središče njegove stranice $BC$ in $F$ točka, za katero je $\overrightarrow{AF}=\frac{1}{3}\cdot \overrightarrow{AB}$. Točka $G$ je četrto oglišče pravokotnika $FBEG$. Kolikšen del ploščine kvadrata $ABCD$ predstavlja ploščina trikotnika $BDG$?

\item Naj bosta $\mathcal{V}$ in $\mathcal{V}'$ podobna večkotnika s koeficientom podobnosti $k$. Potem je:
 $$p_{\mathcal{V}'}=k^2\cdot p_{\mathcal{V}}.$$ Dokaži.

\item Naj bodo $a$, $b$, $c$ in $d$ dolžine stranic, $s$ polobseg in $p$ ploščina poljubnega štirikotnika. Dokaži, da velja:
        $$p=\sqrt{(s-a)(s-b)(s-c)(s-d)}.$$ %Lopandic - nal 924

\item Naj bodo $a$, $b$, $c$ in $d$ dolžine stranic in $p$ ploščina tetivnotangentnega štirikotnika. Dokaži, da velja:
        $$p=\sqrt{abcd}.$$ %Lopandic - nal 925

% Veckotniki

\item Dan je pravokotnik $ABCD$ s stranicama dolžin $a=|AB|$ in $b=|BC|$. Izračunaj ploščino lika, ki predstavlja unijo pravokotnika $ABCD$ in njegove slike pri zrcaljenju čez premico $AC$.

\item Naj bo $ABCDEF$ pravilni šestkotnik, točki $P$ in $Q$ pa središči njegovih stranic $BC$ in $FA$. Kolikšen del ploščine tega šestkotnika predstavlja ploščina trikotnika $PQD$?


% KKrog

\item V kvadratu so včrtani štirje skladni krogi, tako da se vsak krog dotika dveh stranic in dveh krogov. Dokaži, da je vsota ploščin teh krogov enaka ploščini temu kvadratu včrtanega kroga.

\item Izračunaj ploščino kroga, ki je včrtan trikotniku s stranicami dolžin 9, 12 in 15. %resitev 54

\item Naj bo $P$ središče osnovnice $AB$ trapeza $ABCD$, za katerega velja $|BC|=|CD|=|AD|=\frac{1}{2}\cdot |AB|=a$. Izrazi ploščino lika, ki ga določajo osnovnica $CD$ ter krajša krožna loka $PD$ in $PC$ krožnic s središčema $A$ in $B$, kot funkcijo osnovnice $a$.

\item Tetiva $PQ$ ($|PQ|=d$) krožnice $k$ se dotika njene konciklične krožnice $k'$. Izrazi ploščino kolobarja, ki ga določata krožnici $k$ in $k'$, kot funkcijo tetive $d$.

\item Naj bo $r$ polmer včrtanega kroga tetivnega večkotnika $\mathcal{V}$, ki je razdeljen na  trikotnike $\triangle_1,\triangle_2,\ldots,\triangle_n$, tako da nobena dva trikotnika nimata skupnih notranjih točk.  Naj bodo $r_1,r_2,\ldots , r_n$ polmeri včrtanih krožnic teh trikotnikov. Dokaži, da je:
     $$\sum_{i=1}^n r_i\geq r.$$

\end{enumerate}


% DEL 3 - - - - - - - - - - - - - - - - - - - - - - - - - - - - - - - - - - - - - - -
%________________________________________________________________________________
% INVERZIJA
%________________________________________________________________________________


  \del{Inversion} \label{pogINV}
\poglavje{Definition and Basic Properties of Inversion}
\label{odd9DefInv}

V tem poglavju bomo definirali inverzijo - novo preslikavo, ki na
določen način predstavlja ‘‘zrcaljenje’’ čez krožnico.
Pričakujemo, da ima ta nova preslikava podobne lastnosti kot
zrcaljenje čez premico. Jasno je, da inverzija ne more biti izometrija,
toda kljub temu bomo navedli naslednje željene skupne lastnosti ‘‘dveh
zrcaljenj’’ (Figure \ref{sl.inv.9.1.1.pic}):

\begin{figure}[!htb]
\centering
\input{sl.inv.9.1.1.pic}
\caption{} \label{sl.inv.9.1.1.pic}
\end{figure}

\begin{itemize}
  \item zrcaljenje je bijektivna preslikava,
  \item zrcaljenje je involucija,
  \item zrcaljenje  preslika eno od
območij ravnine, ki ga os (krožnica) zrcaljenja določa, v drugo
območje,
  \item vse
fiksne točke zrcaljenja ležijo na osi (krožnici) tega
zrcaljenja,
  \item če je $X'$ slika točke $X$ pri zrcaljenju, potem je premica
  $XX'$
pravokotna na os (krožnico) tega zrcaljenja.
\end{itemize}

 Preslikava, ki jo bomo sedaj
definirali, naj bi torej izpolnjevala vse te lastnosti. Namesto imena
 ‘‘zrcaljenje čez krožnico’’ bomo raje uporabljali
 termin ‘‘inverzija’’\index{inverzija}\footnote{Inverzijo je prvi vpeljal
 nemški matematik
\index{Magnus, L. I.} \textit{L. I. Magnus} (1790--1861) leta 1831.
Prve ideje inverzije so se pojavile že pri starogrškem matematiku
\index{Apolonij} \textit{Apoloniju iz Perge} (3.--2. st. pr. n.
š.) in švicarskem matematiku \index{Steiner, J.} \textit{J.
Steinerju} (1796--1863)}.


 V evklidski ravnini bomo inverzijo definirali na naslednji način.
Naj bo $k(O,r)$ krožnica v evklidski ravnini $\mathbb{E}^2$. Točka $X'$ je slika neke točke $X\in \mathbb{E}^2\setminus
\{O\}$ v \index{inverzija} \pojem{inverziji} $\psi_k$ glede na
krožnico $k$ (njena \index{inverzivna slika} \pojem{inverzivna slika}), če pripada poltraku $OX$ in velja $|OX|\cdot
|OX'|=r^2$, $k$ je \index{krožnica!inverzije}\pojem{krožnica
inverzije}, $O$ pa \index{središče!inverzije} \pojem{središče
inverzije}.

 Inverzijo $\psi_k$ glede na krožnico $k(O,r)$ bomo pogosto označevali
 tudi  $\psi_{O,r}$.

Iz same definicije neposredno sledi, da so edine negibne točke
inverzije ravno točke na krožnici inverzije. In vse točke
krožnice inverzije so negibne. Velja torej ekvivalenca:
$$\psi_k(X)=X\hspace*{1mm}\Leftrightarrow\hspace*{1mm}X\in k.$$

Inverzija $\psi_k$ je bijektivna preslikava na množici
$\mathbb{E}^2\setminus \{O\}$. Velja tudi $\psi_k^{-1}=\psi_k$, torej je
$\psi_k^2$ je identična preslikava. Obe lastnosti sledita
neposredno iz definicije. Direktno sledi tudi, da se zunanje
točke krožnice $k$ z inverzijo $\psi_k$ preslikajo v notranje
točke in obratno (brez točke O).

 Iz same definicije sledi, da je premica $XX'$ pravokotna na
 krožnici inverzije, kar pomeni, da je izpolnjena tudi zadnja
 željena lastnost.

Sedaj bomo konstruirali sliko $X'=\psi_k(X)$ poljubne točke $X$
pri inverziji. Iz vsega povedanega (posebej iz dejstva
$\psi_k^{-1}=\psi_k$) sledi, da zadošča opis konstrukcije, ko je $X$ v zunanjosti krožnice $k$.
 Naj bo v tem primeru $OX$ tangenta
krožnice $k$ v njeni točki $T$, potem iz $|OX|\cdot
|OX'|=r^2=|OT|^2$ sledi, da sta trikotnika $OTX$ in $OX'T$ podobna
in je zato $\angle TX'O=90^0$ (Figure \ref{sl.inv.9.1.2.pic}). Ker
je točka $X'$ tudi na poltraku $OX$, to točko dobimo kot
pravokotno projekcijo točke $T$ na poltrak $OX$.

\begin{figure}[!htb]
\centering
\input{sl.inv.9.1.2.pic}
\caption{} \label{sl.inv.9.1.2.pic}
\end{figure}

%slikaNova7-4
%\includegraphics[width=50mm]{slikaNova7-4.pdf}


 Iz relacije v definiciji inverzije $|OX|\cdot
|OX'|=r^2$ sledi, da če se točka $X$ približuje središču $O$,
 se  njena slika $X'$ ‘‘oddaljuje proti neskončnosti’’. Velja tudi - če je
točka $X$ v bližini krožnice $k$, je tudi njena slika v bližini
te krožnice, seveda na drugi strani. Vse te ugotovitve lahko
zapišemo v bolj formalni obliki.

\bizrek \label{invUrejenost} Če velja $\psi_k:A,B \mapsto A',B'$ in
$\mathcal{B}(O,A,B)$,
 potem je tudi $\mathcal{B}(O,B',A')$.
 \eizrek

 Točka $O$ nima svoje slike. Intuitivno je njena slika točka v
neskončnosti. Več o tem bomo povedali v poglavju \ref{odd9InvRavn}.

Dokažimo naslednjo pomembno lastnost inverzije.

\bizrek \label{invPodTrik} Naj bodo $O$, $A$ in $B$ tri
nekolinearne točke in $\psi_k$ inverzija glede na krožnico
$k(O,r)$. Če sta $A'$ in $B'$ sliki točk $A$ in $B$ v tej
inverziji, sta si trikotnika $OAB$ in $OB'A'$ podobna, torej:
$$\psi_{O,r}:A,B\mapsto A',B' \Rightarrow \triangle OAB \sim \triangle
OB'A'.$$
 \eizrek


\begin{figure}[!htb]
\centering
\input{sl.inv.9.1.3.pic}
\caption{} \label{sl.inv.9.1.3.pic}
\end{figure}

 \textbf{\textit{Proof.}} Ker je $\psi_{O,r}:A,B\mapsto A',B'$,
  je tudi
$OA\cdot OA'=OB\cdot OB'=r^2$ (Figure \ref{sl.inv.9.1.3.pic}).
Podobnost trikotnikov sedaj sledi iz $OA:OB'=OB:OA'$ in $\angle AOB
\cong \angle B'OA'$.
 \kdokaz

 Naslednja trditev je direktna posledica prejšnjega izreka.

\bizrek Naj bodo $O$, $A$ in $B$ tri nekolinearne točke in
$\psi_k$ inverzija glede na krožnico $k(O,r)$. Če sta $A'$ in
$B'$ slike točk $A$ in $B$ v tej inverziji, potem velja:

 (i) $\angle
OAB \cong \angle OB'A'$ in $\angle OBA \cong \angle OA'B'$,

(ii) točke $A$, $B$, $B'$ in $A'$ so konciklične,

(iii) krožnica, ki poteka skozi točke $A$, $B$, $B'$ in $A'$, je
pravokotna na krožnico inverzije.
 \eizrek

\begin{figure}[!htb]
\centering
\input{sl.inv.9.1.4.pic}
\caption{} \label{sl.inv.9.1.4.pic}
\end{figure}

\textbf{\textit{Proof.}}
 \textit{(i)} Skladnost kotov sledi iz podobnosti trikotnikov $OAB$ in
 $OB'A'$(prejšnji izrek \ref{invPodTrik}).


 \textit{(ii)} Naj bo $l$ očrtana krožnica trikotnika
 $ABA'$ (Figure \ref{sl.inv.9.1.4.pic}). Iz definicije inverzije sledi
 $OA\cdot OA'=OB\cdot
 OB'=r^2$. To pomeni, da je potenca točke $O$ na krožnico $l$
 enaka $OA\cdot OA'=OB\cdot
 OB'$, torej točka $B'$ leži na tej krožnici.

 Omenimo, da tetivnost štirikotnika $AA'B'B$ sledi tudi direktno
 iz $\angle
OAB \cong \angle OB'A'$, ker je potem $\angle A'AB + \angle BB'A'
= 180^0 $.

 \textit{(iii)} Iz $\psi_{O,r}(A)=A'$ sledi, da je ena izmed točk
 $A$ in $A'$ v notranjosti, druga pa v zunanjosti krožnice
 inverzije $k$. Po posledici Dedekindovega aksioma
 (\ref{DedPoslKrozKroz}) imata krožnici $k$ in $l$ dve skupni
 točki; eno od njiju označimo s $T$. Ker je $OA\cdot OA'=OB\cdot
 OB'=r^2=OT^2$, sledi, da je $OT$ tangenta krožnice $k$,
 kar pomeni, da sta krožnici $k$ in $l$ pravokotni (izrek
 \ref{pravokotniKroznici}).
  \kdokaz

Na enak način kot v delu (\textit{iii}) prejšnjega izreka,
dokazujemo tudi naslednjo trditev.

 \bizrek \label{invPravKrozn} Naj
bo $X'$ slika točke $X$ ($X\neq X'$) pri inverziji $\psi_k$
glede na krožnico $k(O,r)$. Potem je vsaka krožnica, ki gre
skozi točki $X$ in $X'$ (tudi premica $XX'$) pravokotna na
krožnici inverzije $k$.
 \eizrek

 V naslednjem primeru bomo videli, kako lahko inverzijo s pomočjo
 določenih raztegov prevedemo v inverzijo s koncentrično
 krožnico inverzije.

\bzgled \label{invRazteg} Kompozitum inverzije $\psi_{S,r}$ in
raztega $h_{S,k}$ z istim središčem $S$  in pozitivnim
koeficientom ($k>0$) je inverzija s središčem $S$ in
koeficientom $r\sqrt{k}$.
 \ezgled
\textbf{\textit{Proof.}} Naj bo  $f=h_{S,k}\circ \psi_{S,r}$. Če za
poljubno točko $X$ označimo $X'=f(X)$, je najprej jasno, da točka
$X'$ leži na poltraku $SX$. Označimo še $X_1=\psi_{S,k}(X)$. Iz
definicij inverzije in raztega sledi $|SX_1| \cdot |SX| =r^2$ in
$|SX'| = k\cdot |SX_1| $. Zato je $|SX'| \cdot |SX| =k\cdot
r^2=\left(r\sqrt{k}\right)^2$ oz. $\psi_{S,r\sqrt{k}}(X)=X' =f(X)$.
Ker to velja za poljubno točko $X$, je $f=h_{S,k}\circ
\psi_{S,r}=\psi_{S,r\sqrt{k}}$.
 \kdokaz


%________________________________________________________________________________
 \poglavje{Image of a Circle or Line Under an Inversion} \label{odd9SlokaKrozPrem}

V tem razdelku bomo ugotovili, da inverzija ni kolineacija, kar pomeni, da
ne ohranja relacije kolinearnosti točk. Obravnavali bomo slike
 premic in krožnic pri inverziji. Ker je definicijsko območje
 inverzije evklidska ravnina brez središča inverzije, je smiselno
 vpeljati naslednjo oznako: če je $\Phi$ poljubni lik
evklidske ravnine
 in $S$ točka te ravnine, je
  $$\Phi^S= \Phi \setminus
 \{S\}.$$

\bizrek \label{InverzKroznVkrozn}
   Naj bo $\psi_i$  inverzija glede na krožnico $i(S,r)$ evklidske ravnine.
   Če sta $p$ premica in $k$
krožnica te ravnine, velja (Figure \ref{sl.inv.9.2.1.pic}):

(i) če $S\in p$, je $\psi_i(p^S)=p^S$,

(ii) če $S\notin p$, je $\psi_i(p)=j^S$, kjer je $j$ krožnica, ki
poteka skozi točko $S$,

(iii) če $S\in k$, je y $\psi_i(k^S)=q$, kjer je $q$ premica, ki
ne poteka skozi točko $S$;

(iv) če $S\notin k$,  je $\psi_i(k)=k'$, kjer je k' krožnica, ki
ne poteka skozi točko $S$.
   \eizrek


\begin{figure}[!htb]
\centering
\input{sl.inv.9.2.1.pic}
\caption{} \label{sl.inv.9.2.1.pic}
\end{figure}

 \textbf{\textit{Proof.}}

  (\textit{i}) Iz definicije inverzije sledi, da slika poljubne
  točke
   $X \in p^S$ leži na odprtem poltraku $SX$, zato velja $\psi_i(X)\in
   p^S$. Ker je še $\psi_i^{-1}=\psi_i$, je poljubna točka $Y\in
   p^S$ slika točke $\psi_i^{-1}(Y)=\psi_i(Y)\in
   p^S$. Torej velja $\psi_i(p^S)=p^S$.

   (\textit{ii}) Naj bo $P$ pravokotna projekcija točke
   $S$ na premici $p$ (Figure \ref{sl.inv.9.2.2.pic}).
   Ker $S\notin p$, je potem $P\neq S$, zato
obstaja slika točke $P$ pri inverziji $\psi_i$ -- označimo jo s $P'$.
Naj bo $X$ poljubna točka premice $p$ ($X\neq P$) in $X'=\psi_i(X)$.
Po izreku \ref{invPodTrik} sta si trikotnika $SPX$ in $SX'P'$ podobna,
zato je $\angle SX'P' \cong \angle SPX=90^0$. Torej točka $X'$ leži
na krožnici nad premerom $SP'$ - označimo ga z $j$. Ker je $X'\neq
S$, velja $X'\in j^S$ oz. $\psi_i(p)\subseteq j^S$. Z obratnim
postopkom dokažemo, da je vsaka točka množice $j^S$ slika neke točke
premice $p$, zato je
 $\psi_i(p)= j^S$.

\begin{figure}[!htb]
\centering
\input{sl.inv.9.2.2.pic}
\caption{} \label{sl.inv.9.2.2.pic}
\end{figure}

   (\textit{iii}) Direktna posledica (\textit{ii}), ker je $\psi_i$  involucija,
   oz. $\psi_i^{-1}=\psi_i$.

   (\textit{iv}) (Figure \ref{sl.inv.9.2.2.pic}) Naj bosta $P$ in $Q$
   presečišči krožnice $k$ s premico, ki poteka skozi točko $S$,
   in središče krožnice
   $k$ (primer, ko sta krožnici $i$ in $k$ koncentrični, je enostaven).
    Brez škode za splošnost predpostavimo, da velja $\mathcal{B}(S,P,Q)$.
     Naj bo $P'=\psi_i(P)$,
$Q'=\psi_i(Q)$ in $X$ poljubna točka krožnice $k$, ki je
različna od točk $P$ in $Q$ ter $X'=\psi_i(X)$. Po izreku
\ref{invPodTrik} je $\triangle SPX \sim \triangle SX'P'$ in $\triangle SQX
\sim \triangle SX'Q'$. Zaradi tega je
 $\angle SX'P' \cong \angle SPX$ in $\angle SX'Q'\cong \angle SQX$, torej:
$$\angle P'X'Q'=\angle SX'P'-\angle SX'Q'
=\angle SPX-\angle SQX=\angle PXQ=90^0.$$
 Točka $X'$ potem leži na
krožnici nad premerom $P'Q'$ - označimo jo s $k'$. Z obratnim
postopkom lahko dokažemo, da je tudi vsaka točka krožnice $k'$
slika neke točke krožnice $k$, zato je $\psi_i(k)=k'$

Trditev analogno dokažemo tudi v primeru, kadar velja
$\mathcal{B}(P,S,Q)$.
 \kdokaz

Prejšnji izrek nam poda tudi efektiven način konstrukcije slike
premice oz. krožnice v različnih primerih. Uporabili bomo
oznake iz tega izreka (\ref{InverzKroznVkrozn}).

V primeru (\textit{ii}) - slika \ref{sl.inv.9.2.2a.pic} - je dovolj
določiti sliko pravokotne projekcije $P$ središča inverzije $S$
na premici $p$, ki jo preslikamo. Točka $P'=\psi_i(P)$ s
središčem inverzije določa premer krožnice $j$. Če pa premica
$p$ seka krožnico inverzije $i(S,r)$ npr. v točkah $M$ in $N$,
sta tidve točki fiksni in je slika premice $p$ očrtana krožnica
trikotnika $SMN$ (brez točke $S$). Tangenta krožnice inverzije
$i$ pa se preslika v krožnico (brez točke $S$), ki se od znotraj
dotika krožnice $i$.


\begin{figure}[!htb]
\centering
\input{sl.inv.9.2.2a.pic}
\caption{} \label{sl.inv.9.2.2a.pic}
\end{figure}


V primeru (\textit{iii}) naredimo obraten postopek konstrukciji iz
(\textit{ii}) in dobimo pravokotno projekcijo na premico, ki je
slika dane krožnice. Tudi ostali posebni primeri (ko krožnica
seka ali se dotika krožnice inverzije) so obratni primerom iz
(\textit{ii}).

 Tudi v primeru (\textit{iv}),
 kadar se krožnica preslika v krožnico, nam je postopek
 konstrukcije jasen iz samega izreka - slika $k'$ je določena s
 točkama $P'$ in $Q'$.
 Moramo pa biti pozorni, da se pri tem središče krožnice $k$ ne
 preslika v središče nove krožnice $k'$!
 Kje v tem primeru leži slika središča
 te krožnice? Označimo z $O'$ sliko središča $O$ krožnice
 $k$ pri inverziji $\psi_i$.
 Naj  bosta $t_1=ST_1$ in $t_2=ST_2$ tangenti krožnice $k$
 v točkah $T_1$ in $T_2$ ter $T'_1=\psi_i(T_1)$ in
 $T'_2=\psi_i(T_2)$
(Figure \ref{sl.inv.9.2.3.pic}). Jasno je, da velja $T'_1, T'_2 \in
k'$, toda $t_1$ in $t_2$ sta tudi tangenti krožnice $k'$ (če je
denimo $Y\in k'\cap t_1$, je tudi $X=\psi_i(Y)\in k\cap
t_1=\{T_1\}$, oz. $X=T'_1$). Po izreku \ref{invPodTrik} je
 $\triangle ST_1O \sim \triangle SO'T'_1$ oz. $\angle SO'T'_1 \cong \angle
 ST_1O=90^0$. Iz istih razlogov je tudi $\angle SO'T'_2 \cong \angle
 ST_2O=90^0$. To pomeni, da so točke $T'_1$, $O'$ in $T'_2$
 kolinearne. Torej točko $O'$ lahko dobimo kot presečišče daljice
 $T'_1T'_2$ in poltraka $SO$.

\begin{figure}[!htb]
\centering
\input{sl.inv.9.2.3.pic}
\caption{} \label{sl.inv.9.2.3.pic}
\end{figure}

Še eno vprašanje se ponuja glede primera (\textit{iv}).
Katere krožnice so fiksne pri inverziji? Ena takšna je seveda
sama krožnica inverzije $i$, saj velja, da so vse njene točke
fiksne. Naslednji izrek bo odkril tudi vse druge možnosti.

 \bizrek \label{InverzKroznFiks}
   Naj bo $\psi_i$ inverzija glede na krožnico $i(S,r)$.
   Edine fiksne krožnice te inverzije so krožnica $i$ in krožnice,
   ki so pravokotne na to krožnico, torej:
   $$\psi_i(k)=k \Leftrightarrow k=i \vee k\perp i.$$
   \eizrek

\begin{figure}[!htb]
\centering
\input{sl.inv.9.2.4.pic}
\caption{} \label{sl.inv.9.2.4.pic}
\end{figure}

\textbf{\textit{Proof.}}

 ($\Leftarrow$) Če je $k=i$, je trditev trivialna. Naj bosta $k$ in
 $i$
pravokotni krožnici (Figure \ref{sl.inv.9.2.4.pic}). S $T$
označimo eno njuno presečišče. Premica $ST$ je tangenta
krožnice $k$ (izrek \ref{TangPogoj}). Naj bo $X$ poljubna točka
krožnice $k$ ($X\notin i$) in $X'$ drugo presečišče te
krožnice s poltrakom $SX$. Tedaj je:
 $$p(S,k)= |SX|\cdot |SX'| = |OT|^2 = r^2,$$
zato je $\psi_i(X)=X'\in k$. Analogno je vsaka točka $Y$
krožnice $k$ slika neke točke te krožnice, zato velja
$\psi_i(k)=k$.

 ($\Rightarrow$) Naj bo $\psi_i(k)=k$. Če je $k\neq i$, potem
  obstaja točka $X$ na
krožnici $k$, ki ne leži na krožnici $i$. Naj bo $X'=\psi_i(X)$.
Jasno je, da velja tudi $X'\notin i$. Iz definicije inverzije sledi,
da sta točki $X$ in $X'$ na poltraku z začetno točko $S$, kar
pomeni, da je $S$ zunanja točka krožnice $k$ (ker ni
$\mathcal{B}(X,S,X')$). Torej obstaja tangenta iz točke $S$ na
krožnico $k$ -- s $T$ označimo dotikališče. Tedaj je:
 $$p(S,k)=|ST|^2 = |SX| \cdot |SX'| = r^2,$$
 zato točka $T$ leži na
krožnici $i$. To pomeni, da sta krožnici $k$ in $i$ pravokotni
(izrek \ref{TangPogoj}).
 \kdokaz

 Omenimo, da analogno velja tudi za fiksne premice, le da
 pravokotnost premice s krožnico inverzije pomeni, da premica poteka skozi
 središče inverzije. Torej so edine fiksne premice inverzije
 tiste, ki potekajo skozi središče inverzije (del (\textit{i}) izreka
 \ref{InverzKroznVkrozn}).

 Prejšnji izrek velja tudi, če namesto inverzije govorimo o
 zrcaljenju čez premico. Edine fiksne premice (krožnice)
 tega zrcaljenja so namreč os zrcaljenja in tiste  premice (krožnice),
 ki so pravokotne na to os. To bo motiv, da na določen način
 izenačimo premice in krožnice, s tem pa tudi osno zrcaljenje in
 inverzijo. O tem bomo več izvedeli v naslednjem razdelku.

 \bzgled Dani so kot $\gamma$ ter daljici $c$ in $p$.
Načrtaj trikotnik $ABC$, če je $AB\cong c$, $\angle BCA\cong \gamma$
in $|BA'|\cdot|BC|=p^2$ (daljica $AA'$ je višina trikotnika $ABC$).
 \ezgled

\begin{figure}[!htb]
\centering
\input{sl.inv.9.2.5.pic}
\caption{} \label{sl.inv.9.2.5.pic}
\end{figure}

\textbf{\textit{Solution.}} Naj bo $\triangle ABC$ iskani trikotnik,
ki izpolnjuje pogoje iz naloge (Figure \ref{sl.inv.9.2.5.pic}).
Najprej iz pogoja $\angle BCA\cong \gamma$ sledi, da točka $C$
leži na loku $l$, ki je določen s tetivo $AB$ in obodnim kotom
$\gamma$. Ker je $A'$ nožišče višine, točka $A'$ leži na
krožnici $k$ nad premerom $AB$. Naj bo $\psi_i$ inverzija glede
na krožnico $i(B,p)$. Iz pogoja
 $|BA'|\cdot|BC|=p^2$ sledi $\psi_i(A')=C$. Ker je še $A'\in k$,
 velja tudi $C\in k'$, kjer je $k'=\psi_i(k)$ premica (izrek
 \ref{InverzKroznVkrozn}).

 Če torej najprej načrtamo daljico $AB\cong c$, dobimo točko
 $C$ kot presečišče premice $k'=\psi_i(k)$ z lokom $l$.
\kdokaz

\bzgled \label{MiquelKroznice}(Miquelov\footnote{\index{Miquel, A.}\textit{A. Miquel} (1816–-1851), francoski
matematik, ki je ta izrek objavil leta 1840.} izrek o šestih
krožnicah)
 Naj bodo $k_1$, $k_2$, $k_3$ in $k_4$ takšne krožnice, da
 se krožnici $k_1$ in $k_2$  sekata v točkah $A$ in
$A_0$, krožnici $k_2$ in $k_3$ v točkah $B$ in $B_0$,
krožnici $k_3$ in $k_4$ v točkah $C$ in $C_0$, krožnici
$k_4$ in $k_1$ pa v točkah $D$ in $D_0$. Če so točke $A$, $B$, $C$
in $D$ konciklične, so točke $A_0$, $B_0$, $C_0$ in $D_0$
konciklične ali kolinearne. \index{izrek!Miquelov.}
 \ezgled

\begin{figure}[!htb]
\centering
\input{sl.inv.9.3.7a.pic}
\caption{} \label{sl.inv.9.3.7a.pic}
\end{figure}

\textbf{\textit{Proof.}}  Označimo s $k$ krožnico, ki jo
določajo točke $A$, $B$, $C$ in $D$. Naj bo $\psi_i$ inverzija
glede na poljubno krožnico $i$ s središčem v točki $A$ (Figure
\ref{sl.inv.9.3.7a.pic}) in
 $$\psi_i:\hspace*{1mm}B,C,D,A_0,B_0,C_0,D_0 \mapsto
 B',C',D',A'_0,B'_0,C'_0,D'_0.$$
   Ker gredo krožnice $k$, $k_1$ in
$k_2$ skozi točko $A$, so po izreku \ref{InverzKroznVkrozn}
njihove slike $k'$, $k'_1$ in $k'_2$ pri inverziji $\psi_i$
premice, ki določajo oglišča trikotnika $A_0'B'D'$. Krožnici
$k_3$ in $k_4$ pa ne gresta skozi točko $A$, zato sta njuni sliki
krožnici $k'_3$ in $k'_4$, ki se sekata v točkah $C'\in B'D'$ in
$C'_0$. Naj bo $k'_0$ očrtana krožnica trikotnika
$A'_0D'_0B'_0$. Iz zgleda \ref{Miquelova točka} sledi, da se
krožnice $k'_3$, $k'_4$ in $k'_0$ sekajo v eni točki - $C'_0$,
oz. $A'_0,B'_0,C'_0,D'_0\in k'_0$. Zaradi tega je tudi
$A_0,B_0,C_0,D_0\in k_0=\psi_i(k'_0)$. Po izreku
\ref{InverzKroznVkrozn} pa je $k_0$ krožnica ali premica.
 \kdokaz


%________________________________________________________________________________
 \poglavje{The Inversive Plane}
\label{odd9InvRavn}

Motiv za naslednjo obravnavo je dejstvo, da pri inverziji
definicijsko območje ni cela evklidska ravnina - središče inverzije
nima svoje slike. To nam pogosto oteži delo, kot je to npr. pri
formulaciji izreka \ref{InverzKroznVkrozn}, kjer moramo vedno paziti,
kdaj premica oz. krožnica poteka skozi središče inverzije. Tudi slike
premice in krožnice niso vedno cela premica oz. krožnica, ker moramo v
določenih primerih izločiti središče inverzije.

Zaradi vsega navedenega se ponuja druga rešitev; namesto da iz
definicijskega območja izključimo središče inverzije, lahko
evklidski ravnini formalno dodamo eno točko, ki bo slika središča inverzije
pri tej inverziji. Iz definicije inverzije za sliko $X'$ točke $X$
pri inverziji $\psi_{S,k}$ velja:
 $|SX'|\cdot |SX|=r^2$. Če je $S=X$,  potem formalno dobilmo
  $|SX'|\cdot 0=r^2$, kar pomeni, da mora biti
  $|SX'|=\infty$. Tako novo točko intuitivno vidimo kot
  ‘‘neskončno oddaljeno točko’’, zato jo bomo formalno
  označili s simbolom $\infty$ in imenovali \index{točka!v neskončnosti}\pojem{točka v neskončnosti}. Množico, ki nastane z dodajanjem
  te nove točke evklidski ravnini, imenujemo \index{inverzivna
  ravnina} \pojem{inverzivna ravnina},
  ki jo bomo označili z $\widehat{E}^2$. Torej:
 $$\widehat{E}^2=E^2 \cup \{\infty\}.$$
 Vendar moramo biti pozorni - ne smemo namreč mešati
inverzivne ravnine z \index{razširjena evklidska ravnina}
\pojem{razširjeno evklidsko ravnino}, ki jo dobimo, če evklidski
ravnini dodamo eno (neskončno oddaljeno) premico in opredelimo, da se
vzporedni premici na njej  sekata. Razširjena evklidska ravnina je
model t. i. projektivne geometrije, ki smo jo že omenjali v uvodu.

V primeru inverzivne ravnine smo torej dodali le eno točko
$\infty$ in pri tem zahtevali, da za vsako inverzijo $\psi_{S,k}$
velja $\psi_{S,k}(S)=\infty$. Ker je $\psi_{S,k}$ involucija
(velja $\psi_{S,k}^{-1}=\psi_{S,k}$), naj po dogovoru velja tudi
$\psi_{S,k}(\infty)=S$.

V delu (\textit{iii}) izreka \ref{InverzKroznVkrozn} smo videli, da
je slika lika $k^S$ (krožnice brez točke $S$) pri inverziji
$\psi_{S,k}$ premica $q$, ki ne poteka skozi točko $S$. Če pa liku
$k^S$ dodamo točko $S$, bomo za sliko dobili množico $q \cup
\{\infty\}$. Zato bomo v inverzivni ravnini na naraven način
premicam dodelili točko $\infty$. Če takšne premice z dodano
točko $\infty$ in krožnice skupaj imenujemo $i$-krožnice, lahko
prevedemo izrek \ref{InverzKroznVkrozn} in rečemo, da se z
inverzijo $i$-krožnica preslika v $i$-krožnico. Toda videli smo
tudi določene analogije zrcaljenja čez premico in inverzijo.
Tako bomo obe preslikavi v inverzivni ravnini imenovali
$i$-inverziji. Ni težko videti, da lahko potem posplošimo izrek
\ref{InverzKroznVkrozn} in ga zapišemo v enostavnejši
obliki.


 \bizrek \label{InverzInvRavKvK}
   Z $i$-inverzijo se $i$-krožnica preslika v $i$-krožnico.
   \eizrek

   Dejstvo, da v inverzivni ravnini vsaka premica vsebuje točko
   $\infty$, pomeni, da se dve vzporednici sekata v tej točki. Premici,
   ki se sekata (v navadni točki evklidske ravnine) imata v
   inverzivni ravnini dve skupni točki.
   Ker imata vzporednici
   le eno skupno točko, rečemo tudi, da se dotikata v
   točki $\infty$. Slika dveh vzporednic pri $i$-inverziji bosta
   bodisi vzporednici (če gre za osno zrcaljenje) bodisi
   dve krožnici, ki se dotikata v središču inverzije (če gre
   za inverzijo). Dve krožnici, ki se dotikata (ker imata le eno
   skupno točko), se preslikata bodisi v vzporednici bodisi v
   krožnici, ki se dotikata. Podobno velja tudi v primeru dotika
   premice in krožnice. Vse te primere v inverzivni ravnini lahko
   veliko krajše formuliramo v naslednji obliki.

        \bizrek \label{InverzDotik}
         Naj bo $\psi_i$ poljubna inverzija.
        Če se $i$-krožnici $k$ in $l$ dotikata, se dotikata tudi
        $i$-krožnici $\psi_i(k)$ in  $\psi_i(l)$.
        \eizrek

S pomočjo tega izreka bomo najprej dokazali naslednji pomemben
izrek.

          \bizrek \label{InverzKonf}
          Inverzija je \index{konformna preslikava}konformna preslikava, kar pomeni:

   Kot, pod katerim se premici $p$ in $q$ sekata v presečišču
$A$, je enak kotu, pod katerim se sekata njuni inverzni sliki $p'$
in $q'$ v pripadajoči točki $A'$.
   \eizrek


\begin{figure}[!htb]
\centering
\input{sl.inv.9.3.1.pic}
\caption{} \label{sl.inv.9.3.1.pic}
\end{figure}

\textbf{\textit{Proof.}} Naj bo $\psi_i$ inverzija glede na
krožnico  $i(S,r)$, pri kateri se premici $p$ in $q$ in njuno
presečišče $A$  preslikajo v $p'$, $q'$ in $A'$ (Figure
\ref{sl.inv.9.3.1.pic}).

Primer, če je $A=S$, je trivialen, ker sta tedaj premici $p$ in
$q$ nepremični.

 Predpostavimo najprej, da $A\notin i$. V tem primeru je jasno $A\neq
 A'$.

  Naj bo točka
$P$ presečišče pravokotnice premice $p$ v točki $A$ in
simetrale daljice $AA'$ ter $k_p$ krožnica s središčem $P$, ki
gre skozi točko $A$. Iz konstrukcije krožnice $k_p$ je jasno, da
tudi točka  $A'$ leži na tej krožnici in je premica $p$ njena
tangenta v točki $A$. Če na podoben način s $Q$ označimo
presečišče pravokotnice premice $q$ v točki $A$ in simetrale
daljice $AA'$ ter $k_q$ krožnico s središčem $Q$, ki gre skozi
točko $A$, sledi, da točka  $A'$ leži na krožnici $k_q$ in je
premica $q$ njena tangenta v točki $A$.

Ker gresta obe krožnici $k_p$ in $k_q$  skozi par točk $A$, $A'$
inverzije $\psi_i$, sta $k_p$ in $k_q$ pravokotni na krožnico
inverzije $i$ (izrek \ref{invPravKrozn}). Po izreku
\ref{InverzKroznFiks} je potem $\psi_i(k_p)=k_p$ in
$\psi_i(k_q)=k_q$.

Premici $p$ in $q$ se po vrsti dotikata krožnic $k_p$ in $k_q$.
Po izreku \ref{InverzDotik} se dotikata $i$-krožnici $p'$ in
$\psi_i(k_p)=k_p$, oz. $q'$ in $\psi_i(k_q)=k_q$. Zaradi vsega
tega velja:
 $$\angle p,q \cong \angle k_p,k_q \cong \angle p',q'.$$

  Kadar je $A\in i$, primer prevedemo na prejšnjega
  (z inverzijo glede na koncentrično krožnico) z uporabo izreka
  \ref{invRazteg}, ker razteg
  ohranja velikost kotov.
 \kdokaz

 Jasno je, da bi bil dokaz praktično enak, če bi bili $p$ in $q$
 krožnici oz. krožnica in premica. Zato lahko prejšnji izrek
 zapišemo v bolj splošni obliki.

        \bizrek
          Vsaka $i$-inverzija ohranja kote med dvema $i$-krožnicama.
        \eizrek

 Poseben primer prejšnjega izreka se nanaša na prave kote. Iz
 tega sledi, da je pravokotnost invarianta inverzij in velja
 naslednji izrek.

      \bizrek \label{InverzKonfPrav}
      Vsaka $i$-inverzija preslika dve pravokotni $i$-krožnici v
      pravokotni $i$-krožnici.
      \eizrek

 Sedaj bomo uporabili dokazane trditve, ki veljajo
 v evklidski ravnini.
 Dejstvo, da se krožnica, ki poteka skozi središče
 inverzije, s to inverzijo preslika v premico, nam omogoča
 reševanje različnih načrtovalnih nalog, kjer bomo namesto
 iskane krožnice najprej konstruirali njeno sliko - premico. To
 se nanaša tudi na druge naloge, kjer bomo neko trditev, ki se
 nanaša na krožnico, z inverzijo prevedli v ekvivalentno trditev,
 ki se nanaša na premico. V obeh primerih je zaželeno, da imamo
 vsaj eno točko krožnice, ki jo potem izberemo za središče inverzije.

 Omenimo še, da bomo zaradi vsega navedenega pogosto rekli,
 da je slika premice $p$ ($S\notin p$) pri inverziji $\psi_{S,r}$ kar
 krožnica $j$ ($j\ni S$) namesto $j^S$. In obratno - v istem
 primeru bomo pisali $\psi_{S,r}(j)=p$.

      \bzgled Dane so krožnica $k$ ter točki $A$ in $B$. Načrtaj
       krožnico (premico) $x$, ki je pravokotna na krožnico $k$ ter
      poteka skozi točki $A$ in $B$.
      \ezgled


\begin{figure}[!htb]
\centering
\input{sl.inv.9.3.2.pic}
\caption{} \label{sl.inv.9.3.2.pic}
\end{figure}

\textbf{\textit{Solution.}}  Naj bo $x$ krožnica, ki je
pravokotna na krožnico $k$ in
 poteka skozi točki $A$ in $B$, ter $\psi_i$
inverzija glede na poljubno krožnico $i$ s središčem $A$ (Figure
\ref{sl.inv.9.3.2.pic}).

Predpostavimo najprej, da $A\notin k$. V tem primeru je
$k'=\psi_i(k)$ krožnica in $x'=\psi_i(x)$ premica, ki gre skozi
točko $B'=\psi_i(B)$ (izrek \ref{InverzKroznVkrozn}). Po izreku
\ref{InverzKonfPrav} iz $x \perp k$ sledi $x' \perp k'$. Zato je
$x'$ premica, ki poteka skozi točko $B'$ in je pravokotna na
krožnico $k'$ oz. poteka skozi njeno središče.

Premico $x'$ torej lahko načrtamo kot premico, ki gre skozi
točki $B'=\psi_i(B)$ in središče krožnice $k'=\psi_i(k)$. Na
koncu je $x=\psi_i(x')$.

Če pa velja $A\in k$, je $k'$ premica (izrek
\ref{InverzKroznVkrozn}) in premico $x'$ načrtamo kot njeno
pravokotnico skozi točko $B'$.

Ker vedno lahko narišemo eno samo pravokotnico $x'$, obstaja
natanko ena rešitev tudi za $i$-krožnico $x$. Toda $x$ je
krožnica natanko tedaj, ko $S\notin x'$  oz. kadar točke $A$,
$B$ in središče krožnice $k$ niso kolinearne. V primeru
kolinearnosti teh točk pa je rešitev $x$ premica.
 \kdokaz

     \bzgled \label{TriKroznInv}
    Če se med tremi krožnicami $k_1$, $k_2$ in $k_3$
     neke ravnine po dve od zunaj dotikata,
     je krožnica $k$, ki jo določajo dotikališča,
     pravokotna na vsako od krožnic $k_1$, $k_2$ in $k_3$.
     \ezgled

\begin{figure}[!htb]
\centering
\input{sl.inv.9.3.3.pic}
\caption{} \label{sl.inv.9.3.3.pic}
\end{figure}

\textbf{\textit{Proof.}}   Označimo z $A$ dotikališče krožnic
$k_2$ in $k_3$, z $B$  dotikališče krožnic $k_1$ in $k_3$ ter s
$C$ dotikališče krožnic $k_2$ in $k_1$ (Figure
\ref{sl.inv.9.3.3.pic}). Krožnica $k$ je potem očrtana krožnica
trikotnika $ABC$.

Naj bo $\psi_{A,r}$ inverzija s središčem $A$ in poljubnim
polmerom $r$. Krožnici $k_2$ in $k_3$ se dotikata v središču
inverzije $A$, zato se s to inverzijo preslikata v premici $k'_2$
in $k'_3$ (izrek \ref{InverzKroznVkrozn}), ki nimata skupnih točk
oz. $k'_2\parallel k'_3$. Krožnica $k_1$, ki ne poteka skozi točko
$A$ in se krožnic $k_2$ in $k_3$ dotika v točkah $C$ in $B$, se
preslika v krožnico $k'_1$ (izrek \ref{InverzKroznVkrozn}), ki se
dotika premic $k'_2$ in $k'_3$ (izrek \ref{InverzDotik}) v točkah
$C'=\psi_{A,r}(C)$ in $B'=\psi_{A,r}(B)$. Torej sta premici $k'_2$
in $k'_3$ vzporedni tangenti krožnice $k'_1$ v točkah $C'$ in
$B'$, zato je $B'C'$ njuna skupna pravokotnica, daljica $B'C'$
pa je premer krožnice $k'_1$. Premica $B'C'$ je slika krožnice $k$,
ki poteka skozi središče inverzije $A$ in skozi točki $B$ in $C$
(izrek \ref{InverzKroznVkrozn}), oz. $\psi_{A,r}(k)=k'=B'C'$. Ker
je premica $k'=B'C'$ pravokotna na premici $k'_2$ in $k'_3$ ter
na krožnico $k'_1$, je po izreku \ref{InverzKonfPrav} krožnica
$k$ pravokotna na vsako od krožnic $k_1$, $k_2$ in $k_3$.
 \kdokaz

         \bzgled Po dve od štirih krožnic se od zunaj dotikata v
        točkah $A$, $B$, $C$ in $D$. Dokaži, da so točke $A$, $B$, $C$
        in $D$ bodisi konciklične bodisi kolinearne.
         \ezgled


\begin{figure}[!htb]
\centering
\input{sl.inv.9.3.3c.pic}
\caption{} \label{sl.inv.9.3.3c.pic}
\end{figure}

\textbf{\textit{Proof.}}  Označimo s $k$ krožnico (oz. premico),
ki poteka skozi točke $B$, $C$ in $D$. Naj bodo $k_1$, $k_2$, $k_3$
in $k_4$ takšne krožnice, da se $k_1$ in $k_2$ dotikata v točki
$A$, $k_1$ in $k_4$ v točki $B$, $k_3$ in $k_4$ v točki $C$ ter
$k_3$ in $k_2$ v točki $D$. Označimo s $\psi_i$ inverzijo glede
na poljubno krožnico $i$ s središčem $A$ (Figure
\ref{sl.inv.9.3.3c.pic}). Naj bodo $B'$, $C'$ in $D'$ slike točk
$B$, $C$ in $D$ ter $k'_1$, $k'_2$, $k'_3$ in $k'_4$ slike
krožnic  $k_1$, $k_2$, $k_3$ in $k_4$ v tej inverziji. Iz izrekov
\ref{InverzKroznVkrozn} in \ref{InverzDotik} sledi, da sta $k'_3$
in $k'_4$ krožnici, ki se dotikata v točki $C'$, $k'_1$ in
$k'_2$ pa vzporedni premici, ki sta tangenti krožnic $k'_4$ in
$k'_3$ v točkah $B'$ in $D'$. Iz vzporednosti premic  $k'_1$ in
$k'_2$ sledi
 $\angle k'_2,D'C' \cong \angle k'_1,B'C'$, zato sta po izreku
 \ref{ObodKotTang}
 skladna tudi središčna kota $D'S_2C'$ in $B'S_1C'$. Zaradi tega
 je $\angle D'C'S_2 \cong \angle B'C'S_1$, kar pomeni, da so
 točke $B'$, $C'$ in $D'$ kolinearne oz. je $k'$ premica. Po
 izreku \ref{InverzKroznVkrozn} njena slika glede na inverzijo
 $k=\psi_i(k')$ poteka skozi središče inverzije, zato velja
 $A,B,C,D\in k$.
 \kdokaz

      \bzgled Naj bodo $k$, $l$ in $j$ tri medsebojno pravokotne
      krožnice s skupnimi tetivami $AB$, $CD$ in $EF$. Dokaži da
      se očrtani krožnici trikotnikov $ACE$ in $ADF$ dotikata v
      točki $A$.
      \ezgled


\begin{figure}[!htb]
\centering
\input{sl.inv.9.3.3a.pic}
\caption{} \label{sl.inv.9.3.3a.pic}
\end{figure}

\textbf{\textit{Proof.}} Naj bo $AB$ skupna tetiva krožnic $k$ in
$l$,  $CD$ skupna tetiva krožnic $l$ in $j$ ter  $EF$ skupna
tetiva krožnic $j$ in $k$ (Figure \ref{sl.inv.9.3.3a.pic}).
Označimo z $x$ in $y$ očrtani krožnici trikotnikov $ACE$ in
$ADF$ in s  $\psi_i$ inverzijo glede na poljubno krožnico $i$ s
središčem $A$. Naj bo:
  \begin{eqnarray*}
  && \psi_i:\hspace*{1mm}B,C,D,E,F \mapsto B',C',D',E',F' \hspace*{2mm}
  \textrm{ in
  }\\
  && \psi_i:\hspace*{1mm}k,l,j,x,y \rightarrow k',l',j',x',y'.
  \end{eqnarray*}
   Ker velja $A\in k, l, x, y$ in $A\notin j$,
    so po izreku \ref{InverzKroznVkrozn}
   $k'=E'F'$, $l'=C'D'$, $x'=E'C'$ in
   $y'=D'F'$ premice, $j'\ni C',D',E',F'$ pa krožnica. Iz
   medsebojne pravokotnosti krožnic $k$, $l$ in $j$ ter
    dejstva $B\in k\cap l$ sledi, da sta premici $k'$ in $l'$
    pravokotni v točki $B'$, krožnica $j'$ pa pravokotna na
    obe premici $k'$ in $l'$ (izrek \ref{InverzKonfPrav}). To
    pomeni, da sta $C'D'$ in $E'F'$ pravokotna premera krožnice
    $j'$, zato je štirikotnik $E'C'F'D'$ kvadrat in je $E'C' \parallel
    D'F'$. Iz vzporednosti premic $x'$ in $y'$ sledi, da se njuni
    inverzni sliki (krožnici $x$ in $y$) dotikata v središču
    inverzije $A$.
 \kdokaz

        \bzgled Naj bosta $k$ in $l$ krožnici neke ravnine s središčema
        $O$ in $S$. Naj bodo $t_i$ tangente krožnice $k$,
       ki sekajo krožnico $l$ v točkah $A_i$ in $B_i$. Dokaži, da
       obstaja krožnica, ki se dotika vseh očrtanih krožnic trikotnikov
       $SA_iB_i$.
        \ezgled

\begin{figure}[!htb]
\centering
\input{sl.inv.9.3.4.pic}
\caption{} \label{sl.inv.9.3.4.pic}
\end{figure}

\textbf{\textit{Proof.}} Naj bo $\psi_l$ inverzija glede na
krožnico $l$ (Figure \ref{sl.inv.9.3.4.pic}). Premice $t_i$ se s
to inverzijo preslikajo v krožnice $t'_i$, ki so očrtane
krožnice trikotnikov $SA_iB_i$ (izrek \ref{InverzKroznVkrozn}).
Ker se premice $t_i$ dotikajo krožnice $k$, se vse krožnice $t'_i$
dotikajo krožnice $k'=\psi_l(k)$  (izrek \ref{InverzDotik}).
 \kdokaz

      \bzgled
      Naj bo $ST$ premer krožnice $k$, $t$ tangenta te krožnice v
      točki $T$ ter $PQ$ in $PR$ njeni tangenti  v točkah $Q$ in $R$.
      Dokaži, da če so $L$, $Q'$ in $R'$ presečišča poltrakov $SP$, $SQ$ in
      $SR$ s premico $t$, je točka $L$ središče daljice
      $Q'R'$.
     \ezgled

\begin{figure}[!htb]
\centering
\input{sl.inv.9.3.5.pic}
\caption{} \label{sl.inv.9.3.5.pic}
\end{figure}

\textbf{\textit{Proof.}} Naj bo $\psi_i$ inverzija glede na
krožnico $i(S,ST)$ (Figure \ref{sl.inv.9.3.5.pic}). Točka $T$ je
pravokotna projekcija središča inverzije $S$ na premici $t$. Ker
je še $\psi_i(T)=T$, iz izreka \ref{InverzKroznVkrozn} sledi
$\psi_i(t)=k$ oz. $k'=\psi_i(k)=t$. Zato je tudi $\psi_i(Q)=Q'$
in $\psi_i(R)=R'$. Označimo $P'=\psi_i(P)$. Iz $L\in SP$ sledi,
da so točke $S$, $P$, $L$ in $P'$ kolinearne. Tangenti $q=PQ$ in
$r=PR$ krožnice $k$ se preslikata v krožnici $q'$ in $r'$, ki
potekata skozi točko $S$ in se dotikata premice $k'$ v točkah $Q'$ in $R'$ (izreka
\ref{InverzKroznVkrozn} in \ref{InverzDotik}). Torej je premica
$t=k'$ skupna tangenta krožnic $q'$ in $r'$, ki se sekata v
točkah $S$ in $P'$. Točka $L$ pa leži na njuni potenčni osi
$SP'$, zato je
 $|LQ'|^2=|LR'|^2$, torej je točka $L$ središče daljice $Q'R'$.
\kdokaz


       \bzgled Naj bodo $a$, $b$ in $c_0$ krožnice
        s premeri $PQ$, $PR$ in $RQ$, kjer je $\mathcal{B}(P,R,Q)$.
        Naj bo $c_0$, $c_1$, $c_2$,
       ... $c_n$, ... zaporedje krožnic na istem bregu premice $PQ$, ki
       se dotikajo krožnic $a$, $b$ in se vsaka krožnica v zaporedju
       dotika prejšnje. Dokaži, da je oddaljenost središča krožnice
       $c_n$ od premice $PQ$ $n$-krat večja od premera te
       krožnice\footnote{Ta problem
        je obravnaval \index{Pappus} \textit{Pappus iz Aleksandrije} (4. st.).
         Vzorec, ki nastane s
       polkrožnicami $a$, $b$, $c_0$, je raziskoval že \index{Arhimed}
        \textit{Arhimed
       iz Sirakuze} (3. st. pr. n. š.) - glej zgled \ref{arbelos}.}\\
       (primer iz knjige \cite{Cofman}).
        \ezgled

\begin{figure}[!htb]
\centering
\input{sl.inv.9.3.6.pic}
\caption{} \label{sl.inv.9.3.6.pic}
\end{figure}

\textbf{\textit{Proof.}} (Figure \ref{sl.inv.9.3.6.pic}). Naj bo
$i$ krožnica s središčem v točki $P$, ki je pravokotna na
krožnico $c_n$ (krožnica $i$ poteka skozi dotikališči tangent iz
točke $P$ na krožnico $c_n$). Z inverzijo $\psi_i$ glede na to
krožnico se premica $l=PQ$ in krožnici $a$ in $b$ preslikajo v
premice $l$, $a'$ in $b'$ (izrek \ref{InverzKroznVkrozn}); pri tem
sta premici $a'$ in $b'$ pravokotni na premico $l$ (izrek
\ref{InverzKonfPrav}). Krožnice $c_0$, $c_1$, $c_2$, ..., $c_n$,
... se z isto inverzijo preslikajo v skladne krožnice $c'_0$,
$c'_1$, $c'_2$, ..., $c'_n$, ... , ki se vse dotikajo vzporednic
 $a'$, $b'$ (izrek \ref{InverzDotik})
   in  so vse skladne krožnici $c_n$, ker je
 $c'_n=\psi_i(c_n)=c_n$ (izrek \ref{InverzKroznFiks}).
Ker je tudi $c_0'\perp l$, središče krožnice $c'_0$ leži na
premici $l$, zato je oddaljenost središča krožnice $c_n=c'_n$
od te premice $n$-krat večje od premera te krožnice.
 \kdokaz



        \bzgled
        Naj bo $t$ skupna zunanja tangentna krožnica
         krožnic $k$ in $l$, ki se od zunaj dotikata v
       točki $A$, ter $c_0$, $c_1$, $c_2$, ... $c_n$, ... zaporedje
       krožnic, ki se dotikajo krožnic $k$ in $l$, vsaka krožnica v
       zaporedju se dotika prejšnje, krožnica $c_0$ pa se dotika tudi
       krožnice $t$. Dokaži, da obstaja krožnica (ali premica), ki je
       pravokotna na vsako krožnico iz danega zaporedja.
        \ezgled


\begin{figure}[!htb]
\centering
\input{sl.inv.9.3.8.pic}
\caption{} \label{sl.inv.9.3.8.pic}
\end{figure}

\textbf{\textit{Proof.}} Naj bo $i$ poljubna krožnica s središčem
$A$ in $\psi_i$ inverzija glede na to krožnico (Figure
\ref{sl.inv.9.3.8.pic}). Krožnici $k$ in $l$ se pri tej inverziji
preslikata v premici $k'$ in $l'$ (izrek \ref{InverzKroznVkrozn}), ki
sta vzporedni, ker se krožnici $k$ in $l$ dotikata v središču
inverzije $A$. Krožnice iz zaporedja $c_0$, $c_1$, $c_2$, ...,
$c_n$, ... pa se preslikajo v krožnice $c'_0$, $c'_1$, $c'_2$, ...,
$c'_n$, ..., kjer se vsaki dve sosednji dotikata in se vse krožnice
dotikajo vzporednic $k'$ in $l'$ (izrek \ref{InverzDotik}). Zato so
vse krožnice iz tega
 zaporedja medsebojno skladne. Premica $n'$, ki jo določajo
 njihova
središča, je somernica premic $k'$ in $l'$. Premica $n'$
 je pravokotna na krožnice $c'_0$, $c'_1$, $c'_2$, ..., $c'_n$, ...,
  zato je njena slika $n=\psi_i(n')$ pravokotna na krožnice
   $c_0$, $c_1$, $c_2$, ..., $c_n$, ... (izrek \ref{InverzKonf}).
   Na koncu lahko ugotovimo, da je $n$  krožnica natanko tedaj,
    ko premica $n'$ ne
gre skozi točko $A$ oz. kadar krožnici $k$ in $l$ nista skladni,
sicer je  $n$ premica. \kdokaz


    \bzgled Naj bodo $A$, $B$, $C$ in $D$ štiri poljubne komplanarne
    točke. Dokaži, da je kot, pod katerim se sekata očrtani
    krožnici trikotnikov $ABC$ in $ABD$, enak kotu, pod katerim se
    sekata očrtani
    krožnici trikotnikov $ACD$ in $BCD$.
    \ezgled


\begin{figure}[!htb]
\centering
\input{sl.inv.9.3.9.pic}
\caption{} \label{sl.inv.9.3.9pic}
\end{figure}

\textbf{\textit{Proof.}} Označimo s $k_A$, $k_B$, $k_C$ in $k_D$
očrtane krožnice trikotnikov $BCD$, $ACD$, $ABD$ in $ABC$ (Figure
\ref{sl.inv.9.3.9pic}). Z inverzijo  $\psi_i$ glede na poljubno
krožnico $i$ s središčem $A$ se naša trditev prevede v
ekvivalenten izrek \ref{ObodKotTang}.
 \kdokaz



    \bzgled Naj bodo $A$, $B$ in $C$ točke, ki ležijo na premici
    $l$,
    in $P$ točka izven te premice. Dokaži, da so točka $P$ in
    središča očrtanih krožnic trikotnikov $APB$, $BPC$ in $APC$
    štiri konciklične točke.
    \ezgled


\begin{figure}[!htb]
\centering
\input{sl.inv.9.3.10.pic}
\caption{} \label{sl.inv.9.3.10pic}
\end{figure}

\textbf{\textit{Proof.}} Označimo s $K_A$, $K_B$, $K_C$
središča očrtanih krožnic trikotnikov $BPC$, $APC$ in $APB$
(Figure \ref{sl.inv.9.3.10pic}). Če je $X=\mathcal{S}_{K_A}(P)$,
$Y=\mathcal{S}_{K_B}(P)$ in $Z=\mathcal{S}_{K_C}(P)$ oz.
$h_{P,2}(K_A)=X$, $h_{P,2}(K_B)=Y$ in $h_{P,2}(K_C)=Z$, iz
koncikličnosti točk $X$, $Y$, $Z$ in $P$ sledi koncikličnost
točk $K_A$, $K_B$, $K_C$ in $P$ (iz $X,Y,Z,P\in k$ sledi
$K_A,K_B,K_C,P\in h^{-1}_{P,2}$). Dokažimo torej, da so točke
$X$, $Y$, $Z$ in $P$ konciklične.

Naj bo $\psi_{P,r}$ inverzija s središčem $P$ in poljubnim
polmerom $r$ ter
 $$\psi_{P,r}:\hspace*{1mm} A,B,C,X,Y,Z\mapsto A',B',C',X',Y',Z'.$$
Inverzija $\psi_{P,r}$ preslika očrtane krožnice trikotnikov $BPC$,
$APC$ in $APB$ v premice $B'C'$, $A'C'$ in $A'B'$ (izrek
\ref{InverzKroznVkrozn}). Ker so $PX$, $PY$ in $PZ$ premeri teh
krožnic, so točke $X'$, $Y'$ in $Z'$ pravokotne projekcije središča
inverzije $P$ na premice $B'C'$, $A'C'$ in $A'B'$. Premica $l$ se
po izreku \ref{InverzKroznVkrozn} preslika v očrtano krožnico
trikotnika $A'B'C'$, ki gre tudi skozi točko $P$. Po izreku
\ref{SimpsPrem}  točke $X'$, $Y'$ in $Z'$ ležijo na
\index{premica!Simsonova} Simsonovi premici $s$, kar pomeni (izrek
\ref{InverzKroznVkrozn}), da točke $X$, $Y$ in $Z$ ležijo na
krožnici $\psi_{P,r}(s)$, ki poteka skozi točko $P$, zato so točke
 $X$, $Y$, $Z$ in $P$ konciklične.
 \kdokaz

%________________________________________________________________________________
 \poglavje{Metric Properties of Inversion} \label{odd9MetrInv}

Iz same definicije je jasno, da inverzija ni izometrija. V
prejšnjem poglavju smo ugotovili, da inverzija ne ohranja
relacije kolinearnosti točk, kar pomeni, da ni niti
transformacija podobnosti. Vendar nas zanima, kako se spreminja
razdalja med točkami, čeprav se z inverzijo daljica $AB$ v splošnem
primeru  preslika v lok $A'B'$. Odgovor bo dal
naslednji pomemben izrek.

 \bizrek \label{invMetr} Če sta $A'$ in $B'$ ($A',B'\neq S$) sliki točk $A$ in
 $B$ pri inverziji $\psi_{S,r}$, potem velja:
  $$|A'B'|=\frac{r^2\cdot |AB|}{|SA|\cdot |SB|}$$
  \eizrek

\begin{figure}[!htb]
\centering
\input{sl.inv.9.4.1.pic}
\caption{} \label{sl.inv.9.4.1.pic}
\end{figure}

 \textbf{\textit{Proof.}} Obravnavali bomo dva možna
primera.

\textit{(i)} Naj bodo točke $S$, $A$ in $B$ nekolinearne (Figure
\ref{sl.inv.9.4.1.pic}). Po izreku \ref{invPodTrik} sta trikotnika
$ASB$ in $B'SA'$ podobna in je $A'B':AB=SB':SA$. Ker je $B'$
slika točke $B$ pri inverziji $\psi_{S,r}$, velja tudi $|SB|\cdot
|SB'|=r^2$ oz. $|SB'|=\frac{r^2}{|SB|}$. Iz teh dveh relacij
sledi:
 $$|A'B'|=\frac{|SB'|\cdot |AB|}{|SA|}=
 \frac{r^2\cdot |AB|}{|SA|\cdot |SB|}.$$


\textit{(ii)} Naj bodo $S$, $A$ in $B$ kolinearne točke. Brez
škode za splošnost predpostavimo, da velja $\mathcal{B}(S,A,B)$.
Tedaj je $\mathcal{B}(S,B',A')$ (izrek \ref{invUrejenost}), zato
velja:
 $$|A'B'|=|SA'|-|SB'|=
\frac{r^2}{|SA|}-\frac{r^2}{|SB|}= \frac{(|SB|- |SA|)\cdot
r^2}{|SA|\cdot |SB|}
 = \frac{r^2\cdot |AB|}{|SA|\cdot |SB|} ,$$ kar je bilo treba dokazati. \kdokaz

 Iz prejšnjega izreka vidimo, da razdalja med slikama točk $A'$ in $B'$
   narašča, če se vsaj ena izmed originalov $A$ in $B$
 približa središču inverzije $S$, kar je logično, kajti videli
 smo, da je intuitivno slika tega središča točka v neskončnosti.

 V razdelku \ref{odd7Ptolomej} smo že dokazali Ptolomejev \index{izrek!Ptolomejev splošni} izrek (\ref{izrekPtolomej}), ki se nanaša na
  tetivne
 štirikotnike. Sedaj bomo to trditev posplošili.

\bizrek
 Če je $ABCD$ poljubni konveksni štirikotnik, potem je:
 $$|AB|\cdot |CD|+|BC|\cdot |AD|\geq |AC|\cdot |BD|.$$
  Enakost velja natanko tedaj, ko je $ABCD$ tetivni
štirikotnik.
 \eizrek

\begin{figure}[!htb]
\centering
\input{sl.inv.9.4.2a.pic}
\caption{} \label{sl.inv.9.4.2a.pic}
\end{figure}

 \textbf{\textit{Proof.}} Naj bo $\psi_{A,r}$ inverzija  s središčem v
točki $A$ (Figure \ref{sl.inv.9.4.2a.pic}). S $k$ označimo očrtano
krožnico trikotnika $ABD$. Naj bodo $B'$, $C'$ in $D'$ slike
točk $B$, $C$ in $D$ ter $k'$ slika krožnice $k$ pri inverziji
$\psi_{A,r}$. Po izreku \ref{InverzKroznVkrozn} je $k'$ premica,
ki vsebuje točke $B'$ in $D'$.
 Iz prejšnjega izreka \ref{invMetr} sledi:
 $$|B'C'|=\frac{r^2\cdot |BC|}{|AB|\cdot |AC|}, \hspace*{2mm}
  |C'D'|=\frac{r^2\cdot |CD|}{|AC|\cdot |AD|} \hspace*{1mm} \textrm{ in }
 \hspace*{1mm} |B'D'|=\frac{r^2\cdot |BD|}{|AB|\cdot |AD|}.$$
  Za točke $B'$, $C'$ in $D'$ velja trikotniška neenakost
  \ref{neenaktrik}:
   $$B'C'+C'D'\geq B'D',$$
kjer enakost velja natanko tedaj, ko so točke $B'$, $C'$ in $D'$
kolinearne (in $\mathcal{B}(B',C',D')$) oziroma ko $C'\in k'$ (in
$\mathcal{B}(B',C',D')$). To pa velja natanko tedaj, ko krožnica $k$
vsebuje točko $C$ oziroma ko je štirikotnik $ABCD$ tetiven (in
konveksen).
 Prejšnjo neenakost
lahko zapišemo tudi v obliki:
 $$\frac{r^2\cdot |BC|}{|AB|\cdot |AC|}+\frac{r^2\cdot |CD|}{|AC|\cdot |AD|}
  \geq\frac{r^2\cdot |BD|}{|AB|\cdot |AD|}, \hspace*{2mm} \textrm{oz.}$$
 $$|AB|\cdot |CD|+|BC|\cdot |AD|\geq |AC|\cdot |BD|,$$ kar je bilo treba dokazati. \kdokaz

 Naslednja trditev bo posplošitev zgleda \ref{zgledTrikABCocrkrozP} oz.
 \ref{zgledTrikABCocrkrozPPtol}.

\bzgled \label{zgledABCPinv} Naj bo $k$ očrtana krožnica pravilnega
trikotnika $ABC$. Če je $P$ poljubna točka v ravnini tega
trikotnika, tedaj velja ekvivalenca: točka $P$ ne leži na krožnici $k$,
natanko tedaj, ko obstaja trikotnik s stranicami, ki so skladne z
daljicami $PA$, $PB$ in $PC$.
 \ezgled

\begin{figure}[!htb]
\centering
\input{sl.inv.9.4.3.pic}
\caption{} \label{sl.inv.9.4.3.pic}
\end{figure}

 \textbf{\textit{Proof.}} Naj bo $\psi_{P,r}$ inverzija  s središčem v
točki $P$ (Figure \ref{sl.inv.9.4.3.pic}). Naj bodo $A'$, $B'$ in
$C'$ slike točk $A$, $B$ in $C$ ter $k'$ slika krožnice $k$ pri
inverziji $\psi_{P,r}$.
 Iz izreka \ref{invMetr} sledi:
 $$|A'B'|=\frac{r^2\cdot |AB|}{|PA|\cdot |PB|}, \hspace*{2mm}
  |B'C'|=\frac{r^2\cdot |BC|}{|PB|\cdot |PC|} \hspace*{1mm} \textrm{ in }
 \hspace*{1mm} |A'C'|=\frac{r^2\cdot |AC|}{|PA|\cdot |PC|}.$$
Ker je trikotnik $ABC$ pravilen, iz prejšnjih treh relacij sledi:
$$A'B':B'C':A'C'=PC:PA:PB.$$

Sedaj lahko začnemo z dokazovanjem iskane ekvivalence:
 \begin{eqnarray*}
 P \notin k & \Leftrightarrow& \psi_{P,r}(k) \textrm{ predstavlja krožnico
 \hspace*{2mm}(izrek \ref{InverzKroznVkrozn})}\\
 & \Leftrightarrow& \textrm{točke }A', B', C' \textrm{ so nekolinearne}\\
 & \Leftrightarrow& \textrm{daljice }A'B', B'C', A'C' \textrm{ so stranice
  nekega trikotnika}\\
 & \Leftrightarrow& \textrm{daljice }PC, PA, PB \textrm{ so stranice
  nekega trikotnika,}
 \end{eqnarray*}
  kar je bilo treba dokazati. \kdokaz

  Še ena posplošitev bo povezana s Torricellijevo točko (izrek
  \ref{izrekTorichelijev}).

   \bizrek \label{izrekToricheliFerma}
    Točka, za katero je vsota razdalj od
 oglišč nekega trikotnika minimalna, je
Torricellijeva\footnote{\index{Torricelli, E.} \textit{E.
Torricelli} (1608--1647), italijanski matematik in fizik. Lastnost iz
izreka je dokazal francoski matematik \index{Fermat, P.} \textit{P.
Fermat} (1601--1665). Zaradi tega se ta točka imenuje tudi
\index{točka!Torricellijeva}\index{točka!Fermatova}\pojem{Fermatova
točka}.} točka tega trikotnika.
 \eizrek

\begin{figure}[!htb]
\centering
\input{sl.inv.9.4.4.pic}
\caption{} \label{sl.inv.9.4.4.pic}
\end{figure}

 \textbf{\textit{Proof.}} Uporabimo enake oznake kot pri izreku
 \ref{izrekTorichelijev} (Figure \ref{sl.inv.9.4.4.pic}). Dokazali
 smo,
da se krožnice $k$, $l$ in $j$ sekajo v Torricellijevi točki $P$
trikotnika $ABC$. Ker je $BEC$ pravilni trikotnik, iz izreka
\ref{zgledTrikABCocrkrozP} sledi $|PB|+|PC|=|PE|$ oz.:
 $$|PA|+|PB|+|PC|=|AE|.$$
 Dokažimo, da je ta vsota minimalna ravno za Torricellijevo točko
 $P$. Naj bo $P'\neq P$. Točka $P'$ potem
ne leži na nobeni od krožnic $k$, $l$ in $j$. Brez škode za
splošnost naj bo $P' \notin k$. Iz prejšnje trditve
\ref{zgledABCPinv} sledi, da obstaja trikotnik s stranicami $P'E$,
$P'B$ in $P'C$. Zaradi tega in trikotniške neenakosti (izrek
\ref{neenaktrik}) je $|P'B| + |P'C|
> |P'E|$ oziroma:
 $$|P'A| + |P'B| + |P'C| > |P'A| + |P'E| \geq |AE|=|PA|+|PB|+|PC|,$$ kar je bilo treba dokazati. \kdokaz




              \bnaloga\footnote{37. IMO, India - 1996, Problem 2.}
                Let $P$ be a point inside triangle $ABC$ such that
             $$\angle APB -\angle ACB = \angle APC -\angle ABC.$$
         Let $D$, $E$ be the incentres of triangles $APB$, $APC$, respectively. Show
that $AP$, $BD$, $CE$ meet at a point.
        \enaloga

\begin{figure}[!htb]
\centering
\input{sl.inv.9.4.5.pic}
\caption{} \label{sl.inv.9.4.5.pic}
\end{figure}

 \textbf{\textit{Proof.}}
Naj bo $r>\max\{|AB|, |AC|\}$ poljuben in $\psi_{A,r}$ inverzija s
središčem v točki $A$ in polmerom $r$ ter $B'$, $C'$ in $P'$
slike točk $B$, $C$ in $P$ pri inverziji $\psi_{A,r}$ (Figure
\ref{sl.inv.9.4.5.pic}).

Po izreku \ref{invPodTrik} sta trikotnika $ABC$ in $AC'B'$
podobna, zato je:
 $$\angle AC'B'\cong \angle ABC \textrm{ in } \angle
AB'C'\cong \angle ACB.$$
 Iz podobnosti $\triangle ABP \sim \triangle
AP'B'$ in $\triangle ACP \sim \triangle AP'C'$ pa sledi:
 $$\angle AB'P'\cong \angle APB \textrm{ in } \angle
AC'P'\cong \angle APC.$$
 Če uporabimo dobljene štiri relacije skladnosti kotov in
 začetni pogoj iz naloge ($\angle APB -\angle ACB = \angle APC -\angle
 ABC$), dobimo:
  \begin{eqnarray*}
   \angle C'B'P' &=& \angle AB'P'-\angle AB'C' = \angle APB - \angle
   ACB = \\ &=& \angle APC -\angle ABC = \angle AC'P' -\angle AC'B'
   =\\
   &=& \angle B'C'P'
  \end{eqnarray*}
Sedaj iz $\angle C'B'P'\cong \angle B'C'P'$ sledi $P'B'\cong
P'C'$.  Če uporabimo relacijo iz izreka \ref{invMetr}, dobimo
 $\frac{|PB| \cdot r^2}{|AP|\cdot |AB|}=\frac{|PC| \cdot r^2}{|AP|\cdot
 |AC|}$ oz.:
 $$\frac{|PB| }{ |AB|}=\frac{|PC| }{ |AC|}.$$
 Naj bo $BD \cap AP =X$ in $CE \cap AP =Y$. Potrebno je še
 dokazati, da velja $X=Y$. Ker sta $D$ in $E$ središči
 včrtanih krožnic trikotnikov $APB$
in $APC$, sta premici $BD$ in $CE$ simetrali (notranjih) kotov
$ABP$ in $ACP$. Zato je (izrek \ref{HarmCetSimKota})
 $$ \frac{\overrightarrow{PX}}{\overrightarrow{XA}}=\frac{PB}{BA}
 =\frac{PC}{CA}=\frac{\overrightarrow{PY}}{\overrightarrow{YA}}.$$
 Iz $ \frac{\overrightarrow{PX}}{\overrightarrow{XA}}
 =\frac{\overrightarrow{PY}}{\overrightarrow{YA}}$ sledi (izrek
 \ref{izrekEnaDelitevDaljice}) $X=Y$, kar pomeni, da se premice $AP$, $BD$ in $CE$ sekajo v isti
točki $X=Y$.
 \kdokaz

%________________________________________________________________________________
\poglavje{Inversion and Harmonic Conjugate Points}
\label{odd9InvHarm}

V razdelku \ref{odd7Harm} smo videli, da harmonično četverico točk
definiramo na dva ekvivalentna načina. Za štiri kolinearne točke
$A$, $B$, $C$ in $D$ je $\mathcal{H}(A,B;C,D)$, če je izpolnjen en
izmed dveh (ekvivalentnih) pogojev:
\begin{itemize}
  \item $\frac{\overrightarrow{AC}}{\overrightarrow{CB}}=-
  \frac{\overrightarrow{AD}}{\overrightarrow{DB}}$,
  \item obstaja tak štirikotnik $ABCD$, da velja
  $PQ \cap RS=A$, $PS \cap QR=B$, $PR \cap AB=C$ in $QS \cap AB=D$.
\end{itemize}


Sedaj bomo raziskovali še nekatere možnosti za ekvivalentne definicije
 relacije harmonične četverice točk - najprej s pomočjo inverzije.

 \bizrek \label{invHarm} Naj bo $AB$ premer krožnice $i$ ter $C$ in $D$ točki
 premice $AB$, ki sta različni od $A$ in $B$. Če je $\psi_i$
 inverzija glede na krožnico $i$, potem velja:
 $$\mathcal{H}(A,B;C,D) \hspace*{1mm} \Leftrightarrow \hspace*{1mm}
 \psi_i(C)=D.$$
 \eizrek
\textbf{\textit{Proof.}} Naj bo $O$ središče daljice $AB$ (Figure
\ref{sl.inv.9.6.1.pic}). Tedaj je:
 \begin{eqnarray*}
 \mathcal{H}(A,B;C,D) &\hspace*{1mm} \Leftrightarrow
 \hspace*{1mm}&
  \frac{\overrightarrow{AC}}{\overrightarrow{CB}}=-
  \frac{\overrightarrow{AD}}{\overrightarrow{DB}}\\
 &\hspace*{1mm} \Leftrightarrow
 \hspace*{1mm}&
  \frac{\overrightarrow{OC}-\overrightarrow{OA}}
  {\overrightarrow{OB}-\overrightarrow{OC}}=-
  \frac{\overrightarrow{OD}-\overrightarrow{OA}}
  {\overrightarrow{OB}-\overrightarrow{OD}}\\
 &\hspace*{1mm} \Leftrightarrow
 \hspace*{1mm}&
  \frac{\overrightarrow{OC}+\overrightarrow{OB}}
  {\overrightarrow{OB}-\overrightarrow{OC}}=-
  \frac{\overrightarrow{OD}+\overrightarrow{OB}}
  {\overrightarrow{OB}-\overrightarrow{OD}}\\
 &\hspace*{1mm} \Leftrightarrow
 \hspace*{1mm}& \overrightarrow{OC} \cdot \overrightarrow{OD} =
 OB^2\\
 &\hspace*{1mm} \Leftrightarrow
 \hspace*{1mm}& \psi_i(C)=D,
  \end{eqnarray*}
  kar je bilo treba dokazati.  \kdokaz

\begin{figure}[!htb]
\centering
\input{sl.inv.9.6.1.pic}
\caption{} \label{sl.inv.9.6.1.pic}
\end{figure}

Naslednji izrek bo dal četrto možnost ekvivalentne definicije
 relacije harmonične četverice točk.

 \bizrek \label{harmPravKrozn}
 Naj bodo $A$, $B$, $C$ in $D$ različne kolinearne točke ter $k$ in $l$
  krožnici nad premeroma $AB$ in
$CD$. Tedaj velja ekvivalenca:
$$\mathcal{H}(A,B;C,D) \Leftrightarrow k\perp l.$$
 \eizrek

 \textbf{\textit{Proof.}} Naj bosta $O$ in $S$ središči krožnic $k$ in $l$
 (Figure \ref{sl.inv.9.6.1.pic}). Iz obeh strani ekvivalence
 sledi, da se krožnici
sekata. Eno njuno presečišče označimo s $T$. Iz prejšnjega
izreka \ref{invHarm} je:
 $$\mathcal{H}(A,B;C,D) \hspace*{1mm} \Leftrightarrow \hspace*{1mm}
 \psi_k(C)=D.$$
 Torej moramo dokazati le še:
$$k\perp l \hspace*{1mm} \Leftrightarrow \hspace*{1mm}
 \psi_k(C)=D.$$
 Toda sedaj imamo:
 \begin{eqnarray*}
 k\perp l \hspace*{1mm} &\Leftrightarrow& \hspace*{1mm} OT\perp TS \hspace*{2mm}
 \textrm{ (izrek \ref{pravokotniKroznici})}\\
 &\Leftrightarrow& \hspace*{1mm} OT \textrm{ je tangenta krožnice
 }l \hspace*{2mm}\textrm{ (izrek \ref{TangPogoj})}\\
&\Leftrightarrow& \hspace*{1mm} \overrightarrow{OC}\cdot \overrightarrow{OD}
= OT^2\hspace*{2mm}\textrm{ (izrek \ref{izrekPotenca})}\\
&\Leftrightarrow& \hspace*{1mm} \psi_k(C)=D,
 \end{eqnarray*}
 kar je bilo treba dokazati.  \kdokaz

 Iz prejšnjega izreka \ref{harmPravKrozn} direktno sledita (že prej dokazani) dejstvi, da
 iz $\mathcal{H}(A,B;C,D)$ sledi $\mathcal{H}(C,D;A,B)$ oz.
 $\mathcal{H}(B,A;C,D)$.

 Že od prej nam je znano, da za tri kolinearne točke $A$, $B$ in $C$,
  kjer točka $C$ ni središče daljice $AB$,
 obstaja ena sama točka $D$, tako da velja $\mathcal{H}(A,B;C,D)$. Torej je za tri
dane točke v harmonični četverici točk četrta enolično določena.
Eno od možnih konstrukcij te točke nam da prejšnji izrek
\ref{invHarm} ($D=\psi_k(C)$). Samo dve točki $A$ in $B$ pa nista
dovolj, da bi določili drugi par točk $C$ in $D$. Takšnih parov
$(C,D)$, za katere velja $\mathcal{H}(A,B;C,D)$, je neskončno mnogo (Figure
\ref{sl.inv.9.6.2.pic}). Za njuno določitev
 je potreben še en pogoj. Takšne pogoje bomo obravnavali v
 naslednjih dveh primerih.

\begin{figure}[!htb]
\centering
\input{sl.inv.9.6.2.pic}
\caption{} \label{sl.inv.9.6.2.pic}
\end{figure}

 \bzgled
  Dane so točke $A$, $B$ in $S$. Konstruiraj takšni točki $C$ in $D$,
    da je $S$ središče
daljice $CD$ in velja $\mathcal{H}(A,B;C,D)$.
 \ezgled

 \textbf{\textit{Solution.}}
Naj bo $k$ krožnica nad premerom $AB$. Če je $l$ krožnica nad
premerom $CD$, je $S$ središče te krožnice. Iz izreka
\ref{harmPravKrozn} sledi $k \perp l$ (Figure
\ref{sl.inv.9.6.3.pic}). Torej je dovolj načrtati krožnico $l$, ker
sta potem točki $C$ in $D$ presečišči te krožnice s premico $AB$. Krožnico $l$ pa lahko načrtamo, če najprej konstruiramo tangento
krožnice $k$ iz točke $S$ v točki $T$.
 \kdokaz

\begin{figure}[!htb]
\centering
\input{sl.inv.9.6.3.pic}
\caption{} \label{sl.inv.9.6.3.pic}
\end{figure}


 \bzgled \label{harmDaljicad}
  Dani so daljica $d$ in točki $A$ in $B$. Konstruiraj takšni točki $C$
   in $D$, da velja $\mathcal{H}(A,B;C,D)$ in $CD\cong d$.
 \ezgled
  \textbf{\textit{Solution.}}

  Dovolj je konstruirati središče $S$ daljice $CD$, nato pa nadaljujemo enako
   kot v
  prejšnjem zgledu. Pravokotni trikotnik $OTS$ lahko
konstruiramo, ker sta znani obe kateti $|OT| = \frac{1}{2}\cdot
|AB|$ in $|ST| = \frac{1}{2}\cdot |CD|= \frac{1}{2}\cdot |d|$. Iz
njega pa dobimo hipotenuzo $OS$ (Figure \ref{sl.inv.9.6.3.pic}).
 \kdokaz

V naslednjih primerih si bomo ogledali uporabo prejšnjih dveh
konstrukcij.



\bzgled
  Načrtaj trikotnik s podatki $v_a$, $l_a$ in $b-c$.
 \ezgled

\textbf{\textit{Solution.}}
 Uporabili bomo oznake iz velike naloge \ref{velikaNaloga}
(Figure \ref{sl.inv.9.6.4.pic}). Najprej lahko načrtamo pravokotni
trikotnik $AA'E$, ker je $AA'\cong v_a$ in $AE\cong l_a$. Po
zgledu \ref{harmVelNal} je $\mathcal{H}(A',E;P,Pa)$. Iz velike
naloge pa sledi $PP_a=b -c$, zato lahko po prejšnjem zgledu
\ref{harmDaljicad}  konstruiramo točki $P$ in $P_a$. Nato
narišemo središče $S$ včrtane krožnice trikotnika $ABC$, na
koncu  še včrtano krožnico, njuni tangenti iz točke $A$ ter
oglišči $B$ in $C$.
 \kdokaz

\begin{figure}[!htb]
\centering
\input{sl.inv.9.6.4.pic}
\caption{} \label{sl.inv.9.6.4.pic}
\end{figure}



\bzgled
  Načrtaj trikotnik s podatki $v_a$, $a$ in $r+r_a$.
 \ezgled

\textbf{\textit{Solution.}} Tudi v tem primeru bomo uporabili
oznake iz velike naloge \ref{velikaNaloga} (Figure
\ref{sl.inv.9.6.4.pic}). Po zgledu \ref{harmVelNal} je
$\mathcal{H}(A,L;A',La)$. Ker je $AA'\cong v_a$ in $LL_a =r+r_a$,
lahko po zgledu \ref{harmDaljicad} narišemo najprej daljico
$AA'\cong v_a$, nato pa še točki $L$ in $L_a$. Tako dobimo
$r\cong LA'$ in $r_a\cong L_aA'$.

Iz velike naloge sledi $RR_a = a$. To pomeni, da lahko narišemo
pravokotni trapez $SRR_aS_a$ ($RR_a=a$, $SR=r$ in $S_aR_a=r_a$).
Nato konstruiramo včrtano krožnico $k(S,r)$ in pričrtano
krožnico $k_a(S_a,r_a)$ ter na koncu njune skupne tangente (dve
zunanji in še eno notranjo), ki so nosilke stranic trikotnika
$ABC$.
 \kdokaz

%________________________________________________________________________________
 \poglavje{Feuerbach Points}
\label{odd9Feuerbach}

V tem razdelku bomo obravnavali še nekatere lastnosti včrtane in
pričrtanih krožnic trikotnika.

 \bzgled Naj bo $ABC$ trikotnik s polobsegom $s$ ter $D$ in $E$ takšni
 točki
  na
 premici $BC$, da velja $|AD|=|AE|=s$. Dokaži, da se očrtana
 krožnica trikotnika $ADE$ in stranici $BC$ pričrtana krožnica
 trikotnika $ABC$ dotikata.
 \ezgled

\begin{figure}[!htb]
\centering
\input{sl.inv.9.5.0.pic}
\caption{} \label{sl.inv.9.5.0.pic}
\end{figure}

 \textbf{\textit{Proof.}} Naj bo $l$ očrtana krožnica
 trikotnika $AED$ in $P_a$, $Q_a$ ter $R_a$ točke, v katerih
se pričrtana krožnica $k_a$ trikotnika $ABC$ dotika stranice
  $BC$ ter nosilk $AC$ in $AB$ (Figure \ref{sl.inv.9.5.0.pic}). Z $i$
  označimo krožnico s središčem $A$ in polmerom $AE$ (oz.
  $AD$).
Iz  velike naloge \ref{velikaNaloga} sledi $|AR_a|=|AQ_a|=s=|AE|=|AD|$,
kar
  pomeni, da točki $R_a$ in $Q_a$ ležita na krožnici $i$.
  Po izreku \ref{pravokotniKroznici} sta krožnici $i$ in $k_a$
  pravokotni. Zato se krožnica $k_a$ pri inverziji $\psi_i$ preslika
   vase (izrek \ref{InverzKroznFiks}), premica $BC$ pa v krožnico $l$
  (izrek \ref{InverzKroznVkrozn}).
  Ker se premica $BC$ dotika
  krožnice $k_a$ v točki $P_a$, se njuni sliki (krožnici $l$ in $k_a$)
    dotikata v
  točki $T=\psi_i(P_a)$ (izrek \ref{InverzDotik}).
  \kdokaz

\bzgled \label{InvOcrtVcrt} Naj bodo $P$, $Q$ in $R$ točke, v
katerih se včrtana krožnica
 dotika stranic trikotnika $ABC$. Dokaži, da so višinska točka trikotnika
 $PQR$ in središči
 očrtane ter včrtane krožnice trikotnika $ABC$ kolinearne
 točke.
 \ezgled

  \begin{figure}[!htb]
\centering
\input{sl.inv.9.5.0a.pic}
\caption{} \label{sl.inv.9.5.0a.pic}
\end{figure}

 \textbf{\textit{Proof.}} Naj bosta $l(O,R)$ in $k(S,r)$ očrtana in
 včrtana krožnica
 trikotnika $ABC$ ter $\psi_k$ inverzija glede na krožnico $k$
 (Figure \ref{sl.inv.9.5.0a.pic}).

Označimo še $A'=SA \cap QR$, $B'=SB \cap PR$ in $C'=SC \cap PQ$.
Iz skladnosti trikotnikov $ARA'$ in $AQA'$ (izrek \textit{SAS} \ref{SKS})
sledi, da je točka $A'$ središče stranice $QR$ in velja $AS
\perp RQ$. Analogno sledi tudi, da sta točki $B'$ in $C'$
središči daljic $PR$ in $PQ$ ter velja $BS \perp PR$ in $CS
\perp PQ$.

Iz postopka konstrukcije slike točke pri inverziji sledi:
 $$\psi_k:A,B,C \mapsto A',B',C'.$$
 Torej se očrtana krožnica $l$ z inverzijo $\psi_k$ preslika  v
 očrtano krožnico trikotnika $A'B'C'$ (izrek \ref{InverzKroznVkrozn}).
 Ta pa je Eulerjeva krožnica $e_1$ trikotnika $PQR$, ker gre skozi
 središča njegovih stranic. Središče te krožnice - točka
 $E_1$ - leži na Eulerjevi premici $e_p$ trikotnika $PQR$, ki je
 določena z njegovo višinsko točko $V_1$ in s središčem
 očrtane krožnice $S$.

 Dokažimo še, da točka $O$ leži na premici $e_p$. Čeprav se z
 inverzijo  središče
 krožnice $l$ (točka $O$) ne preslika v središče krožnice
 (točko $E_1$), leži točka $E_1$ na premici $SO$. To
 pomeni tudi, da točka $O$ leži na premici $SE_1=e_p$, zato so
 točke $V_1$, $O$ in $S$ kolinearne.
  \kdokaz

 Naslednja enostavna trditev je le uvod v izrek o t. i. Feuerbachovih
 točkah.

\bizrek Naj bo $k$  včrtana krožnica trikotnika $ABC$ in $k_a$ njegovi stranici $BC$ pričrtana krožnica. Če je $A_1$
središče stranice $BC$ ter $i$ krožnica s središčem $A_1$ in
polmerom $\frac{1}{2}|b-c|$, je
$$\psi_i:k,k_a\rightarrow k, k_a.$$
 \eizrek
 \textbf{\textit{Proof.}} Naj bosta $P$ in $P_a$ dotikališči
  krožnic
$k$ in  $k_a$ z njegovo stranico $BC$ (Figure
\ref{sl.inv.9.5.1.pic}). Iz velike naloge \ref{velikaNaloga}
sledi, da točki $P$ in $P_a$ ležita na krožnici $i$. Torej sta po
izreku \ref{pravokotniKroznici} krožnici $k$ in  $k_a$
pravokotni na krožnico $i$, zato je $\psi_i:k,k_a\rightarrow k,
k_a$ (izrek \ref{InverzKroznFiks}).
 \kdokaz

\begin{figure}[!htb]
\centering
\input{sl.inv.9.5.1.pic}
\caption{} \label{sl.inv.9.5.1.pic}
\end{figure}


         \bizrek \index{krožnica!Eulerjeva}
         Eulerjeva krožnica trikotnika se dotika
        včrtane
         krožnice in vseh treh pričrtanih
        krožnic tega trikotnika\footnote{Točke dotika so t. i.
        \index{točka!Feuerbachova} \pojem{Feuerbachove točke} tega
        trikotnika. \index{Feuerbach, K. W.} \textit{K. W. Feuerbach}
        (1800--1834), nemški matematik, ki je to trditev dokazal leta 1822.}.
        \eizrek

\textbf{\textit{Proof.}}
 Uporabili bomo oznake kot pri veliki nalogi
 \ref{velikaNaloga}. Naj bo še $e$ Eulerjeva krožnica tega
 trikotnika (Figure \ref{sl.inv.9.5.1.pic}). Iz prejšnje trditve inverzija $\psi_i$ glede na
 krožnico $i$ s središčem  $A_1$ in polmerom
 $\frac{1}{2}|b-c|$ (oz. vsebuje točki $P$ in $P_a$) preslika krožnici
 $k$ in $k_a$ vsako vase. Določimo sliko Eulerjeve krožnice $e$
 pri inverziji $\psi_i$.

Krožnica $e$ vsebuje središče inverzije $A_1$, zato se z
inverzijo preslika v neko premico $e'$ (izrek
\ref{InverzKroznVkrozn}). Ostane nam še dokaz, da se
premica $e'$ dotika krožnic $k$ in $k_a$ (izrek
\ref{InverzDotik}). Najprej premica $e'$ vsebuje točko
$\psi_i(A')=E$, ki je presečišče simetrale notranjega kota pri
oglišču $A$ s stranico $BC$. (izreka \ref{harmVelNal} in
\ref{invHarm}). Toda premica $e'$ vsebuje tudi točki
$\psi_i(B_1)=B'_1$ in $\psi_i(C_1)=C'_1$. Po izreku
\ref{invPodTrik} sta si trikotnika $A_1B_1C_1$ in $A_1C'_1B'_1$
podobna. Iz tega sledi, da premica $e'$ s premico $AB$ določa kot
$ACB$. Ker premica $e'$ seka premico $BC$  v točki $E$, ki leži
na simetrali kota $BAC$, $e'$ predstavlja drugo skupno tangento
krožnic $k$ in $k_a$. Označimo s $F'$ in $F'_a$ dotikališči
premice $e'$ s krožnicama $k$ in $k_a$. Potem se krožnica $e$
dotika krožnic $k$ in $k_a$ v točkah $F=\psi_i(F')$ in
$F_a=\psi_i(F'_a)$. Na enak način dokažemo, da se $e$ dotika
tudi krožnic $k_b$ in $k_c$.
 \kdokaz


%________________________________________________________________________________
 \poglavje{Stainer's Theorem}
\label{odd9Stainer}

V tem razdelku bomo na eleganten način (s pomočjo inverzije)
dokazali Stainerjev izrek, ki je povezan s problemom obstoja
zaporedja krožnic, ki se zaporedoma ciklično dotikajo, hkrati pa se
dotikajo tudi dveh danih, od znotraj mimobežnih, krožnic. Jasno
je, da takšno zaporedje v splošnem primeru (za poljubni od znotraj
mimobežni krožnici) ne obstaja. Toda dokazali bomo, da če za
dve  krožnici $k$ in $l$ obstaja vsaj eno
takšno zaporedje, potem začetni člen tega zaporedja lahko
izberemo kot poljubno krožnico, ki se dotika krožnic $k$ in $l$.

 Najprej bomo rešili dve
pomožni nalogi.



 \bzgled \label{StainerjevLema1}
 Naj bosta $k_1(S_1,r_1)$ in $k_2(S_2,r_2)$ dve krožnici neke ravnine. Načrtaj krožnico s
središčem na premici $S_1S_2$, ki je pravokotna na obe krožnici.
 \ezgled


 \begin{figure}[!htb]
\centering
\input{sl.inv.9.7.0.pic}
\caption{} \label{sl.inv.9.7.0.pic}
\end{figure}


 \textbf{\textit{Solution.}}
 Naj bosta $k_1(S_1,r_1)$ in $k_2(S_2,r_2)$ poljubni
 krožnici. V primeru, če rešitev obstaja, središče
 iskane krožnice pripada
potenčni premici $l=p(k_1,k_2)$ dveh krožnic (Figure
\ref{sl.inv.9.7.0.pic}). To je primer, kadar je presečišče $S$
potenčne premice $l$ in centrale $S_1S_2$ zunanja točka obeh
krožnic oz. ko sta krožnici $k_1$ in $k_2$ mimobežni (od znotraj
ali od zunaj). Če iz točke $S$ načrtamo tangento na eno od krožnic
(npr. $k_1$), dobimo tudi polmer $r=ST_1$ iskane krožnice
$k(S,r)$, kjer je točka $T_1$ dotikališče tangente iz $S$ na
krožnico $k_1$. Naj bo točka $T_2$ dotikališče tangente iz $S$
na krožnico $k_2$. Ker točka $S$ leži na potenčni premici dveh
krožnic, je:
 $$|ST_1|=p(S,k_1)=p(S,k_2)=|ST_2|$$
 krožnica $k(S,r)$ pravokotna na krožnici $k_1(S_1,r_1)$ in
 $k_2(S_2,r_2)$ po izreku \ref{pravokotniKroznici}.
 \kdokaz


 %slika

 \bzgled  \label{StainerjevLema2}
 Če je $k$ krožnica v notranjosti krožnice $l$, obstaja inverzija,
  ki krožnici
$k$ in $l$ preslika v dve koncentrični krožnici.
  \ezgled


 \begin{figure}[!htb]
\centering
\input{sl.inv.9.7.1.pic}
\caption{} \label{sl.inv.9.7.1.pic}
\end{figure}


  \textbf{\textit{Proof.}}
 Naj bo $p$ premica, določena s središčema krožnic $k$ in $l$;
 hkrati je tudi pravokotna na
obe krožnici. Iz prejšnjega zgleda \ref{StainerjevLema1} sledi,
da obstaja krožnica $n$ s središčem na premici $p$, ki je
pravokotna na krožnici $k$ in $l$ (Figure
\ref{sl.inv.9.7.1.pic}).
 Ker središče krožnice $n$ leži na
premici $p$, je krožnica $n$ pravokotna tudi na premico $p$. Z $I$
in $J$ označimo presečišči krožnice $n$ s premico $p$. Naj bo $i$
poljubna krožnica s središčem $I$ in $\psi_i$ inverzija glede na to
krožnico. Ker inverzija ohranja kote, se premica $p$ in
krožnica $n$ (brez točke $I$) s $\psi_i$ preslikata v pravokotni premici $p'=p$
in $n'$, krožnici $k$ in $l$ pa v krožnici $k'$ in $l'$, ki sta
pravokotni na ti dve premici (izreka \ref{InverzKroznVkrozn} in
\ref{InverzKonf}). Torej sta krožnici $k'$ in $l'$ koncentrični.
 \kdokaz


 \bzgled
 (Stainerjev izrek \index{izrek!Steinerjev}
 \footnote{\index{Steiner, J.}
 \textit{J. Steiner} (1769--1863), švicarski matematik.}.)
 Naj bo $l$ krožnica v notranjosti krožnice $k$ in $a_1$, $a_2$,
..., $a_n$ zaporedje krožnic, ki se dotikajo krožnic $k$ in $l$,
vsaka pa se  dotika tudi sosednje krožnice v zaporedju.
 Če se dotikata še krožnici $a_n$
in $a_1$, velja ta lastnost neodvisno od izbire prve krožnice
$a_1$ tega zaporedja.
  \ezgled

 \begin{figure}[!htb]
\centering
\input{sl.inv.9.7.2.pic}
\caption{} \label{sl.inv.9.7.2.pic}
\end{figure}


 \textbf{\textit{Proof.}}
 Če uporabimo prejšnji izrek  \ref{StainerjevLema2}, se trditev
 prevede v primer, ko sta krožnici koncentrični (dotik krožnic je
  invarianta inverzije \ref{InverzDotik}). Toda v tem primeru se novo zaporedje krožnic
 enostavno dobi iz prvega z rotacijo s središčem v skupnem središču krožnic
  $\psi_i(k)$ in $\psi_i(l)$ (Figure
\ref{sl.inv.9.7.2.pic}).
 \kdokaz


%________________________________________________________________________________
\poglavje{Problem of Apollonius}
\label{odd9ApolDotik}

 Sedaj se bomo ukvarjali s t. i. \index{problem!Apolonijev}
  Apolonijevimi\footnote{Starogrški
 matematik
 \index{Apolonij}
 \textit{Apolonij iz Perge} (3.--2. st. pr. n. š.), je reševal te probleme.}
 problemi o dotiku
 krožnic, ki jih lahko na
eleganten način rešimo z uporabo inverzije. Gre za probleme, ki
so naslednje oblike:

\textit{
 \vspace*{2mm}
 Načrtaj krožnico $k$, ki izpolnjuje tri
pogoje, od katerih ima vsak eno izmed naslednjih oblik:
\vspace*{2mm}
\begin{itemize}
  \item vsebuje dano točko,
  \item se dotika dane premice,
  \item se dotika dane krožnico.
\end{itemize}}
 Jasno je, da vse točke, premice in krožnice iz omenjenih pogojev ležijo
 v isti ravnini. Nekatere
od teh problemov smo že srečali. Npr. konstruirati krožnico, ki
vsebuje dani točki in se dotika dane premice. Ni težko
ugotoviti, da obstaja deset Apolonijevih problemov. Po navadi jih
navajamo v naslednjem zaporedju:
\begin{enumerate}
  \item načrtaj krožnico, ki vsebuje tri dane točke. $(A,B,C)$,
\item načrtaj krožnico, ki vsebuje dani točki in se dotika dane
premice $(A,B,p)$,
  \item načrtaj krožnico, ki vsebuje dani točki in se dotika
  dane
krožnice $(A,B,k)$,
  \item načrtaj krožnico, ki vsebuje dano točko in se
dotika  dveh danih premic $(A,p,q)$,
  \item načrtaj krožnico, ki vsebuje dano točko ter se
  dotika dane premice in dane
krožnice $(A,p,k)$,
  \item načrtaj krožnico, ki vsebuje dano točko in
se dotika dveh danih krožnic $(A,k_1,k_2)$,
  \item načrtaj krožnico, ki se dotika treh danih premic
  $(p,q,r)$,
  \item načrtaj krožnico, ki se dotika
  dveh danih premic in dane krožnice $(p,q,k)$,
  \item načrtaj krožnico, ki se dotika dane premice in dveh danih krožnic
$(p,k_1,k_2)$,
  \item načrtaj krožnico, ki se dotika
treh danih krožnic $(k_1,k_2,k_3)$.
\end{enumerate}

 Takoj vidimo, da sta prvi in sedmi problem trivialna,
 pa tudi vse ostale probleme lahko
rešimo brez uporabe inverzije. Inverzija  pa nam da splošno
metodo za njihovo reševanje. Ta metoda temelji na dejstvu, da se v
določenem primeru krožnica z inverzijo preslika v premico (izrek
\ref{InverzKroznVkrozn}). Ilustrirali jo bomo na primeru
petega Apolonijevega problema:

\bzgled
 Načrtaj krožnico, ki vsebuje dano točko $A$ ter se
  dotika dane premice $p$ in dane
krožnice $k$.
 \ezgled


 \begin{figure}[!htb]
\centering
\input{sl.inv.9.8.1.pic}
\caption{} \label{sl.inv.9.8.1.pic}
\end{figure}


 \textbf{\textit{Solution.}}

Predpostavimo, da je $x$ krožnica, ki vsebuje točko $A$ ter se
dotika premice $p$ in krožnice $k$ (Figure
\ref{sl.inv.9.8.1.pic}). Obravnavali bomo splošni primer, tako da
točka $A$ ne leži niti na premici $p$ niti na krožnici $k$. Z
$i$ označimo krožnico s središčem $A$ in poljubnim polmerom
$r$. Naj bo $\psi_i$  inverzija glede na to krožnico ter $p'$,
$k'$ in $x'$ slike premice $p$ ter krožnic $k$ in $x$ pri tej
inverziji. Ker velja $A\notin p,k$ in $A\in x$, sta $p'$ in $k'$
krožnici, $x'$ je pa premica (izrek \ref{InverzKroznVkrozn}).
Premica $p$ in krožnica $x$ imata natanko eno skupno točko, zato
to velja tudi za sliki $p'$ in $x'$, torej je premica $x'$ tangenta
krožnice $p'$. Analogno je premica $x'$ tudi tangenta krožnice
$k'$. Torej se problem prevede na konstrukcijo premice $x'$, ki je
skupna tangenta krožnic $p'$ in $k'$ (zgled \ref{tang2ehkroz}).
Potem je $x= \psi_i^{-1}(x')= \psi_i(x')$.

Naloga ima nič, eno, dve, tri ali štiri rešitve, odvisno od
medsebojne lege krožnic $p'$ in $k'$ oz. od števila njunih
skupnih tangent.

V primeru, ko velja ali $A\in p$ ali $A\in k$, je $x'$ premica, ki
se dotika ene krožnice ($k'$ ali $p'$) in je vzporedna z eno
premico ($p'$ ali $k'$).

Če je $A\in p \cap k$, naloga nima rešitev ali pa jih je neskončno mnogo,
odvisno od tega, ali se krožnici sekata ali dotikata. V obeh primerih
je namreč premica $x'$ vzporedna z dvema premicama $p'$ in $k'$,
toda v prvem primeru se premici $p'$ in $k'$ sekata, v drugem sta pa
vzporedni.
 \kdokaz

Takoj ugotovimo, da pri reševanju tega problema ni bilo preveč
pomembno, ali sta $p$ in $k$ ravno premica in krožnica, pomembno pa je
bilo, da sta sliki $p'$ in $k'$ krožnici. Toda $p'$ in
$k'$ sta krožnici tudi v primeru, kadar sta npr. $p$ in $k$ dve
premici in $A\notin p,k$. Torej četrti in šesti Apolonijev problem
se rešita na popolnoma enak način kot peti problem, ki smo ga pravkar
rešili.

Tudi drugi in tretji problem lahko rešimo z uporabo inverzije
glede na poljubno krožnico s središčem $A$. Oba problema se
prevedeta na konstrukcijo tangente iz točke $B'$ na krožnico $p'$ (oz. $k'$).

 Problemi 8, 9 in
10 se po vrsti prevedejo na probleme 4, 5 in 6. Ideja je v tem, da
najprej načrtamo  krožnico, ki je koncentrična z iskano krožnico
in vsebuje središče ene od danih krožnic - tiste z najmanjšim
polmerom.

V nadaljevanju bomo rešili primer, ki je podoben Apolonijevim
problemom o dotiku krožnic.

\bzgled
 Dani so točka $A$, krožnici $k$ in $l$ ter kota $\alpha$ in $\beta$.
  Načrtaj
  krožnico, ki poteka skozi točko $A$,
krožnici $k$ in $l$ pa seka pod kotoma $\alpha$ in $\beta$.
 \ezgled


 \begin{figure}[!htb]
\centering
\input{sl.inv.9.8.2.pic}
\caption{} \label{sl.inv.9.8.2.pic}
\end{figure}

\textbf{\textit{Solution.}}

Predpostavimo, da je $x$ krožnica, ki vsebuje točko $A$ in s
krožnicama $k$ in $l$ določa kota $\alpha$ in $\beta$ (Figure
\ref{sl.inv.9.8.2.pic}). Spet bomo obravnavali splošni primer,
ko točka $A$ ne leži na nobeni izmed krožnic $k$ in $l$. Z
$i$ označimo krožnico s središčem $A$ in poljubnim polmerom
$r$ ter s $\psi_i$  inverzijo glede na to krožnico. Naj bodo $k'$,
$l'$ in $x'$ slike krožnic $k$, $l$ in $x$ pri tej inverziji. Ker
velja $A\notin k,l$ in $A\in x$, sta $k'$ in $l'$ krožnici, $x'$
pa je premica (izrek \ref{InverzKroznVkrozn}). Ker je inverzija
konformna preslikava (izrek \ref{InverzKonf}), premica $x'$ seka
krožnici $k'$ in $l'$ pod kotoma $\alpha$ in $\beta$.

Premico  $x'$ lahko načrtamo kot skupno tangento dveh krožnic
$k'_1$ in $l'_1$, ki sta koncentrični s krožnicama $k'$ in $l'$.
Pri risanju krožnic $k'_1$ in $l'_1$ upoštevamo dejstvo, da
premica $x'$ s krožnicama $k'$ in $l'$ določa tetivi s
središčnima kotoma $2\alpha$ in $2\beta$. Na koncu je še $x=
\psi_i^{-1}(x')= \psi_i(x')$.

Tudi v tem primeru je število rešitev odvisno od medsebojne lege
krožnic $k'_1$ in $l'_1$ oz. števila njunih skupnih tangent.

V primeru, ko velja ali $A\in k$ ali $A\in l$, je $x'$ premica, ki
s premico $k'$ (ali $l'$) določa kot  $\alpha$, s krožnico $p'$
(ali $k'$) pa kot $\beta$.

Če je $A\in p \cap k$, naloga v splošnem nima rešitev,
kajti gre za premico $x'$, ki z danima premicama $k'$ in $l'$
določa kota $\alpha$ in $\beta$. Naloga ima neskončno mnogo rešitev,
kadar premici $k'$ in $l'$ določata kot $|\beta \pm \alpha|$.
 \kdokaz


%________________________________________________________________________________
\poglavje{Constructions With Compass Alone} \label{odd9LeSestilo}

 Pri konstrukcijah v evklidski geometriji smo vedno uporabljali
 ravnilo in šestilo, kar pomeni, da smo uporabljali
 elementarne konstrukcije\index{konstrukcije!z ravnilom in šestilom} (glej razdelek \ref{elementarneKonstrukcije}).

 Toda z uporabo ravnila in šestila, oz. omenjenih elementarnih konstrukcij,  v evklidski geometriji ni možno izpeljati vseh konstrukcij. Najbolj znani so naslednji primeri\footnote{Vsa tri probleme so zastavili že Stari Grki. Kasneje je te probleme poskušalo rešiti veliko znanih matematikov kot tudi laikov. Dejstvo, da omenjenih konstrukcij ni mogoče izpeljati le z uporabo ravnila in šestila, je bilo dokazano šele v 19. stoletju, ko je francoski matematik \index{Galois, E.} \textit{E. Galois} (1811--1832) razvil \index{grupa}teorijo grup. Dokaza za trisekcijo kota in podvojitev kocke je leta 1837 prvi podal francoski matematik \index{Wantzel, P. L.}\textit{P. L. Wantzel} (1814--1848). Dejstvo nerešljivosti kvadrature kroga pa je posledica transcendentnosti števila $\pi$, kar je leta 1882 dokazal nemški matematik \index{Lindemann, C. L. F.}\textit{C. L. F. Lindemann} (1852–-1939).}:
 \begin{itemize}
   \item razdelitev danega kota z dvema poltrakoma  na tri skladne dele (\pojem{trisekcija kota})\index{trisekcija kota};
   \item konstrukcija kvadrata, ki ima enako ploščino kot dani krog (\pojem{kvadratura kroga})\index{kvadratura kroga};
   \item konstrukcija roba kocke, ki ima dvakrat tolikšno prostornino kot kocka z danim robom $a$ (\pojem{podvojitev kocke})\index{podvojitev kocke}.
 \end{itemize}

\index{konstrukcije!pravilnih $n$-kotnikov}Razen tega je znan tudi problem konstrukcije pravilnih $n$-kotnikov z ravnilom in s šestilom, ki ga ni možno rešiti za vsak $n\in\{3,4,\ldots\}$\footnote{Znani nemški matematik \index{Gauss, C. F.}\textit{C. F. Gauss} (1777--1855) je leta 1796 dokazal, da lahko z ravnilom in šestilom konstruiramo pravilen $n$-kotnik natanko tedaj, ko je $n=2^k\cdot p$, kjer je $p$ bodisi enak 1 bodisi praštevilo, ki ga lahko zapišemo v obliki $2^{2^l}+1$, za $l\in\{0,1,2,\ldots\}$ (t. i. \pojem{Fermatova števila}\normalcolor, ki sicer niso vsa prosta - npr. za $n=5$, se imenujejo po francoskem matematiku \index{Fermat, P.}\textit{P. Fermatu} (1601--1665)). Pravilni $n$-kotnik torej lahko konstruiramo z ravnilom in s šestilom, če je $n\in\{3,4,5,6,8,10,12,16,17,\ldots\}$, če pa je $n\in\{7,9,11,13,14,15,18,19,\ldots\}$,  ta konstrukcija ni možna.}.

 V projektivni geometriji, v kateri ni niti vzporednosti niti metrike,
 pri načrtovanju uporabljamo le ravnilo\index{konstrukcije!z ravnilom} oz. samo tiste
 elementarne konstrukcije,
  ki jih lahko naredimo le z ravnilom. Jasno je, da v tej geometriji ni možno
 načrtati (niti definirati) likov, kot so npr. kvadrat,
 paralelogram, krožnica, ...

 Zastavlja se vprašanje, kaj lahko konstruiramo, če uporabljamo
 le šestilo  oz. samo tiste
 elementarne konstrukcije,
  ki jih lahko naredimo samo s šestilom. Pri tem štejemo, da je
  premica načrtana, če sta načrtani dve njeni točki, ne smemo
  pa uporabiti direktne konstrukcije presečišča dveh premic (z
  ravnilom). Presenetljivo je, da na ta način - le s šestilom -
  lahko izpeljemo vse
  konstrukcije, ki jih lahko naredimo s šestilom in z
  ravnilom\footnote{Konstrukcije le s pomočjo šestila sta raziskovala
  italijanski matematik \index{Maskeroni, L.}
  \textit{L. Maskeroni} (1750--1800) (po njem
  jih imenujemo \index{konstrukcije!Maskeronijeve}\index{konstrukcije!s šestilom}
  \pojem{Maskeronijeve konstrukcije}) in že sto
let pred njim danski matematik \index{Mor, G.} \textit{G. Mor}
(1640--1697) v svoji knjigi ‘‘Danski Evklid’’ iz leta 1672. Teoretično
osnovo teh konstrukcij je podal avstrijski matematik \index{Adler,
A.} \textit{A. Adler}, ki je leta 1890 dokazal, da se vsaka
načrtovalna naloga, ki jo lahko rešimo s pomočjo ravnila in šestila, da
rešiti tudi samo z uporabo  šestila.}!

Nadaljevali bomo z naslednjimi konstrukcijami, pri katerih bomo
uporabljali le šestilo. V želji, da pokažemo, da lahko ravnilo
‘‘nadomestimo’’ s šestilom, bo glavno vlogo imela inverzija. V
dosedanji uporabi inverzije pri konstrukcijah smo najpogosteje
uporabljali dejstvo, da inverzija v določenem primeru krožnico
preslika v premico. Tako smo problem konstrukcije iskane krožnice s
pomočjo inverzije prevedli v problem konstrukcije iskane premice,
kar je v večini primerov bolj enostavno. Sedaj bomo poskusili ravno
obratno - probleme povezane s konstrukcijo premice (in uporabe
ravnila) bomo z inverzijo prevedli v probleme konstrukcije krožnice
(in uporabe šestila).

 \bzgled \label{MaskeroniNAB}
 Dani sta točki $A$ in $B$ ter $n\in \mathbb{N}$.
 Načrtaj takšno točko $P_n$, da velja
 $\overrightarrow{AP_n}=n\cdot \overrightarrow{AB}$.
 \ezgled

\begin{figure}[!htb]
\centering
\input{sl.inv.9.9.1.pic}
\caption{} \label{sl.inv.9.9.1.pic}
\end{figure}

%PAZI NA BARVO! \normalcolor

 \textbf{\textit{Solution.}} Konstrukcijo bomo izpeljali
 induktivno. Za $n=1$ je jasno $P_1=B$. Načrtajmo najprej točko
 $P_2$, za katero velja $\overrightarrow{AP_2}=2\cdot \overrightarrow{AB}$
 (Figure
\ref{sl.inv.9.9.1.pic}). Načrtajmo sedaj krožnici $k_1(A,AB)$
in $k_2(B,BA)$. Eno od njunih presečišč označimo s $Q_1$.
Načrtajmo nato krožnico $l_1(Q_1,Q_1B)$. Presečišče krožnic
$l_1$ in $k_2$, ki ni točka $A$, označimo s $Q_2$. Točko $P_2$
dobimo kot eno od presečišč krožnic $l_2(Q_2,Q_2B)$ in $k_2$ (tisto, ki
ni $Q_1$). \normalcolor Relacija $\overrightarrow{AP_2}=2\cdot
\overrightarrow{AB}$ sledi iz dejstva, da so trikotniki $ABQ_1$,
$Q_1BQ_2$ in $Q_2BP_2$ vsi pravilni.

 Predpostavimo, da smo z opisanim postopkom
 načrtali točke $P_k$ ($k\leq n-1$), za katere velja
  $\overrightarrow{AP_k}=k\cdot \overrightarrow{AB}$. Točko $P_n$
  lahko narišemo na enak način.
  Načrtamo najprej krožnico $k_n(P_{n-1},P_{n-1}P_{n-2})$. Presečišče krožnic $l_{n-1}$ in $k_n$, ki ni točka $P_{n-2}$,
označimo s $Q_n$. Točko $P_n$ dobimo kot eno od presečišč
krožnic $l_n$ in $k_n$ (tisto, ki ni $Q_{n-1}$). Trikotniki
$P_{n-2}P_{n-1}Q_{n-1}$, $Q_{n-1}P_{n-1}Q_n$ in $Q_nP_{n-1}P_n$ so
vsi pravilni. Zato je
$\overrightarrow{P_{n-1}P_n}=\overrightarrow{P_{n-2}P_{n-1}}$. Če
uporabimo še indukcijsko predpostavko
$\overrightarrow{AP_k}=k\cdot \overrightarrow{AB}$ za $k=n-1$,
dobimo $\overrightarrow{AP_n}=n\cdot \overrightarrow{AB}$, kar
pomeni, da je $P_n$ iskana točka.
 \kdokaz

 Omenili smo že pomembnost inverzije v naših konstrukcijah.
 Sedaj smo pripravljeni dokazati postopek konstrukcije slike
 točke pri inverziji le s pomočjo šestila.

 \bzgled Dani sta krožnica $i(S,r)$ in točka $X$. Načrtaj točko
 $X'=\psi_i(X)$. \label{MaskeroniInv}
 \ezgled


\begin{figure}[!htb]
\centering
\input{sl.inv.9.9.2.pic}
\caption{} \label{sl.inv.9.9.2.pic}
\end{figure}

 \textbf{\textit{Solution.}} Če $X\in i$ je trivialno $X'=X$.

  Predpostavimo najprej, da je $X$ zunanja točka krožnice
  inverzije $i$ (Figure \ref{sl.inv.9.9.2.pic}). Ker je središče $S$ njena notranja točka,
  krožnica $k(X,XS)$ seka krožnico inverzije v dveh točkah
  (izrek \ref{DedPoslKrozKroz}) - denimo $P$ in $Q$, ki ju lahko
  narišemo. Nato točko $X'$ dobimo kot drugo presečišče
  krožnic $l_P(P,PS)$ in $l_Q(Q,QS)$ (prvo je točka $S$).
  Dokažimo še, da je $X'=\psi_i(X)$. Ker je po konstrukciji $XS\cong
  XP$ in $PX'\cong PS$, je tudi $\angle PX'S\cong \angle PSX'=
   \angle PSX\cong\angle SPX$. To pomeni, da sta trikotnika $PSX'$
   in $XPS$ podobna, zato je $\frac{|PX'|}{|XS|}=\frac{|SX'|}{|PS|}$ oz.
 $|SX|\cdot |SX'|=|PX'|\cdot |PS|=r^2$, torej $X'=\psi_i(X)$.

 Če je $X$ notranja točka  krožnice
  inverzije, potem obstaja takšno naravno število $n$, da velja
  $|n\cdot \overrightarrow{SX}|>r$. Naj bo $X_n$ točka, za katero
  velja $\overrightarrow{SX_n}=n\cdot \overrightarrow{SX}$ in
  $X'_n=\psi_i(X_n)$. Iz prejšnjega zgleda in prvega dela dokaza
  točki $X_n$ in $X'_n$ lahko narišemo.
  Za točko $X'=\psi_i(X)$ velja:
   $$|SX'|=\frac{r^2}{|SX|}=\frac{n\cdot r^2}{|SX_n|}=n\cdot |SX'_n|,$$
 kar pomeni, da lahko tudi točko $X'$ narišemo  s pomočjo
 prejšnjega zgleda.
  \kdokaz

 \bzgled Dani sta točki $A$ in $B$. Načrtaj središče daljice
 $AB$. \label{MaskeroniSred}
 \ezgled


\begin{figure}[!htb]
\centering
\input{sl.inv.9.9.3.pic}
\caption{} \label{sl.inv.9.9.3.pic}
\end{figure}

 \textbf{\textit{Solution.}} Naj bo $\psi_i$ inverzija
 glede na krožnico $i(A,AB)$ (Figure \ref{sl.inv.9.9.3.pic}).
  Najprej načrtamo takšno točko
 $X$, da velja $\overrightarrow{AX}=2\cdot\overrightarrow{AB}$
 (zgled \ref{MaskeroniNAB}),  nato pa $X'=\psi_i(X)$ (zgled \ref{MaskeroniInv}).
 Ker je $X$ zunanja točka krožnice $i$, je $X'$ njena notranja točka.
 Zato je $\mathcal{B}(A,X',B)$. Za točko $X'$ velja tudi
 $|AX'|=\frac{|AB|^2}{|AX|}=\frac{|AB|^2}{2\cdot |AB|}=\frac{1}{2}\cdot |AB|,$
 kar pomeni, da je $X'$ središče daljice $AB$.
 \kdokaz

 \bzgled \label{MaskeroniProj} Dane so tri nekolinearne točke $P$, $Q$ in $R$. Načrtaj
 pravokotno projekcijo točke $P$ na premico $QR$.
 \ezgled


\begin{figure}[!htb]
\centering
\input{sl.inv.9.9.4.pic}
\caption{} \label{sl.inv.9.9.4.pic}
\end{figure}

 \textbf{\textit{Solution.}} Načrtajmo najprej središči daljic
 $PQ$ in $PR$ (zgled \ref{MaskeroniSred}) in ju označimo po vrsti s $Q_1$
 in $R_1$ (Figure \ref{sl.inv.9.9.4.pic}). Točko $P'$ načrtamo
 kot drugo presečišče krožnic $k(Q_1,Q_1P)$ in $l(R_1,R_1P)$. Točka $P'$ leži na krožnicah s premeroma $PQ$ in $PR$. Zato
 je $\angle PP'Q=\angle PP'R=90^0$. Torej točka $R$ leži na
 premici $P'Q$, kar pomeni, da so točke $P'$, $Q$ in $R$
 kolinearne. Zato je točka $P'$ pravokotna projekcija točke $P$
 na premici $QR$.
  \kdokaz

 \bzgled \label{MaskeroniOcrt} Dane so tri nekolinearne točke $A$, $B$ in $C$.
 Načrtaj
 središče očrtane krožnice in očrtano krožnico trikotnika
 $ABC$.
 \ezgled

\begin{figure}[!htb]
\centering
\input{sl.inv.9.9.5.pic}
\caption{} \label{sl.inv.9.9.5.pic}
\end{figure}

 \textbf{\textit{Solution.}}
 Naj bo $l$ očrtana krožnica trikotnika $ABC$ s središčem v
 točki $O$ ter $\psi_i$ inverzija s središčem $A$ in poljubnim
 polmerom (Figure \ref{sl.inv.9.9.5.pic}). Ker $A\in l$, iz izreka
  \ref{InverzKroznVkrozn} sledi,
 da je $l'=\psi_i(l)$ premica. Če je $P'$ pravokotna projekcija
 središča inverzije $A$ na premici $l'$ in $P=\psi_i(P')$, je
 $AP$ premer krožnice $l$ (posledica konstrukcije v dokazu
  (\textit{ii}) izreka
 \ref{InverzKroznVkrozn}).

  Dokazana dejstva nam omogočajo konstrukcijo. Najprej načrtamo
  poljubno krožnico $i$ s središčem v točki $A$ in
  točki $B'=\psi_i(B)$ in $C'=\psi_i(C)$ (zgled
  \ref{MaskeroniInv}), nato pravokotno projekcijo $P'$ točke $A$
  na premici $B'C'$ (zgled \ref{MaskeroniProj}) in $P=\psi_i(P')$.
  Središče očrtane krožnice dobimo kot središče daljice $AP$
  (zgled \ref{MaskeroniSred}). Krožnica $l(O,A)$ je očrtana
  krožnica trikotnika $ABC$.
 \kdokaz

 Eno od elementarnih konstrukcij samo z uporabo šestila predstavlja konstrukcija
 presečišča dveh krožnic. S pomočjo inverzije jo bomo
 uporabili za naslednji elementarni konstrukciji, ki ju štejemo za
 konstrukciji z ravnilom in s
 šestilom.

 \bzgled \label{MaskeroniKrPr} Dane so štiri točke $A$, $B$, $C$ in $D$.
 Načrtaj  presečišče premic $AB$ in $CD$.
 \ezgled


\begin{figure}[!htb]
\centering
\input{sl.inv.9.9.6.pic}
\caption{} \label{sl.inv.9.9.6.pic}
\end{figure}

 \textbf{\textit{Solution.}} Predpostavimo, da premici $AB$ in $CD$
 nista vzporedni.
 Naj bo $S$ poljubna točka, ki ne leži na
 nobeni od premic $AB$ in $CD$. Jasno je, da takšna točka
 obstaja, toda efektivno konstrukcijo takšne točke lahko dobimo,
 če uporabimo postopek konstrukcij točk $Q_1$ in $Q_2 $ iz
 zgleda \ref{MaskeroniNAB} glede na daljico $AB$.
 Naj bo $\psi_i$ inverzija glede na poljubno krožnico $i$ s
 središčem v točki $S$ (Figure \ref{sl.inv.9.9.6.pic}).
 Načrtajmo  točke $A'=\psi_i(A)$, $B'=\psi_i(B)$, $C'=\psi_i(C)$ in
  $D'=\psi_i(D)$ (zgled \ref{MaskeroniInv}) ter očrtane krožnice
  trikotnikov $SA'B'$ in $SC'D'$ (zgled \ref{MaskeroniOcrt}) oz.
  slike premic $AB$ in $CD$ pri inverziji $\psi_i$. Točka $L'$ je
  drugo presečišče dveh očrtanih krožnic (prvo presečišče je točka $S$)
   Le-to obstaja v primeru, kadar premici $AB$ in $CD$
 nista vzporedni. Presečišče $L$ premic $AB$ in $CD$ je slika
 presečišča $L'$ njunih slik - torej $L=\psi_i(L')$ (zgled \ref{MaskeroniInv}).
  \kdokaz


 \bzgled \label{MaskeroniKtKr}Dane so krožnica $k$ ter točki $A$ in $B$.
  Načrtaj  presečišča premice $AB$ in krožnice $k$.
 \ezgled


\begin{figure}[!htb]
\centering
\input{sl.inv.9.9.7.pic}
\caption{} \label{sl.inv.9.9.7.pic}
\end{figure}

 \textbf{\textit{Solution.}} Konstrukcijo lahko izpeljemo na enak
 način kot v prejšnjem primeru, le da je $k'$ očrtana krožnica
 trikotnika, ki ga določajo točke $C'=\psi_i(C)$, $D'=\psi_i(D)$
 in $E'=\psi_i(E)$, kjer so $C$, $D$ in $E$ poljubne točke
 krožnice $k$
  (Figure \ref{sl.inv.9.9.7.pic}).
\kdokaz

 Z zadnjima dvema konstrukcijama smo dokazali, da se vse
 elementarne konstrukcije, ki se nanašajo na ravnilo in šestilo  (glej razdelek \ref{elementarneKonstrukcije}),
 lahko naredijo samo s šestilom. Kot smo že omenili, pri tem štejemo, da je
  premica načrtana, če sta načrtani dve njeni točki (podobno velja za daljico in poltrak), ne smemo
  pa uporabiti direktne konstrukcije presečišča dveh premic oz. premice in krožnice (z uporabo
  ravnila). Iz zadnjih dveh konstrukcij (zgleda \ref{MaskeroniKrPr} in \ref{MaskeroniKtKr}) pa sledi, da je slednje možno tudi samo s šestilom.

 Ker vsako načrtovalno nalogo, ki jo lahko rešimo z ravnilom in s šestilom,
  izpeljemo z elementarnimi konstrukcijami, ki se
 nanašajo na ravnilo in šestilo, iz prejšnjega sledi,
  da to nalogo lahko rešimo le s šestilom.


%NALOGE
%________________________________________________________________________________

\naloge{Exercises}


\begin{enumerate}

  \item
Dokaži, da kompozitum dveh inverzij $\psi_{S,r_1}$ in $\psi_{S,r_2}$ glede na koncentrični
  krožnici predstavlja razteg. Določi središče in koeficient
  tega raztega.

  \item Naj bodo $A$, $B$, $C$ in $D$ štiri kolinearne točke.
  Konstruiraj takšni točki $E$ in $F$, da velja $\mathcal{H}(A,B;E,F)$
in $\mathcal{H}(C,D;E,F)$.

 \item V ravnini so dani točka $A$, premica $p$ in krožnica $k$.
 Načrtaj krožnico, ki poteka skozi  točko $A$ in je
 pravokotna na premico $p$ in krožnico $k$.

  \item Reši tretji, četrti, deveti in deseti Apolonijev problem.

  \item Naj bodo $A$ točka, $p$ premica, $k$ krožnica
   in $\omega$ kot v neki ravnini. Načrtaj krožnico,
ki poteka skozi točko $A$, se dotika premicp $p$ in s krožnico $k$
določa kot $\omega$.

 \item Določi geometrijsko mesto točk dotika dveh krožnic, ki
 se dotikata krakov danega kota v dveh danih točkah $A$ in $B$.

  \item Načrtaj trikotnik, če so znani naslednji podatki:
\begin{enumerate}
 \item $a$, $l_a$, $v_a$
 \item $v_a$, $t_a$, $b-c$
 \item $b+c$, $v_a$, $r_b-r_c$
 \end{enumerate}


   \item Naj bosta $c(S,r)$ in $l$ krožnica in premica v isti ravnini, ki
nimata skupnih točk. Naj bodo še $c_1$, $c_2$ in $c_3$ krožnice
te ravnine, ki se medsebojno (po dve) dotikajo in se vsaka od
njih dotika še $c$ in $l$. Izrazi razdaljo točke $S$ od premice
$l$ s $r$\footnote{Predlog za MMO 1982. (SL 12.)}.

\item Naj bo $ABCD$  pravilni tetraeder. Poljubni točki
$M$, ki leži na robu $CD$, pridružimo točko $P = f(M)$, ki
je presečišče pravokotnice skozi točko $A$ na premico $BM$ in
pravokotnice skozi točko $B$ na premico $AM$. Določi geometrijsko
mesto vseh točk $P$, če točka $M$ zavzame vse vrednosti na
robu $CD$.

\item Naj bo $ABCD$ tetivnotangentni štirikotnik in $P$, $Q$,
$R$ in $S$ dotikališča stranic $AB$, $BC$, $CD$ in $AD$ z včrtano
krožnico tega štirikotnika. Dokaži, da velja $PR\perp QS$.

\item Dokaži, da sta središči tetivnotangentnemu štirikotniku včrtane in očrtane
krožnice ter presečišče njegovih diagonal kolinearne točke
(\index{izrek!Newtonov}Newtonov izrek\footnote{\index{Newton,
I.}\textit{I. Newton} (1643--1727), angleški fizik in matematik}).

\item Naj bosta $p$ in $q$ vzporedni tangenti krožnice $k$.
Krožnica $c_1$ se dotika premice $p$ v točki $P$ in krožnice $k$ v
točki $A$, krožnica $k_2$ pa se dotika premice $q$ ter krožnic $k$
in $k_1$ v točkah $Q$, $B$ in $C$. Dokaži, da je presečišče premic
$PB$ in $AQ$ središče trikotniku $ABC$ očrtane krožnice.

 \item Krožnici $k_1$ in $k_3$ se od zunaj dotikata v točki $P$. Prav
 tako se tudi krožnici $k_2$ in $k_2$ od zunaj dotikata v isti
 točki. Krožnica $k_1$ seka krožnici $k_2$ in $k_4$ še v točkah
 $A$ in $D$,  krožnica $k_3$ pa seka krožnici $k_2$ in $k_4$ še v točkah
 $B$ in $C$. Dokaži, da velja\footnote{Predlog za MMO 2003. (SL 16.)}:
 $$\frac{|AB|\cdot|BC|}{|AD|\cdot|DC|}=\frac{|PB|^2}{|PD|^2}.$$

\item Naj bo $A$ točka, ki leži na krožnici $k$. Samo s
   šestilom načrtaj kvadrat $ABCD$ (oz. njegova oglišča), ki je včrtan  dani
krožnici.

\item Dani sta točki $A$ in $B$. Le z uporabo
  šestila načrtaj takšno točko $C$, da velja
   $\overrightarrow{AC}=\frac{1}{3}\overrightarrow{AB}$.

   \item Samo s pomočjo
   šestila razdeli dano daljico v razmerju $2:3$.
\end{enumerate}









% DEL 10 - - - - - - - - - - - - - - - - - - - - - - - - - - - - - - - - - - - - - - -
%________________________________________________________________________________
% REŠITVE IN NAPOTKI
%________________________________________________________________________________

\del{Solutions and Hints}
\footnotesize
%REŠITVE - Aksiomi evklidske geometrije ravnine
%________________________________________________________________________________

\poglavje{Introduction}

There are no exercises in the first chapter. \color{viol4}$\ddot\smile$ \normalcolor %smile

\poglavje{Axioms of Planar Euclidean Geometry}

\begin{enumerate}

\item \res{Naj bodo $P$, $Q$ in $R$ notranje točke stranic trikotnika $ABC$. Dokaži, da so $P$, $Q$ in $R$ nekolinearne.}

Predpostavimo nasprotno, da točke $P$, $Q$ in $R$ ležijo na
 neki premici $l$. To pomeni, da premica $l$ seka vse tri
stranice trikotnika $ABC$ in po predpostavki ne poteka skozi nobeno
njegovih oglišč. To je v nasprotju s posledico \ref{PaschIzrek}
 Paschevega aksioma \ref{AksPascheva}, zatorej  so $P$, $Q$ in
 $R$ nekolinearne točke.

\item \res{Naj bosta $P$ in $Q$ točki stranic $BC$ in $AC$ trikotnika $ABC$ in hkrati
različni od njegovih oglišč. Dokaži, da se daljici $AP$ in $BQ$
sekata v eni točki.}

 Iz predpostavke, da točka $P$ leži na stranici $BC$ trikotnika
 $ABC$, po definiciji daljice $BC$ velja $\mathcal{B}(B,P,C)$.
 Iz aksioma \ref{AksII2} sledi, da ne velja
 $\mathcal{B}(B,C,P)$, kar pomeni, da točka $B$ ne leži na
 daljici $PC$. V trikotniku $APC$ premica $BQ$ seka stranico $AC$,
 ne seka pa stranice $PC$, zato po posledici \ref{PaschIzrek}
 Paschevega aksioma \ref{AksPascheva} premica $BQ$ seka
 stranico $AP$ tega trikotnika. Torej premica $BQ$ seka daljico
 $AP$ v neki točki $X$ oz. $BQ\cap [AP]=\{X\}$. Analogno je
 tudi $[BQ]\cap AP=\{\widehat{X}\}$. Dokažimo, da je
 $X=\widehat{X}$. Predpostavimo nasprotno, da velja $X\neq
 \widehat{X}$. Ker je $[AP]\subset AP$ in $[BQ]\subset BQ$,
 velja $X,\widehat{X}\in AP\cap BQ$. Po aksiomu \ref{AksI1} predstavljata
  $AP$ in $BQ$ isto premico oz. točke $A$, $B$,
 $P$ in $Q$ ležijo na isti premici $l$. Toda na tej premici, ki
 je določena s točkama $B$ in $P$ (aksiom \ref{AksI1}), leži po
 predpostavki tudi točka $C$, kar pomeni, da so točke $A$, $B$
 in $C$ kolinearne. To pa je v nasprotju s predpostavko, da je
 $ABC$ trikotnik, zato je $X=\widehat{X}$. To pomeni $X\in
 [AP]\cap[BQ]$. Če bi daljici $[AP]$ in $[BQ]$ imeli še eno
 skupno točko, bi ta bila druga skupna točka tudi premic $AP$
 in $BQ$. Dokazali smo, da to ni možno, zatorej je
 $[AP]\cap[BQ]=\{X\}$.

\item  \res{Točke $P$, $Q$ in $R$ ležijo po vrsti na stranicah $BC$, $AC$ in $AB$ trikotnika
 $ABC$ in so različne od njegovih oglišč. Dokaži, da se daljici $AP$
in $QR$ sekata v eni točki.}

Uporabimo prejšnjo nalogo najprej za daljici $AP$ in $BQ$ v
trikotniku $ABC$, nato pa še enkrat  za daljici $AX$
($\{X\}=[AP]\cap[BQ]$) in $QR$ v trikotniku $ABQ$.



\item \res{Premica $p$, ki leži v ravnini štirikotnika, seka njegovo
diagonalo $AC$ in ne poteka skozi nobeno oglišče tega štirikotnika.
Dokaži, da premica $p$ seka natanko dve stranici tega
štirikotnika.}

Dvakrat uporabimo posledico \ref{PaschIzrek} Paschevega aksioma
 \ref{AksPascheva} za premico $p$ in trikotnika $ABC$ in $ADC$.

\item \label{nalAks3}\res{Dokaži, da je polravnina konveksni lik.}

Naj bosta $X$ in $Y$ poljubni točki polravnine $pA$. Potrebno je dokazati, da tudi cela daljica $XY$ leži v tej polravnini. Po definiciji polravnine iz $X,Y\in pA$ sledi $X,A\ddot{-} p$ in $Y,A\ddot{-} p$. Ker je $\ddot{-} p$ ekvivalenčna relacija, je tudi $X,Y\ddot{-} p$. Naj bo $Z$ poljubna točka daljice $XY$. Predpostavimo, da ni $Z,A\ddot{-} p$ oz. da velja $Z,A\div p$ oz. $Z,X\div p$ (ali $Z \in p$). V tem primeru daljica $ZX$ seka premico $p$ v neki točki $T$. Točka $T$ je potem skupna točka premice $p$ in daljice $XY$ (izrek \ref{izrekAksIIDaljica}), kar je v nasprotju z dokazanim $X,Y\ddot{-} p$. Torej velja $Z,A\ddot{-} p$, zato je tudi $Z\in pA$, kar pomeni, da je polravnina $pA$ konveksni lik.

\item \label{nalAks4} \res{Dokaži, da je presek dveh konveksnih likov
konveksni lik.}

Naj bosta $\Phi_1$ in $\Phi_2$ konveksna lika. Dokažimo, da je
potem  tudi $\Phi_1\cap\Phi_2$ konveksni lik. Naj bosta $A, B\in
\Phi_1\cap\Phi_2$ poljubni točki. V tem primeru je $A, B\in
\Phi_1$ in $A,B\in\Phi_2$. Ker sta lika $\Phi_1$ in $\Phi_2$
konveksna, je $[AB]\subseteq\Phi_1$ in $[AB]\subseteq\Phi_2$
oz. $[AB]\subseteq\Phi_1\cap\Phi_2$, kar pomeni, da je tudi
$\Phi_1\cap\Phi_2$ konveksni lik.

\item  \res{Dokaži, da je poljubni trikotnik konveksni lik.}

Trikotnik $ABC$ je presek polravnin $ABC$, $ACB$ in
$BCA$, zato je po prejšnjih dveh nalogah \ref{nalAks3} in
\ref{nalAks4} konveksni lik.

\item  \res{Če je $\mathcal{B}(A,B,C)$ in $\mathcal{B}(D,A,C)$, je tudi
$\mathcal{B}(B,A,D)$. Dokaži.}

Po izreku \ref{izrekAksIIPoltrak} določa točka $A$ na premici $AB$ dva poltraka.
Označimo s $p$ poltrak $AC$ in $p'$ njegov komplementarni (dopolnilni) poltrak. Iz $\mathcal{B}(A,B,C)$  sledi $B,C\ddot{-} A$ iz $\mathcal{B}(D,A,C)$ pa $D,C\div A$, zato velja $B\in p$ in $D\in p'$. Torej točki $B$ in $D$ ne ležita na istem poltraku z začetno točko $A$, zato ni $B,D\ddot{-} A$, kar pomeni, da velja $B,D\div A$ oz.  $\mathcal{B}(B,A,D)$.



\item  \res{Naj bodo $A$, $B$, $C$ in $D$ takšne kolinearne točke, da je
$\neg\mathcal{B}(B,A,C)$ in $\neg\mathcal{B}(B,A,D)$. Dokaži, da
velja $\neg\mathcal{B}(C,A,D)$.}

Iz $\neg\mathcal{B}(B,A,C)$ in $\neg\mathcal{B}(B,A,D)$ sledi $B,C\ddot{-} A$ oz. $B,D\ddot{-} A$. Ker je $\ddot{-} A$ ekvivalenčna relacija, je tudi tranzitivna, zato velja $C,D\ddot{-} A$ oz. $\neg\mathcal{B}(C,A,D)$.

\item \res{Naj bo $A_1A_2\ldots,A_{2k+1}$ poljubni večkotnik z lihim
številom oglišč. Dokaži, da ne obstaja premica, ki seka vse
njegove stranice.}

Predpostavimo nasprotno, da neka premica $p$ seka vse stranice
tega večkotnika. Dokaži, da bi bile v tem primeru točke
$A_1,A_3,\ldots,A_{2k+1}$ na istem bregu premice $p$.

\item \label{nalAks11}\res{Če izometrija $\mathcal{I}$ preslika lika $\Phi_1$ in $\Phi_2$
v lika  $\Phi'_1$ in $\Phi'_2$, potem se  presek
$\Phi_1\cap\Phi_2$ s to izometrijo preslika v presek
$\Phi'_1\cap\Phi'_2$. Dokaži.}

Če za poljubno točko $X$ velja $X\in \Phi_1\cap\Phi_2$, je $X\in \Phi_1$ in $X\in\Phi_2$. Iz tega sledi $\mathcal{I}(X)\in \mathcal{I}(\Phi_1)$ in $\mathcal{I}(X)\in \mathcal{I}(\Phi_2)$ oz. $X'\in \Phi'_1$ in $X'\in \Phi'_2$ (kjer je $X'=\mathcal{I}(X)$). Torej velja $X'\in\Phi'_1\cap\Phi'_2$. Na ta način smo dokazali $\mathcal{I}(\Phi_1\cap\Phi_2)\subseteq \Phi'_1\cap\Phi'_2$. Drugo inkluzijo $\mathcal{I}(\Phi_1\cap\Phi_2)\supseteq \Phi'_1\cap\Phi'_2$ dokažemo na enak način z uporabo izometrije $\mathcal{I}^{-1}$.

\item  \res{Dokaži, da sta poljubna poltraka neke
ravnine med seboj skladna.}

Uporabimo aksiom \ref{aksIII2} in izrek \ref{izrekIzoB}.

\item  \res{Dokaži, da sta poljubni premici neke
ravnine med seboj skladni.}

Uporabimo aksioma \ref{aksIII2} in \ref{aksIII1}.


\item  \res{Naj bosta $k$ in $k'$ dve krožnici
neke ravnine s središčema $O$ in $O'$ ter polmeroma $AB$ in $A'B'$.
Dokaži ekvivalenco: $k\cong k' \Leftrightarrow AB\cong A'B'$.}

($\Leftarrow$) Naj bo $AB\cong A'B'$. Označimo s $P$ in $P'$ poljubni točki krožnic $k$ oz. $k'$. Ker sta $OP$ in $O'P'$ polmera teh krožnic, je $OP\cong O'P'$. Po izreku \ref{izrekAB} obstaja izometrija $\mathcal{I}$, ki preslika točki $O$ in $P$ v točki $O'$ in $P'$. Naj bo $X$ poljubna točka krožnice $k$ in $X'=\mathcal{I}(X)$. Iz $\mathcal{I}:O,X,P\rightarrow O',X',P'$ sledi  $O'X'\cong OX$. Ker $X,P\in k$, je tudi $OX\cong OP$. Iz prejšnjih relacij sledi $O'X'\cong O'P'$ oz. $X'\in k'$. Torej velja $\mathcal{I}(k)\subseteq k'$. Podobno je $\mathcal{I}(k)\supseteq k'$ (uporabimo $\mathcal{I}^{-1}$), zato je $\mathcal{I}(k)= k'$.

($\Rightarrow$) Naj b sedaj $k\cong k'$. Predpostavimo nasprotno - da ni $AB\cong A'B'$. Brez škode za splošnost naj bo $AB>A'B'$.
 Iz $k\cong k'$ sledi, da obstaja izometrija $\mathcal{I}$, ki preslika krožnico $k$ v krožnico $k'$. Naj bo $PQ$ poljubni premer krožnice $k$. Po izreku \ref{premerInS} je $PQ=2\cdot AB$. Naj bo $P'=\mathcal{I}(P)$ in $Q'=\mathcal{I}(Q)$. Torej $P'Q'$ je tetiva krožnice, za katero velja $P'Q'\cong PQ =2\cdot AB>2\cdot A'B'$. Toda relacija $P'Q'>2\cdot A'B'$ ni možna (izrek \ref{premerNajdTetiva}), zato je $AB\cong A'B'$.

\item  \res{Naj bo $\mathcal{I}$
neidentična izometrija ravnine z dvema negibnima točkama $A$ in
$B$. Naj bo $p$ premica te ravnine, ki je vzporedna s premico
$AB$, in $A\notin p$. Dokaži, da na premici $p$ ni negibnih točk
izometrije $\mathcal{I}$.}

Predpostavimo nasprotno - da obstaja fiksna točka $C$ izmetrije $\mathcal{I}$, ki leži na premici $p$. Točka $C$ ne leži na premici $AB$, ker bi bili v nasprotnem vzporednici $AB$ in $p$ enaki, kar pa je v nasprotju s predpostavko $A\notin p$. Torej imamo tri nekolinearne fiksne točke izometrije $\mathcal{I}$, zato je po izreku \ref{IizrekABC2} $\mathcal{I}$ identična preslikava. To je v nasprotju s predpostavko, zato na premici $p$ ni fiksnih točk izometrije $\mathcal{I}$.

\item   \res{ Naj bo $S$ edina negibna točka
izometrije $\mathcal{I}$ v neki ravnini. Dokaži, da če ta izometrija
preslika  premico $p$ vase, je $S\in p$.}

Predpostavimo, da je $S\notin p$. Označimo z $N$ pravokotno projekcijo točke $S$ na premici $p$ in $N'=\mathcal{I}(N)$. Iz $S\notin p$ sledi $N\neq S$. Ker je $S$ edina fiksna točka te izometrije, je $N'\neq N$. Po predpostavki $\mathcal{I}:p\rightarrow p$ je tudi $N'\in p$. Iz $\mathcal{I}:SN,p\rightarrow SN', p$ sledi $\angle SN',p=\angle SN,p=90^0$. V tem primeru bi iz točke $S$ obstajali dve različni pravokotnici na premici $p$, kar po izreku \ref{enaSamaPravokotnica} ni možno. Torej predpostavka $S\notin p$ odpade, kar pomeni, da velja $S\in p$.

\item  \label{nalAks17}\res{Dokaži, da se poljubni dve premici
neke ravnine ali sekata ali sta vzporedni. }

Po aksiomu \ref{AksI1} imata dve različni premici $p$ in $q$ neke ravnine  kvečjemu eno skupno točko. Če imata eno skupno točko, se po definiciji premici sekata, če pa nimata skupnih točk, sta po definiciji vzporedni.

\item  \label{nalAks18}\res{Če neka
premica v ravnini seka eno od dveh vzporednic iste ravnine,
 potem  seka tudi drugo vzporednico. Dokaži.}

Naj bo $p\parallel q$ in $l\cap p =\{A\}$. Če je $p=q$, je jasno tudi $l\cap q =\{A\}$. Druga možnost je pa $p\cap q =\emptyset$. Po Playfairjevem aksiomu je potem $l\cap q \neq\emptyset$. Ker ne more biti niti $l=q$, se po prejšnji nalogi \ref{nalAks17} premici $l$ in $q$ sekata.

\item  \res{Dokaži, da vsaka izometrija preslika vzporednici v vzporednici.}

Uporabimo nalogo \ref{nalAks11}.

\item  \res{Naj bodo $p$, $q$ in $r$ takšne premice neke ravnine, tako da velja
$p\parallel q$ in $r\perp p$. Dokaži, da je $r\perp q$.}

Trditev je direktna posledica naloge \ref{nalAks18} in izreka \ref{KotiTransverzala}.

\item \res{Dokaži, da konveksen $n$-kotnik ne more imeti več kot treh
ostrih kotov.}

Uporabimo dejstvo, da je vsota zunanjih kotov konveksnega $n$-kotnika
enaka $360^0$ (izrek \ref{VsotKotVeckZuna}).


\end{enumerate}


%REŠITVE - Skladnost trikotnikov
%________________________________________________________________________________

\poglavje{Congruence. Triangles and Polygons}

\begin{enumerate}


 \item \res{Naj bo $S$ točka, ki leži v kotu $pOq$, točki $A$ in $B$ pa
 pravokotni projekciji  točke $S$ na krakih $p$ in $q$ tega
kota. Dokaži, da je $SA\cong SB$ natanko tedaj, ko je premica
$OS$ simetrala kota $pOq$.}

V obeh smereh ekvivalence dokaži skladnost trikotnikov $OSA$ in
$OSB$. V dokazu za ($\Rightarrow$) izrek  \textit{SSA}
\ref{SSK}, v dokazu za ($\Leftarrow$) pa izrek \textit{ASA}
\ref{KSK}.

\item \res{Dokaži, da je vsota diagonal konveksnega štirikotnika večja od
vsote dveh njegovih nasprotnih stranic.}

Uporabimo trikotniško neenakost za trikotnika $ASB$ in $CSD$, kjer
je $S$ presečišče diagonal $AC$ in $BD$ konveksnega štirikotnika
$ABCD$.

  \item \res{Dokaži, da je v vsakemu trikotniku
  največ ena stranica krajša od pripadajoče višine.}

  Označimo z $a$, $b$ in $c$ stranice in $v_a$, $v_b$,  $v_c$
  pripadajoče višine trikotnika $ABC$. Predpostavimo nasprotno, da
  je npr. $a<v_a$ in $b<v_b$. Če uporabimo neenakost za
  pravokotne trikotnike (izrek \ref{vecstrveckotHipot}, je
  $v_a\leq b$ in $v_b\leq a$. Če povežemo štiri neenakosti,
  dobimo $a<v_a\leq b<v_b\leq a$ oz. $a<a$, kar ni mogoče. Torej je
  največ ena stranica trikotnika krajša od pripadajoče višine.

  \item \res{Naj bo $AA_1$ težiščnica trikotnika $ABC$. Dokaži,
   da je od dveh kotov, ki jih težiščnica $AA_1$ določa s
stranicama $AB$ in $AC$, večji tisti, ki ga težiščnica določa s
krajšo stranico.}

Uporabimo trikotniško neenakost za trikotnik $ADC$, kjer je
$D=S_{A_1}(A)$ (glej definicijo središčnega zrcaljenja v razdelku
\ref{odd6SredZrc}).

  \item \res{Naj  bosta $BB_1$ in $CC_1$ težiščnici trikotnika $ABC$ ter
  $AB<AC$.
  Dokaži, da je $BB_1<CC_1$.}

Označimo z $AA_1$ tretjo težiščnico in s $T$ težišče trikotnika
$ABC$. Če uporabimo izrek \ref{SkladTrikLema} za trikotnika
$BA_1A$ in $CA_1A$, dobimo $\angle BA_1A< \angle CA_1A$. Če potem
isti izrek uporabimo še za trikotnika $BA_1T$ in $CA_1T$, dobimo
$BT<CT$ oz. $BB_1<CC_1$ (izrek \ref{tezisce}).


  \item \res{Naj bodo $a$, $b$ in $c$ stranice, $t_a$, $t_b$ in $t_c$
   ustrezne težiščnice ter $s$ polobseg poljubnega trikotnika.
Dokaži, da velja:
 \begin{enumerate}
  \item $s <  t_a  + t_b +  t_c  < 2s$
  \item $ta + tb + tc  >  \frac{3}{4}(a + b + c)$
 \end{enumerate}}

 Uporabimo zgled \ref{neenTezisZgl}.

\item \res{Naj bo premica $p$ mimobežnica krožnice $k$. Dokaži, da so
vse točke te krožnice na istem bregu premice $p$.}

Predpostavimo nasprotno, da sta točki $X,Y\in k$ na različnih
bregovih premice $p$. V tem primeru tetiva $XY$ seka premico $p$
v neki točki $N$, ki je po zgledu \ref{tetivaNotrTocke} notranja
točka krožnice $k$. Premica $p$, ki vsebuje notranjo točko $N$
krožnice $k$, je po izreku \ref{DedPoslKrozPrem} njena sekanta,
kar je v nasprotju s predpostavko, da je $p$ mimobežnica.

\item \res{Če krožnica $k$ leži v nekem konveksnem liku $\Phi$, potem tudi
krog, ki je določen s to krožnico, leži v tem liku. Dokaži.}

Dovolj je dokazati, da poljubna notranja točka $X$ krožnice $k$
leži v liku $\Phi$. Naj bo $AB$ poljubna tetiva, ki vsebuje
točko $X$ (obstoj takšne tetive sledi iz izreka
\ref{DedPoslKrozPrem}). Ker je $\Phi$ konveksni lik, iz $A,B\in
k\subset \Phi$ sledi $X\in [AB]\subset \Phi$.

\item \res{Naj bosta $p$ in $q$ različni tangenti krožnice $k$, ki se jo dotikata v
točkah $P$ in $Q$. Dokaži ekvivalenco: $p \parallel q$ natanko
tedaj, ko je $AB$ premer krožnice $k$.}

Uporabimo dejstvo $SP\perp p$ in $SQ\perp q$, kjer je $S$ središče
krožnice $k$.

\item \res{Če je $AB$ tetiva krožnice $k$, potem je presek premice $AB$ in
kroga, ki ga krožnica $k$ določa, enak tej tetivi. Dokaži.}

Označimo s $\mathcal{K}$ omenjeni krog. Potrebno je dokazati
$[AB]=AB\cap \mathcal{K}$. Inkluzija $[AB]\subseteq AB\cap \mathcal{K}$ sledi
direktno iz aksioma \ref{AksII1} in zgleda
\ref{tetivaNotrTocke}. Dokažimo še inkluzijo $AB\cap
\mathcal{K}\subseteq [AB]$. Naj bo $X\in AB\cap \mathcal{K}$. Predpostavimo, da
 $X\notin [AB]$. V tem primeru bi bilo $\mathcal{B}(X,A,B)$ ali
 $\mathcal{B}(A,B,X)$. Ni težko dokazati, da bi v vsakem od teh
 dveh primerov dobilo $SX>SA$, kar ni možno, ker je $X\in \mathcal{K}$.
 Torej velja $X\in[AB]$, oz. $AB\cap
\mathcal{K}\subseteq [AB]$.

\item \res{Naj bo $S'$ pravokotna projekcija središča $S$ krožnice $k$ na
premici $p$. Dokaži, da je $S'$ zunanja točka te krožnice
natanko tedaj, ko premica $p$ krožnice ne seka.}

Uporabimo izrek \ref{TangSekMimobKrit}.

\item \res{Naj bo $V$ višinska točka trikotnika $ABC$, pri katerem velja
$CV \cong AB$. Določi velikost kota $ACB$.}

Označimo z $A'$, $B'$ in $C'$ nožišča višin iz oglišč $A$, $B$
in $C$ trikotnika $ABC$. Najprej sta skladna kota $C'CB$ in
$A'AB$ (kota s pravokotnima krakoma - izrek
\ref{KotaPravokKraki}). Iz skladnosti trikotnikov $CVA'$ in
$ABA'$ (izrek \textit{ASA} \ref{KSK}) sledi $CA'\cong AA'$. To
pomeni, da je $CAA'$ enakokraki pravokotni trikotnik s
hipotenuzo $AC$, zato je po izreku \ref{enakokraki} $\angle
ACB=\angle ACA'=45^0$.

\item \res{Naj bo $CC'$ višina pravokotnega trikotnika $ABC$ ($\angle ACB =
90^0$). Če sta $O$ in $S$ središči včrtanih krožnic trikotnikov
$ACC'$ in $BCC'$, je simetrala notranjega kota $ACB$
pravokotna na premici $OS$. Dokaži.}

Naj bo $I$ središče včrtane krožnice trikotnika $ABC$. Dokažemo, da
je $I$ višinska točka trikotnika $COS$.

\item \res{Naj bo $ABC$ trikotnik, v katerem je $\angle ABC = 15^0$ in
$\angle ACB = 30^0$. Naj bo $D$ takšna točka stranice $BC$, da je
 $\angle BAD=90^0$. Dokaži, da je $BD = 2AC$.}

Označimo s $S$ središče daljice $BD$. Točka $S$ je središče
očrtane krožnice pravokotnega trikotnika $BAD$ (izrek
\ref{TalesovIzrKroz2}). Torej $SA\cong SB\cong SD$. Ker je $BSA$
enakokraki trikotnik z osnovnico $AB$, je po izreku
\ref{enakokraki} $\angle BAS\cong \angle ABS=15^0$. Za zunanji
kot $ASD$ trikotnika $BSA$ potem velja $\angle ASD=\angle BAS +
\angle ABS=30^0$ (izrek \ref{zunanjiNotrNotr}). Torej velja
$\angle ASC=30^0=\angle ACS$, zato je $ASC$ enakokraki trikotnik
z osnovnico $SC$ oz. $AS\cong AC$ (izrek \ref{enakokraki}). Iz
tega sledi $BD = 2SB=2AS=2AC$.


\item \res{Dokaži, da obstaja takšen petkotnik, s katerim je mogoče tlakovati ravnino.}

Lahko izberemo takšen petkotnik $ABCDE$, da je $ABCE$
kvadrat, $ECD$ enakokraki pravokotni trikotnik z osnovnico $CE$
in $A,D\div CE$.

\item \res{Dokaži, da obstaja takšen desetkotnik, s katerim je mogoče tlakovati ravnino.}

Naredimo unijo po dveh ustreznih pravilnih šestkotnikov - celic
 tlakovanja $(6,3)$.

\item \res{V neki ravnini je vsaka točka pobarvana rdeče ali črno.
        Dokaži, da obstaja pravilni trikotnik, ki ima vsa
        oglišča enake barve.}

Uporabimo tlakovanje $(3,6)$.

\item \res{Naj bodo $l_1,l_2,\ldots, l_n$ ($n > 3$) loki, ki vsi ležijo na isti
krožnici. Središčni kot vsakega loka je kvečjemu enak $180^0$.
 Dokaži, da če ima vsaka trojica lokov vsaj eno skupno točko,
 obstaja točka, ki leži na vsakem  loku.}

 Označimo s $k(S,r)$ dano krožnico. Naj bodo $I_i$ ($i\in
 \{1,2,\ldots,n\}$) krožni izseki, ki jih določajo loki $l_k$,
 in $J_i=I_i\setminus\{S\}$. Ker je za vsak $l_i$ središčni kot
 kvečjemu enak $180^0$, so $J_i$ konveksni liki (dokaži). Po
 predpostavki ima vsaka trojica lokov $l_{i_1}$,  $l_{i_2}$ in $l_{i_2}$
 ($i_1,i_2,i_3\in\{1,2,\ldots,n\}$) neko skupno točko
 $X_{i_1i_2i_3}$. Ta točka pripada tudi  vsakemu od likov
 $J_{i_1}$,  $J_{i_2}$ in  $J_{i_2}$. Po Hellyjevem izreku
 \ref{Helly} obstaja točka $Y$, ki pripada vsakemu od likov
 $J_1,J_2,\ldots, J_n$. Če z $X$ označimo presečišče poltraka
 $SY$ s krožnico $k$, sledi, da tudi odprta daljica $(SX]$
 pripada vsakem od teh likov. Ker za vsak $i$ velja $l_i=J_i\cap k$, točka $X$
 pripada vsakem od lokov $l_1,l_2,\ldots, l_n$.

%drugi del

\item
\res{Naj bosta $p$ in $q$ pravokotnici, ki  se sekata v točki $A$. Če
je $B, B'\in p$, $C, C'\in q$, $AB\cong AC'$, $AB'\cong AC$,
$\mathcal{B}(B,A,B')$ in $\mathcal{B}(C,A,C')$, potem
pravokotnica na premico $BC$ skozi točko $A$ poteka skozi središče
daljice $B'C'$. Dokaži.}

Najprej sta po izreku \textit{SAS} \ref{SKS} trikotnika $BAC$ in
$C'AB'$ skladna. Iz tega sledi: $\angle AC'B'\cong\angle
ABC=\beta$ in $\angle AB'C'\cong\angle ACB=90^0-\beta$. Označimo
s $S$ središče daljice $B'C'$, s $P$ pa presečišče premic $BC$
in $AS$. Po izreku \ref{TalesovIzrKroz2} je $SA\cong SB'\cong
SC'$, zato je $\angle CAP=\angle C'AS\cong\angle AC'S=\angle
AC'B'=\beta$ (izrek \ref{enakokraki}). Ker je še $\angle
ACP=\angle ACB=90^0-\beta$, je iz trikotnika $ACP$ po izreku
\ref{VsotKotTrik} $\angle APC=90^0$ oz. $AS\perp BC$. To pomeni,
da pravokotnica $AP$ premice $BC$ poteka skozi središče daljice
$B'C'$.

\item
\res{Dokaži, da se simetrale notranjih kotov pravokotnika, ki ni
kvadrat, sekajo v točkah, ki so oglišča kvadrata.}

Označimo s $s_{\alpha}$, $s_{\beta}$, $s_{\gamma}$ in
$s_{\delta}$ simetrale notranjih kotov ob ogliščih $A$, $B$, $C$
in $D$ pravokotnika $ABCD$ ter $P=s_{\alpha}\cap s_{\beta}$,
$Q=s_{\gamma}\cap s_{\delta}$, $L=s_{\beta}\cap s_{\gamma}$ in
$K=s_{\alpha}\cap s_{\delta}$. Dokažimo, da je $PKQL$ kvadrat.
Najprej iz $\angle PAB=45^0$ in $\angle PBA=45^0$ sledi, da je
$ABP$ enakokraki pravokotni trikotnik z osnovnico $AB$ (izreka
\ref{enakokraki} in \ref{VsotKotTrik}). Torej $\angle APB=90^0$
in $AP\cong BP$. Analogno so tudi ostali notranji koti
štirikotnika $PKQL$ pravi koti. Dovolj je dokazati še $PK\cong
PL$. Iz skladnosti trikotnikov $AKD$ in $BLC$ (izrek \ref{KSK})
je $AK\cong BL$. Če slednje povežemo z že dokazanim $AP\cong
BP$, dobimo  $PK\cong PL$.

\item
 \res{Dokaži, da se simetrale notranjih kotov paralelograma, ki ni
romb, sekajo v točkah, ki so oglišča pravokotnika. Dokaži še, da so diagonale
tega pravokotnika vzporedne s stranicami paralelograma in so
enake razliki sosednjih stranic tega paralelograma.}

Podobno kot pri prejšnji nalogi označimo s $s_{\alpha}$,
$s_{\beta}$, $s_{\gamma}$ in $s_{\delta}$ simetrale notranjih
kotov ob ogliščih $A$, $B$, $C$ in $D$ pravokotnika $ABCD$ ter
$P=s_{\alpha}\cap s_{\beta}$, $Q=s_{\gamma}\cap s_{\delta}$,
$L=s_{\beta}\cap s_{\gamma}$ in $K=s_{\alpha}\cap s_{\delta}$.
Dokažimo, da je $PKQL$ pravokotnik. Naj bosta $E$ in $F$
središči stranic $AD$ in $BC$ paralelograma $ABCD$. V trikotniku
$PAB$ je
 $\angle APB=180^0-\angle PAB-\angle
PBA=180^0-\frac{1}{2}\left(\angle DAB+\angle CBA
\right)=180^0-\frac{1}{2}\cdot 180^0=90^0$ (izreka
 \ref{VsotKotTrik} in \ref{paralelogram}). Analogno so tudi
 ostali notranji koti štirikotnika $PKQL$ pravi koti. Po izreku
 \ref{TalesovIzrKroz2} je $E$ središče očrtane krožnice
 pravokotnega trikotnika $AKD$ s hipotenuzo $AD$ in velja
 $EK\cong EA$. Torej $\angle EKA\cong \angle EAK \cong\angle
 KAB$ (izrek \ref{enakokraki}) oz. po izreku
 \ref{KotiTransverzala} je $EK\parallel AB$). Analogno je tudi $FL\parallel AB$. Ker je še $EF\parallel AB$ (izrek \ref{srednjTrapez}) je tudi
 $KL\parallel AB$. Brez škode za splošnost pedpostavimo, da je $AB>AD$. Označimo s $T$ presečišče simetrale
 $s_{\alpha}$ s stranico $CD$. Po izreku \ref{KotiTransverzala}
 je $\angle DTA\cong\angle TAB\cong\angle DAT$. To pomeni, da je
 $ADT$ enakokraki trikotnik, zato je $AD\cong DT$ (izrek
 \ref{enakokraki}). Ker je $KL\parallel AB\parallel CT$ in
 $s_{\alpha}\parallel s_{\gamma}$, je štirikotnik $KLCT$
 paralelogram. Torej $PQ\cong KL\cong CT=CD-DT=CD-AB$.


\item
\res{Dokaži, da sta simetrali dveh sokotov med seboj
pravokotni.}

Simetrali določata kot, ki je enak polovici ustreznega
iztegnjenega kota.

\item \res{Naj bosta $B'$ in $C'$ nožišči višin iz oglišč $B$ in $C$ trikotnika
$ABC$. Dokaži ekvivalenco $AB\cong AC \Leftrightarrow BB'\cong
CC'$.}

V obe smeri dokaži ekvivalenco  $\triangle BC'C\cong\triangle CB'B$. V
smeri ($\Rightarrow$) uporabi izrek \textit{ASA} \ref{KSK}, v
smeri ($\Leftarrow$) pa \textit{SSA} \ref{SSK}.

\item \res{Dokaži, da je trikotnik pravilen,
če središče trikotniku očrtane krožnice in njegova višinska točka sovpadata.
Ali podobna trditev velja za poljubni dve značilni
točki tega trikotnika?}

Uporabimi dejstvo, da so simetrale stranic v tem primeru nosilke
višin tega trikotnika. Na podoben način bi dokazali tudi, da
podobna trditev velja za poljubni dve značilni točki tega
trikotnika.

\item \res{Dokaži, da sta ostrokotna trikotnika $ABC$ in $A'B'C'$ skladna natanko
tedaj, ko imata skladni višini $CD$ in $C'D'$, stranici $AB$ in
$A'B'$ ter kota $ACD$ in $A'C'D'$.}

Iz skladnosti trikotnikov  $ABC$ in $A'B'C'$ direktno sledi
najprej $AB\cong A'B'$, $AC\cong A'C'$ in $\angle BAC\cong \angle B'A'C'$, nato
pa še $\triangle ACD\cong\triangle A'C'D'$ (izrek \textit{ASA}
\ref{KSK}) oz. $CD\cong C'D'$ in $\angle ACD\cong \angle A'C'D'$.

Predpostavimo, da velja $CD\cong C'D'$, $AB\cong A'B'$ in $\angle ACD\cong \angle A'C'D'$. Iz skladnosti
trikotnikov  $ACD$ in $A'C'D'$ (izrek \textit{ASA} \ref{KSK})
sledi, da obstaja izometrija $\mathcal{I}$, za katero velja
$\mathcal{I}:A,C,D \mapsto A',C',D'$ (izrek \ref{IizrekABC}.
Le-ta preslika poltrak $AD$ v poltrak $A'D'$. Ker ležita $B$ in
$B'$  na poltrakih $AD$ in $A'D'$ in velja $AB\cong A'B'$, je
tudi $\mathcal{I}(B)=B'$. Torej $\mathcal{I}:A,B,C \mapsto
A',B',C'$, kar pomeni, da je $\triangle ABC\cong\triangle A'B'C'$.

\item \res{Če je $ABCD$ pravokotnik ter $AQB$ in $APD$ pravilna
  trikotnika z enako orientacijo, je daljica $PQ$
skladna z diagonalo tega pravokotnika. Dokaži.}

Dokažemo, da sta $PAQ$ in $DAB$ skladna trikotnika.

\item \res{Naj bosta $BB'$ in $CC'$ višini trikotnika $ABC$ ($AC>AB$) ter
 $D$ takšna točka poltraka $AB$, da velja $AD\cong AC$. Točka
$E$ je presečišče premice $BB'$ s premico, ki poteka skozi točko $D$ in je
vzporedna s premico $AC$. Dokaži, da je $BE=CC'-BB'$.}

Naj bo $F$ presečišče premice $ED$ s premico, ki je v točki $B'$
vzporedna s premico $AB$. Štirikotnik $ADFB'$ je paralelogram,
zato je $FB'\cong DA$ in $\angle DFB' \cong\angle DAB'$ (izrek
\ref{paralelogram}). Ker je še po predpostavki $AD\cong AC$,
velja tudi $FB'\cong AC$. To pomeni, da sta pravokotna
trikotnika $FEB'$ in $AC'C$ skladna (izrek \textit{ASA} \ref{KSK}), zato
je $EB'\cong C'C$ oz.  $BE=EB'-BB'=CC'-BB'$.

\item \res{Naj bo $ABCD$ konveksni štirikotnik, pri katerem velja
 $AB\cong BC\cong CD$ in $AC\perp BD$. Dokaži, da je $ABCD$
 romb.}

 Ker je $ABCD$ konveksni štirikotnik, se njegovi diagonali
 sekata v neki točki $S$. Dokaži $\triangle ABS\cong\triangle CBS$ in
 $\triangle CBS\cong\triangle CDS$.

\item \res{Naj bo $BC$ osnovnica enakokrakega trikotnika $ABC$. Če sta $K$ in
$L$ takšni točki, da je $\mathcal{B}(A,K,B)$, $\mathcal{B}(A,C,L)$ in $KB\cong LC$, potem
središče daljice $KL$ leži na osnovnici $BC$. Dokaži.}

Naj bo $O$ središče daljice $KL$. Označimo z $M$ četrto oglišče
paralelograma $CKBM$, skupno središče njunih diagonal $BC$ in
$KM$ (izrek \ref{paralelogram}) pa s $S$. Če uporabimo dejstvo,
da sta trikotnika $ABC$ in $MLC$ enakokraka z osnovnicama $BC$
in $ML$, iz izreka \ref{enakokraki} dobimo $\angle ABC\cong\angle
ACB$ in $\angle CML\cong\angle CLM$. Iz skladnosti trikotnikov
$BKC$ in $CMB$ (izreka \ref{paralelogram} in \textit{SSS}
\ref{SSS}) pa sledi $\angle ABC=\angle KBC\cong\angle MCB$. Kot
$ACM$ je zunanji kot trikotnika $MCL$, zato je po izreku
\ref{zunanjiNotrNotr} $\angle ACM =\angle CML+\angle CLM=2\angle
CML$. Torej $\angle CML=\frac{1}{2}\angle
ACM=\frac{1}{2}\left(\angle ACB+\angle
BCM\right)\frac{1}{2}\left(\angle ACB+\angle ABC\right)=\angle
ABC$. To pomeni, da sta premici $ML$ in $BC$ vzporedni (izrek
\ref{KotiTransverzala}). Daljica $OS$ je srednjica trikotnika
$KML$ z osnovnico $ML$, zato je po izreku \ref{srednjicaTrik}
$SO\parallel MN$. Po Playfairovem aksiomu \ref{Playfair} sta
$SO$ in $BC$ ista premica (sovpadata), zato točka $O$ leži na premici $BC$.
Po Paschevem aksiomu \ref{AksPascheva} za trikotnik $ABC$ in
premico $KL$ točka $O$ leži na stranici $BC$.

\item \res{Naj bo $S$ središče trikotniku $ABC$ včrtane krožnice.
Premica, ki poteka skozi točko $S$ in je vzporedna s stranico $BC$
tega trikotnika, seka stranici $AB$ in $AC$ po vrsti v točkah
$M$ in $N$. Dokaži, da je $BM+NC=NM$.}

Dokažemo, da sta $BSM$ in $SCN$ enakokraka trikotnika z
osnovnicama $BS$ in $SC$.

\item \res{Naj bo $ABCDEFG$ konveksni sedemkotnik. Izračunaj vsoto
konveksnih kotov, ki jih določa lomljenka $ACEGBDFA$.}

Uporabimo dejstvo, da je dvakratna vsota zunanjih kotov
sedemkotnika, ki jih določajo presečišča diagonal sedemkotnika
$ABCDEFG$, enaka $720^0$. Rezultat: $540^0$.

\item \label{nalSkl34}
\res{Dokaži, da so središča stranic in nožišče poljubne višine trikotnika,
v katerem nobeni dve stranici nista skladni, oglišča
enakokrakega trapeza.}

Označimo z $A_1$, $B_1$ in $C_1$ središča stranic $BC$, $CA$ in
$BA$ trikotnika $ABC$ ter $A'$ nožišče višine tega trikotnika iz
oglišča $A$. Iz $|AB|\neq |AC|$ sledi $A'\neq A_1$. Dokažimo, da
je štirikotnik $A'A_1B_1C_1$ enakokraki trapez. Daljica $B_1C_1$
je srednjica trikotnika $ABC$ z osnovnico $BC$, zato je po
izreku \ref{srednjicaTrik} $B_1C_1\parallel BC$. Ker $A',A_1\in
BC$, je tudi $B_1C_1\parallel A',A_1$. Torej je štirikotnik
$A'A_1B_1C_1$ trapez. Dokažimo še, da je enakokrak oz.
$C_1A'\cong B_1A_1$. Toda to dejstvo sledi iz izrekov
\ref{TalesovIzrKroz2} in \ref{srednjicaTrik}, ker je:
$C_1A'=\frac{1}{2}AB= B_1A_1$.

Omenimo še, da lahko z uporabo trditve iz te naloge še na drug
način dokažemo izrek o Eulerjevi krožnici \ref{EulerKroznica}.

 \item \res{Naj bo $ABC$ pravokotni trikotnik s pravim kotom pri oglišču $C$.
Točki $E$ in $F$ naj bosta presečišči simetral notranjih kotov pri
ogliščih $A$ in $B$ z nasprotnima katetama,  $K$ in $L$ pa
pravokotni projekciji točk $E$ in $F$ na hipotenuzo tega
trikotnika. Dokaži, da je $\angle LCK=45^0$.}

Označimo z $\alpha$ in $\beta$ notranja kota pri ogliščih $A$ in
$B$ trikotnika $ABC$. Po izreku \ref{VsotKotTrik} je
$\alpha+\beta=90^0$. Po izreku \textit{ASA} \ref{KSK} velja
$\triangle ACE\cong\triangle AKE$ in $\triangle BCF\cong\triangle BLF$, zato
je $EC\cong EK$ in $FC\cong FL$. Torej sta trikotnika $CEK$ in $CFL$
 enakokraka z osnovnicama $CK$ in $CL$, kar pomeni, da je $\angle ECK\cong\angle EKC$ in
$\angle FCL\cong\angle FLC$. Če
uporabimo izrek \ref{zunanjiNotrNotr} za trikotnik $CFL$ in
izrek \ref{VsotKotTrik} za trikotnik $ALF$, dobimo: $\angle
FCL=\frac{1}{2}\angle LFA=\frac{1}{2}\left(90^0-\alpha \right)$.
Analogno je tudi $\angle ECK=\frac{1}{2}\left(90^0-\beta
\right)$. Iz tega sledi $\angle FCL+\angle
ECK=\frac{1}{2}\left(90^0-\alpha +90^0-\beta\right)=45^0$. Torej
$\angle LCK=90^0-(\angle FCL+\angle ECK)=90^0-45^0=45^0$.

\item \res{Naj bo $M$ središče stranice $CD$ kvadrata $ABCD$ in $P$ takšna točka
 diagonale $AC$, da velja $3AP=PC$. Dokaži, da je $\angle BPM$
pravi kot.}

Naj bo $S$ središče diagonale $AC$ in $V$ središče daljice $SB$. Dokažemo, da je $V$
višinska točka trikotnika $PBC$ (glej še zgled
\ref{zgledPravokotnik}).

 \item \res{Naj bodo $P$, $Q$ in $R$ središča stranic $AB$, $BC$ in $CD$
  paralelograma $ABCD$. Premici $DP$ in $BR$ naj sekata daljico
$AQ$ v točkah $K$ in $L$. Dokaži, da je $KL= \frac{2}{5} AQ$.}

Naj bo $S$ središče daljice $AD$ in $M$ presečišče daljic $SC$
in $BR$. Dokaži najprej, da je $CM\cong AK$, daljica $LQ$
srednjica trikotnika $CBM$ in daljica $PK$ srednjica trikotnika
$LAB$.

 \item  \res{Naj bo $D$ središče hipotenuze $AB$ pravokotnega
trikotnika $ABC$ ($AC>BC$). Točki $E$ in $F$ naj bosta presečišči
poltrakov $CA$ in $CB$ s premico, ki poteka skozi $D$ in je pravokotna
na premico $CD$. Točka $M$ naj bo središče daljice $EF$. Dokaži, da
je $CM\perp AB$.}

Označimo s $T$ presečišče premic $AB$ in $CM$ ter $\angle
CAB=\alpha$ in $\angle CBA=\beta$. Po predpostavki je
$\alpha+\beta=180^0$. Ker je $D$ središče hipotenuze $AB$
pravokotnega trikotnika $ABC$, je po Talesovem izreku
\ref{TalesovIzrKroz2} $DC\cong DA$. To pomeni, da je $\triangle
CDA$ enakokraki trikotnik in velja $\angle DCE=\angle
DCA\cong\angle DAC=\alpha$. Iz $FE\perp CD$ in $FC\perp CE$
sledi $\angle CFE\cong\angle DCE=\alpha$ (izrek
\ref{KotaPravokKraki}). Točka $M$ je središče hipotenuze $FE$
pravokotnega trikotnika $FCE$, zato je (podobno kot pri
trikotniku $ABC$) $\angle FCM\cong\angle CFM=\angle CFE=\alpha$.
V trikotniku $BCT$ je vsota dveh notranjih kotov $\angle
CBT+\angle BCT=\angle CBA+\angle FCM=\alpha+\beta=90^0$. Po
izreku \ref{VsotKotTrik} je torej $\angle CTB=90^0$ oz.
$CM\perp AB$.

\item \res{Naj bosta $A_1$ in $C_1$ središči stranic $BC$ in $AB$ trikotnika $ABC$.
 Simetrala notranjega kota pri oglišču $A$ seka daljico
$A_1C_1$ v točki $P$. Dokaži, da je $\angle APB$  pravi kot.}

Po izreku \ref{KotiTransverzala}, je $\angle C_1PA\cong\angle
PAC$. Torej $\angle C_1PA\cong\angle PAC\cong\angle PAC_1$, kar
pomeni, da je $PAC_1$ enakokraki trikotnik z osnovnico $AP$ oz.
$C_1A\cong C_1P$ (izrek \ref{enakokraki}). Ker je $C_1$ središče
stranice $AB$, je $C_1B\cong C_1A\cong C_1P$. Torej
točka $P$ leži na krožnici s premerom $AB$, zato je po Talesovem
izreku \ref{TalesovIzrKroz2} $\angle APB=90^0$.

 \item \res{Naj bosta $P$ in $Q$ takšni točki stranic $BC$ in $CD$ kvadrata $ABCD$,
  da je premica $PA$ simetrala kota $BPQ$. Določi velikost kota
  $PAQ$.}

  Naj bo $A'=pr_{\perp PQ}(A)$. Dokaži najprej, da velja $\triangle
  ABP\cong\triangle AA'P$ in $\triangle ADQ\cong\triangle AA'Q$.
  Rezultat: $\angle PAQ=45^0$.

\item \res{Dokaži, da  središče očrtane krožnice leži najbliže najdaljši stranici
trikotnika.}

Naj bo $O$ središče očrtane krožnice trikotnika $ABC$ ter $A_1$,
$B_1$ in $C_1$ središča njegovih stranic $BC$, $AC$ in $AB$.
Dovolj je dokazati, da npr. iz $AC>AB$ sledi $OB_1<OC_1$.
Predpostavimo, da velja $AC>AB$. Po izreku \ref{vecstrveckot} je
v tem primeru $\angle ABC>\angle ACB$. Ker je $B_1C_1$ srednjica
trikotnika $ABC$, je $B_1C_1\parallel BC$ (izrek
\ref{srednjicaTrik}). Zato je po izreku \ref{KotiTransverzala}
$\angle AC_1B_1\cong \angle ABC$ in $\angle AB_1C_1\cong \angle
ACB$. Iz $\angle ABC>\angle ACB$ sedaj sledi: $\angle
B_1C_1O=90^0-\angle AC_1B_1= 90^0-\angle ABC<90^0-\angle
ACB=90^0-\angle AB_1C_1=\angle C_1B_1O$ oz. $\angle
B_1C_1O<\angle C_1B_1O$. Po izreku \ref{vecstrveckot}  za
trikotnik $B_1OC_1$ je potem $OB_1<OC_1$.

 \item \res{Dokaži, da je središče včrtane krožnice najbliže oglišču, ki je vrh
največjega notranjega kota trikotnika.}

Označimo z $\alpha$, $\beta$ in $\gamma$ notranje kote pri
ogliščih $A$, $B$ in $C$ trikotnika $ABC$, $S$ središče včrtane
krožnice tega trikotnika ter s $P$ točko, v kateri se ta krožnica
dotika stranice $BC$. Dovolj je dokazati, da npr. iz
$\beta>\gamma$ sledi $SB<SC$. Iz predpostavke $\beta>\gamma$
sledi najprej $\angle
SBP=\frac{1}{2}\beta>\frac{1}{2}\gamma=\angle SCP$. Naj bo $B'$
točka na poltraku $PC$, za katero velja $PB'\cong PB$. Iz
skladnosti trikotnikov $SPB'$ in $SPB$ (izrek \textit{SAS}
\ref{SKS}) sledi $SB'\cong SB$ in $\angle SB'P\cong \angle
SBP>\angle SCP$. Dokažimo, da velja $\mathcal{B}(P,B',C)$ oz.
$PB'<PC$. V nasprotnem bi iz $B'=B$ sledilo $\beta=\gamma$
(skladnost trikotnikov $SBP$ in $SCP$ - izrek \textit{SAS}
\ref{SKS}), iz $\mathcal{B}(P,C,B')$ pa $\beta<\gamma$ (v tem
primeru bi bilo $\angle SCP>\angle SB'P\cong\angle SBP$ - izrek
\ref{zunanjiNotrNotrVecji} za trikotnik $SCB'$). Ker je torej
$\mathcal{B}(P,B',C)$, je $\angle SB'C=180^0-\angle
SB'P=180^0-\frac{1}{2}\beta>90^0$. V topokotnem trikotniku
$SB'C$ je potem $SC<SB'\cong SB$ (izrek \ref{vecstrveckot}).

\item \res{Naj bo $ABCD$ konveksen štirikotnik. Določi točko $P$, tako da
bo vsota $AP+BP+CP+DP$ minimalna.}

S pomočjo trikotniške neenakosti (izrek \ref{neenaktrik})
dokažemo, da je točka $P$ presečišče diagonal tega štirikotnika.

 \item \res{Diagonali $AC$ in $BD$ enakokrakega trapeza $ABCD$ z osnovnico $AB$
se sekata v točki $O$ in velja $\angle AOB=60^0$. Točke $P$, $Q$
in $R$ so po vrsti središča daljic $OA$, $OD$ in $BC$. Dokaži, da
je $PQR$ enakostranični trikotnik.}

Najprej dokažemo, da sta $AOB$ in $COD$ enakostranična trikotnika,
nato pa še relacijo $PQ=\frac{1}{2}AD$ in dejstvo, da sta $CPB$
in $CQB$ pravokotna trikotnika s skupno hipotenuzo $BC$.

\item \res{Naj bo $P$ poljubna notranja točka trikotnika $ABC$, za katero velja
 $\angle PBA\cong \angle PCA$. Točki $M$ in $L$ sta pravokotni
projekciji točke $P$ na stranicah $AB$ in $AC$, točka $N$ pa
središče stranice $BC$. Dokaži, da je $NM\cong
NL$\footnote{Predlog za MMO 1982 (SL 9.).}.}

 Označimo $\angle PBA\cong \angle PCA=\varphi$. Naj bosta $Q$ in
 $R$ središči daljic $PC$ in $PB$. Daljica $NQ$ je srednjica
 trikotnika $PBC$ z osnovnico $BP$, zato je po izreku
 \ref{srednjicaTrik} $NQ\parallel BP$ in
 $|NQ|=\frac{1}{2}|BP|=|RP|$ oz. $NQ\parallel RP$ in $NQ\cong
 RP$. Iz izreka \ref{paralelogram} sledi, da je štirikotnik
 $NQPR$ paralelogram. Po istem izreku je tudi $RN\cong PQ$ in
 $\angle PRN\cong\angle PQN$. Torej imamo:
    \begin{eqnarray} \label{relacijaNalSkl42a}
      NQ\cong
     RP,\hspace*{2mm} NR\cong QP\hspace*{1mm} \textrm{ in }
     \hspace*{1mm}\angle PRN\cong\angle PQN
     \end{eqnarray}
  Daljica $RM$ je težiščnica pravokotnega trikotnika $BPM$ s
  hipotenuzo $BP$, zato je po Talesovem izreku
  \ref{TalesovIzrKroz2} $MR\cong RP\cong RB$. Trikotnik $MBR$ je
  enakokraki trikotnik z osnovnico $MB$ in je $\angle
  BMR=\angle MBR=\varphi$ (izrek \ref{enakokraki}. Za zunanji
  kot $MRP$ trikotnika $MBR$ po izreku \ref{zunanjiNotrNotr}
  velja $\angle MRP=\angle BMR+\angle RBM=2\varphi$. Torej
  velja:
 \begin{eqnarray} \label{relacijaNalSkl42b}
      MR\cong RP  \hspace*{1mm}\textrm{ in }
     \hspace*{1mm}\angle MRP=2\varphi
     \end{eqnarray}
Analogno iz pravokotnega trikotnika $CPL$ in enakokrakega
trikotnika $LQC$ dobimo:
 \begin{eqnarray} \label{relacijaNalSkl42c}
      LQ\cong QP \hspace*{1mm} \textrm{ in }
     \hspace*{1mm}\angle LQP=2\varphi
     \end{eqnarray}
 Iz relacij  \ref{relacijaNalSkl42a},  \ref{relacijaNalSkl42b}
 in  \ref{relacijaNalSkl42c} je: $MR\cong RP \cong NQ$, $NR\cong
 QP \cong LQ$ in $\angle MRP\cong\angle LQP=2\varphi$. Torej:
   \begin{eqnarray} \label{relacijaNalSkl42d}
      MR \cong NQ,\hspace*{2mm} NR \cong LQ\hspace*{1mm} \textrm{ in }
     \hspace*{1mm}\angle MRP\cong\angle LQP
     \end{eqnarray}
 Iz  tretje relacije iz \ref{relacijaNalSkl42a} in tretje
 relacije iz \ref{relacijaNalSkl42d} dobimo $\angle MRN=\angle
 MRP+\angle PRN=\angle LQP+\angle PQN=\angle LQN$. Če to
 povežemo s prvima dvema relacijama iz \ref{relacijaNalSkl42d},
 dobimo:
 \begin{eqnarray} \label{relacijaNalSkl42e}
      MR \cong NQ,\hspace*{2mm} NR \cong LQ\hspace*{1mm} \textrm{ in }
     \hspace*{1mm}\angle MRN\cong\angle LQN
     \end{eqnarray}
 To pomeni, da sta  po izreku \textit{SAS} \ref{SKS} trikotnika
 $MRN$ in $NQL$ skladna, zato je tudi $NM\cong
NL$.

\item \res{Naj bodo $P$, $Q$ in $R$ središča stranic $BC$, $AC$ in $AB$
 trikotnika $ABC$ ($AB<AC$) ter $D$ nožišče višine iz oglišča
$A$. Dokaži, da je $\angle DRP\cong \angle DQP=\angle ABC-\angle ACB$.}

Po nalogi \ref{nalSkl34} je štirikotnik $DPQR$ enakokraki trapez
z osnovnico $DP$. Iz skladnosti trikotnikov $DPR$ in $PDQ$
(izrek \textit{SSS} \ref{SSS}) je $\angle DRP\cong\angle DQP$.
Dokažemo še $\angle DQP=\angle ABC-\angle ACB=\beta-\gamma$.
Štirikotnik $BPQR$ je paralelogram (izrek \ref{srednjicaTrik}),
zato je $\angle RQP=\angle ABC=\beta$. Ker je $\angle ADP=90^0$,
je po Talesovem izreku \ref{TalesovIzrKroz2} $QD\cong QC\cong
QA$. Torej $DCQ$ je enakokraki trikotnik z osnovnico $DC$ in
velja $\angle QDC\cong \angle QCD=\gamma$ (izrek
\ref{enakokraki}). Iz $DP\parallel RQ$ po izreku
\ref{KotiTransverzala} je $\angle RQD\cong \angle QDP$. Na koncu
je: $\angle DQP=\angle RQP-\angle DQP=\beta-\angle
QDP=\beta-\angle QDC=\beta-\gamma$.

\item \res{Naj bo $AD$ simetrala notranjega kota pri oglišču $A$ ($D\in BC$) trikotnika
$ABC$ in $E$ takšna točka stranice $AB$, da velja $\angle
BDE\cong\angle BAC$. Dokaži, da je $DE\cong DC$.}

Naj bo $F$ takšna točka poltraka $AC$, da velja $AF\cong AE$.
Najprej dokažemo, da je $\triangle AED\cong\triangle AFD$ in $\triangle
FDC$ enakokraki trikotnik z osnovnico $FC$.

\item \res{Naj bo $O$ središče kvadrata $ABCD$ ter $P$, $Q$ in $R$ točke,
 ki razdelijo njegov obseg na tri enake dele. Dokaži, da se
minimum vsote $|OP|+|OQ|+|OR|$ doseže, kadar je ena od teh točk
središče stranice kvadrata.}


Naj bodo $P$, $Q$ in $R$ poljubne točke, ki razdelijo obseg
 kvadrata $ABCD$ na tri enake dele. Jasno je, da v tem primeru
 te tri točke ležijo po vrsti na treh stranicah kvadrata. Brez
 škode za splošnost predpostavimo, da je $P\in [AB]$, $Q\in
 [BC]$ in $R\in [AD]$. Naj bo $P_0$ središče daljice $BC$ in
 $Q_0\in [BC]$ in $R_0\in [AD]$ takšni točki, da tudi trojica
 $P_0$, $Q_0$ in $R_0$  razdeli obseg kvadrata $ABCD$ na tri
 enake dele. V tem primeru je jasno $CQ_0\cong DR_0$, zato iz
 skladnosti trikotnikov $COQ_0$ in $DOR_0$ (izrek \textit{SAS}
 \ref{SKS}) sledi $OQ_0\cong OR_0$ in $\angle BQ_0O \cong\angle
 AR_0O$ (ustrezna zunanja kota teh trikotnikov).
  Brez škode za splošnost lahko predpostavimo, da je
 $\mathcal{B}(P_0,P,C)$. Dokažemo, da je v tem primeru
 $|OP_0|+|OQ_0|+|OR_0|<|OP|+|OQ|+|OR|$. Ker trojici $P$, $Q$ in
 $R$ ter $P_0$, $Q_0$ in $R_0$ obe razdelita obseg kvadrata
 $ABCD$ na tri enaka dela, je $|P_0P|=|Q_0Q|=|R_0R|=d$, pri tem
 pa velja še $\mathcal{B}(Q_0,Q,C)$ in $\mathcal{B}(D,R_0,R)$.
 Naj bo $R'$ takšna točka stranice $CD$, da velja $CR'\cong DR$.
 V tem primeru je  $Q_0R'=CR'-CQ_0=DR-DR_0=R_0R=Q_0Q=d$, kar
 pomeni, da je daljica $OQ_0$ težiščnica trikotnika $OR'Q$. Iz
 skladnosti trikotnikov $OQ_0R'$ in $OR_0R$ (izrek \textit{SAS}
 \ref{SKS}) sledi še $OR\cong OR'$. Če uporabimo trditev iz
 zgleda \ref{neenTezisZgl}, dobimo
 $OQ_0<\frac{1}{2}\left(|OQ|+|OR'| \right)$, oz.
 $|OQ_0|+|OR_0|=2|OQ_0|<|OQ|+|OR'|=|OQ|+|OR|$. Ker je iz
 pravokotnega trikotnika $OP_0P$  še $|OP_0|<|OP|$ (izrek
 \ref{vecstrveckot}), je na koncu
 $|OP_0|+|OQ_0|+|OR_0|<|OP|+|OQ|+|OR|$.

\item \res{Dano je končno število premic, ki ravnino razdelijo na območja.
Dokaži, da lahko ravnino pobarvamo z dvema barvama, tako da je
vsako območje pobarvano z eno barvo, sosednji območji pa vedno z
različnima barvama.}

Dokaz izpeljmeo z indukcijo po številu premic. Za $n=1$ ena
premica razdeli ravnino na dve območji in je trditev
izpolnjena.

Predpostavimo, da trditev velja za vsakih $n$ poljubnih premic
oz. da za njih obstaja ustrezno barvanje. Dokažemo, da trditev
velja za $n+1$. Naj bodo $p_1,p_2,\ldots , p_n,p_{n+1}$ poljubne
premice v ravnini $\alpha$. Premica $p_{n+1}$ določa polravnini
$\alpha_1$ in $\alpha_2$. Obravnavajmo območja, ki jih določajo
premice $p_1,p_2,\ldots , p_n$, torej brez premice $p_{n+1}$. Po
indukcijski predpostavki lahko ta območja  pobarvamo z dvema
barvama, tako da je vsako območje pobarvano z eno barvo,
sosednji območji pa vedno z različnima barvama. Omenjeno
barvanje označimo z $\mathcal{B}_n$. V tem primeru ustrezno
barvanje $\mathcal{B}_{n+1}$ območij, ki jih določajo premice
$p_1,p_2,\ldots , p_n,p_{n+1}$, definiramo na naslednji način:
\begin{itemize}
  \item območja, ki so v polravnini $\alpha_1$, ohranijo
isto barvo, ki je določena z barvanjem $\mathcal{B}_n$,
  \item območja, ki so v polravnini $\alpha_2$, spremenijo
 barvo, ki je določena z barvanjem $\mathcal{B}_n$.
\end{itemize}
Barvanje $\mathcal{B}_{n+1}$ izpolnjuje dane pogoje. V primeru,
ko imata sosednji območji mejno stranico, ki leži na premici
$p_{n+1}$, sta po definiciji barvanja  $\mathcal{B}_{n+1}$
območji različne barve. V primeru, ko sosednji območji nimata
 mejne stranice, ki leži na premici $p_{n+1}$, ležita obe v
 polravnini $\alpha_1$ ali oba v polravnini $\alpha_2$. Tudi v
 tem primeru sta območji različno pobarvani, kar sledi direktno
 iz definicij barvanj $\mathcal{B}_n$ in $\mathcal{B}_{n+1}$.


\item \res{Načrtaj trikotnik $ABC$, če so dani podatki (glej oznake v razdelku \ref{odd3Stirik}):} \label{nalSklKonstrTrik}

(\textit{a}) \res{$\alpha$, $\beta$, $s$}

 Naj bosta $E$ in $F$ takšni točki, da velja $\mathcal{B}(E,A,B,F)$, $EA\cong CA$ in $FB\cong CB$.  To omogoča konstrukcijo trikotnika $ECF$ ($EF\cong s$, $\angle CEF = \frac{1}{2}\cdot \alpha$ in $\angle CFE = \frac{1}{2}\cdot \beta$).

(\textit{b}) \res{$a-b$, $c$, $\gamma$}

 Naj bo $D$ takšna točka, da velja $\mathcal{B}(C,D,B)$ in $CD\cong CA$. Najprej načrtamo trikotnik $ADB$ ($AB\cong c$, $DB=a-b$ in $\angle ADB=90^0+\frac{1}{2}\cdot\gamma$).

(\textit{c}) \res{$a$, $\beta-\gamma$, $b-c$}

 Označimo z $D$ tisto točko, da velja $\mathcal{B}(A,D,C)$ in $AD\cong AB$. Najprej načrtamo trikotnik $DBC$ ($BC\cong a$, $DC=b-c$ in $\angle DBC=\frac{1}{2}\cdot\left(\beta-\gamma\right)$).

(\textit{d}) \res{$a$, $\beta-\gamma$, $b+c$}

 Naj bo $D$ takšna točka, da velja $\mathcal{B}(B,A,D)$ in $AD\cong AC$. To omogoča najprej konstrukcijo trikotnika $DBC$ ($BC\cong a$, $DB=b+c$ in $\angle DCB = 90^0-\frac{1}{2}\cdot\left(\beta-\gamma\right)$).

(\textit{e}) \res{$b$, $c$, $v_a$}

 Najprej načrtamo poljubno premico $p$ in točko $A$, ki je od premice $p$ oddaljena $v_a$, nato pa še točki $B\in k(A,c)$ in $C\in k(A,b)$.

(\textit{f}) \res{$b$, $v_a$, $v_b$}

 Naj bo $A'=pr_{\perp BC}(A)$. Najprej načrtamo pravokotni trikotnik $AA'C$ s hipotenuzo $AC=b$ in kateto $AA'=v_a$, nato pa še oglišče $B$ kot presečišče premice $A'C$ z vzporednico $l$ premice $AC$ na razdalji $v_b$.

(\textit{g}) \res{$\alpha$, $v_a$, $v_b$}

Naj bo $B'=pr_{\perp AC}(B)$. Najprej načrtamo pravokotni trikotnik $ABB'$ (iz pogojev: $\angle AB'B=90^0$, $\angle BAB'\cong\alpha$ in $BB'=v_b$), nato pa oglišče $C$ kot presečišče premice $AB'$ in tangente krožnice $k(A,v_a)$ iz točke $B$.

(\textit{h}) \res{$c$, $a+b$, $\gamma$}

 Glej zgled \ref{konstrTrik1}.

(\textit{i}) \res{$v_a$, $\alpha$, $\beta$}

 Naj bo $A'=pr_{\perp BC}(A)$. Najprej načrtamo pravokotni trikotnik $ABA'$ (iz pogojev: $\angle ABB'\cong\beta$, $AA'=v_a$ in $\angle AA'B=90^0$), nato pa oglišče $C$ kot presečišče premice $BB'$ in drugega kraka $p$ kota $\angle BA,p=\alpha$.

(\textit{j}) \res{$b$, $a+c$, $v_c$}

 Naj bo $C'=pr_{\perp AB}(C)$ in $D$ takšna točka, da velja $\mathcal{B}(A,B,D)$ in $BD\cong BC$. Načrtamo po vrsti: daljico $AD=a+c$, vzporednico $l$ premice $AD$ na razdalji $v_c$, $C\in l\cap k(A,b)$ in na koncu točko $B$ kot presečišče premice $AD$ in simetrale daljice $CD$.

(\textit{k}) \res{$b-c$, $v_b$, $\alpha$}


 Naj bo $B'=pr_{\perp AC}(B)$ in $D$ takšna točka, da velja $\mathcal{B}(A,D,C)$ in $AD\cong AB$. Najprej načrtamo pravokotni trikotnik $ABA'$ ($\angle BAB'\cong\alpha$, $BB'=v_b$ in $\angle AB'B=90^0$), nato točko $D$ na poltraku $AC$ s pogojem $AD\cong AB$ in na koncu takšno točko $C$, da velja $\mathcal{B}(A,D,C)$ in $DC=b-c$.

(\textit{l}) \res{$a$, $t_b$, $t_c$}

 Naj bosta $BB_1$ in $CC_1$ težiščnici in $T$ težišče trikotnika $ABC$. Najprej narišemo trikotnik $BTC$ ($BC\cong a$, $BT=\frac{2}{3}\cdot t_b$ in $CT=\frac{2}{3}\cdot t_c$ - izreka \ref{tezisce} in \ref{izrekEnaDelitevDaljice}), nato točki $B_1$ in $C_1$ na poltrakih $BT$ in $CT$ iz pogojev $BB_1\cong t_b$ in $CC_1\cong t_c$ ter na koncu točko $A$ kot presečišče poltrakov $BC_1$ in $CB_1$.

(\textit{m}) \res{$b$, $c$, $t_a$}

 Naj bo $AA_1$ težiščnica trikotnika $ABC$ in $D$ točka, ki je simetrična točki $A$ glede na točko $A_1$. Štirikotnik $ABA_1C$ je paralelogram (izrek \ref{paralelogram}), kar omogoča najprej konstrukcijo trikotnika $ADC$ ($AC\cong b$, $CD\cong c$ in $AD=2\cdot t_a$), nato pa točke $B$, ki je simetrična točki $C$ glede na točko $A_1$.

 (\textit{n}) \res{$t_a$, $t_b$, $t_c$}

  Naj bo $AA_1$ težiščnica in $T$ težišče trikotnika $ABC$ ter $D$ točka, ki je simetrična točki $T$ glede na točko $A_1$. Štirikotnik je paralelogram (izrek \ref{paralelogram}). Po izreku \ref{tezisce} je $TC= \frac{2}{3}\cdot t_c$, $CD\cong BT= \frac{2}{3}\cdot t_b$ in $TD=2\cdot TA_1=\frac{2}{3}\cdot t_a$. To omogoča konstrukcijo najprej trikotnika $TDC$, nato še oglišč $B$ in $A$.

(\textit{o}) \res{$c$, $v_a$, $l_a$}

Naj bosta $AA'$ višina in $AE$ ($E\in BC$) simetrala notranjega kota $BAC$ trikotnika $ABC$. Načrtamo najprej pravokotni trikotnik $AA'E$ ($AA'\cong v_a$, $\angle AA'E=90^0$ in $AE=l_a$), nato oglišče $B\in AE\cap k(A,c)$ in oglišče $C$ kot presečišče premice $AE$ in drugega kraka $p$ kota $\angle EA,p\cong\angle BAE$.

 (\textit{p}) \res{$c$, $v_a$, $t_b$}

 Naj bo $AA'$ višina, $BB_1$ težiščnica trikotnika $ABC$ in $B_2=pr_{\perp BC}(B_1)$. Daljica $B_1B_2$ je srednjica trikotnika $AA'C$, zato je $B_1B_2=\frac{1}{2}\cdot AA'$ (izrek \ref{srednjicaTrik}). Sedaj konstruiramo pravokotni trikotnik $ABA'$ ($AA'\cong v_a$, $\angle AA'B=90^0$ in $AB=c$). Točka $B_1$ naj bo presečišče krožnice $k(B, t_b)$ in vzporednice $l$ premice $BA'$ na razdalji $\frac{1}{2}\cdot v_b$. Oglišče $C$ dobimo kot presečišče premic $BA'$ in $AB_1$.

 (\textit{r}) \res{$b$, $l_a$, $\alpha$}

 Naj bo $AE$ ($E\in BC$) simetrala notranjega kota $BAC$ trikotnika $ABC$. Načrtamo najprej trikotnik $AEC$ ($AE\cong l_a$, $\angle AEC=\frac{1}{2}\cdot\alpha$ in $AC=b$), nato oglišče $A$ kot presečišče premice $CE$ in drugega kraka $p$ kota $EA,p=\frac{1}{2}\cdot\alpha$.

 (\textit{s}) \res{$v_a$, $v_b$, $t_a$}

 Naj bodo $AA'$ in $BB'$ višini in $AA_1$ težiščnica trikotnika $ABC$ ter $A_2=pr_{\perp AC}(A_1)$. Daljica $A_1S_2$ je srednjica trikotnika $BB'C$, zato je $A_1A_2=\frac{1}{2}\cdot BB'$ (izrek \ref{srednjicaTrik}). Sedaj konstruiramo  pravokotni trikotnik $AA'A_1$ ($AA'\cong v_a$, $\angle AA'A_1=90^0$ in $AA_1=t_a$). Oglišče $C$ dobimo kot presečišče premice $A'A_1$ in tangente krožnice $k(A_1,\frac{1}{2}\cdot v_b)$ iz točke $A$. Oglišče $B$ je točka, ki je simetrična oglišču $C$ glede na točko $A_1$.

(\textit{t}) \res{$t_a$, $v_b$, $b+c$}

Naj bodo $A_1$ središče stranice $BC$, $D$ točka, ki je simetrična točki $A$ glede na točko $A_1$, in $E$ takšna točka, da velja $\mathcal{B}(D,B,E)$ in $EB\cong AB$. Ker je $ABDC$ paralelogram (izrek \ref{paralelogram}), je $BD\cong AC$. Naprej je $AD=2\cdot AA_1=2\cdot t_a$, $DE=BD+BE=AC+AB=b+c$, točka $A$ je od premice $ED$ oddaljena $v_b$. Ta dejstva omogočajo konstrukcijo trikotnika $AED$. Oglišče $B$ nato dobimo kot presečišče simetrale daljice $AE$ in premice $ED$, oglišče $C$ pa je četrto oglišče paralelograma $ABDC$.

 (\textit{u}) \res{$a$, $b$, $\alpha-\beta$}

 Naj bo $D$ takšna točka, da velja $\mathcal{B}(C,D,B)$ in $CD\cong CA$. Potem je $BD=a-b$ in $\angle DAB=\frac{1}{2}\left(\alpha-\beta \right)$. Ta dejstva omogočajo konstrukcijo. Najprej načrtamo točke $C$, $B$ in $D$ iz pogojev $\mathcal{B}(C,D,B)$, $CB\cong a$ in $CD\cong b$. Oglišče $A$ je presečišče krožnice $k(C,b)$ in geometrijskega mesta točk, iz katerih se daljica $BD$ vidi pod kotom $\frac{1}{2}\left(\alpha-\beta \right)$ (izrek \ref{ObodKotGMT}).


            \item \res{Načrtaj enakokraki trikotnik $ABC$ , če so dani:}
V vseh primerih predpostavimo, da je $BC$ osnovnica tega trikotnika.

        (\textit{a}) \res{Osnovnica ter vsota kraka in višine na osnovnico.}

Naj bo $A'$ nožišče višine iz oglišča $A$.
Načrtamo najprej trikotnik $DA'B$, kjer je $A'B=\frac{1}{2}a$, $DA'=v_a+b$ in $\angle DA'B=90^0$.

        (\textit{b}) \res{Obseg in višina na osnovnico.}

Naj bo $A'$ nožišče višine iz oglišča $A$.
Najprej načrtamo trikotnik $AA'P$, kjer je $A'P=s$, $AA'=v_a$ in $\angle AA'P=90^0$.

        (\textit{c}) \res{Obe višini.}

Naj bosta $AA'=v_a$ in $BB'=v_b$ višini tega trikotnika, $D$ pa pravokotna projekcija točke $A'$ na kraku $AC$. Ker je $A'$ središče osnovnice $BC$, je $A'D$ središčnica trikotnika $BCD$ za njegovo stranico $BB'$, zato je $A'D=\frac{1}{2}v_b$. To omogoča konstrukcijo pravokotnega trikotnika $AA'D$.

        (\textit{d}) \res{Kot ob osnovnici in odsek njegove simetrale.}

Uporabimo dejstvo, da sta kota ob osnovnici skladna.

        (\textit{e}) \res{Krak in na njem nožišče pripadajoče višine.}

Denimo da je dan krak $AC$ in nožišče $B'$ višine $BB'$ na tem kraku. Oglišče $B$ dobimo kot eno od presečišč krožnice $k(A,AC)$ s pravokotnico kraka $AC$ v točki $B'$ (nosilke višine $BB'$).

        (\textit{f}) \res{Krak in pripadajoča višina.}

Najprej načrtamo pravokotni trikotnik $ABB'$, kjer je $B'$ nožišče višine $BB'$ trikotnika $ABC$.


            \item \res{Načrtaj pravokotni trikotnik $ABC$ s pravim kotom v oglišču $C$, če so dani naslednji podatki:}

 (\textit{a}) \res{$\alpha$, $a+b$}

Naj bo $D$ točka, za katero velja $\mathcal{B}(A,C,D)$ in $CD\cong CB$. Načrtaj najprej trikotnik $ABD$ ($DA=a+b$, $\angle DAB=\alpha$ in $\angle BDA=45^0$).

 (\textit{b}) \res{$\alpha$, $a-b$}

Naj bo $D$ točka, za katero velja $\mathcal{B}(A,D,C)$ in $CD\cong CB$. Načrtaj najprej trikotnik $ABD$ ($DA=a-b$, $\angle DAB=\alpha$ in $\angle BDA=135^0$).

 (\textit{c}) \res{$a$, $b+c$}

Naj bo $D$ točka, za katero velja $\mathcal{B}(C,A,D)$ in $CD\cong AB$. Načrtaj najprej trikotnik $CBD$ ($CD=b+c$, $CB=a$ in $\angle BCD=90^0$).

 (\textit{d}) \res{$c$, $a+b$}

Naj bo $D$ točka, za katero velja $\mathcal{B}(A,C,D)$ in $CD\cong CB$. Načrtaj najprej trikotnik $ABD$ ($DA=a+b$, $AB=c$ in $\angle BDA=45^0$).

 (\textit{e}) \res{$t_a$, $t_c$}

Naj bosta $AA_1=t_a$ in $CC_1=t_c$ težiščnici in $T$ težišče pravokotnega trikotnika $ABC$. Najprej načrtamo težiščnico $AA_1=t_a$, nato težišče $T$ iz pogoja $A_1T=\frac{1}{3}A_1A$, oglišče $C$ kot presečišče krožnice nad premerom $AA_1$ in krožnice $k(T,\frac{2}{3}t_c)$ in na koncu oglišče $B$, ki je simetrično oglišču $C$ glede na točko $A_1$.

 (\textit{f}) \res{$a$, $c-b$}

Naj bo $D$ točka, za katero velja $\mathcal{B}(A,C,D)$ in $AD\cong AB$. Načrtaj najprej trikotnik $CBD$ ($CD=c-b$, $BC=a$ in $\angle BCD=90^0$).

(\textit{g}) \res{$a+v_c$, $\alpha$}

Načrtamo najprej pravokotni trikotnik $CC'B$, kjer je $C'$ nožišče višine trikotnika $ABC$ iz oglišča $C$.

 (\textit{h}) \res{$t_c$, $v_c$}


Najprej načrtamo pravokotni trikotnik $CC'C_1$, kjer je $C'$ nožišče višine trikotnika $ABC$ iz oglišča $C$ in $C_1$ središče njegove stranice $AB$. Nato uporabimo dejstvo $C_1C\cong C_1A\cong C_1B$ (Talesov izrek \ref{TalesovIzrKroz2}).

 (\textit{i}) \res{$a$, $t_a$}

Naj bo $AA_1=t_a$ težiščnica trikotnika $ABC$. Najprej narišemo trikotnik $AA_1C$.

 (\textit{j}) \res{$v_c$, $l_c$}


Načrtamo najprej pravokotni trikotnik $CC'E$, kjer je $C'$ nožišče višine trikotnika $ABC$ iz oglišča $C$ in $E$ presečišče simetrale notranjega kota $ACB$ z njegovo stranico $AB$. Nato uporabimo dejstvo $\angle ECA\cong \angle ECB=45^0$.

                \item \res{Načrtaj pravokotnik $ABCD$, če je dano:}
        \begin{enumerate}
        \item \res{Diagonala in ena stranica.}

Najprej načrtamo pravokotni trikotnik $ABC$.

        \item \res{Diagonala in obseg.}

Najprej načrtamo pravokotni trikotnik $ABC$ iz pogoja $AC=d$ in $AB+BC=\frac{1}{2}o$.

        \item \res{Ena stranica in kot, ki ga oklepata diagonali.}


Najprej načrtamo enakokraki trikotnik $ABS$, kjer je $S$ presečišče diagonal pravokotnika $ABC$.

        \item \res{Obseg in kot, ki ga določata diagonali.}

Naj bo $E$ takšna točka, da velja $\mathcal{B}(C,B,E)$ in $BE\cong AB$. Najprej načrtamo trikotnik $AEC$ iz pogojev: $CE=AB+BC$, $\angle ACE=\angle ACB=\frac{1}{2}\angle ASB$ in $\angle AEC=45^0$.

        \end{enumerate}

            \item \res{Načrtaj romb $ABCD$, če je dano:}

V vseh primerih naj bo $a$ stranica, $e$ in $f$ diagonali ter $\alpha$ notranji kot $BAD$ romba $ABCD$.

        \begin{enumerate}
        \item \res{Stranica in vsota diagonal.}

Naj bo $S$ presečišče diagonal romba $ABCD$ in $E$ takšna točka, da velja $\mathcal{B}(A,S,E)$ in $SE\cong SB$. Najprej načrtamo trikotnik $ABE$ iz pogojev: $AB=a$, $AE=\frac{e+f}{2}$ in $\angle AEB=45^0$.

        \item \res{Stranica in razlika diagonal.}


Naj bo $S$ presečišče diagonal romba $ABCD$ in $E$ takšna točka, da velja $\mathcal{B}(A,E,S)$ in $SE\cong SB$. Najprej načrtamo trikotnik $ABE$ iz pogojev: $AB=a$, $AE=\frac{e-f}{2}$ in $\angle AEB=135^0$.

        \item \res{En kot in vsota diagonal.}


Naj bo $S$ presečišče diagonal romba $ABCD$ in $E$ takšna točka, da velja $\mathcal{B}(A,S,E)$ in $SE\cong SB$. Najprej načrtamo trikotnik $ABE$ iz pogojev: $\angle SAB=\frac{1}{2}\alpha$, $AE=\frac{e+f}{2}$ in $\angle AEB=45^0$.

        \item \res{En kot in razlika diagonal.}


Naj bo $S$ presečišče diagonal romba $ABCD$ in $E$ takšna točka, da velja $\mathcal{B}(A,E,S)$ in $SE\cong SB$. Najprej načrtamo trikotnik $ABE$ iz pogojev: $\angle SAB=\frac{1}{2}\alpha$, $AE=\frac{e-f}{2}$ in $\angle AEB=135^0$.

        \end{enumerate}

            \item \res{Načrtaj paralelogram $ABCD$, če je dano:}

V vseh primerih označimo z $a=AB$ in $b=BC$ stranici, $e=AC$ in $f=BD$ diagonali ter $\alpha$, $\beta$, $\gamma$, $\delta$ notranje kote ob ogliščih $A$, $B$, $C$ in $D$ paralelograma $ABCD$.

        \begin{enumerate}

        \item \res{Ena stranica in diagonali.}

Najprej načrtamo trikotnik $ASB$, kjer je $S$ presečišče diagonal paralelograma $ABCD$.

        \item \res{Ena stranica in višini.}

Predpostavimo, da je dana stranica $AB$.
Načrtamo najprej pravokotni trikotnik $ABB'$, kjer je $B'$ pravokotna projekcija oglišča $B$ na nosilko $AD$ paralelograma $ABCD$.

        \item \res{Ena diagonala in višini.}


Predpostavimo, da je dana diagonala $AC=e$.
Najprej načrtamo pravokotni trikotnik $ASS'$, kjer je $S$ presečišče diagonal paralelograma $ABCD$ in $S'$ pravokotna projekcija te točke na nosilko $AB$ ($SA'=\frac{v_a}{2}$, $\angle SA'A=90^0$ in $AS=\frac{e}{2}$). Oglišče $C$ je simetrično oglišču $A$ glede na točko $S$.
nato konstruiramo pravokotni trikotnik $ACC'$, kjer je $B'$ pravokotna projekcija oglišča $B$ na nosilko $AD$ paralelograma $ABCD$ (uporabi dejstvo $CC'=v_b$ in Talesov izrek \ref{TalesovIzrKroz2}).

        \item \res{Stranica $AB$, kot ob oglišču $A$ in vsota $BC+AC$.}

Naj bo $E$ takšna točka, da velja $\mathcal{B}(B,C,E)$ in $CE\cong CA$. Najprej načrtamo trikotnik $ABE$ iz pogojev: $AB=a$, $BE=e+b$ in $\angle ABC=180^0-\alpha$.

        \end{enumerate}

                 \item \res{Načrtaj trapez $ABCD$, če je dano:}

V vseh primerih naj bosta $a=AB$ in $c=CD$ osnovnici, $b=BC$ in $d=DA$ kraka, $e=AC$ in $f=BD$ diagonali ter $\alpha$, $\beta$, $\gamma$, $\delta$ notranji koti ob ogliščih $A$, $B$, $C$ in $D$ trapeza $ABCD$.

        \begin{enumerate}

        \item \res{Osnovnici, krak in manjši kot, ki ne leži ob tem kraku.}

Brez škode za splošnost denimo, da je dano $a=AB$, $c=CD$, $b=BC$ in $\alpha=\angle DAB$. Naj bo $E$ četrto oglišče paralelograma $BCDE$. Najprej načrtaj trikotnik $AED$ ($AE=a-c$, $\angle DAE=\alpha$ in $DE=b$).

        \item \res{Osnovnici in diagonali.}


Naj bo $E$ četrto oglišče paralelograma $CDBE$. Najprej načrtaj trikotnik $AEC$ ($AE=a+c$, $AC=e$ in $CE=f$).

        \item \res{Osnovnici in kota ob daljši osnovnici.}

Naj bo $E$ četrto oglišče paralelograma $BCDE$. Najprej načrtaj trikotnik $AED$ ($AE=a-c$, $\angle DAE=180^0-\delta$ in $\angle DEA=180^0-\gamma$).

        \item \res{Vsota osnovnic, višina in kota ob daljši osnovnici.}

Naj bo $PQ$ srednjica trapeza $ABCD$. Načrtaj najprej trapez $ABPQ$.

        \end{enumerate}



 \item \res{Načrtaj deltoid $ABCD$, če so dani: diagonala $AC$, ki leži na somernici deltoida, $\angle CAD$ in vsota $AD+DC$.}

Naj bo $E$ takšna  točka, da velja $\mathcal{B}(A,D,E)$ in $DE\cong DC$. Najprej načrtaj trikotnik $ACE$.


 \item \res{Načrtaj štirikotnik $ABCD$, če je dano:}

V vseh primerih označimo z $a=AB$, $b=BC$, $c=CD$ in $d=AD$ stranice, $e=AC$ in $f=BD$ diagonali ter $\alpha$, $\beta$, $\gamma$, $\delta$ notranje kote ob ogliščih $A$, $B$, $C$ in $D$ štirikotnika $ABCD$.

        \begin{enumerate}
        \item \res{Štiri stranice in en kot.}

Predpostavimo, da je dan kot $DAB=\alpha$. Najprej načrtaj trikotnik $DAB$.

        \item \res{Štiri stranice in kot, ki ga oklepata nosilki nasprotnih stranic.}

Naj bo $E$ četrto oglišče paralelograma $BCDE$. Najprej načrtamo trikotnik $ABE$ ($AB=a$, $BE=c$ in $\angle ABE\cong \angle AB,CD$).

        \item \res{Tri stranice in kota ob četrti stranici}

Brez škode za splošnost predpostavimo, da so dane stranice $AB$, $BC$, $AD$ ter kota $BCD$ in $CDA$.
Naj bo $E$ četrto oglišče paralelograma $BCDE$. Najprej načrtamo ta paralelogram ($CB=b$, $CD=c$ in $\angle BCD=\delta$).

        \item \res{Središča treh stranic in daljica, ki je skladna in vzporedna s četrto stranico}

Uporabi izrek \ref{Varignon}.

        \end{enumerate}


\end{enumerate}




%REŠITVE - Skladnost in krožnica
%________________________________________________________________________________

\poglavje{Congruence and Circle}

\begin{enumerate}

\item   \res{Dolžine stranic nekega trikotnika so $6$, $7$ in $9$. Naj bodo $k_1$, $k_2$ in $k_3$ krožnice s središči
v ogliščih tega trikotnika. Krožnice se med seboj dotikajo, tako da
 se krožnica s središčem v oglišču najmanjšega kota trikotnika z ostalima krožnicama dotika od znotraj,
preostali dve krožnici pa se dotikata od zunaj. Izračunaj dolžine polmerov teh treh krožnic.}

Dolžine polmerov dobimo kot rešitev sistema $r_1-r_2=9$, $r_1-r_3=7$, $r_2+r_3=6$. Torej $r_a=11$, $r_2=2$ in $r_3=4$.

\item \res{Dokaži, da je kot, ki ga določata sečnici krožnice, ki se med seboj sekata v zunanjosti krožnice, enak polovici razlike središčnih kotov,  prirejenih lokoma, ki ležita med krakoma tega kota.} \label{nalSkk2}

    Označimo s $S$ središče krožnice, $P$ presečišče sečnic $s_1$ in $s_2$, ki krožnico sekata po vrsti v točkah $N_1$ in $M_1$ oz. $N_2$ in $M_2$ (kjer je $\mathcal{B}(P,N_1,M_1)$ in $\mathcal{B}(P,N_2,M_2)$). Naj bo še: $\alpha=\angle s_1,s_2=\angle N_1PN_2$, $\beta=\angle M_1SM_2$, $\gamma=\angle N_1SN_2$, $\varphi=\angle M_1SN_1$ in  $\psi=\angle M_2SN_2$. Iz štirikotnika $PN_1SN_2$ dobimo (izrek \ref{VsotKotVeck}): $\alpha=360^0-\left( \gamma+90^0-\frac{\varphi}{2} +90^0-\frac{\psi}{2}\right)=180^0-\frac{\varphi+\psi}{2}-\gamma=
    \frac{360^0-\varphi-\psi}{2}-\gamma=\frac{\beta+\gamma}{2}-\gamma=
    \frac{\beta-\gamma}{2}$.

\item \res{Vrh kota $\alpha$ je zunanja točka krožnice $k$. Med krakoma tega kota ležita na krožnici dva
loka, ki sta v razmerju $3:10$. Večji od teh lokov ustreza središčnemu kotu $40^0$. Določi
velikost kota $\alpha$.}

Najprej ugotovimo (s pomočjo ustreznega premega sorazmerja), da drugi središčni kot meri $12^0$. Iz prejšnje naloge (\ref{nalSkk2}) nato sledi $\alpha=\frac{40^0-12^0}{2}=14^0$.

\item  \res{Dokaži, da je kot, ki ga oklepata tangenti krožnice, enak polovici razlike središčnih kotov, prirejenih lokoma, ki ležita med krakoma tega kota.}

    Označimo s $PT_1$ in $PT_2$ tangenti, ki se krožnice s središčem $S$ dotikata v točkah $T_1$ in $T_2$. Z $\beta$ in $\gamma$ ($\beta>\gamma$) označimo središčna kota. Ker je po izreku \ref{TangPogoj} $\angle ST_1P=\angle ST_2P=90^0$ dobimo (iz štirikotnika $PT_1ST_2$ - izrek \ref{VsotKotVeck}): $\angle T_1PT_2=180^0-\gamma=
    \frac{\beta+\gamma}{2}-\gamma=
    \frac{\beta-\gamma}{2}$.

\item \res{Naj bo $L$  pravokotna projekcija poljubne točke $K$ krožnice $k$
 na njeni tangenti $t$ skozi točko $T\in k$ in $X$ točka, ki je
simetrična točki $L$ glede na premico $KT$. Določi geometrijsko
mesto točk $X$.}

Označimo s $S$ središče krožnice $k$ in $\alpha=\angle LTK$. Dokažimo najprej, da so $S$, $X$ in $K$ kolinearne točke. Iz skladnosti trikotnikov $TLK$ in $TXK$ je $\angle XTK\cong \angle LTK=\alpha$, $\angle TXK\cong \angle TLK=90^0$ in $\angle TKX\cong \angle TKL=90^0-\alpha$. Po izrekih \ref{ObodKotTang} in \ref{SredObodKot} je $\angle TSK=2\angle LTK=2\alpha$. Torej iz enakokrakega trikotnika $TSK$ (izreka \ref{enakokraki} in \ref{VsotKotTrik}) dobimo: $\angle TKS=\frac{180^0-2\alpha}{2}=90^0-\alpha=\angle TKX$ oz. $\angle TKS\cong\angle TKX$, zato $X\in SK$. Na koncu je $\angle SXT\cong\angle KXT=90^0$, kar pomeni, da po izreku \ref{TalesovIzrKroz2} točka $X$ leži na krožnici $l$ s premerom $TS$. Ni težko dokazati, da velja tudi obratno - vsako točko $X\in l$ dobimo iz neke točke $K\in k$ po opisanem postopku.

\item \res{Naj bosta $BB'$ in $CC'$ višini trikotnika $ABC$ ter $t$
tangenta očrtane krožnice tega trikotnika v točki $A$. Dokaži, da
je  $B'C'\parallel t$.}

Označimo s $P$ poljubno točko tangente $t$, tako da sta $P$ in $C$ na različnih bregovih te tangente. Po izreku \ref{ObodKotTang} je $\angle PAC'=\angle PAB\cong\angle ACB$. Ker je $BCC'B'$ tetivni štirikotnik (izrek \ref{TalesovIzrKroz2}),  je po izreku \ref{TetivniPogojZunanji} $\angle AC'B'\cong\angle ACB$. Torej velja $\angle AC'B'\cong\angle PAC'$, zato je po izreku \ref{KotiTransverzala} $PA\parallel B'C'$ oz.  $B'C'\parallel t$.

\item \res{V pravokotnem trikotniku $ABC$ je nad kateto $AC$ kot premerom
načrtana krožnica, ki seka hipotenuzo $AB$ v točki $E$. Tangenta te
krožnice v točki $E$ seka drugo kateto $BC$ v točki $D$. Dokaži, da je $BDE$
enakokraki trikotnik.}

Če uporabimo izrek \ref{ObodKotTang}, dobimo $\angle DEC\cong\angle EAC$. Torej:
$\angle BED=90^0-\angle DEC=90^0-\angle EAC=90^0-\angle BAC=\angle ABC=\angle EBD$. Iz
$\angle BED\cong\angle EBD$ po izreku \ref{enakokraki} sledi $BD\cong ED$, kar pomeni, da je $BDE$ enakokraki trikotnik.


\item \res{V pravi kot z vrhom $A$ je včrtana krožnica, ki se dotika krakov tega kota v
točkah $B$ in $C$. Poljubna tangenta te krožnice seka premici $AB$ in $AC$ po vrsti
v točkah $M$ in $N$ (tako, da je $\mathcal{B}(A,M,B)$). Dokaži, da velja:
$$\frac{1}{3}\left(|AB|+|AC|\right) < |MB|+|NC| <
\frac{1}{2}\left(|AB|+|AC|\right).$$}

Označimo s $T$ dotikališče omenjene krožnice s premico $MN$ in $s=\frac{1}{2}(|MN|+|NA|+|AM|)$ polobseg trikotnika $MAN$. Dana krožnica je trikotniku $MAN$ pričrtana krožnica, zato po veliki nalogi velja:
\begin{eqnarray} \label{nalSkk8Eqn1}
|AB|=|AC|=s.
\end{eqnarray}
Dokažimo najprej neenakost $|MB|+|NC| <\frac{1}{2}\left(|AB|+|AC|\right)$.
Če uporabimo izrek \ref{TangOdsek} in trikotniško neenakost \ref{neenaktrik} ter relacijo \ref{nalSkk8Eqn1}, dobimo:
 \begin{eqnarray*}
|MB|+|NC| &=& |MT|+|NT| =|MN|=\\
 &=& \frac{1}{2}\left( |MN|+|MN| \right)\\
 &<& \frac{1}{2}\left( |MN|+|NA|+|AM| \right)=\\
 &=& s=|AB|=\\
 &=& \frac{1}{2}\left(|AB|+|AC|\right).
\end{eqnarray*}
Dokažimo še drugo neenakost: $|MB|+|NC| >\frac{1}{3}\left(|AB|+|AC|\right)$.
Ker je $MAN$ pravokotni trikotnik s hipotenuzo $MN$, po izreku \ref{vecstrveckotHipot} sledi:
\begin{eqnarray} \label{nalSkk8Eqn2}
|MN|>|MA| \hspace*{1mm} \textrm{ in } \hspace*{1mm} |MN|>|NA|.
\end{eqnarray}
Če uporabimo izrek \ref{TangOdsek} ter relaciji \ref{nalSkk8Eqn2} in \ref{nalSkk8Eqn1}, dobimo:
 \begin{eqnarray*}
|MB|+|NC| &=& |MT|+|NT| =|MN|=\\
 &=& \frac{1}{3}\left( |MN|+|MN|+|MN| \right)\\
 &>& \frac{1}{3}\left( |MN|+|NA|+|AM| \right)=\\
 &=& \frac{1}{3}\cdot2s=\frac{1}{3}\cdot2\cdot|AB|=\\
 &=& \frac{1}{3}\left(|AB|+|AC|\right).
\end{eqnarray*}

\item \res{Dokaži, da je v pravokotnem trikotniku vsota katet enaka vsoti
premerov očrtane in včrtane krožnice.}

Naj bo $ABC$ pravokotni trikotnik s pravim kotom ob oglišču $C$. Uporabimo oznake kot pri veliki nalogi \ref{velikaNaloga}. Ker je $\angle PCQ=\angle BCA=90^0$, je štirikotnik $PCQS$ kvadrat, zato  po veliki nalogi (\ref{velikaNaloga}) sledi:
 \begin{eqnarray} \label{nalSkk9Eqn}
r=|SQ|=|CP|=s-c.
\end{eqnarray}
 Iz tega in Talesovega izreka (\ref{TalesovIzrKroz2}) sledi:
 $$R+r=\frac{c}{2}+s-c=b+c.$$

\item \res{Simetrale notranjih kotov konveksnega štirikotnika naj se sekajo v šestih različnih točkah.
Dokaži, da so štiri od teh točk  oglišča tetivnega štirikotnika.}

Uporabi kriterij za tetivnost štirikotnika - izrek \ref{TetivniPogoj}.

\item \res{Naj bodo: $c$ dolžina hipotenuze, $a$ in $b$ dolžini
katet ter $r$ polmer včrtane krožnice pravokotnega trikotnika. Dokaži, da velja:}
\begin{enumerate}
 \item \res{$2r + c \geq 2 \sqrt{ab}$}

Uporabi relacijo \ref{nalSkk9Eqn} ter neenakost aritmetične in geometrijske sredine.

 \item \res{$a + b + c > 8r$}

Uporabi relacijo \ref{nalSkk9Eqn} in Pitagorov izrek \ref{PitagorovIzrek}.
\end{enumerate}

\item \res{Naj bosta $P$ in $Q$ središči krajših lokov $AB$ in $AC$
pravilnemu trikotniku $ABC$ očrtane krožnice. Dokaži, da stranici $AB$ in $AC$ tega trikotnika razdelita
tetivo $PQ$ na tri skladne daljice.}

Označimo z $X$ in $Y$ presečišči tetive $PQ$ s stranicama $AB$ in $AC$ trikotnika $ABC$. Dokažimo, da velja $PX\cong XY\cong YQ$. Po izreku \ref{TockaN} sta $BQ$ in $CP$ simetrali notranjih kotov $ABC$ in $ACB$ trikotnika $ABC$. Po izreku \ref{ObodObodKot} je:
  \begin{eqnarray*}
\angle PAX &=& \angle PAB\cong\angle PCB=\frac{1}{2}\angle ACB=30^0\\
\angle APX &=& \angle APQ\cong\angle ABQ=\frac{1}{2}\angle ABC=30^0.
\end{eqnarray*}
Torej:
 $\angle PAX = \angle APX=30^0$, kar pomeni, da je $APX$ enakokraki trikotnik z osnovnico $AP$ (izrek \ref{enakokraki}) oz. $PX\cong XA$. Po izreku \ref{zunanjiNotrNotr} je še $\angle AXY=\angle APX+\angle PAX=60^0$. Ker je še $\angle XAY=\angle BAC= 60^0$, je $AXY$ pravilni trikotnik, zato je $XA\cong XY\cong YA$. Analogno je tudi $AQY$ enakokraki trikotnik z osnovnico $AQ$ oz. $QY\cong YA$. Če povežemo dokazane relacije skladnosti daljic, dobimo: $PX\cong XA\cong XY\cong AY\cong YQ$.

\item \res{Naj bodo $k_1$, $k_2$, $k_3$, $k_4$ štiri krožnice, od katerih se vsaka od zunaj dotika ene stranice   in dveh nosilk stranic poljubnega konveksnega
štirikotnika. Dokaži, da so središča teh krožnic konciklične točke.}

Dokaži najprej, da dva nasprotna kota dobljenega štirikotnika merita $\frac{\alpha+\beta}{2}$ in $\frac{\gamma+\delta}{2}$, kjer so $\alpha$, $\beta$, $\gamma$ in $\delta$ notranji koti začetnega štirikotnika, nato pa uporabi izreka \ref{TetivniPogoj} in \ref{VsotKotVeck}.

\item \res{Krožnici $k$ in $l$ se od zunaj dotikata v točki $A$. Točki $B$ in $C$ sta dotikališči
skupne zunanje tangente teh dveh krožnic. Dokaži, da je $\angle BAC$ pravi kot.}

Naj bo $L$ presečišče skupne tangente krožnic $k$ in $l$ v točki $A$ z daljico $BC$. Ker sta $XA$ in $XB$ tangenti krožnice $k$ v točkah $A$ in $B$, je $XA\cong XB$ (izrek \ref{TangOdsek}). Analogno sta $XA$ in $XC$ tangenti krožnice $l$ v točkah $A$ in $C$ oz. je $XA\cong XC$. Torej je $X$ središče krožnice s premerom $BC$ in točka $A$ leži na tej krožnici. Po Talesovem izreku \ref{TalesovIzrKroz2} je $\angle BAC=90^0$.

\item \res{Naj bo $ABCD$ deltoid ($AB\cong AD$ in $CB\cong CD$). Dokaži:}
\begin{enumerate}
 \item \res{$ABCD$ je tangentni štirikotnik.}

Uporabi izrek \ref{TangentniPogoj}.

 \item  \res{$ABCD$ je tetivni štirikotnik natanko tedaj, ko je $AB\perp BC$.}

Uporabi izrek \ref{TetivniPogoj}.
\end{enumerate}



\item \res{Krožnici $k$ in $k_1$  se od zunaj dotikata v točki $T$, v kateri se sekata premici $p$ in $q$. Premica $p$ ima
s krožnicama še presečišči $P$ in  $P_1$, premica $q$ pa  $Q$ in $Q_1$.  Dokaži, da je $PQ\parallel P_1Q_1$.}

Naj bo $t$ skupna tangenta krožnic $k$ in $k_1$, $X\in t$ poljubna točka te tangente, za katero velja $P,X\div QQ_1$ in $Y$ točka, ki je simetrična točki $X$ glede na $T$. Po izrekih \ref{ObodKotTang} in \ref{sovrsnaSkladna} velja:
 \begin{eqnarray*}
 \angle QPP_1= \angle QPT \cong\angle QTX\cong\angle Q_1TY\cong\angle Q_1P_1T
 =\angle Q_1P_1P
 \end{eqnarray*}
Torej $\angle QPP_1 \cong\angle Q_1P_1P$, zato je po izreku \ref{KotiTransverzala} $PQ\parallel P_1Q_1$.

\item \res{Naj bo $MN$ skupna tangenta krožnic $k$ in $l$ ($M$ in $N$ sta dotikališči),
ki se sekata v točkah $A$ in $B$. Izračunaj vrednost vsote $\angle MAN+\angle MBN$.}

Brez škode za splošnost predpostavimo, da je $d(A,MN)<d(B,MN)$.
Če uporabimo izreka \ref{ObodKotTang} in \ref{VsotKotTrik}, dobimo:
 \begin{eqnarray*}
 \angle MBN &=& \angle MBA +\angle ABN =\\
&=& \angle NMA +\angle MNA = \\
&=&  180^0-\angle MAN.
 \end{eqnarray*}
Torej $\angle MAN+\angle MBN=180^0$.

\item \res{Naj bo $t$ tangenta trikotniku $ABC$ očrtane krožnice v točki $A$. Premica, ki je vzporedna s tangento $t$, seka stranici $AB$ in $AC$ v točkah $D$ in $E$.
Dokaži, da so točke $B$, $C$, $D$ in $E$ konciklične.}

Označimo s $P$ tisto točko tangente $t$, da velja $T,E\div AB$. Če uporabimo izreka \ref{KotiTransverzala} in \ref{ObodKotTang}, dobimo:
 \begin{eqnarray*}
 \angle ADE \cong \angle PAD =\angle PAB \cong \angle ACB=\angle ECB.
 \end{eqnarray*}
Iz $\angle ADE \cong \angle ECB$ je po izreku \ref{TetivniPogojZunanji} $BCED$ tetivni štirikotnik, torej so točke $B$, $C$, $D$ in $E$  konciklične.

\item \res{Naj bosta $D$ in $E$ poljubni točki polkrožnice, ki je načrtana nad premerom $AB$. Naj bo $AD\cap BE= \{F\}$ in $AE\cap BD= \{G\}$.
Dokaži, da je $FG\perp AB$.}

Po Talesovem izreku \ref{TalesovIzrKroz2} je $\angle ADB\cong\angle AEB=90^0$, torej $BD\perp AF$ in $AE\perp BF$.  $G$ je višinska točka trikotnika $ABF$, zato je po izreku \ref{VisinskaTocka} tudi $FG$ nosilka njegove tretje višine in je $FG\perp AB$.

\item \res{Naj bo $M$ točka na krožnici $k(O,r)$. Določi geometrijsko mesto središč
 vseh tetiv te krožnice, ki imajo eno krajišče v točki $M$.}

Dokažimo, da je iskano geometrijsko mesto $\mathcal{G}$ enako krožnici $k_1$ nad premerom $OM$ oz. dokažimo $\mathcal{G}=k_1$. Dovolj je dokazati, da za poljubno točko $Y$ ravnine velja ekvivalenca:
\begin{eqnarray*}
 Y\in\mathcal{G} \hspace*{1mm}\Leftrightarrow \hspace*{1mm}Y\in k_1.
 \end{eqnarray*}
Za točki $O$ in $M$ očitno velja $O,M\in \mathcal{G}$ in  $O,M\in k_1$. Predpostavimo torej, da je $Y\neq O$ in $Y\neq M$.

($\Rightarrow$) Predpostavimo, da je $Y\in\mathcal{G}$. To pomeni, da je $Y$ središče neke tetive $MX$ krožnice $k$. Trikotnika $MYO$ in $XYO$ sta skladna (izrek \textit{SSS} \ref{SSS}), zato je $\angle OYM\cong\angle OYX=90^0$, kar pomeni (po Talesovem izreku \ref{TalesovIzrKroz2}), da velja $Y\in k_1$.

($\Leftarrow$) Naj bo sedaj $Y\in k_1$. Označimo z $X$ točko, ki je simetrična točki $M$ glede na točko $Y$. Iz $Y\in k_1$ po Talesovem izreku (\ref{TalesovIzrKroz2}) sledi $\angle OYM=90^0$, oz. $\angle OYM\cong\angle OYX=90^0$. Trikotnika $MYO$ in $XYO$ sta skladna (izrek \textit{SAS} \ref{SKS}), zato je $OX\cong OM$, kar pomeni, da točka $X$ leži na krožnici $k$. Torej je $Y$ središče tetive $AX$ krožnice $k$, torej $Y\in\mathcal{G}$.

Bralcu je prepuščeno, da nalogo reši na enostavnejši način - z uporabo središčnega raztega $h_{M,\frac{1}{2}}$ (glej razdelek \ref{odd7SredRazteg}). Ali nalogo v tem smislu posplošimo, če uporabimo središčni razteg $h_{M,k}$ za poljubni koeficient $k$?

\item \res{Naj bosta $M$ in $N$ točki, ki sta simetrični nožišču $A'$ višine $AA'$ trikotnika $ABC$ glede na stranici $AB$ in $AC$, ter $K$ presečišče premic $AB$ in $MN$. Dokaži,
da so točke $A$, $K$, $A'$, $C$ in $N$ konciklične.}

Najprej dokažemo, da je $CK$ višina trikotnika $ABC$, nato pa še, da točke $K$, $A'$ in $N$ ležijo na krožnici s premerom $AC$.

\item \res{Naj bo $D$ nožišče višine iz oglišča $A$ ostrokotnega trikotnika $ABC$ in $O$
središče očrtane krožnice tega trikotnika. Dokaži, da je  $\angle CAD\cong \angle BAO$.}

Najprej dokažemo, da velja $\angle BAD\cong\angle OAC =90^0-\angle ABC$.

\item  \res{Naj bo $ABCD$ tetivni štirikotnik, $E$ višinska točka trikotnika $ABD$ in $F$
 višinska točka trikotnika $ABC$. Dokaži, da je štirikotnik $CDEF$
paralelogram.}

Iz $DE\perp AB$ in $CF\perp AB$ neposredno sledi $DE\parallel CF$. Dokažimo še, da velja $\angle EDC+\angle DCF=180^0$. Če uporabimo izreke \ref{ObodObodKot} in \ref{VsotKotTrik}, velja:
\begin{eqnarray*}
 \angle EDC &=& \angle EDB+\angle BDC=
 90^0-\angle ABD+\angle CAB\\
\angle DCF &=& \angle ACF+\angle DCA=
 90^0-\angle CAB+\angle ABD.
 \end{eqnarray*}
Iz tega sledi $\angle EDC+\angle DCF=180^0$, kar pomeni (po izreku \ref{paralelogram}), da je štirikotnik $CDEF$
paralelogram.

Drugi dokaz tega izreka je dan v delu dokaza trditve iz zgleda \ref{TetivniVisinska}.

\item \res{Krožnici s središčema  $O_1$ in $O_2$ se sekata v točkah $A$ in $B$. Premica $p$,
ki poteka skozi točko $A$, seka ti dve krožnici še v točkah $M_1$ in $M_2$. Dokaži, da je
$\angle O_1M_1B\cong\angle O_2M_2B$.}

Ker sta trikotnika $O_1M_1B$ in $O_2M_2B$ enakokraka z osnovnicama $M_1B$ in $M_2B$, zadošča dokazati (izreka \ref{VsotKotTrik} in \ref{enakokraki}), da sta središčna kota $M_1O_1B$ in $M_2O_2B$ skladna. Brez škode za splošnost predpostavimo, da velja $A, O_1\div BM_1$ in $A,O_2\ddot{-} BM_2$. Naj bo $A'$ točka, ki je simetrična točki $A$ glede na središče $O_1$. Po izrekih \ref{SredObodKot} in \ref{ObodObodKotNaspr} je:
 \begin{eqnarray*}
 \angle M_2O_2B &=& 2\angle M_2AB=2(180^0-\angle M_1AB)=\\
&=& 2\angle M_1A'B=\angle M_1O_1B.
 \end{eqnarray*}

\item \res{Polkrožnica s središčem $O$ je načrtana nad premerom $AB$.
Naj bosta $C$ in $D$ takšni točki na daljici $AB$, da velja $CO\cong OD$. Vzporedni premici skozi točki $C$ in $D$ sekata polkrožnico v točkah $E$ in $F$.
Dokaži, da sta premici $CE$ in $DF$ pravokotni na premico $EF$.}

Naj bo $S$ središče daljice $EF$. Daljica $SO$ je srednjica trapeza $CDFE$ z osnovnicama $CE$ in $DF$, zato je $OS\parallel EC$ oz. $OS\parallel FD$ (izrek \ref{srednjTrapez}). Po izreku \textit{SSS} \ref{SSS} je $\triangle OSE\cong\triangle OSF$, zato je $\angle OSE\cong\angle OSF=90^0$. Torej $OS\perp EF$ oz. $CE\perp EF$ in $DF\perp EF$.

\item \res{Na tetivi $AB$ krožnice $k$ s središčem $O$ leži točka $C$, točka $D$ pa naj bo
drugo presečišče krožnice $k$ z očrtano krožnico trikotnika $ACO$. Dokaži, da je
$CD\cong CB$.}

Ker $A,B\in k$, je $AOB$ enakokraki trikotnik z osnovnico $AB$, zato je $\angle OBA\cong\angle OAB$ (izrek \ref{enakokraki}). Po izreku \ref{ObodObodKot} pa je $\angle OAC\cong\angle ODC$. Če povežemo dokazani relaciji, dobimo:
 \begin{eqnarray*}
 \angle OBC=\angle OBA=\angle OAB=\angle OAC\cong\angle ODC.
 \end{eqnarray*}
Iz $\angle OBC\cong\angle ODC$, $OC\cong OC$ in $OB\cong OD$ sledi (izrek \textit{SSA} \ref{SSK}) $\triangle OBC\cong\triangle ODC$ oz. $CB\cong CD$.


\item \res{Naj bo $AB$ mimobežnica krožnice $k$. Premici $AC$ in $BD$ naj bosta tangenti
krožnice $k$ v točkah $C$ in $D$.
Dokaži, da velja:
 $$||AC|-|BD||< |AB| < |AC|+|BD|.$$}

Ker je $AB$ mimobežnica krožnice $k$, je $A$ zunanja točka te krožnice, zato iz točke $A$ obstajata dve tangenti $AC$ in $AC_1$ ($C,C_1\in k$) na to krožnico. Po izreku \ref{TangOdsek} je $AC\cong AC_1$. Podobno iz točke $B$  obstajata dve tangenti $BD$ in $BD_1$ ($D,D_1\in k$) na krožnico $k$ in velja $BD\cong BD_1$. Torej lahko brez škode za splošnost izberemo takšen par tangent $AC$ in $BD$, da se daljici $AC$ in $BD$ sekata v neki točki $E$. Če uporabimo trikotniško neenakost \ref{neenaktrik}, je:
 \begin{eqnarray*}
 |AB| < |AE|+|BE| < |AC|+|BD|.
 \end{eqnarray*}
Za drugo neenakost spet uporabimo trikotniško neenakost \ref{neenaktrik} in relacijo $EC\cong ED$ (izrek \ref{TangOdsek}):
 \begin{eqnarray*}
 |AB| > ||AE|-|BE||= |\left(|AC|-|EC|\right)-\left(|BD|-|ED|\right)| =  ||AC|-|BD||.
 \end{eqnarray*}


\item \res{Naj bo $S$ presečišče nosilk krakov $AD$ in $BC$
trapeza $ABCD$ z osnovnico $AB$. Dokaži, da se očrtani krožnici trikotnikov $SAB$ in
$SCD$ dotikata v točki $S$.}

Uporabi izrek \ref{TangOdsek}.

\item \res{Premici $PB$ in $PD$ se dotikata krožnice $k(O,r)$ v točkah $B$ in $D$.
Premica $PO$ seka krožnico $k$ v točkah $A$ in $C$ ($\mathcal{B}(P,A,C)$). Dokaži, da je
premica $BA$ simetrala kota $PBD$.}


Uporabi izreka \ref{enakokraki} in \ref{TangOdsek}.

\item \res{Štirikotnik $ABCD$ je včrtan krožnici s središčem $O$. Diagonali $AC$ in
$BD$ sta pravokotni. Naj bo $M$ pravokotna projekcija središča $O$
na premici $AD$. Dokaži, da je
 $$|OM|=\frac{1}{2}|BC|.$$}

Iz skladnosti trikotnikov $OMA$ in $OMD$ (izrek \textit{SSA} \ref{SSK}) je $MA\cong MD$ oz. točka $M$ je središče daljice $AD$.
Označimo z $E$ presečišče diagonal $AC$ in $BD$. Naj bo $D'$ točka, ki je simetrična točki $D$ glede na središče $O$. Jasno je, da velja $D'\in k$. Po Talesovem izreku \ref{TalesovIzrKroz2} je $\angle DAD'=90^0$. Daljica $MO$ je srednjica trikotnika $DAD'$ za stranico $AD'$, zato je $|OM|=\frac{1}{2}|AD'|$. Dovolj je dokazati še $AD'\cong BC$ oz. (po izreku \ref{SklTetSklObKot}) $\angle ADD'\cong \angle BDC$. Če sdaj uporabimo izreka \ref{VsotKotTrik} in \ref{ObodObodKot}, dobimo:
\begin{eqnarray*}
 \angle ADD' &=& 90^0-\angle AD'D =90^0- \angle ACD =\\
&=& 90^0-\angle ECD =\angle EDC =\\
&=& \angle BDC.
 \end{eqnarray*}


\item \res{Daljici $AB$ in $BC$ sta sosednji stranici pravilnega devetkotnika, ki je včrtan krožnici $k$ s središčem $O$.
Točka $M$ je središče stranice $AB$, točka $N$ pa središče
polmera $OX$ krožnice $k$, ki je pravokoten na premico $BC$. Dokaži, da je
$\angle OMN=30^0$.}

Iz $\angle AOB\cong\angle BOC=\frac{360^0}{9}=40^0$ sledi:
 $$\angle AOX=\angle AOB + \angle BOX=\angle AOB + \frac{1}{2}\angle BOC=60^0.$$
Ker je še $OA\cong OB$, je $AOB$ pravilni trikotnik (izreka \ref{enakokraki} in \ref{VsotKotTrik}). Njegova težiščnica $AN$ je hkrati višina oz. $\angle ONA=90^0$. Ker je še $OMA=90^0$, po Talesovem izreku točki $N$ in $M$ ležita na krožnici s premerom $OA$ oz. $ONMA$ je tetiven štirikotnik. Če na koncu uporabimo še izrek \ref{ObodObodKot}, dobimo:
\begin{eqnarray*}
 \angle OMN=\angle OAN=\frac{1}{2}\angle OAX=30^0.
 \end{eqnarray*}

\item \res{Krožnici $k_1$ in $k_2$ se sekata v točkah $A$ in $B$. Naj bo $p$ premica, ki poteka skozi točko $A$, krožnico $k_1$ seka še v točki $C$, krožnico $k_2$ pa še v točki $D$, ter $q$ premica, ki poteka skozi točko $B$, krožnico $k_1$ seka še v točki $E$, krožnico $k_2$ pa še v točki $F$. Dokaži, da
je $\angle CBD\cong\angle EAF$.}

Če uporabimo izreka \ref{VsotKotTrik} in \ref{ObodObodKot}, dobimo:
\begin{eqnarray*}
 \angle CBD &=& 180^0-\angle BCD - \angle BDC =\\
             &=& 180^0-\angle BCA - \angle BDA =\\
             &=& 180^0-\angle BEA - \angle BFA =\\
             &=& 180^0-\angle FEA - \angle EFA =\\
             &=& \angle EAF =\\
 \end{eqnarray*}


\item \res{Krožnici $k_1$ in $k_2$ se sekata v točkah $A$ in $B$. Načrtaj premico $p$, ki poteka skozi točko $A$, tako da bo
dolžina daljice $MN$, kjer sta $M$ in $N$ presečišči premice $p$
s krožnicama $k_1$ in $k_2$,  maksimalna.}

Naj bo $p$ poljubna premica, ki poteka skozi točko $A$ in krožnici $k_1$ in $k_2$ seka v točkah $M$ in $N$.
Označimo s $S_1$ in $S_2$ središči krožnic $k_1$ in $k_2$ ter s $P_1$ in $P_2$ središči daljic $MA$ in $NA$. Ker je $MN=2\cdot P_1P_2$, se problem maksimuma daljice $MN$ prevede na maksimum daljice $P_1P_2$. Iz skladnosti trikotnikov $MS_1P_1$ in $AS_1P_1$ (izrek \textit{SSS} \ref{SSS}) sledi $\angle S_1P_1M\cong\angle S_1P_1A=90^0$ oz. $\angle P_2P_1S_1=90^0$. Analogno je tudi $\angle P_2P_1S_1=90^0$. Torej je $S_1S_2P_2P_1$  pravokotni trapez z višino $P_1P_2$, zato je $P_1P_2\leq S_1S_2$. Enakost se doseže, ko je $S_1S_2P_2P_1$ pravokotnik oz. $P_1P_2\parallel S_1S_2$. To pomeni, da premico $p$ (za katero je dolžina $|MN|$ maksimalna) narišemo kot vzporednico premice $S_1S_2$ skozi točko $A$.

\item \res{Naj bo $L$ pravokotna projekcija poljubne točke $K$ krožnice $k$ na njeni
tangenti skozi točko $T\in k$ ter $X$ točka, ki je simetrična točki $L$ glede na premico
$KT$. Določi geometrijsko mesto točk $X$.}


Označimo s $S$ središče krožnice $k$  in $k_1$ krožnico nad premerom $ST$.
Dokažimo, da je geometrijsko mesto točk $X$ (označimo ga z $\mathcal{G}$) enako krožnici $k_1$. Potrebno je dokazati ekvivalenco:
\begin{eqnarray*}
 X\in \mathcal{G} \hspace*{1mm} \Leftrightarrow \hspace*{1mm} X\in k_1
 \end{eqnarray*}

Za točki $T$ in $S$ je po definiciji jasno, da velja hkrati $T\in \mathcal{G}$ in $T\in k_1$ oz. $S\in \mathcal{G}$ in $S\in k_1$. Predpostavimo naprej, da je $X\neq T$ in $X\neq S$.

Predpostavimo najprej $X\in \mathcal{G}$. Označimo $\angle LTK=\alpha$.
 Trikotnika $XKT$ in $LKT$ sta simetrična glede na premico $KT$ (glej razdelek \ref{odd6OsnZrc}), zato je $\triangle XKT\cong \triangle LKT$ oz. $\angle XTK\cong \angle LTK=\alpha$ in $\angle KXT\cong \angle KLT=90^0$. Dokažimo, da je tudi $\angle SXT=90^0$. Dovolj je dokazati, da so točke $S$, $X$ in $K$ kolinearne oz. da velja $\angle TKX=\angle TKS$. Najprej je (iz $\triangle XKT\cong \triangle LKT$):
 $\angle TKX\cong\angle TKL =90^0-\alpha$. Če uporabimo izrek \ref{enakokraki} enakokrakega trikotnika $KST$ z osnovnico $KT$ in izreka \ref{ObodKotTang} ter \ref{SredObodKot}, je:
 $\angle TKS=\frac{1}{2}(180^0-\angle KST)=\frac{1}{2}(180^0-2\angle LTK)=90^0-\alpha$.
Torej $\angle TKX=\angle TKS=90^0-\alpha$, kar pomeni, da so točke $S$, $X$ in $K$ kolinearne oz. $\angle SXT=90^0$. Iz slednjega je $X\in k_1$.

Predpostavimo sedaj, da je $X\in k_1$. Označimo s $K$ presečišče poltraka $SX$ s krožnico $k$ in z $L$ točko, ki je simetrična točki $X$ glede na premico $TK$.
 Iz $X\in k_1$ po Talesovem izreku \ref{TalesovIzrKroz2} sledi $\angle SXT=90^0$, zato je tudi $\angle KXT=90^0$.
Da dokažemo $X\in \mathcal{G}$, je dovolj dokazati, da $X$ leži na tangenti $t$ krožnice $k$ skozi točko $T$. Naj bo $L'$ poljubna točka na tangenti $t$, za katero velja $L,X\ddot{-} ST$. Dovolj je dokazati $\angle LTK\cong\angle L'TK$.
Označimo $\angle LTK=\alpha$.
 Trikotnika $XKT$ in $LKT$ sta simetrična glede na premico $KT$, zato je $\triangle XKT\cong \triangle LKT$ oz. $\angle XTK\cong \angle LTK=\alpha$ in $\angle KLT\angle KXT\cong =90^0$. Če uporabimo izreka \ref{ObodKotTang} in \ref{SredObodKot}, dobimo:
$\angle L'TK=\frac{1}{2}\angle TSK=\frac{1}{2}(180^0-2\angle SKT)=90^0-\angle SKT=90^0-\angle XKT=\angle XTK=\alpha=\angle LTK$.
Torej $L'\in t$ oz. $X\in \mathcal{G}$.

\item \res{Dokaži, da je tetivni večkotnik z lihim številom oglišč, ki ima vse notranje kote skladne, pravilni večkotnik.}

Označimo z $O$ središče očrtane krožnice tega večkotnika in z $B_i$ središča stranic $A_iB_{i+1}$ ($i\in \{1,2,\ldots , 2n-1 \}$, $A_0=A_{2n-1}$ in $A_{2n}=A_1$). Uporabimo dejstvo, da je $A_iOA_{i+1}$ enakokraki trikotnik z osnovnico $A_iA_{i+1}$ in dokažimo najprej $\triangle OB_{i-1}A_i\cong \triangle OB_{i+1}A_{i+1}$.

\item \res{Dve krožnici se dotikata od znotraj v točki $A$. Daljica $AB$ je premer večje
krožnice, tetiva $BK$ večje krožnice pa se dotika manjše krožnice v točki $C$. Dokaži
da je premica $AC$ simetrala kota $BAK$.}

Označimo z $EA$ skupno tangento krožnic $k$ (večje) in $l$ (manjše) ter z $D$ drugo presečišče premice $AB$ s krožnico $l$. Naj bo še $L$ drugo presečišče poltraka $AK$ s krožnico $l$, $\alpha=\angle DCB$ in $\beta=\angle DBC$.
 Po izreku \ref{ObodKotTang} je:
 \begin{eqnarray} \label{nalSkk36Eqn1}
 \angle CAB=\angle CAD\cong \angle DCB=\alpha
\end{eqnarray}
in
 \begin{eqnarray} \label{nalSkk36Eqn2}
 \angle LDA\cong \angle EAL=\angle EAK\cong \angle ABK=\angle DBC=\beta.
\end{eqnarray}
Kot $CDA$ je zunanji kot trikotnika $CDB$, zato je po izreku \ref{zunanjiNotrNotr}
 $\angle CDA=\angle DCB+\angle CBD=\alpha+\beta$. Iz tega in \ref{nalSkk36Eqn2} sledi:
 \begin{eqnarray} \label{nalSkk36Eqn3}
 \angle LDC= \angle CDA-\angle LDA=\alpha+\beta-\beta=\alpha.
\end{eqnarray}
Iz relacij \ref{nalSkk36Eqn3} in \ref{nalSkk36Eqn1} ter izreka \ref{ObodObodKot}:
\begin{eqnarray*}
 \angle KAC= \angle LAC\cong \angle LDC=\alpha=\angle CAB.
\end{eqnarray*}
To pomeni, da je premica $AC$ simetrala kota $BAK$.

Naloga je poseben primer splošnejše naloge \ref{nalSkk47}.

\item \res{Naj bo $BC$ tetiva krožnice $k$. Določi geometrijsko mesto višinskih točk
vseh trikotnikov $ABC$, kjer je $A$ poljubna točka, ki leži na krožnici $k$.}

Iskano geometrijsko mesto točk je krožnica, ki poteka skozi točki $B$ in $C$. Izračunaj mero kota $BVC$, kjer je $V$ višinska točka trikotnika $ABC$, in uporabi izrek \ref{ObodKotGMT}. Glej tudi izrek \ref{TockaV'a}.

\item \res{Imejmo štirikotnik s tremi topimi notranjimi koti. Dokaži, da daljša
diagonala poteka iz oglišča, ki pripada ostrem kotu.}

Označimo z $ABCD$ štirikotnik, pri katerem so notranji koti pri ogliščih $A$, $B$ in $C$ topi.
Naj bo $k$ krožnica s premerom $BD$. Ker je $\angle BAD>90^0$ in $\angle BCD>90^0$, sta po izreku \ref{obodKotGMTZunNotr} $A$ in $C$ notranji točki krožnice $k$. To pomeni, da je $AC<BD$.

\item \res{Naj bo $ABCDEF$ tetivni šestkotnik ter $AB\cong DE$ in $BC\cong EF$.
Dokaži, da je $CD\parallel AF$.}

Dokaži, da je $\angle ASC\cong\angle DSF$, kjer je $S$ središče očrtane krožnice šestkotnika $ABCDEF$.

\item \res{Naj bo $ABCD$ konveksni štirikotnik, pri katerem je $\angle ABD=50^0$, $\angle ADB=80^0$, $\angle ACB=40^0$ in $\angle DBC=\angle BDC +30^0$. Izračunaj mero kota $\angle DBC$.}

V trikotniku $ABD$ je $\angle DAB=180^0-80^0-50^0=50^0$ (izrek \ref{VsotKotTrik}). Torej je $\angle DAB\cong \angle DBA =50^0$, kar pomeni, da je $ABD$ enakokraki trikotnik (\ref{enakokraki}), torej $DA\cong DB$. Iz tega sledi, da krožnica $k$ s središčem $D$, ki poteka skozi točko $A$, poteka tudi skozi točko $B$. Ker je $\angle ACB=40^0=\frac{1}{2}\angle ADB$, po izreku \ref{ObodKotGMT} tudi točka $C$ leži na krožnici $k$. Iz tega sledi, da je $DAC$ enakokraki trikotnik oz. $\angle DAC\cong\angle DCA=x$ (\ref{enakokraki}). Iz tega je $\angle DCB=\angle DCA+\angle ACB=x+40^0$.
Po izreku \ref{SredObodKot} je $\angle BDC=2\angle BAC=2(50^0-x)=100^0-2x$. Iz danega pogoja je še $\angle DBC=\angle BDC +30^0=130^0-2x$. Če uporabimo vsoto notranjih kotov v trikotniku $BCD$ (izrek \ref{VsotKotTrik}), dobimo enačbo:
 \begin{eqnarray*}
 (100^0-2x)+(x+40^0)+(130^0-2x)=180^0.
 \end{eqnarray*}
Iz rešitve te enačbe $x=30^0$ dobimo $\angle DBC=130^0-2x=70^0$.

\item \res{Naj bo $M$ poljubna notranja točka kota z vrhom $A$, točki $P$ in $Q$ pravokotni projekciji točke $M$ na krakih tega kota, točka $K$
pa  pravokotna projekcija vrha $A$ na premici $PQ$. Dokaži, da je $\angle MAP\cong \angle QAK$.}

Po Talesovem izreku \ref{TalesovIzrKroz2} je štirikotnik $APMQ$ tetiven.
Če uporabimo izreka \ref{ObodObodKot} in \ref{KotaPravokKraki}, dobimo:
\begin{eqnarray*}
 \angle MAP\cong \angle MQP\cong \angle QAK.
\end{eqnarray*}


\item \res{Pri tetivnem osemkotniku $A_1A_2\ldots A_8$ velja $A_1A_2\parallel A_5A_6$, $A_2A_3\parallel A_6A_7$,
$A_3A_4\parallel A_7A_8$. Dokaži, da je $A_8A_1\cong A_4A_5$.}

Najprej iz $A_1A_2\parallel A_5A_6$ in  $A_2A_3\parallel A_6A_7$ po izreku \ref{KotaVzporKraki} sledi $\angle A_1A_2A_3\cong\angle A_5A_6A_7$. Ker sta $A_1A_2A_3A_4$ in $A_5A_6A_7A_8$ tetivna štirikotnika, po izreku \ref{TetivniPogoj} velja:
\begin{eqnarray*}
 \angle A_5A_8A_7=180^0- \angle A_5A_6A_7=180^0-\angle A_1A_2A_3= \angle A_1A_4A_3.
\end{eqnarray*}
Ker je še $A_7A_8\parallel A_3A_4$, je tudi $A_5A_8\parallel A_1A_4$ (izrek \ref{KotaVzporKraki}). Torej je $A_1A_4A_5A_8$ tetivni trapez z osnovnicama $A_5A_8$ in $A_1A_4$, zato sta njegova kraka $A_8A_1$ in $A_4A_5$ skladna (izrek \ref{trapezTetivEnakokr}).

\item \res{Neka krožnica vsako stranico štirikotnika seka v dveh točkah in tako na vseh stranicah štirikotnika določa skladne tetive.
Dokaži, da je ta štirikotnik tangenten.}

Dokaži, da je središče te krožnice enako oddaljeno od vseh štirih stranic trikotnika in tako sovpada s središčem včrtane krožnice.

\item \res{Dolžine stranic tangentnega petkotnika $ABCDE$ so naravna števila in hkrati velja $|AB|=|CD|=1$.
Včrtana krožnica petkotnika se dotika stranice $BC$ v točki $K$.
Izračunaj dolžino daljice $BK$.}

Označimo z $J$, $L$, $M$ in $N$ točke dotika stranic $AB$, $CD$, $DE$ in $EA$ petkotnika $ABCDE$ z njegovo včrtano krožnico. Naj bo še $x=|JB|$, $y=|CL|$ in $z=|EN|$. Iz danih pogojev $|AB|=|CD|=1$ je najprej $|AJ|=1-x$ in $|LD|=1-y$, nato tudi $x<1$ in $y<1$. Po izreku \ref{TangOdsek} je $|BK|=|BJ|=x$ in $|CK|=|CL|=y$ oz. $|BC|=x+y$. Ker so dolžine vseh stranic petkotnika $ABCDE$ naravna števila, je tudi $x+y\in \mathbb{N}$. Iz tega $x<1$ in $y<1$ sledi $x+y=1$ oz. $y=1-x$. Torej $|LD|=1-y=x$.
Po izreku  \ref{TangOdsek} je naprej $|EM|=|EN|=z$, $|AN|=|AJ|=1-x$ in $|DM|=|DL|=x$ oz. $|DE|=z+x\in \mathbb{N}$ in $|EA|=z+1-x\in \mathbb{N}$. Iz zadnjih dveh relacij (z odštevanjem) dobimo $2x-1\in \mathbb{N}$. Ker je še $x<1$, dobimo $|BK|=x=\frac{1}{2}$.

\item \res{Dokaži, da krožnica, ki poteka skozi  sosednji oglišči $A$ in $B$ pravilnega
petkotnika $ABCDE$ in njegovo središče $O$, poteka tudi skozi presečišče
njegovih diagonal $AD$ in $BE$.}

Naj bo $P$ presečišče diagonal $AD$ in $BE$. Dokaži najprej $\angle APB\cong\angle AOB=36^0$.

\item \res{Naj bo $H$ višinska točka trikotnika $ABC$, $l$ krožnica nad  premerom $AH$
ter $P$ in $Q$ presečišči te krožnice s stranicama $AB$ in $AC$. Dokaži, da se
tangenti krožnice $k$ skozi točki $P$ in $Q$ sekata na stranici $BC$.}

Dokaži najprej, da tangenta krožnice $l$ v točki $P$ seka stranico $BC$ v njenem središču. Uporabi izreka \ref{TalesovIzrKroz2} in \ref{ObodKotTang}.

\item \label{nalSkk47}
\res{Krožnica $l$ se od znotraj dotika krožnice $k$ v točki $C$. Naj bo $M$ poljubna točka
krožnice $l$ (različna od $C$). Tangenta krožnice $l$ v točki $M$ seka krožnico $k$ v
točkah $A$ in $B$. Dokaži, da je $\angle ACM \cong \angle MCB$.}

Označimo s $K$ in $L$ središči krožnic $k$ in $l$, z $EC$ njuno skupno tangento v točki $C$ ($E,A\ddot{-} KL$) ter s $F$ in z $G$ drugi presečišči poltrakov $CA$ in $CB$ s krožnico $l$. Po izreku \ref{ObodKotTang} je:
 \begin{eqnarray*}
 \angle FGC\cong\angle FCE=\angle ACE\cong\angle ABC.
 \end{eqnarray*}
Iz $\angle FGC\cong\angle ABC$ po izreku \ref{KotiTransverzala} sledi $FG\parallel AB$. Ker je $AB$ tangenta krožnice $l$ v točki $M$, je $LM\perp AB$. Iz dokazanega $FG\parallel AB$ sledi $LM\perp EF$, kar pomeni, da je nosilka polmera $LM$ krožnice $l$ simetrala njene tetive $EF$. Krožnica $l$ je očrtana krožnica trikotnika $CGF$, točka $M$ pa presečišče te krožnice s simetralo stranice $FG$  tega trikotnika, zato je po izreku \ref{TockaN} $CM$ simetrala kota $FCG$ oz. kota $ACB$. Torej $\angle ACM \cong \angle MCB$.

\item \res{Naj bo $k$ očrtana krožnica trikotnika $ABC$ in $R$ središče tistega loka $AB$ te krožnice,
ki ne vsebuje točke $C$. Daljici $RP$ in $RQ$ sta tetivi te krožnice. Prva je
vzporedna, druga pa pravokotna na simetralo notranjega kota  $\angle BAC$. Dokaži:
\begin{enumerate}
\item premica $BQ$ je simetrala notranjega kota $\angle CBA$,
\item  trikotnik, ki ga določajo premice $AB$, $AC$ in $PR$, je pravilni (enakostranični) trikotnik.
 \end{enumerate}}

\textit{(a)}  Uporabi trditev iz zgleda \ref{tockaNtockePQR}.

\textit{(b)} Označimo z $AED$ dani trikotnik. Uporabi izreka \ref{zunanjiNotrNotr} in \ref{KotaVzporKraki} ter dokaži $\angle DEA\cong \angle EDA=\frac{1}{2}\angle BAC$.

\item \res{Naj bo $X$ takšna notranja točka trikotnika $ABC$, da velja:
 $\angle BXC =\angle BAC+60^0$,  $\angle AXC =\angle ABC+60^0$ in  $\angle AXB =\angle AC B+60^0$. Naj bodo
$P$, $Q$ in $R$ druga presečišča  premic $AX$, $BX$ in $CX$ z očrtano krožnico trikotnika $ABC$. Dokaži, da je  trikotnik $PQR$ pravilen.}

Kot $BXC$ je zunanji kot trikotnika $BRX$, zato je po izreku \ref{zunanjiNotrNotr} $\angle BXC=\angle BRX+\angle RBX$. Če uporabimo še izrek \ref{ObodObodKot}, dobimo:
 \begin{eqnarray*}
 \angle BXC &=& \angle BRX+\angle RBX= \angle BRC+\angle RBQ=\\
 &=& \angle BAC+\angle RPQ.
\end{eqnarray*}
 Iz tega sledi $\angle RPQ=\angle BXC-\angle BAC=60^0$. Podobno je tudi $\angle RQP=60^0$, kar pomeni, da je $PQR$ pravilen trikotnik.

\item \label{nalSkl50}
\res{Dokaži, da dotikališča včrtane krožnice s trikotnikom $ABC$ delijo njegove stranice na
odseke dolžin $s-a$, $s-b$ in $s-c$ ($a$, $b$ in $c$ so dolžine stranic, $s$ pa je polobseg
trikotnika).}

Direktna posledica velike naloge \ref{velikaNaloga}.


 \item \res{Krožnice $k$, $l$ in $j$ se paroma od zunaj dotikajo v nekolinearnih točkah $A$, $B$ in $C$. Dokaži, da je očrtana krožnica trikotnika $ABC$ pravokotna na krožnice $k$, $l$ in $j$.}

Dokaži najprej, da je očrtana krožnica trikotnika $ABC$ hkrati včrtana krožnica trikotnika, ki ga določata središča krožnic $k$, $l$ in $j$. Uporabi prejšnjo nalogo \ref{nalSkl50}.

Še ena rešitev te naloge (s pomočjo inverzije) je dana v zgledu \ref{TriKroznInv}.

 \item
 \res{Naj bo $ABCD$ tetivni štirikotnik s središčem očrtane krožnice v točki $O$. Z $E$ označimo presečišče njegovih diagonal $AC$ in $BD$ ter z $F$, $M$ in $N$ središča daljic $OE$, $AD$ in $BC$. Če so $F$, $M$ in $N$ kolinearne točke, velja $AC\perp BD$ ali $AB\cong CD$. Dokaži.}

 Dovolj je dokazati, da v primeru, kadar ni $AC\perp BD$, velja $AB\cong CD$. Označimo s $P$ in $Q$ središči diagonal $AC$ in $BD$. Daljici $MP$ in $QN$ sta srednjici trikotnikov $DAC$ in $DBC$ za isto osnovnico $DC$, zato je (izrek \ref{srednjicaTrik}) $|MP|=\frac{1}{2}|DC|=|QN|$ in $MP\parallel DC\parallel QN$, kar pomeni, da je štirikotnik $PMQN$ paralelogram in se njegovi diagonali razpolavljata v skupnem središču (izrek \ref{paralelogram}) - označimo ga s $S$. Iz skladnosti  trikotnikov $OAP$ in $OCP$ (izrek \textit{SSS} \ref{SSS}) je $OP\perp AC$. Podobno je tudi $OQ\perp BD$. Če z $G$ označimo presečišče pravokotnic iz točk $P$ in $Q$ na diagonali $BD$ in $AC$, je štirikotnik $QOPG$ tudi paralelogram. Iz predpostavke, da ni $AC\perp BD$, sledi $G\neq E$. Daljica $FS$ je sedaj srednjica trikotnika $GOE$ z osnovnico $GE$, zato je $FS\parallel EG$ (izrek \ref{srednjicaTrik}).
  Ker točki $F$ in $S$ ležita na premici $MN$, premici $FS$ in $MN$ sovpadata ter velja $MN\parallel EG$. Toda po izreku \ref{VisinskaTocka} je $G$ višinska točka trikotnika $PEQ$ (po konstrukciji točke $G$ je $GP\perp EQ$ in $GQ\perp EP$), zato je $GE\perp PQ$. Iz $MN\parallel EG$ in $GE\perp PQ$ sledi $MN\perp PQ$, kar pomeni, da je paralelogram $PMQN$ romb (\ref{RombPravKvadr}). Če izrek \ref{srednjicaTrik} uporabimo še enkrat, dobimo $|AB|=2|PN|=2|NQ|=|CD|$.



\item \res{Načrtaj trikotnik $ABC$ (glej oznake v razdelku \ref{odd3Stirik}):}

V vseh primerih uporabimo oznake kot v veliki nalogi \ref{velikaNaloga}.

 (\textit{a}) \res{$a$, $\alpha$, $r$}

Uporabimo dejstvo, da je $\angle BSC=90^0+\frac{1}{2}\alpha$ in izrek \ref{ObodKotGMT} ter najprej narišimo trikotnik $BSC$.

 (\textit{b}) \res{$a$, $\alpha$, $r_a$}


Uporabimo dejstvo, da je $\angle BS_aC=90^0-\frac{1}{2}\alpha$ in izrek \ref{ObodKotGMT}  ter najprej narišimo trikotnik $BS_aC$.

 (\textit{c}) \res{$a$, $v_b$, $v_c$}

Najprej narišimo trikotnika $BB'C$ in $BC'C$ - točki $B'$ in $C'$ pa ležita na krožnici s premerom $BC$.

 (\textit{d}) \res{$\alpha$, $v_a$, $s$}

Naj bosta $E$ in $D$ takšni točki, da velja $\mathcal{B}(ED,B,C,A)$, $DB\cong BA$ in $EC\cong CA$. V tem primeru je $BC=s$ in $\angle DAE=90^0-\frac{1}{2}\alpha$. To omogoča konstrukcijo trikotnika $DAE$.

 (\textit{e}) \res{$v_a$, $l_a$, $r$}

Najprej načrtamo pravokotni trikotnik $AA'E$ ($AA'=v_a$, $\angle A'=90^0$ in $AE=l_a$), nato pa središče $S$ včrtane krožnice.

(\textit{f}) \res{$\alpha$, $v_a$, $l_a$}

Tudi v tem primeru najprej načrtajmo pravokotni trikotnik $AA'E$, nato uporabimo dejstvo, da je premica $AE$ simetrala kota $BAC$.

 (\textit{g}) \res{$\alpha$, $\beta$, $R$}

Najprej načrtamo enakokraki trikotnik $BOC$ ($OB=OC=R$, $\angle BOC=2\alpha$).

 (\textit{h}) \res{$c$, $r$, $R$}

Analogno kot v zgledu \ref{konstr_Rra}.

 (\textit{i}) \res{$a$, $v_b$, $R$}

Najprej načrtamo enakokraki trikotnik $BOC$ ($OB=OC=R$, $BC=a$) nato točko $B'$ iz pogojev $BB'=v_b$ in $\angle BB'C=90^0$.


 \item \res{Načrtaj krožnico $k$ tako, da:}

  \begin{enumerate}
    \item \res{se dotika dveh danih nevzporednih premic $p$ in $q$, tetiva, ki jo določata dotikališči, pa je skladna z dano daljico $t$}

Najprej načrtamo enakokraki trikotnik, ki ga določata dotikališči in presečišče danih premic.

    \item \res{je središče dana točka $S$, dana premica $p$ pa na njej določa tetivo, ki je skladna z dano daljico $t$}

Uporabimo dejstvo, da je središče tetive pravokotna projekcija točke $S$ na premici $p$.

    \item \res{poteka skozi dani točki $A$ in $B$, središče pa leži na dani krožnici $l$}

 Središče iskane krožnice je eno od presečišč krožnice $l$ s simetralo daljice $AB$.

    \item \res{ima dan polmer $r$ in se dotika dveh danih krožnic $l$ in $j$}

 Središče iskane krožnice je presečišče krožnic $l'$ in $j'$, ki sta koncentrični s krožnicama  $l$ in $j$ s polmeroma, ki sta povečana za $r$.

    \item \res{se dotika premice $p$ v točki $P$ in poteka skozi dano točko $A$}

Središče iskane krožnice je presečišče pravokotnice premice $p$ v točki $P$ in simetrale daljice $PA$.

  \end{enumerate}

  \item \res{Načrtaj kvadrat $ABCD$, če je dano oglišče $B$ ter dve točki $E$ in $F$, ki ležita na nosilkah stranic $AD$ in $CD$.}

Glej zgled \ref{KvadratKonstr4tocke}.

  \item \res{Dana je premica $CD$ ter točki $A$ in $B$ ($A,B\notin CD$). Načrtaj na premici $CD$ takšno točko $M$, da velja $\angle AMC\cong2\angle BMD$.}

Predpostavimo, da je $M$ iskana točka. Naj bo $k$ krožnica s središčem $B$, ki se dotika premice $CD$, in $t$ druga tangenta iz točke $M$ na tej krožnici. Brez škode za splošnost predpostavimo, da je $\mathcal{B}(C,M,D)$. Potem je $\angle t,MD=2\angle BMD=\angle AMC$. To pomeni, da se pri zrcaljenju glede na premico $CD$ premica $t$ preslika v premico $AM$ (glej razdelek \ref{odd6OsnZrc}). Torej točko $M$ dobimo kot presečišče premice $CD$ in tangente iz točke $A$ na sliko krožnice $k$ pri zrcaljenju čez premico $CD$.


  \item \res{Načrtaj trikotnik $ABC$ s podatki:}

V vseh primerih uporabimo oznake kot v veliki nalogi \ref{velikaNaloga}.

   (\textit{a}) \res{$v_a$, $t_a$, $\beta-\gamma$}

Najprej načrtamo pravokotni trikotnik $AA'A_1$ ($AA'=v_a$, $AA_1=t_a$ in $\angle A'=90^0$), nato uporabimo dejstvo, da je $\angle A'AO=\beta-\gamma$ (izrek \ref{TockaNbetagama}).

   (\textit{b}) \res{$v_a$, $l_a$, $R$}

Najprej načrtamo pravokotni trikotnik $AA'E$ ($AA'=v_a$, $AE=l_a$ in $\angle A'=90^0$), nato uporabimo dejstvo, da je $\angle EAO=\angle A'AE$ (izrek \ref{TockaNbetagama}).

   (\textit{c}) \res{$R$, $\beta-\gamma$, $t_a$}

Najprej načrtamo enakokraki trikotnik $AON$ ($AO=ON=R$ in $\angle OAN=\frac{1}{2}\left( \beta-\gamma\right)$ (izrek \ref{TockaNbetagama})), nato točko $A_1$ na premici $ON$ iz pogoja $AA_1=t_a$.

   (\textit{d}) \res{$R$, $\beta-\gamma$, $v_a$}

Uporabimo dejstvo, da je $\angle A'AE\cong\angle EAO\frac{1}{2}\left( \beta-\gamma\right)$  (izrek \ref{TockaNbetagama}). Najprej načrtamo pravokotni trikotnik $AA'E$.

   (\textit{e}) \res{$R$, $\beta-\gamma$, $a$}
Načrtaj najprej tetivo $BC=a$ na krožnici $k(O,R)$, nato pa točko $N$ na krožnici $k$ ($ON\perp BC$) in na koncu oglišče $A$ iz pogoja $\angle OAN\cong\angle ONA=\frac{1}{2}\left( \beta-\gamma\right)$ (izrek \ref{TockaNbetagama}).


\item \res{Na nosilki stranice $AB$ pravokotnika $ABCD$ načrtaj točko $E$, iz katere se stranici $AD$ in $DC$ vidita pod enakim kotom. Kdaj ima naloga rešitev?}

Predpostavimo, da je $E$ iskana točka. V tem primeru je $\angle CED\cong\angle DEC\cong\angle EDC$ (izrek \ref{KotaVzporKraki}). Po izreku \ref{enakokraki} je $DCE$ enakokraki trikotnik z osnovnico $DE$ oz. $CD\cong CE$. Torej točko $E$ lahko dobimo kot presečišče premice $AB$ in krožnice $k(C,CD)$. Število rešitev je odvisno od števila presečišč te premice in krožnice.

    \item \res{V konveksnem štirikotniku $ABCD$ velja $BC\cong CD$. Načrtaj ta štirikotnik, če so dani: stranici $AB$ in $AD$ ter notranja kota ob ogliščih $B$ in $D$.}

Najprej načrtamo trikotnik $ABD$ - uporabimo $\angle ADB-\angle ABD=\angle ADC-\angle ABC$
 (glej nalogo \ref{nalSklKonstrTrik}\textit{(u)} - na koncu poglavja \ref{pogSKL}).

    \item \res{V dano krožnico $k$ včrtaj trikotnik $ABC$, če so dani: oglišče $A$, premica $p$, ki je vzporedna z višino $AA'$, in presečišče $B_2$ nosilke višine $BB'$ in te krožnice.}

Uporabimo dejstvo, da je $\angle BAC=90^0-\angle ABB_2=90^0-\frac{1}{2}\angle AOB_2$, kjer je $O$ središče kroga $k$.


    \item \res{Načrtaj pravilni trikotnik $ABC$, če je njegova stranica $BC$ skladna z daljico $a$, nosilki stranic $AB$ in $AC$ ter simetrala notranjega kota $BAC$ pa gredo po vrsti skozi dane točke $M$, $N$ in $P$.}

 Oglišče $A$ dobimo v preseku dveh geometrijskih mest točk, iz katerih se daljica $MP$ oz. $NP$ vidi pod kotom $30^0$ (izrek \ref{ObodKotGMT}).

  \item \res{Načrtaj trikotnik $ABC$, če so dani:}

V vseh primerih uporabimo oznake kot v veliki nalogi \ref{velikaNaloga}.

  \begin{enumerate}
    \item \res{oglišče $A$, središče očrtane krožnice $O$ in središče včrtane krožnice $S$}

Najprej načrtajmo točko $N$ kot presečišče krožnice $k(O, OA)$ in poltraka $AS$ (izrek \ref{TockaN}), nato pa oglišči $B$ in $C$ kot presečišči krožnic  $k(O, OA)$ in  $k_1(N, NS)$ (izrek \ref{TockaN.NBNC}).

    \item \res{središče očrtane krožnice $O$, središče včrtane krožnice $S$ in središče pričrtane krožnice $S_a$}

Uporabimo dejstvo, da je točka $N$ središče daljice $SS_a$ (velika naloga \ref{velikaNaloga}).

    \item  \res{oglišče $A$, središče očrtane krožnice $O$ in višinska točka $V$}

Uporabimo izrek \ref{TockaV'}.

    \item res{oglišči $B$ in $C$ ter simetrala notranjega kota $BAC$}

Pri zrcaljenju čez simetralo $s$  notranjega kota $BAC$ se nosilka $AB$ preslika v nosilko $AC$. To pomeni, da slika $B'$ točke $B$ pri zrcaljenju čez premico $s$ leži na premici $AC$. Torej oglišče $A$ lahko dobimo kot presečišče premic $s$ in $CB'$.

    \item   \res{oglišče $A$, središče očrtane krožnice $O$ in presečišče $E$ stranice $BC$ s simetralo notranjega kota $BAC$}

Najprej načrtajmo točko $N$ kot presečišče krožnice $k(O, OA)$ in poltraka $AE$ (izrek \ref{TockaN}), nato pa oglišči $B$ in $C$ kot presečišči krožnice  $k(O, OA)$ s pravokotnico premice $ON$ v točki $E$.

    \item \res{točke $M$, $P$ in $N$, v katerih nosilki višine in težiščnice iz oglišča $A$ ter simetrala notranjega kota $BAC$ sekajo očrtano krožnico trikotnika}

Najprej načrtamo očrtano krožnico $k(O,R)$ trikotnika $ABC$, ki je hkrati očrtana krožnica trikotnika $MPN$. Ker je $AA'\parallel ON$ ($A'$ je projekcija točke $A$ na nosilko stranice $BC$), oglišče $A$ dobimo kot presečišče vzporednice premici $ON$ skozi točko $M$. Središče stranice $BC$ - točka $A_1$ je presečišče premic $ON$ in $AP$. Na koncu dobimo oglišči $B$ in $C$ kot presečišči pravokotnice premice $ON$ skozi točko $A_1$ s krožnico $k$.

    \item \res{oglišče $A$, središče očrtane krožnice $O$, točka $N$, v kateri simetrala notranjega kota $BAC$ seka očrtano krožnico trikotnika, in daljica $a$, ki je skladna s stranico $BC$}

Uporabimo dejstvo, da je $BC\perp ON$.

  \end{enumerate}

  \item \res{Načrtaj trikotnik $ABC$ s podatki:}

   (\textit{a}) \res{$a$, $b$, $\alpha=3\beta$}

Naj bosta $D$ in $E$ takšni točki na stranici $BC$, da velja $\angle BAD\cong\angle DAE \cong\angle EAC$. Najprej je $DAB$ enakokraki trikotnik in velja $DA\cong DB$ (izrek \ref{enakokraki}). $\angle CDA$ je zunanji kot tega trikotnika, zato je po izreku \ref{zunanjiNotrNotr}
 $$\angle CDA=\angle DBA+\angle DAB =2\alpha=\angle CAD.$$
To pomeni, da je tudi $CAD$ enakokraki trikotnik oz. velja $CD\cong CA=b$ (izrek \ref{enakokraki}). Iz tega sledi $DA\cong DB =a-b$. Torej lahko najprej načrtamo enakokraki trikotnik $CAD$ ($CA\cong CD=b$ in $DA=a-b$), nato pa še točko $B$ iz pogoja $\mathcal{B}(C,D,B)$ in $DB\cong DA$.

   (\textit{b}) \res{$t_a$, $t_c$, $v_b$}

Naj bodo $AA_1$ in $CC_1$ težiščnici ter $BB'$ višina trikotnika $ABC$. Označimo še s $T$ težišče in z $D$ pravokotno projekcijo točke $A_1$ na premico $AC$. Najprej načrtamo pravokotni trikotnik $AA_1D$ ($AA_1=t_a$, $\angle D=90^0$ in $A_1D=\frac{1}{2}v_b$), nato pa oglišče $C$ iz pogoja $TC=\frac{2}{3}t_c$.

\item \res{Skozi točko $M$, ki leži v notranjosti dane krožnice $k$, načrtaj takšno tetivo, da je razlika njenih odsekov (od točke $M$) enaka dani daljici $a$.}

Najprej načrtamo koncentrično krožnico $k_1$, ki poteka skozi točko $M$, nato pa njeno tetivo $MN=a$.

\item \res{Načrtaj trikotnik $ABC$, če poznaš:}

V vseh primerih uporabimo oznake in lastnosti iz velike naloge \ref{velikaNaloga}.

 (\textit{a}) \res{$b-c$, $r$, $r_a$}

Najprej načrtajmo daljico $PP_a=b-c$ in njeno nosilko, nato krožnici $k(S,r)$ in $k(S_a,r_a)$, nazadnje pa še nosilke stranic $AB$ in $AC$ trikotnika $ABC$, ki sta skupni zunanji tangenti teh dveh krožnic. Presečišče zunanjih tangent je oglišče $A$, njuni presečišči s premico $PP_a$ pa sta točki $B$ in $C$.

 (\textit{b}) \res{$a$, $r$, $r_a$}

Uporabimo dejstvo $RR_a=a$.

 (\textit{c}) \res{$a$, $r_b+r_c$, $v_a$}

Uporabimo dejstvo $A_1M=\frac{1}{2}\left(r_b+r_c \right)$. Najprej načrtamo tetivo $BC$, točko $M$ in očrtano krožnico trikotnika $ABC$.

 (\textit{d}) \res{$b+c$, $r_b$, $r_c$}

Uporabimo dejstvo $R_bR_c=b+c$.

 (\textit{e}) \res{$R$, $r_b$, $r_c$}

Uporabimo najprej dejstvo $A_1M=\frac{1}{2}\left(r_b+r_c \right)$ in načrtajmo  krožnico $k(O,R)$, točke $M$, $N$ in $A_1$ ter stranico $BC$. Nato uporabimo  dejstvo $MM'=\frac{1}{2}\left(r_b-r_c \right)$ in načrtamo tangento iz oglišča $B$ na krožnico $k_1(M,\frac{1}{2}\left(r_b-r_c \right))$ - omenjena tangenta je nosilka stranice $AB$.


 (\textit{f}) \res{$b$, $R$, $r+r_a$}

Najprej načrtamo pravokotni trapez $N'NMM'$, nato pa očrtano krožnico  trikotnika $ABC$.

(\textit{g}) \res{$a$, $v_a$, $r_a-r$}

Uporabimo dejstvo $A_1N=\frac{1}{2}\left(r_a-r \right)$ in načrtajmo najprej tetivo $BC$, nato točko $N$ in očrtano krožnico trikotnika $ABC$.

 (\textit{h}) \res{$\alpha$, $r$, $b+c$}

Najprej načrtamo pravokotni trikotnik $ARS$, nato pa točko $N$ iz pogoja $AN'=\frac{1}{2}\left(b+c \right)$.


 \item \res{Dani so: krožnica $k$, njen premer $AB$ in točka $M\notin k$. Samo z  ravnilom načrtaj pravokotnico iz točke $M$ na premico $AB$.}

Naj bosta $D$ in $E$ presečišči premic $AM$ in $BM$ s krožnico $k$ ter $F$ presečišče premic $AE$ in $BF$. Uporabimo dejstvo, da je $M$ višinska točka trikotnika $ABF$.

 \item \res{Dani so: kvadrat $ABCD$ ter takšni točki $M$ in $N$ na stranicah $BC$ in $CD$, da velja $\angle MAN=45^0$.
 Samo z ravnilom načrtaj pravokotnico iz točke $A$ na premico $MN$.}

Naj bosta $E$ in $F$ presečišči diagonale $BD$ kvadrata $ABCD$ s premicama $AM$ in $AN$. Dokažimo najprej, da sta $MF$ in $NE$ višini trikotnika $AMN$.



\end{enumerate}



%REŠITVE - Vektorji
%________________________________________________________________________________

\poglavje{Vectors}


\begin{enumerate}

  %Vsota  in razlika vektorjev
    %_____________________________________

  \item \res{Načrtaj poljubne vektorje $\overrightarrow{a}$, $\overrightarrow{b}$, $\overrightarrow{c}$ in $\overrightarrow{d}$ tako, da je njihova vsota enaka:}

  (\textit{a}) \res{enemu od teh štirih vektorjev}

Relacija $\overrightarrow{a}+\overrightarrow{b}+\overrightarrow{c}+
\overrightarrow{d}=\overrightarrow{d}$ je ekvivalentna z $\overrightarrow{a}+\overrightarrow{b}+\overrightarrow{c}=\overrightarrow{0}$. Torej za $\overrightarrow{d}$ lahko izberemo poljuben vektor, za ostale pa $\overrightarrow{a}=\overrightarrow{PQ}$, $\overrightarrow{b}=\overrightarrow{QR}$ in $\overrightarrow{c}=\overrightarrow{RP}$, kjer so $P$, $Q$ in $R$ poljubne točke.

  (\textit{b}) \res{razliki dveh od teh štirih vektorjev}

Relacija $\overrightarrow{a}+\overrightarrow{b}+\overrightarrow{c}+
\overrightarrow{d}=\overrightarrow{c}-\overrightarrow{d}$ je ekvivalentna z
 $\overrightarrow{a}+\overrightarrow{b}+
=-2\overrightarrow{d}$. Torej vektorje  $\overrightarrow{a}$, $\overrightarrow{b}$ in $\overrightarrow{c}$ lahko izberemo poljubno, za vektor $\overrightarrow{d}$  pa $\overrightarrow{d}=-\frac{1}{2}\left( \overrightarrow{a}+\overrightarrow{b} \right)$.


   \item \res{Naj bo $ABCDE$ petkotnik v neki ravnini. Dokaži, da v tej ravnini obstaja
petkotnik s stranicami, ki določajo enake vektorje, kot jih določajo
 diagonale petkotnika $ABCDE$.}

Uporabimo relacijo:
$\overrightarrow{AC}+\overrightarrow{CE}+\overrightarrow{EB}+\overrightarrow{BD}
+\overrightarrow{DA}=\overrightarrow{0}$



  \item \res{Naj bodo $A$, $B$, $C$ in $D$ poljubne točke v ravnini. Ali splošno velja:}

  (\textit{a}) \res{$\overrightarrow{AB}+\overrightarrow{BD}=\overrightarrow{AD}+\overrightarrow{BC}$?}

Dana relacija je ekvivalentna z $\overrightarrow{AD}=\overrightarrow{AD}+\overrightarrow{BC}$ oz. $\overrightarrow{BC}=\overrightarrow{0}$. Torej relacija velja le v primeru $B=C$, splošno pa ne velja.

  (\textit{b}) \res{$\overrightarrow{AB}=\overrightarrow{DC}\hspace*{1mm}\Rightarrow \hspace*{1mm} \overrightarrow{AC}+\overrightarrow{BD}=2\overrightarrow{BC}$?}

Relacija na levi strani implikacije je ekvivalentna z $\overrightarrow{AC}+\overrightarrow{CB}=\overrightarrow{DB}+\overrightarrow{BC}$ oz.
$\overrightarrow{AB}=\overrightarrow{DC}$. Torej implikacija velja splošno - za poljubne točke $A$, $B$, $C$ in $D$.

  \item \label{nalVekt4}
\res{Dana je daljica $AB$. Samo z ravnilom z možnostjo risanja vzporednic (konstrukcije v afini geometriji) načrtaj točko $C$, tako da bo:}

    (\textit{a}) \res{$\overrightarrow{AC}=-\overrightarrow{AB}$}

Načrtamo paralelograma $ABPQ$ in $APQC$.

   (\textit{b}) \res{$\overrightarrow{AC}=5\overrightarrow{AB}$}

Načrtamo paralelograme $ABPQ$, $BQPC_1$, $BQC_1C_2$, $BQC_2C_3$ in $BQC_3C$.

   (\textit{c}) \res{$\overrightarrow{AC}=-3\overrightarrow{AB}$}

Načrtamo paralelograme $ABPQ$, $BQPC_1$, $BQC_1C_2$, nato uporabimo (\textit{a}).


   \item \res{Naj bo $ABCD$ štirikotnik in $O$ poljubna točka v ravnini tega štirikotnika. Izrazi vektorje stranic in
diagonal tega štirikotnika z vektorji $\overrightarrow{a}=\overrightarrow{OA}$, $\overrightarrow{b}=\overrightarrow{OB}$, $\overrightarrow{c}=\overrightarrow{OC}$ in $\overrightarrow{d}=\overrightarrow{OD}$.}

Po izreku \ref{vektOdsev} je:
\begin{eqnarray*}
 \overrightarrow{AB}&=&\overrightarrow{OB}-\overrightarrow{OA}=
\overrightarrow{b}-\overrightarrow{a}\\
  \overrightarrow{BC}&=&\overrightarrow{OC}-\overrightarrow{OB}=
\overrightarrow{c}-\overrightarrow{b}\\
  \overrightarrow{CD}&=&\overrightarrow{OD}-\overrightarrow{OC}=
\overrightarrow{d}-\overrightarrow{c}\\
  \overrightarrow{DA}&=&\overrightarrow{OA}-\overrightarrow{OD}=
\overrightarrow{a}-\overrightarrow{d}\\
  \overrightarrow{AC}&=&\overrightarrow{OC}-\overrightarrow{OA}=
\overrightarrow{c}-\overrightarrow{a}\\
  \overrightarrow{BD}&=&\overrightarrow{OD}-\overrightarrow{OB}=
\overrightarrow{d}-\overrightarrow{b}
 \end{eqnarray*}

\item \res{Naj bo $ABCD$ štirikotnik in $O$ poljubna točka v ravnini tega štirikotnika. Ali velja ekvivalenca, da je štirikotnik $ABCD$ paralelogram natanko tedaj, ko je $\overrightarrow{OA}+\overrightarrow{OC}=
    \overrightarrow{OB}+\overrightarrow{OD}$?}

Dana relacija je ekvivalentna z $\overrightarrow{OC}+\overrightarrow{DO}=
    \overrightarrow{AO}+\overrightarrow{OB}$ oz. $\overrightarrow{DC}=
    \overrightarrow{AB}$. Po izreku \ref{vektParalelogram} je zadnja relacija ekvivalentna z izjavo, da je $ABCD$ paralelogram.

 \item \res{Naj bo $ABCD$ paralelogram, $S$
presečišče njegovih diagonal in $M$ poljubna točka v ravnini tega paralelograma. Dokaži, da velja:
        $$\overrightarrow{MS} = \frac{1}{4}\cdot
        \left( \overrightarrow{MA}+\overrightarrow{MB}
         +\overrightarrow{MC} + \overrightarrow{MD} \right).$$}

Ker je $S$ središče daljic $AC$ in $BD$ (izrek \ref{paralelogram}), je po izreku \ref{vektSredOSOAOB}:
\begin{eqnarray*}
 \overrightarrow{MA}+\overrightarrow{MB}
         +\overrightarrow{MC} + \overrightarrow{MD}&=&
\left( \overrightarrow{MA}+\overrightarrow{MB} \right)
         +\left( \overrightarrow{MC} + \overrightarrow{MD} \right)=\\
2\cdot\overrightarrow{MS}
         +2\cdot\overrightarrow{MS}=4\cdot\overrightarrow{MS}.
 \end{eqnarray*}


 \item \res{Naj bodo $ABB_1A_2$,
$BCC_1B_2$ in $CAA_1C_2$ paralelogrami, ki so načrtani nad stranicami trikotnika $ABC$. Dokaži, da velja:
$$\overrightarrow{A_1A_2}+\overrightarrow{B_1B_2}+
\overrightarrow{C_1C_2}=\overrightarrow{0}.$$}

Po izreku \ref{vektParalelogram} je:
\begin{eqnarray*}
 \overrightarrow{A_1A_2}+\overrightarrow{B_1B_2}+
\overrightarrow{C_1C_2}
&=&
\overrightarrow{A_1A_2}+\overrightarrow{B_1B_2}+
\overrightarrow{C_1C_2}+\overrightarrow{0}=\\
&=&
\overrightarrow{A_1A_2}+\overrightarrow{B_1B_2}+
\overrightarrow{C_1C_2}+\overrightarrow{AB}+\overrightarrow{BC}+\overrightarrow{CA}=\\
&=&
\overrightarrow{A_1A_2}+\overrightarrow{B_1B_2}+
\overrightarrow{C_1C_2}
+\overrightarrow{A_2B_1}+\overrightarrow{B_2C_1}+\overrightarrow{C_2A_1}=\\
&=&
\overrightarrow{A_1A_2}+\overrightarrow{A_2B_1}+\overrightarrow{B_1B_2}
+\overrightarrow{B_2C_1}+
\overrightarrow{C_1C_2}
+\overrightarrow{C_2A_1}=\\
&=&
\overrightarrow{A_1A_1}=\overrightarrow{0}.
 \end{eqnarray*}


  \item \res{Pravokotni premici $p$ in $q$, ki se sekata v točki $M$, sekata  krožnico $k$ s središčem $O$ v
točkah $A$, $B$, $C$ in $D$. Dokaži, da velja:
$$\overrightarrow{OA}+ \overrightarrow{OB} + \overrightarrow{OC} + \overrightarrow{OD} = 2\overrightarrow{OM}.$$}

Predpostavimo, da premica $p$ seka krožnico $k$ v točkah $A$ in $B$, premica $q$ pa v točkah $C$ in $D$. Označimo s $P$ in $Q$ središči daljic $AB$ in $CD$.
 Iz skladnosti trikotnikov $MPA$ in $MPB$ (izrek \textit{SSS} \ref{SSS}) sledi $\angle MPA\cong \angle MPB=90^0$ oz. $\angle MPO=90^0$. Analogno je tudi  $\angle MQO=90^0$, kar pomeni (ker je še $p\perp q$), da je $MPOQ$ pravokotnik (tudi paralelogram). Če uporabimo izreka \ref{vektParalelogram} in \ref{vektSredOSOAOB}, dobimo:
\begin{eqnarray*}
 \overrightarrow{OA}+ \overrightarrow{OB} + \overrightarrow{OC} + \overrightarrow{OD}
 = 2\cdot\overrightarrow{OP}+ 2\cdot\overrightarrow{OQ} = 2\cdot\left( \overrightarrow{OP}+ \overrightarrow{OQ} \right)=
2\overrightarrow{OM}.
 \end{eqnarray*}


\item \res{Naj bodo $A$, $B$, $C$ in $D$ poljubne točke v neki ravnini. Ali lahko vseh šest daljic, ki jih določajo te točke, orientiramo tako, da je vsota ustreznih šestih vektorjev enaka vektorju nič?}

Označimo $\overrightarrow{DA}=\overrightarrow{a}$, $\overrightarrow{DB}=\overrightarrow{b}$ in
$\overrightarrow{DC}=\overrightarrow{c}$.
V tem primeru je
$\overrightarrow{AB}=\overrightarrow{b}-\overrightarrow{a}$,
$\overrightarrow{AC}=\overrightarrow{c}-\overrightarrow{a}$ in
$\overrightarrow{BC}=\overrightarrow{c}-\overrightarrow{b}$.
 Potrebno je ugotoviti, ali lahko za poljubne vektorje $\overrightarrow{a}$, $\overrightarrow{b}$ in $\overrightarrow{c}$ izberemo takšno orientacijo vektorjev:
$\pm\overrightarrow{a}$, $\pm\overrightarrow{b}$, $\pm\overrightarrow{c}$,
$\pm\left( \overrightarrow{b}-\overrightarrow{a} \right)$,
$\pm\left( \overrightarrow{c}-\overrightarrow{a} \right)$ in
$\pm\left( \overrightarrow{c}-\overrightarrow{b} \right)$
 (za vsakega $+$ ali $-$), da je njihova vsota enaka $\overrightarrow{0}$. Če seštevamo prve tri vektorje (v vseh različnih orientacijah), dobimo možne vsote:\\
$\pm\left( \overrightarrow{a}+\overrightarrow{b}+\overrightarrow{c} \right)$,
$\pm\left( \overrightarrow{a}+\overrightarrow{b}-\overrightarrow{c} \right)$,
$\pm\left( \overrightarrow{a}-\overrightarrow{b}+\overrightarrow{c} \right)$ in
$\pm\left( -\overrightarrow{a}+\overrightarrow{b}+\overrightarrow{c} \right)$,\\
če pa seštevamo drugo trojico, dobimo:\\
$\pm 2\left( \overrightarrow{b}-\overrightarrow{a} \right)$,
$\pm 2\left( \overrightarrow{c}-\overrightarrow{a} \right)$,
$\pm 2\left( \overrightarrow{c}-\overrightarrow{b} \right)$ in
$\overrightarrow{0}$.\\
Jasno je torej, da če so $\overrightarrow{a}$, $\overrightarrow{b}$ poljubni vektorji oz. $A$, $B$, $C$ in $D$ poljubne točke, ne dobimo kot končno vsoto vektor $\overrightarrow{0}$.

Vsoto nič bi lahko dobili le v primeru, ko je npr. $\overrightarrow{c}=\overrightarrow{a}+\overrightarrow{b}$ oz. ko je $ABCD$ paralelogram.
 V tem primeru lahko izberemo:
$\overrightarrow{AB}+\overrightarrow{BC}+\overrightarrow{CD}+\overrightarrow{DA}
+\overrightarrow{BD}+\overrightarrow{CA}=\overrightarrow{0}$.



\item \res{Naj bodo $A_1$, $B_1$ in $C_1$ središča stranic $BC$, $AC$ in $AB$ trikotnika $ABC$ ter $M$ poljubna točka. Dokaži:}

    (\textit{a}) \res{$\overrightarrow{AA_1}+\overrightarrow{BB_1}+
\overrightarrow{CC_1}=\overrightarrow{0}$}

Uporabimo izrek \ref{vektSredOSOAOB}.

   (\textit{b}) \res{$\overrightarrow{MA}+\overrightarrow{MB}+\overrightarrow{MC}=
   \overrightarrow{MA_1}+\overrightarrow{MB_1}+\overrightarrow{MC_1}$}

Uporabimo izrek \ref{vektSredOSOAOB}.

   (\textit{c}) \res{Obstaja tak trikotnik $PQR$, da za njegova oglišča velja
$\overrightarrow{PQ}=\overrightarrow{CC_1}$, $\overrightarrow{PR}=\overrightarrow{BB_1}$ in $\overrightarrow{RQ}=\overrightarrow{AA_1}$}

Direktna posledica (\textit{a}).


 \item \res{Naj bodo $M$, $N$, $P$, $Q$, $R$ in  $S$ po vrsti središča stranic poljubnega šestkotnika.
Dokaži, da velja:
$$\overrightarrow{MN}+\overrightarrow{PQ}+
\overrightarrow{RS}=\overrightarrow{0}.$$}

Naj bodo $M$, $N$, $P$, $Q$, $R$ in  $S$ po vrsti središča stranic $AB$, $BC$, $CD$, $DE$, $EF$ in $FA$ šestkotnika $ABCDEF$. Daljice $MN$, $PQ$ in $RS$ so srednjice trikotnikov $ABC$, $CDE$ in $EFA$ za stranice $AC$, $CE$ in $EA$, zato je po izreku \ref{srednjicaTrikVekt}:
 \begin{eqnarray*}
\overrightarrow{MN}+\overrightarrow{PQ}+
\overrightarrow{RS}
&=&
 \frac{1}{2}\overrightarrow{AC}+\frac{1}{2}\overrightarrow{CE}+
\frac{1}{2}\overrightarrow{EA}=\\
&=&
\frac{1}{2} \left( \overrightarrow{AC}+\overrightarrow{CE}+
\overrightarrow{EA} \right)=
\overrightarrow{0}.
 \end{eqnarray*}

  %Linearna kombinacija vektorjev
    %_____________________________________

  \item \res{Naj bo $ABCDEF$ konveksni šestkotnik, pri katerem je $AB\parallel DE$, točki $K$ in $L$ pa sta središči daljic, ki jih
določajo središča preostalih parov nasprotnih stranic. Dokaži, da je $K=L$ natanko tedaj, ko je $AB\cong DE$.}

Dokažemo, da velja $\overrightarrow{KL}=\frac{1}{4}\left( \overrightarrow{AB}+\overrightarrow{DE} \right)$. Glej zgled \ref{vektPetkoinikZgled}.


 \item \res{Naj bosta $P$ in $Q$ takšni točki stranic $BC$ in $CD$ paralelograma $ABCD$, da je $BP:PC=2:3$ in
$CQ:QD=2:5$. Točka $X$ je presečišče daljic $AP$ in $BQ$. Izračunaj razmerji, v katerih točka $X$ deli
daljici $AP$ in $BQ$.}

Označimo $\overrightarrow{u}=\overrightarrow{AB}$ in $\overrightarrow{v}=\overrightarrow{AD}$. Potem je
$\overrightarrow{AP}=\overrightarrow{AB}+\overrightarrow{BP}
 =\overrightarrow{u}+\frac{2}{5}\overrightarrow{v}$ in
$\overrightarrow{AQ}=\overrightarrow{AD}+\overrightarrow{DQ}
 =\frac{5}{7}\overrightarrow{u}+\overrightarrow{v}$. Iz tega je
 $\overrightarrow{BQ}=\overrightarrow{AQ}-\overrightarrow{AB}
=-\frac{2}{7}\overrightarrow{u}+\overrightarrow{v}$.
 Ker so $A$, $X$ in $P$ kolinearne točke, za nek $\lambda \in \mathbb{R}$ velja $\overrightarrow{AX}=\lambda \overrightarrow{AP}$ (izrek \ref{vektKriterijKolin}). Podobno  za nek $\mu \in \mathbb{R}$ velja $\overrightarrow{BX}=\mu \overrightarrow{BQ}$.
Torej:
\begin{eqnarray*}
\overrightarrow{AX}
&=& \lambda \overrightarrow{AP}=
 \lambda\left( \overrightarrow{u}+\frac{2}{5}\overrightarrow{v} \right)=
 \lambda \overrightarrow{u}+\frac{2}{5}\lambda\overrightarrow{v}
\\
\overrightarrow{AX}
&=& \overrightarrow{AB}+\overrightarrow{BX}=
\overrightarrow{AB}+\mu \overrightarrow{BQ}=
u+ \mu\left( -\frac{2}{7}\overrightarrow{u}+\overrightarrow{v} \right)=
\left(1-\frac{2}{7}\mu\right)\overrightarrow{u}+\mu\overrightarrow{v}.
 \end{eqnarray*}
Ker sta $\overrightarrow{u}$ in $\overrightarrow{v}$ nekolinearna vektorja, je po izreku \ref{vektLinKomb1Razcep}:
\begin{eqnarray*}
\lambda &=& 1-\frac{2}{7}\mu\\
 \frac{2}{5}\lambda &=& \mu.
 \end{eqnarray*}
Če sistem rešimo po $\lambda$ in $\mu$, dobimo $\lambda=\frac{35}{39}$ in $\mu=\frac{14}{39}$. Iz tega sledi $\overrightarrow{AX}=\lambda \overrightarrow{AP}=\frac{35}{39}\overrightarrow{AP}$ in $\overrightarrow{BX}=\mu \overrightarrow{BQ}=\frac{14}{39}\overrightarrow{BQ}$, zato je $AX:XP=35:4$ in $BX:XQ=14:25$.

\item \res{Naj bodo $A$, $B$, $C$ in $D$ poljubne točke neke ravnine. Točka $E$ je središče daljice $AB$, $F$ in
$G$ takšni točki, da velja $\overrightarrow{EF} = \overrightarrow{BC}$ in $\overrightarrow{EG} = \overrightarrow{AD}$,  ter $S$ središče daljice $CD$. Dokaži, da so $G$, $S$ in $F$
kolinearne točke.}

Dokažemo, da je $S$ tudi središče daljice $FG$.
Uporabimi izrek \ref{vektSredOSOAOB}. Druga možnost je, da najprej dokažemo, da je štirikotnik $FCGD$ paralelogram.

\item \res{Naj bosta $K$ in $L$  takšni točki stranice $AD$ in diagonale $AC$ paralelograma $ABCD$, da velja $\frac{\overrightarrow{AK}}{\overrightarrow{KD}}=\frac{1}{3}$ in
    $\frac{\overrightarrow{AL}}{\overrightarrow{LC}}=\frac{1}{4}$. Dokaži, da so $K$, $L$ in $B$ kolinearne točke.}

Po predpostavki je $\overrightarrow{AK}=\frac{1}{4}\overrightarrow{AD}$ in $\overrightarrow{AL}=\frac{1}{5}\overrightarrow{AC}$. Izrazimo vektor $\overrightarrow{AL}$ kot linearno kombinacijo vektorjev $\overrightarrow{AB}$ in $\overrightarrow{AK}$:
 \begin{eqnarray*}
 \overrightarrow{AL}
&=&
\frac{1}{5}\overrightarrow{AC}
 = \frac{1}{5}\left( \overrightarrow{AB} + \overrightarrow{AD} \right)=\\
&=&
 \frac{1}{5}\left( \overrightarrow{AB} + 4\overrightarrow{AK} \right)=\\
&=&
 \frac{1}{5} \overrightarrow{AB} +\frac{4}{5}\overrightarrow{AK}.
 \end{eqnarray*}
 Ker je $\frac{1}{5}+\frac{4}{5}=1$, so po izreku \ref{vektParamPremica}  $K$, $L$ in $B$ kolinearne točke.

\item \res{Naj bosta $X_n$ in $Y_n$ ($n\in \mathbb{N}$) takšni točki stranic $AB$ in $AC$ trikotnika $ABC$, da velja $\overrightarrow{AX_n}=\frac{1}{n+1}\cdot \overrightarrow{AB}$ in
    $\overrightarrow{AY_n}=\frac{1}{n}\cdot \overrightarrow{AC}$. Dokaži, da obstaja točka, ki leži na vseh premicah
$X_nY_n$  ($n\in \mathbb{N}$).}

Naj bo $D$ četrto oglišče paralelograma $BCAD$. Dokažemo, da točka $D$ leži na vseh premicah
$X_nY_n$  ($n\in \mathbb{N}$) oz. da so točke $D$, $X_n$ in $Y_n$ kolinearne za vsak $n\in \mathbb{N}$. Po predpostavki je:
 \begin{eqnarray*}
 \overrightarrow{AX_n}
&=&
\frac{1}{n+1}\overrightarrow{AB}
 = \frac{1}{n+1}\left( \overrightarrow{AD} + \overrightarrow{AC} \right)=\\
&=&
 \frac{1}{n+1}\left( \overrightarrow{AD} + n\overrightarrow{AY_n} \right)=\\
&=&
 \frac{1}{n+1} \overrightarrow{AD} +\frac{n}{n+1}\overrightarrow{AY_n}.
 \end{eqnarray*}
 Ker je $\frac{1}{n+1}+\frac{n}{n+1}=1$, so po izreku \ref{vektParamPremica}  $D$, $X_n$ in $Y_n$ kolinearne točke.


 \item \res{Naj bodo $M$, $N$, $P$ in $Q$ središča stranic $AB$, $BC$, $CD$ in $DA$ štirikotnika
     $ABCD$. Ali velja ekvivalenca, da je štirikotnik $ABCD$ paralelogram natanko tedaj, ko je:}

    (\textit{a}) \res{$2\overrightarrow{MP}=\overrightarrow{BC}+\overrightarrow{AD}$ in
    $2\overrightarrow{NQ}=\overrightarrow{BA}+\overrightarrow{CD}$?}\\
   (\textit{b}) \res{$2\overrightarrow{MP}+2\overrightarrow{NQ}=
   \overrightarrow{AB}+\overrightarrow{BC}+\overrightarrow{CD}+\overrightarrow{DA}$?}

Ekvivalenca ne velja, kajti trditvi (\textit{a}) in (\textit{b}) po izreku \ref{vektSestSplosno} veljata splošno za poljubni štirikotnik $ABCD$.

 \item \res{Naj bodo $E$, $F$ in $G$ središča stranic $AB$, $BC$ in $CD$ paralelograma $ABCD $, premici $BG$ in $DE$ pa sekata premico $AF$ v točkah $N$ in $M$. Izrazi $\overrightarrow{AF}$, $\overrightarrow{AM}$ in $\overrightarrow{AN}$ kot linearno kombinacijo vektorjev $\overrightarrow{AB}$ in $\overrightarrow{AD}$. Dokaži, da točki $M$ in $N$ delita daljico $AF$ v razmerju $2:2:1$.}

Izrazimo vektor $\overrightarrow{AN}$ na dva načina: $\overrightarrow{AN}=\lambda \overrightarrow{AF}$ in $\overrightarrow{AN}=\overrightarrow{AB}+ \mu\overrightarrow{BG}$, nato izračunamo $\lambda$. Podobno tudi za vektor $\overrightarrow{AM}$.

 \item \res{Točke $K$, $L$, $M$ in $N$ ležijo na stranicah $AB$, $BC$, $CD$ in $DA$
štirikotnika $ABCD$. Če je štirikotnik $KLMN$ paralelogram in velja
$$\frac{\overrightarrow{AK}}{\overrightarrow{KB}}=
\frac{\overrightarrow{BL}}{\overrightarrow{LC}}
=\frac{\overrightarrow{CM}}{\overrightarrow{MD}}=
\frac{\overrightarrow{DN}}{\overrightarrow{NA}}=\lambda$$ za nek $\lambda\neq\pm 1$,
 je tudi štirikotnik $ABCD$
paralelogram. Dokaži.}

Ker je $KLMN$ paralelogram, velja $\overrightarrow{KL}=\overrightarrow{NM}$  (izrek \ref{vektParalelogram}).

Iz danih pogojev je $\overrightarrow{AK}=\lambda\overrightarrow{KB}$, iz tega sledi $\overrightarrow{AB}+\overrightarrow{BK}=\lambda\overrightarrow{KB}$ oziroma:
 \begin{eqnarray} \label{nalVekt23a}
\overrightarrow{KB}=\frac{1}{1+\lambda}\cdot \overrightarrow{AB}.
\end{eqnarray}
Podobno iz $\overrightarrow{BL}=\lambda\overrightarrow{LC}$ sledi $\overrightarrow{BL}=\lambda \left( \overrightarrow{LB}+\overrightarrow{BC} \right)$ oziroma:
 \begin{eqnarray} \label{nalVekt23b}
\overrightarrow{BL}=\frac{\lambda}{1+\lambda}\cdot \overrightarrow{BC}.
\end{eqnarray}
Iz \ref{nalVekt23a} in \ref{nalVekt23b} dobimo:

 \begin{eqnarray} \label{nalVekt23c}
\overrightarrow{KL}=\overrightarrow{KB}+\overrightarrow{BL}=
\frac{1}{1+\lambda}\cdot \overrightarrow{AB}+\frac{\lambda}{1+\lambda}\cdot \overrightarrow{BC}.
\end{eqnarray}
Analogno je:
\begin{eqnarray} \label{nalVekt23d}
\overrightarrow{MN}=\overrightarrow{MD}+\overrightarrow{DN}=
\frac{1}{1+\lambda}\cdot \overrightarrow{CD}+\frac{\lambda}{1+\lambda}\cdot \overrightarrow{DA}.
\end{eqnarray}
Iz $\overrightarrow{KL}=\overrightarrow{NM}$ (oz. $\overrightarrow{KL}+\overrightarrow{MN}=\overrightarrow{0}$), \ref{nalVekt23c} in \ref{nalVekt23d} dobimo:
\begin{eqnarray*}
\frac{1}{1+\lambda}\cdot \overrightarrow{AB}+\frac{\lambda}{1+\lambda}\cdot \overrightarrow{BC}+
\frac{1}{1+\lambda}\cdot \overrightarrow{CD}+\frac{\lambda}{1+\lambda}\cdot \overrightarrow{DA}=\overrightarrow{0}
\end{eqnarray*}
oz.
\begin{eqnarray*}
 \overrightarrow{AB}+\lambda\cdot \overrightarrow{BC}+
 \overrightarrow{CD}+\lambda\cdot \overrightarrow{DA}=\overrightarrow{0}.
\end{eqnarray*}
Iz tega sledi:
\begin{eqnarray*}
\overrightarrow{0}
&=&
\overrightarrow{AB}+\lambda\cdot \overrightarrow{BC}+
 \overrightarrow{CD}+\lambda\cdot \overrightarrow{DA}=\\
 &=&
\overrightarrow{AB}+\overrightarrow{BC}+\left(\lambda-1\right)\cdot \overrightarrow{BC}+
 \overrightarrow{CD}+\overrightarrow{DA}+\left(\lambda-1\right)\cdot \overrightarrow{DA}=\\
 &=&
\overrightarrow{AC}+\left(\lambda-1\right)\cdot \overrightarrow{BC}+
 \overrightarrow{CA}+\left(\lambda-1\right)\cdot \overrightarrow{DA}=\\
 &=&
\left(\lambda-1\right)\cdot \overrightarrow{BC}+
\left(\lambda-1\right)\cdot \overrightarrow{DA}=\\
 &=&
\left(\lambda-1\right)\cdot \left(\overrightarrow{BC}+
 \overrightarrow{DA}\right).
\end{eqnarray*}
Ker je po predpostavki $\lambda\neq 1$, je $\overrightarrow{BC}+
 \overrightarrow{DA}=\overrightarrow{0}$ oz. $\overrightarrow{BC}=
 \overrightarrow{AD}$, kar pomeni (izrek \ref{vektParalelogram}), da je $ABCD$ paralelogram.

\item \res{Naj bo $M$ središče stranice $DE$ pravilnega šestkotnika $ABCDEF$. Točka
$N$ je središče daljice $AM$, točka $P$ pa središče stranice $BC$. Izrazi $\overrightarrow{NP}$ kot linearno kombinacijo vektorjev
$\overrightarrow{AB}$ in $\overrightarrow{AF}$.}

Označimo s $S$ središče pravilnega šestkotnika $ABCDEF$, $\overrightarrow{u}=\overrightarrow{AB}$ in $\overrightarrow{v}=\overrightarrow{AF}$.
Potem je:
\begin{eqnarray*}
 \overrightarrow{NP}
 &=&
 \overrightarrow{AP}-\overrightarrow{AN}=
\overrightarrow{AB}+\overrightarrow{BP}-\frac{1}{2}\overrightarrow{AM}=\\
 &=&
 \overrightarrow{u}+\frac{1}{2}\overrightarrow{BC}
-\frac{1}{2}\left( \overrightarrow{AD}+\overrightarrow{DM} \right)= \\
 &=&
 \overrightarrow{u}+\frac{1}{2}\overrightarrow{AS}
-\frac{1}{2}\left( 2\overrightarrow{AS}-\frac{1}{2}\overrightarrow{u} \right)= \\
 &=&
 \frac{5}{4}\overrightarrow{u}-\frac{1}{2}\overrightarrow{AS}= \\
 &=&
 \frac{5}{4}\overrightarrow{u}-
\frac{1}{2}\left( \overrightarrow{u}+\overrightarrow{v} \right)= \\
 &=&
 \frac{3}{4}\overrightarrow{u}-
\frac{1}{2}\overrightarrow{v}.
\end{eqnarray*}
Drugi način je, da uporabimo direktno izrek \ref{vektSestSplosno} oz.:
\begin{eqnarray*}
\overrightarrow{NP}
&=&
\frac{1}{2}\left( \overrightarrow{AB}+\overrightarrow{MC} \right)=\\
&=&
\frac{1}{2}\left( \overrightarrow{u}+\overrightarrow{MD}+\overrightarrow{DC} \right)=\\
&=&
\frac{1}{2}\left( \overrightarrow{u}+\frac{1}{2}\overrightarrow{u}-\overrightarrow{v} \right)=\\
 &=&
 \frac{3}{4}\overrightarrow{u}-
\frac{1}{2}\overrightarrow{v}.
\end{eqnarray*}

  %Dolžina vektorja
    %_____________________________________

  \item \res{Dokaži, da za poljubne točke $A$, $B$ in $C$ velja:}

    (\textit{a}) \res{$|\overrightarrow{AC}|\leq|\overrightarrow{AB}|+|\overrightarrow{BC}|$; \hspace*{6mm}}
   (\textit{b}) \res{$|\overrightarrow{AC}|\geq|\overrightarrow{AB}|-|\overrightarrow{BC}|$\\
  Pod katerimi pogoji velja enakost?}

Direktna posledica trikotniške neenakosti \ref{neenaktrik}. Enakost velja natanko tedaj, ko je $\mathcal{B}(A,B,C)$.

  \item \res{ Naj bosta $M$ in $N$ takšni točki, ki ležita na daljicah $AD$ oz. $BC$, tako da velja $\frac{\overrightarrow{AM}}{\overrightarrow{MD}}\cdot \frac{\overrightarrow{CN}}{\overrightarrow{NB}}=1$. Dokaži, da je:
      $$|MN|\leq\max\{|AB|, |CD|\}.$$}

Iz danega pogoja  sledi $\frac{\overrightarrow{AM}}{\overrightarrow{MD}}= \frac{\overrightarrow{NB}}{\overrightarrow{CN}}=\lambda$ oz. $\overrightarrow{AM}=\lambda\overrightarrow{MD}$  in $\overrightarrow{NB}=\lambda\overrightarrow{CN}$.
Ker je $\mathcal{B}(A,M,D)$ in $\mathcal{B}(B,N,C)$, je $\lambda\geq 0$. Naprej je:
$\overrightarrow{MN}=\overrightarrow{MA}+\overrightarrow{AB}+\overrightarrow{BN}$ in $\overrightarrow{MN}=\overrightarrow{MD}+\overrightarrow{DC}+\overrightarrow{CN}$.
 Če drugo relacijo množimo z $\lambda$ in dodamo prvo, dobimo:
 $(1+\lambda)\overrightarrow{MN}=\overrightarrow{AB}+\lambda \overrightarrow{DC}$ oz.:
$$\overrightarrow{MN}=\frac{1}{1+\lambda}\overrightarrow{AB}+\frac{\lambda}{1+\lambda} \overrightarrow{DC}.$$
Brez škode za splošnost predpostavimo, da je $|AB|\geq|CD|$. Če uporabimo trikotniško neenakost za vektorje \ref{neenakTrikVekt}, dobimo
 \begin{eqnarray*}
 |\overrightarrow{MN}|
 &=&
 |\frac{1}{1+\lambda}\overrightarrow{AB}+\frac{\lambda}{1+\lambda} \overrightarrow{DC}|\leq\\
 &\leq&
 |\frac{1}{1+\lambda}\overrightarrow{AB}|+|\frac{\lambda}{1+\lambda} \overrightarrow{DC}|=\\
 &=&
\frac{1}{1+\lambda}|\overrightarrow{AB}|+\frac{\lambda}{1+\lambda} |\overrightarrow{DC}|\leq\\
 &\leq&
\frac{1}{1+\lambda}|\overrightarrow{AB}|+\frac{\lambda}{1+\lambda} |\overrightarrow{AB}|=\\
 &=&
\left( \frac{1}{1+\lambda}+\frac{\lambda}{1+\lambda} \right)\cdot|\overrightarrow{AB}|=\\
 &=&|\overrightarrow{AB}|.
 \end{eqnarray*}



  %Težišče
    %_____________________________________

  \item \label{nalVekt24}
\res{Točki $T$ in $T'$ sta težišči $n$-kotnikov $A_1A_2...A_n$ in $A'_1A'_2...A'_n$. Izračunaj:
$$\overrightarrow{A_1A'_1}+\overrightarrow{A_2A'_2}+\cdots+\overrightarrow{A_nA'_n}.$$}

Uporabimo relacijo $\overrightarrow{A_kA'_k}
=\overrightarrow{A_kT}+\overrightarrow{TT'}+\overrightarrow{T'A'_k}$ ($k\in \{1,2,\ldots ,n \}$) in definicijo težišča. Rezultat $n\cdot\overrightarrow{TT'}$.

    \item \res{Dokaži, da imata štirikotnika $ABCD$ in  $A'B'C'D'$ skupno težišče natanko tedaj, ko je:
$$\overrightarrow{AA'}+\overrightarrow{BB'}+\overrightarrow{CC'}+
\overrightarrow{DD'}=\overrightarrow{0}.$$}

 Uporabimo prejšnjo nalogo \ref{nalVekt24}.

    \item \res{Naj bodo $P$, $Q$, $R$ in $S$ težišča trikotnikov $ABD$, $BCA$, $CDB$ in $DAC$. Dokaži, da imata
štirikotnika $PQRS$ in $ABCD$ skupno težišče.}

Naj bo $T$ težišče štirikotnika $ABCD$. Po definiciji je: $\overrightarrow{TA}+\overrightarrow{TB}
+\overrightarrow{TC}+\overrightarrow{TD}=\overrightarrow{0}$. Iz tega in predpostavke, da so $P$, $Q$, $R$ in $S$ težišča trikotnikov $ABD$, $BCA$, $CDB$ in $DAC$ ter izreka \ref{vektTezTrikXT} je:
\begin{eqnarray*}
\overrightarrow{TP}+\overrightarrow{TQ}
+\overrightarrow{TR}+\overrightarrow{TS}
&=&
\frac{1}{3}\left( \overrightarrow{TA}+\overrightarrow{TB}+\overrightarrow{TD} \right)+\\
&+&
\frac{1}{3}\left( \overrightarrow{TB}+\overrightarrow{TC}+\overrightarrow{TA} \right)+\\
&+&
\frac{1}{3}\left( \overrightarrow{TC}+\overrightarrow{TD}+\overrightarrow{TB} \right)+\\
&+&
\frac{1}{3}\left( \overrightarrow{TD}+\overrightarrow{TA}+\overrightarrow{TC} \right)=\\
&=&
 \overrightarrow{TA}+\overrightarrow{TB}+\overrightarrow{TC}+ \overrightarrow{TD}=\overrightarrow{0},
\end{eqnarray*}
 kar pomeni, da je točka $T$ hkrati težišče štirikotnika $PQRS$.

  \item \res{Naj bo $A_1A_2A_3A_4A_5A_6$ poljubni šestkotnik in $B_1$, $B_2$, $B_3$, $B_4$, $B_5$ in $B_6$ po vrsti težišča trikotnikov $A_1A_2A_3$, $A_2A_3A_4$, $A_3A_4A_5$, $A_4A_5A_6$, $A_5A_6A_1$ in $A_6A_1A_2$.
Dokaži, da ta težišča določajo šestkotnik s tremi pari vzporednih stranic.}

Uporabimo izrek \ref{vektTezTrikXT}.

 \item \res{Naj bodo $A$, $B$, $C$ in $D$ štiri različne točke. Točke $T_A$, $T_B$, $T_C$ in $T_D$
        so težišča trikotnikov $BCD$, $ACD$, $ABD$ in $ABC$. Dokaži, da se daljice $AT_A$, $BT_B$, $CT_C$ in $DT_D$ sekajo v eni točki $T$. V katerem razmerju točka $T$ deli te daljice?}

Uporabimo izrek \ref{vektTezTrikXT}. Iskano razmerje je $3:1$.
Uporabimo tudi zgled \ref{TetivniTezisce}.

 \item \label{nalVekt29}
 \res{Naj bo $CC_1$ težiščnica trikotnika $ABC$ in $P$ poljubna točka na stranici
$AB$ tega trikotnika. Vzporednica $l$ premice $CC_1$ skozi točko $P$ seka premici $AC$ in $BC$ v točkah $M$ in $N$. Dokaži, da velja:
$$\overrightarrow{PM} + \overrightarrow{PN}= \overrightarrow{AC} + \overrightarrow{BC}.$$}

Najprej je $\overrightarrow{AC} + \overrightarrow{BC}=-(\overrightarrow{CA} + \overrightarrow{CB})=-2\overrightarrow{CC_1}=2\overrightarrow{C_1C}$ (izrek \ref{vektSredOSOAOB}).
Po Talesovem izreku \ref{TalesovIzrek} je: $\frac{\overrightarrow{PN}}{\overrightarrow{C_1C}}=
\frac{\overrightarrow{BP}}{\overrightarrow{BC_1}}=
\frac{|BP|}{|BC_1|}$ oz. $\overrightarrow{PN}=\frac{|BP|}{|BC_1|}\cdot \overrightarrow{C_1C}$.
Podobno je tudi:
$\frac{\overrightarrow{PM}}{\overrightarrow{C_1C}}=
\frac{\overrightarrow{AP}}{\overrightarrow{AC_1}}=
\frac{|AP|}{|AC_1|}$ oz. $\overrightarrow{PM}=\frac{|AP|}{|AC_1|}\cdot \overrightarrow{C_1C}$.
 Torej:
 \begin{eqnarray*}
 \overrightarrow{PM} + \overrightarrow{PN}
&=&
\frac{|BP|}{|BC_1|}\cdot \overrightarrow{C_1C}+\frac{|AP|}{|AC_1|}\cdot \overrightarrow{C_1C}=\\
&=&
\frac{|BP|+|AP|}{\frac{1}{2}|AB|}\overrightarrow{C_1C}=2\overrightarrow{C_1C}=\\
&=&
\overrightarrow{AC} + \overrightarrow{BC}.
 \end{eqnarray*}

 %Hamilton in Euler
 %______________________________________________________


 \item \res{Naj bodo $A$, $B$, $C$ točke neke ravnine, ki ležijo na isti strani premice $p$, in $O$ točka na
premici $p$, za katero velja  $|\overrightarrow{OA}| = |\overrightarrow{OB}| = |\overrightarrow{OC}| =1$. Dokaži, da potem velja tudi: $$|\overrightarrow{OA} + \overrightarrow{OB} + \overrightarrow{OC}| \geq 1.$$}

Točka $O$ je središče očrtane krožnice $k(O,1)$ trikotnika $ABC$. Po Hamiltonovem izreku \ref{Hamilton} je $\overrightarrow{OA} + \overrightarrow{OB} + \overrightarrow{OC}=\overrightarrow{OV}$, kjer je $V$ višinska točka trikotnika $ABC$. Ker točke $A$, $B$, $C$ ležijo na isti strani premice $p$, je središče $O$ zunanja točka tega trikotnika. To pomeni, da je $ABC$ topokotni trikotnik. Brez škode za splošnost naj bo $\angle BAC>90^0$. To pomeni $\mathcal{B}(A',A,V)$ (kjer je $A'$ višina tega trikotnika) oz. $V$ je zunanja točka krožnice $k$ in velja $|\overrightarrow{OV}|>1$.
 Lahko pogledamo tudi nalogo \ref{OlimpVekt15} (razdelek \ref{odd5DolzVekt}).

\item \res{Izračunaj kote, ki jih določajo vektorji $\overrightarrow{OA}$, $\overrightarrow{OB}$ in $\overrightarrow{OC}$, če točke $A$, $B$ in $C$ ležijo na krožnici s središčem
$O$ in dodatno velja še:
$$\overrightarrow{OA} + \overrightarrow{OB} + \overrightarrow{OC} = \overrightarrow{0}.$$}

Po Hamiltonovem izreku \ref{Hamilton} je $\overrightarrow{0}=\overrightarrow{OA} + \overrightarrow{OB} + \overrightarrow{OC}=\overrightarrow{OV}$, kjer je $V$ višinska točka trikotnika $ABC$. Torej $V=O$, kar pomeni, da je $ABC$ enakostranični trikotnik so vsi iskani koti enaki $120^0$.

 \item \res{Naj bodo $A$, $B$, $C$ in $D$ točke, ki ležijo na krožnici s središčem
$O$, in velja
$$\overrightarrow{OA} + \overrightarrow{OB} + \overrightarrow{OC} + \overrightarrow{OD} = \overrightarrow{0}.$$
Dokaži, da je $ABCD$ pravokotnik.}

Uporabimo Hamiltonov izrek \ref{Hamilton}.

\item \res{Naj bodo $\overrightarrow{a}$, $\overrightarrow{b}$ in $\overrightarrow{c}$ vektorji neke ravnine, za katere velja $|\overrightarrow{a}| = |\overrightarrow{b}| = |\overrightarrow{c}| =x$. Razišči, v katerem primeru velja tudi $|\overrightarrow{a} + \overrightarrow{b} + \overrightarrow{c}| = x$.}

Predpostavimo najprej, da so $\overrightarrow{a}$, $\overrightarrow{b}$ in $\overrightarrow{c}$ nekolinearni vektorji.
Naj bo $O$ poljubna točka in $A$, $B$ in $C$ takšne točke, da velja $\overrightarrow{OA}=\overrightarrow{a}$, $\overrightarrow{OB}=\overrightarrow{b}$ in $\overrightarrow{OC}=\overrightarrow{c}$ (izrek \ref{vektAvObst1TockaB}). Iz nekolinearnosti vektorjev $\overrightarrow{a}$, $\overrightarrow{b}$ in $\overrightarrow{c}$ sledi nekolinearnost točk $A$, $B$ in $C$. Iz pogoja $|\overrightarrow{a}| = |\overrightarrow{b}| = |\overrightarrow{c}| =x$ sledi $|\overrightarrow{OA}| = |\overrightarrow{OB}| = |\overrightarrow{OC}| =x$, kar pomeni, da je točka $O$ središče očrtane krožnice trikotnika $ABC$. Po Hamiltonovem izreku \ref{Hamilton} je:
 \begin{eqnarray*}
x
 &=&
|\overrightarrow{a} + \overrightarrow{b} + \overrightarrow{c}|=\\
 &=&
|\overrightarrow{OA} + \overrightarrow{OB} + \overrightarrow{OC}|=\\
 &=& |\overrightarrow{OV}|,
\end{eqnarray*}
 kjer je $V$ višinska točka, ki torej leži na očrtani krožnici trikotnika $ABC$. To je možno natanko tedaj, ko je $ABC$ pravokotni trikotnik.

Če so $\overrightarrow{a}$, $\overrightarrow{b}$ in $\overrightarrow{c}$ kolinearni vektorji, je dani pogoj izpolnjen natanko tedaj, ko $\overrightarrow{a}$, $\overrightarrow{b}$ in $\overrightarrow{c}$ niso vsi iste orientacije.


 %Tales
 %______________________________________________________


\item \res{Dano daljico $AB$ razdeli:}

  (\textit{a}) \res{na pet enakih daljic,}\\
  (\textit{b}) \res{v razmerju $2:5$,}\\
  (\textit{c}) \res{na tri daljice, ki so v razmerju $2:\frac{1}{2}:1$.}

Uporabimo zgled \ref{izrekEnaDelitevDaljice}.


  \item \res{Dana je daljica $AB$. Samo z uporabo ravnila z možnostjo risanja vzporednic (konstrukcije v afini geometriji) načrtaj točko $C$, če je:}

(\textit{a}) \res{$\overrightarrow{AC}=\frac{1}{3}\overrightarrow{AB}$,} \hspace*{3mm}
   (\textit{b}) \res{$\overrightarrow{AC}=\frac{3}{5}\overrightarrow{AB}$,} \hspace*{3mm}
   (\textit{c}) \res{$\overrightarrow{AC}=-\frac{4}{7}\overrightarrow{AB}$.}

Uporabimo nalogo \ref{nalVekt4} in zgled \ref{izrekEnaDelitevDaljiceNan}.


  \item \res{Dane so daljice $a$, $b$ in $c$. Načrtaj daljico $x$, tako da bo:}

     (\textit{a}) \res{$a:b=c:x$;} \hspace*{3mm}
    (\textit{b}) \res{$x=\frac{a\cdot b}{c}$,} \hspace*{3mm}
   (\textit{c}) \res{$x=\frac{a^2}{c}$,}\hspace*{3mm}\\
   (\textit{č}) \res{$x=\frac{2ab}{3c}$,}\hspace*{3mm}
   (\textit{č}) \res{$(x+c):(x-c)=7:2$.}

Uporabimo talesov izrek \ref{TalesovIzrek}.

 \item \res{Naj bosta $M$ in $N$ točki na kraku $OX$,  $P$ točka na kraku $OY$ kota
$XOY$ ter $NQ\parallel MP$ in $PN\parallel QS$ ($Q\in OY$, $S\in OX$). Dokaži, da je
$|ON|^2=|OM|\cdot |OS|$ (za daljico $ON$ v tem primeru pravimo, da je \index{geometrijska sredina daljic}\pojem{geometrijska sredina} \color{green1} daljic $OM$
in $OS$).}

Po Talesovem izreku  \ref{TalesovIzrek} je:
 $$\frac{|ON|}{|OM|}=\frac{|OQ|}{|OP|}=\frac{|OS|}{|ON|},$$ iz tega sledi iskana relacija.

\item \res{Naj bo  $ABC$ trikotnik ter $Q$, $K$, $L$, $M$, $N$ in $P$ takšne točke poltrakov $AB$, $AC$, $BC$,
$BA$, $CA$ in $CB$, da velja $AQ\cong CP\cong AC$, $AK\cong BL\cong AB$ in $BM\cong CN\cong BC$.
Dokaži, da so $MN$, $PQ$ in $LK$ tri vzporednice.}

Uporabimo izreka \ref{vektParamPremica} in \ref{vektDelitDaljice}.

\item \res{Naj bo $P$ središče težiščnice $AA_1$ trikotnika $ABC$. Točka $Q$ je presečišče premice $BP$
s stranico $AC$. Izračunaj razmerji $AQ:QC$ in $BP:PQ$.}

Naj bo $R$ središče daljice $QC$. Po izreku \ref{srednjicaTrikVekt} (za trikotnika $BQC$ in $AA_aR$) je:
 $$\overrightarrow{PQ}=\frac{1}{2}\overrightarrow{A_1R}=\frac{1}{4}\overrightarrow{BQ},$$
 zato je $BP:PQ=3:1$. Iz $\overrightarrow{PQ}=\frac{1}{2}\overrightarrow{A_1R}$ sledi $PQ\parallel A_1R$, zato je po Talesovem izreku $AQ:QR=AP:PA_1=1:1$. Torej $AQ\cong QR \cong RC$ oz. $AQ:QC=1:2$.

Glej tudi zgled \ref{TezisceSredisceZgled}.

\item \res{Točki $P$ in $Q$ ležita na stranicah $AB$ in $AC$ trikotnika $ABC$, pri tem pa velja $\frac{|\overrightarrow{PB}|}{|\overrightarrow{AP}|}
    +\frac{|\overrightarrow{QC}|}{|\overrightarrow{AQ}|}=1$. Dokaži, da težišče tega trikotnika leži na daljici $PQ$.}

Naj bo $\widehat{T}$ presečišče premice $PQ$ s težiščnico $AA_1$ tega trikotnika. Dovolj je dokazati $\widehat{T}=T$.
Ker so točke $A$, $\widehat{T}$ in $A_1$ kolinearne, je $\overrightarrow{A\widehat{T}}=\lambda\overrightarrow{AA_1}$ za nek $\lambda\in R$. Za $\widehat{T}=T$ je dovolj dokazati, da je $\lambda=\frac{2}{3}$.

Označimo $\alpha=\frac{|\overrightarrow{PB}|}{|\overrightarrow{AP}|}$ in $\beta=\frac{|\overrightarrow{QC}|}{|\overrightarrow{AQ}|}$. Po predpostavki je $\alpha+\beta=1$. Iz danih pogojev je
 $\overrightarrow{AB}=\left(1+\alpha\right)\cdot\overrightarrow{AP}$ in
 $\overrightarrow{AC}=\left(1+\beta\right)\cdot\overrightarrow{AQ}$. Če uporabimo slednji relaciji in izrek \ref{vektSredOSOAOB}, dobimo:
 \begin{eqnarray*}
 \overrightarrow{A\widehat{T}} &=& \lambda\overrightarrow{AA_1}=\\
 &=&
\frac{\lambda}{2}\cdot \left(\overrightarrow{AB}+\overrightarrow{AC}\right)=\\
 &=&
\frac{\lambda}{2}\cdot \left(\left(1+\alpha\right)\cdot\overrightarrow{AP}
+\left(1+\beta\right)\cdot\overrightarrow{AQ}\right)=\\
 &=&
\frac{\lambda}{2}\cdot \left(1+\alpha\right)\cdot\overrightarrow{AP}
+\frac{\lambda}{2}\left(1+\beta\right)\cdot\overrightarrow{AQ}.
 \end{eqnarray*}
Ker točka $T$ leži na premici $PQ$ in ker je $\alpha+\beta=1$, je po izreku \ref{vektParamPremica}:
 \begin{eqnarray*}
1&=&\frac{\lambda}{2}\cdot \left(1+\alpha\right)
+\frac{\lambda}{2}\left(1+\beta\right)=\\
&=&\frac{\lambda}{2}\cdot \left(1+\alpha+1+\beta\right)=\\
 &=&\frac{3\lambda}{2}\\
 \end{eqnarray*}
oz. $\lambda=\frac{2}{3}$.


 \item \res{Naj bodo $a$, $b$ in $c$ trije poltraki s skupnim izhodiščem $S$ ter $M$
točka na poltraku $a$. Če se točka $M$ “giblje” po poltraku $a$,
 je razmerje razdalj te točke od premic $b$ in $c$ konstantno. Dokaži.}

Uporabimo Talesov izrek  \ref{TalesovIzrek}.

 \item \res{Naj bo $D$ točka, ki leži na stranici $BC$ trikotnika $ABC$ ter
 $F$ in $G$ točki, v katerih premica, ki poteka skozi točko $D$ in je vzporedna  s težiščnico $AA_1$, seka premici $AB$ in $AC$.
Dokaži, da je vsota $|DF|+|DG|$ konstantna, če se točka $D$ “giblje” po stranici
$BC$.}

Dokažemo, da je $|DF|+|DG|=|AA_1|$. Glej nalogo \ref{nalVekt29}.


   \item
   \res{Načrtaj trikotnik s podatki:}

V obeh primerih uporabi iste oznake kot v veliki nalogi \ref{velikaNaloga} in izreku \ref{velNalTockP'}.

   (\textit{a}) \res{$v_a$, $r$, $b-c$}

Najprej narišemo pravokotni trikotnik $SPP_a$ ($SP=r$, $\angle P=90^0$ in $PP_a=b-c$), točko $P'$ in na koncu uporabimo dejstvo, da oglišče $A$ leži na poltraku $P_aP'$.

   (\textit{b}) \res{$\beta$, $r$, $b-c$}

Podobno kot v prejšnjem primeru.


\end{enumerate}




%REŠITVE -  Izometrije
%________________________________________________________________________________

\poglavje{Isometries}

\begin{enumerate}
  \item
  \res{Dana je premica $p$ ter točki $A$ in $B$, ki ležita na
  nasprotnih straneh premice $p$. Konstruiraj točko
  $X$, ki leži na
premici $p$, tako da bo razlika $|AX|-|XB|$ maksimalna.}

Naj bo $A'=\mathcal{S}_p(A)$. Dokažemo, da je $X=A'B\cap p$ iskana
točka (glej tudi zgled \ref{HeronProbl}).

  \item \res{V ravnini so dane premice $p$, $q$ in $r$. Konstruiraj
  enakostranični trikotnik $ABC$, tako da
   oglišče $B$ leži na premici $p$, $C$ na $q$,
višina iz oglišča $A$ pa na premici $r$.}

Ker je $\mathcal{S}_r(B)=C$ in $B\in p$, oglišče $C$ dobimo iz
pogoja $C\in \mathcal{S}_r(p)\cap q$.

\item \res{Dan je štirikotnik $ABCD$ in točka $S$. Načrtaj paralelogram
s središčem v točki $S$, tako
da njegova oglišča ležijo na nosilkah
stranic danega štirikotnika.}

Uporabimo središčno zrcaljenje $\mathcal{S}_S$.

\item \res{Naj bo $\mathcal{I}$ indirektna izometrija ravnine, ki
 preslika  točko $A$ v točko $B$,
$B$ pa v $A$. Dokaži, da je $\mathcal{I}$ osno zrcaljenje.}

Naj bo $s$ simetrala daljice $AB$. Kompozitum
$\mathcal{I}\circ\mathcal{S}_s$ je direktna izometrija z dvema
fiksnima točkama $A$ in $B$, zato po izreku \ref{izo2ftIdent}
predstavlja identiteto oz.
$\mathcal{I}\circ\mathcal{S}_s=\mathcal{E}$. Če množimo enakost
z desne s $\mathcal{S}_s$, dobimo $\mathcal{I}=\mathcal{S}_s$.

\item  \res{Naj bosta $K$ in $L$ točki, ki sta simetrični z ogliščem
$A$ trikotnika $ABC$ glede na
simetrali notranjih kotov ob ogliščih  $B$ in $C$. Točka $P$ naj bo
dotikališče včrtane krožnice tega trikotnika in stranice $BC$.
Dokaži, da je $P$ središče daljice $KL$.}

Označimo s $s_\beta$ in $s_\gamma$ omenjene simetrale notranjih
kotov ob ogliščih  $B$ in $C$ ter s $Q$ in z $R$ dotikališči
včrtane krožnice trikotnika $ABC$ s stranicama $AC$ in $AB$. Po
izreku \ref{TangOdsek} je $AQ\cong AR$, $BR\cong BP$ in $CP\cong
CQ$. Torej $\mathcal{S}_{s_\beta}:A,R\mapsto K,P$ in
$\mathcal{S}_{s_\gamma}:A,Q\mapsto L,P$. Iz tega sledi $KP\cong
AR\cong AQ\cong LP$, kar pomeni, da je $P$ središče daljice
$KL$.

\item  \res{Naj bosta $k$ in $l$ krožnici na različnih bregovih premice
$p$. Načrtaj enakostranični trikotnik $ABC$, tako da njegova višina
$AA'$ leži na premici $p$, oglišče $B$ na krožnici $k$, oglišče
$C$ pa na krožnici $l$.}

Ker velja $\mathcal{S}_p(B)=C$ in $B\in k$, točko $C$ dobimo iz
pogoja $C\in \mathcal{S}_p(k)\cap l$.

\item \res{Naj bo $k$ krožnica ter $a$, $b$ in $c$ premice v isti ravnini.
Načrtaj trikotnik $ABC$, ki je
včrtan krožnici $k$, tako da bodo njegove stranice $BC$, $AC$ in
 $AB$ vzporedne po vrsti s premicami $a$, $b$ in $c$.}

 Najprej lahko načrtamo simetrale stranic $BC$, $AC$ in $AB$ kot
 premice, ki gredo skozi središče $O$ krožnice $k$ in so
 pravokotne po vrsti s premicami $a$, $b$ in $c$. Označimo te
 simetrale s $p$, $q$ in $r$. Ker ima kompozitum
 $\mathcal{I}=\mathcal{S}_r\circ\mathcal{S}_q\circ\mathcal{S}_p$
  fiksni točki $O$ in $A$, je po izreku \ref{izo1ftIndZrc}
 $\mathcal{I}=\mathcal{S}_{OA}$. Torej lahko premico $OA$
 narišemo kot simetralo daljice $XX'$, kjer je $X$ poljubna
 točka. Oglišče $A$ dobimo kot presečišče premice $OA$ in
 krožnice $k$. Nato je $B=\mathcal{S}_p(A)$ in
 $C=\mathcal{S}_r(A)$.

\item  \res{Naj bo $ABCDE$ tetivni petkotnik, v katerem je $BC\parallel
DE$ in $CD\parallel EA$.
Dokaži, da oglišče $D$ leži na simetrali daljice $AB$.}

Označimo z $O$ središče očrtane krožnice petkotnika $ABCDE$. Ker
je $BC\parallel DE$, imata premici $BC$ in $DE$ isto pravokotnico
iz središča $S$, ki je pravzaprav skupna simetrala stranic $BC$
in $DE$. Označimo jo s $p$. Podobno imata stranici $CD$ in $EA$
skupno simetralo $q$. Označimo še z $r$ simetralo daljice $AB$.
Dokažemo, da velja $D\in r$. Kompozitum
 $\mathcal{S}_{OD}\circ\mathcal{S}_q
\circ\mathcal{S}_p\circ\mathcal{S}_r
\circ\mathcal{S}_q\circ\mathcal{S}_p$ je direktna izometrija, ki
ima fiksni točki $O$ in $D$, zato po izreku \ref{izo2ftIdent}
predstavlja identiteto. Torej:
$$\mathcal{S}_{OD}\circ\mathcal{S}_q
\circ\mathcal{S}_p\circ\mathcal{S}_r
\circ\mathcal{S}_q\circ\mathcal{S}_p=\mathcal{E}.$$

Premice $p$, $q$ in $r$ so iz istega šopa $\mathcal{X}_O$, zato
kompozitum $\mathcal{S}_q \circ\mathcal{S}_p\circ\mathcal{S}_r$
predstavlja osno zrcaljenje (izrek \ref{izoSop}) oz.
$$\mathcal{S}_q
\circ\mathcal{S}_p\circ\mathcal{S}_r=\mathcal{S}_l
=\mathcal{S}^{-1}_l=\left(\mathcal{S}_q
\circ\mathcal{S}_p\circ\mathcal{S}_r\right)^{-1}=\mathcal{S}_r
\circ\mathcal{S}_p\circ\mathcal{S}_q.$$
 Torej je:
 \begin{eqnarray*}
\mathcal{E}&=&\mathcal{S}_{OD}\circ\mathcal{S}_q
\circ\mathcal{S}_p\circ\mathcal{S}_r
\circ\mathcal{S}_q\circ\mathcal{S}_p =\\
&=&\mathcal{S}_{OD}\circ\mathcal{S}_r
\circ\mathcal{S}_p\circ\mathcal{S}_q
\circ\mathcal{S}_q\circ\mathcal{S}_p =
\mathcal{S}_{OD}\circ\mathcal{S}_r.
\end{eqnarray*}
 Iz
$\mathcal{E}=\mathcal{S}_{OD}\circ\mathcal{S}_r$ pa sledi
$\mathcal{S}_{OD}=\mathcal{S}_r$ oz. $OD=r$ in $D\in r$.


\item  \res{Premice $p$, $q$ in $r$ ležijo v isti ravnini. Dokaži
ekvivalenco $\mathcal{S}_r\circ\mathcal{S}_q\circ\mathcal{S}_p
 =\mathcal{S}_p\circ\mathcal{S}_q\circ \mathcal{S}_r$ natanko
 tedaj, ko premice $p$, $q$ in $r$ pripadajo istem šopu.}

Označimo
$\mathcal{I}=\mathcal{S}_r\circ\mathcal{S}_q\circ\mathcal{S}_p$.

($\Leftarrow$) Predpostavimo, da premice $p$, $q$ in $r$
pripadajo istemu šopu. Po izreku \ref{izoSop} izometrija
$\mathcal{I}$ predstavlja osmo zrcaljenje. Iz tega sledi
$\mathcal{I}=\mathcal{I}^{-1}$ oz.
$\mathcal{S}_r\circ\mathcal{S}_q\circ\mathcal{S}_p
 =\mathcal{S}_p\circ\mathcal{S}_q\circ \mathcal{S}_r$

($\Rightarrow$) Predpostavimo, da velja
 $\mathcal{S}_r\circ\mathcal{S}_q\circ\mathcal{S}_p
 =\mathcal{S}_p\circ\mathcal{S}_q\circ \mathcal{S}_r$. Iz danega
 pogoja sledi $\mathcal{I}=\mathcal{I}^{-1}$ oz.
$\mathcal{I}^2=\mathcal{E}$. Predpostavimo nasprotno - da
premice $p$, $q$ in $r$ niso iz istega šopa. Po izreku
\ref{izoZrcdrsprq} je
$\mathcal{I}=\mathcal{G}_{2\overrightarrow{AB}}$
($\overrightarrow{AB}\neq \overrightarrow{0}$). Toda v tem
primeru bi bilo
$\mathcal{E}=\mathcal{I}^2=\mathcal{G}^2_{2\overrightarrow{AB}}
=\mathcal{T}_{4\overrightarrow{AB}}$ (izrek
\ref{izoZrcdrsZrcdrs}), kar ni možno, ker je po predpostavki
$\overrightarrow{AB}\neq \overrightarrow{0}$. Zatorej  premice
$p$, $q$ in $r$ pripadajo istemu šopu.

\item \res{Naj bodo $O$, $P$ in $Q$ tri nekolinearne točke. Konstruiraj
kvadrat $ABCD$ (v ravnini $OPQ$) s središčem v  točki $O$, tako
da točki $P$ in $Q$ po vrsti
ležita na premicah $AB$ in $BC$.}

Uporabimo dejstvo, da $Q,\mathcal{R}_{O,90^0}(P)\in BC$.

\item \res{Naj bosta $\mathcal{R}_{S,\alpha}$ rotacija in $\mathcal{S}_p$
osno zrcaljenje v isti ravnini ter $S\in p$.
Dokaži, da kompozituma $\mathcal{R}_{S,\alpha}\circ\mathcal{S}_p$
in $\mathcal{S}_p\circ\mathcal{R}_{S,\alpha}$ predstavljata osno
zrcaljenje.}

Uporabimo izreka \ref{rotacKom2Zrc} in \ref{izoSop}.

\item \res{Dani sta točka $A$ in krožnica $k$ v isti ravnini. Načrtaj
kvadrat $ABCD$, tako da krajišči diagonale $BD$ ležita na krožnici $k$.}

Iz $B,D\in k$ in $\mathcal{R}_{A,90^0}(B)=D$ sledi $D\in k\cap
\mathcal{R}_{A,90^0}(k)$, kar omogoča konstrukcijo oglišča $D$.
Nato je $B=\mathcal{R}_{A,-90^0}(D)$ in
$C=\mathcal{R}_{D,90^0}(A)$.

\item \res{Naj bo $ABC$ poljubni trikotnik. Dokaži:
 $$\mathcal{R}_{C,2\measuredangle BCA}\circ
 \mathcal{R}_{B,2\measuredangle ABC}\circ
 \mathcal{R}_{A,2\measuredangle CAB}=\mathcal{E}.$$}

 Po izreku \ref{rotacKom2Zrc} je:
  \begin{eqnarray*}
 &&\mathcal{R}_{C,2\measuredangle BCA}\circ
 \mathcal{R}_{B,2\measuredangle ABC}\circ
 \mathcal{R}_{A,2\measuredangle CAB}=\\ &&=
 \mathcal{S}_{CA}\circ\mathcal{S}_{CB}\circ\mathcal{S}_{BC}
 \circ\mathcal{S}_{BA}\circ\mathcal{S}_{AB}\circ\mathcal{S}_{AC}=
 \mathcal{E}.
 \end{eqnarray*}

\item \res{Dokaži, da kompozitum osnega zrcaljenja $\mathcal{S}_p$
in središčnega zrcaljenja $\mathcal{S}_S$ ($S\in p$)
predstavlja osno zrcaljenje.}

  Naj bo $q$ pravokotnica premice $p$ skozi točko $S$. Po izreku
  \ref{izoSrZrcKom2Zrc} je:
   $$\mathcal{S}_S\circ\mathcal{S}_p=
   \mathcal{S}_q\circ\mathcal{S}_p\circ\mathcal{S}_p=\mathcal{S}_q.$$

\item \res{Naj bodo $O$, $P$ in $Q$ tri nekolinearne točke. Konstruiraj
kvadrat $ABCD$ (v ravnini $OPQ$) s središčem v točki $O$, tako
 da točki $P$ in $Q$
ležita po vrsti na premicah $AB$ in $CD$.}

Uporabimo dejstvo, da $Q,\mathcal{S}_O(P)\in CD$.

\item \res{Kaj predstavlja kompozitum translacije in središčnega zrcaljenja?}

Uporabimo izreka \ref{transl2sred} in \ref{izoKomp3SredZrc}.

\item \res{Dani so premica $p$ ter krožnici  $k$ in $l$, ki
ležijo v isti ravnini. Načrtaj premico, ki je vzporedna s
premico $p$, tako da na krožnicah $k$ in $l$ določa skladni tetivi.}

Označimo s $K$ in $L$ središči krožnic $k$ in $l$.  Naj bo $q$
iskana premica, ki je vzporedna s premico $p$, krožnici $k$ in
$l$ pa seka po vrsti v takšnih točkah $A$, $B$, $C$ in $D$, da
velja $AB\cong CD$. Naj bo še $\mathcal{B}(A,B,C,D)$.
Označimo s $K_q$ in $L_q$ pravokotni projekciji središč $K$ in
$L$ na premici $q$ ter $K_p$ in $L_p$ pravokotni projekciji
središč $K$ in $L$ na premici $p$. Iz skladnosti trikotnikov
$KAK_q$ in $KBK_q$ (izrek \textit{SSA} \ref{SSK}) sledi, da je
$K_q$ središče tetive $AB$. Analogno je $L_q$ središče tetive
$CD$. Iz $AB\cong CD$ potem sledi $K_qB\cong L_qD$ oz.
$\overrightarrow{K_qB} = \overrightarrow{L_qD}$. Torej:
$$\overrightarrow{BD}=\overrightarrow{BD}+\overrightarrow{0}=
\overrightarrow{BD}+\overrightarrow{K_qB}+\overrightarrow{DL_q}=
\overrightarrow{K_qL_q}=\overrightarrow{K_pL_p}.$$ Vektor
$\overrightarrow{v}=\overrightarrow{K_pL_p}$ lahko konstruiramo.
To pomeni, da  prejšnja analiza omogoča konstrukcijo, ker je
$\mathcal{T}_{\overrightarrow{v}}(B)=D$ oz. $D\in
\mathcal{T}_{\overrightarrow{v}}(k)\cap l$.

\item \res{Naj bo $c$ premica, ki seka vzporednici $a$ in $b$, ter $l$
 daljica. Načrtaj enakostranični trikotnik $ABC$, tako da velja
  $A\in a$, $B\in b$, $C\in c$ in $AB\cong l$.}

  Najprej konstruiramo poljubni enakostranični trikotnik
  $A_1B_1C_1$, ki izpolnjuje pogoje $A_1\in a$, $B_1\in b$ in
  $A_1B_1\cong l$, nato uporabimo ustrezno translacijo.

\item \res{Dokaži, da kompozitum rotacije in osnega zrcaljenja
neke ravnine
predstavlja zrcalni zdrs natanko tedaj, ko središče
rotacije ne leži na osi osnega zrcaljenja.}

Uporabimo izreke \ref{rotacKom2Zrc}, \ref{izoZrcdrsprq} in
\ref{izoSop}.

\item \res{Naj bo $ABC$ enakostranični trikotnik. Dokaži,
da kompozitum $\mathcal{S}_{AB}
    \circ\mathcal{S}_{CA}
    \circ\mathcal{S}_{BC}$
predstavlja zrcalni zdrs. Določi še vektor in os tega zdrsa.}

Označimo $\mathcal{I}=\mathcal{S}_{AB} \circ\mathcal{S}_{CA}
    \circ\mathcal{S}_{BC}$. Po izreku \ref{izoZrcdrsprq} je
    $\mathcal{I}$ zrcalni zdrs. Naj bodo $A_1$, $B_1$ in $C_1$ po
    vrsti središča daljic $BC$, $AC$ in $AB$ trikotnika $ABC$.
    Ker je le-ta pravilen je $\mathcal{I}(A_1C_1)=A_1C_1$, kar
    pomeni, da je premica $A_1C_1$ os tega zrcalnega zdrsa. Ni
    težko dokazati, da za točko $A'_1=\mathcal{I}(A_1)$ (obe
    ležita na osi $A_1C_1$) velja
    $\overrightarrow{A_1A'_1}=3\overrightarrow{A_1C_1}$, zato je
    $\mathcal{I}=\mathcal{G}_{3\overrightarrow{A_1C_1}}$.

\item  \res{Dani sta  točki $A$ in $B$ na istem bregu premice
$p$.
Načrtaj daljico  $XY$, ki leži na premici $p$ in je skladna
z dano daljico $l$, tako da bo vsota
$|AX|+|XY|+|YB|$ minimalna.}

Naj bo $A'=\mathcal{G}_{\overrightarrow{MN}}(A)$ (kjer $M,N\in
p$ in $MN\cong l$). Točko $Y$ dobimo kot presečišče premic $p$
in $X'Y$ (glej tudi zgled \ref{HeronProbl}).

\item  \res{Naj bo $ABC$ enakokraki pravokotni trikotnik s pravim
kotom pri oglišču $A$. Kaj predstavlja kompozitum
$\mathcal{G}_{\overrightarrow{AB}}\circ \mathcal{G}_{\overrightarrow{CA}}$?}

Naj bosta $p$ in $q$ simetrali stranic $CA$ in $AB$ trikotnika
$ABC$. Po izreku \ref{izoZrcDrsKompSrOsn} je:
 $$\mathcal{G}_{\overrightarrow{AB}}\circ
 \mathcal{G}_{\overrightarrow{CA}}=
 \mathcal{S}_q\circ\mathcal{S}_A\circ\mathcal{S}_A\circ\mathcal{S}_p=
 \mathcal{S}_q\circ\mathcal{S}_p.$$ Ker je $ABC$ enakokraki
 pravokotni trikotnik s pravim kotom pri oglišču $A$, sta
 premici $p$ in $q$ pravokotni in se sekata v središču $S$
 hipotenuze $BC$. Zatorej je
 $\mathcal{G}_{\overrightarrow{AB}}\circ
 \mathcal{G}_{\overrightarrow{CA}}=\mathcal{S}_q
 \circ\mathcal{S}_p=\mathcal{S}_S$.

\item \res{V isti ravnini so dane  premice  $a$, $b$ in $c$.
Načrtaj točki $A\in a$ in $B\in b$
tako, da bo $\mathcal{S}_c(A)=B$.}

Iz $A\in a$ sledi $\mathcal{S}_c(A)\in \mathcal{S}_c(a)$ oz.
$B\in \mathcal{S}_c(a)$. Ker je še $B\in b$, dobimo točko $B$ iz
pogoja $B\in \mathcal{S}_c(a)\cap b$. Nato je še
$A=\mathcal{S}_c(B)$.

\item  \res{Dani sta premici $p$ in $q$ ter točka $A$ v isti ravnini.
Načrtaj točki $B$ in $C$ tako,
da bosta premici $p$ in $q$ simetrali notranjih kotov pri
ogliščih $B$ in $C$ trikotnika $ABC$.}

Uporabimo dejstvo, da je premica $BC$ določena s točkama
$\mathcal{S}_p(A)$ in $\mathcal{S}_q(A)$.

\item  \res{Naj bodo $p$, $q$ in $r$ premice ter $K$ in $L$ točki v
isti ravnini. Načrtaj
premici $s$ in $s'$, ki gresta po vrsti skozi točki $K$ in $L$,
tako da velja $\mathcal{S}_r\circ\mathcal{S}_q\circ\mathcal{S}_p(s)=s'$.}

Označimo
$\mathcal{I}=\mathcal{S}_r\circ\mathcal{S}_q\circ\mathcal{S}_p$.
Premica $s'$ je določena s točkama $L$ in $\mathcal{I}(K)$,
premica $s$ pa s točkama $K$ in $\mathcal{I}^{-1}(K)$. Če je
$\mathcal{I}(K)=L$, ima naloga neskončno mnogo rešitev - premica $s$
je poljubna premica, ki poteka skozi točko $K$.


\item  \res{Naj bo $s$ simetrala enega od kotov, ki jih določata premici $p$
in $q$. Dokaži, da je $\mathcal{S}_s\circ\mathcal{S}_p =
\mathcal{S}_q\circ\mathcal{S}_s$.}

Dokažemo lahko celo več - da velja ekvivalenca (pod
predpostavko, da se premici $p$ in $q$ sekata). Če dano enakost
množimo z desne s $\mathcal{S}_s$, dobimo ekvivalentno enakost
$\mathcal{S}_s\circ\mathcal{S}_p\circ\mathcal{S}_s =
\mathcal{S}_q$. Po izreku \ref{izoTransmutacija} je
ekvivalentna z enakostjo $\mathcal{S}_{\mathcal{S}_s(p)} =
\mathcal{S}_q$ oz. $\mathcal{S}_s(p) = q$, kar je ekvivalentno s
tem, da je $s$ simetrala enega od kotov, ki ga določata premici
$p$ in $q$.

\item \label{nalIzo27}
\res{Naj bo $S$ središče trikotniku $ABC$ včrtane krožnice in $P$ točka,
 v kateri se ta krožnica dotika stranice $BC$. Dokaži: $$\mathcal{S}_{SC}
 \circ\mathcal{S}_{SA}\circ\mathcal{S}_{SB} =\mathcal{S}_{SP}.$$}

 Enakost je direktna posledica izreka \ref{izo1ftIndZrc}, ker
 je\\
 $\mathcal{S}_{SC}\circ\mathcal{S}_{SA}\circ\mathcal{S}_{SB}:S,P\mapsto
 S,P$.

\item  \res{Premice $p$, $q$ in $r$ neke ravnine potekajo skozi središče
$S$ krožnice $k$.
Načrtaj trikotnik $ABC$, ki je očrtan  tej krožnici, tako da
bodo premice $p$, $q$ in $r$ simetrale notranjih kotov pri
ogliščih $A$, $B$ in $C$ tega trikotnika.}

Uporabimo prejšnjo nalogo \ref{nalIzo27}.

\item  \res{Premice $p$, $q$, $r$, $s$ in $t$ neke ravnine se sekajo
v točki $O$, točka $M$ pa leži na premici $p$.
Načrtaj tak petkotnik, da bo $M$ središče ene njegove stranice,
premice $p$, $q$, $r$, $s$ in $t$ pa simetrale stranic.}

Točka $M$ leži na eni od simetral stranic. Brez škode za
splošnost naj bo $M\in p$. Naj bo $ABCDE$ takšen petkotnik, da
so premice $p$, $q$, $r$, $s$ in $t$  po vrsti simetrale
njegovih stranic $AB$, $BC$, $CD$, $DE$ in $EA$. Naj bo
$\mathcal{I}=\mathcal{S}_t\circ\mathcal{S}_s\circ\mathcal{S}_r
\circ\mathcal{S}_q\circ\mathcal{S}_p$. Ker
$\mathcal{I}:O,A\mapsto O,A$, je po izreku \ref{izo1ftIndZrc}
$\mathcal{I}=\mathcal{S}_{OA}$. Premico $OA$ dobimo kot
simetralo daljice $XX'$, kjer je $X$ poljubna točka in
$X'=\mathcal{I}(X)$, nato pa oglišče $A$ kot presečišče premice
$OA$ in pravokotnice premice $p$ skozi točko $M$.

\item  \res{Točka $P$ leži v ravnini trikotnika $ABC$. Dokaži, da
premice, ki so simetrične s
premicami $AP$, $BP$ in $CP$ glede na simetrale notranjih kotov
 ob ogliščih $A$, $B$ in $C$ tega trikotnika, pripadajo istem šopu.}

Označimo z $s_{\alpha}$, $s_{\alpha}$ in $s_{\alpha}$ simetrale notranjih
kotov ob ogliščih $A$, $B$ in $C$ trikotnika $ABC$
ter $a=\mathcal{S}_{s_{\alpha}}(AP)$, $b=\mathcal{S}_{s_{\beta}}(BP)$ in $c=\mathcal{S}_{s_{\gamma}}(CP)$.
Dokažimo, da premice $a$, $b$ in $c$ pripadajo istemu šopu. Ker $\mathcal{S}_{s_{\alpha}}:AC, p\rightarrow AB,
 a$ je $\measuredangle CAP=\measuredangle a,AP$. Torej $\mathcal{R}_{A,2\measuredangle CAP}
 =\mathcal{R}_{A,2\measuredangle a,AP}$ oz. $\mathcal{S}_{AC}\circ \mathcal{S}_{AP}
 =\mathcal{S}_a\circ \mathcal{S}_{AB}$. Iz tega sledi  $\mathcal{S}_a=\mathcal{S}_{AC}
 \circ\mathcal{S}_{AP}\circ \mathcal{S}_{AB}$. Analogno je: $\mathcal{S}_b=\mathcal{S}_{BA}
 \circ\mathcal{S}_{BP}\circ \mathcal{S}_{BC}$ in $\mathcal{S}_c=\mathcal{S}_{CB}\circ\mathcal{S}_{CP}
 \circ \mathcal{S}_{CA}$. Zatorej je (izreka \ref{izoSop} in \ref{izoTransmutacija}):
 \begin{eqnarray*}
 \mathcal{I}&=&\mathcal{S}_a\circ\mathcal{S}_b\circ\mathcal{S}_c=\\
 &=&
 \mathcal{S}_{AC}\circ\mathcal{S}_{AP}\circ \mathcal{S}_{AB}\circ
 \mathcal{S}_{BA}\circ\mathcal{S}_{BP}\circ \mathcal{S}_{BC}\circ
 \mathcal{S}_{CB}\circ\mathcal{S}_{CP}\circ \mathcal{S}_{CA}=\\
 &=&
 \mathcal{S}_{AC}\circ\mathcal{S}_{AP}\circ \mathcal{S}_{BP}\circ\mathcal{S}_{CP}\circ \mathcal{S}_{CA}=\\
 &=&
 \mathcal{S}_{AC}\circ\mathcal{S}_{AX}\circ \mathcal{S}_{CA}=\\
 &=&
 \mathcal{S}_{\mathcal{S}_{AC}(AX)}.
 \end{eqnarray*}
Po izreku \ref{izoSop} premice  $a$, $b$ in $c$ pripadajo istem šopu.

\item  \res{Izračunaj kot, ki ga določata premici $p$ in $q$, če velja:
$\mathcal{S}_p\circ\mathcal{S}_q\circ\mathcal{S}_p =
\mathcal{S}_q\circ\mathcal{S}_p\circ\mathcal{S}_q$.}

 Če dano enakost množimo z leve po vrsti z $\mathcal{S}_q$,
$\mathcal{S}_p$ in $\mathcal{S}_q$, dobimo ekvivalentno enakost:
$\mathcal{S}_q\circ\mathcal{S}_p\circ\mathcal{S}_q\circ
\mathcal{S}_p\circ\mathcal{S}_q\circ\mathcal{S}_p =
\mathcal{E}$ oz.
$\left(\mathcal{S}_q\circ\mathcal{S}_p\right)^3 = \mathcal{E}$.

Če je $p=q$, je zadnja enakost jasno izpolnjena.

V primeru $p\parallel q$ je $\mathcal{S}_q\circ\mathcal{S}_p$
translacija, zato je vedno
$\left(\mathcal{S}_q\circ\mathcal{S}_p\right)^3
=\mathcal{T}^3_{\overrightarrow{v}}=\mathcal{T}_{3\overrightarrow{v}}\neq
\mathcal{E}$.

Če se premici $p$ in $q$ sekata, predstavlja kompozitum
$\mathcal{S}_q\circ\mathcal{S}_p$ rotacijo
$\mathcal{R}_{S,\omega}$ ($p\cap q=\{S\}$ in
$\omega=2\measuredangle p,q$). V tem primeru je torej
$\left(\mathcal{S}_q\circ\mathcal{S}_p\right)^3
=\mathcal{R}^3_{S,\omega}=\mathcal{R}_{S,3\omega}= \mathcal{E}$.
To pomeni $3\omega=360^0$ oz. $\omega=120^0$, zato premici $p$
in $q$ določata kot $60^0$.

\item  \res{Naj bosta $\mathcal{R}_{A,\alpha}$ in $\mathcal{R}_{B,\beta}$
rotaciji v isti ravnini. Določi vse točke $X$ v tej ravnini, za
katere velja
$\mathcal{R}_{A,\alpha}(X)=\mathcal{R}_{B,\beta}(X)$.}

Pogoj $\mathcal{R}_{A,\alpha}(X)=\mathcal{R}_{B,\beta}(X)$ je
ekvivalenten s pogojem
$\mathcal{R}_{A,\alpha}\circ\mathcal{R}^{-1}_{B,\beta}(X)=X$
oz. $\mathcal{R}_{A,\alpha}\circ\mathcal{R}_{B,-\beta}(X)=X$.
Potrebno je torej določiti fiksne točke izometrije
$\mathcal{I}=\mathcal{R}_{A,\alpha}\circ\mathcal{R}_{B,-\beta}$.
Uporabimo izrek \ref{rotacKomp2rotac}. Obravnavali bomo več
primerov.

 \textit{1}) Če je $A=B$ in $\alpha-\beta=k\cdot360^0$ (za nek
 $k\in \mathbb{Z}$), je $\mathcal{I}=\mathcal{E}$, zato pogoj
 jasno velja za vsako točko te ravnine.

 \textit{2}) Če je $A=B$ in $\alpha-\beta\neq k\cdot360^0$ (za
 vsak $k\in \mathbb{Z}$), je
 $\mathcal{I}=\mathcal{R}_{A,\alpha-\beta}$ in je pogoj
 izpolnjen le za točko $X=A$.

 \textit{3}) Če je $A\neq B$ in $\alpha-\beta=k\cdot360^0$ (za
 nek $k\in \mathbb{Z}$), predstavlja $\mathcal{I}$ translacijo,
 zato pogoj ne ustreza nobeni točki te ravnine.

 \textit{4}) Če je $A\neq B$ in $\alpha-\beta\neq k\cdot360^0$
 (za vsak $k\in \mathbb{Z}$), je $\mathcal{R}_{C,\alpha-\beta}$
 (kjer  velja $\measuredangle CAB=\frac{\alpha}{2}$ in
 $\measuredangle BAC=\frac{\beta}{2}$), kar pomeni, da pogoj
 velja le za točko $C$.

\item  \res{Premici $p$ in $q$ se pod kotom $60^0$ sekata v središču $O$ enakostraničnega
 trikotnika $ABC$. Dokaži, da
 sta odseka, ki jih na  premicah določata stranici
 trikotnika $ABC$, skladni daljici.}

Uporabimo rotacijo $\mathcal{R}_{O,120^0}$.

\item   \res{Točka $S$ naj bo središče pravilnega petkotnika $ABCDE$.
Dokaži, da velja:
 $$\overrightarrow{SA} + \overrightarrow{SB} + \overrightarrow{SC}
  + \overrightarrow{SD} + \overrightarrow{SE} = \overrightarrow{0}.$$}

Naj bo $\overrightarrow{SA} + \overrightarrow{SB} +
 \overrightarrow{SC} + \overrightarrow{SD} + \overrightarrow{SE}
  = \overrightarrow{SX}$. Ker je $ABCDE$ pravilni petkotnik,
  velja $\mathcal{R}_{S,72^0}:A,B,C,D,E,S,X\mapsto
  B,C,D,E,A,S,X'$. Zato je $ \overrightarrow{SB} +
 \overrightarrow{SC} + \overrightarrow{SD} + \overrightarrow{SE}
 +\overrightarrow{SA} = \overrightarrow{SX'}$. To pomeni, da je
 $\overrightarrow{SX}= \overrightarrow{SX'}$ oz.
 $X=X'=\mathcal{R}_{S,72^0}(X)$. Ker je $S$ edina fiksna točka
 rotacije $\mathcal{R}_{S,72^0}$, je $X=S$. Zatorej je:
 $\overrightarrow{SA} + \overrightarrow{SB} +
 \overrightarrow{SC} + \overrightarrow{SD} + \overrightarrow{SE}
  = \overrightarrow{SS}=\overrightarrow{0}$.

\item   \res{Dokaži, da se diagonale pravilnega petkotnika
sekajo v točkah, ki so tudi
oglišča pravilnega petkotnika.}

Uporabimo rotacijo s središčem v središču pravilnega petkotnika za
kot $72^0$.

\item   \res{Naj bosta $ABP$ in $BCQ$  pravilna trikotnika z
isto orientacijo in $\mathcal{B}(A,B,C)$. Točki $K$ in $L$ sta
središči daljic $AQ$ in $PC$. Dokaži, da je tudi $BLK$ pravilni
trikotnik.}

Z rotacijo $\mathcal{R}_{B,-60^0}$ se točki $A$ in $Q$
preslikata v točki $P$ in $C$. Torej se daljica $AQ$ in njeno
središče $K$ preslikata v daljico $PC$ in njeno središče $L$. Iz
$\mathcal{R}_{B,-60^0}(K)=L$ pa sledi, da je $BLK$ pravilni
trikotnik.

\item   \res{Dane so tri koncentrične krožnice in premica v isti ravnini.
Načrtaj pravilni trikotnik tako, da njegova oglišča
po vrsti ležijo na  teh krožnicah, ena stranica pa bo vzporedna
dani premici.}

Najprej načrtamo poljubni tak trikotnik brez pogoja, da je ena
njegova stranica vzporedna dani premici. To lahko naredimo tako,
da izberemo poljubno oglišče $A$ na eni od krožnic in uporabimo
rotacijo $\mathcal{R}_{A,60^0}$. Nato pa uporabimo ustrezno
rotacijo s središčem v središču koncentričnih krožnic, ki
načrtani trikotnik preslika v trikotnik, pri katerem je ena
njegova stranica vzporedna dani premici.

\item  \res{Točka $P$ je notranja točka pravilnega trikotnika
$ABC$, tako da velja
$\angle APB=113^0$ in $\angle BPC=123^0$. Izračunaj velikosti
kotov trikotnika, čigar stranice so skladne z daljicami
$PA$, $PB$ in $PC$.}

Najprej je jasno $\angle APC=360^0-113^0-123^0=124^0$. Označimo
$P'=\mathcal{R}_{A,60^0}(P)$. $APP'$ je pravilni trikotnik, zato
je $PA\cong P'P$. Ker je še $C=\mathcal{R}_{A,60^0}(B)$, velja
$\triangle ABP\cong\triangle ACP'$, zato je $PB\cong P'C$ in $\angle
APB\cong\angle AP'C$. To pomeni, da so stranice $P'P$, $P'C$ in
$PC$ trikotnika $PCP'$ po vrsti skladne z daljicami $PA$, $PB$
in $PC$. Pri tem je $\angle P'PC=124^0-60^0=64^0$, $\angle
PP'C=113^0-60^0=53^0$ in $\angle PP'C=180^0-64^0-53^0=63^0$.

\item   \res{Dane so nekolinearne točke $P$, $Q$ in $R$.
Načrtaj trikotnik $ABC$, tako
da bodo $P$, $Q$ in $R$  središča kvadratov, ki so konstruirani
nad stranicami $BC$, $CA$ in $AB$ tega trikotnika.}

Poiščemo fiksno točko kompozituma
$\mathcal{R}_{Q,90^0}\circ\mathcal{R}_{R,90^0}\circ\mathcal{R}_{P,90^0}$.

\item   \res{Naj bosta  $A$ in $B$  točki ter $p$ premica v isti
ravnini. Dokaži, da je
kompozitum $\mathcal{S}_B\circ\mathcal{S}_p\circ\mathcal{S}_A$
 osno zrcaljenje natanko tedaj, ko je $AB\perp p$.}

Središčni zrcaljenji $\mathcal{S}_A$ in $\mathcal{S}_B$
predstavimo kot kompozitume
$\mathcal{S}_A=\mathcal{S}_a\circ\mathcal{S}_{a_1}$ in
$\mathcal{S}_B=\mathcal{S}_{b_1}\circ\mathcal{S}_b$, kjer sta
$a$ in $b$ pravokotni na premico $p$.

\item   \res{Naj bodo $p$, $q$ in $r$ tangente trikotniku $ABC$ včrtane krožnice,
ki so vzporedne z njegovimi stranicami
$BC$, $AC$ in $AB$. Dokaži,
da premice $p$, $q$, $r$, $BC$, $AC$ in $AB$ določajo tak
 šestkotnik, v katerem so pari nasprotnih stranic skladne daljice.}

Uporabimo središčno zrcaljenje $\mathcal{S}_S$, kjer je $S$
središče včrtane krožnice trikotnika $ABC$.

\item   \res{Načrtaj trikotnik s podatki:  $\alpha$, $t_b$, $t_c$.}

Najprej lahko načrtamo težiščnico $BB_1\cong t_b$ in težišče
$T$. Ker oglišče $A$ leži na loku $l$ nad tetivo $BB_1$ in
obodnim kotom $\alpha$, oglišče $B$ na krožnici
$k(T,\frac{2}{3}t_c)$, pri tem pa velja
$C=\mathcal{S}_{B_1}(A)$, dobimo oglišče $C$ iz pogoja $C\in
k(T,\frac{2}{3}t_c)\cap\mathcal{S}_{l}(A)$.

\item \res{Naj bosta $ALKB$ in $ACPQ$ kvadrata, ki sta načrtana zunaj trikotnika $ABC$
nad stranicama $AB$ in
$AC$, ter $X$ središče stranice $BC$. Dokaži, da
je $AX\perp LQ$ in
$|AX|=\frac{1}{2}|QL|$.}

Naj bo $Q'=\mathcal{S}_A(C)$. V tem primeru velja
$R_{A,90^0}:Q,L\mapsto Q',B$. Torej $QL\cong Q'B$ in $QL\perp
Q'B$ (izrek \ref{rotacPremPremKot}). Ker je $AX$ srednjica
trikotnika $BCQ$ za osnovnico $BQ'$, je
$\overrightarrow{AX}=\frac{1}{2}\overrightarrow{BQ'}$, zatorej
je $AX\perp LQ$ in $|AX|=\frac{1}{2}|QL|$.

\item \res{Naj bo $O$ središče pravilnega trikotnika $ABC$ ter $D$ in
$E$ točki na stranicah $CA$ in $CB$, tako
 da velja $CD\cong CE$. Točka $F$ je četrto oglišče paralelograma
 $BODF$. Dokaži,
da je trikotnik $OEF$ pravilen.}

Naj bo $\mathcal{I}=\mathcal{T}_{\overrightarrow{OB}}\circ
\mathcal{R}_{C,-60^0}$. Po izreku \ref{izoKompTranslRot} je
$\mathcal{I}$ rotacija za isti kot $-60^0$ s središčem v neki
točki $\widehat{O}$. Torej
$\mathcal{T}_{\overrightarrow{OB}}\circ
\mathcal{R}_{C,-60^0}=\mathcal{R}_{\widehat{O},-60^0}$. Pri tem
je
$\mathcal{R}_{\widehat{O},-60^0}(E)=\mathcal{T}_{\overrightarrow{OB}}\circ
\mathcal{R}_{C,-60^0}(E)=F$, zato je $\widehat{O}EF$ pravilni
trikotnik. Potrebno je še dokazati, da je $\widehat{O}=O$ oz.
da je $O$ fiksna točka rotacije
$\mathcal{R}_{\widehat{O},-60^0}$. Če je
$O'=\mathcal{R}_{C,-60^0}(O)$, je $O'$ središče simetričnega
enakostraničnega trikotnika $AB'C$ (kjer je
$C'=\mathcal{S}_{AC}(B)$). Zato je
$\overrightarrow{O'O}=\overrightarrow{OB}$ oz.
$\mathcal{T}_{\overrightarrow{OB}}(O')=O$. Iz tega sledi
$\mathcal{R}_{\widehat{O},-60^0}(O)=O$ oz. $\widehat{O}=O$, kar
pomeni, da je $OEF$
 pravilni trikotnik.

\item \res{Naj bo $L$ točka, v kateri se trikotniku
$ABC$ včrtana krožnica dotika njegove stranice $BC$.
Dokaži: $$\mathcal{R}_{C,\measuredangle ACB}\circ\mathcal{R}_{A,\measuredangle BAC}
\circ\mathcal{R}_{B,\measuredangle CBA} =\mathcal{S}_L.$$}

Označimo z $M$ in $N$ točki, v katerih se včrtana krožnica
trikotnika $ABC$ dotika njegovih stranic $AC$ in $AB$. Po izreku
\ref{rotacKomp2rotac} je $\mathcal{R}_{C,\measuredangle ACB}
\circ\mathcal{R}_{A,\measuredangle BAC}\circ
\mathcal{R}_{B,\measuredangle CBA}=\mathcal{S}_{\widehat{L}}$.
Ker je:
\begin{eqnarray*}
 \mathcal{S}_{\widehat{L}}(L)&=&
 \mathcal{R}_{C,\measuredangle
ACB} \circ\mathcal{R}_{A,\measuredangle BAC}\circ
\mathcal{R}_{B,\measuredangle CBA}(L)=\\
&=&
 \mathcal{R}_{C,\measuredangle
ACB} \circ\mathcal{R}_{A,\measuredangle BAC}(N)=
 \mathcal{R}_{C,\measuredangle ACB}(M)= L,
 \end{eqnarray*}
  je $\widehat{L}=L$ oz. $\mathcal{R}_{C,\measuredangle ACB}
\circ\mathcal{R}_{A,\measuredangle BAC}\circ
\mathcal{R}_{B,\measuredangle CBA}=\mathcal{S}_L$.

\item \res{Točke $P$ in $Q$ ter $M$ in $N$ so središča po dveh kvadratov,
ki so načrtani zunaj nad
nasprotnimi stranicami poljubnega štirikotnika. Dokaži, da je
$PQ\perp MN$ in $PQ\cong MN$.}

 Naj bo $ABCD$ dani štirikotnik, $P$ in $Q$ središči kvadratov,
 ki sta konstruirani nad stranicama $AB$ in $CD$ ter $M$ in $N$
 središči kvadratov, ki sta konstruirani nad stranicama $BC$ in
 $AD$. Kompozitum $\mathcal{I}=\mathcal{R}_{N,90^0}\circ
\mathcal{R}_{P,90^0}$ po izreku  \ref{rotacKomp2rotac}
predstavlja središčno zrcaljenje. Ker je pri tem
$\mathcal{I}(B)=D$, je središče tega zrcaljenja pravzaprav
središče diagonale $BD$; označimo ga s $S$. Torej
$\mathcal{R}_{N,90^0}\circ \mathcal{R}_{P,90^0}=\mathcal{S}_S$.
Po istem izreku \ref{rotacKomp2rotac} za točko $S$ velja $\angle
NPS=\frac{1}{2}90^0=45^0$ in $\angle PNS=\frac{1}{2}90^0=45^0$.
To pomeni, da je $PNS$ enakokraki pravokotni trikotnik z
osnovnico $NP$ in pravim kotom ob oglišču $S$. Analogno je tudi
$MSQ$ enakokraki pravokotni trikotnik z osnovnico $MQ$ in s pravim
kotom ob oglišču $S$. Iz teh dveh dejstev sledi $R_{S,90^0}:
M,N\mapsto Q,P$, zato je $MN\cong QP$ in $MN\perp QP$ (izrek
\ref{rotacPremPremKot}).


\item  \res{Naj bosta $APB$ in $ACQ$ pravilna trikotnika, ki sta zunaj trikotnika $ABC$
načrtana  nad
stranicama $AB$ in $AC$. Točka $S$ je središče
stranice $BC$ in $O$ središče trikotnika $ACQ$. Dokaži, da je
$|OP|=2|OS|$.}

Naj bo $\mathcal{I}=\mathcal{R}_{O,120^0}\circ
\mathcal{R}_{P,60^0}$. Po izreku \ref{rotacKomp2rotac} je:
 $$\mathcal{I}=\mathcal{R}_{O,120^0}\circ
\mathcal{R}_{P,60^0}=\mathcal{R}_{\widehat{S},180^0}
=\mathcal{S}_{\widehat{S}},$$ kjer je $\widehat{S}$ oglišče
 trikotnika $OP\widehat{S}$ in $\angle
 \widehat{S}PO=\frac{1}{2}60^0=30^0$ ter $\angle
 PO\widehat{S}=\frac{1}{2}120^0=60^0$. Ker je
 $\mathcal{S}_{\widehat{S}}(S)=\mathcal{I}(S)=S$ oz.
 $\widehat{S}=S$.

 Če označimo $O'=\mathcal{S}_S$, je $POO'$ pravilni trikotnik,
 zato je $|OP|=|OO'|=2|OS|$.

\item \res{Dokaži, da osno zrcaljenje in translacija neke ravnine
komutirata natanko tedaj, ko je os tega zrcaljenja vzporedna z
vektorjem translacije.}

Če uporabimo izrek \ref{izoTransmutacija}, dobimo:
 \begin{eqnarray*}
\mathcal{T}_{\overrightarrow{v}}\circ \mathcal{S}_p=
\mathcal{S}_p\circ\mathcal{T}_{\overrightarrow{v}}
&\Leftrightarrow&
\mathcal{T}_{\overrightarrow{v}}\circ \mathcal{T}_p\circ\mathcal{T}^{-1}_{\overrightarrow{v}}=
 \mathcal{S}_p\\
&\Leftrightarrow&
\mathcal{S}_{\mathcal{T}_{\overrightarrow{v}}(p)}=
 \mathcal{S}_p
\Leftrightarrow
\mathcal{T}_{\overrightarrow{v}}(p)=
 p
\Leftrightarrow
\overrightarrow{v}\parallel
 p.
\end{eqnarray*}

\item  \res{V isti ravnini so dani premica $p$, krožnici $k$ in $l$ ter daljica $d$.
Načrtaj romb $ABCD$ s stranico, ki je skladna daljici $d$, stranica $AB$ leži na
premici $p$, oglišči $C$ in $D$ pa po vrsti ležita na krožnicah $k$ in $l$.}

Uporabi translacijo za vektor $\overrightarrow{v}$, ki je
vzporeden s premico $p$ in $|\overrightarrow{v}|=|d|$. V tem
primeru je $D\in \mathcal{T}_{\overrightarrow{v}}(k)\cap l$.

\item  \res{Naj bo $p$ premica, $A$ in $B$ pa točki, ki ležita na istem bregu
premice $p$, ter $d$ daljica v isti ravnini.
Načrtaj točki $X$ in $Y$ na premici $p$ tako, da bo $AX\cong BY$ in $XY\cong d$.}

Naj bo  $\overrightarrow{v}$ vektor, ki je vzporeden s premico
$p$ in $|\overrightarrow{v}|=|d|$. Naj bo
$A'=\mathcal{T}_{\overrightarrow{v}}(A)$, $Y$ presečišče
simetrale daljice $A'B$ s premico $p$ in
$X=\mathcal{T}^{-1}_{\overrightarrow{v}}(Y)$.

Točka $Y$ leži na simetrali daljice $A'B$, zato je $A'Y\cong YB$. Ker
$\mathcal{T}_{\overrightarrow{v}}:A,X\mapsto A',Y$ je štirikotnik $AYYA'$
paralelogram, zatorej je $AX\cong A'Y\cong BY$. Iz $\mathcal{T}_{\overrightarrow{v}}(X)=Y$
pa sledi $\overrightarrow{XY}=\overrightarrow{v}$ oz. $|XY|=|\overrightarrow{v}|=|d|$.

\item  \res{Naj bo $H$ višinska točka trikotnika $ABC$ in $R$ polmer očrtane krožnice tega
trikotnika. Dokaži, da je $|AB|^2+|CH|^2=4R^2$.}

Naj bo $O$ središče očrtane krožnice $k(O,R)$ trikotnika $ABC$
in $A'=\mathcal{S}_O(A)$. Daljica $AA'$ je premer krožnice $k$,
zato je $\angle ACA'=90^0$ oz. $A'C\perp AC$. Ker je še
$BH\perp AC$, je $A'C\parallel BH$. Analogno je tudi
$A'B\parallel CH$, kar pomeni, da je štirikotnik $BA'CH$
paralelogram. Torej velja $CH\cong A'B$. Ker je $A'BA$
pravokotni trikotnik, je po Pitagorovem izreku
\ref{PitagorovIzrek}:
$|AB|^2+|CH|^2=|AB|^2+|A'B|^2=|AA'|^2=4R^2$.

\item  \res{Naj bo $EAB$ trikotnik, ki je načrtan nad stranico $AB$ kvadrata
$ABCD$. Naj bo tudi $M=pr_{\perp AE}(C)$ in $N=pr_{\perp BE}(D)$ ter točka $P$
presečišče premic $CM$ in $DN$. Dokaži, da je $PE\perp AB$.}

Uporabimo translacijo za vektor $\overrightarrow{CB}$.

\item  \res{Načrtaj enakostranični trikotnik $ABC$ tako, da njegova oglišča po vrsti
ležijo na treh vzporednicah $a$, $b$ in $c$ v isti ravnini,
središče tega trikotnika pa leži na premici $s$, ki seka
premice $a$, $b$ in $c$.}

    Najprej narišmo poljubni pravilni trikotnik $A_1B_1C_1$, kjer $A_1\in a$, $B_1\in b$ in $C_1\in c$,
    nato pa uporabimo translacijo za vektor $\overrightarrow{S_1S}$, kjer je $S_1$ središče trikotnika
    $A_1B_1C_1$, točka $S$ presečišče premice $s$ z vzporednico premic $a$, $b$ in $c$ skozi točko $S_1$.

\item \res{Če ima petkotnik vsaj dve osi simetrije,  je pravilen. Dokaži.}

    Naj bo $\mathfrak{G}(\mathcal{V}_5)$ grupa simetrij
    našega petkotnika $\mathcal{V}_5$ ter $p$ in $q$ njegovi
    osi simetrije. Jasno je, da je ta grupa končna, zato po
    izreku \ref{GrupaLeonardo} predstavlja bodisi ciklično grupo
    $\mathfrak{C}_n$ bodisi diedrsko grupo $\mathfrak{D}_n$. Ker
    vsebuje osne simetrije, je
    $\mathfrak{G}(\mathcal{V}_5)=\mathfrak{D}_n$. Pri tem je
    jasno $n\leq 5$. Ker $\mathcal{S}_p, \mathcal{S}_q \in
    \mathfrak{G}(\mathcal{V}_5)=\mathfrak{D}_n$, se osi $p$ in
    $q$ sekata v neki točki $S$, kompozitum $\mathcal{S}_q\circ
    \mathcal{S}_p$ predstavlja rotacijo $\mathcal{R}_{S,
    \alpha}$. Ker grupa
    $\mathfrak{G}(\mathcal{V}_5)=\mathfrak{D}_n$ vsebuje vsaj
    dve osni simetriji, je $n\geq 2$. Torej $n\in \{2,3,4,5\}$.
    Pri tem je osnovna rotacija te grupe (za najmanjši kot)
    $\mathcal{R}_{S, \frac{360^0}{n}}$. Ker ima petkotnik pet
    oglišč, je število $n$ delitelj števila 5, kar pomeni
    $n=5$. Torej $\mathfrak{G}(\mathcal{V}_5)=\mathfrak{D}_5$,
    torej je $\mathcal{V}_5$ pravilni petkotnik.

\item \res{Naj bodo $A$, $B$ in $C$ tri kolinearne točke. Kaj predstavlja
kompozitum $\mathcal{G}_{\overrightarrow{BC}}\circ \mathcal{S}_A$?}

Označimo s $p$ premico, ki poteka skozi točke $A$, $B$ in $C$.
Naj bosta $b$ in $a$ pravokotnici premice $p$ v točkah $A$ in $B$ ter $s$ simetrala daljice $BC$. Tedaj je:
 $$\mathcal{G}_{\overrightarrow{BC}}\circ \mathcal{S}_A=
 \mathcal{S}_s\circ \mathcal{S}_b\circ \mathcal{S}_p
 \circ \mathcal{S}_p\circ \mathcal{S}_a=
 \mathcal{S}_s\circ \mathcal{S}_b\circ \mathcal{S}_a.$$
 Ker so premice $a$, $b$ in $s$ pravokotne na premico $p$, so iz šopa vzporednic,
 zato je po izreku \ref{izoSop} kompozitum $\mathcal{S}_s
 \circ \mathcal{S}_b\circ \mathcal{S}_a$ (oz. $\mathcal{G}_{\overrightarrow{BC}}
 \circ \mathcal{S}_A$) osno zrcaljenje.

\item \res{Naj bodo $p$, $q$ in $r$ premice, ki niso iz istega šopa, ter $A$ točka v isti
ravnini. Načrtaj premico $s$, ki poteka skozi točko $A$, tako da velja
$\mathcal{S}_r\circ \mathcal{S}_q\circ \mathcal{S}_p(s)=s'$ in $s\parallel s'$.}

Po izreku \ref{izoZrcdrsprq} je kompozitum $\mathcal{S}_r\circ \mathcal{S}_q\circ
\mathcal{S}_p(s)=s'$ zrcalni zdrs - označimo ga z $\mathcal{G}_{2\overrightarrow{PQ}}$.
Premica $s$ je vzporedna ali pravokotna na os zrcalnega zdrsa. Le-ta je določena s
središčema daljic $XX'$ in $YY'$, kjer sta $X$ in $Y$ poljubni točki ter $\mathcal{S}_r
\circ \mathcal{S}_q\circ \mathcal{S}_p: X, Y\mapsto X', Y'$.

%nove naloge
%___________________________________

\item \res{Naj bosta $Z$ in $K$ notranji točki pravokotnika $ABCD$.
Načrtaj točke $A_1$, $B_1$, $C_1$ in $D_1$, ki po vrsti ležijo na
stranicah $AB$, $BC$, $CD$ in $DA$ tega pravokotnika, tako da velja
$\angle ZA_1A\cong\angle B_1A_1B$, $\angle A_1B_1B\cong\angle C_1B_1C$,
$\angle B_1C_1C\cong\angle D_1C_1D$ in  $\angle C_1D_1D\cong\angle KD_1A$.}

    Najprej načrtamo točki $Z'=S_{CB}\circ S_{AB}(Z)=S_B(Z)$ in $K'=S_{CD}\circ S_{AD}(Z)=S_D(K)$.
     Nato dokažemo in uporabimo dejstvo, da so točke $Z'$, $B_1$, $C_1$ in $K'$ kolinearne.


\item \res{Točka $A$ leži na premici $a$, točka $B$ pa na premici $b$.
Določi rotacijo, ki preslika premico $a$ v premico $b$ in točko $A$ v točko $B$.}

    Če sta premici $a$ in $b$ vzporedni, je iskana rotacija
    središčno zrcaljenje s središčem, ki je središče daljice
    $AB$. Če pa se premici $a$ in $b$ sekata v točki
    $O$, je središče rotacije presečišče simetrale daljice $AB$
    in simetrale kota $aOb$, kot rotacije pa enak kotu $aOb$.

\item \res{V središču kvadrata se sekata dve pravokotnici.
Dokaži, da ti pravokotnici sekata stranice kvadrata v točkah, ki so
oglišča novega kvadrata.}

    Uporabimo rotacijo $\mathcal{R}_{S,90^0}$, kjer je $S$ središče kvadrata.

\item \res{Dana je krožnica $k$ ter premice $a$, $b$, $c$, $d$ in $e$, ki ležijo
v isti ravnini. Krožnici $k$ včrtaj petkotnik s stranicami, ki
so po vrsti vzporedne s premicami  $a$, $b$, $c$, $d$ in $e$.}


    Najprej načrtamo simetrale $p$, $q$, $r$, $s$ in $t$ stranic
    iskanega petkotnika $ABCDE$, ki so pravzaprav pravokotnice na
    premice  $a$, $b$, $c$, $d$ in $e$  iz središča $O$
    krožnice $k$. Kompozitum $\mathcal{I}= \mathcal{S}_t\circ
    \mathcal{S}_s\circ \mathcal{S}_r\circ \mathcal{S}_q\circ
    \mathcal{S}_p$ je indirektna izometrija s fiksnima  točkama
    $O$ in $A$, zato je po izreku \ref{izo1ftIndZrc}
    $\mathcal{I}=\mathcal{S}_{OA}$. Os $OA$ lahko načrtamo kot
    simetralo daljice $XX'$, kjer je $X$ poljubna točka in
    $X'=\mathcal{I}(X)$. To omogoča konstrukcijo oglišča $A$,
    nato pa ostalih oglišč petkotnika $ABCDE$.

\item \res{Točka $P$ leži v notranjosti kota $aOb$. Načrtaj premico $p$ skozi točko $P$,
 ki s krakoma $a$ in $b$ določa trikotnik z najmanjšo ploščino.}

    Dokažimo, da ima trikotnik najmanjšo ploščino,
    če je daljica $OP$ njegova težiščnica. Označimo ta
    trikotnik z $OAB$ ($A\in a$ in $B\in b$). Naj bosta $X$ in
    $Y$ presečišča poljubne premice (različne od $AB$) skozi
    točko $P$. Dokažimo, da je ploščina trikotnika $XOY$ večja
    od ploščine trikotnika $AOB$. Brez škode za splošnost naj bo
    $\mathcal{B}(O,X,A)$ oz. $\mathcal{B}(O,B,Y)$. V tem primeru
    točki $A$ in $B$ načrtamo na naslednji način. Naj bo
    $a'=\mathcal{S}_P(a)$, $\{B\}=b\cap a'$ in
    $A=\mathcal{S}_P(B)$. Označimo z $X'=\mathcal{S}_P$. Jasno
    je $X'\in a'$. Ker $\mathcal{S}_P:X,A,P\mapsto X',B,P$, sta
    trikotnika $XAP$ in $X'BP$ skladna, zato imata enako
    ploščino. Če z $p_{V_1V_2\cdots V_n}$ označimo ploščino
    poljubnega večkotnika $V_1V_2\cdots V_n$, je:
    \begin{eqnarray*}
    p_{AOB}&=&p_{AXPB}+p_{XAP}=\\
    &=&p_{AXPB}+p_{X'BP}>p_{AXPB}+p_{X'BP}+
    p_{X'BY}=\\
    &=&p_{XOY}.
    \end{eqnarray*}

\item \res{Paralelogram $PQKL$ naj bo včrtan v paralelogram $ABCD$ (oglišča prvega ležijo na stranicah
drugega). Dokaži, da imata paralelograma skupno središče.}

    Uporabimo središčno zrcaljenje $\mathcal{S}_S$, kjer je $S$ presečišče diagonal paralelograma $ABCD$.

\item \res{Loki $l_1, l_2,\cdots , l_n$ ležijo na krožnici $k$ in je vsota njihovih
dolžin manjša od polobsega te krožnice. Dokaži, da obstaja tak premer
$PQ$ krožnice $k$, da nobeno od njegovih krajišč ne leži na katerem od
lokov $l_1, l_2,\cdots , l_n$.}

    Naj bo $S$ središče krožnice $k$. Naj bo
    $\mathcal{S}_S:l_1, l_2,\cdots , l_n\rightarrow l'_1, l'_2,\cdots , l'_n$.
    Predpostavimo nasprotno, da za vsak premer $PQ$ eno od njegovih krajišč leži
    na katerem od lokov $l_1, l_2,\cdots , l_n$. Ker je $\mathcal{S}_S(P)=Q$,
    ležita obe krajišči na nekem loku $l_1, l_2,\cdots , l_n$, $l'_1, l'_2,\cdots , l'_n$.
    To pomeni, da tudi vsaka točka krožnice $k$ leži na katerem od lokov $l_1, l_2,\cdots ,
    l_n$, $l'_1, l'_2,\cdots , l'_n$, kar ni možno, saj je skupna dolžina teh lokov manjša od obsega krožnice $k$.

\item \res{Dan je krog $K(S,20)$. Igralca $\mathcal{A}$ in $\mathcal{B}$
izmenično rišeta kroge s polmeri $x_i$ ($1<x_i<2$), ki ležijo v
notranjosti kroga $K$, tako da noben nima skupnih točk
z nobenim od prej narisanih krogov. Zmaga igralec, ki
nariše zadnji krog. Ali obstaja zmagovalna strategija za katerega od
igralcev $\mathcal{A}$ ali $\mathcal{B}$?}

    Igralec $\mathcal{A}$ ima zmagovalno strategijo. V prvi
    potezi nariše krog s središčem $S$, nato pa na vsako
    potezo igralca $\mathcal{B}$ - krog $K_i$ - odgovori s
    potezo - krogom $\mathcal{S}_S(K_i)$.

\item \res{Naj bosta $AB$ in $CD$ tetivi krožnice $k$, ki nimata skupnih
točk, in $P$ poljubna točka, ki leži na tetivi $CD$. Načrtaj takšno
točko $X$ na krožnici $k$, da tetivi $XA$ in $XB$ sekata tetivo $CD$
v točkah $Y$ in $Z$ tako, da je točka $P$ središče daljice $ZY$.}

   Predpostavimo, da so $X$, $Y$ in $Z$ točke, ki izpolnjujejo pogoje iz naloge.
   Iz pogoja $X\in k$ sledi, da je znan kot $\angle CXD=\omega$ - kot nad tetivo
   $CD$ (izrek \ref{ObodObodKot}). Naj bo $C'=\mathcal{S}_P(C)$. Točka $P$ je
   skupno središče daljic $YZ$ in $CC'$, zato je štirikotnik $YC'ZC$ paralelogram.
   Ker je torej $CX\parallel C'Z$, sledi $\angle XZC'\cong \angle YXZ=\angle CXD=\omega$.
   To pomeni, da je $C'ZD=180^0-\omega$, kar omogoča konstukcijo točke $Z$ (izrek \ref{ObodKotGMT}).

\item \res{Zunaj paralelograma $ABCD$ so nad njegovimi stranicami  konstruirani enakostranični
 trikotniki. Dokaži, da so središča teh trikotnikov oglišča novega paralelograma.}

    Naj bodo $APB$, $BQC$, $CMD$ in $AND$ dani enakostranični trikotniki ter $S$
    središče (presečišče diagonal) paralelograma $ABCD$. Ker
     $\mathcal{S}_S:A,B\mapsto C,D$, preslika zrcaljenje
     $\mathcal{S}_S$ trikotnika $APB$ in $BQC$ v trikotnika
     $CMD$ in $AND$
      oz. točki $P$ in $Q$ v točki $M$ in $N$.
      To pomeni, da imata daljici $PQ$ in $MN$ skupno središče, zato je $PQMN$ paralelogram.

\item \res{Načrtaj trapez, tako da bosta osnovnici skladni z danima daljicama $a$ in
$c$, diagonali pa skladni z danima daljicama $e$ in $f$.}

    Če je $ABCD$ iskani trapez, najprej načrtamo trikotnik $DBC'$, kjer je $C'=\mathcal{T}_{\overrightarrow{AB}}(C)$.

\item \res{Mesti (točki) $A$ in $B$ sta na različnih bregovih reke
(pasa, ki ga določata vzporednici $p$ in $q$). Potrebno je narediti
most (daljico $PQ\perp p$, $P\in P$ in $Q\in q$) čez reko, ki bo povezal
mesti $A$ in $B$, tako da bo pot med mestoma najkrajša ($|AP|+|PQ|+|QB|$ minimalno).}

    Uporabimo translacijo  $\mathcal{T}_{\overrightarrow{PQ}}$.

\item \res{V ravnini so dane premice $a$, $b$ in $p$ ter daljica $d$.
Načrtaj vzporednico $q$ premice $p$, ki s premicama $a$ in $b$ določa
daljico, ki je skladna z daljico $d$.}

    Uporabimo translacijo $\mathcal{T}_{\overrightarrow{v}}$,
    kjer je $\overrightarrow{v}\parallel p$ in $|\overrightarrow{v}|=|d|$.

\item \res{Trikotnik $ABE$ je zunaj pravokotnika $ABCD$ načrtan nad stranico $AB$.
Pravokotnici premic $AE$ in $BE$ iz točk $C$ in $D$ se sekata v točki
$P$.  Dokaži, da je $PE\parallel BC$.}

    Uporabimo translacijo $\mathcal{T}_{\overrightarrow{CB}}$ in dokažemo,
     da je $\mathcal{T}_{\overrightarrow{CB}}(P)$ višinska točka trikotnika $ABE$.

\item \res{Točka $M$ leži v notranjosti kvadrata $ABCD$. Dokaži, da obstaja
štirikotnik s pravokotnima diagonalama in s stranicami, ki so skladne z
daljicami $MA$, $MB$, $MC$ in $MD$.}

    Če je $M_1=\mathcal{S}_{BC}(M)$ in
    $M_2=\mathcal{T}_{\overrightarrow{AB}}\circ\mathcal{S}_{AD}(M)$,
    potem dokažemo, da je $BM_1CM_2$ iskan štirikotnik.

\item \res{Skladni krožnici se sekata v točkah $P$ in $Q$. Premica $l$ je
vzporedna s premico $m$, ki poteka skozi središči dveh krožnic, in $l$ seka
krožnici po vrsti v točkah $A$ in $B$ ter $C$ in $D$. Dokaži, da mera
kota $APC$ ni odvisna od izbire premice $l$.}

    Če je $P_1=\mathcal{T}_{\overrightarrow{AC}}(P)$, najprej dokažemo $\angle APC=\angle PCP_1$.

\item \label{nalIzo75} \res{Kaj predstavlja kompozitum $\mathcal{S}_A\circ \mathcal{S}_p$?}

    Središčno zrcaljenje $\mathcal{S}_A$ predstavimo kot
    kompozitum osnih zrcaljenj:
    $\mathcal{S}_A=\mathcal{S}_n\circ\mathcal{S}_q$, kjer je
    $q\parallel p$. V primeru $A\in p$ je kompozitum
    $\mathcal{S}_A\circ \mathcal{S}_p$ osno zrcaljenje, v
    primeru $A\notin p$ pa zrcalni zdrs (glej tudi izrek
    \ref{izoZrcDrsKompSrOsn}).



\item \res{Naj bo $t$ tangenta očrtane krožnice trikotnika $ABC$ v oglišču $A$.
Dokaži, da velja:
$\mathcal{G}_{\overrightarrow{CA}} \circ \mathcal{G}_{\overrightarrow{BC}}
\circ \mathcal{G}_{\overrightarrow{AB}} =\mathcal{S}_t $.}

Naj bodo $p$, $q$ in $r$ simetrale stranic $BC$, $CA$ in $AB$, ki se
sekajo v središču $O$ očrtane krožnice trikotnika $ABC$. Kompozitum
$\mathcal{S}_q \circ
 \mathcal{S}_p  \circ \mathcal{S}_r$ je inirektna izometrija s fiksnima
  točkama $O$ in $A$, zato je po izreku \ref{izo1ftIndZrc}
 $\mathcal{S}_q \circ \mathcal{S}_p  \circ
 \mathcal{S}_r=\mathcal{S}_{OA}$. Zatorej je (po izreku
 \ref{izoZrcDrsKompSrOsn}):
\begin{eqnarray*}
\mathcal{I}&=&\mathcal{G}_{\overrightarrow{CA}} \circ
\mathcal{G}_{\overrightarrow{BC}} \circ \mathcal{G}_{\overrightarrow{AB}}
=
 \mathcal{S}_A \circ \mathcal{S}_q \circ
 \mathcal{S}_p \circ \mathcal{S}_B \circ
 \mathcal{S}_B \circ \mathcal{S}_r\\
 &=&
 \mathcal{S}_A \circ \mathcal{S}_q \circ
 \mathcal{S}_p  \circ \mathcal{S}_r
  =
 \mathcal{S}_A \circ \mathcal{S}_{OA}
 =
 \mathcal{S}_t\circ\mathcal{S}_{OA}\circ \mathcal{S}_{OA}
=\mathcal{S}_t.
\end{eqnarray*}



\item \res{Kaj predstavlja kompozitum $\mathcal{S}_A\circ
\mathcal{S}_B\circ \mathcal{S}_{AB}$?}

    Po izreku \ref{transl2sred} je:
    $\mathcal{S}_A\circ \mathcal{S}_B\circ \mathcal{S}_{AB}=
    \mathcal{T}_{2\overrightarrow{BA}} \circ \mathcal{S}_{AB}=
    \mathcal{G}_{2\overrightarrow{BA}}$.

\item \res{Naj bo $ABC$ enakostranični trikotnik. Določi os in vektor
zrcalnega zdrsa, ki je določen s kompozitumom
$\mathcal{S}_{BC}\circ \mathcal{S}_{AB}\circ \mathcal{S}_{CA}$.}

    Naj bosta $A_1$ in $B_1$ središči stranic $BC$ in $AC$
    trikotnika $ABC$. Najprej dokažemo, da je $\mathcal{S}_{BC}\circ
     \mathcal{S}_{AB}\circ \mathcal{S}_{CA}(A_1B_1)=A_1B_1$, nato
     poiščemo $\mathcal{S}_{BC}\circ \mathcal{S}_{AB}\circ \mathcal{S}_{CA}(B_1)$.

\item \res{V isti ravnini sta dana konveksni sedemkotnik $PQRSTUV$ in krožnica $k$.
Načrtaj sedemkotnik $ABCDEFG$, ki je včrtan dani krožnici,
njegove stranice pa so vzporedne s stranicami danega sedemkotnika.}

    Naj bodo $p$, $q$, $r$, $s$, $t$, $u$
    in $v$ simetrale stranic sedemkotnika $ABCDEFG$, ki so
    pravzaprav pravokotnice nosilk stranic sedemkotnika  $PQRSTUV$
      iz središča $O$ krožnice $k$. Kompozitum
    $\mathcal{I}=\mathcal{S}_v\circ \mathcal{S}_u\circ \mathcal{S}_t
    \circ \mathcal{S}_s\circ \mathcal{S}_r\circ \mathcal{S}_q\circ
    \mathcal{S}_p$ je indirektna izometrija s fiksnima  točkama $O$
    in $A$, zato je po izreku \ref{izo1ftIndZrc} $\mathcal{I}=\mathcal{S}_{OA}$.
    Os $OA$ lahko načrtamo kot simetralo daljice $XX'$, kjer je $X$
    poljubna točka in $X'=\mathcal{I}(X)$. To omogoča konstrukcijo
    oglišča $A$, nato pa ostalih oglišč sedemkotnika $ABCDEFG$.

    Če rešitev obstaja, jo dobimo po opisanem postopku. Če pa ne obstaja, pa
    po tem postopku dobimo neenostavno sklenjeno lomljenko, ki ima posamezne daljice
    vzporedne sedemkotniku $PQRSTUV$.

\item \res{Dokaži, da velja:
$\mathcal{S}_A\circ\mathcal{S}_B\circ\mathcal{S}_C=
\mathcal{S}_C\circ\mathcal{S}_B\circ\mathcal{S}_A$.}

    Po izreku \ref{izoKomp3SredZrc} je kompozitum treh
    središčnih zrcaljenj središčno zrcaljenje. Ker je še
    središčno zrcaljenje involucija (izrek \ref{izoSredZrcInv}),
    velja:
    $\mathcal{S}_A\circ\mathcal{S}_B\circ\mathcal{S}_C=\mathcal{S}_X=
    \mathcal{S}_X^{-1}=
    \left(\mathcal{S}_A\circ\mathcal{S}_B\circ\mathcal{S}_C\right)^{-1}=
    \mathcal{S}_C\circ\mathcal{S}_B\circ\mathcal{S}_A$.

\item \res{Naj bodo $A$, $B$ in $C$ tri nekolinearne točke. Določi točko
$S$, za katero velja: $$\mathcal{S}_S\circ\mathcal{S}_A\circ
\mathcal{S}_S\circ\mathcal{S}_B\circ
\mathcal{S}_S\circ\mathcal{S}_C=\mathcal{E}.$$}
    Po izrekih \ref{transl2sred} in \ref{translKomp} je:
    $\mathcal{E}=\mathcal{S}_S\circ\mathcal{S}_A\circ
\mathcal{S}_S\circ\mathcal{S}_B\circ
\mathcal{S}_S\circ\mathcal{S}_C=
\mathcal{T}_{\overrightarrow{AS}}\circ
\mathcal{T}_{\overrightarrow{BS}}\circ
\mathcal{T}_{\overrightarrow{CS}}=
 \mathcal{T}_{\overrightarrow{CS}+\overrightarrow{BS}+\overrightarrow{AS}}$.
  Torej velja:
  $\overrightarrow{CS}+\overrightarrow{BS}+\overrightarrow{AS}
  =\overrightarrow{0}$ oz.
  $\overrightarrow{SC}+\overrightarrow{SB}+\overrightarrow{SA}
  =\overrightarrow{0}$. Po izreku \ref{tezTrikVektObr} je točka
  $S$ težišče trikotnika $ABC$.

\item \res{Če je $S$ središče daljice $AB$, je
$\mathcal{S}_S\circ\mathcal{S}_A\circ\mathcal{S}_S=\mathcal{S}_B$. Dokaži.}

Uporabi izrek o transmutaciji središčnega zrcaljenja
\ref{izoTransmSredZrc}.

%olimp
%____________________________________________________

\item  \res{Naj bosta $ABC$ in $A'B'C'$ skladna
enakostranična trikotnika. Dokaži, da so središča daljic $AA'$, $BB'$ in $CC'$
bodisi kolinearne točke bodisi oglišča novega enakostraničnega trikotnika\footnote{Predlog za MMO 1971. (SL 12.)}.}

    Če sta trikotnika enako orientirana, je trditev direktna
    posledica zgleda \ref{RotacZglVeck}. Če pa trikotnika nista
    enako orientirana, je trditev direktna posledica izreka
    \ref{Chasles-Hjelmsleva}.

\item \res{Premica $l$ poteka skozi višinsko
točko trikotnika $ABC$. Označimo z $l_a$, $l_b$ in $l_c$ zrcalne slike
premice $l$ glede na premice $BC$, $AC$ in $BC$. Dokaži, da se premice
  $l_a$, $l_b$ in $l_c$ sekajo v skupni točki, ki leži na očrtani krožnici trikotnika $ABC$\footnote{Predlog za MMO 1967. (LL 41.)}.}

    Naj bo $k$ očrtana krožnica in $V$ višinska točka
    trikotnika $ABC$. Označimo z $V_a=\mathcal{S}_{BC}(V)$,
     $V_b=\mathcal{S}_{AC}(V)$ in $V_c=\mathcal{S}_{AB}(V)$.
     Po izreku \ref{TockaV'} ležijo točke $V_a$, $V_b$ in $V_c$
     na krožnici $k$ in velja $\measuredangle V_aCV_b =2\measuredangle BCA=2\gamma$.

    Po predpostavki je $\mathcal{S}_{BC}(l)=l_a$,
    $\mathcal{S}_{AC}(l)=l_b$ in $\mathcal{S}_{AB}(l)=l_c$,
     zato je $V_a\in l_a$, $V_b\in l_b$ in $V_c\in l_c$.
     Iz $\mathcal{S}_{BC}(l)=l_a$ (oz. $\mathcal{S}_{BC}(l_a)=l$)
      in $\mathcal{S}_{AC}(l)=l_b$ sledi tudi (izrek \ref{rotacKom2Zrc})
      $\mathcal{R}_{C,2\gamma}(l_a)=
     \mathcal{S}_{AC}
    \circ\mathcal{S}_{BC}(l_a)=l_b$,
     zato je po izreku \ref{rotacPremPremKot}  $\angle l_a, l_b=2\gamma$.

    Označimo z $X$ presečišče premic $l_a$ in $l_b$
    (po prejšnji ugotovitvi $l_a$ in $l_b$ nista vzporedni).
    Ker je še $V_a\in l_a$ in $V_b\in l_b$, je $V_aXV_b=\angle l_a,
     l_b=2\gamma=\measuredangle V_aCV_b$, kar pomeni, da točka $X$
     leži na krožnici $k$. Torej se premici $l_a$ in $l_b$ sekata
     na krožnici $k$. Isto velja za premici $l_a$ in $l_c$ oz. $l_b$
     in $l_c$. Ker $l_c\neq V_aV_b$,  tudi premica $l_c$  seka krožnico $k$ v točki $X$.

\item \res{Enakostranični trikotniki
 $BAM$, $DCP$, $BCN$ in $DAQ$ so konstruirani nad stranicami konveksnega
 štirikotnika $ABCD$. Prva dva trikotnika sta konstruirana zunaj, druga
 dva pa znotraj tega štirikotnika. Kaj lahko rečemo o štirikotniku $MNPQ$\footnote{Predlog za MMO 1982. (SL 20.)}?}

Daljico $AC$  preslika rotacija $R_{B, 60^0}$ v daljico $MN$.
Isto daljico $AC$ pa preslika rotacija $R_{D, 60^0}$ v daljico $QP$.
Zato sta daljici $MN$ in $QP$ skladni in vzporedni
(izrek \ref{rotacPremPremKot}). Torej je štirikotnik $MNPQ$ paralelogram.

Drugi način je, da uporabimo kompozitum
$\mathcal{I}=\mathcal{R}_{Q,60^0}\circ
\mathcal{R}_{P,-60^0}\circ \mathcal{R}_{N,60^0}\circ
\mathcal{R}_{M,-60^0}$. Izometrija $\mathcal{I}$ po
izreku \ref{rotacKomp2rotac} predstavlja identiteto
ali translacijo. Ker je $\mathcal{I}(A)=A$, je
$\mathcal{I}=\mathcal{E}$. Torej $\mathcal{I}=
\mathcal{R}_{Q,60^0}\circ \mathcal{R}_{P,-60^0}=
\mathcal{R}^{-1}_{M,-60^0}\circ \mathcal{R}^{-1}_{N,60^0}$
oz. $\mathcal{I}=\mathcal{R}_{Q,60^0}\circ \mathcal{R}_{P,-60^0}=
\mathcal{R}_{M,60^0}\circ \mathcal{R}_{N,-60^0}$.
Ni težko dokazati (kot posledico izreka \ref{rotacKomp2rotac}),
da je potem $\overrightarrow{PQ}=\overrightarrow{NM}$, kar pomeni, da je
štirikotnik $MNPQ$ paralelogram.


 \item \res{Naj bo $\mathcal{R}_{D,90^0}\circ
 \mathcal{R}_{C,90^0}\circ\mathcal{R}_{B,90^0}\circ\mathcal{R}_{A,90^0}
 =\mathcal{E}$. Dokaži, da je $AC\perp BD$ in $AC\cong BD$.}

Enakost $\mathcal{R}_{D,90^0}\circ
 \mathcal{R}_{C,90^0}\circ\mathcal{R}_{B,90^0}\circ\mathcal{R}_{A,90^0}
 =\mathcal{E}$ je ekvivalentna z $\mathcal{R}_{D,90^0}\circ
 \mathcal{R}_{C,90^0}=\mathcal{R}_{A,-90^0}\circ\mathcal{R}_{B,-90^0}$
 oz. $\mathcal{S}_P=\mathcal{S}_Q$, kjer sta $CPD$ in $BQA$
 enakokraka pravokotna trikotnika s hipotenuzama $CD$ in $AB$
 (izrek \ref{rotacKomp2rotac}). Iz $\mathcal{S}_P=\mathcal{S}_Q$
 pa sledi $P=Q$, kar pomeni, da sta $CPD$ in $BPA$ enakokraka
 pravokotna trikotnika s hipotenuzama $CD$ in $AB$. Torej
 $\mathcal{R}_{P,90^0}:D,B\mapsto C,A$, zato je $DB\perp CA$ in
 $DB\cong CA$.

\end{enumerate}


%REŠITVE -  Podobnost
%________________________________________________________________________________

\poglavje{Similarity}


\begin{enumerate}



%Ttales

\item \label{nalPod1}
\res{Naj bo $S$ presečišče diagonal $AC$ in $BD$ trapeza $ABCD$. Naj bosta
 $P$ in $Q$ presečišči vzporednice osnovnic $AB$ in $CD$ skozi točko $S$ s krakoma tega trapeza. Dokaži, da je $S$ središče daljice $PQ$.}

Predpostavimo, da je $P\in AD$ in $Q\in BC$. Po Talesovem izreku \ref{TalesovIzrek} je:
 \begin{eqnarray*}
 \frac{\overrightarrow{PS}}{\overrightarrow{AB}}
=\frac{\overrightarrow{DS}}{\overrightarrow{SB}}
=\frac{\overrightarrow{CS}}{\overrightarrow{SA}}
=\frac{\overrightarrow{SQ}}{\overrightarrow{AB}},
 \end{eqnarray*}
 zato je $\overrightarrow{PS}=\overrightarrow{SQ}$, tore je točka $S$ središče daljice $PS$.

Dokazana trditev je poseben primer trditve iz zgleda \ref{vektTrapezZgled}.

\item
\res{Naj bo $ABCD$ trapez z osnovnico $AB$, točka $S$ presečišče njegovih
diagonal in $E$ presečišče nosilk krakov tega trapeza. Dokaži, da
premica $SE$ poteka skozi  središči osnovnic $AB$ in $CD$.}

Označimo s $F$ in $G$ presečišči premice $SE$ s premicama $AB$ in $CD$ ter $P$ in $Q$ točke kot v prejšnji nalogi \ref{nalPod1}. Po  Talesovem izreku \ref{TalesovIzrek} je:
 \begin{eqnarray*}
 \frac{\overrightarrow{AF}}{\overrightarrow{PS}}
=\frac{\overrightarrow{EF}}{\overrightarrow{ES}}
=\frac{\overrightarrow{FB}}{\overrightarrow{SQ}}.
 \end{eqnarray*}
Ker je po prejšnji nalogi \ref{nalPod1} $\overrightarrow{PS}=\overrightarrow{SQ}$, je
 $\overrightarrow{AF}=\overrightarrow{FB}$, torej je točka $F$ središče stranice $AB$. Podobno je točka $G$ središče stranice $CD$.

\item
\res{Naj bodo $P$, $Q$ in $R$ točke, v katerih poljubna premica skozi točko $A$ seka
nosilke stranic $BC$ in $CD$ ter diagonalo $BD$ paralelograma
$ABCD$. Dokaži, da je $|AR|^2=|PR|\cdot |QR|$.}

Ker je $AD\parallel BC$ in $AB\parallel CD$, je po  Talesovem izreku \ref{TalesovIzrek}:
\begin{eqnarray*}
 \frac{AR}{PR}=\frac{DR}{BR}=\frac{QR}{AR},
 \end{eqnarray*}
iz tega sledi iskana relacija.

\item
 \res{Točki $D$ in $K$ ležita na stranicah $BC$ in $AC$ trikotnika $ABC$ tako, da velja $BD:DC=2:5$ in $AK:KC=3:2$. Izračunaj razmerje, v katerem premica $BK$ deli daljico
$AD$.}

 Rešitev: $21:4$.

\item
\res{Naj bo $P$ takšna točka stranice $AD$ paralelograma $ABCD$, da velja $\overrightarrow{AP}=
\frac{1}{n}\overrightarrow{AD}$, ter $Q$ presečišče  premic $AC$ in $BP$. Dokaži, da velja:
 $$\overrightarrow{AQ}=\frac{1}{n + 1}\overrightarrow{AC}.$$}


Ker je $AD\parallel BC$ in $AB\parallel CD$, je po  Talesovem izreku \ref{TalesovIzrek}:
\begin{eqnarray*}
 \frac{\overrightarrow{AQ}}{\overrightarrow{QC}}
=\frac{\overrightarrow{AP}}{\overrightarrow{BC}}=
\frac{\overrightarrow{AP}}{\overrightarrow{AD}}=\frac{1}{n}.
 \end{eqnarray*}
Torej velja $\overrightarrow{AQ}=\frac{1}{n}\overrightarrow{QC}$, iz tega pa sledi iskana relacija.

%Homotetija

\item \label{nalPod6}
\res{Dani so: točka $A$, premici $p$ in $q$ ter daljici $m$ in $n$. Načrtaj premico $s$ skozi točko $A$,
ki naj seka premici $p$ in $q$ v takšnih točkah $X$ in $Y$, da velja $XA:AY= m:n$.}

Predpostavimo, da je $s$ iskana premica in $\mathcal{B}(X,A,Y)$.
Naj bo $h_{A,k}$ središčni razteg s središčem $A$ in koeficientom $k=-\frac{|n|}{|m|}$. Potem je $Y=h_{A,k}(X)$. Ker je $X\in p$, velja tudi  $Y=h_{A,k}(X)\in h_{A,k}(p)$. Torej točko $Y$ dobimo iz pogoja $Y\in h_{A,k}(p)\cap q$. Druge rešitve (za $\mathcal{B}(A,X,Y)$ oz. $\mathcal{B}(X,Y,A)$) dobimo, če izberemo druge vrednosti za koeficient $k$.

\item \res{Dani so: točka $S$, premice $p$, $q$ in $r$ ter daljici $m$ in $n$. Načrtaj premico $s$ skozi točko $S$,
ki naj seka  premice $p$, $q$ in $r$ v takšnih točkah  $X$, $Y$ in $Z$, da velja $XY:YZ=m:n$.}

Uporabimo prejšnjo nalogo (\ref{nalPod6}) in najprej načrtamo premico $s'$, ki poteka skozi poljubno točko $Y'\in q$ in premici $p$ in $r$ seka v takšnih točkah $X'$ in $Y'$, da velja $X'Y':Y'Z'=m:n$. Po posledici Talesovega izreka \ref{TalesPosl3} je iskana premica $s$ vzporedna s premico $s'$.

 \item \res{V dani trikotnik $ABC$ včrtaj takšen pravokotnik $PQRS$, da stranica $PQ$ leži na stranici $BC$,
oglišči $R$ in $S$ ležita na stranicah $AB$ in $AC$, pri tem pa še velja $PQ=2QR$.}

Najprej načrtamo poljuben pravokotnik $P'Q'R'S'$, tako da stranica $P'Q'$ leži na stranici $BC$,
oglišče $R'$ leži na stranici $AB$, pri tem pa še velja $P'Q'=2Q'R'$. Potem uporabimo središčni razteg s središčem $B$. Glej zgled \ref{sredRaztegZgledKvadrat}.


\item
\res{Načrtaj:}

(\textit{a}) \res{romb, če sta dani stranica $a$ in razmerje diagonal $e:f$}.

Naj bo $ABCD$ romb, pri katerem je $AB\cong a$ in $\frac{AC}{BD}=\frac{e}{f}$. Označimo s $S$ presečišče diagonal $AC$ in $BD$. $ASB$ je pravokotni trikotnik s hipotenuzo $AB\cong a$, za njegovi kateti pa velja $\frac{AS}{BS}=\frac{e}{f}$. Najprej načrtamo poljubni pravokotni trikotnik $A'SB'$ s hipotenuzo $A'B'$, pri katerem je $\frac{A'S}{B'S}=\frac{e}{f}$. Trikotnik $ASB$ dobimo kot sliko trikotnika $A'SB'$ pri središčnem raztegu $h_{S,\frac{a}{A'B'}}$. Potem je še $C=\mathcal{S}_S(A)$ in $D=\mathcal{S}_S(B)$.

 (\textit{b}) \res{trapez, če so dani: notranja kota $\alpha$ in $\beta$ ob eni osnovnici, razmerje te osnovnice in višine $a:v$ ter druga osnovnica $c$}

Naj bo $a:v=m:n$, kjer sta $m$ in $n$ dani daljici. Načrtamo najprej trapez $A'B'C'D$, pri katerem je $\angle DA'B'\cong\alpha$, $\angle A'B'C'\cong\beta$,  $A'B'=m$, višina pa je skladna daljici $n$. Iskani trapez dobimo kot sliko trapeza $A'B'C'D$ pri središčnem raztegu $h_{D,\frac{c}{DC'}}$.

\item
\res{Naj bodo: $A$ točka v notranjosti kota $pSq$, $l$ premica in $\alpha$ kot v neki ravnini. Načrtaj takšen trikotnik
$APQ$, da velja: $P\in p$, $Q\in q$, $\angle PAQ\cong \alpha$ in
$PQ\parallel l$.}

Naj bosta $P'$ in $Q'$ presečišči poljubne vzporednice premice $l$ s krakoma $p$ in $q$ ter $A'$ presečišče poltraka $SA$ z lokom, ki je geometrijsko mesto točk $X$, za katere je $\angle P'XQ'=\alpha$ (izrek \ref{ObodKotGMT}). Trikotnik $APQ$ je potem slika trikotnika $A'P'Q'$ pri središčnem raztegu $h_{S,\frac{SP}{SP'}}$.

\item
\res{Načrtaj krožnico $l$, ki se dotika dane krožnice $k$ in dane premice $p$, če je dano še dotikališče:}

(\textit{a}) \res{$l$ in $p$}

Naj bo $S$ središče krožnice, ki se dotika premice $p$ v točki $P$ in krožnice $k(K,r)$ zunaj v točki $Q$. Označimo s $P'$ takšno točko  premice $SP$, da velja $\mathcal{B}(S,P,P')$ in $PP'\cong KQ=r$. Ker je $|SP'|=|SP|+|PP'|=|SQ|+|QK|=|SK|$, je $SP'K$ enakokraki trikotnik z osnovnico $P'K$, zato točka $S$ leži na simetrali daljice $P'K$.
Dokazana dejstva omogočajo konstrukcijo. V našem primeru so dani premica $p$, krožnica $k(K,r)$ in točka $P$. Najprej načrtamo takšno točko $P'$, da velja: $P'P\perp p$, $P'P\cong r$ in $P', K\div p$, nato pa točko $S$ kot presečišče premice $P'P$ s simetralo $s_{P'K}$ daljice $P'K$ in na koncu krožnico $l(S,SP)$.

Dokažemo še, da je $l(S,SP)$ iskana krožnica. Po konstrukciji je $PP'\perp p$, zato je premica $p$ tangenta krožnice $l(S,SP)$ v točki $P$. Naj bo $Q_1$ presečišče premice $SK$ in krožnice $k$. Po konstrukciji leži točka $S$ na simetrali $s_{P'K}$ daljice $P'K$, zato je $SP'\cong SK$. Iz tega in konstrukcije sledi $|SQ_1|=|SK|-|KQ_1|=|SP'|-r=|SP|+|PP'|-r=|SP|+r-r=|SP|$. To pomeni, da velja $Q_1\in l$. Ker so točke $S$, $Q_1$ in $K$ kolinearne, se krožnici $l$ in $k$ dotikata v točki $Q_1$ (pravzaprav je $Q_1=Q$).

Naloga ima v splošnem primeru še eno rešitev, ki jo dobimo, če v konstrukciji izberemo $\mathcal{B}(S,P,P')$.


(\textit{b}) \res{$l$ in $k$}

Naj bo $S$ središče krožnice, ki se dotika premice $p$ v točki $P$ in krožnice $k(K,r)$ zunaj v točki $Q$. Označimo s $S_1$ poljubno točko na poltraku $KQ$ in $l_1(S_1,S_1P_1)$. Krožnico, ki se dotika premice $p$ v točki $P_1$, daljico $KS_1$ pa seka v točki $Q_1$. Naj bo še $O$ presečišče premic $p$ in $KQ$.

Najprej narišemo krožnico $l_1$, nato uporabimo središčni razteg $h_{O,\frac{OQ}{OQ_1}}$.

Drugi način je, da direktno načrtamo točko $T$ iz pogoja
 $\angle OQP=\angle SQP=\frac{1}{2}\angle OSP=\frac{1}{2}\left( 90^0-\angle KOP \right)$.

Nalogi (\textit{a}) in (\textit{b}) lahko rešimo tudi z uporabo inverzije (glej razdelek \ref{odd9ApolDotik}).

\item \label{nalPod12}
\res{Načrtaj krožnico, ki se dotika krakov kota $pSq$ in:}

(\textit{a}) \res{poteka skozi dano točko}

Gre za Apolonijev problem o dotiku krožnic (glej razdelek \ref{odd9ApolDotik}), ki ga lahko rešimo z inverzijo.
Lahko uporabimo tudi središčni razteg s središčem v točki $S$, tako da najprej načrtamo poljubno krožnico, ki se dotika krakov $p$ in $q$.

(\textit{b}) \res{se dotika dane krožnice}

Najprej načrtamo koncentrično krožnico, ki poteka skozi središče dane krožnice in se dotika premic $p'$ in $q'$, ki sta na razdalji polmera dane krožnice vzporedni s krakoma $p$ in $q$.

\item
\res{Naj bodo: $p$ in $q$ premici, $S$ točka ($S\notin p$ in $S\notin q$) ter daljici $m$ in $n$.
Načrtaj krožnici $k$ in $l$, ki se od zunaj dotikata v točki $S$, prva se dotika premice $p$,
druga premice $q$, razmerje polmerov pa je enako $m:n$.}

S središčnim raztegom $h_{S,-\frac{n}{m}}$ se iskana krožnica $k$ preslika v iskano krožnico $l$, njena tangenta $p$ pa v tangento $p'$ krožnice $l$. Ker lahko načrtamo premico $p'$, se problem prevede na konstrukcijo krožnice $l$, ki poteka skozi točko $S$ in se dotika dveh premic: $q$ in $p'$. Naprej uporabimo prejšnjo nalogo \ref{nalPod12}.

\item
\res{V ravnini so dani: premica $p$ ter  točki $B$ in $C$, ki ležita na krožnici $k$. Načrtaj takšno točko $A$
na krožnici $k$, da težišče trikotnika $ABC$ leži na premici $p$.}

Uporabimo središčni razteg $h_{A_1,2}$, kjer je $A_1$ središče daljice $BC$.

\item
\res{Naj bo $k$ krožnica s premerom $PQ$. Načrtaj kvadrat $ABCD$ tako, da velja $A,B\in PQ$ in $C,D\in k$.}

Načrtaj poljubnen kvadrat $A'B'C'D'$ tako, da velja $A',B'\in PQ$, središče $S$ daljice $A'B'$ pa je hkrati središče krožnice $k$. Nato uporabimo ustrezni središčni razteg s središčem v točki $S$.

\item
\res{V isti ravnini so dani: premici $p$ in $q$, točka $A$ ter daljici $m$ in $n$. Načrtaj takšen
pravokotnik $ABCD$, da velja $B\in p$, $D\in q$
in $AB:AD=m:n$.}

Rotacijski razteg $\rho_{A,\frac{n}{m},90^0}$ preslika točko $B$ v točko $D$, zato je $D\in q\cap \rho_{A,\frac{n}{m},90^0}(p)$.

\item \res{V dani trikotnik $ABC$ včrtaj trikotnik tako, da bodo njegove stranice vzporedne z danimi premicami $p$, $q$ in $r$.}

Najprej načrtamo trikotnik $X'Y'Z'$ tako, da izpustimo pogoj $X'\in BC$. Torej naj velja le: $Y'\in AC$, $Z'\in AB$, $Y'Z'\parallel p$, $X'Z'\parallel q$ in $X'Y'\parallel r$. Nato načrtamo točko $X=AX'\cap BC$ in uporabimo središčni razteg $h_{A,\frac{\overrightarrow{AX}}{\overrightarrow{AX'}}}$, ki trikotnik $X'Y'Z'$ preslika v iskani trikotnik $XYZ$.


\item \res{Naj bodo $p$, $q$ in $r$ tri premice neke ravnine. Načrtaj premico $t$, ki je
pravokotna  na premico $p$ in premice $p$, $q$ in $r$ seka po vrsti v takšnih točkah $P$, $Q$ in $R$, da velja $PQ\cong QR$.}

Če je $p\parallel r$, je naloga trivialna.
Naj bo $S=p\cap r$.

Predpostavimo, da je $t$ premica, ki je pravokotna na premico $p$ in seka premice $p$, $q$ in $r$ po vrsti v takšnih točkah $P$, $Q$ in $R$, da velja $PQ\cong QR$.
Naj bo $h_{S,k}$ središčni razteg s poljubnim koeficientom $k$ in
 $h_{S,k}:\hspace*{1mm} P,Q,R\mapsto P',Q',R'$. Potem velja: $R'\in r$, $P'\in p$, $P'R'\parallel t$. Ker je po predpostavki $Q$ središče daljice $PR$ in središčni razteg $h_{S,k}$ transformacija podobnosti, je tudi $Q'$ središče daljice $P'R'$. Te ideje nam omogočajo konstrukcijo.

Načrtamo poljubno točko $P'$ premice $p$, točko $R'$ kot presečišče pravokotnice $t'$ premice $p$ v točki $P'$ in premice $r$, središče  $Q'$ daljice $P'R'$, točko $Q$ kot presečišče premic $SQ'$ in $q$ ter točki $P$ in $R$ kot presečišči pravokotnice $t$ premice $p$ v točki $Q$ s premicama $p$ in $r$.

Dokažemo še, da je $t$ iskana premica.
Po konstrukciji je $t$ pravokotnica premice $p$, ki seka premice  $p$, $q$ in $r$ po vrsti v točkah $P$, $Q$ in $R$. Dokažemo, da velja $PQ\cong QR$. Po konstrukciji je še  $t'\perp p$, zato je tudi $t'\parallel t$. Po posledici Talesovega izreka \ref{TalesPosl3} je: $\frac{\overrightarrow{PQ}}{\overrightarrow{QR}}=
\frac{\overrightarrow{P'Q'}}{\overrightarrow{Q'R'}}$. Ker je po konstrukciji točka $Q'$ središče daljice $P'R'$, je $\frac{\overrightarrow{P'Q'}}{\overrightarrow{Q'R'}}=1$, zato je tudi $\frac{\overrightarrow{PQ}}{\overrightarrow{QR}}=1$, torej $PQ\cong QR$.

Naloga ima eno samo rešitev natanko tedaj, ko ni $p\perp q$. V primeru $p\perp q$ ni rešitve.





%Ppodobnost ttrikotnikov

\item \res{Naj bo $P$ notranja točka trikotnika $ABC$ ter $A_1$, $B_1$ in $C_1$ pravokotne projekcije
točke $P$ na trikotnikovih stranicah $BC$, $AC$ in $AB$. Analogno so točke $A_2$, $B_2$ in $C_2$ določene s točko $P$
in trikotnikom $A_1B_1C_1$,..., točke $A_{n+1}$, $B_{n+1}$ in $C_{n+1}$  s točko $P$ in trikotnikom $A_nB_nC_n$... Kateri od trikotnikov $A_1B_1C_1$, $A_2B_2C_2$, ... so podobni trikotniku $ABC$?}


Označimo:
\begin{eqnarray*}
\angle BAP= \alpha_1, \hspace*{3mm}  \angle PAC= \alpha_2,\\
\angle CBP= \beta_1, \hspace*{3mm}   \angle PBA= \beta_2,\\
\angle ACP= \gamma_1, \hspace*{3mm}  \angle PCB= \gamma_2.
\end{eqnarray*}

Štirikotniki $AC_1PB_1$, $BA_1PC_1$ in $CB_1PA_1$ so tetivni (izrek \ref{TetivniPogoj}), zato je:
\begin{eqnarray*}
\angle B_1A_1P= \gamma_1, \hspace*{3mm}  \angle PA_1C_1= \beta_2,\\
\angle C_1B_1P= \alpha_1, \hspace*{3mm}   \angle PB_1A_1= \gamma_2,\\
\angle A_1C_1P= \beta_1, \hspace*{3mm}  \angle PC_1B_1= \alpha_2.
\end{eqnarray*}

Če postopek nadaljujemo, dobimo notranje kote $\angle A_n$, $\angle B_n$ in $\angle C_n$ trikotnikov $A_nB_nC_n$; $n\in \{0,1,2,\ldots \}$ (pri tem je $\triangle A_0B_0C_0=\triangle ABC$):


\vspace*{3mm}

\hspace*{12mm}
\begin{tabular}{|c|c|c|c|c|c|}
  \hline
  % after \\: \hline or \cline{col1-col2} \cline{col3-col4} ...
  $n$ & 0 & 1 & 2 & 3 & $\cdots$ \\
  \hline
  $\angle A_n$ & $\alpha_1+\alpha_2$ & $\gamma_1+\beta_2$ & $\beta_1+\gamma_2$ & $\alpha_1+\alpha_2$ & $\cdots$ \\
  \hline
  $\angle B_n$ & $\beta_1+\beta_2$ & $\alpha_1+\gamma_2$ & $\gamma_1+\alpha_2$ & $\beta_1+\beta_2$ & $\cdots$ \\
  \hline
  $\angle C_n$ & $\gamma_1+\gamma_2$ & $\beta_1+\alpha_2$ & $\alpha_1+\beta_2$ & $\gamma_1+\gamma_2$ & $\cdots$ \\
  \hline
\end{tabular}

\vspace*{3mm}

To pomeni, da so trikotniki $A_3B_3C_3$, $A_7B_7C_7$, ..., $A_{4n-1}B_{4n-1}C_{4n-1}$ ,... podobni trikotniku $ABC$.

\item
\res{Naj bo $k$ očrtana krožnica štirikotnika $ABCD$, $E$ presečišče njegovih diagonal ter $CB\cong CD$. Dokaži, da je $\triangle ABC \sim\triangle BEC$.}

Uporabimo: $\angle EBC=\angle DBC \cong \angle DAC\cong\angle BAC$.

\item
\res{Naj bo $ABCD$ paralelogram. V točkah $E$ in $F$ trikotniku $ABC$ očrtana krožnica seka premici $AD$ in $CD$. Dokaži, da je $\triangle EBC\sim\triangle EFD$.}

Brez škode za splošnost (v ostalih primerih je dokaz podoben) predpostavimo, da velja $\mathcal{B}(A,D,E)$ in $\mathcal{B}(C,D,F)$.
Ker je $ABCD$ paralelogram, $ABCE$ pa tetivni štirikotnik, po izrekih \ref{paralelogram} in \ref{TetivniPogoj} velja:
\begin{eqnarray*}
 \angle EDF\cong \angle ADC=180^0-\angle DAB=180^0-\angle EAB=\angle ECB.
\end{eqnarray*}
Po izreku \ref{ObodObodKot} je še:
\begin{eqnarray*}
 \angle EFD\cong \angle EFC\cong\angle EBC.
\end{eqnarray*}
Iz izreka \ref{PodTrikKKK} sledi $\triangle EFD\sim\triangle EBC$.


\item
 \res{Naj bosta $AA'$ in $BB'$ višini ostrokotnega trikotnika $ABC$. Dokaži, da je
$\triangle ABC\sim\triangle A'B'C$.}

Najprej dokažemo, da je $AB'A'B$ tetivni štirikotnik, nato pa
 $\angle A'B'C\cong\angle ABC$.

\item \label{nalPod23}
\res{Naj bo nosilka višine $AD$ trikotnika $ABC$ hkrati tangenta očrtane krožnice tega trikotnika. Dokaži, da velja $|AD|^2=|BD|\cdot |CD|$.}

Ker je po izreku \ref{ObodKotTang} $\angle DAB\cong\angle BCA=\angle DCA$, sta pravokotna trikotnika
$DAB$ in $DCA$ podobna (izrek \ref{PodTrikKKK}). Iz tega sledi
$\frac{AD}{CD}=\frac{BD}{AD}$ oz. $|AD|^2=|BD|\cdot |CD|$.

Trditev je tudi direktna posledica izreka o potenci točke glede na krožnico \ref{izrekPotenca}.

\item
\res{V trikotniku $ABC$ naj bo notranji kot ob oglišču $A$ dvakrat večji
od notranjega kota ob oglišču $B$. Dokaži, da velja $|BC|^2= |AC|^2+|AC|\cdot |AB|$.}

Naj bo $D$ točka, v kateri simetrala notranjega kota $AC$ trikotnika $ABC$ seka njegovo stranico $BC$. Po predpostavki je $\angle BAD\cong\angle DAC\cong\angle CBA$. Torej je trikotnik $DAB$ enakokrak z osnovnico $AB$, zato je $AD\cong BD$.
Iz pogoja $\angle DAC\cong\angle CBA$ je po izreku \ref{PodTrikKKK} še $\triangle CAD\sim \triangle CBA$. Iz te podobnosti in dokazanega $AD\cong BD$ sledi:
 \begin{eqnarray*}
\frac{AC}{BC}=\frac{CD}{CA}
 \hspace*{2mm} \textrm{ in }  \hspace*{2mm}
 \frac{AC}{BC}=\frac{AD}{BA}=\frac{BD}{BA}
\end{eqnarray*}
oziroma:
\begin{eqnarray*}
|CD|\cdot |BC|=|AC|^2
 \hspace*{2mm} \textrm{ in }  \hspace*{2mm}
 |BC|\cdot |BD|=|AC|\cdot |AB|.
\end{eqnarray*}
 Če seštejemo dobljeni enakosti in uporabimo $|CD|+|BD|=|BC|$ (ker je $\mathcal{B}(C,D,B)$), dobimo iskano relacijo.

\item
\res{Dokaži, da sta polmera očrtanih krožnic dveh podobnih trikotnikov sorazmerna z ustreznima stranicama teh dveh trikotnikov.}

Uporabimo transformacijo podobnosti $f$, ki prvi trikotnik preslika v drugega.


 \item \res{V krožnico s središčem $S$ je včrtan štirikotnik $ABCD$. Diagonali tega štirikotnika sta pravokotni in se sekata v točki $E$. Premica, ki poteka skozi točko $E$ in je pravokotna na stranico $AD$, seka stranico $BC$ v točki $M$.}


(\textit{a}) \res{Dokaži, da je točka $M$ središče daljice $BC$.}

Glej zgled \ref{TetivniLemaBrahm}.

(\textit{b}) \res{Določi množico vseh točk $M$, če se diagonala $BD$ spreminja in je vedno pravokotna na diagonalo $AC$.}

Označimo s $k$ očrtano krožnico štirikotnika $ABCD$. Ker je $h_{C,\frac{1}{2}}(B)=M$ in $B\in k$, je $M\in k'=h_{C,\frac{1}{2}}(k)$. Velja tudi obratno - za točko $M\in k'\setminus \{ C\}$ je $B=h_{C,\frac{1}{2}}^{-1}(M)\in k$.  Iskana množica vseh točk $M$ je torej $M\in k'\setminus \{ C\}$. Pri tem je $k'=h_{C,\frac{1}{2}}(k)$ pravzaprav krožnica s premerom $CS$ (izrek \ref{RaztKroznKrozn}).


\item
\res{Naj bo $t$ tangenta očrtane krožnice $l$ trikotnika $ABC$ v oglišču $A$.
Naj bo $D$ takšna točka premice $AC$, da je $BD\parallel t$. Dokaži, da velja
$|AB|^2=|AC|\cdot |AD|$.}

Uporabimo izreka \ref{ObodKotTang} in \ref{KotiTransverzala} ter dokažemo $\triangle ABD \sim \triangle ACB$.

 \item
 \res{Višinska točka ostrokotnega trikotnika naj deli njegovi višini v enakem razmerju (od oglišča do nožišča višine). Dokaži, da gre za enakokraki trikotnik.}

 Naj bosta $BB'$ in $CC'$ višini, $V$ višinska točka trikotnika $ABC$, pri katerem velja $\frac{BV}{VB'}=\frac{CV}{VC'}$ oz. $\frac{BV}{CV}=\frac{VB'}{VC'}$. Ker je $\angle C'VB\cong \angle B'VC$, sta si po izreku \ref{PodTrikKKK} pravokotna trikotnika $C'VB$ in $B'VC$ podobna. Iz tega sledi $\frac{BV}{CV}=\frac{VC'}{VB'}$. Iz prejšnjih dveh relacij sledi $\frac{BV}{CV}=\frac{VC'}{VB'}=\frac{VB'}{VC'}$ oz. $VB'\cong VC'$ in $VB\cong VC$ ter na koncu $BB'\cong CC'$. Iz skladnosti pravokotnih trikotnikov $ABB'$ in $ACC'$ (izrek \textit{ASA} \ref{KSK}) sledi $AB\cong AC$, kar pomeni, da je $ABC$ enakokraki trikotnik.

\item
 \res{V trikotniku $ABC$ se višina $BD$ dotika očrtane krožnice tega trikotnika.
Dokaži:}

(\textit{a}) \res{da je razlika kotov ob osnovnici $AC$ enaka $90^0$}

(\textit{b}) \res{da velja $|BD|^2=|AD|\cdot |CD|$}

Glej nalogo \ref{nalPod23}.

 \item
\res{Krožnica s središčem na osnovnici $BC$ enakokrakega trikotnika $ABC$ se dotika
krakov $AB$ in $AC$. Točki $P$ in $Q$ sta presečišči teh krakov s poljubno
tangento te krožnice. Dokaži, da velja $4\cdot |PB|\cdot |CQ|=|BC|^2$.}

Dokažemo $\triangle BDP\sim\triangle CQD$.


\item
\res{Naj bo $V$ višinska točka ostrokotnega trikotnika $ABC$, točka $V$ središče višine
$AD$, višino $BE$ pa točka $V$ deli v razmerju $3:2$. Izračunaj razmerje, v katerem $V$ deli višino $CF$.}

Rešitev je $10:1$.

\item
\res{Naj bo $S$ zunanja točka krožnice $k$. $P$ in $Q$
sta točki, v katerih se krožnica $k$ dotika svojih tangent iz točke $S$, $X$ in $Y$ pa sta presečišči te krožnice s poljubno premico, ki poteka skozi točko $S$. Dokaži da je $XP:YP=XQ:YQ$.}

Uporabimo podobnost trikotnikov $XPS$ in $PYS$ ter trikotnikov $XQS$ in $QYS$ in dokažemo
 $XP:YP=SX:SP=SX:SQ=XQ:YQ$.

\item
\res{Naj bo $D$  točka, ki leži na stranici $BC$ trikotnika $ABC$. Točki $S_1$ in $S_2$ naj bosta središči očrtanih krožnic trikotnikov $ABD$ in $ACD$. Dokaži, da velja
 $\triangle ABC\sim\triangle AS_1S_2$.}

Daljica $AD$ je skupna tetiva danih krožnic s središčema $S_1$ in $S_2$, zato je premica $S_1S_2$ simetrala kota $AS_1S_2$. Če uporabimo izrek \ref{SredObodKot}, dobimo:
 $\angle AS_1S_2=\frac{1}{2}\angle AS_1D=\angle ABD=\angle ABC$ oz. $\angle AS_1S_2\cong\angle ABC$. Podobno je tudi $\angle AS_2S_1\cong\angle ACB$. Po izreku \ref{PodTrikKKK} je  $\triangle ABC\sim\triangle AS_1S_2$.

\item
\res{Točka $P$ leži na hipotenuzi $BC$ trikotnika  $ABC$. Pravokotnica premice $BC$ v točki $P$ seka premici $AC$ in $AB$ v točkah $Q$ in $R$ ter očrtano krožnico trikotnika $ABC$ v točki $S$. Dokaži, da velja $|PS|^2=|PQ|\cdot |PR|$.}

Dokažemo $\triangle QPC\sim\triangle BPR$, nato uporabimo izrek \ref{izrekVisinski}

\item
\res{Točka $A$  leži na kraku $OP$ pravega kota $POQ$. Naj bodo
$B$, $C$ in $D$ takšne točke kraka $OQ$, da velja $\mathcal{B}(O,B,C)$, $\mathcal{B}(B,C,D)$ in
$OA\cong OB\cong BC\cong CD$. Dokaži, da velja tudi $\triangle ABC\sim\triangle DBA$.}

Označimo $|OA|=a$. Po predpostavki je tudi $|OB|=|BC|=|CD|=a$, po Pitagorovem izreku \ref{PitagorovIzrek} pa $|AB|=a\sqrt{2}$. Torej $\frac{AB}{DB}=\frac{a\sqrt{2}}{2a}=\frac{\sqrt{2}}{2}$ in $\frac{BC}{BA}=\frac{a}{a\sqrt{2}}=\frac{1}{\sqrt{2}}=\frac{\sqrt{2}}{2}$ oz. $\frac{AB}{DB}=\frac{BC}{BA}$. Ker je še $\angle ABC\cong\angle DBA$, je po izreku  \ref{PodTrikSKS} $\triangle ABC\sim\triangle DBA$.

\item \res{Načrtaj trikotnik, če so znani naslednji podatki:}

(\textit{a}) \res{$\alpha$, $\beta$, $R+r$}

Načrtmo poljubni podobni trikotnik $A'B'C'$, nato uporabimo $a:a'=(R+r):(R+r')$ in Talesov izrek \ref{TalesovIzrekDolzine}.

 (\textit{b}) \res{$a$, $b:c$, $t_c-v_c$}

Načrtamo poljubni podobni trikotnik $A'B'C'$, nato uporabimo $a:a'=(t_c-v_c):(t'_c-v'_c)$ in Talesov izrek \ref{TalesovIzrekDolzine}.

 (\textit{c}) \res{$v_a$, $v_b$, $v_c$}

Uporabimo $2\cdot p_{\triangle}=a\cdot v_a=b\cdot v_b=c\cdot v_c$ (izrek \ref{PloscTrik}). Iz tega namreč sledi $a=\frac{2\cdot p_{\triangle}}{v_a}$, $b=\frac{2\cdot p_{\triangle}}{v_b}$ in $c=\frac{2\cdot p_{\triangle}}{v_c}$ oz.:
$$a:b:c=\frac{1}{v_a}=\frac{1}{v_b}=\frac{1}{v_c}.$$
Na ta način lahko načrtamo najprej poljubni trikotnik $A'B'C'$ s stranicami $a'=\frac{x^2}{v_a}$, $b'=\frac{x^2}{b_a}$ in $c'=\frac{x^2}{v_c}$ za poljubno daljico $x$ (uporabimo Talesov izrek \ref{TalesovIzrekDolzine}).

\item
\res{Naj bosta $AB$ in $CD$ osnovnici enakokrakega tangentnega trapeza $ABCD$, $r$ pa naj bo polmer včrtane krožnice. Dokaži, da je $|AB|\cdot |CD|=4r^2$.}

Naj bo $T$ dotikališče včrtane krožnice $k(S,r)$ trapeza $ABCD$ z njegovim krakom $BC$. Ker sta $BS$ in $CS$ simetrali notranjih kotov $ABC$ in $BCD$ tega trapeza, je po izreku \ref{KotiTransverzala}
 $\angle SBC+\angle SCB=\frac{1}{2}\left( \angle ABC+\angle BCD\right)=\frac{1}{2}\cdot 189^0=90^0$. Iz tega sledi $\angle BSC=90^0$ (izrek \ref{VsotKotTrik}).
 Torej je $ST$ višina pravokotnega trikotnika $CSB$ s hipotenuzo $CB$, zato je po izreku \ref{izrekVisinski}
\begin{eqnarray} \label{eqnPodNal37}
|ST|^2= |CT|\cdot |TB|.
 \end{eqnarray}
Naj bosta $E$ in $F$ središči osnovnic  $AB$ in $CD$. Ker gre za enakokraki trapez, je premica $EF$ somernica tega trapeza in sta $E$ in $F$ dotikališči osnovnic $AB$ in $CD$ z včrtano krožnico $k$ tega trapeza. Če uporabimo \ref{eqnPodNal37} in izrek \ref{TangOdsek}, dobimo:
\begin{eqnarray*}
4r^2=4\cdot |ST|^2= 4\cdot |CT|\cdot |TB|=4\cdot |CF|\cdot |BE|=|AB|\cdot |CD|.
 \end{eqnarray*}


%Harmon cetverica

\item \res{Dane so krožnica $k$ ter točki $A$ in $B$. Načrtaj takšno točko $X$ na krožnici $k$, da bo $AX:XB=2:5$.}

Uporabimo Apolonijevo krožnico oz. izrek \ref{ApolonijevaKroznica}.

\item \res{Načrtaj trikotnik z danimi podatki:}\\
(\textit{a}) \res{$a$, $v_a$, $b:c$,} \hspace*{3mm}
 (\textit{b}) \res{$a$, $t_a$, $b:c$,}\hspace*{3mm}
 (\textit{c}) \res{$a$, $b$, $b:c$,}\\
 (\textit{d}) \res{$a$, $\alpha$, $b:c$,}\hspace*{3mm}
  (\textit{e}) \res{$a$, $l_a$, $b:c$.}

Pri vseh primerih uporabimo izrek \ref{HarmCetSimKota} in Apolonijevo krožnico $\mathcal{A}_{BC,c:b}$ - izrek \ref{ApolonijevaKroznica}.

\item \res{Načrtaj trikotnik, če so dani naslednji podatki:}

(Pri vseh primerih uporabimo oznake in rezultate velike naloge \ref{velikaNaloga} in njene posledice \ref{harmVelNal}.)

(\textit{a}) \res{$v_a$, $r$, $\alpha$}

Uporabimo $\mathcal{H}(A,A';L,L_a)$ in tako najprej načrtamo $r_a$.

(\textit{b}) \res{$v_a$, $r_a$, $a$}

Uporabimo $\mathcal{H}(A,A';L,L_a)$ in $RR_a=a$.

(\textit{c}) \res{$v_a$, $t_a$, $b-c$}

Uporabimo $\mathcal{H}(A',E;P,P_a)$, $PP_a=b-c$ in dejstvo, da je $A_1$ središče daljice $PPa$.


\item \res{Načrtaj paralelogram, pri katerem sta ena stranica in ustrezna višina skladni z danima daljicama $a$ in $v_a$, diagonali pa sta v razmerju $3:5$.}

Najprej načrtamo stranico $AB\cong a$ paralelograma $ABCD$, nato pa njegovo središče $S$ kot presečišče Apolonijeve krožnice $\mathcal{A}_{AB,3:5}$ (izrek \ref{ApolonijevaKroznica}) in vzporednice nosilke $AB$ na razdalji $\frac{1}{2}v$.

\item \res{Točka $E$ naj bo presečišče simetrale notranjega kota $BAC$ trikotnika $ABC$ z njegovo stranico $BC$. Dokaži da velja:
    $$\overrightarrow{AE}=\frac{|AC|}{|AB|+|AC|}\cdot\overrightarrow{AB}+
    \frac{|AB|}{|AB|+|AC|}\cdot\overrightarrow{AC}.$$}

Uporabimo zgled \ref{vektDelitDaljice} in izrek \ref{HarmCetSimKota}.

\item
\res{Dane so štiri kolinearne točke, za katere velja $\mathcal{H}(A,B;C,D)$. Načrtaj točko $L$, iz katere se daljice $AC$, $CB$ in $BD$ vidijo pod enakim kotom.}

Naj bo $L$ točka, ki leži na preseku krožnic $k_{AB}$ in $k_{CD}$, ki sta načrtani nad premeroma $AB$ in $CD$. Dokažemo, da velja $\angle ALC\cong\angle CLB\cong\angle BLD$.

Krožnica $k_{CD}$ je pravzaprav Apolonijeva krožnica $\mathcal{A}_{AB,\frac{AC}{CB}}$ (izrek \ref{ApolonijevaKroznica}), zato za točko $L\in k_{CD}$ velja $\frac{LA}{LB}=\frac{CA}{CB}$. Iz tega sledi (izrek \ref{HarmCetSimKota}), da je $LC$ simetrala kota $ALB$, torej $\angle ALC\cong\angle CLB$. Podobno iz $L\in k_{CD}$ sledi
$\angle CLB\cong\angle BLD$, kar pomeni, da je $L$ iskana točka.

\item \label{nalPod44}
\res{Naj bo $AE$ ($E\in BC$) simetrala notranjega kota trikotnika $ABC$ ter $a=|BC|$, $b=|AC|$ in  $c=|AB|$. Dokaži, da velja:
$$|BE|=\frac{ac}{b+c} \hspace*{1mm} \textrm{ in } \hspace*{1mm}  |CE|=\frac{ab}{b+c}.$$}

Po izreku \ref{HarmCetSimKota} je $\frac{BE}{EC}=\frac{BA}{AC}=\frac{c}{b}$, zato je:
 $$|BE|=|BC|\cdot\frac{BE}{BC}=a\cdot\frac{c}{b+c}=\frac{ac}{b+c}.$$
Podobno je tudi $|CE|=\frac{ab}{b+c}$.

\item \res{Naj bosta $AE$ ($E\in BC$) in $BF$ ($F\in AC$) simetrali notranjih kotov ter $S$ središče včrtane krožnice trikotnika $ABC$. Dokaži, da je $ABC$ enakokraki trikotnik (z osnovnico $AB$) natanko tedaj, ko je $AS:SE=BS:SF$.}


Predpostavimo, da velja $AS:SE=BS:SF$.
Premica $BS$ je hkrati simetrala notranjega kota $ABE$ trikotnika $ABE$, zato je po izreku \ref{HarmCetSimKota} $\frac{AS}{SE}=\frac{AB}{BE}$. Ker je še $AS$ simetrala notranjega kota $BAF$ trikotnika $BAF$, je tudi $\frac{BS}{SF}=\frac{BA}{AF}$. Ker je po predpostavki $\frac{AS}{SE}=\frac{BS}{SF}$, velja $BE\cong AF$. Po prejšnji nalogi \res{nalPod44} je:
$|BE|=\frac{ac}{b+c}$ in $|AF|=\frac{bc}{a+c}$. Torej $\frac{ac}{b+c}=\frac{bc}{a+c}$ oz. $a(a+c)=b(b+c)$. Zadnja enakost je ekvivalentna z $a^2-b^2+ac-bc=0$ oz.
 $(a-b)(a+b+c)=0$. Ker je $a+b+c\neq 0$, je $a=b$, kar pomeni, da je $ABC$ enakokraki trikotnik.

Če je $ABC$ enakokraki trikotnik, je relacija $AS:SE=BS:SF$ trivialno izpolnjena.


%Menelaj Ceva


\item \res{Dokaži, da simetrale zunanjih kotov poljubnega trikotnika sekajo nosilke nasprotnih stranic v treh kolinearnih točkah.}

Uporabimo Menelajev izrek \ref{izrekMenelaj}.

%Ppitagorov iizrek

\item \res{Če so $a$, $b$ in $c$ ($a>b$) dane daljice, načrtaj takšno daljico $x$, da velja:}\\
(\textit{a}) \res{$x=\sqrt{a^2+b^2}$,} \hspace*{3mm}
(\textit{b}) \res{$x=\sqrt{a^2-b^2}$,} \hspace*{3mm}
(\textit{c}) \res{$x=\sqrt{3ab}$,}\\
(\textit{d}) \res{$x=\sqrt{a^2+bc}$,} \hspace*{3mm}
(\textit{e}) \res{$x=\sqrt{3ab-c^2}$,} \hspace*{3mm}
(\textit{f}) \res{$x=\frac{a\sqrt{ab+c^2}}{b+c}$.}

Uporabi Pitagorov izrek \ref{PitagorovIzrek} in višinski izrek \ref{izrekVisinski}.

%Stewartov iizrek

\item
\res{Naj bodo $a$, $b$ in $c$ stranice nekega trikotnika in velja $a^2+b^2=5c^2$. Dokaži, da sta težiščnici
 $t_a$ in $t_b$ med seboj pravokotni.}

Označimo s $T$ težišče danega trikotnika $ABC$ ter $t_a$ in $t_b$ dolžini ustreznih težiščnic. Dokažimo, da je $ATB$ pravokotni trikotnik.
Po posledici \ref{StwartTezisc} Stewartovega izreka \ref{StewartIzrek} je:
 \begin{eqnarray*}
t_a^2=\frac{b^2}{2}+\frac{c^2}{2}-\frac{a^2}{4}\\
t_b^2=\frac{a^2}{2}+\frac{c^2}{2}-\frac{b^2}{4}.
 \end{eqnarray*}
Iz tega in iz predpostavke $a^2+b^2=5c^2$ sledi:
\begin{eqnarray*}
|AT|^2+|BT|^2
 &=&
\left(\frac{2}{3}t_a\right)^2+\left(\frac{2}{3}t_b\right)^2=\\
 &=&
\frac{4}{9}\left(\frac{b^2}{2}+\frac{c^2}{2}-\frac{a^2}{4}
+\frac{a^2}{2}+\frac{c^2}{2}-\frac{b^2}{4}\right)=\\
 &=&
\frac{4}{9}\left(\frac{a^2+b^2}{4}+c^2\right)=\\
 &=& c^2=|AB|^2.
 \end{eqnarray*}
Po obratnem Pitagorovem izreku \ref{PitagorovIzrekObrat} je $ATB$ pravokotni trikotnik s hipotenuzo $AB$, kar pomeni, da sta ustrezni težiščnici pravokotni.

\item
\res{Naj bodo $a$, $b$, $c$ in $d$ stranice, $e$ in $f$ diagonali ter $x$ daljica, ki je določena s središčema
 stranic $b$ in $d$ nekega štirikotnika. Dokaži:
$$x^2 = \frac{1}{4} \left(a^2 +c^2 -b^2 -d^2 +e^2 +f^2 \right).$$}

Uporabimo posledico \ref{StwartTezisc} Stewartovega izreka \ref{StewartIzrek}.

\item
\res{Naj bodo $a$, $b$ in $c$ stranice trikotnika $ABC$. Dokaži, da je razdalja središča $A_1$ stranice $a$ od nožišča $A'$ višine na
to stranico enaka:
$$|A_1A'|=\frac{|b^2-c^2|}{2a}.$$}

Brez škode za splošnost predpostavimo, da velja $b\geq c$.
Označimo $x=|A_1A'|$, $v_a=|AA'|$ in $t_a=|AA_1|$. Če uporabimo Pitagorov izrek \ref{PitagorovIzrek} za trikotnika $AA'A_1$ in $AA'B$, dobimo:
 \begin{eqnarray*}
 v_a^2 &=& t_a^2-x^2;\\
 v_a^2 &=& c^2- \left( \frac{a}{2} -x \right)^2.
 \end{eqnarray*}
 Po odštevanju enačb in reševanju dobljene enačbe po $x$ dobimo:
  \begin{eqnarray*}
x=\frac{1}{a}\left( t_a^2-c^2+ \frac{a^2}{2} \right).
 \end{eqnarray*}
Na koncu uporabimo še relacijo za $t_a^2$ iz izreka \ref{StwartTezisc}:
 \begin{eqnarray*}
x &=& \frac{1}{a}\left( t_a^2-c^2+ \frac{a^2}{4} \right)=\\
 &=& \frac{1}{a}\left( \frac{b^2}{2}+\frac{c^2}{2}-\frac{a^2}{4}-c^2+ \frac{a^2}{4} \right)=\\
 &=& \frac{b^2-c^2}{2a}.
 \end{eqnarray*}

%Pappus in Ppascal

\item
\res{Naj bosta ($A$, $B$, $C$) in ($A_1$, $B_1$, $C_1$) trojici kolinearnih točk neke ravnine, ki nista na
isti premici. Če je $AB_1\parallel A_1B$ in $AC_1\parallel A_1C$, tedaj je tudi $CB_1\parallel C_1B$. (\textit{Pappusov izrek}\footnote{Pappus iz Aleksandrije\index{Pappus} (3. st.), starogrški matematik. Gre za posplošitev Pappusovega izreka (glej izrek \ref{izrek Pappus}), če za $X$, $Y$ in $Z$ izberemo točke v neskončnosti.})}

Uporabimo Talesov izrek \ref{TalesovIzrek} in obratni Talesov izrek \ref{TalesovIzrekObr}.

%Desarguesov iizrek


\item \label{nalPodDesarg1}
\res{Naj bodo $P$, $Q$ in $R$ takšne točke stranic $BC$, $AC$ in $AB$
trikotnika $ABC$, da so premice $AP$, $BQ$ in
$CR$ iz istega šopa. Dokaži: Če je $X=BC\cap QR$, $Y=AC\cap PR$ in $Z=AB\cap PQ$, so točke
$X$, $Y$ in $Z$
kolinearne.}

Uporabimo Desarguesov izrek \ref{izrekDesarguesEvkl} za trikotnika $ABC$ in $PQR$.


\item
\res{Naj bodo $AA'$, $BB'$ in $CC'$ višine trikotnika $ABC$ ter $X=B'C'\cap BC$,  $Y=A'C'\cap AC$ in $Z=A'B'\cap AB$. Dokaži, da so $X$, $Y$ in $Z$ kolinearne točke.}

Direktna posledica izreka \ref{VisinskaTocka} in prejšnje naloge \ref{nalPodDesarg1}.

\item
\res{Naj bosta $A$ in $B$ točki izven premice $p$. Načrtaj presečišče premic $p$ in $AB$ brez direktnega risanja premice
$AB$.}

Načrtaj poljubne kolinearne točke $C$, $C'$ in $S$, nato pa: $Y=p\cap AC$, $X=p\cap BC$, $A'=SA\cap C'Y$ in $B'=SB\cap  C'X$. Po Desarguesovem izreku \ref{izrekDesarguesEvkl} se premici $AB$ in $A'B'$ sekata v točki $Z$, ki je kolinearna s točkama $X$ in $Y$, torej leži na premici $p$. To pomeni, da presečišče premic $p$ in $AB$ dobimo kot presečišče premic $p$ in $A'B'$, torej brez direktnega risanja premice
$AB$.


\item
\res{Naj bosta $p$ in $q$ premici neke ravnine, ki se sekata v točki $S$, ki je ‘‘izven papirja’’, in $A$
točka te ravnine. Načrtaj premico, ki poteka skozi točki $A$ in $S$.}

Uporabimo izrek \ref{izrekDesarguesOsNesk}.

\item
\res{Načrtaj trikotnik tako, da njegova oglišča ležijo na treh danih vzporednih premicah, nosilke njegovih stranic pa potekajoo skozi tri dane
točke.}

Uporabimo Desarguesov izrek \ref{izrekDesarguesEvkl}. Glej zgled \ref{zgled 3.2}.


%Ppotenca

\item
\res{Dana je  krožnica $k(S,r)$.}

(\textit{a}) \res{Katere vrednosti vse lahko ima potenca točke glede na krožnico $k$?}

Ker po izreku \ref{izrekPotenca} potenco neke točke $P$ glede na krožnico $k$ dobimo kot: $p(S,k)=|PS|^2-r^2$, se vrednost potence giblje na intervalu $[-r^2,\infty)$.

 (\textit{b}) \res{Katera je najmanjša vrednost te potence in za katero točko se ta minimalna vrednost doseže?}

Iz prejšnjega primera je jasno, da je minimalna vrednost potence $-r^2$ in se doseže za središče $S$ krožnice $k$.

(\textit{c}) \res{Določi množico vseh točk, za katere je potenca glede na krožnico enaka $\lambda\in \mathbb{R}$.}

Pogoja $p(S,k)=|PS|^2-r^2=\lambda$ in $|PS|^2=r^2+\lambda$ sta si ekvivalentna.

V primeru, ko je $\lambda>-r^2$, gre za krožnico $k(S,r^2+\lambda)$. Če je  $\lambda=-r^2$, je iskana množica enaka $\{S \}$, v primeru $\lambda<-r^2$ pa prazni množici $\emptyset$.

\item \res{Naj bo $k_a(S_a,r_a)$ pričrtana in $l(O,R)$ očrtana krožnica nekega trikotnika. Dokaži enakost\footnote{Trditev je posplošitev Eulerjeve formule za krožnico (glej izrek \ref{EulerjevaFormula}). \index{Euler, L.}
        \textit{L. Euler}
        (1707--1783), švicarski matematik.}:
   $$S_aO^2=R^2+2r_aR.$$}

Dokaz je podoben dokazu izreka \ref{EulerjevaFormula}.


\item
\res{Načrtaj krožnico, ki poteka skozi dani točki $A$ in $B$ in se dotika dane krožnice $k$.}

Iskano krožnico označimo z $x$.
Načrtamo poljubno krožnico $j$, ki poteka skozi točki $A$ in $B$, krožnico $k$ pa seka v točkah $C$ in $D$. Uporabimo dejstvo, da je presečišče premic $AB$ in $CD$ potenčno središče krožnic $k$, $j$ in $x$.

Naloga predstavlja enega od desetih Apolonijevih problemov o dotiku krožnic (glej razdelek \ref{odd9ApolDotik}).

\item
\res{Dokaži, da so središča daljic, ki so določena s skupnimi tangentami dveh krožnic, kolinearne točke.}

Omenjene točke ležijo na potenčni premici dveh krožnic.

\item
\res{Načrtaj krožnico, ki je pravokotna na dve dani krožnici, tretjo dano krožnico
pa seka v točkah, ki določata premer te tretje krožnice.}

Označimo dane krožnice po vrsti z $k(K, r_k)$, $j(J,r_j)$ in $l(L,r_l)$, iskano krožnico pa z $x(X,r_x)$.

Naj bo $P\in x\cap k$, $Q\in x\cap l$ in $R, R_1\in x\cap j$.
Po predpostavki je $x\perp k, l$. Po izreku \ref{TangPogoj} sta $XP$ in $XQ$ tangenti krožnic $k$ in $l$ v točkah $P$ in $Q$. Ker je $XP\cong XQ$, je $p(X,k)=p(X,l)$ oz. središče $X$ krožnice $X$ leži na potenčni premici $p(k,l)$ krožnic $k$ in $l$.

Poiščemo še eno geometrijsko mesto točk za $X$, tako da vključimo pogoj za krožnico $j$.
Po predpostavki je $RR_1$ premer krožnice $j$. To pomeni, da je $J$ središče daljice $RR_1$, iz skladnosti trikotnikov $XJR$ in $XJR_1$ (izrek \textit{SSS} \ref{SSS}) pa sledi $\angle XJR=90^0$. Po Pitagorovem izreku \ref{PitagorovIzrek} (za $\triangle XJR$) je potem:
 \begin{eqnarray*}
 |XR|^2=|XJ|^2+r_j^2.
 \end{eqnarray*}
Iz pravokotnega trikotnika $XQL$  po istem izreku dobimo:
 \begin{eqnarray*}
 |XQ|^2=r_l^2-|XL|^2.
 \end{eqnarray*}
Ker je $XR\cong XQ$, iz prejšnjih dveh relacij sledi $|XJ|^2+r_j^2=r_l^2-|XL|^2$ oz.:
\begin{eqnarray*}
 |XJ|^2+|XL|^2=r_l^2-r_j^2.
 \end{eqnarray*}
Torej točka $X$ leži na neki krožnici $g$, ki jo lahko načrtamo (glej izrek \ref{GMTmnl}). To pomeni, da točko $S$ dobimo kot presečišče krožnice $g$ in potenčne premice $p(k,l)$ krožnic $k$ in $l$.



%Razno

\item
\res{Naj bosta $M$ in $N$ presečišči stranic $AB$ in $AC$ trikotnika $ABC$ s premico,
ki poteka skozi središče včrtane krožnice tega trikotnika in je vzporedna z njegovo
stranico $BC$. Izrazi dolžino daljice $MN$ kot funkcijo dolžin stranic trikotnika $ABC$.}

Uporabimo izrek \ref{HarmCetSimKota} in dokažemo $\frac{AS}{SE}=\frac{b+c}{a}$.
Nato uporabimo \ref{TalesovIzrek} in dokažemo $\frac{MN}{BC}=\frac{AS}{AE}$. Rezultat: $|MN|=\frac{a(b+c)}{a+b+c}$.

\item
\res{Naj bo $AA_1$ težiščnica trikotnika $ABC$. Točki $P$ in $Q$ naj bosta presečišči
simetral kotov $AA_1B$ in $AA_1C$ s stranicama $AB$ in $AC$. Dokaži, da je
$PQ\parallel BC$.}

Po izreku \ref{HarmCetSimKota} je:
$$\frac{AP}{PB}=\frac{AA_1}{A_1B}=\frac{AA1}{A_1C}=\frac{AQ}{QC}.$$
Iz obratnega Talesovega izreka \ref{TalesovIzrekObr} sledi $PQ\parallel BC$.

\item
\res{V trikotniku $ABC$ naj bo vsota (ali razlika) notranjih kotov $ABC$ in $ACB$ enaka pravemu
kotu. Dokaži, da je $|AB|^2+|AC|^2=4r^2$, kjer je $r$ polmer očrtane krožnice tega trikotnika.}

Naj bo $\alpha=\angle BAC$, $\beta=\angle ABC$ in $\gamma=\angle ACB$ ter $k(O,r)$ očrtana krožnica trikotnika $ABC$.

Če je $\beta+\gamma=90^0$, je $\alpha=90^0$ (izrek \ref{VsotKotTrik}), kar pomeni, da je $ABC$ pravokotni trikotnik s hipotenuzo $BC$; trditev je v tem primeru trivialna posledica izreka \ref{TalesovIzrKroz2} in Pitagorovega izreka \ref{PitagorovIzrek}.

Naj bo $\beta-\gamma=90^0$ oz. $\gamma=\beta-90^0$. Označimo $C'=\mathcal{S}_O(C)$. Po izreku  \ref{TalesovIzrKroz2} je najprej $\angle CAC'=90^0$. Ker je $BCC'A$ tetivni štirikotnik, je po izreku \ref{TetivniPogoj}:
 \begin{eqnarray*}
\angle C'CA &=& 90^0-\angle AC'C=\\
            &=& 90^0-(180^0-\beta)=\beta-90^0=\\
            &=& \gamma =\angle BCA.
 \end{eqnarray*}

Po izreku \ref{ObodObodKot} je $AC'\cong AB$, zato iz Pitagorovega izreka \ref{PitagorovIzrek} dobimo:
 \begin{eqnarray*}
|AB|^2+|AC|^2=|AC'|^2+|AC|^2=|C'C|^2=4r^2.
 \end{eqnarray*}

\item
\res{Naj bo $AD$ višina trikotnika $ABC$. Dokaži, da je
vsota (ali razlika) notranjih kotov $ABC$ in $ACB$ enaka pravemu kotu natanko tedaj, ko je:
$$\frac{1}{|AB|^2}+\frac{1}{|AC|^2}=\frac{1}{|AD|^2}.$$}

Obravnavamo trikotnika $ADB$ in $CDA$.

\item
\res{Izrazi razdaljo med težiščem in središčem očrtane krožnice trikotnika kot
funkcijo dolžin njegovih stranic in polmera očrtane krožnice.}

Uporabimo Stewartov izrek \ref{StewartIzrek} za trikotnik $OAA_1$ in izrek \ref{StwartTezisc}; $k(O,R)$ je očrtana krožnica, $T$ težišče in $A_1$ središče stranice $BC$ trikotnika $ABC$.

 Rezultat:
$|OT|=\sqrt{R^2-\frac{1}{9} \left(a^2+b^2+c^2 \right)}$.

\item
\res{Dokaži, da pri trikotniku $ABC$ simetrala zunanjega kota ob oglišču $A$ in simetrali
notranjih kotov ob ogliščih $B$ in $C$ sekajo nosilke nasprotnih stranic v treh kolinearnih točkah.}

Uporabimo Menelajev izrek \ref{izrekMenelaj} in izrek \ref{HarmCetSimKota}.

\item
\res{Dokaži, da so pri trikotniku $ABC$ središče višine $AD$, središče včrtane krožnice in točka, v kateri se stranica $BC$ dotika pričrtane krožnice tega trikotnika, tri
kolinearne točke.}

Uporabimo izrek \ref{velNalTockP'}.

\item
\res{Dokaži Simsonov izrek \ref{SimpsPrem} z uporabo Menelajevega izreka \ref{izrekMenelaj}.}

Naj bo $S$ poljubna točka očrtane krožnice trikotnika $ABC$ ter $P$, $Q$ in $R$ pravokotne projekcije te točke na nosilkah $BC$, $AC$ in $BC$. Uporabi:
$\triangle SRA\sim\triangle SPC$, $\triangle SPB\sim\triangle SQA$ in $\triangle SQC\sim\triangle SRB$.


\item
\res{Skozi točko $M$ stranice $AB$ trikotnika $ABC$ je konstruirana premica, ki seka
premico $AC$ v točki $K$. Izračunaj razmerje, v katerem premica $MK$ deli stranico $BC$,
če je $AM:MB=1:2$ in $AK:AC=3:2$.}

Naj bo $P$ presečišče premic $MK$ in $BC$. Po Menelajevem izreku je
$$-1=\frac{\overrightarrow{BP}}{\overrightarrow{PC}}\cdot
\frac{\overrightarrow{CK}}{\overrightarrow{KA}}\cdot
\frac{\overrightarrow{AM}}{\overrightarrow{MB}}=
\frac{\overrightarrow{BP}}{\overrightarrow{PC}}\cdot
\frac{-1}{3}\cdot\frac{1}{2},$$ zato je $\frac{\overrightarrow{BP}}{\overrightarrow{PC}}=6:1$.

\item
\res{Naj bo $A_1$ središče stranice $BC$ trikotnika $ABC$ in naj bosta $P$ in $Q$ takšni točki stranic
$AB$ in $AC$, da velja $BP:PA=2:5$ in $AQ:QC=6:1$. Izračunaj razmerje, v katerem premica  $PQ$ deli težiščnico $AA_1$.}

Naj bo $R$ presečišče premic $PQ$ in $BC$. Uporabimo Menelajev izrek \ref{izrekMenelaj} najprej za trikotnik $ABC$ in premico $PQ$, nato pa za trikotnik $AA_1C$ in isto premico. Rezultat: $17:60$.

\item
\res{Dokaži, da se pri poljubnem trikotniku premice, ki so določene z oglišči in
dotikališči ene pričrtane krožnice z nosilkami nasprotnih stranic, sekajo v skupni točki.}

Uporabimo Cevov izrek \ref{izrekCeva} in veliko nalogo \ref{velikaNaloga}.

\item
\res{Kaj predstavlja množica vseh točk, iz katerih odseka tangent glede na dve dani krožnici predstavljata dve skladni daljici?}

Ker je v tem primeru potenca iz teh točk glede na krožnici enaka, je iskana množica del njune potenčne premice, ki je v zunanjosti obeh krožnic.

\item
\res{Naj bosta $PP_1$ in $QQ_1$ zunanji tangenti krožnic $k(O,r)$ in $k_1(O_1,r_1)$ (točke $P$, $P_1$, $Q$ in $Q_1$ so ustrezna dotikališča). Točka $S$ naj bo presečišče teh dveh tangent, $A$ eno od presečišč krožnic $k$ in $k_1$ ter $L$ in $L_1$ presečišči
premice $SO$ s premicama $PQ$ in $P_1Q_1$. Dokaži, da velja $\angle LAO\cong\angle L_1AO_1$.}

Brez škode za splošnost naj bo $\mathcal{B}(S,O,O_1)$.
Naj bo še $h_{S,\lambda}$ središčni razteg s koeficientom $\lambda=\frac{\overrightarrow{SO_1}}{\overrightarrow{SO}}$. Potem je $h_{S,\lambda}(k)=k_1$ in:
 $$h_{S,\lambda}:\hspace*{1mm} O,P,Q,L \mapsto O_1,P_1,Q_1,L_1.$$
Naj bo še $A_1=h_{S,\lambda}(A)$. Iz $h_{S,\lambda}(k)=k_1$ in $A\in k$ sledi
 $A_1\in k_1$. Po izreku \ref{homotOhranjaKote} je $\angle L_1A_1O_1\cong\angle LAO$, torej je dovolj dokazati $\angle L_1A_1O_1\cong\angle L_1AO_1$.

Iz podobnosti pravokotnih trikotnikov $SP_1O_1$ in $SL_1P_1$ (izrek \ref{PodTrikKKK}) in izreka \ref{izrekPotenca} sledi:
 \begin{eqnarray*}
\overrightarrow{SO_1}\cdot \overrightarrow{SL_1}=|SP_1|^2=p(S.k_1)=
\overrightarrow{SA}\cdot \overrightarrow{SA_1}.
 \end{eqnarray*}
To pomeni, da je $O_1L_1AA_1$ tetivni štirikotnik, zato je $\angle L_1A_1O_1\cong\angle L_1AO_1$.


\item
\res{Dokaži, da je stranica pravilnega desetkotnika enaka
večjemu delu delitve polmera očrtane krožnice tega  desetkotnika v razmerju zlatega reza.}

Naj bo $k(S,r)$ središče očrtane krožnice in $AB$ ($a=|AB|$) ena stranica pravilnega desetkotnika. V trikotniku $ASB$ merijo notranji koti:
 $\angle ASB=36^0$ in $\angle SAB=\angle SBA=72^0$.

Označimo s $P$ takšno točko, da velja $\mathcal{B}(B,A,P)$ in $AP\cong AS$. Ker je $SAP$ enakokraki trikotnik z osnovnico $PS$, je po izreku \ref{enakokraki} $\angle SPA\cong\angle PSA$. Iz izreka \ref{zunanjiNotrNotr} (za trikotnik $SAP$) sledi
 $\angle APS=\angle PSA=\frac{1}{2}72^0=36^0$. Torej velja $\triangle SPB\sim\triangle ASB$ (izrek \ref{PodTrikKKK}), zato je $\frac{PB}{SB}=\frac{SB}{AB}$ oz. $\frac{a+r}{r}=\frac{r}{a}$, iz tega pa direktno sledi naša trditev.


\item \label{nalPod75}
\res{Naj bodo $a_5$, $a_6$ in $a_{10}$ stranice pravilnega petkotnika, šestkotnika in desetkotnika, ki so včrtani isti krožnici. Dokaži, da velja:
 $$a_5^2=a_6^2+a_{10}^2.$$}

Uporabimo podobnost trikotnikov iz prejšnje naloge \ref{nalPod75}, pri tem pa upoštevamo ustrezni višini teh dveh trikotnikov.

\item
 \res{Naj bodo $t_a$, $t_b$ in $t_c$ težiščnice in $s$ polobseg nekega trikotnika. Dokaži, da velja:
    $$t_a^2+t_b^2+t_c^2\geq s^2.$$} % zvezek - dodatni MG

Če uporabimo izrek \ref{StwartTezisc2}:
 \begin{eqnarray*}
t_a^2+t_b^2+t_c^2 &=& \frac{3}{4}\left(a^2+ b^2+c^2\right)=\\
  &\geq& \frac{3}{4}\cdot 3\cdot\left(\frac{a+ b+c}{3}\right)^2=\\
 &=& s^2.
 \end{eqnarray*}

\end{enumerate}





%REŠITVE -  Ploščina
%________________________________________________________________________________

\poglavje{Area of Figures}


\begin{enumerate}

% Ttrikotniki

\item \res{Naj bo $T$ težišče trikotnika $ABC$. Dokaži:
   $$p_{TBC}=p_{ATC}=p_{ABT}.$$}

Označimo z $A_1$ središče stranice $BC$. Potem je po izreku \ref{PloscTrik}: $p_{AA_1B}=p_{AA_1C}$ in $p_{TA_1B}=p_{TA_1C}$. Če odštejemo enakosti, dobimo $p_{AA_1B}-p_{TA_1B}$ in  $p_{AA_1C}-p_{TA_1C}$ oz. po izreku \ref{ploscGlavniIzrek} \textit{4)}: $p_{ABT}=p_{ATC}$. Analogno dobimo tudi $p_{TBC}=p_{ATC}$.


\item \res{Naj bodo $c$ hipotenuza, $v_c$ pripadajoča višina ter $a$ in $b$ kateti pravokotnega trikotnika. Dokaži, da velja $c+v_c>a+b$.} % zvezek - dodatni MG

Po izreku \ref{PloscTrik} za ploščino $p$ tega trikotnika velja $p=\frac{1}{2}ab=\frac{1}{2}cv_c$, zato je $ab=cv_c$, če pa še uporabimo Pitagorov izrek \ref{PitagorovIzrek}:
 \begin{eqnarray*}
  (a+b)^2=a^2+2ab+b^2=
c^2+2cv_c<c^2+2cv_c+v_c^2=(c+v_c)^2.
 \end{eqnarray*}


  \item \res{Načrtaj kvadrat, ki ima enako ploščino kot pravokotni trikotnik s katetama $a$ in $b$.} % (Hipokratovi luni)

Če je $x$ stranica iskanega kvadrata, je $x^2=\frac{1}{2}ab$. Torej lahko stranico $x$ konstruiramo po višinskem izreku \ref{izrekVisinski} iz relacije $x=\sqrt{\frac{a}{2}\cdot b}$.


\item \res{Naj bo $ABC$ enakokraki pravokotni trikotnik z dolžinama katet $|CB|=|CA|=a$. Označimo z $A_1$, $B_1$, $C_1$ in $C_2$ točke, za katere velja: $\overrightarrow{CA_1}=\frac{n-1}{n}\cdot \overrightarrow{CA}$, $\overrightarrow{CB_1}=\frac{n-1}{n}\cdot \overrightarrow{CB}$, $\overrightarrow{AC_1}=\frac{n-1}{n}\cdot \overrightarrow{AB}$ in  $\overrightarrow{AC_2}=\frac{n-2}{n}\cdot \overrightarrow{AB}$ za nek $n\in \mathbb{N}$. Izrazi ploščino štirikotnika, ki ga določajo premice $AB$, $A_1B_1$, $CC_1$ in $CC_2$ kot funkcijo $a$ in $n$.}

Označimo z $X$ in $Y$ presečišči daljic $CC_2$ in $CC_1$ z daljico $A_1B_1$. Potem je (izreka \ref{PloscTrik} in \ref{ploscGlavniIzrek} \textit{4)}):
  \begin{eqnarray*}
  p_{C_1C_2XY}&=&p_{CC_1C_2}-p_{CXY}=\frac{1}{n}p_{ABC}-\frac{1}{n}p_{A_1B_1C}=\\
&=& \frac{1}{n}\cdot \frac{1}{2}\cdot a^2-\frac{1}{n}\cdot \frac{1}{2}\cdot
\left(\frac{n-1}{n}\cdot a \right)^2=\frac{2n-1}{2n^3}\cdot a^2.
 \end{eqnarray*}

\item \res{Dan je pravokotni trikotnik $ABC$ s hipotenuzo $AB$ in ploščino $p$. Naj bo $C'=\mathcal{S}_{AB}(C)$,  $B'=\mathcal{S}_{AC}(B)$ in  $A'=\mathcal{S}_{BC}(A)$. Izrazi ploščino trikotnika $A'B'C'$ kot funkcijo ploščine $p$.}


Iz  $C'=\mathcal{S}_{AB}(C)$ sledi $\triangle ACB \cong \triangle AC'B$, iz $B'=\mathcal{S}_{AC}(B)$ in  $A'=\mathcal{S}_{BC}(A)$ pa
  $\mathcal{S}_{C}:\hspace*{1mm}A,B,C\mapsto A',B',C$ oz. $\triangle ACB \cong \triangle A'CB'$. Iz tega sledi $|A'B'|=|AB|=c$, višina trikotnika $A'B'C'$ iz oglišča $C'$ pa je enaka trikratni višini $v_c$ trikotnika $ABC$ iz oglišča $C$. Po izreku \ref{PloscTrik} je:
 $$p_{A'B'C'}=\frac{1}{2}\cdot c\cdot 3v_c=3\cdot\frac{1}{2}\cdot cv_c=3p_{ABC}.$$


\item \res{Naj bosta $R$ in $Q$ točki, v katerih se trikotniku $ABC$ včrtana krožnica dotika njegovih stranic $AB$ in $AC$. Simetrala notranjega kota $ABC$ naj seka premico $QR$ v točki $L$. Določi razmerje ploščin trikotnikov $ABC$ in $ABL$.}
  % pripremni zadaci - naloga 193

Označimo z $\alpha$, $\beta$ in $\gamma$ notranje kote ob ogliščih $A$, $B$ in $C$, z $A_1$ središče starnice $BC$ ter s $S$ središče včrtane krožnice trikotnika $ABC$. Kot $BRQ$ je zunanji kot enakokrakega trikotnika $ARQ$, zato je $\angle BRE =\angle BRQ = 90^0+\frac{1}{2}\alpha$ (izrek \ref{VsotKotTrik}). Ker je po predpostavki še $\angle EBR =\angle SBA = \frac{1}{2}\beta$, po izreku \ref{VsotKotTrik}) za trikotnik $BRE$ velja:
\begin{eqnarray*}
 \angle QES &=& \angle REB=\\
  &=& 180^0-(90^0+\frac{1}{2}\alpha+\frac{1}{2}\beta)=\\
 &=&  \frac{1}{2}\gamma = \angle ACS= \angle QCS.
 \end{eqnarray*}
 Torej $\angle QES \cong \angle QCS$, zato je po izreku \ref{ObodKotGMT} $SCEQ$ tetivni štirikotnik. To pomeni (izrek \ref{ObodObodKot}), da velja:
 $$\angle BEC=\angle SEC=\angle SQC=90^0.$$
 Torej je $EA_1$ težiščnica pravokotnega trikotnika $BEC$ s hipotenuzo $BC$, zato je po izreku \ref{TalesovIzrKroz2} $|EA_1|=\frac{1}{2}|BC|=|BA_1|$. Iz tega sledi, da je $BA_1E$ enakokraki trikotnik z osnovnico $BC$ oz.
$\angle A_1EB \cong\angle EBA_1\cong \angle ABE$ (izrek \ref{enakokraki}) oz. $A_1E\parallel BA$ (izrek \ref{KotiTransverzala}). Iz slednjega sledi (izrek \ref{PloscTrik}):
 $$p_{AEB}=p_{AA_1B}=\frac{1}{2}\cdot p_{ABC}.$$


\item \res{Določi točko v notranjosti trikotnika $ABC$, za katero je produkt njenih razdalj od stranic tega trikotnika maksimalen.} % zvezek - dodatni MG

Označimo z $a$, $b$ in $c$ dolžine stranic $BC$, $AC$ in $AB$ ter z $x$, $y$ in $z$ razdalje poljubne točke $P$ v notranjosti trikotnika $ABC$ od njegovih stranic  $BC$, $AC$ in $AB$. Ker so $a$, $b$ in $c$ konstante, se maksimum produkta $xyz$ doseže natanko tedaj, ko se doseže maksimum produkta $ax\cdot by\cdot cz$. Če uporabimo znano neenakost geometrijske in aritmetične sredine, dobimo:
 \begin{eqnarray*}
 ax\cdot by\cdot cz\leq \left( \frac{ax+by+cz}{3}\right)^3=\left( \frac{2p_{ABC}}{3}\right)^3.
 \end{eqnarray*}
Pri tem se enakost doseže (prav tako maksimum produkta $ax\cdot by\cdot cz$, ker je izraz $\left( \frac{2p_{ABC}}{3}\right)^3$ konstanta) natanko tedaj, ko je $ax=by=cz$. V tem primeru je:
 \begin{eqnarray*}
 \left(ax\right)^3= \left( \frac{2p_{ABC}}{3}\right)^3=\left( \frac{a\cdot v_a}{3}\right)^3,
 \end{eqnarray*}
oz. $x=\frac{v_a}{3}$. Analogno je tudi $y=\frac{v_b}{3}$ in $z=\frac{v_c}{3}$, kar pomeni, da se maksimum produkta $ax\cdot by\cdot cz$ oz. $xyz$ doseže natanko tedaj, ko je točka $P$ težišče trikotnika $ABC$.

\item \res{Trikotniku
             $ABC$ s stranicami $a$, $b$ in $c$ je včrtana krožnica. Načrtane so
             tangente te krožnice, ki so vzporedne  stranicam
             trikotnika.
             Vsaka od tangent v notranjosti trikotnika določa ustrezne daljice dolžin $a_1$, $b_1$ in $c_1$. Dokaži, da velja:
 $$\frac{a_1}{a}+\frac{b_1}{b}+\frac{b_1}{b}=1.$$} % zvezek - dodatni MG

Naj bodo: $r$ polmer včrtane krožnice, $s$ polobseg ter $v_a$, $v_b$ in $v_c$ pripadajoče višine trikotnika $ABC$. Če uporabimo Talesov izrek \ref{TalesovIzrek}  ter izreka \ref{PloscTrik} in \ref{PloscTrikVcrt}, dobimo:
  \begin{eqnarray*}
 \frac{a_1}{a}+\frac{b_1}{b}+\frac{b_1}{b}&=&  \frac{v_a-2r}{v_a}+\frac{v_b-2r}{v_b}+\frac{v_c-2r}{v_b}=\\
&=& 3-\left(\frac{2r}{v_a}+\frac{2r}{v_b}+\frac{2r}{v_b}\right)=\\
&=& 3-\left(\frac{ar}{p_{ABC}}+\frac{br}{p_{ABC}}+\frac{cr}{p_{ABC}}\right)=\\
&=& 3-\frac{2sr}{p_{ABC}}=\\
&=& 3-2=1.
 \end{eqnarray*}

\item \res{Naj bosta $T$ in $S$ težišče in središče včrtane krožnice trikotnika $ABC$. Naj bo tudi $|AB|+|AC|=2\cdot |BC|$. Dokaži, da je $ST\parallel BC$.} % zvezek - dodatni MG


Naj bodo: $r$ polmer včrtane krožnice, $s$ polobseg, $a$, $b$ in $c$ dolžine stranic ter $v_a$ pripadajoča višina trikotnika $ABC$. Če uporabimo dani pogoj $b+c=2a$ ter izreka \ref{PloscTrik} in \ref{PloscTrikVcrt}, dobimo:
  \begin{eqnarray*}
 r=\frac{p_{ABC}}{s}=\frac{av_a}{3a}=\frac{v_a}{3}.
 \end{eqnarray*}
To pomeni, da točki $S$ in $T$ obe ležita na premici $h_{A,\frac{2}{3}}(BC)$, zato sta premici $ST$ in $BC$ vzporedni.

\item \res{Naj bo $p$ ploščina trikotnika $ABC$, $R$ polmer očrtane krožnice in $s'$ obseg pedalnega trikotnika. Dokaži, da velja $p=Rs'$.} %Lopandic - nal 918

En dokaz te trditve je podan v izreku \ref{ploscTrikPedalni}. Na tem mestu bomo izpeljali še drugo varianto dokaza.

Označimo z $A'$, $B'$ in $C'$ nožišča višin iz oglišč $A$, $B$ in $C$, z $A_1$, $B_1$ in $C_1$ središča stranic $BC$, $AC$ in $AB$ ter z $O$ središče očrtane krožnice s polmerom $R$ trikotnika $ABC$. Ker je $\angle BB'C=\angle CC'B=90^0$, je
štirikotnik $BCC'B'$ tetiven (Talesov izrek \ref{TalesovIzrKroz2}), zato je  $\angle B'C'A=\angle BCB'=\angle BCA=\gamma$ (izrek \ref{TetivniPogojZunanji}). Iz tega po izreku \ref{PodTrikKKK} sledi $\triangle ABC \sim \triangle AB'C'$. Torej velja:
 \begin{eqnarray} \label{nalPlo1}
BC:B'C'=AB:AB'.
 \end{eqnarray}
Toda velja tudi $\angle BOA_1=\frac{1}{2}\angle BOC=\angle BAC=\alpha$ (izrek \ref{SredObodKot}), zato je
$\triangle ABB' \sim \triangle OBA_1$ (izrek \ref{PodTrikKKK}) oz.:
 \begin{eqnarray} \label{nalPlo2}
AB:AB'=OA:OA_1=R:OA_1.
 \end{eqnarray}
Iz relacij \ref{nalPlo1} in \ref{nalPlo2} sledi:
 \begin{eqnarray} \label{nalPlo3}
 |B'C'|=\frac{|BC|\cdot |OA_1|}{R}.
 \end{eqnarray}
Analogno je tudi;
 \begin{eqnarray} \label{nalPlo4}
 |A'B'|=\frac{|AB|\cdot |OC_1|}{R} \hspace*{1mm} \textrm{ in }
 \hspace*{1mm} |A'C'|=\frac{|AC|\cdot |OB_1|}{R}.
 \end{eqnarray}
Če na koncu uporabimo relacije \ref{nalPlo3} in \ref{nalPlo4} ter izreka \ref{PloscTrik} in \ref{ploscGlavniIzrek} \textit{4)}, dobimo:
\begin{eqnarray*}
 s'&=& \frac{1}{2}\cdot \left(|A'B'|+|B'C'|+|A'C'|\right)=\\
&=&\frac{1}{R} \cdot
\left(\frac{|AB|\cdot |OC_1|}{2}+\frac{|BC|\cdot |OA_1|}{2} +\frac{|AC|\cdot |OB_1|}{2} \right) =\\
&=&\frac{1}{R} \cdot
\left(p_{OAB}+p_{OBC} +p_{OAC} \right) =\\
&=& \frac{1}{R} \cdot p_{ABC},
 \end{eqnarray*}
 oz. $p=Rs'$.
% Sstirikotniki


\item \res{Naj bo $L$ poljubna točka v notranjosti paralelograma $ABCD$. Dokaži, da velja:
        $$p_{LAB}+p_{LCD}=p_{LBC}+p_{LAD}.$$} %Lopandic - nal 890
Naj bosta $a$ in $b$ dolžini stranic ter $v_a$ in $v_b$ pripadajoči višini paralelograma $ABCD$. Označimo še $v_{a_1}$ in $v_{a_2}$ višini trikotnikov $LAB$ in $LCD$ iz oglišča $L$ ter z $v_{b_1}$ in $v_{b_2}$ višini trikotnikov $LBC$ in $LAD$ iz oglišča $L$. Ker je $v_{a_1}+v_{a_2}=v_a$ in  $v_{b_1}+v_{b_2}=v_b$, po izrekih \ref{PloscTrik}, \ref{ploscParal} in \ref{ploscGlavniIzrek} \textit{4)} velja:
\begin{eqnarray*}
&& p_{LAB}+p_{LCD}=\frac{av_{a_1}}{2}+\frac{av_{a_2}}{2}=\frac{1}{2}av_a=\frac{1}{2} p_{ABCD}\\
&& p_{LBC}+p_{LAD}=\frac{av_{b_1}}{2}+\frac{av_{b_2}}{2}=\frac{1}{2}bv_b=\frac{1}{2} p_{ABCD}.
 \end{eqnarray*}

\item \res{Naj bo $P$ središče kraka $BC$ trapeza $ABCD$. Dokaži:
 $$p_{APD}=\frac{1}{2}\cdot p_{ABCD}.$$}

Naj bo $v$ višina trapeza, $Q$ središče kraka $AD$ oz. $|PQ|=m$  srednjica trapeza $ABCD$.
Po izrekih \ref{PloscTrik}, \ref{PloscTrik}, \ref{ploscTrapez} in \ref{ploscGlavniIzrek} \textit{4)} velja:

\begin{eqnarray*}
 p_{APD}=p_{APQ}+p_{QPD}=\frac{1}{2}m\frac{v}{2}+\frac{1}{2}m\frac{v}{2}=
 \frac{1}{2}mv=\frac{1}{2} p_{ABCD}.
 \end{eqnarray*}

\item \res{Naj bodo $o$ obseg, $v$ višina in $p$ ploščina tangentnega trapeza. Dokaži, da velja: $p=\frac{o\cdot v}{4}$.}

Uporabimo izrek \ref{ploscTetVec}.

\item \res{Načrtaj premico $p$, ki poteka skozi oglišče $D$ trapeza $ABCD$ ($AB>CD$), tako da razdeli ta trapez na dva ploščinsko enaka lika.}

Naj bodo: $v$ višina trapeza ter $a$ in $c$ dolžini njegovih osnovnic $AB$ in $CD$.
Če je $X$ točka, v kateri premica $p$ seka osnovnico $AB$, iz danega pogoja velja:
 $p_{AXD}=\frac{1}{2}p_{ABCD}$ oz.:
\begin{eqnarray*}
 \frac{|AX|\cdot v}{2} =\frac{1}{2}\cdot \frac{a+c}{2}\cdot v.
 \end{eqnarray*}

Torej je daljica $AX$ enaka srednjici trapeza oz. $|AX|=\frac{a+c}{2}$, kar omogoča konstrukcijo točke $X$ in premice $p$.


\item \res{Načrtaj premici $p$ in $q$, ki potekata skozi oglišče $D$ kvadrata $ABCD$ in ga razdelita na ploščinsko enake like.}

Če sta $P$ in $Q$ točki, v katerih premica $p$ oz. $q$ seka stranico $AB$ oz. $BC$ kvadrata $ABCD$, dokažemo, da je $AP:PB=2:1$ in $BQ:QC=1:2$.

\item \res{Naj bo $ABCD$ kvadrat, $E$ središče njegove stranice $BC$ in $F$ točka, za katero je $\overrightarrow{AF}=\frac{1}{3}\cdot \overrightarrow{AB}$. Točka $G$ je četrto oglišče pravokotnika $FBEG$. Kolikšen del ploščine kvadrata $ABCD$ predstavlja ploščina trikotnika $BDG$?}

Označimo z $a$ dolžino stranice kvadrata $ABCD$.
Naj bo $S$ središče tega kvadrata in $H=\mathcal{T}_{\overrightarrow{AF}}(D)$. Po izrekih \ref{PloscTrik}, \ref{ploscKvadr} in \ref{ploscGlavniIzrek} \textit{4)} je:
\begin{eqnarray*}
 p_{BDG} &=& p_{BSG}+p_{SDG}= p_{FSG}+p_{SHG}=\\
 &=& p_{FSH}=\frac{1}{2}\cdot |FH| \cdot |GS|= \\
 &=& \frac{1}{2}\cdot a \cdot \left( \frac{a}{2}-\frac{a}{3} \right)=\\
 &=& \frac{a^2}{12} = \frac{1}{12}\cdot p_{ABCD}.
 \end{eqnarray*}


\item \label{nalPloKoef}
\res{Naj bosta $\mathcal{V}$ in $\mathcal{V}'$ podobna večkotnika s koeficientom podobnosti $k$. Dokaži, da je:
 $$p_{\mathcal{V}'}=k^2\cdot p_{\mathcal{V}}.$$}

najprej dokažemo, da trditev velja za trikotnike (uporabimo izrek \ref{PloscTrik}), nato pa uporabimo izrek \ref{ploscGlavniIzrek} \textit{4)}.


\item \label{nalPloHeronStirik}
 \res{Naj bodo $a$, $b$, $c$ in $d$ dolžine stranic, $s$ polobseg in $p$ ploščina poljubnega štirikotnika. Dokaži, da velja:
        $$p=\sqrt{(s-a)(s-b)(s-c)(s-d)}.$$} %Lopandic - nal 924

Označimo z $A$, $B$, $C$ in $D$ oglišča danega štirikotnika, tako da je: $|AB|=a$, $|BC|=b$, $|CD|=c$ in $|DA|=d$. Če je $ABCD$ paralelogram, je zaradi njegove tetivnosti pravokotnik (izreka \ref{paralelogram} in \ref{TetivniPogoj}) in je trditev trivialna.

Brez škode za splošnost torej predpostavimo, da se nosilki $BC$ in $AD$ sekata v neki točki $P$ in pri tem velja še $a>c$. Označimo še $|PC|=x$, $|PD|=y$ in $p'=p_{PCD}$.
Po izreku \ref{TetivniPogojZunanji} je $\angle PCD\cong \angle BAD$, zato je $\triangle CDP \cong \triangle ABP$ (izrek \ref{PodTrikKKK}). Koeficient podobnosti teh dveh trikotnikov je $k=\frac{AB}{CD}=\frac{a}{c}$. Torej po trditvi iz prejšnje naloge \ref{nalPloKoef} velja:
 $$\frac{p_{ABP}}{p_{CDP}}=\frac{a^2}{c^2}$$
oz. po izreku \ref{ploscGlavniIzrek} \textit{4)}
$$\frac{p_{ABCD}+p_{CDP}}{p_{CDP}}=\frac{a^2}{c^2}.$$
Iz tega dobimo:
\begin{eqnarray} \label{nalPloEqnHeronStirik1}
 p=\frac{a^2-c^2}{c^2}\cdot p'.
 \end{eqnarray}
Izračunajmo najprej $p'=p_{PCD}$. Iz omenjene podobnosti $\triangle CDP \cong \triangle ABP$ sledi še:
\begin{eqnarray*}
 \frac{x+b}{y}=\frac{y+d}{x}=\frac{a}{c}.
 \end{eqnarray*}
 Če preoblikujemo enakosti, dobimo sistem enačb za $x$ in $y$:
 \begin{eqnarray*}
  && ax-cy=cd\\
 && cx-ay=-bc
 \end{eqnarray*}
in njegove rešitve:
\begin{eqnarray*}
  x &=& \frac{c(ad+bc)}{a^2-c^2}\\
 y &=& \frac{c(ab+cd)}{a^2-c^2}
 \end{eqnarray*}
Iz tega dobimo:
\begin{eqnarray} \label{nalPloEqnHeronStirik2}
  x+y &=& \frac{c(b+d)}{a+c}\\
 x-y &=& \frac{c(d-b)}{a-c}
 \end{eqnarray}
Naj bo $s'$ polobseg trikotnika $PCD$. Z uporabo relacij \ref{nalPloEqnHeronStirik2} dobimo:
 \begin{eqnarray*}
 s' &=& \frac{x+y+c}{2}=\frac{c}{a-c}\cdot (s-c)\\
 s'-c &=& \frac{x+y-c}{2}=\frac{c}{a-c}\cdot (s-a)\\
 s'-x &=& \frac{c+x+y}{2}=\frac{c}{a+c}\cdot (s-d)\\
 s'-y &=& \frac{c+x+y}{2}=\frac{c}{a+c}\cdot (s-b)
 \end{eqnarray*}
Iz prejšnjih relacij po Heronovi formuli za ploščino trikotnika (izrek \ref{PloscTrikHeron}) velja:
\begin{eqnarray*}
 p' &=& p_{PCD}=\sqrt{s'(s'-c)(s'-x)(s'-y)}=\\
 &=& \frac{c^2}{a^2-c^2}\sqrt{(s-a)(s-b)(s-c)(s-d)}.
 \end{eqnarray*}
Če to uvrstimo v relacijo \ref{nalPloEqnHeronStirik1}, dobimo:
\begin{eqnarray*}
 p = \frac{a^2-c^2}{c^2}\cdot p'=\sqrt{(s-a)(s-b)(s-c)(s-d)}.
 \end{eqnarray*}


\item \res{Naj bodo $a$, $b$, $c$ in $d$ dolžine stranic in $p$ ploščina tetivnotangentnega štirikotnika. Dokaži, da velja:
        $$p=\sqrt{abcd}.$$} %Lopandic - nal 925

Naj bo $s$ polobseg danega štirikotnika.
Ker gre za tangentni štirikotnik, za njegove stranice po izreku \ref{TangentniPogoj} velja $a+c=b+d$. Zato je:
\begin{eqnarray*}
 s-a &=& \frac{b+c+d-a}{2}=c\\
 s-b &=& \frac{a+c+d-b}{2}=d\\
 s-c &=& \frac{a+b+d-c}{2}=a\\
 s-d &=& \frac{a+b+c-d}{2}=b.
 \end{eqnarray*}
Ker je štirikotnik še tetiven, lahko uporabimo trditev iz prejšnje naloge \ref{nalPloHeronStirik} in dobimo:
$$p=\sqrt{(s-a)(s-b)(s-c)(s-d)}=\sqrt{abcd}.$$

% Veckotniki

\item \res{Dan je pravokotnik $ABCD$ s stranicama $a=|AB|$ in $b=|BC|$. Izračunaj ploščino lika, ki predstavlja unijo pravokotnika $ABCD$ in njegove slike pri zrcaljenju čez premico $AC$.}

Naj bo $B'=\mathcal{S}_{AC}(B)$ in $D'=\mathcal{S}_{AC}(D)$, $P$ presečišče daljic $AB'$ in $CD$ ter $S$ presečišče diagonal $AC$ in $BD$ pravokotnika $ABCD$. Označimo še $d=|AC|=|BD|$ in $v=|SP|$. Iz podobnosti trikotnikov $AB'C$ in $ASP$ (izrek   \ref{PodTrikKKK}) dobimo $v:b=\frac{d}{2}:a$ oz. $v=\frac{bd}{2a}$. Če uporabimo dobljeno relacijo, izreke \ref{PloscTrik}, \ref{ploscPravok} in \ref{ploscGlavniIzrek} ter Pitagorov izrek \ref{PitagorovIzrek}, za ploščino $p$ iskanega lika velja:

\begin{eqnarray*}
 p &=& 2\cdot \left(p_{AB'C}+p_{ADC}-p_{APC}\right)=
2\cdot \left(p_{ABC}+p_{ADC}-p_{APC}\right)=\\
&=& 2\cdot \left(p_{ABCD}-p_{APC}\right)=2\cdot \left( ab-\frac{dv}{2} \right)=\\
&=& 2\cdot \left( ab-\frac{bd^2}{4a} \right)=2ab-\frac{b(a^2+b^2)}{2a}=\\
&=& \frac{b(3a^2-b^2)}{2a}.
 \end{eqnarray*}

\item \res{Naj bo $ABCDEF$ pravilni šestkotnik, točki $P$ in $Q$ pa središči njegovih stranic $BC$ in $FA$. Kolikšen del ploščine tega šestkotnika predstavlja ploščina trikotnika $PQD$?}

Uporabimo izrek \ref{srednjTrapez}. Rezultat: $p_{PQD}=\frac{3}{8}p_{ABCDEF}$.


% KKrog

\item \res{V kvadratu so včrtani štirje skladni krogi, tako da se vsak krog dotika dveh stranic in dveh krogov. Dokaži, da je vsota ploščin teh krogov enaka ploščini temu kvadratu včrtanega kroga.}

Če z $a$ označimo dolžino stranice kvadrata, sta $r=\frac{a}{2}$ oz. $r_1=\frac{a}{4}$ polmera včrtanega kroga oz. vsakega od manjših včrtanih krogov. Potem je omenjena vsota ploščin enaka:
$$4\cdot r_1^2\pi=4\left(\frac{a}{4} \right)^2\cdot \pi=\left(\frac{a}{2} \right)^2\cdot \pi=r^2\pi.$$

\item \res{Izračunaj ploščino kroga, ki je včrtan trikotniku s stranicami dolžin 9, 12 in 15.}

Uporabimo Heronovo formulo \ref{PloscTrikHeron} in izrek \ref{PloscTrikVcrt}. Rezultat: ploščina trikotnika  $p_{\triangle}=54$; ploščina kroga $p=9\pi$.

\item \res{Naj bo $P$ središče osnovnice $AB$ trapeza $ABCD$, za katerega velja $|BC|=|CD|=|AD|=\frac{1}{2}\cdot |AB|=a$. Izrazi ploščino lika, ki ga določajo osnovnica $CD$ ter krajša krožna loka $PD$ in $PC$ krožnic s središčema $A$ in $B$, kot funkcijo osnovnice $a$.}

Ker je po predpostavki $\overrightarrow{PB}=\overrightarrow{DC}$, je štirikotnik $PBDC$ paralelogram, zato je $BC\cong PD$. Ker je po predpostavki $AD\cong BC$ in $AD\cong AP$, je $PD\cong AD\cong AP$ in $APD$ je enakostranični trikotnik. Torej $\angle PAD= 60^0$. Analogno je tudi $\angle CBP=60^0$. Iskano ploščino $p_0$ dobimo kot razliko ploščine trapeza $p$ in dvakratne ploščine krožnega izseka s središčnim kotom $60^0$:
$$p_0=\frac{a^2}{12}\left(9\sqrt{3}-4\pi \right).$$


\item \res{Tetiva $PQ$ ($|PQ|=d$) krožnice $k$ se dotika njene konciklične krožnice $k'$. Izrazi ploščino kolobarja, ki ga določata krožnici $k$ in $k'$, kot funkcijo tetive $d$.}

 Uporabimo Pitagorov izrek \ref{PitagorovIzrek}. Rezultat: $\frac{d^2\pi}{4}$.

\item \res{Naj bo $r$ polmer včrtanega kroga tetivnega večkotnika $\mathcal{V}$, ki je razdeljen na  trikotnike $\triangle_1,\triangle_2,\ldots,\triangle_n$, tako da nobena dva trikotnika nimata skupnih notranjih točk.  Naj bodo $r_1,r_2,\ldots , r_n$ polmeri včrtanih krožnic teh trikotnikov. Dokaži, da je:
     $$\sum_{i=1}^n r_i\geq r.$$}

Naj bo $s$ polobseg in $p$ ploščino večkotnika $\mathcal{V}$ ter $s_i$ polobseg in $p_i$ ($i\in \{1,2,\ldots , n\}$)  ploščina trikotnika $\triangle_i$. Če uporabimo izreka \ref{ploscTetVec} in \ref{PloscTrikVcrt} ter dejstvo, da za vsak $i\in \{1,2,\ldots , n\}$ velja $s\geq s_i$, dobimo:
 $$sr=p= \sum_{i=1}^n p_i = \sum_{i=1}^n s_ir_i\leq \sum_{i=1}^n sr_i=s\cdot\sum_{i=1}^n r_i.$$


\end{enumerate}




%REŠITVE -  Inverzija
%________________________________________________________________________________

\poglavje{Inversion}


\begin{enumerate}

 \item \res{Dokaži da kompozitum dveh inverzij $\psi_{S,r_1}$ in $\psi_{S,r_2}$ glede na koncentrični
  krožnici predstavlja razteg. Določi središče in koeficient
  tega raztega.}

 Iz definicije inverzije sledi, da je kompozitum inverzij
 $\psi_{S,r_2}\circ\psi_{S,r_1}$
  razteg s središčem $S$ in koeficientom $\frac{r_2^2}{r_1^2}$.
  Trditev je tudi direktna
  posledica naloge \ref{invRazteg}.


  \item \res{Naj bodo $A$, $B$, $C$ in $D$ štiri kolinearne točke.
  Konstruiraj takšni točki $E$ in $F$, da velja $\mathcal{H}(A,B;E,F)$
in $\mathcal{H}(C,D;E,F)$.}

  Uporabimo izrek \ref{harmPravKrozn} - najprej narišemo krožnico, ki
  je pravokotna na krožnici s premeroma $AB$ in $CD$,


 \item \res{V ravnini so dani točka $A$, premica $p$ in krožnica $k$.
 Načrtaj krožnico, ki poteka skozi  točko $A$ in je
 pravokotna na premico $p$ in krožnico $k$.}

 Uporabi inverzijo s središčem $A$.

  \item \res{Reši tretji, četrti, deveti in deseti Apolonijev problem.}

  Pri tretjem in četrtem Apolonijevem problemu uporabimo inverzijo
  s središčem v eni od danih točk. Pri devetem in desetem
  Apolonijevem problemu najprej načrtamo krožnico, ki je
  koncentrična z iskano krožnico in poteka skozi središče ene od
  krožnic. Na ta način problema prevedemo na peti oz. šesti
  Apolonijev problem.

  \item \res{Naj bodo: $A$ točka, $p$ premica, $k$ krožnica
   in $\omega$ kot v neki ravnini. Načrtaj krožnico,
ki poteka skozi točko $A$, se dotika premice $p$ in s krožnico $k$
določa kot $\omega$.}

  Uporabimo inverzijo s središčem $A$.


 \item \res{Določi geometrijsko mesto točk dotika dveh krožnic, ki
 se dotikata krakov danega kota v dveh danih točkah $A$ in $B$.}

 Uporabimo inverzijo s središčem $B$.

  \item \res{Načrtaj trikotnik, če so znani naslednji podatki:
\begin{enumerate}
 \item $a$, $l_a$, $v_a$
 \item $v_a$, $t_a$, $b-c$
 \item $b+c$, $v_a$, $r_b-r_c$
 \end{enumerate}}

  Uporabimo veliko nalogo (glej izrek \ref{velikaNaloga}) in
  ustrezne harmonične četverice točk.


\item \res{Naj bosta $c(S,r)$  in $l$ krožnica in premica v isti ravnini, ki
nimata skupnih točk. Naj bodo še $c_1$, $c_2$ in $c_3$ krožnice te
ravnine, ki se medseboj (po dve) dotikajo in se vsaka od njih
dotika še $c$ in $l$. Izrazi razdaljo točke $S$ od premice $l$ s
$r$\footnote{Predlog za MMO 1982. (SL 12.)}.}

   Najprej dokažemo, da obstaja krožnica $n$, ki je pravokotna na
   premico $l$ in krožnico $c$. Naj bo $Y$ eno od presečišč
   krožnice $n$ in pravokotnice na premico $l$ iz središča krožnice
   $c$. Uporabimo kompozitum $f=\psi_i\circ \mathcal{R}\circ \psi_i$,
   kjer je $i$ poljubna krožnica s središčem $Y$, $\mathcal{R}$
    pa rotacija s središčem v središču krožnice $l'=\psi_i(l)$,
    ki preslika dotikališče krožnic $l'$ in $c'_3=\psi_i(c_3$ v točko
    $Y$. Če je:
    $f:\hspace*{1mm}l, c, c_1, c_2, c_3\mapsto
    \widehat{l}, \widehat{c}, \widehat{c_1}, \widehat{c_2}, \widehat{c_3},$
    dokažemo, da je $\widehat{l}=l$, $\widehat{c}=c$, $\widehat{c_3}$
    premica, ki je vzporedna s premico $l$, ter $\widehat{c_1}$ in
     $\widehat{c_2}$ skladni krožnici, ki se med seboj dotikata in
     se hkrati dotikata tudi vzporednic $l$  in $\widehat{c_3}$.
     Na koncu sledi, da je razdalja središča krožnice $c$ od premice $l$
     enaka $7r$.


\item \res{Naj bo $ABCD$  pravilni tetraeder. Poljubni točki
$M$, ki leži na robu $CD$, pridružimo točko $P = f(M)$, ki je
presečišče pravokotnice skozi točko $A$ na premico $BM$ in
pravokotnice skozi točko $B$ na premico $AM$. Določi geometrijsko
mesto vseh točk $P$, če točka $M$ zavzame vse vrednosti na robu
$CD$.}

Točka $P$ je višinska točka trikotnika $ABM$. Če je $S$ središče roba
$AB$, najprej dokažemo, da velja $\overrightarrow{SP}\cdot
\overrightarrow{SM}=\frac{a^2}{4}$, kjer je $a$ rob
pravilnega tetraedra. Nato uporabimo inverzijo
$\psi_{S,\frac{a}{2}}$ (v ravnini $SCD$). Geometrijsko mesto točk
je potem slika daljice $CD$ pri tej inverziji oz. ustrezni krožni
lok s središčem $S$, s krajišči v višinskih točkah trikotnikov
$ACD$ in $BCD$.


\item \res{Naj bo $ABCD$ tetivnotangentni štirikotnik ter $P$, $Q$,
$R$ in $S$ dotikališča stranic $AB$, $BC$, $CD$ in $AD$ z včrtano
krožnico tega štirikotnika. Dokaži, da velja $PR\perp QS$.}

Uporabimo inverzijo glede na včrtano krožnico. Dokažemo, da so
slike oglišč štirikotnika $ABCD$ oglišča pravokotnika, stranice
tega pravokotnika pa vzporedne z daljicama $PR$ in $QS$.


\item \res{Dokaži, da sta središči tetivnotangentnemu štirikotniku včrtane in očrtane
krožnice ter presečišče njegovih diagonal kolinearne točke
(\index{izrek!Newtonov}Newtonov izrek\footnote{\index{Newton,
I.}\textit{I. Newton} (1643--1727), angleški fizik in matematik}).}

Uporabimo prejšnjo nalogo in dokažemo, da je središče $G$
pravokotnika iz te naloge hkrati središče daljice, ki jo določata
središče včrtane krožnice štirikotnika $ABCD$ (središče inverzije)
in presečišče daljic $PR$ in $QS$. Točka $G$ je namreč središče
slike očrtane krožnice pri omenjeni inverziji.


\item \res{Naj bosta $p$ in $q$ vzporedni tangenti krožnice $k$.
Krožnica $c_1$ se dotika premice $p$ v točki $P$ in krožnice $k$ v
točki $A$, krožnica $k_2$ pa se dotika premice $q$ in krožnic $k$
in $k_1$ v točkah $Q$, $B$ in $C$. Dokaži, da je presečišče premic
$PB$ in $AQ$ središče trikotniku $ABC$ očrtane krožnice.}

Uporabimo inverzijo s središčem $B$ in najprej dokažemo, da je $PB$
skupna tangenta krožnic $k$ in $k_2$, nato pa, da je presečišče
premic $PB$ in $AQ$ potenčno središče krožnic $k$, $k_1$ in $k_2$.

\item \res{Krožnici $k_1$ in $k_3$ se od zunaj dotikata v točki $P$. Prav
 tako se tudi krožnici $k_2$ in $k_2$  od zunaj dotikata v isti
 točki. Krožnica $k_1$ seka krožnici $k_2$ in $k_4$ še v točkah
 $A$ in $D$,  krožnica $k_3$ pa seka krožnici $k_2$ in $k_4$ še v točkah
 $B$ in $C$. Dokaži, da velja\footnote{Predlog za MMO 2003. (SL 16.)}:
 $$\frac{|AB|\cdot|BC|}{|AD|\cdot|DC|}=\frac{|PB|^2}{|PD|^2}.$$}

Naj bo $\psi_P$ inverzija s poljubnim premerom $r$. Ta
preslika štirikotnik $A'B'C'D'$ v paralelogram (izrek
\ref{InverzDotik}). Torej velja $A'B'\cong C'D'$. Če uporabimo
izrek \ref{invMetr}, dobimo
 $\frac{|AB|\cdot r^2}{|PA|\cdot|PB|}=\frac{|CD|\cdot
 r^2}{|PC|\cdot|PD|}$ oz.
 $\frac{|AB|}{|CD|}=\frac{|PA|\cdot|PB|}{|PC|\cdot|PD|}$. Na
 podoben način iz $C'B'\cong A'D'$ sledi
  $\frac{|CB|}{|AD|}=\frac{|PC|\cdot|PB|}{|PA|\cdot|PD|}$. Z
  množenjem  dveh relacij dobimo:
  $\frac{|AB|\cdot|BC|}{|AD|\cdot|DC|}=\frac{|PB|^2}{|PD|^2}$.

  \item \res{Naj bo $A$ točka, ki leži na krožnici $k$. Samo
   s šestilom načrtaj kvadrat $ABCD$ (oz. njegova oglišča), ki je včrtan dani
krožnici.}

  Najprej načrtamo pravilni šestkotnik  $AB_1B_2CD_1D_2$, ki je
  včrtan dani krožnici.

 \item \res{Dani sta točki $A$ in $B$. Samo z uporabo
   šestila načrtaj takšno točko $C$, da velja
   $\overrightarrow{AC}=\frac{1}{3}\overrightarrow{AB}$.}

 Uporabimo podoben postopek kot v nalogi \ref{MaskeroniSred}.
  Najprej narišemo točko $X$, za katero velja
  $\overrightarrow{AX}=3\cdot \overrightarrow{AB}$, nato pa iskano
  točko
  $X'=\psi_k(X)$, kjer je $k$ krožnica s središčem $A$ in polmerom $AB$.

   \item \res{Samo s pomočjo
   šestila razdeli dano daljico v razmerju $2:3$.}

   Za dano daljico $AB$ najprej narišemo točko $X$, za katero je
   $\overrightarrow{AX}=2\cdot \overrightarrow{AB}$ (podobno kot
   pri prejšnji nalogi), nato pa
    točko $Y$, za katero velja
   $\overrightarrow{AY}=\frac{1}{5}\cdot \overrightarrow{AX}$. $Y$
   je iskana točka, ker velja
   $\overrightarrow{AY}=\frac{2}{5}\cdot \overrightarrow{AB}$.
\end{enumerate}
\newpage


\normalsize

%________________________________________________________________________________
% LITERATURA - - - - - - - - - - - - - - - - - - - - - - - - - - - - - - - - - - - - - - -
%________________________________________________________________________________
\begin{thebibliography}{1}


        \bibitem{Berger}  Berger, M.
\emph{Geometry}, Springer-Verlag, Berlin 1987.

        \bibitem{Cofman}  Cofman, J.
\emph{What to solve?}, Oxford University Press, Oxford, 1990.

        \bibitem{CoxeterRevisited}  Coxeter, H. S. M.; Greitzer, S. L.
\emph{Geometry Revisited}, Random House, New York, 1976.

        \bibitem{Djerasimovic}  Djerasimovi\'c, B.
\emph{Zbirka zadataka iz geometrije}, Stručna knjiga, Beograd, 1987.

        \bibitem{MMO}  Djuki\'c, D.; Jankovi\'c, V.;
        Mati\'c, I.; Petrovi\'c, N.
\emph{The IMO Compendium}, Springer, New York, 2006.

         \bibitem{Efimov}  Efimov, N. V.
\emph{Višja geometrija}, Nauka, Moskva, 1978.

         \bibitem{Evklid}  Evklid, \emph{Elementi}, Naučna knjiga, Beograd, 1949.

         \bibitem{Fetisov}  Fetisov, A. I.
\emph{O euklidskoj i neeuklidskim geometrijama}, Školska knjiga, Zagreb, 1981.

        \bibitem{KratkaZgodCasa} Hawking, S. W.
\emph{Kratka zgodovina časa}, DMFA, Ljubljana, 2003.

        \bibitem{Hilbert}  Hilbert, D. \emph{Osnove geometrije}, Naučno delo, Beograd, 1957.

        \bibitem{ZutaKnjiga}  Lopandi\'c, D.
\emph{Geometrija}, Naučna kniga, Beograd, 1979.

        \bibitem{Lopandic}  Lopandi\'c, D.
\emph{Zbirka zadataka iz osnova geometrije}, PMF, Beograd, 1971.

        \bibitem{Lucic}  Luči\'c, Z.
\emph{Euklidska i hiperbolička geometrija}, Matematički
fakultet, Beograd, 1994.

        \bibitem{Martin} \emph{Martin, G.}, Transformation Geometry, Springer-Verlang, New York, 1982.

        \bibitem{Mitrovic}  Mitrovi\'c, M.
\emph{Projektivna geometrija}, DMFA, Ljubljana, 2009.

        \bibitem{MitrovicMG}  Mitrovi\'c, M.; Ognjanovi\'c, S.; Veljkovi\'c, M.; Petkovi\'c, L.; Lazarevi\'c, N.
\emph{Geometrija za prvi razred Matematičke gimnazije}, Krug, Beograd, 1996.

        \bibitem{Nice}  Niče, V.
\emph{Uvod u sintetičku geometriju}, Školska knjiga, Zagreb,
1956.

        \bibitem{Prasalov} Prasalov, V. V. \emph{Zadači po geometriji}, Nauka, Moskva, 1986.

        \bibitem{Prvanovic} Prvanovi\'c, M. \emph{Osnove geometrije}, Gradjevinska knjiga, Beograd, 1987.

        \bibitem{Tosic}  Toši\'c, R.;  Petrovi\'c, V.
\emph{Zbirka zadataka iz osnova geometrije}, Gradjevinska
knjiga, Novi Sad, 1982.

        \bibitem{Stojanovic}  Stojanović, V. \emph{Matematiskop III}, Nauka, Beograd, 1988.

        \bibitem{Struik}  Struik, D. J.
\emph{Kratka zgodovina matematike}, Državna založba Slovenije,
Ljubljana, 1978.

        \bibitem{Oblika} Weeks, J. R.
\emph{Oblika prostora}, DMFA, Ljubljana, 1998.

%%%% Pregledano v celoti! Roman

\end{thebibliography}

\footnotesize

 \printindex

\end{document}
