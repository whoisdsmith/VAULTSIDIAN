\documentclass[11pt]{book}

%PAPIR - US TRADE
\paperwidth 15.24cm
\paperheight 22.86cm

%TEKST
\textwidth 11.9cm \textheight 19.4cm
\oddsidemargin=-0.5cm
\evensidemargin=-1.2cm
\topmargin=-15mm

\headheight=13.86pt

%\usepackage[slovene]{babel}
\usepackage[english]{babel}

%\usepackage[cp1250]{inputenc}
\usepackage[utf8]{inputenc}


\usepackage[T1]{fontenc}
\usepackage{amsmath}
\usepackage{color}
\usepackage{amsfonts}
\usepackage{makeidx}
\usepackage{calc}
\usepackage{gclc}
%\usepackage[dvips]{hyperref}
\usepackage{amssymb}
\usepackage[dvips]{graphicx}
\usepackage{fancyhdr}

%za slike
\usepackage{caption}
\DeclareCaptionFormat{empty}

\def\contentsname{Content}

\makeindex

\newcommand{\ch}{\mathop {\mathrm{ch}}}
\newcommand{\sh}{\mathop {\mathrm{sh}}}
\newcommand{\tgh}{\mathop {\mathrm{th}}}
\newcommand{\tg}{\mathop {\mathrm{tg}}}
\newcommand{\ctg}{\mathop {\mathrm{ctg}}}
\newcommand{\arctg}{\mathop {\mathrm{arctg}}}
\newcommand{\arctgh}{\mathop {\mathrm{arcth}}}

\def\indexname{Index}

\definecolor{green1}{rgb}{0,0.5,0}
\definecolor{viol}{rgb}{0.5,0,0.5}
\definecolor{viol1}{rgb}{0.2,0,0.9}
\definecolor{viol3}{rgb}{0.3,0,0.6}
\definecolor{viol4}{rgb}{0.6,0,0.6}
\definecolor{grey}{rgb}{0.5,0.5,0.5}

 \def\qed{$\hfill\Box$}
\newcommand{\kdokaz}{\color{red}\qed\vspace*{2mm}\normalcolor}

\newcommand{\res}[1]{\color{green1}\textit{#1}\normalcolor}

\newtheorem{izrek}{Theorem}[section]
\newtheorem{lema}{Lemma}[section]
\newtheorem{definicija}{Definition}[section]
\newtheorem{aksiom}{Axiom}[section]
\newtheorem{zgled}{Exercise}[section]
\newtheorem{naloga}{Problem}
\newtheorem{trditev}{Proposition}[section]
\newtheorem{postulat}{Postulate}
\newtheorem{ekv}{E}


%BARVA

\newcommand{\pojem}[1]{\color{viol4}\textit{#1}\normalcolor}

%\newcommand{\pojemFN}[1]{\textit{#1}}

\newcommand{\blema}{\color{blue}\begin{lema}}
\newcommand{\elema}{\end{lema}\normalcolor}

\newcommand{\bizrek}{\color{blue}\begin{izrek}}
\newcommand{\eizrek}{\end{izrek}\normalcolor}

\newcommand{\bdefinicija}{\begin{definicija}}
\newcommand{\edefinicija}{\end{definicija}}

\newcommand{\baksiom}{\color{viol3}\begin{aksiom}}
\newcommand{\eaksiom}{\end{aksiom}\normalcolor}

\newcommand{\bzgled}{\color{green1}\begin{zgled}}
\newcommand{\ezgled}{\end{zgled}\normalcolor}

\newcommand{\bnaloga}{\color{red}\begin{naloga}}
\newcommand{\enaloga}{\end{naloga}\normalcolor}

\newcommand{\btrditev}{\color{blue}\begin{trditev}}
\newcommand{\etrditev}{\end{trditev}\normalcolor}


\newcommand{\del}[1]{\chapter{#1}}
\newcommand{\poglavje}[1]{\section{#1}}
%\newcommand{\naloge}[1]{\color{red}\section*{#1}\normalcolor}
\newcommand{\naloge}[1]{\section{#1}}
\newcommand{\ppoglavje}[1]{\subsection{#1}}

\setlength\arraycolsep{2pt}

\author{Milan Mitrovi\'c}
\title{\textsl{\Huge{\textbf{Euclidean Plane Geometry}}}}

\date{}

%_________________________________________________________________________________________

\begin{document}
\pagestyle{fancy}
\lhead[\thepage]{\textsl{\nouppercase{\rightmark}}}
\rhead[\textsl{\nouppercase{\leftmark}}]{\thepage}
\cfoot[]{}


 \vspace*{-12mm}

\hspace*{24mm} \textsl{\Huge{\textbf{Euclidean }}}\\

\hspace*{24mm} \textsl{\Huge{\textbf{Plane}}}\\

\hspace*{24mm} \textsl{\Huge{\textbf{Geometry}}}

 \vspace*{8mm}\hspace*{60mm}Milan Mitrovi\'c
% \normalcolor

 \vspace*{0mm}

\hspace*{-21mm}
\input{sl.NASL2A4.pic}

%\color{viol1}
 \hspace*{48mm}Sevnica

 \hspace*{49mm}
2013
% \normalcolor
  %slikaNova0-1-1
%\includegraphics[width=120mm]{slikaNaslov.pdf}

 \setcounter{section}{0}
 \thispagestyle{empty}
 \newpage


%\pagecolor{white}

%\color{viol1}
\vspace*{17mm}
 \hspace*{77mm} \textit{To Boris and Jasmina}
%\normalcolor
  %slikaNova0-1-1
%\includegraphics[width=120mm]{slikaNaslov.pdf}


 \thispagestyle{empty}
\newpage

%________________________________________________________________________

%PREDGOVOR
%{\hypertarget{Vsebina}\tableofcontents}
 %\printindex
\thispagestyle{empty}


%_______________________________________________________________________


\chapter*{Preface}

\thispagestyle{empty}


\thispagestyle{empty}

The present book is the result of the experience I gained as a professor in teaching geometry at the Mathematical Gymnasium in Belgrade for many years and preparing students in Slovenia for the International Mathematical Olympiad (IMO).

Formally, the substance is presented in such a way that it does not rely on prior knowledge of geometry. In the book, we will deal only with planar Euclidean geometry - all definitions and statements refer to the plane.

The first two chapters deal with the history and axiomatic design of geometry.
The consequences of the axioms of incidence,  congruence and parallelism are discussed in detail, while in the other two groups (axioms of order and continuity) the consequences are mostly not proven.
Chapters three and four deal with the relation of the congruence of figures, the use of the triangle congruence theorems, and a circle.
In the fifth chapter, a vectors are defined. Thales's theorem of proportion is proven.
Chapter six deals with isometries and their use. Their classification has been performed.
Chapters 7 and 8 deal with similarity transformations, figure similarity relation, and area of figures. The ninth chapter presents the inversion.
At the end of each chapter (except the introductory one) are exercises. Solutions and instructions can be found in the last, tenth chapter.


The book contains 341 theorems, 247 examples and 418 solved problems (28 of them from the IMO). In this sense, the book in front of you is at the same time a preparing guide for the IMO.

For some well-known theorems and problems, are given brief historical remarks. That can help high school and college students to better understand the development of geometry over the centuries.

A lot of help in writing the book was selflessly offered to me by Prof. Roman Drstvenšek, who read the manuscript in its entirety. With his professional and linguistic comments, he made a great contribution to the final version of the book.
In that work, he was assisted by Prof. Ana Kretič Mamič. I kindly thank both of them for the effort and time they have generously devoted to this book.

I would especially like to thank Prof. Kristjan Kocbek, who read the partial manuscript
 and with his critical remarks contributed to
 significant improvement of the book.

%I sincerely thank prof. Gordana Kenda Kozinc, who read and proofread the introductory chapter, and thanks also to prof. Alenka Brilej, who helped Roman with the language test with quite a few useful tips.

I also thank Prof. Dr Predrag Janiči\'c, who wrote the wonderful software package \textit {GCLC} for \ LaTeX {}. Almost all pictures in this book were made with this package.

Last but not least, I would like to thank the students of the Bežigrad Grammar School, the 1st Grammar School in Celje and the Brežice Grammar School for their inspiration and support. Students from these schools attended the renewed course of geometry which I have already taken before
years in Belgrade.

\thispagestyle{empty}

\vspace*{12mm} Sevnica, December 2013 \hfill Milan
Mitrovi\'c
%\newpage

%________________________________________________________________________

 \tableofcontents

 %\thispagestyle{empty}

%\newpage



% DEL 1 - - - - - - - - - - - - - - - - - - - - - - - - - - - - - - - - - - - - - - -
%________________________________________________________________________________
% O DEDUKTIVNI IN INDUKTIVNI METODI
%________________________________________________________________________________

 \del{Introduction} \label{pogUVOD}

%________________________________________________________________________________
\poglavje{Deductive and Inductive Method} \label{odd1DEDUKT}

We learn a lot of geometric concepts in elementary school, such as:
triangle, circle, right angle, etc. Later we also learn some
propositions: propositions about the congruence of triangles,
Pythagoras' and Tales' proposition. At the beginning we do not
prove the propositions, but we verify the facts based on several
individual examples. This way of reasoning is called inductive
method. Inductive method (lat. inductio -- introduction) is thus a
way of reasoning, in which we come to general conclusions from
individual examples. Later we start proving individual
propositions. Through these proofs we first encounter the so-called
deductive way of reasoning, or deduction. Deduction (lat. deductio
-- deduction) is a way of reasoning, in which we come from general
to individual conclusions. The idea of this method is thus to
deduce a general conclusion by proving it and then use it in
individual examples. Since we cannot verify all examples using
inductive method, because their number is usually infinite, we can
also come to wrong conclusions using this method. With deductive
method we always get correct conclusions, if the assumptions we
use in the proof are correct. Let us analyze both of these methods
using the following example. We try to come to the conclusion:
 \btrditev \label{TalesUvod}
 The diameter of a circle subtends a right angle to any point on the circle.
 \etrditev


\begin{figure}[!htb]
\centering
\input{sl.1.2.1.6.pic}
\caption{} \label{sl.sl.1.2.1.6.pic}
\end{figure}

  %slikaNova0-1-1
%\includegraphics[width=50mm]{slikaNova0-1-1.pdf}

If we used inductive method, we would verify whether this proposition
is true in some individual examples; for example in the case when
the apex of the angle is in the center of the circle and similar
(Figure \ref{sl.sl.1.2.1.6.pic}). If we only deduced the general
conclusion from these individual examples, of course we could not
be sure that the proposition is not true in some example we did not
verify.

We will now use the deductive method. Let $AB$ be the radius of the circle with center $O$ and $L$ any point on this circle, different from points $A$ and $B$ (Figure \ref{sl.sl.1.2.1.6.pic}). We will prove that the angle $ALB$ is a right angle. Because $OA\cong OB\cong OL$, it follows that the triangles $AOL$ and $BOL$ are isosceles, therefore $\angle ALO\cong\angle LAO=\alpha$ and $\angle BLO\cong\angle LBO=\beta$. Then $\angle ALB=\alpha+\beta$. The sum of the interior angles in triangle $ALB$ is equal to $180^0$, therefore $2\alpha+2\beta=180^0$. From this it follows:
 $$\angle ALB=\alpha+\beta=90^0$$

We notice that in the case of using the deductive method or in the proof of the statement, we have not considered a certain point $L$ on the circle, but an arbitrary point (in general position). This means that the statement is valid for every point on the circle (except $A$ and $B$), if the proof is correct, of course. But is the proof correct? In this proof, we used the following two statements:
 \btrditev
 If two sides in a triangle are congruent, then the angles opposite the congruent sides are congruent angles.
 \etrditev
 \btrditev
  The sum of the interior angles of a triangle is equal to $180^0$.
   \etrditev
We also used concepts such as: isosceles triangle, angle congruence; in the statement itself, we also used the concepts: diameter, circle, angle over the diameter and right angle. In order to be sure that the statement we proved is true, we must be sure that the two statements we used in the proof are also true. In our case, we assume that we have already proved the aforementioned two statements and that we have introduced all the concepts mentioned. It is clear that this problem arises with every statement - even with the two on which we relied in the proof. This requires a certain systematization of the entire geometry. The question arises as to how to start if we again refer to previously proven statements in the proof of each statement. This process could then continue indefinitely. Thus we come to the need for initial statements - \index{aksiomi} \pojem{aksiomih}. The same applies to concepts - we need t. i. \pojem{začetni pojmi}.\index{začetni pojmi} In this way, each geometry (there can be more of them), which we consider, depends on the choice of initial concepts and axioms. We call this approach to building a geometry \pojem{sintetični postopek}, and we call the geometry itself \pojem{sintetična geometrija}\index{geometrija!sinteti\v{c}na}.


%________________________________________________________________________________
\poglavje{Basic Terms and Basic Theorems} \label{odd1POJMI}

In some theory (like geometry) we introduce every new
concept with a \index{definition} \pojem{definition}, which describes this concept
with the help
 of some initial or already defined concepts.
 The connections between concepts and their appropriate properties are given by
statements,
 which we call
\pojem{theory statements}. As we have already mentioned, we call the initial statements
\index{aksiomi} \pojem{aksiomi}, the statements derived from them
are  \pojem{izreki} of this theory. Formally,
\pojem{proof} \index{dokaz izreka} of some statement $\tau$ is a sequence of statements, which logically follow one
 from the other, each of which
is either an axiom or a statement derived from the axioms (izrek), and the last one in this
sequence is precisely the statement $\tau$.

 Although the choice of axioms is not uniquely determined, it cannot be arbitrary.
 When making this choice
we must be careful not to lead to contradictory statements
or to a contradiction. This means that, for a given choice of
axioms, there is no such statement that both the statement and its
negation are izreki in this theory. We also need enough
axioms so that we can determine, for every statement that we can formulate in this
theory, whether it is true or not. This means that
either the statement or its negation is an izrek in this theory. For a system of axioms that satisfies the first requirement, we say that it is
\index{sistem aksiomov!neprotisloven} \pojem{neprotisloven}, for one that satisfies the second requirement, we say that it is \index{sistem
aksiomov!popoln} \pojem{popoln}. When choosing axioms, there is also a third requirement -- that the system of axioms is \index{sistem
aksiomov!minimalen} \pojem{minimalen}, which means that none of the axioms can be derived from the others. We mention that the last
 requirement is not as important as the first two.

We must also add that we do not build Euclidean geometry
independently from algebra and logic. We will use concepts such as
set, function, relation with properties that apply to them. We will
also use so-called rules of inference, such as
the method of contradiction. For mathematical disciplines that we use in this way to build geometry, we say that they are
\pojem{predpostavljene teorije}.

%________________________________________________________________________________
\poglavje{A Brief Historical Overview of the Development of Geometry}
\label{odd1ZGOD}

People started dealing with geometry in early history.
At first, it was only the observation of characteristic shapes, such as
a circle or a square. Based on the drawings found on the walls of old caves, we conclude
that people in prehistory were interested in the symmetry of shapes.

In further development, man was discovering various properties
of geometric shapes. This was due to practical needs, e.g.
measuring the area of ​​land - which is also how the word
"geometry" came about. In this period, geometry developed as
an inductive science. This means that geometric propositions were coming from experiences - by means of measurements and checking on
individual examples. In this sense, geometry was developed by
all ancient civilizations: Chinese, Indian and especially
Egyptian.

In Egypt, geometry developed mainly as a study of
measurements. Because the Nile river often flooded,
the land had to be measured very often. In addition,
the knowledge of geometry was used in construction. They knew
e.g. the formula for calculating the volume of a pyramid and a truncated
pyramid, although they came to it empirically. So geometry was for
the Egyptians primarily a pragmatic discipline.
The oldest records date back to approximately 1700 BC.

Geometry of three-dimensional space was not
discussed as much as in Egypt.

There is not as much data on Chinese geometry as on Egyptian,
although we know that it was also very developed. In
the oldest preserved records we find a description of the calculation of
the volume of a prism, pyramid, cylinder, cone, truncated pyramid and
truncated cone.

Indian geometry is much younger than the previous three. It dates
back to approximately the 5th century BC. In it we already see the first
attempts at proof. Later it developed parallel to
Greek geometry.

The turning point in the development of geometry took place in Ancient Greece. This is when the deductive method was first used in geometry. The first geometric proofs are associated with Tales\footnote{The Ancient Greek philosopher and mathematician \textit{Tales} \index{Tales} from Miletus (640--546 BC).}. We connect his name with the well-known statement about the proportionality of segments in parallel lines. He also proved the statement that the angles above the diameter of a circle are right, even though this claim was known without proof to the Babylonians 1000 years earlier. This way of developing geometry was continued by other Ancient Greek philosophers, of which Pythagoras\footnote{The Ancient Greek philosopher and mathematician \textit{Pythagoras} \index{Pythagoras} from the island of Samos (ca. 580--490 BC).} was one of the most important. Of course, his \index{statement!Pythagorean}\textit{Pythagorean statement} is famous. However, the Egyptians knew this statement as a fact 3000 years BC (maybe the statement was known even before that), but Pythagoras gave the first known proof. Archimedes\footnote{The Ancient Greek philosopher and mathematician \textit{Archimedes} \index{Archimedes} from Syracuse (287--212 BC).} was the first to present a theoretical calculation of the number $\pi$, by considering the inscribed and circumscribed polygons with $96$ sides. The statements about the congruence of triangles were also proven. With the rapid progress of geometry, reflected in the large number of proven statements, the need for systematization and, with that, the need for axioms, became apparent. The need for axioms was first described by Plato\footnote{The Ancient Greek philosopher and mathematician \textit{Plato} \index{Plato} (427--347 BC).} and Aristotle\footnote{The Ancient Greek philosopher and mathematician \textit{Aristotle} \index{Aristotle} from Athens (384--322 BC).}. Plato is also known in mathematics for his research of regular polygons: the tetrahedron, the cube, the octahedron, the dodecahedron and the icosahedron, which is why we also call them Platonic solids after him.

One of the first attempts at an axiomatic approach to geometry - and the only one that has been preserved from that time - was made by the most famous mathematician of that time, Euclid, Plato's student, in his well-known work \textit{The Elements}, which consists of 13 books. In it, he systematized all the existing knowledge of geometry. He divided the initial statements into axioms and so-called postulates, of which the latter are purely geometric content (today we also call them axioms). \textit{The Elements} became one of the most important and influential books in the history of mathematics. The geometry, which he developed in this way, with minor unimportant changes, is the one that is taught in schools today. Proofs, such as the one about the central and peripheral angle, have been preserved in practically unchanged form. We give the postulates as Euclid gave them (Figure \ref{sl.sl.1.3.1.9.pic}):
\color{viol3}
\begin{postulat}
  We can draw a straight line from any point to any point.
 \end{postulat}
 \begin{postulat}
We can produce a finite straight line continuously in a straight line.
 \end{postulat}
 \begin{postulat}
We can describe a circle with any center and distance.
 \end{postulat}
  \begin{postulat}
All right angles are equal to one another.
 \end{postulat}
 \begin{postulat}
If a straight line falling on two straight lines makes the interior angles on the same side less than two right angles, the straight lines, if produced indefinitely, will meet on that side on which the angles are less that two right angles.
 \end{postulat}
\normalcolor

\begin{figure}[!htb]
\centering
\input{sl.1.3.1.9.pic}
\caption{} \label{sl.sl.1.3.1.9.pic}
\end{figure}

  %slikaNova1-3-3
%\includegraphics[width=100mm]{slikaNova1-3-3.pdf}


However, the system of axioms that Euclid gave was not perfect. In some proofs, he took certain parts as obvious and did not prove them. Of course, we should not be too critical, because this was a revolutionary work for those times. \textit{The Elements} were an example and inspiration for mathematicians for centuries and laid the foundations for the further development of geometry to this day. The last axiom, i.e. the \index{aksiom!fifth Euclid's axiom} \pojem{fifth Euclid's axiom}, was particularly important for the further development of geometry. The problem of its independence from the other axioms was open for the next 2000 years!

The last in a series of great ancient Greek mathematicians were
Apolonius\footnote{The Greek mathematician \textit{Apolonius}
\index{Apolonius} from Perga (262--190 BC).},
Menelaus\footnote{The Greek mathematician \textit{Menelaus}
\index{Menelaus} from Alexandria
  (ca. 70--130).} and Pappus\footnote{The Greek mathematician \textit{Pappus}
  \index{Pappus} from
  Alexandria (ca. 290--350).}. In his book \textit{On the Cutting of a Cone}, Apolonius defined the ellipse, parabola and hyperbola as the intersections of a plane and a circular (infinite) cone. This allowed him to consider their properties simultaneously, which was quite a modern approach for the time. Menelaus and Pappus proved certain theorems which didn't become relevant until the 19th century with the development of projective geometry. So the ideas of these three mathematicians were quite modern and in a way we can say that they were on the brink of discovering the first non-Euclidean geometry.

After the final fall of the Old Greece under the Roman Empire,
the period of the glorious ancient Greek geometry came to an end. Although the Old Romans took over a large part of the ancient Greek culture and built roads,
aqueducts and so on, it is interesting that they never really showed much interest in
the ancient theoretical mathematics. So their contribution to the development
of geometry is very modest.

An important role in the further development of geometry was taken over by the Arabs.
First of all, we should say that all the works of the Ancient Greeks
including Euclid's \textit{Elements} are known to us today because they
were translated and thus preserved by the Arabs. After the foundation of
Baghdad in 762, in the next 100 years they translated most of the works
of ancient Greek and Indian mathematics. They also made a synthesis
of ancient Greek mostly geometric and Indian mostly
algebraic approach. We mention that the word itself \pojem{algebra}
is of Arabic origin. In addition, the Arabs continued the development
of \pojem{trigonometry}, which was designed by the Ancient Greeks. A. R. al-Biruni\footnote{The Arab mathematician \textit{ A. R. al-Biruni} \index{al-Biruni, A. R.} (973--1048).} proved the now known \pojem{sine theorem}.

The development of geometry in Europe began in the 12th century, when
Arabic and Jewish mathematicians brought their knowledge to
Spain and Sicily. (Euclid's \textit{Elements} were translated from
Arabic into Latin around 1200); but Europe did not experience its
true flowering until the 16th century. In the Middle Ages,
mathematics developed very slowly. In the Middle Ages,
Western European mathematicians only learned the Greek geometric
heritage from Arabic translations, but this process was not
quick. When this knowledge was accumulated and social and
political conditions changed, a new era began in the
development of geometry. The first new results were given by Italian
mathematicians of that time, who paid a lot of attention to
constructions using a ruler and a compass.

As we mentioned earlier, Euclid's fifth axiom had a very big impact on
the further development of geometry. Because of its
formulation, which is not as simple as the previous axioms, and
also because of its importance, many mathematicians of that time
thought that it did not need to be considered as an axiom, but it
could be proven as a theorem with the other axioms. If we read
Euclid's other initial propositions, it is really true that the
fifth axiom is more complex. The problem of the independence of the
fifth axiom from the others occupied many
mathematicians in the following centuries. Until the second half of
the 19th century the problem was not solved. In many attempts to
prove the fifth axiom from the others, propositions were used,
whose proof was omitted.
It was later shown that these propositions could not even be
proven from the other axioms if the fifth axiom was omitted from the
list.
Just as they follow from the fifth axiom, these propositions
also follow from the fifth axiom (of course using the other
axioms).
We therefore call them \pojem{equivalents of the fifth Euclidean
axiom}.
We give some examples of these equivalents (Figure
\ref{sl.sl.1.3.1.9a.pic}):

\color{blue}
\begin{ekv}
If $ ABCD $ is a quadrilateral with two right angles on the side $BC$
and the sides $AB$ and $CD$ are congurent, then the two remain angles
of this quadrilateral are also right angles.\footnote{This equivalent was set by
Italian mathematician \index{Saccheri, G. G.} \textit{G. G.
Saccheri} (1667--1733).}
\end{ekv}
\begin{ekv}
A line perpendicular to one arm of an acute angle intersects
his other arm.
\end{ekv}
\begin{ekv}
Every triangle can be circumscribed.
\end{ekv}
\begin{ekv}
If three angles of a quadrilateral are right angles, then the fourth angle is also a right angle.\footnote{\index{Lambert, J. H.}\textit{J.
H. Lambert} (1728--1777), French mathematician.}
\end{ekv}
\begin{ekv}
The sum of the interior angles in every triangle is $180^0$.\footnote{\index{Legendre, A. M.} \textit{A. M. Legendre}
(1752--1833), French mathematician.}
\end{ekv}
\begin{ekv}
For any given line $p$ and point $A$ not on $p$, in the plane containing both line $p$ and point $A$ there is just one line
 through point $A$ that do not intersect line $p$\footnote{\index{Playfair, J.} \textit{J. Playfair}
(1748--1819), Scottish mathematician.}.
 \end{ekv}
\normalcolor


\begin{figure}[!htb]
\centering
\input{sl.1.3.1.9a.pic}
\caption{} \label{sl.sl.1.3.1.9a.pic}
\end{figure}

So mathematicians were able to prove the fifth Euclidean axiom with
the help of each of these assertions, but eventually it turned out that
none of them could be proved without the fifth axiom. Therefore,
these assertions, as we have already mentioned, are equivalent to the fifth axiom.
Today, Playfair's equivalent is most commonly used, which was later
added to Euclid's axioms instead of the fifth axiom.

But how did mathematicians find out that the fifth Euclidean axiom could not
be derived from the other axioms? Just the fact that they were not able
to prove it did not mean that it was not possible. The answer to this
question came at the end of the 19th century and, as we will see,
brought much more to the development of geometry than just the fact of
the unprovability of the fifth axiom.

The next breakthrough in the development of geometry was the discovery
\index{geometrija!neevklidska}\pojem{non-Euclidean geometries} in the 19th century. For the beginning of this development
we count N.~I.~Lobačevskega\footnote{\index{Lobačevski, N.
I.}\textit{N. I. Lobačevski} (1792--1856), Russian mathematician.}. He also dealt with the problem of the independence of the fifth Euclidean axiom.
Based on its negation or the negation of Playfair's equivalent statement, Lobačevski postulated that through a point that does not lie on a straight line, there are at least two straight lines that do not intersect with that line and are coplanar. In an attempt to come to a contradiction (thus the fifth axiom would be proven), he built a whole
sequence of new statements. One of them, for example, is that the sum of the interior angles
of a triangle is always less than the extended angle. But none of these statements
were in contradiction with the other axioms, if of course we exclude
the fifth Euclidean axiom from the list. From this he got the idea that it is possible to build
a completely new geometry that is non-contradictory and based on all
Euclidean axioms except the fifth, which we replace with its
negation. Today we call this geometry
\index{geometrija!hiperbolična} \pojem{hyperbolic geometry}
or \pojem{Lobačevski geometry}.

Independently of Lobačevski, J.
Bolyai\footnote{\index{Bolyai, J.} \textit{J. Bolyai} (1802--1860),
Hungarian mathematician.}. As it often happens, the ideas of Lobačevski during his lifetime
unfortunately were not accepted. The complete confirmation of these ideas
or the proof of the non-contradiction of this new geometry was at the end of the 19th century, that is, only after the death of Lobačevski, presented by A.
Poincar\'{e}\footnote{\index{Poincar\'{e}, J. H.} \textit{J. H.
Poincar\'{e}} (1854--1912), French mathematician.}.
Poincar\'{e} built a model on
the basis of which he showed that a possible contradiction of Lobačevski geometry
would at the same time be a contradiction of Euclidean geometry.
Later, the discovery of other non-Euclidean geometries followed.

Although at the end of the 19th and the beginning of the 20th century the system of Euclid's geometry axioms was already almost completely built, the first correct and complete system was given by D. Hilbert\footnote{\index{Hilbert, D.} \textit{D. Hilbert} (1862--1943), German mathematician.} in his famous book \textit{The Foundations of Geometry}, published in 1899. We use a very similar system of axioms in an almost unchanged form even today.

Parallel to the research of the fifth Euclid's axiom and the development of non-Euclidean geometries, other important methods have also emerged in the study of geometry. As early as around 1637, R. Descartes\footnote{\index{Descartes, R.} \textit{R. Descartes} (1596--1650), French mathematician.} in his book \textit{Geometry} showed that every point in a plane can be described by a suitable pair of two real numbers and similarly in space as a triple of three real numbers. He connected this with the concept of a coordinate representation of the dependence of one quantity (function) on another (variable), which was known earlier. Today, we call such a way of determining points in space after him \pojem{Cartesian coordinate system}. Lines and planes can then be described as sets of solutions of appropriate linear equations, where the unknowns are coordinates of points.

Thus, under the influence of the ideas of F. Vi\'{e}te\footnote{\index{Vi\'{e}te, F.} \textit{F. Vi\'{e}te} (1540--1603), French mathematician.}, Descartes and P. Fermat\footnote{\index{Fermat, P.} \textit{P. Fermat} (1601--1665), French mathematician.}, the two very important mathematical disciplines began to develop - first \index{geometry!analytical} \pojem{analytical geometry}, then \index{linear algebra} \pojem{linear algebra}, which represent the connection between algebra and geometry. The further development of these two disciplines allowed the development of \index{geometry!multidimensional} \pojem{multidimensional geometry}, in which spaces of dimensions greater than three can be considered, because in algebra there are no such limitations as we have in the geometric perception of space. Thus, we can define the so-called \pojem{polytope} - objects of multidimensional space, which are the analogy of two-dimensional polygons and three-dimensional polyhedrons.

Later, the discovery of other non-Euclidean geometries followed.
In the 19th century, the so-called \index{geometrija!projektivna}\pojem{projective geometry} developed, but
its development was not axiomatic like in hyperbolic
geometry; the appropriate system of axioms was set only
later. In this geometry, there are no lines in the plane that do not intersect.

One of the first motives for the beginning of the
 development
 of projective geometry originates from painting or from the desire to transfer
 the feeling of three-dimensional space into a plane. Already in very
 early painting, we encounter a very important property--that
 in the picture, parallel lines are represented as lines that intersect.

 In the 15th century, Italian artists were very interested in
 the geometry of space. The theory of perspective was first considered by F.
 Brunellechi\footnote{\index{Brunellechi, F.} \textit{F.
 Brunellechi} (1377--1446), Italian architect.} in 1425.
   His work was continued by L. B. Alberti\footnote{\index{Alberti, L. B.}
  \textit{L. B.
 Alberti} (1404--1472), Italian mathematician and painter.} and A.
 D\"{u}rer\footnote{\index{D\"{u}rer, A.} \textit{A.
 D\"{u}rer} (1471--1528), German painter.}. Alberti's book from
 1435 represents the first presentation of central projection.

 For the beginning of the development of projective geometry as a mathematical
 discipline, we consider the period when
 J. Kepler\footnote{\index{Kepler, J.} \textit{J. Kepler} (1571--1630),
  German astronomer.}  and G. Desargues\footnote{\index{Desargues, G.}
  \textit{G. Desargues}
 (1591--1661), French architect.}
 independently introduced the concept of points at infinity.
  Kepler showed that a parabola has two foci, one of which
   is a point at infinity. Desargues
 was writing in 1639: ‘‘Two parallel lines have a common endpoint at
 an infinite distance.’’ In 1636, he wrote a book on perspective,
 and in 1639, he wrote about cones. The famous \textit{Desargues' Theorem}
 was published  in 1648.

With the further development of projective geometry we connect French mathematics.
  The genius B. Pascal\footnote{\index{Pascal, B.} \textit{B. Pascal} (1623--1662),
  French philosopher and mathematician.} was
  already as a sixteen year old when he proved an important theorem about
  cones, which we today call \textit{Pascal's theorem}.
  This theorem, which is one of the basic
  theorems of projective geometry, was published in 1640.
  G. Monge\footnote{\index{Monge, G.} \textit{G. Monge}
  (1746--1818), French mathematician.}
  was among the first mathematicians who we can consider a specialist; he
  is in fact the first true geometer. He developed \pojem{descriptive geometry} as
  a special discipline. In his research in descriptive geometry
  we find many ideas of projective geometry.
  The most original Monge's student was
  J. V. Poncelet\footnote{\index{Poncelet, J. V.}
  \textit{J. V. Poncelet} (1788--1867), French mathematician.}.
  Although Pappus\footnote{\index{Pappus} \textit{Pappus of Alexandria} (3rd century), Greek mathematician.}
  discovered the first projective theorems, Poncelet
  with a completely projective way of reasoning proved them only in the 19th century.
  In 1822
  Poncelet published his famous "Treatise on the projective properties of figures",
  in which all the important concepts characteristic for
  projective geometry appear: harmonic quadruple, perspectivity, projectivity,
  involution, etc. Poncelet introduced a line at infinity for all
  planes that are parallel to a given plane. Poncelet and J. D.
  Gergonne\footnote{\index{Gergonne, J. D.} \textit{J. D. Gergonne} (1771--1859), French mathematician.} independently
  from each other studied duality in projective geometry,
  C.~J.~Brianchon\footnote{\index{Brianchon, C. J.} \textit{C. J. Brianchon}
   (1783--1864), French mathematician.} however
 proved a theorem which is dual to Pascal's theorem.
 M.~Chasles\footnote{\index{Chasles, M.} \textit{M. Chasles} (1793--1880), French mathematician.} was the last of
  the great
 school of
 French projective  geometers of that time.

A typical representative of so-called pure geometry
 (today we would say synthetic geometry) was
 J. Steiner\footnote{\index{Steiner, J.} \textit{J. Steiner}
 (1796--1863),
 Swiss mathematician.}.
 Steiner developed projective geometry very systematically,
  from perspective to projectivity and then to conics.

 In the middle of the 19th century, German mathematicians took over the lead in the development of projective geometry. They advocated a synthetic approach
 h geometry. All mathematicians until then had designed projective geometry
  based on Euclidean metric geometry -- by adding
  points at infinity. But C.~G.~C.~Staudt
  \footnote{\index{Staudt, K. G. C.} \textit{C. G. C. Staudt} (1798--1867),
  German mathematician.}
   was the first to try to make it independent and
    design it only on incidence axioms, without the help of metric.
    This led to the abolition of the difference between points at infinity and ordinary
     points
    or the transition from extended Euclidean to projective space.

  F. Klein\footnote{\index{Klein, F. C.} \textit{F. C. Klein} (1849--1925), German mathematician.} set the projective geometry on algebraic foundations in 1871
   with the help of
    so-called \pojem{homogeneous coordinates}, which were discovered independently of each other in 1827,
     by K. W. Feuerbach\footnote{\index{Feuerbach, K. W.} \textit{K. W. Feuerbach}
      (1800--1834), German mathematician.}
      and A. F. M\"{o}bius\footnote{\index{M\"{o}bius, A. F.} \textit{A. F. M\"{o}bius}
       (1790--1868), German mathematician.}.
     A. Cayley\footnote{\index{Cayley, A.} \textit{A. Cayley} (1821--1895), English mathematician.} and
     Klein
     found the use of projective geometry in other non-Euclidean geometries.
     They discovered the model of hyperbolic geometry and models of other geometries
     in projective.

   The first to completely axiomatically design projective geometry, were
    G. Fano\footnote{\index{Fano, G.} \textit{G. Fano} (1871--1952), Italian mathematician.}
     in 1892 and M. Pieri\footnote{\index{Pieri, M.} \textit{M. Pieri} (1860--1913), Italian mathematician.}
      in 1899.

Because of its relative simplicity, the development of classical
(synthetic) projective geometry was almost completed by the end of
the 19th century. Its development today continues within the framework
of other theories -- especially in algebra and algebraic geometry as
$n$-dimensional projective geometry.

G. F. B. Riemann\footnote{\index{Riemann, G. F. B.} \textit{G. F.
B. Riemann} (1828--1866), German mathematician.} defined the space of
arbitrary dimension in his book \textit{On the hypotheses which lie at
the foundations of geometry}, which is not always of constant
curvature. After him, we today call it the \pojem{Riemann metric
space}\index{Riemann spaces}. Euclidean geometry is then obtained as a
special case: if the curvature is constant and equal to 0; hyperbolic
geometry is obtained if we choose that the curvature is constant and
negative. If the curvature is constant and positive, we obtain the so-
called \index{geometry!elliptic} \pojem{elliptic geometry}. The latter
geometry is actually projective geometry, if we add a metric to it.
This research was also the beginning of the development of a new
discipline in mathematics, namely the \index{geometry!differential}
\pojem{differential geometry}.

If we think about non-Euclidean geometries, it may seem strange to us
that in mathematics there can be considered more different theories,
such as Euclidean geometry and hyperbolic geometry, which are in
contradiction with each other. For modern mathematics, it is most
important that both geometries are determined by systems of axioms,
which are (each for itself) consistent and complete. To the question
of which of these two geometries is valid, it is pointless to seek an
answer within mathematics. This is because it depends on which
axioms we have chosen. Such a question would be the same as the
question of which axioms are valid. But the axioms are assumed by
definition without proof. Of course, we can ask the question of what
the geometry of space is in the physical sense and how we can describe
it with axioms."

To answer this question, a physical interpretation of the basic geometric concepts is needed. For example, it is most natural to interpret a line as a ray of light. In this sense, physical space is not Euclidean. It is not even determined by hyperbolic geometry. With the advent of Einstein's\footnote{\index{Einstein, A.} \textit{A. Einstein} (1879--1955), famous German physicist.} theory of relativity at the beginning of the 20th century, it turned out that in space of cosmic dimensions it is more convenient to use non-Euclidean geometry with variable curvature (Riemann metric spaces!). We can say that the geometry of the universe is locally different in each point, depending on the proximity and size of some mass. Einstein's theory also tells us that space and time are interrelated and that time (which is of course surprising) does not flow evenly in each point of the universe. In connection with the mentioned connection of space and time, the so-called \pojem{Minkowsky space of four dimensions}\footnote{This space was discovered by \index{Minkowsky, H.} \textit{H. Minkowsky} (1864--1909), German mathematician.} is important.

Since the 20s of the 20th century and the development of the theory of the primordial soup, we know that the universe is not static and that it is expanding. Its destiny depends on what geometry globally describes it best. But we still do not know for sure what the shape of the universe is or what its destiny is. We do not even know if the universe is finite or infinite. As S. Hawking writes in his famous popular book from 1988 A Brief History of Time (\cite{KratkaZgodCasa}), the universe may even be finite and unlimited. The latter seems paradoxical, although we can imagine it if, instead of three-dimensional, we imagine a "two-dimensional universe". So the beings of this two-dimensional universe could once find out that their universe is actually not a plane but a sphere that is finite but unlimited. The sphere is part of three-dimensional space. So we can theoretically imagine the universe as a three-dimensional sphere in four-dimensional space. The three-dimensional sphere is one of the 3-manifolds. In this case, the geometry of the universe would be globally elliptical.

Although the question of the shape and destiny of the universe is a question of theoretical physics and cosmology, we see how modern geometry (non-Euclidean geometry, geometry of multi-dimensional spaces, etc.) is closely related to this problem (\cite{Oblika}). It is important to understand that geometry (and every other mathematical discipline) develops and is treated as an abstract discipline in which physical interpretations are only inspiration - in this sense, only axioms and initial concepts remain on which we then build mathematical theory.

At the end, we mention another one of the most important geometers of the 20th century H. S. M. Coxeter\footnote{\index{Coxeter, H. S. M.}
\textit{H. S. M. Coxeter} (1907--2003), Canadian mathematician. One
of the greatest geometers of the 20th century.}. Coxeter further researched polytopes in arbitrary dimensions, primarily regular polytopes. In addition, he
was mostly occupied with groups of isometries in hyperbolic geometry and
with multidimensional hyperbolic geometry.

It needs to be added that with everything
that we have
said, the development of geometry is far from over. Quite the opposite -- contrary to the usual
perception -- geometry and mathematics in general are now developing
even faster than ever before. Even today, there are in mathematics (in geometry in particular from the non-Euclidean
geometries) many
problems that are still unresolved.







% DEL 2 - - - - - - - - - - - - - - - - - - - - - - - - - - - - - - - - - - - - - - -
%________________________________________________________________________________
%  AKSIOMI RAVNINSKE EVKLIDSKE GEOMETRIJE
%________________________________________________________________________________


\del{Axioms of Planar Euclidean Geometry} \label{pogAKS}

In what follows, we will illustrate the axiomatic design of planar Euclidean
geometry. We will list the initial concepts and initial statements -
axioms, and then derive some new concepts and statements.
We mention that we have chosen the axioms of the plane, because in this book
we will only deal with the geometry of the Euclidean plane.

Let $\mathcal{S}$ be a non-empty set. Its elements are called
\index{point} \pojem{points} and we denote them with $A, B, C, \ldots$
Certain subsets of the set $\mathcal{S}$ are called \index{line}
\pojem{lines} and we denote them with $a, b, c, \ldots$ The set
$\mathcal{S}$ (the set of all points) is also called the
\index{plane} \pojem{plane}. In addition to these basic concepts,
there are also two relations on the set $\mathcal{S}$. The first is
the \index{relation!$\mathcal{B}$} \pojem{relation $\mathcal{B}$}
and it applies to three points. The fact that points $A$, $B$ and $C$
are in this relation, we will denote with $\mathcal{B}(A,B,C)$ and
read: Point $B$ is between points $A$ and $C$. The second is the
\index{relation!compatibility of point pairs} \pojem{relation of
compatibility of point pairs}; the fact that pairs of points $A, B$
and $C, D$ are in this relation, we will denote with $(A,B) \cong
(C,D)$ and read: The pair of points $(A,B)$ is compatible with the
pair of points $(C,D)$.

 With the help of the aforementioned basic concepts, we can
also define the following derived concepts:

If point $A$ belongs to line $p$ ($A\in p$), or line $p$ contains
point $A$ ($p\ni A$), we will say that
 point $A$\index{relation!lies on a line} \pojem{lies on} line $p$, or that line $p$ \index{relation!goes through a point}\pojem{goes
 through} point $A$.
 For three or
more points we say that they are \index{collinear points}\pojem{collinear},
if they lie on the same line, otherwise they are
\index{non-collinear points}\pojem{non-collinear}. Two
different lines \pojem{intersect}, if their intersection (the
intersection of two subsets) is not an empty set. We call their
intersection the \index{intersection of two lines} \pojem{intersection} of two lines. Any non-empty subset $\Phi$ of the set $\mathcal{S}$ ($\Phi\subset\mathcal{S}$) is called a \index{figure} \pojem{figure}. We say that figures $\Phi_1$ and $\Phi_2$ \index{figures!coincide}\pojem{coincide} (or they are \index{figures!identical}\pojem{identical}), if $\Phi_1=\Phi_2$.

 Now we will
also list some basic theorems - axioms. By their nature, they are
divided into five groups:

\begin{enumerate}
  \item incidence axioms (three axioms),
  \item ordering axioms (four axioms),
  \item congruence axioms (four axioms),
  \item continuity axiom (one axiom),
  \item Playfair's axiom (one axiom).
\end{enumerate}



%________________________________________________________________________________
 \poglavje{Incidence Axioms}
  \label{odd2AKSINC}

 Because lines as basic notions represent certain sets of points,
 we can
consider appropriate relations between elements
and sets
of points and lines: $\in$ and $\ni$ - relations we also call
\index{relacija!incidencije}\pojem{relations of incidence}. These axioms describe just the basic properties of these relations  (Figure
\ref{sl.aks.2.1.1.pic}):

\vspace*{3mm}

        \baksiom \label{AksI1} For every pair of distinct points $A$ and $B$
        there is exactly one line $p$ such that $A$ and $B$  lie on $p$.
        \eaksiom

        \baksiom \label{AksI2}
        For every line there exist at least two distinct points such that both  lie on.
         \eaksiom

        \baksiom \label{AksI3} There exist three points that do not all lie on any one line.
        \eaksiom

\vspace*{3mm}


\begin{figure}[!htb]
\centering
\input{sl.aks.2.1.1.pic}
\caption{} \label{sl.aks.2.1.1.pic}
\end{figure}



 From the first two axioms \ref{AksI1} and \ref{AksI2} it follows that each line is determined by
 its two distinct points. Therefore
the line $p$, which is determined by the points $A$ and $B$, we also call the line $AB$.

 From the first axiom \ref{AksI1} it follows that the intersection of two
 lines that intersect is one point. If, for example, two lines
 had one more common point, according to this axiom they would coincide (they would be identical),
 but in the definition of lines that intersect, we required
 that
they are different.
 The fact that the lines $p$ and $q$ intersect in the point $A$, we will
 write $p\cap q=\{A\}$ or shorter $p\cap q=A$ (Figure \ref{sl.aks.2.1.2.pic}).



\begin{figure}[!htb]
\centering
\input{sl.aks.2.1.2.pic}
\caption{} \label{sl.aks.2.1.2.pic}
\end{figure}



The third axiom \ref{AksI3} can also be expressed as follows: There exist
at least three  points that are not collinear.

So we deduced the first consequences of the incidence axioms;
 for simplicity, we did not express them in the form of propositions. These are almost all the consequences that arise from the first group of axioms.
 Because of this, geometry, which is based solely on the axioms of incidence,
 is too simple. In it, we could only prove the existence of three points and three
 lines. So we need new axioms.



%________________________________________________________________________________
 \poglavje{Ordering Axioms}
 \label{odd2AKSURJ}

The axioms in this group describe the basic characteristics of the relation
$\mathcal{B}$, which we listed as the basic concept.
\vspace*{3mm}


        \baksiom \label{AksII1} If $\mathcal{B} (A, B, C)$, then $A$, $B$
         and $C$ are three  distinct collinear points, and also
         $\mathcal{B} (C, B, A)$ (Figure \ref{sl.aks.2.2.1.pic}).
        \eaksiom


\begin{figure}[!htb]
\centering
\input{sl.aks.2.2.1.pic}
\caption{} \label{sl.aks.2.2.1.pic}
\end{figure}


        \baksiom \label{AksII2} If $A$, $B$ and $C$ are three distinct
         collinear points,
          exactly one of the relations holds: $\mathcal{B}(A,B,C)$,
        $\mathcal{B}(A,C,B)$, $\mathcal{B}(C,A,B)$
        (Figure \ref{sl.aks.2.2.2.pic}).
        \eaksiom


\begin{figure}[!htb]
\centering
\input{sl.aks.2.2.2.pic}
\caption{} \label{sl.aks.2.2.2.pic}
\end{figure}



        \baksiom \label{AksII3} Given a pair of distinct points $A$ and $B$ there is a point $C$ on line $AB$, so that
        is $\mathcal{B}(A,B,C)$
        (Figure \ref{sl.aks.2.2.3.pic}).
        \eaksiom

\vspace*{-1mm}

\begin{figure}[!htb]
\centering
\input{sl.aks.2.2.3.pic}
\caption{} \label{sl.aks.2.2.3.pic}
\end{figure}

\baksiom \label{AksPascheva}\index{aksiom!Paschev}
        (Pasch's\footnote{\index{Pasch, M.}
         \textit{M. Pasch}
        (1843--1930), German mathematician, who introduced the concept of ordering points
        in his 'Lectures on Modern Geometry' from 1882. These
        axioms were later supplemented by Italian mathematician \index{Peano, G.} \textit{G. Peano}
        (1858--1932), in 'Principles of Geometry', then by German mathematician
        \index{Hilbert, D.}\textit{D. Hilbert} (1862--1943) in his famous book
         'Foundations of Geometry' from
        1899.} axiom)
        Let $A$, $B$ and $C$ be three noncollinear points and $l$ be a line that does not contain point $A$.
        If there is a point $P$ on $l$ that is $\mathcal{B}(B,P,C)$ then either $l$ contains a point $Q$ that is  $\mathcal{B}(A,Q,C)$ or $l$ contains a point $R$ that is $\mathcal{B}(A,R,B)$ (Figure \ref{sl.aks.2.2.4.pic}).
        \eaksiom


\begin{figure}[!htb]
\centering
\input{sl.aks.2.2.4.pic}
\caption{} \label{sl.aks.2.2.4.pic}
\end{figure}


In the previous axiom, we did not particularly emphasize that the line $l$ lies in the plane $ABC$, because we are building a plane Euclidean geometry, where all points lie in the same plane.

 At this point we will not derive all the consequences of the ordering axioms.
The formal derivation of all the facts is not so simple and would take up a lot of space. Most of the proofs can be found in \cite{Lucic}.

We prove the first consequence of the ordering axioms.


        \bizrek \label{izrekAksUrACB}
        Given a pair of distinct points $A$ and $B$ there is a point $C$, so that
        is $\mathcal{B}(A,C,B)$.
        \eizrek


\begin{figure}[!htb]
\centering
\input{sl.aks.2.2.5.pic}
\caption{} \label{sl.aks.2.2.5.pic}
\end{figure}

\textbf{\textit{Proof.}} By Axiom \ref{AksI1} there exists exactly one line that goes through points $A$ and $B$ - we'll mark it with $AB$.
By Axiom \ref{AksI3} there are at least three non-linear points.
Therefore, there is at least one point outside of line $AB$ - we'll mark it with $D$
  (Figure \ref{sl.aks.2.2.5.pic}). Next, by
Axiom \ref{AksII3} there is such a point $E$, that $\mathcal{B}(B,D,E)$ is true, and then such a point $F$, that $\mathcal{B}(A,E,F)$ is true. $A$, $B$ and $E$ are non-linear points,
because otherwise point $D$ would lie on line $AB$ (Axiom \ref{AksI1}).
Line $FD$ does not go through point $A$, because by Axiom \ref{AksI1} points $F$, $D$, $A$ and $E$
would be linear, and so would point $B$. But that is not possible, because it would imply that point $D$ lies on line $AB$.
We'll now use Pasch's Axiom \ref{AksPascheva} on
points $A$, $B$ and $E$ and line $FD$. Line $FD$ intersects line $EB$ in such a point $D$,
that $\mathcal{B}(B,D,E)$ is true, and therefore it intersects either line $AE$ in such a point $F$, that $\mathcal{B}(A,F,E)$ is true, or
line $AB$ in such a point $C$, that $\mathcal{B}(A,C,B)$ is true. But since $\mathcal{B}(A,E,F)$ is already true, by Axiom \ref{AksII2} $\mathcal{B}(A,F,E)$ cannot be true as well. Therefore line $FD$ intersects
line $AB$ in such a point $C$, for which $\mathcal{B}(A,C,B)$ is true.
\kdokaz

Relation $\mathcal{B}$ and the order axioms related to it, allow us to define new concepts.


 Let $A$ and $B$ be any two different points.
  \index{distance!open}\pojem{Open distance} $AB$ with notation $(AB)$ is the set of all points
  $X$, for
  which $\mathcal{B}(A,X,C)$ is true (Figure \ref{sl.aks.2.2.6.pic}).

\begin{figure}[!htb]
\centering
\input{sl.aks.2.2.6.pic}
\caption{} \label{sl.aks.2.2.6.pic}
\end{figure}

If we add points $A$ and $B$ to an open line segment $AB$, we get a \pojem{line segment} (or a \pojem{closed line segment}) $AB$, which we also denote with $[AB]$. Points $A$ and $B$ are its \pojem{endpoints}, and all other points on it are \pojem{interior points} of the line segment (Figure \ref{sl.aks.2.2.6.pic}). More formally: a line segment (or a closed line segment) is the union of an open line segment and the set $\{A,B\}$ or $[AB]=(AB)\cup \{A,B\}$.

Similarly, we define a \pojem{half-open line segment}: $(AB]=(AB)\cup \{B\}$, or $[AB)=(AB)\cup \{A\}$ (Figure \ref{sl.aks.2.2.6a.pic}).


\begin{figure}[!htb]
\centering
\input{sl.aks.2.2.6a.pic}
\caption{} \label{sl.aks.2.2.6a.pic}
\end{figure}

  From Axiom \ref{AksII1} it follows immediately that line segments $AB$ and $BA$ are the same. From the same axiom it also follows that line segment $AB$ is a subset of the line $AB$. Therefore, we say that line segment $AB$ \pojem{lies on the line} $AB$, and we call the line $AB$ the \pojem{line supporting the line segment} $AB$  (Figure \ref{sl.aks.2.2.6b.pic}).


\begin{figure}[!htb]
\centering
\input{sl.aks.2.2.6b.pic}
\caption{} \label{sl.aks.2.2.6b.pic}
\end{figure}

By Theorem \ref{izrekAksUrACB}, line segment $AB$ has, besides its endpoints $A$ and $B$, at least one more point $C_1$. In this way, we can get an infinite sequence of points $C_1$, $C_2$, ..., for which $\mathcal{B}(A, C_n, C_{n-1})$ is true ($n\in \{2,3,\cdots\}$)  (Figure \ref{sl.aks.2.2.6c.pic}). At this point, we will not formally prove the fact that all points in the sequence are different and that all of them lie on line segment $AB$. From this statement it follows that every line segment (and consequently every line) has infinitely many points.

\begin{figure}[!htb]
\centering
\input{sl.aks.2.2.6c.pic}
\caption{} \label{sl.aks.2.2.6c.pic}
\end{figure}

Let's define the relation $\mathcal{B}$, which relates to more than three collinear points. We say that $\mathcal{B}(A_1,A_2,\ldots,A_n)$ ($n\in\{4,5,\ldots\}$), if for every $k\in\{1,2,\ldots,n-2\}$ it holds that $\mathcal{B}(A_k,A_{k+1},A_{k+2})$ (Figure \ref{sl.aks.2.2.6d.pic}).


\begin{figure}[!htb]
\centering
\input{sl.aks.2.2.6d.pic}
\caption{} \label{sl.aks.2.2.6d.pic}
\end{figure}

 Let $S$ be a point that lies on the line $p$. On the set $p\setminus \{S\}$ (all points of the line $p$ without the point $S$) we define two relations.
We say that the points $A$ and $B$ ($A,B\in p\setminus \{S\}$) \index{relacija!na različnih straneh točke} \pojem{on different sides of the point} $S$ (which we denote by $A,B\div S$), if $B(A,S,B)$, otherwise the points $A$ and $B$ ($A,B\in p\setminus \{S\}$) \index{relacija!na isti strani točke} \pojem{on the same side of the point} $S$ (which we denote by $A,B\ddot{-} S$). So for points $A,B\in p\setminus \{S\}$ it holds that $A,B\ddot{-} S$, if it is not $A,B\div S$  (Figure \ref{sl.aks.2.2.7.pic}).


\begin{figure}[!htb]
\centering
\input{sl.aks.2.2.7.pic}
\caption{} \label{sl.aks.2.2.7.pic}
\end{figure}


Let $A$ and $B$ be two different points. The set of all such points $X$, for which $B,X\ddot{-} A$ including the point $A$, we call \index{poltrak}\pojem{poltrak} $AB$ with \pojem{starting point} or \pojem{origin} $A$. The line $AB$ is
\index{nosilka!poltraka}  \pojem{the carrier of the poltrak} $AB$ (Figure \ref{sl.aks.2.2.8.pic}).


\begin{figure}[!htb]
\centering
\input{sl.aks.2.2.8.pic}
\caption{} \label{sl.aks.2.2.8.pic}
\end{figure}

From the definition itself it follows that the poltrak is a subset of its carrier or that it lies on its carrier. From the relation $B,X\ddot{-} A$ it follows that $B$, $X$ and $A$ are collinear points, so the point $X$ lies on the line $AB$.

We will not prove other important properties of lines and line segments that we will use later. Let's mention some of these properties.

If $C$ is the an interior point of the line segment $AB$, then that line segment can be expressed as the union
the line segments $AC$ and $CB$.

            \bizrek \label{izrekAksIIDaljica}
            If $C$ is an interior point of the line segment $AB$, then that line segment can be expressed as the union
            the line segments $AC$ and $CB$ (Figure \ref{sl.aks.2.2.9.pic}).
            \eizrek

\begin{figure}[!htb]
\centering
\input{sl.aks.2.2.9.pic}
\caption{} \label{sl.aks.2.2.9.pic}
\end{figure}


            \bizrek \label{izrekAksIIPoltrak}
            Each point lying on the line determines exactly
            two rays on it. The union of these rays is equal to that line  (Figure \ref{sl.aks.2.2.9.pic}).
            \eizrek

The proof of the previous statement is based on the fact that the relation $\ddot{-} A$ is equivalent to the relation that has two classes. Each of the classes is a suitable open ray.

The rays from the previous statement, which are determined by the same initial point on the line, are called \index{poltrak!komplementarni}\pojem{complementary rays}.

The concepts of line and ray allow us to define new concepts.

Let $A_1$, $A_2$, ... $A_n$ be such points in the plane that no three of them are collinear. The union of the lines $A_1A_2$, $A_2A_3$,... $A_{n-1}A_n$ is called \index{lomljenka} \pojem{broken line} $A_1A_2\cdots A_n$ or \index{poligonska
črta}\pojem{polygonal line} $A_1A_2\cdots A_n$ (Figure \ref{sl.aks.2.2.10.pic}). The points  $A_1$, $A_2$, ... $A_n$ are \index{oglišče!lomljenke} \pojem{vertices of the broken line},
the lines $A_1A_2$, $A_2A_3$,... $A_{n-1}A_n$ are \index{stranica!lomljenke} \pojem{sides of the broken line}. The sides of the broken line with a common vertex are \index{sosednji stranici!lomljenke} \pojem{adjacent sides of the broken line}.


\begin{figure}[!htb]
\centering
\input{sl.aks.2.2.10.pic}
\caption{} \label{sl.aks.2.2.10.pic}
\end{figure}

If the sides of a broken line do not have common points, except for adjacent sides that have a common vertex, such a broken line is called a \index{lomljenka!enostavna} \pojem{simple broken line} (Figure \ref{sl.aks.2.2.10a.pic}).

\begin{figure}[!htb]
\centering
\input{sl.aks.2.2.10a.pic}
\caption{} \label{sl.aks.2.2.10a.pic}
\end{figure}

A broken line $A_1A_2\cdots A_nA_{n+1}$, for which $A_{n+1}=A_1$ and $A_n$, $A_1$ and $A_2$ are non-linear points, is called a \index{lomljenka!sklenjena} \pojem{closed broken line} $A_1A_2\cdots A_n$ (Figure \ref{sl.aks.2.2.10b.pic}).

\begin{figure}[!htb]
\centering
\input{sl.aks.2.2.10b.pic}
\caption{} \label{sl.aks.2.2.10b.pic}
\end{figure}

We will be particularly interested in \pojem{simple closed broken lines} (Figure \ref{sl.aks.2.2.10b.pic}).

Let $p$ and $q$ be two semi-lines with a common starting point $O$ (Figure \ref{sl.aks.2.2.10c.pic}). The union of these two semi-lines is called a \index{kotna lomljenka} \pojem{angular broken line} $pq$ (or $pOq$).

\begin{figure}[!htb]
\centering
\input{sl.aks.2.2.10c.pic}
\caption{} \label{sl.aks.2.2.10c.pic}
\end{figure}


For a figure $\Phi$ we say that it is \index{lik!konveksen}\pojem{convex}, if for any two of its points $A,B\in \Phi$ the distance $AB$ is a subset of this figure or if the following is true (Figure \ref{sl.aks.2.2.10d.pic}):
 $$(\forall A)(\forall B)\hspace*{1mm} (A,B\in \Phi \Rightarrow [AB]\subseteq \Phi).$$
For a figure that is not convex, we say that it is \index{lik!nekonveksen}\pojem{non-convex} (Figure \ref{sl.aks.2.2.10d.pic}).


\begin{figure}[!htb]
\centering
\input{sl.aks.2.2.10d.pic}
\caption{} \label{sl.aks.2.2.10d.pic}
\end{figure}

It follows directly from the definition that a straight line is a convex figure. As a result of the axioms of this group, it can be proved that a distance and a semi-line are convex figures.

We say that a figure $\Phi$ is
\index{figure!connected}\pojem{connected}, if for every two of its points $A,B\in \Phi$ there exists a broken line $AT_1T_2\cdots T_nB$, which is a subset of this figure, or if the following is true (Figure \ref{sl.aks.2.2.10e.pic}):
 $$(\forall A\in \Phi)(\forall B\in \Phi)(\exists T_1,T_2,\cdots , T_n)\hspace*{1mm}  AT_1T_2\cdots T_nB\subseteq \Phi.$$


\begin{figure}[!htb]
\centering
\input{sl.aks.2.2.10e.pic}
\caption{} \label{sl.aks.2.2.10e.pic}
\end{figure}

 A figure that is not connected, we call \index{figure!not connected}\pojem{not connected}.

 It is clear that every convex figure is also connected. For the broken line it is enough to take the distance $AB$. The converse is of course not true. There are figures that are connected, but not convex, which we will discover later.



Now we will define two relations that are analogous to the relations $\ddot{-} S$ and $\div S$.
 Let $p$ be a line that lies in the plane $\alpha$ (because we axiomatically build only the Euclidean geometry of the plane, in fact all the points that exist to us are in this plane). On the set $\alpha\setminus p$ (all points except the points of the line $p$) we define two relations.
We say that the points $A$ and $B$ ($A,B\in \alpha\setminus p$) are \index{relation!on different sides of the line} \pojem{on different sides of the line} $p$ (which we denote by $A,B\div p$), if the line $AB$ has a common point with the line $p$, otherwise the points $A$ and $B$ ($A,B\in \alpha\setminus p$) are \index{relation!on the same side of the line} \pojem{on the same side of the line} $p$ (which we denote by $A,B\ddot{-} p$). So for points $A,B\in \alpha\setminus p$ it is $A,B\ddot{-} p$, if it is not $A,B\div p$ (Figure \ref{sl.aks.2.2.11.pic}).

\begin{figure}[!htb]
\centering
\input{sl.aks.2.2.11.pic}
\caption{} \label{sl.aks.2.2.11.pic}
\end{figure}

Let $A$ be a point that does not lie on the line $p$. The set of all such points $X$, for which $A,X\ddot{-} p$, is called
\index{polravnina!odprta}\pojem{open half-line} $pA$. The union of the open half-lines $pA$ and the line $p$ is the \index{polravnina!zaprta}\pojem{closed half-line} or just
\index{polravnina}\pojem{half-line} $pA$. The line $p$ is the \index{rob!polravnine} \pojem{edge} of this half-line (Figure \ref{sl.aks.2.2.11a.pic}). If the points $B$ and $C$ lie on the edge $p$ of the half-line $pA$, we will call this half-line the half-line $BCA$. In addition, we will denote half-lines by Greek letters $\alpha$, $\beta$, $\gamma$,...

\begin{figure}[!htb]
\centering
\input{sl.aks.2.2.11a.pic}
\caption{} \label{sl.aks.2.2.11a.pic}
\end{figure}

Similarly to the case of a segment, it can be shown (as a consequence of the axioms of this group), that each line $p$ in the plane determines two half-lines $\alpha$ and $\alpha'$, which have the line $p$ as an edge (Figure \ref{sl.aks.2.2.11a.pic}). We say that in this case $\alpha$ and $\alpha'$ are \index{polravnina!komplementarna}\pojem{complementary half-lines}.
It turns out that the union of two complementary half-lines is the whole plane. Similarly to the case of a segment, the proof of these statements is based on the fact that the relation $\ddot{-} p$ is equivalent to the relation with two classes. Each of the classes is the appropriate open half-line.

Let $pq$ or $pOq$ be an angle. Define a new relation on the set of all points of the plane except the points that lie on the angle. We say that the points $A$ and $B$ are on the same side of the angle $pq$ (which we denote by $A,B\ddot{-} pq$), if there exists an angle $AT_1T_2\cdots T_nB$, which does not intersect the angle $pq$ or does not have any common points with it (Figure \ref{sl.aks.2.2.12.pic}).


\begin{figure}[!htb]
\centering
\input{sl.aks.2.2.12.pic}
\caption{} \label{sl.aks.2.2.12.pic}
\end{figure}

The relation $\ddot{-} pq$ is also an equivalence relation that has two classes. The union of each of these two classes with the angular bracket $pq$ is called
 \index{kot}\pojem{the angle} $pq$, which is denoted by $\angle pq$, or $\angle pOq$.
  The angular bracket therefore determines two angles. We will soon resolve the dilemma of which angle is meant by the notation  $\angle pOq$.
 The segments $p$ and $q$ are
\index{krak!kota}\pojem{the sides of the angle} and the point $O$ \index{vrh kota}\pojem{the vertex of the angle}.
If $P\in p$ and $Q\in q$ are points that lie on the sides of the angle $pOq$ and differ from its vertex $O$, we will also call the angle $POQ$ and denote it by $\angle POQ$ (Figure \ref{sl.aks.2.2.12a.pic}). If we know which angle it is, we will denote it by its vertex: $\angle O$. We will also denote angles by Greek letters $\alpha$, $\beta$, $\gamma$,...

All points of the angle $pOq$, which do not lie on either of the sides $p$ and $q$, are called \index{notranje točke!kota} \pojem{internal points of the angle}, the set of all these points is called \index{notranjost!kota}\pojem{the interior of the angle}. It is clear that these are points of the appropriate class determined by the relation $\ddot{-} pq$. The points of the other class are \index{zunanje!točke kota}\pojem{external points of the angle}, the whole class is \index{zunanjost!kota}\pojem{the exterior of the angle}. The points that lie on the sides, or on the angular bracket $pOq$, which determines the angle $pOq$, are \index{robne točke!kota}\pojem{the boundary points of the angle}, the whole bracket is \index{rob!kota}\pojem{the boundary of the angle}.

\begin{figure}[!htb]
\centering
\input{sl.aks.2.2.12a.pic}
\caption{} \label{sl.aks.2.2.12a.pic}
\end{figure}

If the sides of the angle are complementary segments, such an angle is called
\index{kot!iztegnjeni}\pojem{an extended angle} (Figure \ref{sl.aks.2.2.12b.pic}). As a set of points, this angle is essentially the same as the half-line with the boundary that is the carrier of both sides.


\begin{figure}[!htb]
\centering
\input{sl.aks.2.2.12b.pic}
\caption{} \label{sl.aks.2.2.12b.pic}
\end{figure}

If the angular bisector $pOq$ does not determine the extended angle or is not equal to the line, it turns out that $pOq$ determines two angles that represent a convex and a concave shape - we call them the \index{kot!konveksen}\pojem{convex (protruding) angle} and the \index{kot!nekonveksen}\pojem{concave (indented) angle}. We will omit the formal proof of this fact here. Unless otherwise stated, we will always mean the convex angle under the label $\angle pOq$ (or $\angle pq$ or $\angle POQ$). In this sense, it is clear from the definition of the angle that (convex) angle $pOq$ and $qOp$ represent the same angle.

The angle $pOq$ and $qOr$, which have a common leg $q$, which is also their intersection (as a set of points), are \index{kot!sosednji}\pojem{adjacent angles} (Figure \ref{sl.aks.2.2.12c.pic}). If in addition the segments $p$ and $r$ are complementary (or determine the extended angle), we say that $pOq$ and $qOr$ are \index{kota!sokota}\pojem{sokota}  (Figure \ref{sl.aks.2.2.12c.pic}).

\begin{figure}[!htb]
\centering
\input{sl.aks.2.2.12c.pic}
\caption{} \label{sl.aks.2.2.12c.pic}
\end{figure}

The angle $pOq$ and $rOs$ are \index{kota!sovršna}\pojem{sovršna kota}, if $p$ and $r$ or $q$ and $s$ are a pair of complementary (supplementary) segments (Figure \ref{sl.aks.2.2.12d.pic}).

\begin{figure}[!htb]
\centering
\input{sl.aks.2.2.12d.pic}
\caption{} \label{sl.aks.2.2.12d.pic}
\end{figure}

 Let $A_1A_2\cdots A_n$ ($n\in \{3,4,5,\cdots\}$) be a simple closed curve.
 Similarly to the angular bisector, on the set of all points in the plane except for the points that lie on the curve $A_1A_2\cdots A_n$, we can define the following relation: we say that the points $B$ and $C$ are on the same side of the simple closed curve $A_1A_2\cdots A_n$ (which we denote by $B,C\ddot{-} A_1A_2\cdots A_n$), if there exists a curve $BT_1T_2\cdots T_nC$, which does not intersect the simple closed curve $A_1A_2\cdots A_n$ or does not have any common points with it (Figure \ref{sl.aks.2.2.13.pic}).

\begin{figure}[!htb]
\centering
\input{sl.aks.2.2.13.pic}
\caption{} \label{sl.aks.2.2.13.pic}
\end{figure}

 It can also be proven in this case that it is an equivalence relation with two classes - for one class there is a line that lies entirely within it, but for the other class there is no such line. The union of the class that does not contain any line (intuitively - the one that is limited) and the simple closed broken line, $A_1A_2\cdots A_n$ is called
\index{polygon}\pojem{polygon} $A_1A_2\cdots A_n$ or
\index{$n$-gon}\pojem{$n$-gon} $A_1A_2\cdots A_n$ (Figure \ref{sl.aks.2.2.13a.pic}). All points of the aforementioned class that do not contain any line are called \index{internal points!of a polygon} \pojem{internal points of a polygon}, the entire class is \index{interior!of a polygon}\pojem{interior of a polygon}. The points of the other class are \index{external!points of a polygon}\pojem{external points of a polygon}, the entire class is \index{exterior!of a polygon}\pojem{exterior of a polygon}. The points that lie on the broken line $A_1A_2\cdots A_n$, which determines the polygon, are \index{vertices!of a polygon}\pojem{vertices of a polygon}, the entire broken line is \index{border!of a polygon}\pojem{border of a polygon}.


\begin{figure}[!htb]
\centering
\input{sl.aks.2.2.13a.pic}
\caption{} \label{sl.aks.2.2.13a.pic}
\end{figure}


An equivalent statement holds true for the internal points of a polygon. We will state the statement without proof.

            \bizrek
            A point $N$ is an interior point of a polygon if and only
            if any ray from the point $N$, that does not contain the vertices of the
            polygon, intersects an odd number of sides of the polygon (Figure \ref{sl.aks.2.2.13b.pic}).
            \eizrek

\begin{figure}[!htb]
\centering
\input{sl.aks.2.2.13b.pic}
\caption{} \label{sl.aks.2.2.13b.pic}
\end{figure}

The points $A_1$, $A_2$,..., $A_n$ (the vertices of the broken line) are the \index{oglišče!večkotnika}\pojem{vertices of the polygon}, the segments $A_1A_2$, $A_2A_3$, ... $A_{n-1}A_n$, $A_nA_1$ are the \index{stranica!večkotnika}\pojem{sides of the polygon}. The lines $A_1A_2$, $A_2A_3$, ... $A_{n-1}A_n$, $A_nA_1$ are the \index{nosilka!stranice}\pojem{supports of the sides} $A_1A_2$, $A_2A_3$, ... $A_{n-1}A_n$, $A_nA_1$. The sides that contain a common vertex are the \index{stranica!sosednja}\pojem{adjacent sides}, otherwise the sides are \index{stranica!nesosednja}\pojem{non-adjacent}. If the vertices are at the same time the endpoints of the same side, we say that the vertices are \index{oglišče!sosednje}\pojem{adjacent}, otherwise the vertices are \index{oglišče!nesosednje}\pojem{non-adjacent}.
It is clear from the definition that each vertex has exactly two adjacent vertices. Similarly, each side has exactly two adjacent sides.
 The segment determined by two non-adjacent vertices is called the
\index{diagonala!večkotnika}\pojem{diagonal of the polygon} (Figure \ref{sl.aks.2.2.13c.pic}).

\begin{figure}[!htb]
\centering
\input{sl.aks.2.2.13c.pic}
\caption{} \label{sl.aks.2.2.13c.pic}
\end{figure}

The following statement is true for the diagonals of the polygon.

            \bizrek
            The number of diagonals of an $n$-gon is $\frac{n(n-3)}{2}$.
            \eizrek

We will give the proof of this statement in section \ref{odd3Helly}, where we will separately consider the combinatorial properties of sets of points in the plane.

Let's define the angles of a polygon. Let $O$ be any vertex of a polygon, and $P$ and $Q$ be its two adjacent vertices. The line segments $OP$ and $OQ$ are denoted by $p$ and $q$. In this case, the angular bisector $pOq$ determines two angles. The angle for which it is true that every line segment with initial point $O$, which belongs to this angle and does not contain any other vertices of the polygon, intersects the edge of the polygon except at point $O$ in an odd number of points, is called the \index{angle!inner polygon}\pojem{inner angle of the polygon} or, more briefly, the \index{angle!inner}\pojem{angle of the polygon} at vertex $O$ (Figure \ref{sl.aks.2.2.13d.pic}). If the inner angle of the polygon is convex, its supplement is called the \index{angle!outer polygon}\pojem{outer angle of the polygon} (Figure \ref{sl.aks.2.2.13d.pic}).
  The angles of the polygon are
  \index{angle!adjacent}\pojem{adjacent angles}, if their vertices are adjacent vertices of the polygon, otherwise the angles are \index{angle!non-adjacent}\pojem{non-adjacent}.


\begin{figure}[!htb]
\centering
\input{sl.aks.2.2.13d.pic}
\caption{} \label{sl.aks.2.2.13d.pic}
\end{figure}

The most simple $n$-gon and at the same time one of the most used shapes in the geometry of a plane is the case $n=3$ - \index{trikotnik}\pojem{trikotnik}.
In the case of the triangle $ABC$ (we will mark it with $\triangle ABC$), the points $A$, $B$ and $C$ are its \pojem{oglišča}, and the lines $AB$, $BC$ and $CA$ are its
\index{stranica!trikotnika}\pojem{stranice} (Figure \ref{sl.aks.2.2.14.pic}). Obviously, every two sides of the triangle are adjacent. The same goes for every two vertices. The triangle therefore has no diagonal. We say that the vertex $A$ ($B$ and $C$) or the angle $BAC$ ($ABC$ and $ACB$) is the \index{oglišče!nasprotno trikotnika}\pojem{nasprotno oglišče} or the \index{kot!nasprotni trikotnika}\pojem{nasprotni kot} of the side $BC$ ($AC$ and $AB$) of the triangle $ABC$. And also the side $BC$ ($AC$ and $AB$) is the \index{stranica!nasprotna trikotnika}\pojem{nasprotna stranica} of the vertex $A$ ($B$ and $C$) or the angle $BAC$ ($ABC$ and $ACB$) of the triangle $ABC$. The angles (internal) of the triangle $ABC$ at the vertices $A$, $B$ and $C$ are often marked with $\alpha$, $\beta$ and $\gamma$, the corresponding external ones with $\alpha_1$, $\beta_1$ and $\gamma_1$.

\begin{figure}[!htb]
\centering
\input{sl.aks.2.2.14.pic}
\caption{} \label{sl.aks.2.2.14.pic}
\end{figure}

The triangle $ABC$ can also be defined as the intersection of the half-planes $ABC$, $ACB$ and $BCA$. We will not prove the equivalence of these two definitions here.

 Pasch's axiom in terms of triangles can now be expressed in a shorter form:

If a line in the plane of a triangle intersects one of its sides
and does not pass through any of its vertices, then intersects exactly one more side of this triangle



             \bizrek \label{PaschIzrek}
           If a line, not passing through any vertex of a triangle, intersects one side of the triangle
           then the line intersects exactly one more side of the triangle (Figure \ref{sl.aks.2.2.14a.pic}).
             \eizrek

\begin{figure}[!htb]
\centering
\input{sl.aks.2.2.14a.pic}
\caption{} \label{sl.aks.2.2.14a.pic}
\end{figure}

In the case of $n=4$ for the $n$-gon we get
\index{štirikotnik}\pojem{štirikotnik}. Because each vertex of the quadrilateral has exactly one non-adjacent vertex, we will also call this vertex the \index{oglišče!nasprotno štirikotnika}\pojem{opposite vertex} of the quadrilateral. Similarly, the non-adjacent sides will be the \index{stranica!nasprotna štirikotnika}\pojem{opposite sides} of the quadrilateral. The diagonal of the quadrilateral is therefore determined by the opposite vertices.
 From the definition itself, it is clear that the quadrilateral has two diagonals\index{diagonala!štirikotnika} (Figure \ref{sl.aks.2.2.14b.pic}).

\begin{figure}[!htb]
\centering
\input{sl.aks.2.2.14b.pic}
\caption{} \label{sl.aks.2.2.14b.pic}
\end{figure}

For the non-adjacent angle of the quadrilateral, we also say that they are the
 \index{kot!nasprotni štirikotnika}\pojem{opposite angles} of the quadrilateral. The angles (internal) of the quadrilateral $ABCD$ at the vertices $A$, $B$, $C$ and $D$ are usually denoted by $\alpha$, $\beta$, $\gamma$ and $\delta$, and the corresponding external (those that exist) by $\alpha_1$, $\beta_1$, $\gamma_1$ and $\delta_1$ (Figure \ref{sl.aks.2.2.14c.pic}).

\begin{figure}[!htb]
\centering
\input{sl.aks.2.2.14c.pic}
\caption{} \label{sl.aks.2.2.14c.pic}
\end{figure}



As a result of the axioms of order, we will also introduce the concepts of orientation of a triangle and orientation of an angle in this section.

The triangle $ABC$, for which the vertices are a triple $(A,B,C)$, is called the
  \index{orientacija!trikotnika} \pojem{oriented triangle}.
We say that the oriented triangles $ABC$ and $BCA'$ are of the \pojem{same orientation}, if $A,A'\ddot{-} BC$, and of the opposite orientation, if $A,A'\div BC$
 (Figure \ref{sl.aks.2.2.15.pic}).

\begin{figure}[!htb]
\centering
\input{sl.aks.2.2.15.pic}
\caption{} \label{sl.aks.2.2.15.pic}
\end{figure}

When we talk about the orientation of two triangles, from now on we will always mean an oriented triangle (we will often omit the word oriented). Triangles $ABC$ and $A'B'C'$ are of the \pojem{same orientation} or are \pojem{equally oriented}, if there exists such a sequence of triangles: $\triangle ABC=\triangle P_1P_2P_3$, $\triangle P_2P_3P_4$, $\triangle P_3P_4P_5$, ..., $\triangle P_{n-2}P_{n-1}P_n=\triangle A'B'C'$, that in this sequence the number of changes in orientation of two adjacent triangles is even
 (Figure \ref{sl.aks.2.2.15a.pic}).

\begin{figure}[!htb]
\centering
\input{sl.aks.2.2.15a.pic}
\caption{} \label{sl.aks.2.2.15a.pic}
\end{figure}

It can be proven that the relation of the same orientation of two triangles is equivalent to a relation that has two classes. For two triangles that are not in the same class, we say that they are of \pojem{different orientation} or are \pojem{differently oriented}. Each of the two classes determines the \index{orientation!plane}\pojem{orientation of the plane}. We call them the \pojem{positive orientation} and the \pojem{negative orientation}. For the sake of easier understanding, let us agree that the orientation that corresponds to the direction of the rotation of the hour hand is negative, and the opposite is the positive orientation
 (Figure \ref{sl.aks.2.2.15b.pic}).

\begin{figure}[!htb]
\centering
\input{sl.aks.2.2.15b.pic}
\caption{} \label{sl.aks.2.2.15b.pic}
\end{figure}


Let us also define the orientation of angles. Angles $ASB$ and $A'S'B'$, neither of which is a right angle, are of the \pojem{same orientation}, if:
\begin{itemize}
  \item both are convex or both are concave, and triangles $ASB$ and $A'S'B'$ are of the same orientation (Figure \ref{sl.aks.2.2.16.pic}),
  \item one angle is convex and the other is concave, and triangles $ASB$ and $A'S'B'$ are of opposite orientation (Figure \ref{sl.aks.2.2.16c.pic}).
\end{itemize}

\begin{figure}[!htb]
\centering
\input{sl.aks.2.2.16.pic}
\caption{} \label{sl.aks.2.2.16.pic}
\end{figure}

\begin{figure}[!htb]
\centering
\input{sl.aks.2.2.16c.pic}
\caption{} \label{sl.aks.2.2.16c.pic}
\end{figure}

 If $\angle ASB$ is an obtuse angle, $\angle A'S'B'$ is a convex angle, then the angles $ASB$ and $A'S'B'$ have the same orientation, if there is a point $C$ inside the angle $ASB$, such that the angles $ASC$ and $A'S'B'$ have the same orientation (Figure \ref{sl.aks.2.2.16d.pic}). We do the same if the angle $A'S'B'$ is obtuse or if both angles $ASB$ and $A'S'B'$ are obtuse.

\begin{figure}[!htb]
\centering
\input{sl.aks.2.2.16d.pic}
\caption{} \label{sl.aks.2.2.16d.pic}
\end{figure}


It turns out that the relation of the same orientation of angles is an equivalent relation that has two classes. In this case, the positive orientation of the angle represents the class in which the triangle $ASB$ has a negative orientation for the convex angle $ASB$ from this class (Figure \ref{sl.aks.2.2.16a.pic}). In this sense, the angles $ASB$ and $BSA$ are oriented in the opposite direction.
   \index{orientation!angle} \pojem{Oriented angle} $ASB$  we will denote with $\measuredangle ASB$.


\begin{figure}[!htb]
\centering
\input{sl.aks.2.2.16a.pic}
\caption{} \label{sl.aks.2.2.16a.pic}
\end{figure}

\begin{figure}[!htb]
\centering
\input{sl.aks.2.2.16b.pic}
\caption{} \label{sl.aks.2.2.16b.pic}
\end{figure}

If $C$ is an arbitrary point that does not lie on the edge of the angle $ASB$, we will define the sum of the oriented angles $\measuredangle ASC$ and $\measuredangle CSB$
 (Figure \ref{sl.aks.2.2.16b.pic}):
 \begin{eqnarray}
 \measuredangle ASC+\measuredangle CSB = \measuredangle ASB.
 \label{orientKotVsota}
 \end{eqnarray}


%________________________________________________________________________________
 \poglavje{Congruence Axioms}
 \label{odd2AKSSKL}


The following axioms are needed to introduce the concept and properties of
the congruence of figures. With the previous axioms, we could introduce and
consider the concepts: distance, segment, angle, polygon, ... but not
the concepts related to congruence: circle, right angle,
congruence of triangles,~...

The intuitive idea of the compatibility of shapes that we used in
elementary school is associated with the movement that the first
shape transforms into the other. We will now use this idea to
formally define the concept of compatibility and its properties.
We will first start with the basic, already mentioned, concept of
compatibility of pairs of points $(A,B)\cong (C,D)$ (Figure
\ref{sl.aks.2.3.1.pic}) and formally define the concept of
‘‘movement’’.

\begin{figure}[!htb]
\centering
\input{sl.aks.2.3.1.pic}
\caption{} \label{sl.aks.2.3.1.pic}
\end{figure}

 Using the compatibility of pairs of points, we first define the compatibility of an $n$-tuple of points.
 We say that
two $n$-tuples of points are compatible (Figure \ref{sl.aks.2.3.2.pic}) or
$$(A_1 , A_2,\ldots ,A_n ) \cong ( A'_1 , A'_2 ,\ldots , A'_n ),$$
if: $(A_i,A_j)\cong (A'_i,A'_j)$ for every $i,j\in \{1,2,\ldots,
n\}$.

\begin{figure}[!htb]
\centering
\input{sl.aks.2.3.2.pic}
\caption{} \label{sl.aks.2.3.2.pic}
\end{figure}


A bijective mapping of a plane into a plane
$\mathcal{I}:\mathcal{S}\rightarrow \mathcal{S}$ is
\index{izometrija}\pojem{izometrija} or \pojem{izometrijska
transformacija}, if it preserves the relation of compatibility of
pairs of points (Figure \ref{sl.aks.2.3.3.pic}) or if for every two
points $A$ and $B$ it holds:
 $$(\mathcal{I}(A),\mathcal{I}(B))\cong (A,B).$$

\begin{figure}[!htb]
\centering
\input{sl.aks.2.3.3.pic}
\caption{} \label{sl.aks.2.3.3.pic}
\end{figure}


With the following axioms, we will introduce the properties of the
newly defined mapping.


\vspace*{3mm}


            \baksiom \label{aksIII1} Isometries preserve the relation
            $\mathcal{B}$  (Figure \ref{sl.aks.2.3.4.pic}), which means that for every
              isometry $\mathcal{I} $ holds:
            $$\mathcal{I}: A, B,C\mapsto A',B',C'\hspace*{2mm}\wedge \hspace*{2mm}
            \mathcal{B}(A,B,C)
             \hspace*{1mm}\Rightarrow\hspace*{1mm} \mathcal{B}(A',B',C').$$
             \eaksiom

\begin{figure}[!htb]
\centering
\input{sl.aks.2.3.4.pic}
\caption{} \label{sl.aks.2.3.4.pic}
\end{figure}


            \baksiom  \label{aksIII2} If $ABC$ and $A'B'C'$ are two half-planes
             (Figure \ref {asl.aks.2.3.5.pic}), then there is a single isometry
             $\mathcal{I}$, which maps:

            \begin{itemize}
            \item point $A$ to point $A'$,
               \item half-line $AB$ in half-line $A'B'$,
              \item half-plane $ABC$ to half-plane $A'B'C'$.
            \end{itemize}
            If $(A,B)\cong (A',B')$ holds,
            then it is $\mathcal{I}(B)=B'$.
            \\ If in addition
            $(A,B,C)\cong (A',B',C')$ also holds,
             then it is $\mathcal{I}(B)=B'$ and $\mathcal{I}(C)=C'$.
            \eaksiom


\begin{figure}[!htb]
\centering
\input{sl.aks.2.3.5.pic}
\caption{} \label{asl.aks.2.3.5.pic}
\end{figure}

            \baksiom  \label{aksIII3} For every two points $A$ and $B$ there exists
             isometry such that holds $$\mathcal{I}: A, B\mapsto B,A.$$
             If $(S,A)\cong (S,B)$ and $S\in AB$, then for each isometry $\mathcal{I}$ with this property  holds $\mathcal{I}(S)=S$ (Figure \ref{sl.aks.2.3.6.pic}).
            \eaksiom

\begin{figure}[!htb]
\centering
\input{sl.aks.2.3.6.pic}
\caption{} \label{sl.aks.2.3.6.pic}
\end{figure}

            \baksiom  \label{aksIII4} The set of all isometries with respect to the composition of mappings form a group, which means that:
            \begin{itemize}
            \item composition of two isometries  $\mathcal{I}_2\circ \mathcal{I}_1$ is isometry,
            \item identity map $\mathcal{E}$ is isometry,
            \item if $\mathcal{I}$ is  isometry, then its inverse transformation
            $\mathcal{I}^{-1}$ is also isometry.
            \end{itemize}
             \eaksiom

\vspace*{3mm}

We mention that in the structure of a group the property of associativity is also required, i.e. $\mathcal{I}_1\circ (\mathcal{I}_2\circ \mathcal{I}_3)=
  (\mathcal{I}_1\circ \mathcal{I}_2)\circ \mathcal{I}_3$ (for any isometries
  $\mathcal{I}_1$, $\mathcal{I}_2$ and $\mathcal{I}_3$), which is automatically fulfilled
  in the operation of the composition of functions. We also mention the \pojem{identity} \index{identity}
 $\mathcal{E}$ from the previous axiom of the mapping, for which
 $\mathcal{E}(A)=A$ for every point on the plane. The mapping
 $\mathcal{I}^{-1}$ is the \pojem{inverse mapping} for the isometry
 $\mathcal{I}$, if $\mathcal{I}^{-1}\circ \mathcal{I}
 =\mathcal{I}\circ\mathcal{I}^{-1}=\mathcal{E}$. According to the previous
 axiom, the identity and the inverse mapping are therefore also isometries of every isometry.



We prove the first consequences of the compatibility axioms. First, we will
consider the following properties of isometries.



            \bizrek \label{izrekIzoB} Isometry maps a line to a line, a line segment to a line segment, a ray to a ray,
            a half-plane to a half-plane, an angle to an angle and an $n$-gon to an $n$-gon.
             \eizrek

\textbf{\textit{Proof.}}
 According to axiom \ref{aksIII1}, isometries preserve the relation
 $\mathcal{B}$. Therefore, all points of the line $AB$ are mapped by isometry $I$
  into points that lie on the line $A'B'$, where $A'=\mathcal{I}(A)$ and
  $B'=\mathcal{I}(B)$. Since the inverse mapping $\mathcal{I}^{-1}$ is also an isometry (axiom \ref{aksIII4}), each point of the line
  $A'B'$ is the image of some point that lies on the line $AB$. So with
  isometry $\mathcal{I}$
  the line $AB$ is mapped into the line $A'B'$.

 We defined the other shapes from the statement using the relation
 $\mathcal{B}$, so the proof is similar to the one for the line.
 \kdokaz

From the proof of the previous statement it follows that the endpoints of the line $AB$
are mapped by isometry into the endpoints of the image $A'B'$. In a similar
way, the starting point of the ray is mapped into the starting point of the ray, the edge of the half-plane into the edge of the half-plane, the vertex of the angle into the vertex of the angle and
the vertices of the polygon into the vertices of the polygon.

Isometries are defined as bijective maps, which preserve the
congruence of pairs of points. But does it also hold true, that for every congruent pair of points
there is an isometry, which maps the first pair into the second one? Let us answer this question with the following theorem.


            \bizrek \label{izrekAB} If $(A,B)\cong (A',B')$, then there is an isometry
             $\mathcal{I}$, which maps the points $A$ and $B$ to the points
            $A'$ and $B'$,
             i.e.:
            $$\mathcal{I}: A, B\mapsto A',B'.$$
            \eizrek


\begin{figure}[!htb]
\centering
\input{sl.aks.2.3.7.pic}
\caption{} \label{sl.aks.2.3.7.pic}
\end{figure}


\textbf{\textit{Proof.}}
 Let $C$ be a point, which does not lie on the line $AB$, and $C'$ be a point,
 which does not lie on the line $A'B'$ (Figure \ref{sl.aks.2.3.7.pic}).
 By Axiom \ref{aksIII2} there is a single isometry $\mathcal{I}$, which
 maps the point $A$ into the point $A'$, the segment $AB$ into the segment $A'B'$
 and the plane $ABC$ into the plane $A'B'C'$. Because of $(A,B)\cong (A',B')$ from the same Axiom \ref{aksIII2}, it follows that $\mathcal{I}(B)=B'$.
 \kdokaz

The proof of the following theorem, which will later be stated in a different form as the first theorem about the congruence of triangles, is similar.



            \bizrek \label{IizrekABC} Let $(A,B,C)$ and $(A',B',C')$ be
            triplets of non-collinear points such that $$(A,B,C)\cong (A',B',C'),$$
            then there is a single isometry  $\mathcal{I}$, that maps the points
             $A$, $B$ and $C$ into the points $A'$, $B'$ and $C'$, i.e.:
            $$\mathcal{I}: A, B,C\mapsto A',B',C'.$$
            \eizrek


\begin{figure}[!htb]
\centering
\input{sl.aks.2.3.5.pic}
\caption{} \label{sl.aks.2.3.5.pic}
\end{figure}

\textbf{\textit{Proof.}}
By Axiom \ref{aksIII2} there exists a single isometry $\mathcal{I}$,
that maps point $A$ to point $A'$, line segment $AB$ to line segment
$A'B'$ and plane $ABC$ to plane $A'B'C'$ (Figure \ref{sl.aks.2.3.5.pic}).
Because, by the assumption $(A,B,C)\cong (A',B',C')$ from the same
Axiom \ref{aksIII2}, it follows that $\mathcal{I}(B)=B'$ and
$\mathcal{I}(C)=C'$.

  It is necessary to prove that $\mathcal{I}$ is the only such isometry.
  Assume that there exists such an isometry $\mathcal{\widehat{I}}$, that
  $\mathcal{\widehat{I}}: A, B,C\mapsto A',B',C'$. By
  Theorem \ref{izrekIzoB} isometry $\mathcal{\widehat{I}}$
  also maps line segment $AB$ to line segment $A'B'$ and plane $ABC$
  to plane $A'B'C'$. From Axiom \ref{aksIII2} it follows that
   $\mathcal{\widehat{I}}=\mathcal{I}$.
 \kdokaz

A direct consequence is the following theorem.


                \bizrek \label{IizrekABCident} Let $A$, $B$ and $C$ be three non-collinear points, then the identity map
                 $\mathcal{E}$ is the only isometry that maps points $A$, $B$, and $C$ to the same points
                $A$, $B$ and $C$.
                \eizrek


\begin{figure}[!htb]
\centering
\input{sl.aks.2.3.5a.pic}
\caption{} \label{sl.aks.2.3.5a.pic}
\end{figure}


\textbf{\textit{Proof.}} (Figure \ref{sl.aks.2.3.5a.pic})

First, the identical mapping $\mathcal{E}$, that maps points $A$, $B$
and $C$ to the same points $A$, $B$ and $C$, is an isometry by Axiom
\ref{aksIII4}. From the previous theorem \ref{IizrekABC} it follows
that such an isometry is unique.
 \kdokaz

For point $A$ we say that it is \index{point!fixed} \pojem{fixed point}
(or \index{point!invariant} \pojem{invariant point}) of isometry
$\mathcal{I}$, if $\mathcal{I}(A)=A$. The previous theorem tells us
that the only isometries that have three fixed non-collinear points
are identities.

We will discuss isometries in more detail in chapter
\ref{pogIZO}, but here we will use them primarily to help us
introduce the concept of congruence of figures. Two figures $\Phi$
and $\Phi'$ are \index{figures!congruent}\poem{congruent} (we will
write $\Phi\cong \Phi'$), if there exists an isometry $I$, that
transforms figure $\Phi$ into figure $\Phi'$.

A direct consequence of axiom \ref{aksIII4} is the following
proposition.

\bizrek
             Congruence of figures is an equivalence relation. \label{sklRelEkv}
            \eizrek

\textbf{\textit{Proof.}}

\textit{Reflexivity.} For every figure $\Phi$ it holds that $\Phi \cong
\Phi$, because the identity transformation $\mathcal{E}$ is an isometry
(axiom \ref{aksIII4}) and $\mathcal{E}:\Phi\rightarrow\Phi$.

\textit{Symmetry.} From $\Phi \cong \Phi_1$ it follows that there exists an isometry $\mathcal{I}$, that transforms figure $\Phi$ into figure $\Phi_1$.
The inverse transformation $\mathcal{I}^{-1}$, which is an isometry according to axiom \ref{aksIII4}, transforms figure $\Phi_1$ into figure $\Phi$,
so $\Phi_1 \cong \Phi$ holds.

\textit{Transitivity.} From $\Phi \cong \Phi_1$ and $\Phi_1 \cong
\Phi_2$ it follows that there exist such isometries $\mathcal{I}$ and
$\mathcal{I}'$, that $\mathcal{I}:\Phi\rightarrow\Phi_1$ and
$\mathcal{I}':\Phi_1\rightarrow\Phi_2$ hold.
Then the composition $\mathcal{I}'\circ\mathcal{I}$,
  which is an isometry according to axiom \ref{aksIII4}, transforms figure $\Phi$
into figure $\Phi_2$, so $\Phi \cong \Phi_2$ holds.
\kdokaz


  The concept of congruence of figures also applies to lines. We have intuitively
  associated the congruence of lines with the congruence of pairs of points.
  Now we will prove the equivalence of both relations.

            \bizrek  \label{izrek(A,B)} $AB \cong A'B' \Leftrightarrow
            (A,B)\cong (A',B')$
             \eizrek

\textbf{\textit{Proof.}}

 ($\Rightarrow$) If $(A,B)\cong
(A',B')$, according to proposition \ref{izrekAB} there exists an isometry
$\mathcal{I}$, that transforms points $A$ and $B$ into points $A'$ and
$B'$. From proposition \ref{izrekIzoB} it follows that isometry $\mathcal{I}$
transforms line $AB$ into line $A'B'$ or $AB \cong A'B'$ holds.

($\Leftarrow$) If $AB \cong A'B'$, there exists an isometry
$\mathcal{I}$, which maps the line segment $AB$ to the line segment $A'B'$. By
the consequence of Theorem \ref{izrekIzoB}, the endpoint of the line segment is mapped to the endpoint of the line segment. This means that either
$\mathcal{I}:A,B\mapsto A',B'$ or $\mathcal{I}:A,B\mapsto
B',A'$ holds. From the first relation it follows that $(A,B)\cong (A',B')$ and from the second that $(A,B)\cong (B',A')$. But from the second example we also get
$(A,B)\cong (A',B')$, which is a consequence of Axioms \ref{aksIII3} and
\ref{aksIII4}.
\kdokaz

 Because of the previous theorem, in the following we will always write
  the relation $(A,B)\cong (A',B')$ instead of $AB\cong
 A'B'$.


            \bizrek \label{ABnaPoltrakCX}
            For each line segment $AB$ and each ray $CX$, there is exactly
            one point $D$ on the ray
            $CX$ that $AB\cong CD$ holds.
            \eizrek


\begin{figure}[!htb]
\centering
\input{sl.aks.2.3.5b.pic}
\caption{} \label{sl.aks.2.3.5b.pic}
\end{figure}



 \textbf{\textit{Proof.}} Let $P$ be a point that does not lie on the line $AB$ and $Q$ be a point that does not lie on the line $CX$ (Figure \ref{sl.aks.2.3.5b.pic}).
  By Axiom \ref{aksIII2}, there is only one
  isometry $\mathcal{I}$, which
 maps the point $A$ to the point $C$, the line segment $AB$ to the line segment $CX$
 and the half-plane $ABP$ to the half-plane $CXQ$.
 Let $D=\mathcal{I}(C)$, then $AB \cong CD$ holds.

 We assume that
 on the line segment $CX$ there is another point $\widehat{D}$, for
 which
 $AB \cong C\widehat{D}$ holds. Because the line segments
 $CX$ and $CD$ are congruent, and the isometry $\mathcal{I}$ maps the point
 $A$ to the point $C$, the line segment $AB$ to the line segment $CD$
 and the half-plane $ABP$ to the half-plane $CDQ$,
  from Axiom \ref{aksIII2} it follows that $\mathcal{I}(C)=\widehat{D}$ or
 $\widehat{D}=D$.
 \kdokaz

\bizrek \label{izomEnaC'} Let $A$, $B$, $C$ be three non-collinear points
                 and $A'$, $B'$ points of the edge of a half-plane $\pi$ such that $AB \cong A'B'$.
                  Then there is exactly one point $C'$ in the half-plane $\pi$ such that $AC \cong A'C'$ and $BC \cong B'C'$.
                 \eizrek


\begin{figure}[!htb]
\centering
\input{sl.aks.2.3.11a.pic}
\caption{} \label{sl.aks.2.3.11a.pic}
\end{figure}


 \textbf{\textit{Proof.}} (Figure \ref{sl.aks.2.3.11a.pic})

 By Axiom \ref{aksIII2} there is only one
  isometry $\mathcal{I}$, which
 maps the point $A$ into the point $A'$, the segment $AB$ into the segment $A'B'$
 and the plane $ABC$ into the plane $\pi$ and holds $\mathcal{I}(B)=B'$.
 Let $C'=\mathcal{I}(C)$, then it holds $AC \cong A'C'$ and
 $BC \cong B'C'$. We assume that there is such a point
 $\widehat{C}'$, which lies in the plane $\pi$ and holds $AC \cong A'\widehat{C}'$ and
 $BC \cong B'\widehat{C}'$. Because $AB \cong
A'B'$, by Theorem \ref{IizrekABC} there is only one isometry
$\mathcal{\widehat{I}}$, which maps the points $A$, $B$ and $C$ into
the points $A'$, $B'$ and $\widehat{C}'$. But this also maps
the segment $AB$ into the segment $A'B'$ and the plane $ABC$ into the plane
$A'B'\widehat{C}'=\pi$. By Axiom \ref{aksIII2} is
$\mathcal{\widehat{I}}=\mathcal{I}$ and therefore also
$\widehat{C}'=\mathcal{\widehat{I}}(C)=\mathcal{I}(C)=C'$.
 \kdokaz


            \bizrek \label{izoABAB} If $\mathcal{I}$ is an isometry that maps a points $A$ and $B$ into the same points
            $A$ and $B$ (i.e. $\mathcal{I}(A)=A$ and $\mathcal{I}(B)=B$), then it also holds for each point $X$ on the line
            $AB$ (i.e. $\mathcal{I}(X)=X$).
             \eizrek

\begin{figure}[!htb]
\centering
\input{sl.aks.2.3.8.pic}
\caption{} \label{sl.aks.2.3.8.pic}
\end{figure}

\textbf{\textit{Proof.}} Let $X$ be an arbitrary point on the line
$AB$. Without loss of generality, we assume that the point $X$ lies
on the segment $AB$ (Figure \ref{sl.aks.2.3.8.pic}). We prove that
$\mathcal{I}(X)=X$.

Let $P$ be a point that does not lie on the line $AB$ and
$P'=\mathcal{I}(P)$. The isometry $\mathcal{I}$
maps the point $A$ to the point $A$, the segment $AB$ to the segment $AB$
 (or the segment $AX$ to the segment $AX$)
 and the half-plane $ABP$ to the half-plane $ABP'$
 (or the half-plane $AXP$ to the half-plane $AXP'$).
 By Axiom \ref{aksIII2}
  from $AX\cong AX$ it follows that $\mathcal{I}(X)=X$.
 \kdokaz

 We introduce new concepts related to distances.

We say that the line $EF$ \index{vsota!daljic}\pojem{vsota daljic}
$AB$ and $CD$, which we denote $EF=AB+CD$, if there exists such a point $P$
on the line $EF$, that $AB \cong EP$ and $CD \cong PF$ (Figure
\ref{sl.aks.2.3.9.pic}).

\begin{figure}[!htb]
\centering
\input{sl.aks.2.3.9.pic}
\caption{} \label{sl.aks.2.3.9.pic}
\end{figure}


The line $EF$ is \index{razlika!daljic}\pojem{razlika daljic} $AB$
and $CD$, which we denote $EF=AB-CD$, if $AB=EF+CD$ (Figure
\ref{sl.aks.2.3.9.pic}).

 In a similar way, we can also define
  multiplication of a line by a natural and a positive rational
  number. For the lines $AB$ and $CD$ it is $AB=n\cdot CD$
  ($n\in \mathbb{N}$), if  there exist such points
  $X_1$, $X_2$,..., $X_{n-1}$, that
  $\mathcal{B}(X_1,X_2,\ldots,X_{n-1})$ and
  $AX_1 \cong X_1X_2 \cong X_{n-1}B \cong CD$ (Figure
\ref{sl.aks.2.3.10.pic}).
  In this case, it is also $CD=\frac{1}{n}\cdot AB$.

  At this point, we will not formally prove the fact that for every line $PQ$ and every natural number $n$ there exists a line $AB$, for which $AB=n\cdot PQ$, and a line $CD$, for which $CD=\frac{1}{n}\cdot PQ$.

\begin{figure}[!htb]
\centering
\input{sl.aks.2.3.10.pic}
\caption{} \label{sl.aks.2.3.10.pic}
\end{figure}

We introduce multiplication of a line segment with a positive rational number in the following way. For $q=\frac{n}{m} \in \mathbb{Q^+}$ we have:
$$q\cdot AB=\frac{n}{m}\cdot AB = n\cdot\left(\frac{1}{m}\cdot AB\right)$$

If for a point $P$ on the line segment $AB$ it holds that $AP=\frac{n}{m}\cdot PB$, we say that the point $P$ \pojem{divides} the line segment $AB$ in the \index{razmerje} \pojem{ratio} $n:m$, which we write as $AP:PB=n:m$.

The line segment $AB$ is \index{relacija!urejenosti daljic}\pojem{longer} than the line segment $CD$, which we denote $AB>CD$, if there exists such a point $P\neq B$ on the line segment $AB$, that it holds $CD \cong AP$ (Figure \ref{sl.aks.2.3.11.pic}). In this case we also say that the line segment $CD$ is \pojem{shorter} than the line segment $AB$ (notation $CD<AB$).

\begin{figure}[!htb]
\centering
\input{sl.aks.2.3.11.pic}
\caption{} \label{sl.aks.2.3.11.pic}
\end{figure}

It is not hard to prove that for the line segments $AB$ and $CD$ exactly one of the relations $AB>CD$, $AB<CD$ or $AB \cong CD$ holds. This is a consequence of \ref{ABnaPoltrakCX}.

The point $S$ is the \index{središče!daljice} \pojem{midpoint (bisector)} of the line segment $AB$, if it lies on that line segment and it holds that $AS \cong SB$ (Figure \ref{sl.aks.2.3.12.pic}). Obviously, the midpoint divides the line segment in the ratio $1:1$. We still need to prove that such a point always exists.

\begin{figure}[!htb]
\centering
\input{sl.aks.2.3.12.pic}
\caption{} \label{sl.aks.2.3.12.pic}
\end{figure}

                \bizrek
              For every line segment, there is exactly one midpoint.
                 \eizrek


\begin{figure}[!htb]
\centering
\input{sl.aks.2.3.13.pic}
\caption{} \label{sl.aks.2.3.13.pic}
\end{figure}

\textbf{\textit{Proof.}}
 Let $AB$ be a line segment and $C$ an arbitrary point that does not lie on the line segment $AB$ (Figure \ref{sl.aks.2.3.13.pic}). We denote by $\pi$ the plane $ABC$ and by $\pi'$ the complementary plane of the plane $\pi$. By Axiom \ref{aksIII2}, there exists a single isometry $\mathcal{I}$, which maps the point $A$ to the point $B$, the line segment $AB$ to the line segment $BA$, and the plane $\pi$ to the plane $\pi'$. From $AB\cong BA$ (a consequence of Axiom \ref{aksIII3}) by the same Axiom it follows that $\mathcal{I}(B)=A$.

 Let $C'=\mathcal{I}(C)$, then $AC \cong B'C'$ and $BC \cong A'C'$. Because $C$ and $C'$ are on different sides of the line segment $AB$, the line segment $CC'$ intersects the line segment $AB$ at some point $S$. If $\widehat{C}=\mathcal{I}(C')$, then $A'C' \cong B\widehat{C}$ and $B'C' \cong A\widehat{C}$. Because $AC \cong B'C'$ and $BC \cong A'C'$, by Theorem \ref{izomEnaC'} it follows that $\widehat{C}=C$ or $\mathcal{I}(C')=C$. Therefore, the isometry $\mathcal{I}$ maps the line segment $AB$ and the line segment $CC'$ to themselves, so:
 $$\mathcal{I}(S)=\mathcal{I}(AB\cap CC')=
 \mathcal{I}(AB)\cap \mathcal{I}(CC')=
 AB\cap CC'=S.$$
 Now from $\mathcal{I}:A,S\mapsto B,S$ it follows that $AS\cong SB$.

 To prove that the point $S$ is indeed the center of the line segment $AB$, it is necessary to prove that the point $S$ lies on the line segment $AB$. Assume the contrary. Without loss of generality, let $\mathcal{B}(A,B,S)$. But in this case, on the line segment $SA$ there are two such points $A$ and $B$, that $SA\cong SB$, which contradicts Theorem \ref{ABnaPoltrakCX}.

 We also prove that the line segment has only one center. Let $\widehat{S}\neq S$ be a point on the line segment $AB$ and $A\widehat{S}\cong \widehat{S}B$. From Axiom \ref{aksIII3} it follows that $\mathcal{I}(A\widehat{S})=A\widehat{S}$. This means (from Theorem \ref{izoABAB}), that for every point $X\in AB$ it holds that $\mathcal{I}(X)=X$ and also $\mathcal{I}(A)=A$, which is not possible. Therefore, $\widehat{S}= S$.
 \kdokaz

We say that point $A$ is \index{symmetry!with respect to a point}\emph{symmetric} to point $B$ with respect to point $S$, if $S$ is the center of the line segment $AB$. The symmetry with respect to a translation over a point (i.e. central reflection) will be
further discussed in section \ref{odd6SredZrc}.

Now we will introduce the concepts and derive
the properties that relate
to angles and are analogous to those that we introduced for line segments.
If the statement \ref{ABnaPoltrakCX} is intuitively related to the transfer of a line segment with a compass
to a ray, the next statement will represent the transfer of an angle to a given
ray.


                 \bizrek \label{KotNaPoltrak}
                 For each angle $\alpha$ and each ray $Sp$ lies on a line $p$,
                there is exactly one ray $Sq$ in one of the half-planes determined by the line $p$, such that
                $\alpha \cong \angle pSq$.
                \eizrek

\begin{figure}[!htb]
\centering
\input{sl.aks.2.3.14.pic}
\caption{} \label{sl.aks.2.3.14.pic}
\end{figure}

 \textbf{\textit{Proof.}}
 Let $\alpha=\angle BAC$ and $\pi'$ be one of the half-planes determined by the line $p$ that contains the ray $Sp$ (Figure \ref{sl.aks.2.3.14.pic}).

 By statement \ref{ABnaPoltrakCX} there is
 only one point $P$ on the ray $Sp$, such that $AB \cong SP$.
 By axiom \ref{aksIII2} there is only one isometry $\mathcal{I}$, that
 maps point $A$ to point $S$, the ray $AB$ to the ray $Sp$
 and the half-plane $ABC$ to the half-plane $\pi'$.
  If $Q=\mathcal{I}(C)$, then the ray $AC$ is mapped by this isometry
  to the ray $SQ$. Therefore, the ray $SQ=Sq$ lies in the half-plane
  $\pi'$ and we have $\angle BAC\cong pSq$.

Let's assume that $S\widehat{q}$ is also a segment, lying in the plane
  $\pi'$ and $\angle BAC\cong pS\widehat{q}$. From the definition of congruence
   it follows that there exists an isometry $\mathcal{\widehat{I}}$, which
   maps the angle $BAC$ to the angle $\angle BAC\cong pS\widehat{q}$. Because
   the isometry $\mathcal{\widehat{I}}$
 also maps the point $A$ to the point $S$, the segment $AB$ to the segment $Sp$
 and the plane $ABC$ to the plane $\pi'$, by the axiom
 \ref{aksIII2} $\mathcal{\widehat{I}}=\mathcal{I}$. Therefore,
  $$S\widehat{q}=\mathcal{\widehat{I}}(AB)=\mathcal{I}(AB)=Sq,$$ which was needed to be proven. \kdokaz

 Because the carrier of the segment $Sp$ from the previous statement determines two
 planes, there exist two angles with the leg $Sp$, which are congruent to the angle
 $\alpha$. The mentioned angles are differently oriented. This means that one of these two angles has the same
 orientation as the angle $\alpha$ (Figure \ref{sl.aks.2.3.15.pic}).

\begin{figure}[!htb]
\centering
\input{sl.aks.2.3.15.pic}
\caption{} \label{sl.aks.2.3.15.pic}
\end{figure}


Similarly to the case of distances, we define certain operations and
relations also among angles.

 The angle $pq$ with the vertex $S$ is the \index{vsota!kotov}\pojem{sum of angles} $ab$ and $cd$ or
 $\angle pq = \angle ab + \angle cd$, if there exists a segment
 $s=SX$, lying in the angle $pq$ and $\angle ps \cong \angle ab$
 and $\angle sq \cong \angle cd$ (Figure \ref{sl.aks.2.3.16.pic}).
  In this case we also say that
 the angle $ab$ is the \index{razlika!kotov}\pojem{difference of angles} $pq$ and $cd$, or
  $ \angle ab= \angle pq - \angle cd$.

\begin{figure}[!htb]
\centering
\input{sl.aks.2.3.16.pic}
\caption{} \label{sl.aks.2.3.16.pic}
\end{figure}

Similarly to the case of distances, for the angle $ab$ we define the angles
$n\cdot \angle ab$ and  $\frac{1}{n}\cdot \angle ab$ ($n\in
\mathbb{N}$) and $q\cdot \angle ab$ ($q\in \mathbb{Q}$).

We say that the angle $ab$ with the vertex $S$ \index{relation!order of angles}\pojem{is greater} than the angle $cd$ ($\angle ab > \angle cd$), if there exists a segment $s=SX$ in the angle $ab$, such that $\angle as \cong \angle cd$ (Figure \ref{sl.aks.2.3.17.pic}). In this case, the angle $cd$ is also \pojem{smaller} than the angle $ab$ ($\angle cd< \angle ab$). It is not difficult to prove that for two angles $ab$ and $cd$ one of the relations holds: $\angle ab > \angle cd$, $\angle ab < \angle cd$ or $\angle ab \cong \angle cd$.


\begin{figure}[!htb]
\centering
\input{sl.aks.2.3.17.pic}
\caption{} \label{sl.aks.2.3.17.pic}
\end{figure}


The angles are \index{angles!supplementary}\pojem{supplementary}, if their sum is equal to the straight angle  (Figure
\ref{sl.aks.2.3.18.pic}).

\begin{figure}[!htb]
\centering
\input{sl.aks.2.3.18.pic}
\caption{} \label{sl.aks.2.3.18.pic}
\end{figure}


The segment $s=SX$ is the \index{bisector of an angle}\pojem{bisector of the angle} $\angle pSq=\alpha$ (Figure \ref{sl.aks.2.3.19.pic}), if it lies in this angle and
it holds $\angle ps \cong \angle sq$. The carrier of this bisector  is the \index{simetrala!kota}\pojem{simetrala kota} $pSq$ (the line $s_{\alpha}$).

\begin{figure}[!htb]
\centering
\input{sl.aks.2.3.19.pic}
\caption{} \label{sl.aks.2.3.19.pic}
\end{figure}



Similarly to the center of a line, the following statement holds for the bisector of an angle.

            \bizrek \label{izrekSimetralaKota}
             An angle has exactly one bisector.
             %(oz. eno samo simetralo).
            \eizrek

\textbf{\textit{Proof.}}
 Let $\alpha=pSq$ be an arbitrary angle, $P$ an arbitrary point, which lies on the
 segment $Sp$ ($P\neq S$) and $Q$ a point, which lies on the segment $Sq$ and it holds
 $SP\cong SQ$.

\begin{figure}[!htb]
\centering
\input{sl.aks.2.3.20.pic}
\caption{} \label{sl.aks.2.3.20.pic}
\end{figure}

Let the angle $\alpha$ be the extended angle
 (Figure \ref{sl.aks.2.3.20.pic}), which determines
 the line $\pi$. Let $A$ be any point on it. By the statement
 \ref{izomEnaC'} in the line $\pi$ there is only one point $B$,
 so that $(P,Q,A)\cong (Q,P,B)$. From the statement \ref{IizrekABC} it follows that there is only one izometry $\mathcal{I}$, which maps points $P$, $Q$ and
 $A$ into points $Q$, $P$ and $B$. Let
 $\mathcal{I}(B)=\widehat{A}$. Because
 $(Q,P,B)\cong(P,Q,\widehat{A})$, by the statement \ref{izomEnaC'}
 $\widehat{A}=A$. Therefore:
  $$\mathcal{I}:P,Q,A,B\mapsto Q,P,B,A.$$
Therefore, the centers $S$ and $L$ of the lines $PQ$ and $AB$ map into each other
(axiom \ref{aksIII4}), which then also holds for the line segment $s=SL$ and
every point on it (statement \ref{izoABAB}). Therefore, the izometry
$\mathcal{I}$ maps the line segment $pSs$ into the line segment $sSq$, so
  the line segment $pSs\cong sSq$ or the line segment $s$ is the bisector of the angle $pSq$.

  We will prove that $s$ is the only bisector of the angle $\alpha$. Let
  $\widehat{s}=S\widehat{L}$
  be a line segment that lies in the angle $\alpha$ and $pS\widehat{s}\cong
  \widehat{s}Sq$. Then there is an izometry $\mathcal{\widehat{I}}$, which
  maps the angle $pS\widehat{s}$ into the angle $\widehat{s}Sq$. This izometry
  maps the point $S$ into the point $S$, the line segment $p$ into the line segment $q$ and
  the line $\pi$ into the line $\pi$, so by the axiom
  \ref{aksIII2} $\mathcal{\widehat{I}}=\mathcal{I}$. Therefore
  $\mathcal{I}(\widehat{s})=
  \mathcal{\widehat{I}}(\widehat{s})=\widehat{s}$. If $\widehat{L} \notin
  s$, the izometry $\mathcal{I}$ maps three non-collinear points
  $S$, $L$ and $\widehat{L}$ into itself and is the identical mapping
  (statement \ref{IizrekABCident}), which is not possible. Therefore $\widehat{L} \in
  s$ or $\widehat{s}=s$.

If $\alpha$ is an unextended convex angle  (Figure \ref{sl.aks.2.3.20.pic}),
 then points $S$, $P$ and $Q$
 are nonlinear, so according to  Theorem \ref{IizrekABC} there is only one
  isometry $\mathcal{I}$, which maps points $P$, $S$ and $Q$ to
 points $Q$, $S$ and $P$. With $L$ we denote the center of the line $PQ$.
 By the Axiom \ref{aksIII3} we have $\mathcal{I}(L)=L$. Then also all
 points of the ray $s=SL$ are mapped to themselves (Theorem \ref{izoABAB}). As
$\alpha$ is a convex angle, this means that the point $L$ and then also
the ray $s$ lie within this angle.
 Therefore the isometry $\mathcal{I}$ maps angle $pSs$ to angle $sSq$, so
  $pSs\cong sSq$ or the ray $s$ is the bisector of angle $pSq$.

Similarly as in the previous example we prove that angle $\alpha$ has no other bisectors.

If $\alpha$ is a non-convex angle, the bisector is obtained as the
complementary (supplementary) ray of the ray $s$.
 \kdokaz

We prove two theorems, which relate to right angles and perfect angles.\index{angle!right} \index{angle!perfect}



               \bizrek
              The adjacent supplementary angles  of two congruent angles are also congruent. \label{sokota}
             \eizrek

\begin{figure}[!htb]
\centering
\input{sl.aks.2.3.20a.pic}
\caption{} \label{sl.aks.2.3.20a.pic}
\end{figure}

\textbf{\textit{Proof.}} Let $\alpha'=\angle P'OQ$ and $\alpha_1'=\angle P_1'O_1Q_1$ be the supplement angles of two adjacent angles $\alpha=\angle POQ$ and $\alpha_1=\angle P_1O_1Q_1$ (Figure \ref{sl.aks.2.3.20a.pic}). By Axiom \ref{aksIII2}, there exists a single isometry $\mathcal{I}$, which maps point $O$ to point $O_1$, line segment $OP$ to line segment $O_1P_1$, and line $POQ$ to line $P_1O_1Q_1$. Let $Q_2=\mathcal{I}(Q)$. Then $\angle P_1O_1Q_2\cong \angle POQ$. Isometry $\mathcal{I}$ maps line $POQ$ to line $P_1O_1Q_1$, so point $Q_2$ (and also line segment $O_1Q_2$) lies on line $P_1O_1Q_1$. Since, by assumption, $\angle POQ\cong\angle P_1O_1Q_1$, by  \ref{KotNaPoltrak} Theorem, $OQ_1$ and $OQ_2$ represent the same line segment. Therefore, point $Q_2$ lies on line segment $O_1Q_1$. Let $P_2'=\mathcal{I}(P')$. Since isometries map line segments to line segments (Theorem \ref{izrekIzoB}), point $P_2'$ lies on line segment $O_1P_1'$. From $\mathcal{I}:P',O,Q\mapsto P_2',O_1,Q_2$ it follows that isometry $\mathcal{I}$ maps angle $P'OQ$ to angle $P_2'O_1Q_2$  (Theorem \ref{izrekIzoB}), so
 $\angle P'OQ\cong \angle P_2'O_1Q_2=\angle P_1'O_1Q_1$.
\kdokaz


                \bizrek \label{sovrsnaSkladna}
               Vertical angles are congruent.
                \eizrek

\begin{figure}[!htb]
\centering
\input{sl.aks.2.3.20b.pic}
\caption{} \label{sl.aks.2.3.20b.pic}
\end{figure}


\textbf{\textit{Proof.}} Let $\alpha=\angle POQ$ and $\alpha'=\angle P'OQ'$ be two adjacent angles, where points $P$, $O$, $P'$ (or $Q$, $O$, $Q'$) are collinear (Figure \ref{sl.aks.2.3.20b.pic}). Angle $\beta=\angle QOP'$ is the supplement of both angles $\alpha$ and $\alpha'$. Since $\beta\cong\beta$, by the previous Theorem \ref{sokota}, $\alpha\cong\alpha'$.
\kdokaz

\bizrek \label{sredZrcObstoj}
            For each point $S$ there exists an isometry $\mathcal{I}$ such that $\mathcal{I}(S)=S$.
            In addition, for each point $X\neq S$ the following holds:\\ if $\mathcal{I}(X)=X'$, then $S$ is the midpoint of the line segment $XX'$.
            \eizrek

\begin{figure}[!htb]
\centering
\input{sl.aks.2.3.20c.pic}
\caption{} \label{sl.aks.2.3.20c.pic}
\end{figure}


\textbf{\textit{Proof.}} Let $P$ be an arbitrary point different from $S$  (Figure \ref{sl.aks.2.3.20c.pic}). By Axiom \ref{AksII3}, there exists a point $Q$ on the line $SP$ such that $\mathcal{B}(P,S,Q)$. We mark the half-planes determined by the edge $SP$ with $\alpha$ and $\alpha'$. By Axiom \ref{aksIII2}, there exists (one and only one) isometry $\mathcal{I}$, which maps the point $S$ to the point $S$, the line segment $SP$ to the line segment $SQ$, and the half-plane $\alpha$ to the half-plane $\alpha'$.

We mark the line $SP$ with $p$.
The point $P'=\mathcal{I}(P)$ lies on the line segment $SQ$ or on the line $p$. Therefore, since  $\mathcal{I}:S,P \mapsto S,P'$, the line $SP$ is mapped to the line $SP'$ by Axiom \ref{aksIII1}, i.e. $\mathcal{I}:p\rightarrow p$.
The image of the half-plane $\alpha'$ with the edge $p$ is therefore a half-plane with the same edge (Proposition \ref{izrekIzoB}). This half-plane cannot be $\alpha'$, since the isometry  $\mathcal{I}$ is a bijective mapping and it maps the half-plane  $\alpha$ to the half-plane $\alpha'$.  Therefore, $\mathcal{I}:\alpha'\rightarrow \alpha$.

Now it is clear that, without loss of generality, it is enough to carry out the proof only for points that lie in the half-plane $\alpha$ (without the edge or only the line segment $SP$).

Let $X\in \alpha\setminus p$ and $X'=\mathcal{I}(X)$. We immediately see that $X'\in \alpha'\setminus p$. By Axiom \ref{AksII3} there exists on the line $SX$ such a point $X_1$, that $\mathcal{B}(X,S,X_1)$ is true. Because $\angle PSX$ and $\angle P'SX_1$ are perfect angles, by Theorem \ref{sovrsnaSkladna} they are also compatible. But from $\mathcal{I}:S,P,X \mapsto S,P',X'$ it follows that $\angle PSX \cong \angle P'SX'$. Therefore $\angle P'SX_1\cong \angle P'SX'$ is true (Theorem \ref{sklRelEkv})), so by Theorem \ref{KotNaPoltrak} the line segments $SX_1$ and $SX'$ are identical. This means that the point $X'$ lies on the line segment $SX_1$ or $\mathcal{B}(X,S,X')$ is true. Because of $\mathcal{I}:S,X \mapsto S,X'$ it is also true that $SX\cong SX'$, so by definition the point $S$ is the center of the line $XX'$.

 Let in the end $Y$ be an arbitrary point on the line segment $SP$, which is different from the point $S$, and $Y'=\mathcal{I}(Y)$. The point $Y'$ lies on the line segment $SQ$, so $\mathcal{B}(Y,S,Y')$ is true. Because of $\mathcal{I}:S,Y \mapsto S,Y'$ it is also true that $SY\cong SY'$, so by definition the point $S$ is the center of the line $YY'$.
\kdokaz

In section \ref{odd6SredZrc} we will discuss the isometry, which is mentioned in the previous Theorem \ref{sredZrcObstoj}, in more detail.


 Let us define new types of angles.
 A \index{kot!ostri}\pojem{acute angle} is a \index{kot!pravi}
 \pojem{right angle} or a \index{kot!topi}\pojem{obtuse angle}, if it is
 smaller, equal to or greater than its supplement (Figure \ref{sl.aks.2.3.21.pic}).


\begin{figure}[!htb]
\centering
\input{sl.aks.2.3.21.pic}
\caption{} \label{sl.aks.2.3.21.pic}
\end{figure}



From the definition it follows that acute (or obtuse) angles are those convex angles, which are smaller (or greater) than a right angle.

From Theorem \ref{izrekSimetralaKota} it follows that a right angle
exists, since the bisector of an extended angle divides it into two compatible supplements.

It is not difficult to prove that every two right angles are compatible and that an angle, which is compatible with a right angle, is also a right angle.

If the sum of two angles is a right angle, we say that the angles are
\pojem{complementary} (Figure
\ref{sl.aks.2.3.22.pic}).


\begin{figure}[!htb]
\centering
\input{sl.aks.2.3.22.pic}
\caption{} \label{sl.aks.2.3.22.pic}
\end{figure}



We will now introduce an extremely important relation between lines. If
the lines $p$ and $q$ contain the segments of a right angle, we say that
$p$ and $q$ are \pojem{perpendicular}, which we denote $p \perp q$
(Figure \ref{sl.aks.2.3.23.pic}).

\begin{figure}[!htb]
\centering
\input{sl.aks.2.3.23.pic}
\caption{} \label{sl.aks.2.3.23.pic}
\end{figure}

From the definition itself it is clear that the perpendicularity is a
symmetric relation, i.e. from $p \perp q$ it follows that $q \perp p$. If
$p \perp q$ and $p \cap q=S$, we say that the line $p$ is
\pojem{perpendicular} to the line $q$ at the point $S$ or that $p$ is a
\pojem{perpendicular} of the line $q$ at this point.



The following theorem is the most important theorem that characterizes
the relation of perpendicularity.



                \bizrek \label{enaSamaPravokotnica}
                For each point $A$ and each line $p$, there is a unique line $n$
            going through the point $A$, which is perpendicular on the line $p$.
                \eizrek

\textbf{\textit{Proof.}}
Assume that point $A$ does not lie on line $p$. Let $B$ and $C$ be any points that lie on line $p$ (Figure \ref{sl.aks.2.3.24.pic}). We denote the plane $BCA$ with $\pi$, and the complementary plane with $\pi_1$. By izreku \ref{izomEnaC'}, there exists only one point $A_1\in \pi_1$, for which $(A,B,C) \cong (A_1,B,C)$. From izreku \ref{IizrekABC} it follows that there exists only one izometrija $\mathcal{I}$, which maps points $A$, $B$ and $C$ into points $A_1$, $B$ and $C$. We denote the line $AA_1$ with $n$. Because $A$ and $A_1$ are on different sides of line $p$, line $n$ intersects line $p$ in some point $S$. From $\mathcal{I}:B,C \mapsto B,C$ it follows that $\mathcal{I}(S)=S$ (izrek \ref{izoABAB}). Therefore, izometrija $\mathcal{I}$ maps angle $ASB$ into angle $A_1SB$. It follows that $\angle ASB$ and $\angle A_1SB$ are complementary angles, therefore they are right angles. Therefore, $n \perp p$.

\begin{figure}[!htb]
\centering
\input{sl.aks.2.3.24.pic}
\caption{} \label{sl.aks.2.3.24.pic}
\end{figure}

We will prove that $n$ is the only perpendicular to line $p$ through point $A$. Let $\widehat{n}$ be a line, for which $A\in \widehat{n}$ and $\widehat{n} \perp p$. Let point $\widehat{S}$ be the intersection of lines $\widehat{n}$ and $p$. By the assumption, $\angle A\widehat{S}B$ is a right angle and is compatible with its complementary angle $\angle B\widehat{S}A_2$ ($A_2$ is such a point that $\mathcal{B}(A,\widehat{S},A_2)$), which is also a right angle.

From $\mathcal{I}:B,C \mapsto B,C$ it follows
$\mathcal{I}(\widehat{S})=\widehat{S}$ (statement \ref{izoABAB}).
Therefore, the isometry $\mathcal{I}$ maps the angle $A\widehat{S}B$ into the angle
$A_1\widehat{S}B$. It follows that $\angle A\widehat{S}B$ and $\angle
A_1\widehat{S}B$ are congruent, so $\angle A_1\widehat{S}B$
is a right angle. Therefore, the angle $A_1\widehat{S}B$ and $A_2\widehat{S}B$
are right angles and are therefore congruent. It follows that the line segments
$\widehat{S}A_1$ and $\widehat{S}A_2$ are the same, so $A_1 \in
\widehat{S}A_2=\widehat{n}$ or $\widehat{n}=AA_1=n$.

 In the case when the point $A$ lies on the line $p$, the rectangle $n$
 is the symmetry of the corresponding extended angle (statement \ref{izrekSimetralaKota}).
\kdokaz


The previous statement has the following important consequence - the existence of pairs of disjoint lines in the plane - or those that do not have common points. This is the content of the following two statements.


            \bizrek \label{absolGeom1}
             Let $p$ and $q$ be a lines perpendicular on a line $PQ$ in the points $P$ and $Q$.
            Then the lines $p$ and $q$ do not have a common points i.e. $p\cap q=\emptyset$.
            \eizrek

\begin{figure}[!htb]
\centering
\input{sl.aks.2.3.25b.pic}
\caption{} \label{sl.aks.2.3.25b.pic}
\end{figure}

\textbf{\textit{Proof.}} The statement is a direct consequence of the previous statement \ref{enaSamaPravokotnica}. Namely, if the lines $p$ and $q$ intersected at some point $S$, we would have two rectangles on the line $PQ$ from the point $S$ (Figure \ref{sl.aks.2.3.25b.pic}), which is in contradiction with the aforementioned statement.
 \kdokaz



            \bizrek \label{absolGeom}
            If $A$ is a point that does not lie on a line $p$, then there exists at least
            one line (in the same plane) passing through the point $A$ and not intersecting the line
            $p$ (Figure \ref{sl.aks.2.3.25a.pic}).
            \eizrek


\begin{figure}[!htb]
\centering
\input{sl.aks.2.3.25a.pic}
\caption{} \label{sl.aks.2.3.25a.pic}
\end{figure}

\textbf{\textit{Proof.}} By the statement \ref{enaSamaPravokotnica} there exists (exactly one) rectangle $n$ of the line $p$, which goes through the point $A$. We mark with $A'$ the intersection of the lines $p$ and $n$. From the same statement it follows that there exists another rectangle $q$ of the line $n$ in the point $A$. By the previous statement \ref{absolGeom1} the line $q$ goes through the point $A$ and doesn't have any common points with the line $p$.
 \kdokaz

The point $A'$ is the \index{node}\pojem{node} or the
\index{orthogonal projection}\pojem{orthogonal projection} of the
point $A$ on the line $p$, if the rectangle of the line $p$ through
the point $A$ intersects the line in the point $A'$. We will mark
it with $A'=pr_{\perp p}(A)$ (Figure \ref{sl.aks.2.3.25.pic}). From
the previous statement it follows that for every point and line
there exists only one orthogonal projection.

\begin{figure}[!htb]
\centering
\input{sl.aks.2.3.25.pic}
\caption{} \label{sl.aks.2.3.25.pic}
\end{figure}

The line, which goes through the center $S$ of the distance $AB$ and is perpendicular to the line $AB$, is called the \index{symmetry!of a line}\pojem{symmetry of the line} $AB$ and we mark it with $s_{AB}$
(Figure \ref{sl.aks.2.3.26.pic}). The properties of the symmetry of a line we will discuss in the next chapter.

\begin{figure}[!htb]
\centering
\input{sl.aks.2.3.26.pic}
\caption{} \label{sl.aks.2.3.26.pic}
\end{figure}


We say that the point $A$ is \index{symmetry!with respect to a line}\pojem{symmetric} to the point $B$ with respect to the line $s$, if the line $s$ is the symmetry of the line $AB$. The symmetry with respect to a line (as a mapping - i.e. the basic mirroring) we will discuss in more detail in the section \ref{odd6OsnZrc}.

Let $S$ be a point and $AB$ a line. The set of all points $X$, for
which it holds that $SX \cong AB$, is called the
\index{circle}\pojem{circle} with
\index{center!of a circle}\pojem{center} $S$ and \index{radius
of a circle}\pojem{radius} $AB$; we mark it with $k(S,AB)$ (Figure
\ref{sl.aks.2.3.27.pic})  i.e.:
 $$k(S,AB)=\{X;\hspace*{1mm}SX \cong AB\}.$$

\begin{figure}[!htb]
\centering
\input{sl.aks.2.3.27.pic}
\caption{} \label{sl.aks.2.3.27.pic}
\end{figure}

 Of course, a circle is a set
 of points in a plane, because in this book we only consider
 planar geometry (all points and all figures belong to the same plane).

 From the definition it is clear that for the radius we can choose any distance
 that is consistent with
 the distance $AB$, that is, any distance $SP$, where $P$ is any
 point on the circle. Since the radius is not tied to
 a specific distance, we usually denote it with a small letter $r$. So we can also write the circle like this:
 $$k(S,r)=\{X;\hspace*{1mm}SX \cong r\}.$$
 The set

$$\{X;\hspace*{1mm}SX \leq r\}$$
we call the \index{krog}\pojem{circle} with center $S$ and radius $r$ (Figure \ref{sl.aks.2.3.28.pic}) and denote it with $\mathcal{K}(S,r)$.
The set
 $$\{X;\hspace*{1mm}SX < r\}$$
 is the \index{notranjost!kroga}
 \pojem{interior of the circle} $\mathcal{K}(S, r)$, and its points are
 \pojem{interior points of the circle}.
 This means that the circle is actually the union of its interior and the corresponding circle.

The set
 $$\{X;\hspace*{1mm}SX > r\}$$
 we call the \index{zunanjost!kroga}\pojem{exterior of the circle} $\mathcal{K}(S, r)$
  and its points \pojem{exterior points of the circle}.

 For practical reasons, we will call the interior of the circle $\mathcal{K}(S, r)$ the \index{notranjost!krožnice}\pojem{interior} of the corresponding circle $k(S,r)$, the exterior of the circle $\mathcal{K}(S, r)$ the \index{zunanjost!krožnice}\pojem{exterior} of the corresponding circle $k(S,r)$. We define the \pojem{interior} and \pojem{exterior} points of the circle in the same way.

\begin{figure}[!htb]
\centering
\input{sl.aks.2.3.28.pic}
\caption{} \label{sl.aks.2.3.28.pic}
\end{figure}

If $P$ and $Q$ are two points on the circle $k(S, r)$, the distance $PQ$
   is called the \index{tetiva krožnice} \pojem{tetiva} of
  the circle. If
the tether contains the center of the circle, it is called
\index{premer krožnice}\pojem{premer} or \index{diameter
krožnice}\pojem{diameter} of the circle (Figure
\ref{sl.aks.2.3.29.pic}).

\begin{figure}[!htb]
\centering
\input{sl.aks.2.3.29.pic}
\caption{} \label{sl.aks.2.3.29.pic}
\end{figure}

We prove the following statement.


             \bizrek \label{premerInS}
            The centre $S$ of the circle $k(S, r)$ is at the same time the midpoint of each
            diameter of that circle.
            \eizrek

\textbf{\textit{Proof.}}

\begin{figure}[!htb]
\centering
\input{sl.aks.2.3.30.pic}
\caption{} \label{sl.aks.2.3.30.pic}
\end{figure}

 If $PQ$ is the diameter of the circle $k(S, r)$, the points $P$ and $Q$ lie
on the circle, which means: $SP \cong SQ \cong r$ (Figure
\ref{sl.aks.2.3.30.pic}). Because the point $S$ lies on the line $PQ$,
it follows that the point $S$ is the center of this line.
 \kdokaz

 From the previous statement it follows that the diameter is equal to two radii, because:
$PQ = PS + SQ = 2\cdot PS = 2\cdot r$. This means
that all diameters of a circle are consistent with each other.


Let $P$ and $Q$ be any two points on the circle  $k(S, r)$. The intersection
of the circle $k$ with one of the planes (in the plane of this circle) with the edge
$s=PQ$ is called the \index{krožni!lok} \pojem{krožni lok} $PQ$ (or
shorter \pojem{lok}) with the endpoints $P$ and $Q$.

Thus, each tether $PQ$ on a circle $k$ determines two arcs. Assume that the center $S$ does not lie on the edge of the plane that generates the circular arc.
If this plane
contains the center $S$ of the circle $k$,
it is a \pojem{veliki lok} $PQ$, otherwise it is a \pojem{mali
lok} $PQ$.
 But if
  the center $S$ is on the edge $PQ$ of the plane, then each of the two arcs $PQ$
 \index{polkrožnica}\pojem{polkrožnica} $PQ$ (Figure
\ref{sl.skk.4.2.1.pic}).

\begin{figure}[!htb]
\centering
\input{sl.skk.4.2.1.pic}
\caption{} \label{sl.skk.4.2.1.pic}
\end{figure}

Since the locus is not uniquely determined by its endpoints, we must also know
at least one point on the circle that belongs or does not belong to this locus.

In a similar way, we define certain parts of the circle.

Let $P$ and $Q$ be arbitrary points on the circle $k(S, r)$. The intersection
of the circle $\mathcal{K}(S, r)$ with one of the planes (in the plane of this circle) with the edge
$s=PQ$ is called
\index{krožni!odsek} \pojem{circular segment}.
 So each chord $PQ$ on some circle $k(S, r)$ determines on the circle $\mathcal{K}(S, r)$ two circular segments. Assume that the center $S$ does not lie on the edge of the plane that generates the circular segment.
If this plane
contains the center $S$ of the circle $k$,
it is a \pojem{major circular segment} $PQ$, otherwise it is a \pojem{minor
circular segment} $PQ$.
 If the
  center $S$ is on the edge $PQ$ of the plane, then each of the two circular segments
\index{polkrog}\pojem{arc}
 is an \pojem{arc}.
 From the definition it is clear that the edge of the circular segment is the union of the chord $PQ$ and the corresponding arc (Figure
\ref{sl.skk.4.2.1b.pic}).

\begin{figure}[!htb]
\centering
\input{sl.skk.4.2.1b.pic}
\caption{} \label{sl.skk.4.2.1b.pic}
\end{figure}


We will also define one more term related to the circle.
Let $P$ and $Q$ be arbitrary points on the circle $k(S, r)$. The intersection
of the circle $\mathcal{K}(S, r)$ with one of the angles $PSQ$ is called
\index{krožni!izsek} \pojem{circular sector}. Again we have two circular sectors. If the angle $PSQ$ is an obtuse angle, we get two arcs, otherwise a convex and a concave circular sector, depending on whether the angle $PSQ$ is convex or concave (Figure
\ref{sl.skk.4.2.1c.pic}).

\begin{figure}[!htb]
\centering
\input{sl.skk.4.2.1c.pic}
\caption{} \label{sl.skk.4.2.1c.pic}
\end{figure}





%________________________________________________________________________________
 \poglavje{Continuity Axiom} \label{odd2AKSZVE}

We have already learned in elementary school, when introducing the numerical line and the coordinate system, that it is possible to establish a connection in which each point of a line corresponds to a certain real number, and vice versa, each real number can be assigned a point that lies on that line. This is related to the following axiom.

\baksiom \label{aksDed}\index{aksiom!Dedekind's}
(Dedekind's\footnote{\index{Dedekind, R.}
\textit{R. Dedekind} (1831--1916),
German mathematician.}
axiom)
Suppose that all points on open line segment $AB$ are divided into the union of two nonempty disjoint sets $\mathcal{U}$ and
$\mathcal{V}$ such that no point of $\mathcal{U}$ is
between two points of  $\mathcal{V}$ and vice versa: no point of $\mathcal{V}$ is
between two points of  $\mathcal{U}$. Then there is a unique point $C$ on open line segment $AB$ such that
$B(A',C,B')$ for any two points $A'\in
\mathcal{U}\setminus{C}$ and $B'\in \mathcal{V} \setminus {C}$
(Figure \ref{sl.aks.2.4.1.pic}).
\eaksiom

\begin{figure}[!htb]
\centering
\input{sl.aks.2.4.1.pic}
\caption{} \label{sl.aks.2.4.1.pic}
\end{figure}

Let us state without proof two important consequences of the axiom
of continuity\footnote{Until the 19th century, mathematicians did not
feel the need to prove these two statements, or the need to
introduce the axiom of continuity. Even \index{Evklid} Euclid from
Alexandria (3rd century BC) in his famous work
'Elements', does not prove that a certain circle intersects.}.


\bizrek \label{DedPoslKrozPrem} Let $k$ be a circle and $P$ a point inside that circle.
Then any line $p$ passing through the point $P$ and the circle $k$ has exactly two common points (Figure
\ref{sl.aks.2.4.2.pic}).
\eizrek

\begin{figure}[!htb]
\centering
\input{sl.aks.2.4.2.pic}
\caption{} \label{sl.aks.2.4.2.pic}
\end{figure}



\bizrek \label{DedPoslKrozKroz} If $k$ and $l$ are circles such that $l$ contains at least one point inside and one point outside $k$,
then the circles has exactly two points (Figure \ref{sl.aks.2.4.3.pic}).
\eizrek

\poglavje{The basics of Geometry} \label{osn9Geom}
We have already learned in elementary school, when introducing the numerical line and the coordinate system, that it is possible to establish a connection in which each point of a line corresponds to a certain real number, and vice versa, each real number can be assigned a point that lies on that line. This is related to the following axiom.

\baksiom \label{aksDed}\index{aksiom!Dedekind's}
(Dedekind's\footnote{\index{Dedekind, R.}
\textit{R. Dedekind} (1831--1916),
German mathematician.}
axiom)
Suppose that all points on open line segment $AB$ are divided into the union of two nonempty disjoint sets $\mathcal{U}$ and
$\mathcal{V}$ such that no point of $\mathcal{U}$ is
between two points of  $\mathcal{V}$ and vice versa: no point of $\mathcal{V}$ is
between two points of  $\mathcal{U}$. Then there is a unique point $C$ on open line segment $AB$ such that
$B(A',C,B')$ for any two points $A'\in
\mathcal{U}\setminus{C}$ and $B'\in \mathcal{V} \setminus {C}$
(Figure \ref{sl.aks.2.4.1.pic}).
\eaksiom

\begin{figure}[!htb]
\centering
\input{sl.aks.2.4.1.pic}
\caption{} \label{sl.aks.2.4.1.pic}
\end{figure}

Let us state without proof two important consequences of the axiom
of continuity\footnote{Until the 19th century, mathematicians did not
feel the need to prove these two statements, or the need to
introduce the axiom of continuity. Even \index{Evklid} Euclid from
Alexandria (3rd century BC) in his famous work
'Elements', does not prove that a certain circle intersects.}.


\bizrek \label{DedPoslKrozPrem} Let $k$ be a circle and $P$ a point inside that circle.
Then any line $p$ passing through the point $P$ and the circle $k$ has exactly two common points (Figure
\ref{sl.aks.2.4.2.pic}).
\eizrek

\begin{figure}[!htb]
\centering
\input{sl.aks.2.4.2.pic}
\caption{} \label{sl.aks.2.4.2.pic}
\end{figure}



\bizrek \label{DedPoslKrozKroz} If $k$ and $l$ are circles such that $l$ contains at least one point inside and one point outside $k$,
then the circles has exactly two points (Figure \ref{sl.aks.2.4.3.pic}).
\eizrek

\begin{figure}[!htb]
\centering
\input{sl.aks.2.4.3.pic}
\caption{} \label{sl.aks.2.4.3.pic}
\end{figure}

Dedekind's axiom is used in a few different ways to establish the set of real numbers. This is reminiscent of the already mentioned connection between the set of points on a line and the set of real numbers.

We have already defined the multiplication operation of line segment $AB$ by any positive rational number $q$. Now we can extend the concept of multiplication to any positive real number $\lambda$. The definition of line segment $\lambda\cdot AB$, ($\lambda \in \mathbb{R}^+$), which we will not formally derive here, is associated with two sets of points on line segment $CD$:
 \begin{eqnarray*}
&& \{X\in CD;\hspace*{1mm}CX=q\cdot
AB,\hspace*{1mm}q\leq\lambda,\hspace*{1mm}q\in
\mathbb{Q}^+ \} \hspace*{1mm}\textrm{ in}\\
&& \{X\in CD;\hspace*{1mm}CX=q\cdot
AB,\hspace*{1mm}q>\lambda,\hspace*{1mm}q\in \mathbb{Q}^+ \}
 \end{eqnarray*}
 and
Dedekind's axiom \ref{aksDed}.

 With the help of the axiom of continuity \ref{aksDed} we can also introduce the concepts of measuring line segments and angles.

 When measuring line segments we will assign each line segment $AB$ a positive real number $\textsl{m}(AB)$ in the following way.
  Let $\mathcal{D}$ be the set of all line segments and $\mathbb{R}^+$ be the set of all positive real numbers. The mapping $\textsl{m}:\mathcal{D}\rightarrow\mathbb{R}^+$, which satisfies the following properties:
  \begin{itemize}
    \item $(\exists A_0B_0\in\mathcal{D})\hspace*{1mm}\textsl{m}(A_0B_0)=1$,
    \item $(\forall AB, CD\in\mathcal{D})\hspace*{1mm}(AB\cong CD \Rightarrow\textsl{m}(AB)=\textsl{m}(CD))$,
    \item $(\forall AB, CD, EF\in\mathcal{D})\hspace*{1mm}(AB+CD=EF\Rightarrow \textsl{m}(AB)+\textsl{m}(CD)=\textsl{m}(EF))$,
  \end{itemize}
  is called the \index{dolžina!daljice}\pojem{length of a line segment} or the \index{mera!daljice}\pojem{measure of a line segment}, and the triple $\textsl{M}=(\mathcal{D},\mathbb{R}^+,\textsl{m})$ is called the \index{sistem merjenja!daljic}\pojem{system of measuring line segments}.

The length of the line $AB$ (or $\textsl{m}(AB)$) will usually be denoted by $|AB|$.

  It is intuitively clear that there are an infinite number of measuring systems, which depend on the choice of the unit length line $A_0B_0$ - the one that has a length of 1, or $\textsl{m}(A_0B_0)=1$. In one measuring system, the length of any line $AB$ is represented by a positive real number $x$, for which $AB=x\cdot A_0B_0$. So $\textsl{m}(AB)=x$ exactly when $AB=x\cdot A_0B_0$ (Figure \ref{sl.aks.2.4.4.pic}). Now it is clear why we need the axiom of connectivity - without it, we would have problems with the definition of the length of the diagonal of a square with a unit length side (of length 1)\footnote{The ancient Greeks always represented the length as a rational number, so they called the side
and the diagonal of the square \pojem{incomparable lines}.}.


\begin{figure}[!htb]
\centering
\input{sl.aks.2.4.4.pic}
\caption{} \label{sl.aks.2.4.4.pic}
\end{figure}


        \bizrek \label{meraDalj1}
        For any measuring system of lengths it holds:

          (\textit{i}) $AB<CD\Rightarrow |AB|<|CD|$;

          (\textit{ii}) $|AB|=|CD|\Rightarrow AB\cong CD$.
        \eizrek

    \textbf{\textit{Proof.}}

\begin{figure}[!htb]
\centering
\input{sl.aks.2.4.5.pic}
\caption{} \label{sl.aks.2.4.5.pic}
\end{figure}

          (\textit{i}) From $AB<CD$ it follows that there is a point $T$ on the line $CD$, such that $CT\cong AB$. Because $\mathcal{B}(C,T,D)$, it is clear that $CD=CT+TD$ (Figure \ref{sl.aks.2.4.5.pic}). From the definition of the measure, it then follows: $|CD|=|CT|+|TD|=|AB|+|TD|>|AB|$.

\begin{figure}[!htb]
\centering
\input{sl.aks.2.4.6.pic}
\caption{} \label{sl.aks.2.4.6.pic}
\end{figure}

(\textit{ii}) Suppose that $AB\not\cong CD$. Without loss of generality, let $AB<CD$. But in this case, from the proven (\textit{i}), it follows that $|AB|<|CD|$, which is in contradiction with the assumption $|AB|=|CD|$. Therefore $AB\cong CD$ (Figure \ref{sl.aks.2.4.6.pic}).
    \kdokaz

Because, by definition, the lengths of the lines and the previous statement \ref{meraDalj1} are consistent only when they have the same length, we will often look at the \index{polmer krožnice}polmer $r$ of the circle $k(S,r)$ as the length of this radius.

The next statement will be given without proof.

            \bizrek \label{meraDaljice}
            Let $\textsl{m}:\mathcal{D}\rightarrow\mathbb{R}^+$ be the measure of the line. The mapping $\textsl{m}_1:\mathcal{D}\rightarrow\mathbb{R}^+$ also represents the measure exactly when there is such a positive real number $\mu$, that for every line $AB$ it holds:
            $$\textsl{m}_1(AB)=\mu\cdot\textsl{m}(AB).$$
            \eizrek

The measure of the line allows us to define a new concept.
 \index{razmerje!daljic}\pojem{Razmerje daljic} $AB$ and $CD$ with labels $AB:CD$ or $\frac{AB}{CD}$ is the ratio of the lengths of these two lines (Figure \ref{sl.aks.2.4.7a.pic}). Therefore:
  $$AB:CD=\frac{AB}{CD}=\frac{|AB|}{|CD|}.$$

\begin{figure}[!htb]
\centering
\input{sl.aks.2.4.7a.pic}
\caption{} \label{sl.aks.2.4.7a.pic}
\end{figure}

  It is clear that the ratio of two lines must always be the same number, regardless of the measuring system.

 We will therefore confirm the correctness of the previous definition with the following statement.

            \bizrek
            The ratio of two lines is not dependent on the measuring system.
            \eizrek

\textbf{\textit{Proof.}} Let $(\mathcal{D},\mathbb{R}^+,\textsl{m})$ and $(\mathcal{D},\mathbb{R}^+,\textsl{m}_1)$ be two measuring systems. By the previous statement \ref{meraDaljice} there exists a $\mu\in\mathbb{R}^+$, such that $\textsl{m}_1(PQ)=\mu\cdot\textsl{m}(PQ)$ for any distance $PQ$. For any distances $AB$ and $CD$ it therefore holds:
     $$\frac{\textsl{m}_1(AB)}{\textsl{m}_1(CD)}=
     \frac{\mu\cdot\textsl{m}(AB)}{\mu\cdot\textsl{m}(CD)}=
     \frac{\textsl{m}(AB)}{\textsl{m}(CD)},$$ which had to be proven. \kdokaz


 The concept of dividing a distance in a given ratio is easily extended, so that the ratio is a positive real number.
  We say that a point $T$ divides the distance $AB$ in the \index{delitev daljice!v razmerju}\pojem{ratio} $\lambda \in \mathbb{R}^+$, if $\mathcal{B}(A,T,B)$ and $\frac{AT}{TB}=\lambda$ (Figure \ref{sl.aks.2.4.7.pic}).

\begin{figure}[!htb]
\centering
\input{sl.aks.2.4.7.pic}
\caption{} \label{sl.aks.2.4.7.pic}
\end{figure}

 The equality of two ratios we will call the \pojem{ratio}. So the distances $AB$, $CD$, $EF$ and $GH$ (in this order)  are proportional, if:
  $$\frac{AB}{CD}=\frac{EF}{GH}.$$


On a similar way we define the measure of an angle.

  Let $\mathcal{K}$ be the set of all angles and $\mathbb{R}^+$ the set of all positive real numbers. A mapping $\textsl{l}:\mathcal{K}\rightarrow\mathbb{R}^+$, which satisfies the following properties:
  \begin{itemize}
    \item $(\exists \alpha_0\in\mathcal{K})\hspace*{1mm}\textsl{l}(\alpha_0)=1$,
    \item $(\forall \alpha, \beta\in\mathcal{K})\hspace*{1mm}(\alpha\cong \beta \Rightarrow\textsl{l}(\alpha)=\textsl{l}(\beta))$,
    \item $(\forall \alpha, \beta, \gamma\in\mathcal{K})\hspace*{1mm}(\alpha+\beta=\gamma\Rightarrow \textsl{l}(\alpha)+\textsl{l}(\beta)=\textsl{l}(\gamma))$.
  \end{itemize}
  we call the \index{mera!kota}\pojem{measure of an angle}, and the triple $\textsl{L}=(\mathcal{K},\mathbb{R}^+,\textsl{l})$ the \index{sistem merjenja!kotov}\pojem{measuring system of angles}.

The fact that the measure of the angle $\alpha$ is equal to $x$ (or $\textsl{l}(\alpha)=x$) will be more often written in the form $\alpha=x$.


  Similarly to measuring distances, there exists an infinite number of systems of measuring angles, which depend on the unit angle $\alpha_0$ - the one for which the measure is equal to 1, or $\textsl{l}(\alpha_0)=1$. So in one measuring system, the measure of any angle $\alpha$ is a positive real number $x$, for which $\alpha=x\cdot \alpha_0$. Therefore $\textsl{l}(\alpha)=x$ exactly when $\alpha=x\cdot \alpha_0$. Of course, we must also use the axiom of continuity in order to introduce multiplication of an angle by a positive real number.

  Of all measuring systems, we will highlight two.
  \begin{itemize}
    \item In the first system, which we will use most often, the unit angle is the 180-th part of the extended angle. For this angle we will say that it measures \pojem{one angular degree} and this measure is denoted by $1^0$ (Figure \ref{sl.aks.2.4.8.pic}). So if we use the properties of the measure function $\textsl{l}$, in this system the extended angle measures $180^0$, and the right angle measures $90^0$.
    \item In the second measuring system with \pojem{radians}, denoted by $[\textrm{rad}]$, the extended angle measures $\pi$ (where $\pi$ is an irrational number - $\pi\doteq 3,14$), and the right angle measures $\frac{\pi}{2}$. In this measuring system of angles, the notation $[\textrm{rad}]$ is usually not written.
  \end{itemize}


\begin{figure}[!htb]
\centering
\input{sl.aks.2.4.8.pic}
\caption{} \label{sl.aks.2.4.8.pic}
\end{figure}

  So in both systems we can write the measure of the extended angle $\alpha$ as $\alpha=180^0=\pi$, and the measure of the right angle $\beta$ as $\beta=90^0=\frac{\pi}{2}$ (Figure \ref{sl.aks.2.4.8a.pic}).

\begin{figure}[!htb]
\centering
\input{sl.aks.2.4.8a.pic}
\caption{} \label{sl.aks.2.4.8a.pic}
\end{figure}

The general relation between the two measuring systems can be written with the following formulas:

$$1\hspace*{1mm}\textrm{rad}=\frac{180^0}{\pi}, \textrm{ or }
1^0=\frac{\pi}{180^0}\hspace*{1mm}\textrm{rad}.$$



%________________________________________________________________________________
 \poglavje{Playfair's Axiom} \label{odd2AKSVZP}

As a result of the axioms of the previous four groups, we have already proven (the statement \ref{absolGeom}), that for a point $A$, which does not lie on the line $p$, there exists (in this plane) at least one line $q$, that goes through the point $A$ and does not intersect the line $p$ (Figure \ref{sl.aks.2.5.0.pic}).
But is such a line just one? Intuitively, the answer is affirmative. But can we prove it with the axioms we have so far?\footnote{This question is connected with the already mentioned problem of the fifth Euclidean postulate and was open for almost 2000 years.} It turns out that we cannot answer this question if we stay only with the first four groups of axioms.\footnote{This fact was first realized by the Russian mathematician \textit{N. I. Lobachevski} (1792--1856) and the Hungarian mathematician  \textit{J. Bolyai} (1802--1860). The first one to formally prove it was the French mathematician \index{Poincar\'{e}, J. H.} \textit{J. H.
Poincar\'{e}} (1854--1912).} The axioms we have so far form an incomplete system of axioms, because there is a statement, that we can formulate in this
theory, but we cannot determine if it is true or not.
  Therefore, we need to add a new axiom, with which we will decide, whether there is just one such line $q$ or there are more of them.


\begin{figure}[!htb]
\centering
\input{sl.aks.2.5.0.pic}
\caption{} \label{sl.aks.2.5.0.pic}
\end{figure}

        \baksiom \label{Playfair}\index{aksiom!Playfairjev}
         (Playfair's\footnote{
        \index{Playfair, J.}
        \textit{J. Playfair}
        (1748--1819), škotski matematik je predlagal to trditev kot ekvivalent petega evklidovega postulata.} axiom)
        For any given line $p$ and point $A$ not on $p$ (in the plane containing both line $p$ and point $A$), there is
        just one line  through the point $A$ that do not intersect the line $p$
         (Figure \ref{sl.aks.2.5.1.pic}).
        \eaksiom

\begin{figure}[!htb]
\centering
\input{sl.aks.2.5.1.pic}
\caption{} \label{sl.aks.2.5.1.pic}
\end{figure}

The second option offers the following axiom.

\baksiom \label{Lobac}\index{aksiom!Lobačevskega}
(Lobachevsky's\footnote{Russian mathematician \textit{N. I. Lobachevsky} (1792--1856) and Hungarian mathematician  \textit{J. Bolyai} (1802--1860) independently from each other built the first non-Euclidean geometry, which is based on
this axiom and the axioms of the first four groups.} axiom)
For any given line $p$ and point $A$ not on $p$ (in the plane containing both line $p$ and point $A$), there are
at least two lines  through the point $A$ that do not intersect the line $p$
(Figure \ref{sl.aks.2.5.1.pic}).
\eaksiom



In this way, we get two geometries, each of which is internally consistent.
The first geometry, which is determined by the axioms of the first four groups and Playfair's axiom \ref{Playfair}, is called
\index{geometrija!evklidska}\pojem{planar Euclidean geometry}. The second geometry, which is determined by the axioms of the first four groups and
Lobachevsky's axiom \ref{Lobac}, is called \index{geometrija!hiperbolična}\pojem{planar hyperbolic
geometry}.

Although they are obviously different, the aforementioned geometries have a lot in common. This is clear already because the first four groups of axioms are the same - they only differ in the fifth. The consequences of the first four
groups of axioms that we have considered so far, are valid in
both Euclidean and hyperbolic geometry.
The geometry that is based
 only on the first four groups of axioms, is called
\index{geometrija!absolutna}\pojem{planar absolute geometry}. It determines the common properties of Euclidean and hyperbolic geometry. Because the system of axioms that determines it is incomplete, we say that absolute geometry is a \pojem{incomplete theory}.

Despite the similarity, in hyperbolic geometry there are strange statements that are, of course, a consequence of Lobachevsky's axiom \ref{Lobac} or the negation of Playfair's axiom \ref{Playfair}. The sum of the internal angles of a triangle is always less than $180^0$ and is not constant; a rectangle of one acute angle does not always intersect the other angle; there are even triangles for which there is no inscribed circle, etc. Of course, these statements seem contradictory to us, but they are only contradictory in Euclidean geometry; in hyperbolic geometry they are not. It turns out that hyperbolic geometry - like Euclidean geometry - is a consistent theory\footnote{The consistency of hyperbolic geometry was first proved by the French mathematician \index{Poincaré, J. H.} \textit{J. H. Poincaré} (1854--1912), who built a model of hyperbolic geometry in Euclidean geometry. So the contradiction of hyperbolic geometry would mean a contradiction in Euclidean geometry.}.

In addition to the mentioned geometries, there are also other non-Euclidean geometries. Geometry in which every two lines of a plane intersect is called \index{geometry!elliptical}\pojem{elliptical geometry\footnote{This geometry was developed by the German mathematician \index{Riemann, G. F. B.} \textit{G. F. B. Riemann} (1828--1866).}}. From the already mentioned statement \ref{absolGeom} it is clear that elliptical geometry cannot be built on the axioms of the first four groups or absolute geometry. In this sense, this geometry differs more from Euclidean and hyperbolic geometry.

We also mention \index{geometry!projective}\pojem{projective geometry}. In a certain way, this is the most simple of all the mentioned geometries, because it is based only on three groups of axioms. Also in this geometry every two lines intersect, but unlike elliptical geometry, we do not have a defined metric or a relation of compliance. In projective geometry, it is possible to make models of all three geometries: Euclidean, hyperbolic and elliptical.

We will reiterate that in this book we are only building the Euclidean geometry of a plane. So, at the beginning, we called the set of all points $\mathcal{S}$ a plane. From now on, we will also denote it with $\mathbb{E}^2$ (or $\mathbb{E}^2=\mathcal{S}$). We have chosen only the axioms of a plane and thus obtained the system of axioms of Euclidean geometry of a plane. With a slightly different choice of basic concepts (the basic set of all points $\mathcal{S}$ is called space, certain subsets are lines and planes) and by adding new axioms (we would need to add them already in the first group), we would obtain the \pojem{Euclidean geometry of space} or, shorter, the \index{geometry!Euclidean}\pojem{Euclidean geometry}. In a similar way, we would obtain the \pojem{hyperbolic geometry}\index{geometry!hyperbolic} and the \index{geometry!absolute}\pojem{absolute geometry}.

We will only consider the Euclidean geometry of a plane, so we will always assume that all points, lines, and figures lie in the same plane (because, in this geometry, there are no others anyway). However, for the sake of clarity, we will occasionally emphasize this fact again.

In Euclidean geometry of a plane, it is now possible to introduce a new concept. We say that the lines $p$ and $q$ are \index{vzporednost!premic}\pojem{parallel}, which we denote $p\parallel q$, if they coincide or if they have no common points.
 $$p\parallel q \hspace*{2mm} \Leftrightarrow \hspace*{2mm}p=q \hspace*{1mm}\vee \hspace*{1mm} p\cap q=\emptyset .$$

Of course, in Euclidean geometry (of space), we would also need to add the condition that $p$ and $q$ lie in the same plane.

We can now express Playfair's axiom in the following form.



                \bizrek \label{Playfair1}
                For any given line $p$ and point $A$ not on $p$, there is
                just one line  through the point $A$ that is parallel to the line $p$
                (Figure \ref{sl.aks.2.5.1a.pic}).
                \eizrek


\begin{figure}[!htb]
\centering
\input{sl.aks.2.5.1a.pic}
\caption{} \label{sl.aks.2.5.1a.pic}
\end{figure}

 We will now prove an important property of the defined relation of parallelism.

\bizrek
        The relation of being parallel is an equivalence relation.
        \eizrek

\begin{figure}[!htb]
\centering
\input{sl.aks.2.5.2.pic}
\caption{} \label{sl.aks.2.5.2.pic}
\end{figure}

 \textbf{\textit{Proof.}} It is necessary (and sufficient) to prove that the relation is reflexive, symmetric and transitive (Figure \ref{sl.aks.2.5.2.pic}).


 (\textit{R}) The relation is reflexive by definition, because for every line $p$ it holds $p\parallel p$.

 (\textit{S}) If $p\parallel q$, then by definition it also holds $q\parallel p$, which means that the relation is symmetric.

 (\textit{T}) Assume that $p\parallel q$ and $q\parallel r$ (all three lines are in the same plane). We will prove that $p\parallel r$ also holds. If at least two of the three lines coincide, the proof is trivial. Assume that lines $p$ and $r$ intersect in a point $A$. In this case, through point $A$ there pass at least two lines that do not intersect line $q$, which is in contradiction with Playfair's axiom \ref{Playfair}. Therefore $p\parallel r$ holds, which means that the relation is transitive.
\kdokaz

We will also define the relation of parallelism for segments and distances. Segments (or distances) are \index{parallelism!segments}\index{parallelism!distances}\pojem{parallel}, if their supporting lines are parallel lines.

We will now prove an important theorem of Euclidean geometry.



        \bizrek \label{KotiTransverzala}
        Let $l$ be a line intersecting two distant lines $a$ and $b$ in points $A$ and $B$, respectively.
        If $X\in a$ and $Y\in b$ are points such that $X,Y\div l$, then:
         $$\angle XAB\cong\angle YBA\hspace*{1mm} \Leftrightarrow \hspace*{1mm}  a\parallel b.$$
        \eizrek


\begin{figure}[!htb]
\centering
\input{sl.aks.2.5.3.pic}
\caption{} \label{sl.aks.2.5.3.pic}
\end{figure}

 \textbf{\textit{Proof.}}

 ($\Rightarrow$) We will first assume that $\angle XAB\cong\angle YBA$. We will mark with $S$ the center of distance $AB$ and with $N$ the orthogonal projection of point $S$ on line $a$ (Figure \ref{sl.aks.2.5.3.pic}).

According to the statement \ref{sredZrcObstoj} there exists an isometry $\mathcal{I}$, which maps the point $S$ to the point $S$, for every point $T\neq S$ and its image $T'=\mathcal{I}(T)$ it holds that $S$ is the center of the line $TT'$. So first $\mathcal{I}:A,B\mapsto B,A$.
 Let $\mathcal{I}(X)=X'$ and $\mathcal{I}(N)=M$. We mark with $n$ the line $NM$. We prove that $n$ is the common perpendicular of the lines $a$ and $b$.

We prove first $\mathcal{I}:a\rightarrow b$ and $M\in b$. Because $S$ is the center of the line $XX'$, $X,X'\div S$ or $X,X'\div l$ and $X',Y\ddot{-} l$. From $\mathcal{I}:B,A,X\mapsto A,B,X'$ it follows $\angle XAB\cong \angle X'BA$. Because $\angle XAB\cong\angle YBA$, also $\angle X'BA\cong\angle YBA$ holds. Because the points $Y$ and $X'$ are also in the same plane $ABY$, according to the statement \ref{KotNaPoltrak} the line segment $BX'$ and $BY$ correspond. This means that the point $X'$ lies on the line segment $BY$, so $X'\in b$ holds. From $\mathcal{I}:A,X\mapsto B,X'$ it now follows (the axiom \ref{aksIII1}) $\mathcal{I}:a\rightarrow b$.
Because $N\in a$, also $\mathcal{I}(N)\in \mathcal{I}(a)$ or $M\in b$ holds.

From $\mathcal{I}:S,N\mapsto S,M$ according to the axiom \ref{aksIII1} it follows $\mathcal{I}:n\rightarrow n$.
So it holds $\mathcal{I}:a,n\rightarrow b,n$, so $\angle b,n\cong a,n=90^0$ or $b\perp n$ holds.
Because $n$ is the common perpendicular of different lines $a$ and $b$, according to the statement  \ref{absolGeom} the lines $a$ and $b$ don't have common points, which means that $a\parallel b$ holds.

\begin{figure}[!htb]
\centering
\input{sl.aks.2.5.3a.pic}
\caption{} \label{sl.aks.2.5.3a.pic}
\end{figure}

($\Leftarrow$) Let $a\parallel b$ now (Figure \ref{sl.aks.2.5.3a.pic}). We assume the opposite - that is, that $\angle XAB\cong\angle YBA$ does not hold. According to Theorem \ref{KotNaPoltrak}, there exists such a line segment $BZ$ in the plane $ABY$, that $\angle ZBA\cong\angle XAB$ holds. We mark the line $ZB$ with $b'$. From the first part of the proof ($\Rightarrow$) it follows that $b'\parallel a$. Therefore, both $b$ and $b'$ go through the point $B$ and are parallel to the line $a$ or, in other words, they do not have any common points with it (because $a\neq b$). According to Playfair's axiom, this is not possible, which means that $b=b'$. Therefore, the point $Z$ lies on the line segment $BY$, so $\angle YBA=\angle ZBA \cong\angle XAB$.
\kdokaz

A line $l$, that intersects lines $a$ and $b$, is called their \pojem{transversal}. The angles, that are determined by the line $l$ and lines $a$ and $b$, are called the \index{koti!ob transverzali}\pojem{angles on the transversal}. From the previous theorem \ref{KotiTransverzala} and theorem \ref{sovrsnaSkladna} it follows, that each of the two angles on the transversal $l$ of the parallel lines $a$ and $b$ is either congruent or supplementary (Figure \ref{sl.aks.2.5.3b.pic}).
Therefore, the following theorem holds.

\begin{figure}[!htb]
\centering
\input{sl.aks.2.5.3b.pic}
\caption{} \label{sl.aks.2.5.3b.pic}
\end{figure}


            \bizrek \label{KotiTransverzala1}
            If two parallel lines $a$ and $b$  are cut by a transversal $l$,
            then the angles on the transversal are either congruent or supplementary.
             \index{kota!z vzporednimi kraki}
             \eizrek



 Let's prove a generalization of the previous theorem.



            \bizrek \label{KotaVzporKraki}
            Angles with parallel sides are either congruent or supplementary.
             \index{kota!z vzporednimi kraki}
             \eizrek


\begin{figure}[!htb]
\centering
\input{sl.aks.2.5.4.pic}
\caption{} \label{sl.aks.2.5.4.pic}
\end{figure}

\textbf{\textit{Proof.}} Let $\angle aSb$ and $\angle a'S'b'$ be such angles that $a\parallel a'$ and $b\parallel b'$ (Figure \ref{sl.aks.2.5.4.pic}).
 If $b'\parallel a$, by Playfair's axiom, the sides of the $\angle a'S'b'$ match. From this it also follows for the sides of $\angle aSb$, which means that $\angle aSb$ and $\angle a'S'b'$ are both extended and consistent.
 With $S_1$ we mark the intersection of the sides of $\angle aSb$ and $\angle b'S'$. By the previous theorem \ref{KotiTransverzala1}, each of the two angles at the transversal $SS_1$ of the sides of $\angle b$ and $\angle b'$ is either congruent or supplementary to the angle at the transversal $S_1S'$ of the sides of $\angle a$ and $\angle a'$. Similarly, the angle $\angle aSb$ and $\angle a'S'b'$ are either congruent or supplementary.
 \kdokaz


In the following we will deal with the internal and external angles of a polygon. We start with a triangle.


         \bizrek \label{VsotKotTrik} The sum of the interior angles of a triangle is equal to
         $180^0$.
          \eizrek

\begin{figure}[!htb]
\centering
\input{sl.aks.2.5.7.pic}
\caption{} \label{sl.aks.2.5.7.pic}
\end{figure}

 \textbf{\textit{Proof.}} Let $ABC$ be an arbitrary triangle  (Figure \ref{sl.aks.2.5.7.pic}). By the consequence of Playfair's axiom \ref{Playfair1}, there is only one line $l$, which is parallel to the line $BC$. Let $Y$ and $Z$ be such points of the line $l$, that $Y,B\div AC$ and $Z,C\div AB$. The line $AB$ is the transversal of the sides of $BC$ and $l$. Because $Z,C\div AB$, by theorem \ref{KotiTransverzala}
 $\angle ABC\cong\angle ZAB$. Similarly, the line $AC$ is the transversal of the sides of $BC$ and $l$, so from  $Y,B\div AC$ and theorem \ref{KotiTransverzala} it follows that $\angle BCA\cong\angle YAC$. In the end we have:
  $$\angle ABC +\angle BAC +\angle BCA = \angle ZAB +\angle BAC +\angle CAY=\angle ZAY=180^0,$$ which was to be proven. \kdokaz

 The following theorems are very useful.


          \bizrek \label{zunanjiNotrNotr}
          An exterior angle of a triangle is equal to the sum of the two opposite interior angles.
           \eizrek

\textbf{\textit{Proof.}}  Let $ABC$ be an arbitrary triangle  (Figure \ref{sl.aks.2.5.6.pic}). We denote its internal angles at the vertices $A$, $B$ and $C$ with $\alpha$, $\beta$ and $\gamma$, and the corresponding external angles with  $\alpha'$, $\beta'$ and $\gamma'$. Without loss of generality, it is enough to prove that $\alpha+\beta=\gamma'$.
 From the previous  \ref{VsotKotTrik} follows: $$\alpha+\beta+\gamma=180^0.$$ Because both are external and internal angle to him by definition of a supplement, it is also $$\gamma'+\gamma=180^0.$$ From the previous two relations we obtain:
  $$\alpha+\beta=\gamma',$$ which had to be proven. \kdokaz

\begin{figure}[!htb]
\centering
\input{sl.aks.2.5.6.pic}
\caption{} \label{sl.aks.2.5.6.pic}
\end{figure}

A direct consequence is the following theorem.


       \bizrek \label{zunanjiNotrNotrVecji}
       An exterior angle of a triangle is greater than either opposite interior angle.
        \eizrek

\textbf{\textit{Proof.}} We introduce the same notation as in the previous  \ref{zunanjiNotrNotr} (Figure \ref{sl.aks.2.5.6.pic}). Without loss of generality, it is enough to prove that $\gamma'>\alpha$ and $\gamma'>\beta$. The relations are a direct consequence of the proven inequality $\alpha+\beta=\gamma'$ from the previous  \ref{zunanjiNotrNotr}.
 \kdokaz

Based on the internal angles, we can consider three types of triangles.
 We have already proven
  that in any triangle the sum of the internal angles is equal to $180^0$
  (the statement  \ref{VsotKotTrik}).
This means that at most one of these angles is a right angle or a reflex angle,
or at least two are acute. So we have three possibilities (Figure
\ref{sl.aks.2.6.4a.pic}):
\begin{itemize}
  \item The triangle is \index{trikotnik!ostrokotni}
  \pojem{acute-angled}, if all of its internal angles are acute.
  \item The triangle is \index{trikotnik!pravokotni}\pojem{right-angled},
    if it has one internal angle that is a right angle. The side opposite
the right angle of a right-angled triangle is called
\index{hipotenuza}\pojem{the hypotenuse}, the other two sides are
\index{kateta}\pojem{the catheti}. Because the sum of the internal angles in any
  triangle is equal to $180^0$, the
angles at the hypotenuse of a right-angled triangle are complementary.
  \item The triangle is \index{trikotnik!topokotni}\pojem{obtuse-angled},
  if it has one internal angle that is obtuse.
  The other two internal angles are then acute.
\end{itemize}

\begin{figure}[!htb]
\centering
\input{sl.aks.2.6.4a.pic}
\caption{} \label{sl.aks.2.6.4a.pic}
\end{figure}

%slika 33.1

 We will now prove a statement that applies to any $n$-gon.



         \bizrek \label{VsotKotVeck}
         The sum of the interior angles of any $n$-gon is equal to
         $(n - 2) \cdot 180^0$.
          \eizrek




\begin{figure}[!htb]
\centering
\input{sl.aks.2.5.5c.pic}
\caption{} \label{sl.aks.2.5.5c.pic}
\end{figure}

\textbf{\textit{Proof.}} First, let's assume that $A_1A_2\ldots A_n$ is a convex $n $-gon (Figure \ref{sl.aks.2.5.5c.pic}). Its $n-3$ diagonals $A_1A_3$, $A_1A_4$, ... $A_1A_{n-1}$ divide this polygon into $n-2$ triangles $\triangle_1$, $\triangle_2$, ..., $\triangle_{n-2}$. Because each internal angle of the $n $-gon $A_1A_2\ldots A_n$ is divided into the appropriate internal angles of the aforementioned triangles, the sum of all internal angles
 $n $-gon  $A_1A_2\ldots A_n$ is equal to the sum of all angles of triangles  $\triangle_1$, $\triangle_2$, ..., $\triangle_{n-2}$. By Theorem \ref{VsotKotTrik} in the end this sum is equal to exactly $(n - 2) \cdot 180^0$.

 At this point we will not prove the fact that even in the case of a non-convex $n $-gon  $A_1A_2\ldots A_n$ it can be divided into $n-2$ triangles.
 \kdokaz


         \bizrek \label{VsotKotVeckZuna}
         The sum of the exterior angles of any $n$-gon is equal to
         $360^0$.
          \eizrek



\begin{figure}[!htb]
\centering
\input{sl.aks.2.5.5b.pic}
\caption{} \label{sl.aks.2.5.5b.pic}
\end{figure}

 \textbf{\textit{Proof.}} We mark with $\alpha_1$, $\alpha_2$,..., $\alpha_n$ the internal and $\alpha'_1$, $\alpha'_2$, ..., $\alpha'_n$ the corresponding external angles at the vertices $A_1$, $A_2$,...,$A_n$ of a convex $n $-gon  $A_1A_2\ldots A_n$. (Figure \ref{sl.aks.2.5.5b.pic}). For the appropriate internal and external angle we have:

 \begin{eqnarray*}
 & & \alpha_1+\alpha'_1=180^0\\
 & & \alpha_2+\alpha'_2=180^0\\
 & & \vdots\\
 & & \alpha_n+\alpha'_n=180^0
 \end{eqnarray*}

 If we add all the equalities and take into account the proven equality from the previous theorem \ref{VsotKotVeck} $$\alpha_1+\alpha_2+\cdots + \alpha_n=(n - 2) \cdot 180^0,$$
  we get $(n - 2) \cdot 180^0+\alpha'_1+\alpha'_2+\cdots + \alpha'_n=n\cdot 180^0$ or
  $$\alpha'_1+\alpha'_2+\cdots + \alpha'_n=2 \cdot 180^0=360^0,$$ which was to be proven. \kdokaz

From the statement \ref{VsotKotVeck} it follows that the sum of all the internal angles of an arbitrary quadrilateral is equal to $360^0$, and from the statement \ref{VsotKotVeckZuna} it also follows that the sum of all the external angles of a convex quadrilateral is equal to $360^0$ (Figure \ref{sl.aks.2.5.5a.pic}).



\begin{figure}[!htb]
\centering
\input{sl.aks.2.5.5a.pic}
\caption{} \label{sl.aks.2.5.5a.pic}
\end{figure}



 Because a triangle (as the intersection of three planes) is a convex figure, the sum of all the external angles of an arbitrary triangle is equal to $360^0$ (Figure \ref{sl.aks.2.5.5d.pic}).

\begin{figure}[!htb]
\centering
\input{sl.aks.2.5.5d.pic}
\caption{} \label{sl.aks.2.5.5d.pic}
\end{figure}


A similar claim, based on the statement \ref{KotaVzporKraki}, which refers to an angle with parallel sides, also applies to an angle with perpendicular sides.



        \bizrek \label{KotaPravokKraki}
        Angles with perpendicular sides are either congruent or supplementary.
         \index{kota!s pravokotnimi kraki}
        \eizrek



\begin{figure}[!htb]
\centering
\input{sl.aks.2.5.5.pic}
\caption{} \label{sl.aks.2.5.5.pic}
\end{figure}


 \textbf{\textit{Proof.}} Let $\angle aSb$ and $\angle a'S'b'$ be such angles that $a\perp a'$ and $b\perp b'$ (Figure \ref{sl.aks.2.5.5.pic}). Let $A$ be the intersection of the sides of $a$ and $a'$, and $B$ be the intersection of the sides of $b$ and $b'$. In the quadrilateral $SAS'B$, the internal angles at the vertices $A$ and $B$ each measure $90^0$. According to the statement \ref{VsotKotVeck}, the sum of all the internal angles of this quadrilateral is equal to $360^0$. Therefore, the internal angle $BSA$ and $AS'B$ of this quadrilateral measure together $180^0$, which means that two angles, determined by the sides of $a$ and $b$ or $a'$ and $b'$, are supplementary. However, if we replace one of these angles with its supplement, the corresponding angles are congruent.
 \kdokaz


%________________________________________________________________________________
\naloge{Exercises}
\begin{enumerate}

\item Let $P$, $Q$ and $R$ be the internal points of the sides of a triangle
$ABC$. Prove that $P$, $Q$ and $R$ are non-collinear.

\item Let $P$ and $Q$ be points on sides $BC$ and $AC$ of triangle $ABC$ and at the same time different from its vertices. Prove that the line segments $AP$ and $BQ$ intersect in one point.

\item Points $P$, $Q$ and $R$ are located in order on sides $BC$, $AC$ and $AB$ of triangle $ABC$ and are different from its vertices. Prove that the line segments $AP$ and $QR$ intersect in one point.

\item A line $p$, which lies in the plane of the quadrilateral, intersects its diagonal $AC$ and does not pass through any vertex of this quadrilateral. Prove that the line $p$ intersects exactly two sides of this quadrilateral.

\item Prove that the half-line is a convex figure.

\item Prove that the intersection of two convex figures is a convex figure.

\item Prove that any triangle is a convex figure.

\item If $\mathcal{B}(A,B,C)$ and $\mathcal{B}(D,A,C)$, then $\mathcal{B}(B,A,D)$ is also true. Prove it.

\item Let $A$, $B$, $C$ and $D$ be such collinear points that $\neg\mathcal{B}(B,A,C)$ and $\neg\mathcal{B}(B,A,D)$ are true. Prove that $\neg\mathcal{B}(C,A,D)$ is also true.

\item Let $A_1A_2\ldots,A_{2k+1}$ be an arbitrary polygon with an even number of vertices. Prove that there is no line that intersects all of its sides.

\item If izometry $\mathcal{I}$ maps figure $\Phi_1$ and $\Phi_2$ into figure  $\Phi'_1$ and $\Phi'_2$, then the intersection $\Phi_1\cap\Phi_2$ is mapped by this izometry into the intersection $\Phi'_1\cap\Phi'_2$. Prove it.

\item Prove that any two strips of a plane are compatible with each other.

\item Prove that any two lines of a plane are compatible with each other.


\item Let $k$ and $k'$ be two circles of a plane with centers $O$ and $O'$ and radii $AB$ and $A'B'$. Prove the equivalence: $k\cong k' \Leftrightarrow AB\cong A'B'$.

\item Let $\mathcal{I}$ be a non-identical izometry of a plane with two fixed points $A$ and $B$. If $p$ is a line of this plane, which is parallel to the line $AB$ and $A\notin p$. Prove that there are no fixed points of izometry $\mathcal{I}$ on the line $p$.

\item   Let $S$ be the only fixed point
of the isometry $\mathcal{I}$ in some plane. Prove that if this isometry
maps the line $p$ to itself, then $S\in p$.

\item  Prove that any two lines
of a plane either intersect or are parallel.

\item  If a
line in a plane intersects one of the two parallel lines of the same plane,
 then it also intersects the other parallel line. Prove.

\item  Prove that every isometry maps a parallel line to a parallel line.

\item  Let $p$, $q$ and $r$ be such lines of a plane that
$p\parallel q$ and $r\perp p$. Prove that $r\perp q$.

\item Prove that a convex $n$-gon cannot have more than three
acute angles.

\end{enumerate}



% DEL 3 - - - - - - - - - - - - - - - - - - - - - - - - - - - - - - - - - - - - - - -
%________________________________________________________________________________
% SKLADNOST TRIKOTNIKOV. VEČKOTNIKI
%________________________________________________________________________________

 \del{Congruence. Triangles and Polygons} \label{pogSKL}

%________________________________________________________________________________
 \poglavje{Triangle Congruence Theorems} \label{odd3IzrSkl}

From the general definition of the congruence of figures it follows that two
triangles are congruent if there exists an isometry that maps the first triangle to
the second. It is clear that from the congruence of two triangles follows the
congruence of the corresponding sides and interior angles. We are interested in the inverse problem:
When does the congruence of some of
the corresponding sides and angles imply the congruence of two triangles? This is
expressed by the following \index{statement!about the congruence
of triangles}\emph{triangle congruence theorems}\footnote{The first, third and fourth triangle congruence theorems are attributed to \index{Pythagoras} \textit{Pythagoras of Samos} (6th century BC), while the second theorem is assumed to have been known to \index{Thales} \textit{Thales of Miletus} (7th--6th century BC). All four
are mentioned by \index{Euclid} \textit{Euclid of Alexandria}
(3rd century BC) in the first book of his \textit{Elements}.}:

\bizrek \label{SSS} (\textit{SSS})
                 Triangles are congruent if three sides of one triangle are congruent
                 to the corresponding sides of the other triangle,
                i.e. (Figure
                \ref{sl.skl.3.1.1.pic}):
           \begin{eqnarray*}
            \left.
             \begin{array}{l}
              AB \cong A'B'\\
             BC \cong B'C'\\
             AC \cong A'C'
            \end{array}
            \right\}\hspace*{1mm}\hspace*{1mm}\Rightarrow\triangle ABC \cong \triangle A'B'C'
            \end{eqnarray*}
             \eizrek

\begin{figure}[!htb]
\centering
\input{sl.skl.3.1.1.pic}
\caption{} \label{sl.skl.3.1.1.pic}
\end{figure}

\textbf{\textit{Proof.}}
 From the congruence of sides by \ref{izrek(A,B)} there is also $(A,B)
 \cong (A',B')$, $(B,C) \cong (B',C')$
 and $(A,C)
 \cong (A',C')$ or $(A,B,C) \cong (A',B',C')$. From
 \ref{IizrekABC} it follows that there is an isometry $\mathcal{I}$, which
 maps points $A$, $B$ and $C$ to points $A'$, $B'$ and $C'$.
 It maps triangle $ABC$ to triangle $A'B'C'$, which is
 a consequence of \ref{izrekIzoB}. So triangles $ABC$ and
 $A'B'C'$ are congruent.
 \kdokaz

\bizrek \label{SKS} (\textit{SAS})
                Triangles are congruent if two pairs of sides and the included angle of one triangle
                 are congruent to the corresponding sides and angle of the other triangle,
           i.e. (Figure
         \ref{sl.skl.3.1.2.pic}):
           \begin{eqnarray*}
            \left.
             \begin{array}{l}
              AB \cong A'B'\\
             AC \cong A'C'\\
             \angle BAC \cong \angle B'A'C'
            \end{array}
            \right\}\hspace*{1mm}\hspace*{1mm}\Rightarrow\triangle ABC \cong \triangle A'B'C'
            \end{eqnarray*}
            \eizrek


\begin{figure}[!htb]
\centering
\input{sl.skl.3.1.2.pic}
\caption{} \label{sl.skl.3.1.2.pic}
\end{figure}

\textbf{\textit{Proof.}}
  Because angles $BAC$ and $B'A'C'$ are congruent,
  there exists an isometry $\mathcal{I}$, which maps angle $BAC$ to angle $B'A'C'$.
  This isometry maps point $A$ to point $A'$, and segments $AB$ and $AC$ to $A'B'$ and $A'C'$.
  Let $\mathcal{I}(B)=\widehat{B}'$ and  $\mathcal{I}(C)=\widehat{C}'$.
  From this
  it follows that $AB \cong A'\widehat{B}'$ and $AC \cong A'\widehat{C}'$,
  but according to \ref{ABnaPoltrakCX} we have that $\widehat{B}'=B'$ and
  $\widehat{C}'=C'$.
  So the isometry $\mathcal{I}$
  maps points $A$, $B$ and $C$ to points $A'$, $B'$ and $C'$,
  or triangle $ABC$ to triangle $A'B'C'$, which means that triangles $ABC$ and
  $A'B'C'$ are congruent.
\kdokaz

\bizrek \label{KSK} (\textit{ASA})
                Triangles are congruent if two pairs of angles and the included side of one triangle
                 are congruent to the corresponding angles and side of the other triangle
                i.e. (Figure
                \ref{sl.skl.3.1.3.pic}):
           \begin{eqnarray*}
            \left.
             \begin{array}{l}
             AB \cong A'B'\\
             \angle BAC \cong \angle B'A'C'\\
             \angle ABC \cong \angle A'B'C'
            \end{array}
            \right\}\hspace*{1mm}\hspace*{1mm}\Rightarrow\triangle ABC \cong \triangle A'B'C'
            \end{eqnarray*}
            \eizrek

\begin{figure}[!htb]
\centering
\input{sl.skl.3.1.3.pic}
\caption{} \label{sl.skl.3.1.3.pic}
\end{figure}

\textbf{\textit{Proof.}}
 From axiom \ref{aksIII2} it follows that there exists an isometry  $\mathcal{I}$,
  that maps
 point $A$ to point $A'$, line segment $AB$ to line segment $A'B'$ and
 plane $ABC$ to plane $A'B'C'$.
 Because $AB \cong A'B'$, from the same axiom it follows
 $\mathcal{I}(B)=B'$. Let $\mathcal{I}(C)=\widehat{C}'$.
 Then $\angle BAC \cong \angle B'A'\widehat{C}'$ and
   $\angle ABC \cong \angle A'B'\widehat{C}'$.
  Because according to the assumption also $\angle BAC \cong \angle B'A'C'$ and
   $\angle ABC \cong \angle A'B'C'$, from 
   \ref{KotNaPoltrak} it follows that line segments $A'\widehat{C}'$ and
   $A'\widehat{C}'$ (or $B'C'$ and
   $B'\widehat{C}'$) are equal. Therefore $\widehat{C}'=A'\widehat{C}'\cap
   A'\widehat{C}'=A'C'\cap
   A'C'=C'$. So $\mathcal{I}:A,B,C \mapsto A',B',C'$, therefore triangles $ABC$ and $A'B'C'$ are congruent.
 \kdokaz

We will omit the proof of the fourth statement about congruent triangles.

\bizrek \label{SSK} (\textit{SSA})
            Triangles are congruent if two pairs of sides and the angle opposite to the longer side of one triangle
                 are congruent to the corresponding sides and angle of the other triangle.
            (Figure \ref{sl.skl.3.1.4.pic}).
            \eizrek

\begin{figure}[!htb]
\centering
\input{sl.skl.3.1.4.pic}
\caption{} \label{sl.skl.3.1.4.pic}
\end{figure}


One of the most important consequences of theorems about the congruence of triangles is the following statement.



             \bizrek \label{enakokraki}
             If two sides of a triangle are congruent, then angles opposite those sides are congruent.\\
             And vice versa:\\
            If angles opposite those sides are congruent, then two sides of a triangle are congruent.
            \eizrek


\begin{figure}[!htb]
\centering
\input{sl.skl.3.1.5.pic}
\caption{} \label{sl.skl.3.1.5.pic}
\end{figure}


 \textbf{\textit{Proof.}}
 Let $ABC$ be such a triangle that $AB \cong AC$
 (Figure \ref{sl.skl.3.1.5.pic}). Because $AC \cong AB$ and $BC \cong CB$ still hold, from the \textit{SSS} theorem it follows that
the triangles $ABC$ and $ACB$ are congruent (these two triangles have
different orientations). Therefore, $\angle ABC \cong \angle CBA$. In the same way, we could have proven the converse statement. In this case, we would have used the \textit{ASA} theorem.
 \kdokaz

 A triangle (as is the triangle $ABC$ from the previous theorem),
 which has at least two
sides congruent, is called
\index{trikotnik!enakokraki}\pojem{an isosceles triangle}. Each of the
two congruent sides is a \index{krak!enakokrakega
trikotnika}\pojem{arm}, and the third side is
\index{osnovnica!enakokrakega trikotnika}\pojem{the base} of this
triangle. So, according to the previous theorem, the angles at the
base of an isosceles triangle are congruent. And vice versa - if two
angles of a triangle are congruent, then this triangle is isosceles.

\bzgled
            Let the $E$ and $F$ be a points lies on the line containing  the hypotenuse $AB$ of a perpendicular
            triangle $ABC$ and let $B(E,A,B,F)$, $EA\cong AC$ and
              $FB\cong BC$. What is the measure of the angle $ACB$?
            \ezgled


\begin{figure}[!htb]
\centering
\input{sl.skl.3.1.6.pic}
\caption{} \label{sl.skl.3.1.6.pic}
\end{figure}


 \textbf{\textit{Solution.}} (Figure \ref{sl.skl.3.1.6.pic})

 The internal angle of the triangle $ABC$ at the
 vertices
 $A$ and $B$ we denote by $\alpha$ and
  $\beta$.
 The triangle $EAC$ and $CBF$ are
 isosceles triangles with the bases $CE$ and $CF$, so $\angle CEA \cong \angle ACE$ and
 $\angle CFB \cong \angle BFC$ (statement \ref{enakokraki}). The
 $\alpha$ is the external angle of the triangle $EAC$, so by the statement \ref{zunanjiNotrNotr}:
  $\alpha = 2\angle ECA$ or  $\angle ECA = \frac{1}{2} \alpha$.
  Similarly, $\angle FCB = \frac{1}{2} \beta$. Therefore:
\begin{eqnarray*}
   \angle ECF&=&\angle ECA+\angle ACB+\angle BCF=\\
   &=&\frac{1}{2}
   \cdot\alpha+90^0+
    \frac{1}{2} \cdot\beta=90^0+
    \frac{1}{2} \cdot\left(\alpha+\beta\right)=\\
    &=&90^0+
    \frac{1}{2}\cdot 90^0=135^0,
    \end{eqnarray*}
or $\angle ECF=135^0$. \kdokaz

A triangle in which all sides are equal is called
\index{trikotnik!enakostranični}\pojem{an equilateral triangle}
(Figure \ref{sl.skl.3.1.7.pic}), which is a special case
of an isosceles triangle. Therefore, from the above statement it follows that
all the internal angles of an equilateral triangle are equal. Since their sum is equal to $180^0$, each of these angles measures $60^0$. The converse is also true: if at least two angles of a triangle are equal to $60^0$ (and therefore also the third one), then this triangle is equilateral.


\begin{figure}[!htb]
\centering
\input{sl.skl.3.1.7.pic}
\caption{} \label{sl.skl.3.1.7.pic}
\end{figure}

\bzgled
Let $ABC$ be an equilateral triangle and $P$, $Q$ and $R$ points such that $\mathcal{B}(A,B,R)$, $\mathcal{B}(B,C,Q)$, $\mathcal{B}(C,A,P)$ and
$BR\cong CQ\cong AP$. Prove that $PQR$ is also an equilateral triangle.
\ezgled

\begin{figure}[!htb]
\centering
\input{sl.skl.3.1.8.pic}
\caption{} \label{sl.skl.3.1.8.pic}
\end{figure}


 \textbf{\textit{Proof.}} (Figure \ref{sl.skl.3.1.8.pic})

 From the given conditions, it follows first that
  $AR\cong BQ\cong CP$. The triangle $ABC$ is an equilateral triangle,
  so all three internal angles measure $60^0$. From this it follows
  that $\angle PAR\cong \angle RBQ\cong QCP$. By the \textit{SAS} theorem,
  the triangles $PAR$, $RBQ$ and $QCP$ are similar and $PR\cong RQ\cong QP$.
  This means that $PQR$ is also an equilateral triangle.
 \kdokaz


In the previous chapter we defined the perpendicular bisector of a line segment as the line that is perpendicular to the line segment and goes through its center. Now we prove an equivalent definition of the perpendicular bisector, which will be very important in the sequel. \index{perpendicular bisector}


             \bizrek \label{simetrala}
             The perpendicular bisector of a line segment $AB$ is the set of all points $X$
              that are equidistant from its endpoints, i.e. $AX \cong BX$.
             \eizrek

\begin{figure}[!htb]
\centering
\input{sl.skl.3.1.9.pic}
\caption{} \label{sl.skl.3.1.9.pic}
\end{figure}

 \textbf{\textit{Proof.}}
   Let $s$ be a line that is the perpendicular bisector of the line
segment $AB$ in some plane. By definition, $s$ is perpendicular to the line
$AB$ through its center - the point $S$. We denote by $\mathcal{M}$
the set of all points $X$ in this plane, for which $AX \cong BX$.
It is necessary to prove that $s =\mathcal{M}$. We will prove this by two inclusions (Figure \ref{sl.skl.3.1.9.pic}).

($s\subseteq \mathcal{M}$). Let $X \in s$. We will prove that
then $X \in \mathcal{M}$ is true. From the relations $AS \cong BS$, $XS \cong XS$
and $\angle ASX \cong \angle BSX = 90^0$ it follows that the triangles
$ASX$ and $BSX$ are congruent (theorem \ref{SAS}). Therefore $AX \cong BX$
or $X \in \mathcal{M}$.

($\mathcal{M}\subseteq s$). Now let $X \in \mathcal{M}$.
We will prove that $X \in s$ is true. From $X \in \mathcal{M}$ it follows that $AX \cong BX$.
Now from $AS \cong BS$, $XS \cong XS$ and $AX \cong BX$ it follows that
the triangles $ASX$ and $BSX$ are congruent (theorem \ref{SSS}). Therefore the angles
$ASX$ and $BSX$ are congruent and the angle between the lines is both a right angle. This means that
the line $XS$ is perpendicular to the line segment $AB$ in its center.
Therefore, the line $XS$ is the perpendicular bisector of $s$ or $X \in s$.
 \kdokaz

The next problem is an example of multiple use of the theorem of an isosceles triangle (\ref{enakokraki}).

      \bnaloga\footnote{42. IMO USA - 2001, Problem 5.}
      In a triangle $ABC$, let $AP$ bisect $\angle BAC$, with $P$ on $BC$, and let $BQ$ bisect $\angle ABC$, with $Q$ on $CA$.
It is known that $\angle BAC=60^0$ and that $|AB|+|BP|=|AQ|+|QB|$.
What are the possible measures of  interior angles of triangle $ABC$?
        \enaloga


\begin{figure}[!htb]
\centering
\input{sl.skk.4.9.IMO1.pic}
\caption{} \label{sl.skk.4.9.IMO1.pic}
\end{figure}

\textbf{\textit{Solution.}} Let's mark the interior angles of triangle $ABC$ with
$\alpha=60^0$, $\beta$ and $\gamma$. Let $D$ and $E$ be such
points that $BD\cong BP$, $\mathcal{B}(A,B,D)$, $QE \cong QB$
and $\mathcal{B}(A,Q,E)$ (Figure \ref{sl.skk.4.9.IMO1.pic}). From these
conditions it follows that $DBP$ and $BQE$ are isosceles triangles with
bases $DP$ and $BE$. From the given condition $|AB|+|BP|=|AQ|+|QB|$
it also follows that $AD\cong AE$, which means that $ADE$ is also an isosceles triangle with base $DE$.

Since $DBP$ is an isosceles triangle, from the statements \ref{enakokraki} and \ref{zunanjiNotrNotr} it follows: $\angle BDP\cong \angle BPD =\frac{1}{2}\angle ABC=\frac{1}{2}\beta$.
Since $BQE$ is also an isosceles triangle, $\angle QBE\cong\angle QEB$ follows.
From the congruence of triangles $ADP$ and $AEP$ (the statement \textit{SAS} \ref{SKS}) it follows that $\angle ADP\cong \angle AEP$ and $PD\cong PE$.

If we connect the proven relations, it holds:
 \begin{eqnarray*}
&& \angle AEP\cong \angle BDP
=\frac{1}{2}\beta\cong \angle QBP\\
&&\textrm{ and } \angle AEB\cong\angle QBE
 \end{eqnarray*}

 Let's first assume that $QB>QC$ or $\mathcal{B}(Q,C,E)$ holds. In this case:
 \begin{eqnarray*}
 \angle PEB &=&\angle AEB-\angle AEP=\\
 &=&\angle QBE-\angle QBP=\\
 &=&\angle PBE.
 \end{eqnarray*}
This means that $PBE$ is an isosceles triangle with the base $BE$ or $PE\cong PB$. But from the already proven $PE\cong PD$ and the assumption $PB\cong BD$ it follows that $PD\cong PB\cong BD$, therefore $BDP$ is an equilateral triangle. From this it follows that $\beta=2\angle BDP=2\cdot 60^0=120^0$, or $\alpha+\beta=60^0+120^0=180^0$, which is not possible (the statement \ref{VsotKotTrik}). Therefore the relation $QB>QC$ is not possible.

In a similar way, the relation $QB>QC$ leads to a contradiction. This means that only $QB\cong QC$ is possible. In this case $C=E$ and $\gamma=\angle ACB=\angle AEB\cong AEP=\frac{1}{2}\beta$ holds. From $\alpha+\beta+\gamma=180^0$, it follows that $60^0+\beta+\frac{1}{2}\beta=180^0$ or $\beta=80^0$.

We have shown that from the conditions of the task it follows that $\beta=80^0$. So the only possible solution is $\beta=80^0$. It is still necessary to show that $\beta=80^0$ is a solution, or that from $\alpha=60^0$, $\beta=80^0$ it follows that $|AB|+|BP|=|AQ|+|QB|$.
First, from $\angle QCB=\gamma=\frac{1}{2}\beta=40^0=\angle QBC$ it follows
 (from the statement \ref{enakokraki}) $QC\cong QB$ or
 $|AQ|+|QB|=|AQ|+|QC|=|AC|$.
If we define the point $D$ in the same way as in the first part, we again
get $\angle ADP\cong \angle BPD=\frac{1}{2}\angle ABC=40^0=\angle ACB=\angle ACP$.
This means that the triangles
$ADP$ and $ACP$ are congruent (from the statement \textit{ASA} \ref{KSK}) or
$AD\cong AC$. Therefore, in the end:
 $$|AB|+|BP|=|AB|+|BD|=|AD|=|AC|=|AQ|+|QB|,$$ which had to be proven. \kdokaz



%________________________________________________________________________________
 \poglavje{Constructions in Geometry} \label{odd3NacrtNaloge}

At the next example we will describe the so-called design tasks.


             \bzgled \label{načrt1odd3}
              Two congruent line segments $AB$ and $A'B'$ in a plane are given.
              Construct a point $C$ such that $\triangle ABC \cong \triangle A'B'C$.
             \ezgled


\begin{figure}[!htb]
\centering
\input{sl.skl.3.1.10.pic}
\caption{} \label{sl.skl.3.1.10.pic}
\end{figure}

 \textbf{\textit{Solution.}} We assume that $C$ is a point in the plane of the segments, for
 which $\triangle ABC \cong \triangle A'B'C$. Then $AC\cong A'C$
 and  $BC\cong B'C$ or the point $C$ lies on the perpendiculars of the segments $AA'$
 and $BB'$ (from the statement  \ref{simetrala}). This fact allows us to construct
  (Figure \ref{sl.skl.3.1.10.pic}).

 We draw the perpendiculars of the segments $AA'$
 and $BB'$. We get the point $C$ as their intersection.

 We prove that $C$ is the desired point or that it satisfies the conditions of
 the task. By assumption, $AB\cong A'B'$ already. Because we got the point $C$ as
 the intersection of the perpendiculars of the segments $AA'$
 and $BB'$, it is  $AC\cong A'C$ and $BC\cong B'C$. From the statement \ref{SSS} (SSS) it follows that the triangles $ABC$ and $A'B'C$
 are congruent.

The task has a solution (one) exactly when the lines of symmetry
  $AA'$
 and $BB'$ intersect, or when the lines $AA'$
 and $BB'$ are not parallel.
  \kdokaz

  The previous example is therefore called the \index{task!design}
   \pojem{design task}, in which
  for the given data it is necessary to plan or construct a new element or figure,
  which in relation to the given data
  satisfies certain conditions. The \pojem{planning} or \index{construction}
  \pojem{construction}
   means
  the use of a ruler and a compass or the use of
  \index{construction!elementary}
  \pojem{elementary construction}:\label{elementarneKonstrukcije}
\begin{itemize}
  \item for the given points $A$ and $B$ we draw:
 \begin{itemize}
  \item the line $AB$,
  \item the line segment $AB$,
  \item the midpoint $AB$;
\end{itemize}
 \item we draw the circle $k$:
\begin{itemize}
  \item with the center $S$, which goes through the given point $A$,
  \item with the center $S$ and the radius, which is consistent with the given
  line segment;
\end{itemize}
\item we draw the circular arc with the given center and radius,
\item we draw the intersection (or intersections):
\begin{itemize}
  \item of two lines,
  \item of a line and a circle,
  \item of two circles.
\end{itemize}
\end{itemize}

  The solution to the design task (draw the figure $\Phi$, which satisfies the conditions
  $\mathcal{A}$) is formally composed of four steps:
\begin{itemize}
  \item \textit{analysis} - in which we assume that
  the figure $\Phi$ is already designed and satisfies the conditions $\mathcal{A}$, then
  we look for new conditions $\mathcal{B}$, which the figure satisfies.
  These follow from the conditions $\mathcal{A}$ and are more favorable for
  the construction of the figure $\Phi$. We prove
  $\mathcal{A}\Rightarrow \mathcal{B}$.
  \item \textit{construction} - we plan the figure $\Phi'$, which satisfies
   the conditions
   $\mathcal{B}$. We exactly describe the course of the design.
  \item \textit{proof} - we prove that $\Phi' = \Phi$ or
  $\mathcal{B}\Rightarrow \mathcal{A}$.
  \item \textit{discussion} - we investigate
  the number of task solutions, depending on the conditions  $\mathcal{A}$.
\end{itemize}

\bzgled \label{konstrTrik1}
Line segments $a$, $l$ and an angle $\alpha$ are given. Construct a triangle
            $ABC$, such that the side $BC$ is congruent to the line segment $a$, the sum of the sides $AB + AC$
             equal to the line segment $l$ and the interior angle $BAC$ congruent to the angle $\alpha$  ($a$, $b+c$, $\alpha$).
        \ezgled


\begin{figure}[!htb]
\centering
\input{sl.skl.3.1.10a.pic}
\caption{} \label{sl.skl.3.1.10a.pic}
\end{figure}


 \textbf{\textit{Analysis.}} Let $ABC$ be a triangle, where $BC \cong a$, the sum of $AB+AC$ is equal to the given line segment $l$ and
$\angle BAC\cong \alpha$ (Figure \ref{sl.skl.3.1.10a.pic}). Let $D$ be a point on the line segment $BA$, such that $AD\cong AC$ and points $B$ and $D$ are on different sides
of point $A$. Therefore $BD=BA+AD=AB+AC=l$ or $BD\cong l$. Triangle $ACD$ is isosceles, so the angles $ADC$ and $ACD$
are congruent by Theorem \ref{enakokraki}. Because they are also the interior
angles of triangle $CAD$, both are equal to half of the exterior angle $BAC$ of this triangle (Theorem  \ref{zunanjiNotrNotr}), or
 $\angle BDC=\angle ADC\cong \angle ACD=\frac{1}{2}\angle BAC=\frac{1}{2}\alpha$.
 This allows us
to construct triangle $BCD$.

\textbf{\textit{Construction.}} First, let's construct triangle $BCD$, where
$\angle BDC=\frac{1}{2}\alpha$,
 $BC\cong a$ and $BD\cong l$,
then point $A$ as the intersection of the line segment $CD$'s perpendicular with line segment $BD$. We will prove that $ABC$
is the desired triangle.

\textbf{\textit{Proof.}} First, $BC\cong a$ by construction. By construction, point $A$ lies on the perpendicular of line
segment $CD$, so $AD\cong AC$ (Theorem \ref{simetrala}). Therefore triangle $CAD$ is isosceles with the base $CD$, so it is (Theorem \ref{enakokraki}) also $\angle ADC\cong \angle ACD$. Because of this, (Theorem \ref{zunanjiNotrNotr}) $\angle BAC = 2 \cdot \angle BDC= 2\cdot\frac{1}{2}\alpha=\alpha$. From $AD\cong AC$ it follows
 $AB + AC = AB + AD = BD \cong l$
.

\textbf{\textit{Discussion.}} The task has a solution (namely one or two) exactly when
the line segment $DC$ intersects the circle $k(B,a)$
and the perpendicular of line segment $CD$ intersects line segment $BD$.
 \kdokaz

In the future, we will not carry out all the steps in every design task.
In most cases, we will only do the first step and thus indicate
the course of the solution.




%________________________________________________________________________________
 \poglavje{Triangle inequality} \label{odd3NeenTrik}

First, we will prove two important expressions that are a consequence
of theorems about the congruence of triangles.

            \bizrek \label{vecstrveckot}
            One side of a triangle is longer than another side of a triangle if and only if
            the measure of the angle opposite the longer side is greater than the angle opposite the shorter side.
            \eizrek

\begin{figure}[!htb]
\centering
\input{sl.skl.3.2.1.pic}
\caption{} \label{sl.skl.3.2.1.pic}
\end{figure}

 \textbf{\textit{Proof.}} Let $ABC$ be a triangle in which $AC > AB$ (Figure \ref{sl.skl.3.2.1.pic}).
 We prove that then $\angle ABC > \angle ACB$. Because $AC > AB$, there is
such a point $B'$ between points $A$ and $C$, for which $AB \cong AB'$.
Then the triangle $BAB’$ is isosceles and $\angle ABB' \cong \angle AB'B$ (theorem \ref{enakokraki}).
The segment $BB'$ is inside the angle $ABC$, so $\angle ABC > \angle ABB '$.
Then $\angle AB'B$ is the external angle of the triangle $BCB'$. By
  theorem  \ref{zunanjiNotrNotrVecji} this angle is greater than its adjacent internal
angle $B'CB$. If we use what has been proven so far, we get:
 $$\angle ABC >
\angle ABB ' \cong \angle AB'B > \angle B'CB \cong \angle ACB.$$
 Therefore $\angle ABC
> \angle ACB$. In a similar way, we prove that the converse is also true.
\kdokaz

 If we denote the lengths of the sides $BC$, $AC$ and
 $AB$ of the triangle $ABC$ with $a$, $b$ and $c$, and the measures of the opposite angles at the vertices $A$, $B$ and $C$ with $\alpha$, $\beta$ and $\gamma$, we can write the previous theorem
 in the form:
  $$a > b \Leftrightarrow \alpha > \beta,$$
  the theorem about the isosceles triangle \ref{enakokraki} in the form:
 $$a = b \Leftrightarrow \alpha = \beta.$$


 This means that both expressions $a-b$ and $\alpha-\beta$ are both positive,
 both negative or both equal to zero. Thus we have proven the following property:

\bzgled \label{vecstrveckotAlgeb}
            For each triangle $ABC$ is:
             $$(a-b)(\alpha-\beta)\geq 0, \hspace*{4mm}
             (b-c)(\beta-\gamma)\geq 0, \hspace*{4mm}
             (c-a)(\gamma-\alpha)\geq 0$$
            \ezgled

The following proposition follows directly from 
\ref{vecstrveckot}:


            \bizrek \label{vecstrveckotHipot}
            The hypotenuse of a right-angled triangle is longer than its
            two legs.
            The longest side of an obtuse triangle is the one opposite to the obtuse angle.
             \eizrek

\begin{figure}[!htb]
\centering
\input{sl.skl.3.2.2.pic}
\caption{} \label{sl.skl.3.2.2.pic}
\end{figure}


\textbf{\textit{Proof.}} (Figure \ref{sl.skl.3.2.2.pic})

 The sum of the inner angles of a triangle is equal to $180^0$. Therefore, in
a right-angled triangle, the largest angle is a right angle. The hypotenuse of a right-angled triangle is the longest side of this triangle according to the previous proposition. We can similarly prove this for an obtuse triangle.
 \kdokaz

A line $AA'$ is the \index{height!of a triangle}\pojem{height} of a triangle $ABC$, if $AA'\perp BC$ and $A'\in BC$. The latter of the two relations means that the point $A'$ lies on the line $BC$, but not necessarily on the line segment $BC$. The relation $\mathcal{B}(B,A',C)$ is valid exactly when both of the internal angles at the vertices $B$ and $C$ are acute (Figure \ref{sl.skl.3.2.3.pic}). This is a consequence of the theorem about the sum of the internal angles of any triangle (theorem \ref{VsotKotTrik}). In the case that $\angle ABC\geq 90^0$ and $\mathcal{B}(B,A',C)$, the sum of the internal angles in the triangle $ABA'$ would be greater than $180^0$. So the height of a triangle is not always inside the triangle. In a right triangle, the heights from the two vertices with acute angles are equal to the corresponding catheti. The heights from the vertices $A$, $B$ and $C$ are usually denoted by $v_a$, $v_b$ and $v_c$. From the previous theorem \ref{vecstrveckotHipot} it follows that the length of the height of any triangle is less than or equal to the length of the opposite side of that triangle, e.g.: $v_a\leq b$, $v_a\leq c$, ...

\begin{figure}[!htb]
\centering
\input{sl.skl.3.2.3.pic}
\caption{} \label{sl.skl.3.2.3.pic}
\end{figure}

Now we will solve a design problem in which the height of a triangle is given as data.

\bzgled
        	 Construct a triangle $ABC$ such that the sides $AB$,
            $AC$ and the altitude  from the vertex $B$ are congruent to the three given line segments $c$, $b$ and $v_b$.
        \ezgled



\begin{figure}[!htb]
\centering
\input{sl.skl.3.2.4a.pic}
\caption{} \label{sl.skl.3.2.4a.pic}
\end{figure}

\textbf{\textit{Analysis.}} Let $ABC$ be a triangle for which $AB\cong c$,
 $AC\cong b$ and $AD\cong v_b$ (where $BD$ is the height of this triangle from the vertex $B$). In the right triangle $ABD$
therefore the known hypotenuse $AB\cong c$ and the cathetus
$AD\cong v_b$ are known, which means that we can design it. The third vertex $C$ of the triangle $ABC$ lies on the line $AD$ (Figure \ref{sl.skl.3.2.4a.pic}).

\textbf{\textit{Construction.}} Let's first draw a rectangular triangle
$ABD$ (with conditions: $AB \cong c$, $\angle ADB = 90^0$ and $BD \cong v_b$). On the line $AD$ then determine such a point $C$,
so that $AC \cong b$. Prove that $ABC$ is the desired triangle.


\textbf{\textit{Proof.}} First, $AB \cong c$ and
 $AC \cong b$ already by construction. Since $\angle ADB = 90^0$, $BD$ is the height of the triangle $ABC$ from the vertex $B$ and is consistent with the distance $v_b$ by construction.


\textbf{\textit{Discussion.}} The task has a solution exactly when it is possible
construction of the triangle $ABD$ or $hb \leq c$. In the construction of point $C$
there are two possibilities - on different sides of point $A$, which means that we have two solutions for triangle $ABC$. In the case $hb \cong c$, the solutions are a right triangle and a congruent triangle.
 \kdokaz



        \bzgled
        If $v_a$, $v_b$ and $v_c$ are altitudes corresponding
        to the sides $a$, $b$ and $c$ of a triangle, then:
        $$\frac{v_a}{b+c}+\frac{v_b}{a+c}+\frac{v_c}{a+b}<\frac{3}{2}.$$
        \ezgled

\textbf{\textit{Proof.}}
By adding the inequalities $v_a \leq b$,
$v_a \leq c$
we get $2v_a \leq b + c$ or $\frac{v_a}{b + c} \leq \frac{1}{2}$.
Similarly, we get $\frac{v_b}{a + c} \leq \frac{1}{2}$ and
$\frac{v_c}{a + b} \leq \frac{1}{2}$. Since all inequalities cannot be true at the same time,
by adding we get the desired inequality.
 \kdokaz

The next property of the triangle will be called \index{triangle
inequality} \pojem{triangle inequality}.


             \bizrek \label{neenaktrik}
             The sum of any two sides of a triangle is greater than the third side.
            \eizrek


\begin{figure}[!htb]
\centering
\input{sl.skl.3.2.4.pic}
\caption{} \label{sl.skl.3.2.4.pic}
\end{figure}

\textbf{\textit{Proof.}} Let $ABC$ be an arbitrary triangle. We will prove that $AB + AC > BC$. We mark a point $D$ so that $\mathcal{B}(B,A,D)$ and $AD \cong  AC$ (Figure \ref{sl.skl.3.2.4.pic}). By the \ref{enakokraki} theorem ($\triangle CAD$ is an isosceles triangle with the base $CD$), we also have that $\angle BDC=\angle ADC  \cong  \angle ACD$. The line segment $CA$ is inside the angle $DCB$, so $\angle ACD <  \angle DCB$. Then $\angle BDC <  \angle DCB$ as well. From \ref{vecstrveckot} (referring to the triangle $BCD$) it follows that: $$BC < BD = AB + AD = AB + AC,$$ which was to be proven. \kdokaz

From the triangle inequality we obtain a criterion for the existence of such a triangle, that its sides are consistent with three given line segments.


             \bzgled
            Let $a$, $b$ and $c$ be three line segments. A triangle with sides $a$,
            $b$ and $c$ exist if and only if:
             $$b + c > a,\hspace*{2mm}
             a + c > b \hspace*{1mm}\textrm{ in }\hspace*{1mm}
             a + b > c.$$
             \ezgled

\begin{figure}[!htb]
\centering
\input{sl.skl.3.2.5.pic}
\caption{} \label{sl.skl.3.2.5.pic}
\end{figure}

\textbf{\textit{Proof.}}  If such a triangle exists, then the three relations are direct consequences of the triangle inequality \ref{neenaktrik}. So we assume that all three relations are true. Without loss of generality, let $a$ be the longest side of this triangle (it is enough that it is not shorter than any other side) and let $B$ and $C$ be any points for which $BC \cong a$ (Figure \ref{sl.skl.3.2.5.pic}). Because $b + c > a$, this means that the circles $k(B,c)$ and $k(C,b)$ intersect in some point $A$ (a consequence of Dedekind's axiom - \ref{DedPoslKrozKroz} theorem, because each of them contains the inner points of the other), which is not on the line segment $BC$. The triangle $ABC$ is then the desired triangle. \kdokaz

If we know which of the three sides is the longest,
 it is enough to check only one inequality, as
 the other two are automatically fulfilled. The proof
 of the previous statement can also be used for the following, equivalent
 criterion.

             \bzgled \label{neenaktrik1}
               Let $a$, $b$ and $c$ be three line segments, such that $a \geq b,c$. A triangle with sides $a$,
            $b$ and $c$ exist if and only if $b + c > a$.
              \ezgled

For example, we can determine that a triangle exists with sides that have lengths 7, 5 and 3 (because $5+3>7$), but a triangle with sides that have lengths 9, 6 and 2
does not exist (because $6+2$ is not greater than 9).

 Let's look at some consequences of the previous statements.


             \bzgled
              If $X$ is an arbitrary point of the side $BC$ of a triangle $ABC$,
                then:
               $$AX < AB + AC.$$
             \ezgled

\begin{figure}[!htb]
\centering
\input{sl.skl.3.2.6.pic}
\caption{} \label{sl.skl.3.2.6.pic}
\end{figure}

\textbf{\textit{Proof.}} If we use the triangle inequality for
the triangles $ABX$ and $AXC$ (Figure \ref{sl.skl.3.2.6.pic}), we get:
 $$AX < AB + BX \hspace*{1mm} \textrm{ and }\hspace*{1mm}  AX < AC + CX.$$
By adding these two inequalities and using the triangle inequality for the triangle $ABC$, we get:
 $$2AX < AB + AC +
BC < 2(AB + AC),$$ which was to be proven. \kdokaz

%%  !!! Dosegel magično stran - 100!!! Wow Bravo!!!

The next inequality is a generalization of the previous one. In this sense, the previous statement is its consequence and it was not necessary to prove it separately.


            \bzgled
            Let $X$ be an arbitrary point of the side $BC$, different from the vertices $B$ and $C$ of a triangle $ABC$.
            Then the line segment $AX$ is shorter than at least one of the two line segments $AB$ and $AC$ i.e.:
            $$AX < \max\{AB, AC\}.$$
            \ezgled

\begin{figure}[!htb]
\centering
\input{sl.skl.3.2.7.pic}
\caption{} \label{sl.skl.3.2.7.pic}
\end{figure}

\textbf{\textit{Proof.}} Because for the point $X$ it holds that $\mathcal{B}(B, X, C)$,
one of the sides $AXB$ and $AXC$ is not acute. Without loss of
generality, let it be $AXC$ (Figure \ref{sl.skl.3.2.7.pic}). Then
it is the largest angle in the triangle $AXB$, which means that \\
$AX < AC$ (statement \ref{vecstrveckot}). Similarly, if the angle $AXB$ is not acute,
it holds that $AX < AB$.
 \kdokaz

We will especially consider the case of the distance $AX$, if the point $X$ is from
the previous two statements the center of the side $BC$ (Figure
\ref{sl.skl.3.2.8.pic}). Such a distance, which is determined by the vertex
and the center of the opposite side of the triangle, is called
\index{težiščnica trikotnika} \pojem{težiščnica} trikotnika.
Težiščnice, ki ustrezajo ogliščem $A$, $B$ in $C$ trikotnika $ABC$,
običajno označujemo s $t_a$, $t_b$ in $t_c$. Zadnji
dve trditvi lahko uporabimo tudi za težiščnice. Toda za težiščnice
velja še dodatna lastnost, ki jo bomo sedaj dokazali.

\begin{figure}[!htb]
\centering
\input{sl.skl.3.2.8.pic}
\caption{} \label{sl.skl.3.2.8.pic}
\end{figure}



             \bzgled \label{neenTezisZgl}  If $a$, $b$, $c$ are the sides and $t_a$
             the corresponding median of a triangle $ABC$, then:
            $$\frac{b+c-a}{2}<t_a<\frac{b+c}{2}.$$
              \ezgled

\begin{figure}[!htb]
\centering
\input{sl.skl.3.2.9.pic}
\caption{} \label{sl.skl.3.2.9.pic}
\end{figure}

\textbf{\textit{Proof.}} We mark with $A_1$ the center of the side $BC$
of the triangle $ABC$. Then $t_a  =AA_1$  (Figure
\ref{sl.skl.3.2.9.pic}).

If we use the triangle inequality for the triangle $ABX$ and $ACX$ (statement \ref{neenaktrik}), we get: $AA_1 + A_1B > AB$ and
$AA_1 + A_1C > AC$  or:
$$t_a+\frac{a}{2}>c \hspace*{2mm} \textrm{ in } \hspace*{2mm}
 t_a+\frac{a}{2}>b.$$
If we add these two inequalities, we get $\frac{b+c-a}{2}<t_a$.
We will also prove $t_a<\frac{b+c}{2}$.
Let $D$ be the point, for which
 $A_1D \cong AA_1$ and $\mathcal{B}(A,A_1,D)$. The triangles $AA_1B$ and $DA_1C$
  are
congruent (statement \textit{SAS} \ref{SKS}), which means that $AB \cong DC$. If
we use the triangle inequality again (statement \ref{neenaktrik}) for
the triangle $ACD$, we get:
$$b+c = AC + AB = AC + CD > AD = 2AA_1 = 2t_a,$$ which had to be proven. \kdokaz

 We will show some more examples of the use of the triangle inequality.


             \bzgled
             Let $M$ be an arbitrary point of the bisector of the exterior angle at
            the vertex $C$ of a triangle $ABC$. Then
            $$MA + MB \geq CA + CB.$$
             \ezgled

\begin{figure}[!htb]
\centering
\input{sl.skl.3.2.10.pic}
\caption{} \label{sl.skl.3.2.10.pic}
\end{figure}


 \textbf{\textit{Proof.}} Let $D$ be such a point on the line segment $BC$, that $AC \cong CD$
 and $\mathcal{B}(A,C,D)$ (Figure
\ref{sl.skl.3.2.10.pic}). The triangles $ACM$ and $DCM$ are congruent,
by  statement \textit{SAS}  \ref{SKS} ($AC \cong DC$, $CM \cong CM$, $\angle ACM \cong
\angle DCM$). Therefore $MA \cong MD$. If we use the triangle inequality now, we get: $$MA + MB = MD + MB \geq BD = DC +
CB = CA + CB.$$
 Of course, the equality holds when the points $B$, $M$ and $D$ are collinear or $M = C$.
 \kdokaz



             \bzgled
            If the bisector of the interior angle at
            the vertex $A$ of a triangle $ABC$ intersects the side $BC$ in the point $E$, then
            $$AB > BE \hspace*{2mm} \textrm{ in }\hspace*{2mm}  AC > CE .$$
             \ezgled

\begin{figure}[!htb]
\centering
\input{sl.skl.3.2.11.pic}
\caption{} \label{sl.skl.3.2.11.pic}
\end{figure}

 \textbf{\textit{Proof.}}  Because $\mathcal{B}(B,E,C)$, the angle
 $AEB$ is the external angle  of triangle $AEC$ (Figure
\ref{sl.skl.3.2.11.pic}). Therefore $\angle BEA > \angle EAC \cong
\angle BAE$ (statement \ref{zunanjiNotrNotrVecji}). Opposite the larger
angle in triangle $BAE$ is the larger side, or $AB > BE$
(statement \ref{vecstrveckot}). Similarly, we prove that the other of the two relations is also true.
 \kdokaz


             \bzgled \label{zgled3.2.9}
            If $M$ is the interior point of a triangle $ABC$, then \\
            $BA + AC > BM + MC.$
            \ezgled

\begin{figure}[!htb]
\centering
\input{sl.skl.3.2.12.pic}
\caption{} \label{sl.skl.3.2.12.pic}
\end{figure}

\textbf{\textit{Proof.}} Let $N$ be the intersection of lines $BM$ and
$CA$ (Figure \ref{sl.skl.3.2.12.pic}). Because $M$ is an interior point of
triangle $ABC$, $\mathcal{B}(B,M,N)$ and
$\mathcal{B}(A,N,C)$ are true. If we now use the triangle inequality (statement \ref{neenaktrik}) twice, we get:
 \begin{eqnarray*}
\hspace*{-4mm}BM + MC &<& BM + (MN + NC) = (BM + MN) + NC = BN + NC\\
 \hspace*{-4mm}&<& (BA +
AN) + NC = BA + (AN + NC) = BA + AC.
  \end{eqnarray*}

Let's define two new concepts. The sum of all sides of a polygon is called its \index{obseg!večkotnika} \pojem{obseg}. Half of this sum is the \pojem{polobseg} of this polygon.



             \bzgled
            If $M$ is the interior point and $s$ the semiperimeter of a triangle $ABC$, then
             $$s < AM + BM + CM < 2s.$$
             \ezgled

\begin{figure}[!htb]
\centering
\input{sl.skl.3.2.13.pic}
\caption{} \label{sl.skl.3.2.13.pic}
\end{figure}

\textbf{\textit{Proof.}}
  We first get the inequality, if
we use the triangle inequality for the triangles $MAB$,
$MBC$ and $MCA$ and then add them up. The second inequality is obtained,
if we use the previous statement (Example \ref{zgled3.2.9})
and add the corresponding inequalities (Figure \ref{sl.skl.3.2.13.pic}).
 \kdokaz


             \bzgled
             In each convex pentagon there exist three diagonals,
             which are congruent to the sides of a triangle.
             \ezgled

\begin{figure}[!htb]
\centering
\input{sl.skl.3.2.14.pic}
\caption{} \label{sl.skl.3.2.14.pic}
\end{figure}

\textbf{\textit{Proof.}}
 Let $AD$ be the longest diagonal
of the pentagon $ABCDE$ (not shorter than any other
diagonal). We prove that $AD$, $AC$ and $BD$ are the desired diagonals,
i.e. those for which there exists a triangle, whose sides
 are congruent to these diagonals (Figure \ref{sl.skl.3.2.14.pic}). Because
 $AD\geq AC$ and $AD\geq BD$, it is enough to prove (Example \ref{neenaktrik1}),
that $AC + BD > AD$. The pentagon $ABCDE$ is convex,
so its diagonals $AC$ and
$BD$ intersect in some point $S$. Then: $$AC + BD > AS + SD > AD,$$ which was to be proven. \kdokaz

The following consequence of the theorems about the congruence of
triangles is very important. We will also need the triangle inequality in this proof.


             \bizrek \label{SkladTrikLema}
             Let $ABC$ and $A'B'C'$ triangles such that $AB \cong A'B'$
              and $AC \cong A'C'$. Then $BC > B'C'$ if and only if
             $\angle BAC > \angle B' A'C'$ i.e.
             $$BC > B'C' \Leftrightarrow \angle BAC > \angle B'
            A'C'.$$
             \eizrek

\begin{figure}[!htb]
\centering
\input{sl.skl.3.2.15.pic}
\caption{} \label{sl.skl.3.2.15.pic}
\end{figure}

\textbf{\textit{Proof.}} (Figure \ref{sl.skl.3.2.15.pic})

($\Leftarrow$) Let $\angle BAC > \angle B' A'C'$. Then there exists
within the angle $BAC$ such a line segment $l$, that $\angle BA,l \cong \angle B'A'C'$.
With $C''$ we mark the point of the line segment $l$, for which
$AC'' \cong A'C'$. Then both triangle $ABC''$ and
$A'B'C'$ are congruent and $BC'' \cong B'C'$. It is enough to prove that
$BC > BC''$. If $C''$ lies on the side $BC$, it is trivially
fulfilled. We assume that the point $C''$ does not lie on the side $BC$.
Let the point $E$ be the intersection of the perpendicular bisector of the angle $CAC''$ and the side $BC$.
By the \textit{SAS} theorem, the triangles $ACE$ and $AC''E$ are congruent, therefore
$CE \cong C''E$. Now it is:
$$BC = BE + EC = BE + EC'' \hspace{0.1mm} > BC'' = B'C'.$$
 ($\Rightarrow$) Let $BC > B'C'$. The relation $\angle BAC \cong
  \angle B' A'C'$ is not true,
  because then (by the \textit{SAS} theorem) the triangle
$ABC$ and $A'B'C'$ would be congruent and then also $BC \cong B'C'$. If
it would be true that $\angle BAC < \angle B' A'C'$,  then from what has already been proven it would follow that $BC < B'C'$. Therefore $\angle BAC > \angle B' A'C'$.
 \kdokaz

             \bizrek \label{neenakIzlLin}
             If $A_1A_2\ldots A_n$ ($n\in \mathbb{N}$, $n\geq 3$) is polygonal chain, then
             $$|A_1A_2|+|A_2A_2|+\cdots +|A_{n-1}A_n|\geq |A_1A_n|.$$
             \eizrek


\begin{figure}[!htb]
\centering
\input{sl.skl.3.2.16.pic}
\caption{} \label{sl.skl.3.2.16.pic}
\end{figure}

\textbf{\textit{Proof.}} We will carry out the proof by induction over $n$
(Figure \ref{sl.skl.3.2.16.pic}).

 In the case $n=3$ we get the triangle inequality - izrek
 \ref{neenaktrik}.

Let's assume that the inequality is true for $n=k$ ($k\in \mathbb{N}$, $k> 3$) or
  $|A_1A_2|+|A_2A_2|+\cdots +|A_{k-1}A_k|\geq |A_1A_k|.$ We will prove that
  the inequality is also true for $n=k+1$ or
  $|A_1A_2|+|A_2A_2|+\cdots +|A_kA_{k+1}|\geq |A_1A_{k+1}|.$ If
  we first use the induction
  assumption, and then the triangle inequality, we get:
  \begin{eqnarray*}
   && |A_1A_2|+|A_2A_2|+\cdots +|A_{k-1}A_k|+|A_kA_{k+1}|\geq\\
   && \geq|A_1A_k|+|A_kA_{k+1}|\geq |A_1A_{k+1}|,
  \end{eqnarray*}
 which is what needed to be proven. \kdokaz

We will now prove another inequality that is true in any triangle.


             \bzgled
             If $a$, $b$, $c$ are the sides and $\alpha$, $\beta$, $\gamma$
              the opposite interior angles of a triangle, then
              $$60^0\leq \frac{a\alpha+b\beta +c\gamma}{a+b+c} < 90^0.$$
               \ezgled

\textbf{\textit{Proof.}} We will prove each of the inequalities separately. In doing so, we will use the statement about the sum of the interior angles of a triangle (statement \ref{VsotKotTrik}). First, we will prove the second inequality:
 \begin{eqnarray*}
  \frac{a\alpha+b\beta +c\gamma}{a+b+c} < 90^0
  &\Leftrightarrow& a\alpha+b\beta +c\gamma - 90^0(a+b+c)<0\\
  &\Leftrightarrow& a(\alpha-90^0)+b(\beta-90^0) +c(\gamma-90^0)<0\\
  &\Leftrightarrow& a(180^0-2\alpha)+b(180^0-2\beta) +c(180^0-2\gamma)>0\\
  &\Leftrightarrow& a(\beta+\gamma-\alpha)+b(\alpha+\gamma-\beta) +
  c(\alpha+\beta-\gamma)>0\\
  &\Leftrightarrow& \alpha(b+c-a)+\beta(a+c-b) +
  \gamma(a+b-c)>0
 \end{eqnarray*}
 The last inequality is fulfilled because, according to the triangle inequality (statement \ref{neenaktrik}), $b+c-a>0$, $a+c-b>0$ and  $a+b-c>0$ hold.
 We will now prove the first inequality:
 \begin{eqnarray*}
  && \frac{a\alpha+b\beta +c\gamma}{a+b+c} \geq 60^0\Leftrightarrow\\
  &\Leftrightarrow& a\alpha+b\beta +c\gamma - 60^0(a+b+c)\geq 0\\
  &\Leftrightarrow& a(\alpha-60^0)+b(\beta-60^0) +c(\gamma-60^0)\geq 0\\
  &\Leftrightarrow& a(3\alpha-180^0)+b(3\beta-180^0) +c(3\gamma-180^0)\geq 0\\
  &\Leftrightarrow& a(2\alpha-\beta-\gamma)+b(2\beta-\alpha-\gamma) +
  c(2\gamma-\alpha-\beta)\geq 0\\
  &\Leftrightarrow& a(\alpha-\beta+\alpha-\gamma)+b(\beta-\alpha+\beta-\gamma) +
  c(\gamma-\alpha+\gamma-\beta)\geq 0\\
  &\Leftrightarrow& a(\alpha-\beta)+a(\alpha-\gamma)+
  b(\beta-\alpha)+b(\beta-\gamma) +
  c(\gamma-\alpha)+c(\gamma-\beta)\geq 0\\
  &\Leftrightarrow& (a-b)(\alpha-\beta)+(a-c)(\alpha-\gamma)+
  (b-c)(\beta-\gamma) \geq 0
 \end{eqnarray*}
 The last inequality is a consequence of statement \ref{vecstrveckotAlgeb}.
 \kdokaz

 The next statement will be the motivation for defining the distance of a point from a line.

             \bizrek Let $A'=pr_{\perp p}(A)$ be the foot of the perpendicular from a point  $A$ on a line $p$.
            If $X\in p$ and $X\neq A'$, then $AX>AA'$.
            \eizrek

\begin{figure}[!htb]
\centering
\input{sl.skl.3.2.17.pic}
\caption{} \label{sl.skl.3.2.17.pic}
\end{figure}

  \textbf{\textit{Proof.}}
 By definition, $AA'\perp p$
(Figure \ref{sl.skl.3.2.17.pic}), which means that $AA'X$
 is a right angled triangle with hypotenuse $AX$. From 
 \ref{vecstrveckotHipot} it follows that $AX>AA'$.
 \kdokaz

 If $A'=pr_{\perp
p}(A)$, we say that the length of the line $AA'$ \index{distance!point
 from a line} \pojem{distance of point $A$ from line $p$}.
We denote it by $d(A,p)$. So $d(A,p)=|AA'|$.




%________________________________________________________________________________
 \poglavje{Circle and Line} \label{odd3KrozPrem}

In the following we will deal with a circle and the mutual
position of a circle and a line. We prove first a property of the
diameter of a circle, which is a simple consequence of the
triangle inequality.


            \bizrek \label{premerNajdTetiva}
               The longest chord of a circle is its diameter.
             \eizrek

\begin{figure}[!htb]
\centering
\input{sl.skl.3.3.1.pic}
\caption{} \label{sl.skl.3.3.1.pic}
\end{figure}

 \textbf{\textit{Proof.}}  Let
$AB$ be any chord of a circle that is not a diameter, and $C$ a
point on the line $AS$, for which $CS\cong SA$ and $\mathcal{B}(A,S,C)$
(Figure \ref{sl.skl.3.3.1.pic}). Then point $C$ lies on
the circle $k$ and $AC$ is its diameter. We have already shown (a consequence
of \ref{premerInS}) that all diameters of a circle
are mutually congruent. So it is enough to show that $AC>AB$. This follows 
from the triangle inequality (triangle $ASB$). It holds:
 $$AC=AS+SC=AS+SB>AB,$$ which was to be proved. \kdokaz

It follows another property of a chord of a circle, as a consequence of 
\ref{vecstrveckotHipot}.


         \bzgled \label{tetivaNotrTocke}
        Every point that lies on a chord of a circle, except its endpoints,
         is an interior point of the circle.
         \ezgled

\begin{figure}[!htb]
\centering
\input{sl.skl.3.3.2.pic}
\caption{} \label{sl.skl.3.3.2.pic}
\end{figure}

\textbf{\textit{Proof.}} Let $X$ be an inner point of the segment $AB$
with shorter sides on the circle $k(S,r)$ (Figure \ref{sl.skl.3.3.2.pic}). The angle $AXS$ and
$BXS$ are adjacent angles, which means that they are not both acute angles. Without
loss of generality, assume that the angle $BXS$ is not an acute angle. Then in
the triangle $SXB$ the side $SB$ is the longest  (by
\ref{vecstrveckotHipot}), which means that:
 $$SX<SB=r.$$
Therefore, $X$ is an inner point of the circle  $k(S,r)$.
 \kdokaz

Similarly, we will prove the following important statement.


        \bizrek \label{KrogKonv}
         A circular disc is a convex set.
          \eizrek

\begin{figure}[!htb]
\centering
\input{sl.skl.3.3.3.pic}
\caption{} \label{sl.skl.3.3.3.pic}
\end{figure}

\textbf{\textit{Proof.}} Let $A$ and $B$ be two points of the circle
$\mathcal{K}(S,r)$ (Figure \ref{sl.skl.3.3.3.pic}). It is necessary to prove that
the whole segment $AB$ lies in this circle, or that this is true for any
point $X$, for which $\mathcal{B}(A,X,B)$. Because points $A$ and $B$
lie in the circle $\mathcal{K}$, then $SA, SB\leq r$.
As in the proof of the previous statement, we write: because
$\mathcal{B}(A,X,B)$, then at least one of the adjacent angles $AXS$ and $BXS$ is not
acute. Without loss of generality, let $\angle BXS\geq 90^0$. If
we use the statement \ref{vecstrveckotHipot} for the triangle $SXB$, we get:
$$SX<SB\leq r.$$
 Therefore, the point $X$ lies in the circle $\mathcal{K}$, which means that $\mathcal{K}$
is a convex figure.
 \kdokaz

It is intuitively clear that a line and a circle can have at most
two common points. We will now prove this fact.

        \bizrek \label{KroznPremPresek}
        A line and a circle can have at most two common points.
        \eizrek

\begin{figure}[!htb]
\centering
\input{sl.skl.3.3.4.pic}
\caption{} \label{sl.skl.3.3.4.pic}
\end{figure}

\textbf{\textit{Proof.}} We assume the opposite, that the circle
$k(S,r)$ and the line $p$ have at least three different common
points: $A$, $B$ and $C$, or $A,B,C\in p\cap k$ (Figure
\ref{sl.skl.3.3.4.pic}). If the center $S$ lies on the line $p$,
then on this line there are only two points that are distant from
the point $S$ by the radius $r$ (statement \ref{ABnaPoltrakCX}). Let
$S\notin p$. From the condition $A,B,C\in p\cap k$ it follows that
$SA=SB=SC=r$, which means that the triangles $ASC$, $ASB$ and $BSC$
are isosceles. Without loss of generality, we assume that
$\mathcal{B}(A,C,B)$. From this it follows (statement
\ref{enakokraki}), that the angles:
 $$\angle  SCA\cong \angle SAC \cong \angle SBC \cong \angle SCB.$$
are congruent. Therefore, the angle $SCA$ and the angle $SCB$ are
congruent and are both right angles. Then the angle $SAC$ is also
a right angle. But this is not possible, because in this case the
triangle $SAC$ would have two right internal angles. This means
that the assumption $A,B,C\in p\cap k$ is false. \kdokaz

 From statement \ref{KroznPremPresek} it follows that a line and a circle can have two, one or no common
points. In the first case we say that the line and the circle
\pojem{intersect}, in the second case they \pojem{touch}, in the
third case they are
 \pojem{non-intersecting} (Figure
\ref{sl.skl.3.3.5.pic}). The line is in the first case \index{secant
of a circle}\pojem{secant} or \pojem{secant}, in the second case
\index{tangent of a circle}\pojem{tangent} or \pojem{tangent}, and
in the third case \index{non-secant of a circle}\pojem{non-secant}.
The point in which the tangent touches the circle is called
\index{tangent point}\pojem{tangent point}.

\begin{figure}[!htb]
\centering
\input{sl.skl.3.3.5.pic}
\caption{} \label{sl.skl.3.3.5.pic}
\end{figure}

We often use the following criterion for the tangent, which is
actually a necessary and sufficient condition for a line to be a
tangent to a circle.

\bizrek \label{TangPogoj}
Let $T$ be a point lying on the circle $k(S, r)$. A line
$PT$ (lying in the plane of the circle) is a tangent of the circle at the point $T$ if and only if $PT \perp TS$.
\eizrek

\textbf{\textit{Proof.}}  ($\Rightarrow$) Let $PT$ be a tangent
of the circle $k$ at the point $T$. If the angle $PTS$ is not a right angle, one
of the angles, determined by the lines $PT$ and $TS$, is acute. Without loss of
generality, let $\angle STX =w < 90°$ (Figure
\ref{sl.skl.3.3.6.pic}). With $l$ we denote
 the half-line with the origin $S$, lying in the plane $STX$, so that
 $\angle ST,l = 180° - 2w $. If $Y$ is the intersection of the half-lines $TX$ and $l$,
the triangle $STY$ is isosceles ($\angle STY = \angle SYT =w$ ) and $ST = SY = r$. But this is not possible, because $PT$ is a tangent of the circle $k$
and they have only one common point. Therefore, the angle $PTS$ is a right angle or $PT \perp TS$.

\begin{figure}[!htb]
\centering
\input{sl.skl.3.3.6.pic}
\caption{} \label{sl.skl.3.3.6.pic}
\end{figure}

($\Leftarrow$) Now let $PT \perp TS$ (Figure
\ref{sl.skl.3.3.6.pic}). For each point $T_1\in PT$  ($T_1 \neq T$)
the triangle $STT_1$ is a right triangle with the hypotenuse $ST_1$ and then
it holds (from \ref{vecstrveckotHipot}):
 $$ST_1 > ST = r.$$
 Therefore, none of the points $T_1$ ($T_1 \neq T$), lying on
the line $PT$, lies on the circle $k$. This means that the line
$PT$ is a tangent of this circle.
 \kdokaz

From the proof of the previous statement ($\Leftarrow$) it follows that all
points lying on the tangent of the circle (except for its
point of contact), are external points of this circle. With this
property we will prove the following statement.


\bzgled \label{tangKrozEnaStr}
All points of a circle are on the one side of its tangent.
\ezgled

\begin{figure}[!htb]
\centering
\input{sl.skl.3.3.7.pic}
\caption{} \label{sl.skl.3.3.7.pic}
\end{figure}

\textbf{\textit{Proof.}} Let $T$ be the point of tangency of the circle $k(S, r)$ and its tangent $t$ (Figure \ref{sl.skl.3.3.7.pic}). The tangent $t$ divides the plane in which $k$ and $t$ lie into two half-planes. The half-plane in which the point $S$ lies, we denote by $\alpha_1$, the other half-plane by $\alpha_2$. We prove that all points of the circle $k$ lie in the half-plane $\alpha_1$. Let $X$ be an arbitrary point of the half-plane $\alpha_2$. Since the points $S$ and $X$ are on different sides of the line $t$, it follows that the open line segment $SX$ intersects at some point $Y$. Then we have:
 $$SX = SY + YX > SY \geq ST = r,$$
  which means that the point $X$ does not lie on the circle $k$ and is its external point. Therefore, none of the points of the half-plane $\alpha_2$ lies on the circle $k$, that is, all of them are in the half-plane $\alpha_1$ with the edge $t$. \kdokaz

A direct consequence of the statement \ref{TangPogoj} is also that in each point of the circle we can draw only one tangent. If $X$ is an internal point of the circle $k(S, r)$, then no tangent passes through this point, since all lines through $X$ are intersecting, which is a consequence of Dedekind's axiom (statement \ref{DedPoslKrozPrem}). Later (statement \ref{tangentiKroznice}) we will find that through each external point of the circle we can draw exactly two tangents. For the time being, we prove the following statement (the reader will remember that this is a statement that we have already considered at the beginning of the introductory chapter - statement \ref{TalesUvod}).

\bizrek \label{TalesovIzrKroz} \index{izrek!Talesov za krožnico}
            Thales' theorem for a circle\footnote{Starogrški
            filozof in matematik \textit{Tales}
            \index{Tales} iz Mileta (640--546 pr. n. š.)
             ni prvi, ki je odkril to trditev. Kot empirično dejstvo so jo poznali
             že stari Egipčani in Babilonci. Izrek imenujemo po Talesu, ki ga je prvi dokazal.
             V dokazu je uporabljal lastnosti enakokrakih trikotnikov in dejstvo, da je
             vsota notranjih kotov trikotnika enaka vsoti dveh pravih kotov. Torej je
             dokaz enak temu, ki ga bomo izpeljali tukaj.}:\\
            Let $AB$ be a diameter of a circle $k$. Then for any point $X$ of this circle different from $A$ and $B$
            ($X\in k$ in $X\neq A$ in $X\neq B$) is $\angle AXB=90^0$
            \index{izrek!Talesov za krožnico}
            \eizrek

\begin{figure}[!htb]
\centering
\input{sl.skl.3.3.8.pic}
\caption{} \label{sl.skl.3.3.8.pic}
\end{figure}

\textbf{\textit{Proof.}} Naj bo $O$ središče krožnice $k$ (Figure
\ref{sl.skl.3.3.8.pic}). Ker $A,B,X\in k$, je
 $OA\cong OB\cong OX$. Torej sta
 trikotnika $AOX$ in $BOX$  enakokraka, zato je (izrek
\ref{enakokraki}):
 $\angle AXO\cong\angle XAO=\alpha$ in $\angle BXO\cong\angle XBO=\beta$.
Tedaj je $\angle AXB=\alpha+\beta$.
 Vsota notranjih kotov v
trikotniku $AXB$ je enaka $180^0$ (izrek \ref{VsotKotTrik}), torej
je $2\alpha+2\beta=180^0$. Iz tega sledi
 $$\angle AXB=\alpha+\beta=90^0,$$ kar je bilo treba dokazati. \kdokaz

 Dokažimo tudi obratno trditev.


             \bizrek \label{TalesovIzrKrozObrat}
            If $A$, $B$ and $X$ are non-collinear points such that
            $\angle AXB=90^0$, then the point $X$ lies on a circle with the diameter $AB$.
             \eizrek

\begin{figure}[!htb]
\centering
\input{sl.skl.3.3.9.pic}
\caption{} \label{sl.skl.3.3.9.pic}
\end{figure}


\textbf{\textit{Proof.}} Let $O$ be the center of the line $AB$ and $k$
the circle with center $O$ and radius $OA$ or diameter $AB$
(Figure \ref{sl.skl.3.3.9.pic}). From $\angle AXB=90^0$ it follows from
the theorem \ref{VsotKotTrik}:
 \begin{eqnarray}
 \angle XAB+ \angle XBA = 90^0 \label{relacija336}
 \end{eqnarray}

We prove $X\in k$. We assume the contrary, i.e. that the point $X$ does not
lie on the circle $k$. In this case $OX\neq OA$. Let $X_1$
be the point on the segment $OX$, for which $OX_1\cong OA$ (theorem
\ref{ABnaPoltrakCX}). This means that the point $X_1$ lies on the circle
$k$ and by the theorem \ref{TalesovIzrKroz} we have $\angle
AX_1B=90^0$.

By our assumption $OX\neq OA$ it is clear that $X\neq X_1$.
We will consider two possibilities:

\textit{1)} Let $OX_1<OX$ or $\mathcal{B}(O,X_1,X)$. In this
case $X_1$ is an inner point of the angles $XAB$ and $XBA$, therefore $\angle X_1AB<\angle XAB$ and $\angle X_1BA<\angle XBA$. From this and
relation \ref{relacija336} it follows:
 $$\angle X_1AB+ \angle X_1BA<\angle XAB+ \angle XBA = 90^0.$$
 Since $\angle AX_1B=90^0$, the sum of the angles in the triangle $AX_1B$
 is less than $180^0$, which by the theorem \ref{VsotKotTrik} is not possible.


\textit{2)} Let $OX_1>OX$ or $\mathcal{B}(O,X,X_1)$. Similarly
as in the first case we get:
 $$\angle X_1AB+ \angle X_1BA>\angle XAB+ \angle XBA = 90^0.$$
 In this case the sum of the angles in the triangle $AX_1B$ is greater than
$180^0$, which by the theorem \ref{VsotKotTrik} is not possible.

It follows that $OX=OX_1$  or $X\in k$.
 \kdokaz


 Use the previous theorem for the design of tangents.


             \bzgled \label{tangKrozKonstr}
             Let $A$ be an exterior point of a circle $k(S,r)$.
             Construct all tangents of this circle passing through the point $A$.
             \ezgled

\begin{figure}[!htb]
\centering
\input{sl.skl.3.3.10.pic}
\caption{} \label{sl.skl.3.3.10.pic}
\end{figure}


 \textbf{\textit{Solution.}} Let $l$ be a circle with diameter $SA$
(Figure \ref{sl.skl.3.3.10.pic}). Because $S$ is an interior, $A$ is an exterior point of the given circle $k$, the circle $k$ and $l$ have exactly two common points $T_1$ and $T_2$ (statement \ref{DedPoslKrozKroz}).
By Tales' statement \ref{TalesovIzrKroz},
$\angle ST_1A\cong \angle ST_2A=90^0$. Because $ST_1$ and  $ST_2$
are radii of the circle $k$, $AT_1$ and $AT_2$ are tangents of the circle $k$
through the point $A$ (statement \ref{TangPogoj}).

 We prove that $AT_1$ and $AT_2$ are the only tangents of the circle $k$
  from the point $A$. If $AT$ is a tangent from the point $A$, which the circle $k$
 touches in the point $T$, by  statement \ref{TangPogoj} $\angle ATS=90^0$.
 This means that the point $T$ lies on the circle $l$
 (statement \ref{TalesovIzrKrozObrat}) or $T\in k\cap l$. Therefore $T$ is one
 of the points $T_1$ and $T_2$, so $AT_1$ and $AT_2$ are the only tangents of the circle
  $k$  from the point $A$.
  \kdokaz

  The following statement follows from the previous construction.



        \bizrek \label{tangentiKroznice}
         If $V$ is an exterior point of a circle $k(S,r)$, then there are exactly
       two tangents of the circle $k$ through the point $V$.
        \eizrek


 We prove some more properties of a tangent of a circle.


            \bzgled \label{TangOdsek}
            If $VA$ and $VB$ are tangents of a circle $k(S,r)$ in a points
            $A$ and $B$ of the circle, then the centre $S$ lies on the bisector
            of the angle $AVB$ and $VA \cong VB$.
             \ezgled

\begin{figure}[!htb]
\centering
\input{sl.skl.3.3.11.pic}
\caption{} \label{sl.skl.3.3.11.pic}
\end{figure}

\textbf{\textit{Proof.}} From izrek \ref{TangPogoj} it follows: $VA
\perp AS$ and $VB \perp BS$
(Figure \ref{sl.skl.3.3.11.pic}). Therefore $ASV$ and $BSV$ are right-angled
triangles with a shared hypotenuse $SV$. Because $SA \cong SB = r$,
these two triangles are congruent (izrek \textit{SSA} \ref{SSK}). Therefore the angles
$AVS$ and $BVS$ are also congruent, which means that the line $VS$
is the bisector of angle $AVB$. From the congruence of these two triangles it also follows
that $VA \cong VB$.
 \kdokaz

The converse is also true.


             \bzgled \label{SimKotaKraka}
             If a point $S$ lies on the bisector of a convex angle,
            then it is the centre of a circle touching both sides of this angle.
            \ezgled

\begin{figure}[!htb]
\centering
\input{sl.skl.3.3.12.pic}
\caption{} \label{sl.skl.3.3.12.pic}
\end{figure}

\textbf{\textit{Proof.}} Let $A$ and $B$ be the right-angled projections
of point $S$ on the sides of given angle with the vertex $V$ (Figure
\ref{sl.skl.3.3.12.pic}). Triangles $ASV$ and $BSV$ are congruent
(izrek \textit{ASA} \ref{KSK}), because they have a shared side $VS$ and
two pairs of congruent angles - from $\angle AVS\cong \angle BVS$ and $\angle
SAV\cong \angle SBV=90^0$ it follows that $\angle ASV\cong \angle BSV$. Therefore $SA\cong SB$ and $k(S,SA)$ is the desired circle. The sides
of given angle are tangents to the circle by izrek \ref{TangPogoj}.
 \kdokaz

Now we will prove another criterion for the mutual position of a line and
a circle in a plane.



        \bizrek \label{TangSekMimobKrit}
        Let $P$ be the foot of the perpendicular from the centre of a circle  $k(S,r)$
            on a line $p$ (lying in the plane of the circle). Then the line $p$ is:

        (i) secant, if and only if $SP < r$;

        (ii) tangent, if and only if  $SP \cong r$;

         (iii) non-intersecting line,  if and only if  $SP> r$.
        \eizrek

\begin{figure}[!htb]
\centering
\input{sl.skl.3.3.13.pic}
\caption{} \label{sl.skl.3.3.13.pic}
\end{figure}

\textbf{\textit{Proof.}} (Figure \ref{sl.skl.3.3.13.pic})

The statement (ii) follows directly from the criterion for tangency (the statement \ref{TangPogoj}).

(i) In the proof of the direct direction of equivalence we use the fact that the hypotenuse of a right angled triangle is longer than either of the two shorter sides (the statement \ref{vecstrveckotHipot}). If $A$ and $B$ are the intersections of the secant $p$ and the circle $k$, then $SA$ is the hypotenuse of the right angled triangle $ASP$ and it holds:
 $r \cong SA > SP$.

 In the proof of the inverse direction of equivalence we use the consequence of Dedekind's axiom (the statement \ref{DedPoslKrozPrem}). Because in this case $P$ is an inner point of this circle, each straight line of this plane that goes through
the point $P$ is a secant of the circle $k$.

(iii) It follows from the proven (i) and (ii). Because if $SP > r$,
then neither $SP < r$ nor $SP \cong r$. From the equivalences (i) and (ii) it follows that the straight line $p$ is neither a secant nor a tangent. Therefore $p$
is a parallel of the circle $k$. In the same way we prove the inverse direction of equivalence.
  \kdokaz

%________________________________________________________________________________
 \poglavje{Quadrilaterals} \label{odd3Stirik}


In section \ref{odd2AKSURJ} we introduced the concept of a quadrilateral as
a polygon with four sides and four vertices.
We defined the concepts of adjacent and opposite sides, adjacent and
opposite angles and diagonal. To a quadrilateral $ABCD$ with the lengths of its sides $AB$, $BC$, $CD$ and $DA$ we usually assign the letters $a$,
$b$, $c$ and $d$, and to the lengths of its diagonals $AC$ and $BD$ the letters $e$ and
$f$  (Figure \ref{sl.skl.3.4.1.pic}).

\begin{figure}[!htb]
\centering
\input{sl.skl.3.4.1.pic}
\caption{} \label{sl.skl.3.4.1.pic}
\end{figure}

In the same section, we introduced the concepts of internal and external angles of a quadrilateral. We also mentioned that the internal angles at vertices $A$, $B$, $C$ and $D$ of a quadrilateral $ABCD$ are usually denoted by $\alpha$, $\beta$, $\gamma$ and $\delta$, and its external angles by $\alpha'$, $\beta'$, $\gamma'$ and $\delta'$. We proved (as a consequence of the general statement \ref{VsotKotVeck}) that the sum of all four internal angles of an arbitrary quadrilateral is equal to $360^0$ (Figure \ref{sl.skl.3.4.1.pic}). The sum of external angles is also equal to $360^0$ (in a convex quadrilateral). So:
 \begin{eqnarray*}
 \alpha+\beta+\gamma+\delta=360^0,\\
 \alpha'+\beta'+\gamma'+\delta'=360^0
 \end{eqnarray*}

Let us also add that we call the internal angle \pojem{adjacent} or \pojem{opposite}, if the corresponding vertices are adjacent or opposite.



Now we will consider some types of quadrilaterals in more detail.

 A quadrilateral
$ABCD$ is a \pojem{trapezoid}, if $AB\parallel CD$ (Figure \ref{sl.skl.3.4.2.pic}).
Sides $AB$ and $CD$ are the \pojem{bases}, $BC$ and $AD$ are the
\pojem{legs} of this trapezoid.
The line $PQ$ ($P\in AB$, $Q\in CD$ and $PQ\perp AB$) is the \pojem{height} of the trapezoid. We often denote it by $v$.


A trapezoid is
 \pojem{isosceles},
 if $BC \cong AD$ and $BC \not\parallel AD$,
 or \pojem{right}, if at least
one of the internal angles is a right angle.




\begin{figure}[!htb]
\centering
\input{sl.skl.3.4.2.pic}
\caption{} \label{sl.skl.3.4.2.pic}
\end{figure}


Two internal angles at the same leg of a trapezoid are supplementary, because
they are angles with parallel legs (statement \ref{KotiTransverzala}). The supplementarity of these angles is also
a sufficient condition for a quadrilateral to be a trapezoid. It follows from this that a right trapezoid has at least two right internal angles.

We get another group of quadrilaterals as a special type of trapezoids. These are \pojem{parallelograms}. They can be defined in different ways. We will choose one, and prove that the others are equivalent.

Quadrilateral $ABCD$ is a \index{paralelogram} \pojem{parallelogram}, if $AB \parallel CD$ and $AD \parallel BC$ (Figure \ref{sl.skl.3.4.3.pic}). Line $PQ$ ($P\in AB$, $Q\in CD$ and $PQ\perp AB$) and $MN$ ($M\in BC$, $N\in AD$ and $MN\perp BC$) are the \index{višina!paralelograma}\pojem{heights of the parallelogram}. We often denote them with $v_a$ and $v_b$.



\begin{figure}[!htb]
\centering
\input{sl.skl.3.4.3.pic}
\caption{} \label{sl.skl.3.4.3.pic}
\end{figure}

 A parallelogram is therefore a quadrilateral that has two pairs of parallel sides. The term congruence is not used in the definition of a parallelogram. We can also consider parallelograms (and trapezoids) in so-called \index{geometrija!afina} \pojem{affine geometry}. This is a geometry that is based on all the axioms of Euclidean geometry, with the exception of the third group of axioms - the axioms of congruence.

We will now prove the aforementioned equivalents for the definition of a parallelogram.

\bizrek  \label{paralelogram}
Let $ABCD$ be a convex quadrilateral.
Then the following statements are equivalent:
\begin{enumerate}
  \item The quadrilateral $ABCD$ is a parallelogram.
  \item Any two adjacent interior angles of  the quadrilateral $ABCD$ are supplementary.
  \item Any two opposite interior angles of  the quadrilateral $ABCD$ are congruent.
 \item $AB \parallel CD$ and $AB \cong CD$\footnote{This equivalent
    in a slightly different form is given by \index{Euclid}
    \textit{Euclid of Alexandria} (3rd century BC) in
    the first book of his 'Elements'.}.
 \item $AB \cong CD$ and $AD \cong BC$.
 \item The diagonals of the quadrilateral $ABCD$ bisect each other, i.e.
   line segments $AC$ and $BD$ have a common midpoint.
\end{enumerate}
 \eizrek

 \textbf{\textit{Proof.}}
It is enough to prove the equivalence of all the statements $(1)-(6)$. To avoid proving all equivalences (two implications - for example, with the statement (1), which would give 10 implications in total), we will simplify the proof a little, so that we prove implications according to the following scheme.

\vspace*{5mm}
\hspace*{25mm}
$\begin{array}{ccccccc}
  \textit{(1)} & \Leftarrow & \textit{(2)} & \Leftarrow & \textit{(3)} &   &   \\
  \Downarrow &   &   &   & \Uparrow &   &   \\
  \textit{(4)} &   & \Rightarrow &   & \textit{(5)} & \Leftrightarrow & \textit{(6)}
\end{array}$

\vspace*{5mm}

 As we can see, this is enough because the implication $\textit{(1)}
\Rightarrow \textit{(2)}$ follows directly from: $\textit{(1)}\Rightarrow \textit{(4)}\Rightarrow
\textit{(5)}\Rightarrow \textit{(3)} \Rightarrow \textit{(2)}$.

Let's mark with $\alpha$, $\beta$, $\gamma$ and $\delta$ the internal angles at
the vertices $A$, $B$, $C$ and $D$ of the quadrilateral $ABCD$. The quadrilateral
$ABCD$ is convex, which means that its diagonals intersect in
some point $S$.

$\textit{(2)}\Rightarrow \textit{(1)}$. Let the angles $\alpha$ and
$\beta$ be complementary (Figure \ref{sl.skl.3.4.4.pic}). Then the angles at the transversal $AB$ are congruent to the angles $AD$ and $BC$, so $AD\parallel
BC$ (by Theorem \ref{KotiTransverzala}). Similarly, from the complementarity of the angles $\beta$ and $\gamma$ it follows that $AB\parallel CD$. Therefore, the quadrilateral $ABCD$ is a parallelogram.

\begin{figure}[!htb]
\centering
\input{sl.skl.3.4.4.pic}
\caption{} \label{sl.skl.3.4.4.pic}
\end{figure}

$\textit{(3)}\Rightarrow\textit{(2)}$. Let $\alpha =\gamma$ and $\beta =\delta$ (Figure \ref{sl.skl.3.4.4.pic}).
Since $\alpha + \beta +\gamma +\delta = 360°$ (the sum of all
internal angles in a quadrilateral is $360°$ - by Theorem \ref{VsotKotVeck}), it follows that
$\alpha + \beta =180°$ and $\beta +\gamma = 180°$.


\begin{figure}[!htb]
\centering
\input{sl.skl.3.4.4a.pic}
\caption{} \label{sl.skl.3.4.4a.pic}
\end{figure}

$\textit{(1)}\Rightarrow\textit{(4)}$. Let the quadrilateral $ABCD$
be a parallelogram, i.e. let $AB \parallel CD$ and $AD \parallel BC$
(Figure \ref{sl.skl.3.4.4a.pic}). We will prove that then also $AB \cong CD$.
The line $AC$ is a transversal of the parallels $AB$ and $CD$, which
means that the angles $CAB$ and $ACD$ are alternate angles at this
transversal and are therefore congruent. Similarly, from the parallelism of the lines
$AD$ and $BC$ it follows that the angles $ACB$ and $CAD$ are congruent. Since $AC \cong AC$, the triangles $ACB$ and $CAD$ are congruent (by Theorem \ref{KSK} - \textit{ASA}). Therefore,  $AB \cong CD$.

\begin{figure}[!htb]
\centering
\input{sl.skl.3.4.4b.pic}
\caption{} \label{sl.skl.3.4.4b.pic}
\end{figure}

 $\textit{(4)}\Rightarrow \textit{(5)}$. Let $ABCD$ be such a quadrilateral that
 $AB \parallel CD$ and $AB \cong CD$ (Figure \ref{sl.skl.3.4.4b.pic}).
 We will prove that then also $AD \cong BC$.
 The line $AC$ is a transversal of the parallels $AB$ and
$CD$, which means that the angles $CAB$ and $ACD$ are alternate angles at this
transversal and are therefore congruent. Since $AC \cong AC$, the triangles $ACB$ and $CAD$ are congruent (by Theorem \ref{SKS} - \textit{SAS}). Therefore, $BC \cong AD$.

\begin{figure}[!htb]
\centering
\input{sl.skl.3.4.4c.pic}
\caption{} \label{sl.skl.3.4.4c.pic}
\end{figure}

 $\textit{(5)}\Rightarrow \textit{(3)}$. Let $ABCD$ be a quadrilateral such that $AB \cong CD$ and
 $AD \cong BC$ (Figure \ref{sl.skl.3.4.4c.pic}). We prove that
then $\beta =\delta$ and $\alpha =\gamma$. Because $AC \cong
AC$, the triangles $ACB$ and $CAD$ are congruent (\ref{SSS} - \textit{SSS}).
It follows that $\angle ABC \cong \angle CDA$ or $\beta =\delta$. In a
similar way we prove that $\alpha =\gamma$.

\begin{figure}[!htb]
\centering
\input{sl.skl.3.4.4d.pic}
\caption{} \label{sl.skl.3.4.4d.pic}
\end{figure}

$\textit{(5)}\Leftrightarrow \textit{(6)}$. Let $ABCD$ be a quadrilateral such that
$AB \cong CD$ and $AD \cong BC$ (Figure \ref{sl.skl.3.4.4d.pic}). We prove that the point $S$
is the common center of its diagonals $AC$ and $BD$. Because $AC \cong
AC$, the triangle $ACB$ is congruent to the triangle $CAD$ (\ref{SSS} - \textit{SSS}). Therefore
$\angle ACB \cong \angle CAD$ or $\angle SCB \cong \angle
SAD$. From the congruence of the right angles $CSB$ and $ASD$, it follows that the
angles $SBC$ and $SDA$ are also congruent. Because $AD \cong BC$, it follows that
$\triangle CSB \cong \triangle ASD$ (\ref{KSK} - \textit{ASA}). Therefore $SB
\cong SD$ and $SC \cong SA$ or the point $S$ is the common center
of the diagonals $AC$ and $BD$.


\begin{figure}[!htb]
\centering
\input{sl.skl.3.4.4e.pic}
\caption{} \label{sl.skl.3.4.4e.pic}
\end{figure}


Conversely, let $S$ be the common center of the diagonals $AC$ and $BD$ (Figure \ref{sl.skl.3.4.4e.pic}). Then
$SB \cong SD$ and $SC \cong SA$. The right angles $CSB$ and $ASD$ are also congruent, so
$\triangle CSB \cong \triangle ASD$ (\ref{SKS} - \textit{SAS}). It follows that $AD \cong BC$. In a similar way
we prove that $AB \cong CD$.
 \kdokaz

We recommend to the reader to prove the previous statement by using a
similar scheme. This will be a good exercise in using theorems about the congruence of triangles.

Let us now define a type of quadrilateral for which it will
be shown that they are special cases of parallelograms (Figure
\ref{sl.skl.3.4.5.pic}).

\begin{figure}[!htb]
\centering
\input{sl.skl.3.4.5.pic}
\caption{} \label{sl.skl.3.4.5.pic}
\end{figure}

A quadrilateral with all sides congruent is called a \index{rhombus}
\pojem{rhombus}.

A quadrilateral with all interior angles congruent (and therefore
equal to $90^0$, because their sum is $360^0$) is a \index{rectangle}
\pojem{rectangle}.

A quadrilateral with all sides congruent and all interior angles
congruent (and equal to $90^0$) is called a \index{square}
\pojem{square}.


It is not difficult to prove that each of these quadrilaterals is also a parallelogram. This is a direct consequence of the previous statement. The rhombus is a parallelogram due to $\textit{(5)}\Rightarrow\textit{(1)}$; the rectangle is a parallelogram due to
$\textit{(2)}\Rightarrow\textit{(1)}$ (or $\textit{(3)}\Rightarrow\textit{(1)}$). For the square it is
clear that it is also a rectangle and a rhombus, so it is also
a parallelogram.

From the previous statement \ref{paralelogram} - equivalent (\textit{5}) it follows that a parallelogram with two adjacent sides congruent is a rhombus. Similarly,
 according to the same statement from the equivalents \textit{(2)} and \textit{(3)} it follows that a parallelogram with at least one right angle is a rectangle.

The next statement provides additional criteria when a parallelogram is also a rhombus,
a rectangle or a square. This statement refers to diagonals. In
a parallelogram, the diagonals always intersect, but in a rhombus,
a rectangle or a square we will have additional properties.



        \bizrek \label{RombPravKvadr} $ $  (Figure \ref{sl.skl.3.4.5a.pic})

        a) A parallelogram is a rhombus if and only if their diagonals are perpendicular.

         b) A parallelogram is a rectangle if and only if their diagonals are congruent.

        c) A parallelogram is a square if and only if their diagonals are perpendicular and congruent.
        \eizrek

\begin{figure}[!htb]
\centering
\input{sl.skl.3.4.5a.pic}
\caption{} \label{sl.skl.3.4.5a.pic}
\end{figure}

 \textbf{\textit{Proof.}}
 Let $ABCD$ be a parallelogram and $S$ the intersection of its diagonals $AC$ and $BD$. According to
the previous statement \ref{paralelogram}, $S$ is their common
center. From the same statement it also follows that $AB \cong CD$ and $AD \cong
BC$.

\begin{figure}[!htb]
\centering
\input{sl.skl.3.4.6.pic}
\caption{} \label{sl.skl.3.4.6.pic}
\end{figure}

\textit{a)}  (Figure \ref{sl.skl.3.4.6.pic})

 If $ABCD$ is a rhombus, all sides are congruent. Therefore the
triangles $ABS$ and
  $ADS$
are congruent (statement \ref{SSS} - \textit{SSS}). Then the angles $ASB$
and $ASD$ are also congruent and are (as the supplement of the angle) both right angles. This means that the diagonals are perpendicular.

If the diagonals of the parallelogram $ABCD$ are perpendicular, the triangles
$ABS$ and $ADS$ are congruent (statement \ref{SKS} - \textit{SAS}). Therefore the sides
$AB$ and $AD$ are congruent. In a similar way, we prove that all sides of this parallelogram are congruent, which means that the parallelogram is a rhombus.


\begin{figure}[!htb]
\centering
\input{sl.skl.3.4.6a.pic}
\caption{} \label{sl.skl.3.4.6a.pic}
\end{figure}


\textit{b)}  (Figure \ref{sl.skl.3.4.6a.pic})

 If $ABCD$ is a rectangle, all interior angles are congruent and are right angles. Then
the triangles $ABC$ and $DCB$ are congruent (statement \ref{SKS} - \textit{SAS}).
Therefore $AC \cong DB$.

If in the parallelogram $ABCD$ it holds that $AC \cong DB$, the triangles
$ABC$ and $DCB$ are congruent (statement \ref{SSS} - \textit{SSS}). From this it follows that
the interior angles at the vertices $B$ and $D$ of the parallelogram
$ABCD$ are congruent. According to the previous statement \ref{paralelogram} they are supplementary, which means that they are both right angles. Similarly, all angles of this parallelogram are right angles, which means that the parallelogram is a rectangle.

 \textit{c)} A parallelogram is a square if and only if it is a rhombus and a rectangle at the same time.
 The latter is equivalent to
the fact that the diagonals are perpendicular and
congruent, which follows from the proven  (\textit{a.} and \textit{b.}).
 \kdokaz


\begin{figure}[!htb]
\centering
\input{sl.skl.3.4.7.pic}
\caption{} \label{sl.skl.3.4.7.pic}
\end{figure}

Since the diagonals of a rectangle are perpendicular and intersect, there is a circle that contains all the vertices of this rectangle (Figure \ref{sl.skl.3.4.7.pic}). This is called the \index{circumscribed circle!rectangle} \pojem{circumscribed circle of the rectangle}. Its center is the intersection of its diagonals. If the point $S$ is the intersection of the diagonals of the rectangle $ABCD$, then from the previous equation \ref{RombPravKvadr} it follows:
 $$SA \cong SC \cong SB \cong SD.$$
The radius of this circle is equal to half the diagonal of the rectangle. Since the square is a special type of rectangle, it also has a circumscribed circle.

We will now prove an important property of isosceles trapezoids.


     \bizrek \label{trapezEnakokraki}
     Interior base angles of an isosceles trapezium are congruent.
     The diagonals of an isosceles trapezium are congruent line segments.
     \eizrek


\begin{figure}[!htb]
\centering
\input{sl.skl.3.4.8.pic}
\caption{} \label{sl.skl.3.4.8.pic}
\end{figure}

  \textbf{\textit{Proof.}}
  Let $ABCD$ be an isosceles trapezoid with the base $AB$ (Figure \ref{sl.skl.3.4.8.pic}). Without loss of generality, assume that $AB>CD$. Let $C'$ and $D'$ be the orthogonal projections of the vertices $C$ and $D$ onto the line $AB$. The quadrilateral $D'C'CD$ is a parallelogram with a right angle, so it is a rectangle. From the fact that $D'C'CD$ is a parallelogram, it follows that $CC'\cong DD'$ (equation \ref{paralelogram}). Since $\angle CC'B\cong \angle DD'A=90^0$, the triangles $CC'B$ and $DD'A$ are congruent (the \textit{SSA} equation \ref{SSK}), so $\beta=\angle CBC'\cong \angle DAD'=\alpha$.

  We will now prove that the diagonals $AC$ and $BD$ are congruent. This follows from the congruence of the triangles $ABC$ and $BAD$ (the \textit{SAS} equation \ref{SKS}).
  \kdokaz


            \bzgled
            Let $ABCD$, $AEBK$ and $CEFG$ be equally oriented squares in a plane.
            Then $B$, $D$ and $F$ are collinear points and the point $B$ is a midpoint of the line segment $DF$.
             \ezgled

\begin{figure}[!htb]
\centering
\input{sl.skl.3.4.9.pic}
\caption{} \label{sl.skl.3.4.9.pic}
\end{figure}

 \textbf{\textit{Proof.}} (Figure \ref{sl.skl.3.4.9.pic})

 The triangles $CAE$ and $FBE$ are congruent, because $CE\cong FE$, $AE\cong BE$  and
 $\angle AEC=90^0-\angle CEB=\angle BEF$ (statement \ref{SKS} - \textit{SAS}). Therefore $\angle EBF$ is a right angle and the points $D$, $B$ and $F$ are collinear. From
 the congruence of these two triangles it also follows that $BF\cong AC\cong BD$.
  \kdokaz


\begin{figure}[!htb]
\centering
\input{sl.skl.3.4.10.pic}
\caption{} \label{sl.skl.3.4.10.pic}
\end{figure}

Except for trapezoids and parallelograms, we will define one more group
of quadrilaterals. A quadrilateral $ABCD$ is
\index{deltoid}\pojem{deltoid}, if its diagonals are perpendicular and
one of the diagonals bisects the other (Figure \ref{sl.skl.3.4.10.pic}).
The following statement is related to the deltoid and is equivalent to its
definition.


        \bzgled
        A quadrilateral is a deltoid if and only if it has two pairs of congruent adjacent sides.
        \ezgled

\begin{figure}[!htb]
\centering
\input{sl.skl.3.4.11.pic}
\caption{} \label{sl.skl.3.4.11.pic}
\end{figure}

\textbf{\textit{Proof.}} (Figure \ref{sl.skl.3.4.11.pic})

 Let the quadrilateral $ABCD$ be a deltoid. Then the diagonals
$AC$ and $BD$ are perpendicular and one of the diagonals bisects the other. Without
loss of generality, let the diagonal $BD$ bisect the diagonal $AC$.
It follows that the right triangles $ABS$ and $CBS$ are congruent (statement
\ref{SKS} - \textit{SAS}). Then $AB \cong CB$. From the congruence of the triangles $ADS$ and $CDS$ it follows that $AD \cong CD$.

 Let $ABCD$ be a quadrilateral, in which $AB \cong CB$ and $AD \cong CD$ hold.
 The triangles $ABD$ and $CBD$ are
congruent (statement \ref{SSS} - \textit{SSS}), so the angles $ADS$
and $CDS$ are also congruent. From this it follows that the triangles $ADS$ and
$CDS$ are also congruent (statement \ref{SKS} - \textit{SAS}). Therefore $S$ is the center of the diagonal $AC$, and the angles $DSA$
and $DSC$ are right angles, because they are congruent with the angles.
 \kdokaz

\bzgled
            Let $k_1(S_1,r)$, $k_2(S_2,r)$, $k_3(S_3,r)$ be congruent circles and
            $k_1\cap k_2=\{B,A_3\}$, $k_2\cap k_3=\{B,A_1\}$, $k_3\cap k_1=\{B,A_2\}$.
            Prove that the lines $S_1A_1$,
             $S_2A_2$ and $S_3A_3$ intersect at a single point .
            \ezgled

\begin{figure}[!htb]
\centering
\input{sl.skk.4.2.12.pic}
\caption{} \label{sl.skk.4.2.12.pic}
\end{figure}

\textbf{\textit{Proof.}}
 We will prove that the lines $O_1A_1$, $O_2A_2$ and $O_3A_3$
 have the same center, or that the corresponding quadrilaterals
are parallelograms (Figure \ref{sl.skk.4.2.12.pic}). Because $k_1$, $k_2$
and $k_3$ are congruent circles, the quadrilaterals $O_1A_2O_3B$ and $O_2A_1O_3B$
 are rhombuses. Because of this, the lines $O_1A_2$ and $O_2A_1$ are congruent and
parallel, which means that the quadrilateral $O_1A_2A_1O_2$ is a parallelogram
(statement \ref{paralelogram}). From the same statement it follows that its
diagonals $O_1A_1$ and $O_2A_2$ have a common center. In a similar way we prove that the lines $O_2A_2$ and $O_3A_3$ have a common center, which means that this is also true for all three lines $O_1A_1$,
$O_2A_2$ and $O_3A_3$ at the same time. \kdokaz


            \bzgled
            Construct a rectangle $ABCD$ if its diagonals and the difference of its sides
             are congruent with the two given line segments $d$ and $l$.
            \ezgled


\begin{figure}[!htb]
\centering
\input{sl.skl.3.4.10a.pic}
\caption{} \label{sl.skl.3.4.10a.pic}
\end{figure}

\textbf{\textit{Solution.}} Let $ABCD$ be a rectangle, where $AC\cong d$ and $AB-BC=l$ (Figure \ref{sl.skl.3.4.10a.pic}). Let $E$ be a point on side $AB$ such that $EB\cong BC$. In this case, $AE=AC-EB=AC-BC=l$. Because $EBC$ is an isosceles right triangle, $\angle CEB\cong\angle ECB=45^0$ (\ref{enakokraki} and \ref{VsotKotTrik}) or $\angle AEC=135^0$.
This allows us to first construct the triangle $AEC$ ($AC\cong d$, $\angle AEC=135^0$ and $AE\cong l$), and then the rectangle $ABCD$.
 \kdokaz


        \bzgled
        Construct a triangle with given $b$, $c$ and $t_a$ (sides $AC$, $AB$ and triangle median $AA_1$).
        \ezgled



\begin{figure}[!htb]
\centering
\input{sl.skl.3.4.10b.pic}
\caption{} \label{sl.skl.3.4.10b.pic}
\end{figure}

\textbf{\textit{Solution.}} Let $ABC$ be a triangle, where $AC\cong b$, $AB\cong c$ and $AA_1\cong t_a$, where $A_1$ is the center of line segment $BC$ (Figure \ref{sl.skl.3.4.10b.pic}). Let $D$ be a point on line segment $AA_1$ such that $DA_1\cong AA_1$ and $\mathcal{B}(A, A_1,D)$. This means that $A_1$ is the common center of line segments $BC$ and $AD$, so by \ref{paralelogram} quadrilateral $ABDC$ is a parallelogram. By the same \ref{paralelogram}, $CD\cong AB\cong c$. This allows us to first construct the triangle $ADC$ ($AC\cong b$, $CD\cong c$ and $AD=2t_a$), and then point $A$.
 \kdokaz

We will now introduce a shorter form of data entry for designing triangles. Similarly, as we had in the previous task with the notation:  $b$, $c$, $t_a$, for the elements of the triangle $ABC$ we will usually use the following labels:
 \begin{itemize}
   \item $a$, $b$, $c$ - sides,
   \item $\alpha$, $\beta$, $\gamma$ - internal angles,
   \item $v_a$, $v_b$, $v_c$ - altitudes,
   \item $t_a$, $t_b$, $t_c$ - centroids,
   \item $l_a$, $l_b$, $l_c$ - distances that are determined by the vertex and the intersection of the internal angle's symmetry line at that vertex with the opposite side;
   \item $s$ - semi-perimeter  ($s=\frac{a+b+c}{2}$),
   \item $R$ - radius of the circumscribed circle (see section \ref{odd3ZnamTock}),
   \item $r$ - radius of the inscribed circle (see section \ref{odd3ZnamTock}),
   \item  $r_a$, $r_b$, $r_c$ - radii of the excircles (see section \ref{odd4Pricrt}).
 \end{itemize}


        \bzgled
        Construct a trapezium if its sides are congruent with the four given line segments $a$, $b$, $c$ and $d$.
        \ezgled


\begin{figure}[!htb]
\centering
\input{sl.skl.3.4.10c.pic}
\caption{} \label{sl.skl.3.4.10c.pic}
\end{figure}

\textbf{\textit{Solution.}} Without loss of generality, we will first assume that $a\geq c$. Let $ABCD$ be a trapezium, in which sides $AB\cong a$, $BC\cong b$, $CD\cong c$ and $DA\cong d$ (Figure \ref{sl.skl.3.4.10c.pic}). In this case, $AB\geq CD$, so on the side $AB$ there exists a point $E$, such that $AE\cong CD$. Because $AB\parallel CD$, by \ref{paralelogram} the quadrilateral $AECD$ is a parallelogram, so by the same theorem $CE\cong DA\cong d$. It also holds that $EB=AB-AE=AB-CD=a-c$. This allows us to construct the triangle $EBC$ ($EB=a-c$, $BC\cong c$ and $CE\cong d$), and then the vertices $A$ and $D$ (from the condition $AE\cong CD\cong c$).
 \kdokaz



%________________________________________________________________________________
 \poglavje{Regular Polygons}\label{odd3PravilniVeck}

The concept of a square fits into the general definition of a new type of polygons.
 A polygon is
\index{pravilni!večkotniki} \pojem{regular}, if all
of its sides are congruent and all of its interior angles are congruent (Figure
\ref{sl.skl.3.5.1.pic}).

\begin{figure}[!htb]
\centering
\input{sl.skl.3.5.1.pic}
\caption{} \label{sl.skl.3.5.1.pic}
\end{figure}

 A square is therefore a regular quadrilateral. An equilateral triangle
 \index{trikotnik!pravilni}\pojem{regular triangle}
is also a regular polygon.
This is due to the fact that, in an equilateral triangle,
all angles are also equal.

We have already established that the sum of the interior angles of any
$n$-gon is equal to $(n - 2) \cdot 180^0$ (Theorem \ref{VsotKotVeck}). Because
in a regular $n$-gon, all interior angles are congruent, we can calculate the interior angle
by dividing the sum of all angles by the number $n$. Thus, we have
proved the following theorem (Figure \ref{sl.skl.3.5.2.pic}).


             \bizrek \label{pravVeckNotrKot}
             The measure of each interior angle of a regular $n$-gon is:
            $$\frac{(n - 2)\cdot 180^0}{n}.$$
            \eizrek

 Thus, the interior angle of a regular triangle measures $60^0$, the interior angle of a regular quadrilateral measures $90^0$, the interior angle of a regular pentagon measures $108^0$, the interior angle of a regular hexagon measures $120^0$, ...


\begin{figure}[!htb]
\centering
\input{sl.skl.3.5.2.pic}
\caption{} \label{sl.skl.3.5.2.pic}
\end{figure}

We shall now prove two important properties of regular polygons.


        \bizrek \label{sredOcrtaneKrozVeck}
        For each regular polygon, there exists a circle passing through each of its vertices.
        \eizrek

\textbf{\textit{Proof.}} Let $A_1A_2\ldots A_n$ be a regular
$n$-gon (Figure \ref{sl.skl.3.5.2.pic}). Then all sides are
congruent and all internal angles are congruent and equal to $\frac{(n
- 2)\cdot 180^0}{n}$. Let $s_1$ and $s_2$ be the lines of symmetry of sides
$A_1A_2$ and $A_2A_3$ of this polygon and let $S$ be their
intersection point. From  \ref{simetrala} it follows that $SA_1 \cong SA_2$
and $SA_2 \cong SA_3$  or:
 $$SA_1 \cong SA_2 \cong SA_2 \cong SA_3.$$
 It follows that the triangle $A_1SA_2$ and $A_2SA_3$
are congruent (from \ref{SSS} - \textit{SSS}). Then the angles
$SA_1A_2$, $SA_2A_1$, $SA_2A_3$ and $SA_3A_2$ are congruent. From $\angle SA_2A_1 \cong
\angle SA_2A_3$  it follows that both angles are equal to half of the internal
angle of the polygon or $\frac{\alpha}{2}=\frac{(n-2)\cdot
180^0}{2n}$. Therefore:
 $$\angle SA_3A_4 =\alpha - \frac{\alpha}{2}=\frac{\alpha}{2} = \angle
 SA_3A_2.$$
Thus the triangle $A_2SA_3$ and $A_3SA_4$ are congruent (from \ref{SKS} - \textit{SAS}). Because of this,  $SA_3 \cong SA_4$  or:
 $$SA_1 \cong SA_2 \cong SA_2 \cong SA_3\cong SA_4.$$
 If we continue this process, we get:
 $$SA_1 \cong SA_2 \cong SA_2 \cdots \cong SA_n,$$
which means that the point $S$ is the center of the circle $k(S, SA_1)$, which
contains all its vertices.
 \kdokaz

The circle from the previous theorem is called the \index{circumscribed circle!regular polygon} \pojem{circumscribed circle
of a regular polygon}. From the proof of the previous theorem it is clear that
its center lies at the intersection of the lines of symmetry of all its sides.

We prove the following theorem in an analogous way.


        \bizrek \label{sredVcrtaneKrozVeck}
        For each regular polygon, there exists a circle  touching each of its sides.
         \eizrek


\begin{figure}[!htb]
\centering
\input{sl.skl.3.5.3.pic}
\caption{} \label{sl.skl.3.5.3.pic}
\end{figure}

\textbf{\textit{Proof.}} Let $A_1A_2\ldots A_n$ be a regular $n$-gon
 (Figure \ref{sl.skl.3.5.3.pic}).
Define the point $S$ as in the proof of the previous statement.
We have shown that:  $SA_1 \cong SA_2 \cong SA_2 \cdots \cong
SA_n$. From this, by statement \ref{SSS} - \textit{SSS}, it follows that the congruence of the isosceles triangles:
 $$\triangle A_1SA_2 \cong \triangle A_2 SA_3 \cong \cdots \cong
  \triangle A_{n-1}SA_n \cong \triangle A_nSA_1.$$
Because of this, all angles at the bases of these triangles are also congruent.
Therefore, the lines $SA_1$, $SA_2$,..., $SA_n$ are the altitudes of the polygon $A_1A_2\ldots A_n$. Let $P_1$, $P_2$,...,
$P_n$ be the points of intersection of these altitudes with the polygon $A_1A_2\ldots A_n$.
From the congruence of the triangles $\triangle A_1SP_1$, $\triangle A_2SP_1$,
$\triangle A_2SP_2$, ..., $\triangle A_1SP_n$ (statements \ref{SSK} and
\ref{KSK}), it follows that the segments $SP_1$, $SP_2$,..., $SP_n$ are congruent. By statement \ref{TangPogoj}, the circle $k(S, SP_1)$ touches all sides
of the polygon $A_1A_2\ldots A_n$.
 \kdokaz

The circle from the previous statement is called the \index{circumscribed circle!of a regular polygon} \pojem{circumscribed circle of a regular
polygon}. From the proof of this statement, it is clear that the center of the circumscribed circle of a regular polygon lies at the intersection of the altitudes of all its internal angles. From the proof it is also clear that the points at which this circle touches the sides of the regular polygon are also the centers of these sides. The center of the circumscribed and inscribed circle is the same point and is therefore also called the \index{center!of a regular polygon}\pojem{center of a regular polygon}.

We have also seen that all triangles,
determined by the center
of the regular $n$-gon and by its sides, are isosceles and all
congruent. The radius of the circumscribed and inscribed circle of the $n$-gon is equal
to the leg or. the height of each of these triangles. The angles at the top of these
triangles are also congruent and because there are a total of $n$ (the same as
the sides of the $n$-gon), each of them measures (Figure \ref{sl.skl.3.5.4.pic}):
 $$\varphi = \frac{360^0}{n}.$$

\begin{figure}[!htb]
\centering
\input{sl.skl.3.5.4.pic}
\caption{} \label{sl.skl.3.5.4.pic}
\end{figure}

 For a regular hexagon, or for $n = 6$, it holds:
 $$\varphi = \frac{360^0}{6}=60^0.$$

 This means that the aforementioned triangles are regular.
Therefore, a regular hexagon is composed of six
regular triangles  (Figure \ref{sl.skl.3.5.4.pic}).

In the following we will consider the properties of regular $n$-gons.

 Let $n$ be an even number and $k = \frac{n}{2}+1$.
We say that $A_k$ is the \pojem{opposite vertex} of vertex $A_1$
of the regular $n$-gon $A_1A_2\ldots A_n$ (Figure
\ref{sl.skl.3.5.5.pic}). Analogously, $A_2$ and $A_{k+1}$, $A_3$ and
$A_{k+2}$, ... , $A_{k-1}$ and $A_n$ are opposite vertices of this
$n$-gon. Similarly, the sides $A_1A_2$ and $A_kA_{k+1}$, ... ,
$A_{k-1} A_k$ and $A_nA_1$ of the polygon
$A_1A_2\ldots A_n$ are \pojem{opposite sides}. We notice that it holds:
 $$\angle A_1SA_k=\frac{n}{2}\varphi = \frac{n}{2}\cdot
  \frac{360^0}{n}=180^0,$$
which means that the diagonal $A_1A_k$ of this $n$-gon contains its
center. Therefore, this diagonal represents the diameter of the circumscribed circle.
Analogously, this holds for all diagonals determined by opposite
vertices. Because of this, we call such diagonals \index{velika diagonala
pravilnega $n$-kotnika} \pojem{major diagonals} of a regular
$n$-gon. The radius of the circumscribed circle is
 equal to half the major diagonal.

\begin{figure}[!htb]
\centering
\input{sl.skl.3.5.5.pic}
\caption{} \label{sl.skl.3.5.5.pic}
\end{figure}

It can be proven in a similar way that for an even number $n$ of centers of the opposite sides of a regular $n$-gon
$A_1A_2\ldots A_n$ determine the diameters of the inscribed circle of this
$n$-gon. If we consider the previous labels, we get:
 \begin{eqnarray*}
 \angle P_1SP_k&=&\angle P_1SA_2 + \angle A_2SA_{k-1} + \angle
 A_{k-1}SP_k\\
 &=&  \frac{\varphi}{2}+\frac{n-2}{2}\cdot \varphi+\frac{\varphi}{2}
 =\frac{n}{2}\cdot\varphi
 =180^0.
  \end{eqnarray*}


 The distances that are determined by a pair of centers of opposite sides of an $n$-gon
 $A_1A_2\ldots A_n$ or the distances $P_1P_k$,
$P_2P_{k+1}$, ... , $P_{k-1}P_n$, are called the \pojem{heights}
\index{height!of a regular $n$-gon} of this $n$-gon. The radius of the inscribed circle is equal to half the height.

 So every regular $n$-gon, where $n$ is an even number, contains
$\frac{n}{2}$ big diagonals (equal to the diameter of the circumscribed circle) and
$\frac{n}{2}$ heights (equal to the diameter of the inscribed circle). Each of them
goes through the center of this $n$-gon.


 Let $n$ be an odd number now (Figure \ref{sl.skl.3.5.6.pic}) and
  $k=\frac{n+1}{2}+1$.
 Then:
\begin{eqnarray*}
 \angle P_1SA_k=\angle P_1SA_2 + \angle A_2SA_k=
  \frac{\varphi}{2}+\frac{n-1}{2}\cdot \varphi
 =\frac{n}{2}\cdot\varphi
 =180^0.
  \end{eqnarray*}

\begin{figure}[!htb]
\centering
\input{sl.skl.3.5.6.pic}
\caption{} \label{sl.skl.3.5.6.pic}
\end{figure}

This means that the distance $P_1A_k$ contains the center $S$ of a regular
$n$-gon  $A_1A_2\ldots A_n$. This distance is called the \index{height!of a regular $n$-gon}
\pojem{height} of this $n$-gon, the side
$A_1A_2$ and the point $A_k$ are \pojem{opposite}. We define the remaining $n$ heights and $n$ pairs of opposite sides and points in a similar way. In a similar way we can prove that the other heights of this $n$-gon also contain its center.

In a right (isosceles) triangle, we therefore have three
heights, which intersect in its center (Figure
\ref{sl.skl.3.5.7.pic}). If this is true for
any triangle, we will find out later


\begin{figure}[!htb]
\centering
\input{sl.skl.3.5.7.pic}
\caption{} \label{sl.skl.3.5.7.pic}
\end{figure}

 In a square, its  diagonals are also the main
 diagonals and intersect in
its center  (Figure \ref{sl.skl.3.5.7.pic}). The heights
of the square are consistent with its side, which is not difficult to prove.

We have already mentioned that a regular hexagon is composed of six
triangles, which intersect in its center. We will consider a regular
hexagon in more detail later. We will also prove
some properties of a regular pentagon, heptagon, nonagon and dodecagon. The problem
of designing regular $n$-gons for any number $n$ will be particularly interesting.

We will now prove an interesting property of a regular nonagon.

             \bzgled
            If $a$ is a side and $d$ and $e$ are the shortest and longest
            diagonal of a regular nonagon ($9$-gon), then $e - d = a$.
             \ezgled


\begin{figure}[!htb]
\centering
\input{sl.skl.3.5.8.pic}
\caption{} \label{sl.skl.3.5.8.pic}
\end{figure}

\textbf{\textit{Proof.}} Let $d = CE$ and $e = BF$ be the shortest and longest
diagonals of a regular nonagon $ABCDEFGHI$ with side
$a$ and $P$ the intersection of lines $BC$ and $FE$  (Figure
\ref{sl.skl.3.5.8.pic}). The internal angle of this nonagon measures
 $\angle CDE=\frac{9-2}{9}\cdot 180^0=140^0$,
so $\angle ECD = \angle CED = 20^0$. From this it follows
 $\angle BCE = \angle FEC = 120^0$, i.e.:
 $$\angle ECP = \angle CEP = 60^0.$$
 Therefore, triangle $CPE$ is regular. Since $CB = EF=a$
and $\angle BPF \cong \angle CPE = 60^0$, triangle $BPF$ is also regular. Therefore:
 $$e = BF = BP = BC + CP = BC + CE = a + d,$$ which was to be proved. \kdokaz


%%________________________________________________________________________________
 \poglavje{Midsegment of Triangle} \label{odd3SrednTrik}

We will now look at a very important property of a triangle, which we will use often. Let $P$ and $Q$ be the centers of the sides $AB$ and $AC$ of the triangle $ABC$. The distance $PQ$ is called the \index{midsegment!of a triangle} \pojem{midsegment of the triangle} $ABC$, which corresponds to the side $BC$ (Figure \ref{sl.skl.3.6.1.pic}). We also say that $PQ$ is the midsegment of the triangle $ABC$ with the base $BC$. We prove the basic property that relates to the midsegment.


\begin{figure}[!htb]
\centering
\input{sl.skl.3.6.1.pic}
\caption{} \label{sl.skl.3.6.1.pic}
\end{figure}



           \bizrek \label{srednjicaTrik}
             Let $PQ$ be the midsegment of a triangle $ABC$ corresponding to the side $BC$. Then:
            $$ PQ = \frac{1}{2} BC\hspace*{2mm}
            \textrm{ in } \hspace*{2mm} PQ \parallel BC.$$
           \eizrek



 \textbf{\textit{Proof.}} Let $R$ be a point such that $PQ \cong QR$ and $\mathcal{B}(P,Q,R)$ (Figure \ref{sl.skl.3.6.1.pic}). The line segments $AC$ and $PR$ have a common center, so the quadrilateral $APCR$ is a parallelogram (\ref{paralelogram}). Therefore, the line segments $AP$ and $RC$ are congruent and parallel. The point $P$ is the center of the line segment $AB$, so the line segments $PB$ and $RC$ are congruent and parallel. This means that the quadrilateral $PBCR$ is a parallelogram. It follows that the line segments $BC$ and $PR$ are congruent and parallel. The final conclusion follows from the fact that the point $Q$ is the center of the line segment $PR$.
 \kdokaz


            \bzgled
            Let $AB$ and $A'B'$ be congruent line segments, $C$ and $D$ the midpoints of the line segments
            $AA'$ and $BB'$. Suppose that $CD =\frac{1}{2}  AB$.
            What is a measure of the angle between the lines $AB$ and $A'B'$?
             \ezgled

\textbf{\textit{Solution.}} Let point $S$ be the center of line
$A'B$ (Figure \ref{sl.skl.3.6.2.pic}). Lines $CS$ and $DS$ are the
medians of triangles $A'AB$ and $BA'B'$, so: $$CS = \frac{1}{2}AB =
CD = \frac{1}{2}A'B'= DS,$$ or $SCD$ is an equilateral triangle.
Angles $\angle AB,A'B'$ and $\angle CSD$ have corresponding sides.
Therefore: $\angle AB, A'B' \cong \angle CSD = 60^0$.
 \kdokaz

The next consequence of \ref{srednjicaTrik} applies to a trapezium.


             \bizrek \label{srednjTrapez}
             Let $P$ and $Q$ be the midpoints of legs $BC$ and $DA$  of a trapezium $ABCD$.
            Suppose that $M$ and $N$ are the midpoints of the diagonals $AC$ and $BD$ of that trapezium.
            Then the points $M$ and $N$ lie on the line $PQ$, which is parallel to the bases $AB$ in $CD$ of the trapezium,
             and also:
             $$PQ = \frac{1}{2}( AB + CD)     \hspace*{2mm}
             \textrm{ in } \hspace*{2mm} MN=\frac{1}{2}( AB - CD).$$
            \eizrek

\begin{figure}[!htb]
\centering
\input{sl.skl.3.6.3.pic}
\caption{} \label{sl.skl.3.6.3.pic}
\end{figure}

\textbf{\textit{Proof.}}  The lines $PN$, $NQ$ and $PM$ are in turn medians of
the triangles
$DAC$, $ACB$ and $ADB$ for
the corresponding bases $DC$, $AB$ and $AB$ (Figure
\ref{sl.skl.3.6.3.pic}). Because of this, all three lines $PN$, $NQ$
and $PM$ are parallel to the bases $CD$ and $AB$. Since through each point (first $N$, then $P$) there is only one parallel to the line
$AB$ (Playfair's\footnote{\index{Playfair, J.}\textit{J.
Playfair} (1748--1819), Scottish mathematician.} axiom
\ref{Playfair}), the points $P$, $N$, $M$ and $Q$ are collinear. It also holds (from the statement \ref{srednjicaTrik}):
 $PN = \frac{1}{2} CD$ and
  $NQ = PM = \frac{1}{2} AB$. From this it follows:
   \begin{eqnarray*}
   PQ&=& PN+NQ=\frac{1}{2}CD+
   \frac{1}{2}AB=\frac{1}{2}\left(AB+CD\right)\\
  NM&=& PM-PN=\frac{1}{2}AB-
   \frac{1}{2}CD=\frac{1}{2}\left(AB-CD\right),
  \end{eqnarray*}
  which was to be proven.  \kdokaz

The line $PQ$ from the previous statement is called
\index{srednjica!trapeza} \pojem{median of the trapezoid}.

 The following statements apply to an arbitrary quadrilateral.


             \bizrek \label{Varignon}
             Let $ABCD$ be an arbitrary quadrilateral and $P$, $Q$, $K$ and $L$
            the midpoints of the sides $AB$, $CD$, $BC$ and $AD$, respectively. Then the quadrilateral $PKQL$ is
            a parallelogram (so-called \index{paralelogram!Varignonov}
              Varignon\footnote{\index{Varignon, P.}
              \textit{P. Varignon} (1654--1722),
             French mathematician,
            who was the first to prove this property. However, the statement was not published until
            after his death in 1731. Given the simplicity, it is
            quite surprising that this statement "waited" so long to be
            discovered.} parallelogram).
            \eizrek

\begin{figure}[!htb]
\centering
\input{sl.skl.3.6.4.pic}
\caption{} \label{sl.skl.3.6.4.pic}
\end{figure}

\textbf{\textit{Proof.}} (Figure \ref{sl.skl.3.6.4.pic})
The lines $PK$ and $LQ$ are the medians of the triangles $ABC$ and $ADC$ for the same base $AC$, so they are congruent and
parallel. Therefore, the quadrilateral $PKQL$ is a parallelogram.
 \kdokaz

In a special case, Varignon's parallelogram can even be a rectangle,
rhombus or square. When is this possible? This question gives us an idea for
the next statement. We know that a parallelogram is a rectangle if it has
at least one internal  angle right. The other (equivalent) condition  is that it has congruent diagonals. A similar treatment can be used for rhombus and square.


            \bzgled \label{VarignonPoslPravRomb}
            Let $ABCD$ be an arbitrary quadrilateral and $P$, $K$, $Q$ and $L$
            the midpoints of the sides $AB$, $BC$, $CD$ and
            $DA$, respectively. Then:

            a) $AC \perp BD \Leftrightarrow PQ \cong KL$;

             b) $AC \cong BD \Leftrightarrow PQ \perp KL$.
             \ezgled


\begin{figure}[!htb]
\centering
\input{sl.skl.3.6.5.pic}
\caption{} \label{sl.skl.3.6.5.pic}
\end{figure}


\textbf{\textit{Proof.}} (Figure \ref{sl.skl.3.6.5.pic})
 From the previous statement \ref{Varignon} it follows that
 the quadrilateral $PKQL$ is always a parallelogram – Varignon's
parallelogram. The lines $PL$ and $PK$ are the medians of the triangles $ABD$
and $ABC$ for the bases $AD$ and $AC$. Therefore, we have:
 $PL= \frac{1}{2}BD$ and $PL \parallel BD$ and $PK= \frac{1}{2}AC$ and $PK \parallel AC$.
 Therefore, we have:

 a) $AC \perp BD \Leftrightarrow PL \perp PK
\Leftrightarrow PKQL \textrm{ rectangle} \Leftrightarrow PQ
\cong KL$;

 b) $AC \cong BD \Leftrightarrow PL \cong PK \Leftrightarrow
  PKQL \textrm{ rhombus } \Leftrightarrow PQ \perp KL$.
 \kdokaz

If Varignon's parallelogram is a square, all four
conditions from the previous equivalences are fulfilled, or in this case: $AB \perp CD$, $AB \cong CD$, $PQ \perp KL$ and $PQ \cong KL$.

Let's look at one more use of the property of Varignon's parallelogram.

\bzgled \label{VagnanPosl}
Let $ABCD$ be an arbitrary quadrilateral. If $P$, $Q$, $K$, $L$, $M$ and
$N$ are the midpoints of the line segments $AB$, $CD$, $BC$, $AD$, $AC$ and $BD$,  respectively,
then the line segments $PQ$, $KL$ and
$MN$  have a common midpoint.
\ezgled


\begin{figure}[!htb]
\centering
\input{sl.skl.3.6.6.pic}
\caption{} \label{sl.skl.3.6.6.pic}
\end{figure}


\textbf{\textit{Proof.}}
 (Figure \ref{sl.skl.3.6.6.pic}) Štirikotnik $PKQL$ je Varignonov
paralelogram (izrek \ref{Varignon}). Na podoben način dokazujemo, da
je tudi štirikotnik $LNKM$ paralelogram (srednjice trikotnikov $ADB$
in $ACB$). Paralelograma $PKQL$ in $LPKQ$ imata skupno diagonalo
$LK$. Ker se diagonali poljubnega paralelograma razpolavljata (izrek
\ref{paralelogram}), imajo daljice $LK$, $PQ$ in $MN$  skupno
središče.
 \kdokaz

Točko iz prejšnjega primera, v kateri se daljice sekajo, imenujemo
\index{težišče!štirikotnika} \pojem{težišče štirikotnika}. Več
o tem bomo povedali v razdelku \ref{odd5TezVeck}.

Naslednja trditev je lep primer kombiniranja neenakosti trikotnika
in srednjice trikotnika. Trditev je pravzaprav posplošitev izreka
o srednjici trapeza \ref{srednjTrapez}.


              \bizrek
             If $P$ and $Q$ are the midpoints of the sides $AB$ and $CD$ of
             an arbitrary quadrilateral
            $ABCD$, then: $$PQ \leq \frac{1}{2}\left( BC + AD\right).$$
             \eizrek


\begin{figure}[!htb]
\centering
\input{sl.skl.3.6.7.pic}
\caption{} \label{sl.skl.3.6.7.pic}
\end{figure}

\textbf{\textit{Proof.}} Let $S$ be the center of the diagonal $AC$ of the quadrilateral $ABCD$ (Figure \ref{sl.skl.3.6.7.pic}). If we use the theorem about the median of a triangle (\ref{srednjicaTrik}) and the triangle inequality (\ref{neenaktrik}), we get: $$BC + AD = 2PS + 2SQ = 2(PS + SQ) \geq 2PQ.$$ The equality holds when the points $P$, $S$ and $Q$ are collinear, i.e. when the quadrilateral $ABCD$ is a trapezoid with the base $BC$. \kdokaz

 We mention that the inequality from the previous example also holds when the points $A$, $B$, $C$ and $D$ are not in the same plane, i.e. when the $ABCD$ is a \pojem{tetrahedron}.


        \bzgled \label{TezisceSredisceZgled}
        Let $P$ be the midpoint of the median $AA_1$ of a triangle $ABC$ and
        $Q$
        the intersection of the side $AC$ and the line $BP$. Determine the ratios
        $AQ :QC$ and $BP : PQ$.
        \ezgled


\begin{figure}[!htb]
\centering
\input{sl.skl.3.6.8.pic}
\caption{} \label{sl.skl.3.6.8.pic}
\end{figure}


\textbf{\textit{Solution.}} Let $R$ be the center of the line segment $QC$ (Figure \ref{sl.skl.3.6.8.pic}). The line segment $A_1R$ is the median of the triangle $BCQ$ for the base $BQ$, so (by the theorem \ref{srednjicaTrik}) $BQ = 2A_1R$ and $BQ\parallel A_1R$. From this parallelism and the definition of the point $P$ it follows that $PQ$ is the median of the triangle $AA_1R$ for the base $A_1R$, so (by the theorem \ref{srednjicaTrik} and Playfair's axiom \ref{Playfair}) the point $Q$ is the center of the line segment $AR$ and it holds $A_1R = 2PQ$. Therefore: $AQ \cong QR \cong RC$ i.e. $AQ:QC=1:2$. In the end it is also $BQ=2A_1R=4PQ$ i.e. $BP:PQ=3:1$.
 \kdokaz

\bnaloga\footnote{19. IMO Yugoslavia - 1977, Problem 1.}
                Equilateral triangles $ABP$, $BCL$, $CDM$, $DAN$ are constructed inside the
                square $ABCD$. Prove that the midpoints of the segments $LM$, $MN$, $NP$, $PL$, $AN$, $LB$,
                  $BP$, $CM$, $CL$, $DN$, $DM$ in $AP$
                are the twelve vertices of a regular dodecagon ($12$-gon).
                \enaloga

\begin{figure}[!htb]
\centering
\input{sl.skl.3.6.IMO1.pic}
\caption{} \label{sl.skl.3.6.IMO1.pic}
\end{figure}

\textbf{\textit{Solution.}} We mark with $a$ the length of the side and
with $O$ the center of the square $ABCD$ (Figure \ref{sl.skl.3.6.IMO1.pic}).

We first prove that the quadrilateral $MNPL$ is also a square with the same
center $O$. Because $ABP$, $BCL$, $CDM$ and $DAN$ are all right
triangles, the diagonals $MP$ and $LN$ of the quadrilateral $MNPL$ are on
the similitudes of the sides $AB$ and $BC$ of the square $ABCD$. From this it follows that $MP \perp LN$.
Since $d(M,AB)=d(P,CD)=a-v$ (where $v$ is the length
of the height of the aforementioned right triangles), it also holds that $OM\cong OP$.
Similarly, $OL\cong ON$, which means that the quadrilateral $MNPL$
is really a square with the same center~$O$.

We now prove that $LAM$ is a right triangle. Because $AB\cong
AD\cong BL\cong DM=a$ and $\angle LBA\cong\angle MDA
=90^0-60^0=30^0$, the triangles $LBA$ and $MDA$ are similar (by the \textit{SAS} \ref{SKS}). This means that $LA\cong MA$ and $\angle
DAL= 90^0-\angle LAB=15^0$. Similarly, $\angle BAM=15^0$ or
$\angle LAM = 90^0-2\cdot 15^0=60^0$. Therefore, $LAM$ is a right triangle, so the side of the square $MNPL$ has length
$b=|LM|=|LA|$.

Let's mark the center of the line $LM$ with $S$. The centers of the sides of the square $MNPL$ lie on the circle $k(O,\frac{b}{2})$, which is the inscribed circle of this square. We will prove that the point $T$ - the center of the line $AN$ - also lies on this circle. The line $OT$ is the median of the triangle $LAN$ for the base $LA$, so $OT\parallel LA$ and $|OT|=\frac{1}{2}|LA|=\frac{b}{2}$. Therefore, the point $T$ and, analogously, all the points defined in the problem of the $12$-gon lie on the circle $k(O,\frac{b}{2})$.

We will also prove that the aforementioned $12$-gon is regular. Without loss of generality, it is enough to prove that $\angle SOT=\frac{360^0}{12}=30^0$. But from the already proven fact $OT\parallel LA$ it follows that $\angle SOT\cong \angle LAS=\frac{1}{2}\angle LAM=30^0$.
 \kdokaz

%________________________________________________________________________________
 \poglavje{Triangle Centers} \label{odd3ZnamTock}

 We will continue our research with a triangle - the most simple
 polygon,
 which is at the same time
a figure that has unexpectedly many interesting properties. We will
discuss some of them in this section, some of them later, when we will
deal with other concepts, such as isometries and similarity.

 Now we will consider four \index{značilne točke
 trikotnika}\pojem{značilne točke
 trikotnika}\footnote{These four points are mentioned by the Ancient Greeks, although they
 (especially the centroid) were probably known long before that.} and
 their use in quadrilaterals and polygons.

We will start with the first among the characteristic points, which is related to
 the medians of the triangle. We have already defined
  the median in chapter \ref{odd3NeenTrik}.



         \bizrek \label{tezisce}
         The medians of a triangle intersect at one point.
        That point divides  the medians in the ratio $2:1$
         (from the vertex to the midpoint of the opposite side).
        \eizrek

\begin{figure}[!htb]
\centering
\input{sl.skl.3.7.1.pic}
\caption{} \label{sl.skl.3.7.1.pic}
\end{figure}

\textbf{\textit{Proof.}}
 Let $AA_1$, $BB_1$ and $CC_1$ be the altitudes of the triangle $ABC$
   (Figure \ref{sl.skl.3.7.1.pic}).
Because of Pasch's axiom, \ref{AksPascheva} with respect to the triangle
$BCB_1$ and the line $AA_1$, the line $AA_1$ intersects the line segment $BB_1$.
Similarly, the line $BB_1$ intersects the line $AA_1$, which means that the
altitudes $AA_1$ and $BB_1$ intersect at some point $T$. Let $A_2$
and $B_2$ be the midpoints of the line segments $AT$ and $BT$. The line segments $A_1B_1$ and $A_2B_2$
are the medians of the triangles $ABC$ and $ABT$ with the same base $AB$. This
means that the line segments $A_1B_1$ and $A_2B_2$ are parallel and equal
to half of the side $AB$. Therefore, the quadrilateral $B_2A_1B_1A_2$
is a parallelogram (statement \ref{paralelogram}), which means that its
diagonals $A_1A_2$ and $B_1B_2$ intersect at their common center $T$. So it holds:
 $A_1T\cong TA_2 \cong A_2A$  and $B_1T\cong TB_2 \cong B_2B$
 or $AT:TA_1=2:1$ and $BT:TB_1=2:1$ and
  $$A_1T=\frac{1}{3}A_1A \textrm{ and }    B_1T = \frac{1}{3}B_1B.$$
  In the same way
 we prove that the altitudes $AA_1$ and $CC_1$ intersect at some
point $T'$, for which it holds $AT':T'A_1=2:1$ and $CT':T'C_1=2:1$ or:
 $$A_1T'=\frac{1}{3}A_1A \textrm{ and }    C_1T' = \frac{1}{3}C_1C.$$
This means that $T$ and $T'$ are points on the line segment $A_1A$, for which it holds
$A_1T\cong A_1T'=\frac{1}{3}A_1A$, so according to  \ref{ABnaPoltrakCX} $T = T'$, which means that the altitudes $AA_1$,
$BB_1$ and $CC_1$ intersect at the point $T$ and it holds
$AT:TA_1=BT:TB_1=CT:TC_1=2:1$.
 \kdokaz


The point from the previous statement, in which all altitudes of the triangle intersect, is called the
\index{težišče!trikotnika}\pojem{center of the triangle}.

The center of a triangle in the physical sense represents the point that is the center of mass of that triangle. This will be even more clear when we in section \ref{odd8PloTrik} prove the fact that the center divides the triangle into triangles with the same area. In the next chapter \ref{pogVEKT} (section \ref{odd5TezVeck}) we will consider the center of any polygon.

In section \ref{odd3PravilniVeck} we found that for any regular polygon there exist circumscribed (which contains all its vertices) and inscribed (which touches all its sides) circle with the same center. This property is also transferred to regular or equilateral triangles. But how is it with any triangle? We will prove that there exist the mentioned circles for any triangle, just that in the general case they have different centers.

\bizrek \label{SredOcrtaneKrozn}
       The perpendicular bisectors of the sides of any triangle intersect at a single point,
             which is the centre of a circle containing all its vertices.
        \eizrek

\begin{figure}[!htb]
\centering
\input{sl.skl.3.7.2.pic}
\caption{} \label{sl.skl.3.7.2.pic}
\end{figure}


\textbf{\textit{Proof.}}
  Let $p$, $q$ and $r$ be the perpendicular bisectors of sides $BC$, $AC$ and $AB$ of triangle $ABC$ (Figure \ref{sl.skl.3.7.2.pic}). The perpendicular bisectors $p$ and $q$ are not parallel (because in that case by Playfair's axiom \ref{Playfair} the lines $BC$ and $AC$ would be parallel too) and they intersect in some point $O$. Because this point lies on the perpendicular bisectors $p$ and $q$ of sides $BC$ and $AC$, we have $OB \cong OC$ and $OC \cong OA$. From this follows $OA \cong OB$, which means that the point $O$ also lies on the perpendicular bisector $r$ of the line $AB$. So the perpendicular bisectors $p$, $q$ and $r$ intersect in one point.

Because $OA \cong OB \cong OC$, the point $O$ is the center of the circle $k(O,OA)$, which contains all vertices of the triangle $ABC$.
 \kdokaz

The circle from the previous theorem, which contains all the vertices of the triangle, is called the \index{circumscribed circle!triangle} \pojem{circumscribed circle of the triangle}, and its center is the \index{center!circumscribed circle!triangle} \pojem{center of the circumscribed circle of the triangle}.

\bizrek \label{SredVcrtaneKrozn}
The bisectors of the interior angles of any triangle intersect at a single point, which is the centre of a circle touching all its sides.
\eizrek

\begin{figure}[!htb]
\centering
\input{sl.skl.3.7.3.pic}
\caption{} \label{sl.skl.3.7.3.pic}
\end{figure}

\textbf{\textit{Proof.}}
Let $p$, $q$, and $r$ be the angle bisectors of the angles at the vertices $A$, $B$, and $C$ of the triangle $ABC$ (Figure \ref{sl.skl.3.7.3.pic}). We shall prove that the bisectors $p$ and $q$ are not parallel. Otherwise, by Theorem \ref{KotiTransverzala}, the sum of the halves of the angles at the vertices $A$ and $B$ would be equal to $180°$, which, by Theorem \ref{VsotKotTrik}, is not possible. Therefore, $p$ and $q$ intersect at some point $S$. Since the point $S$ lies on the angle bisectors of the angles at the vertices $A$ and $B$, it is equidistant from the sides $AC$ and $AB$ (Theorem \ref{SimKotaKraka}). From this it follows that the point $S$ is also equidistant from the sides $BA$ and $BC$, which means that it lies on the bisector $r$ of the angle at the vertex $C$. Therefore, the angle bisectors $p$, $q$, and $r$ intersect at the point $S$.

With $P$, $Q$ in $R$ we denote the orthogonal projections of point $S$ onto
the sides $BC$, $CA$ and $AB$. Because $\frac{1}{2}\angle CBA<90^0$
and $\frac{1}{2}\angle BCA<90^0$, it follows that $\mathcal{B}(B,P,C)$. Similarly,
$\mathcal{B}(C,Q,A)$ and $\mathcal{B}(A,R,B)$ also hold. Because of the already proven properties of point $S$, it follows that $SP \cong SQ \cong SR$. Therefore, the point $S$ is the center of the circle $l$, which passes through points $P$, $Q$ and
$R$. Because of the perpendicularity of the radii $SP$, $SQ$ and $SR$ to the respective sides, they are tangent to the circle $l$. Because $\mathcal{B}(B,P,C)$,  $\mathcal{B}(C,Q,A)$ and $\mathcal{B}(A,R,B)$ hold, the circle $l$ is tangent to all
         sides of the triangle $ABC$.
 \kdokaz


 The circle from the previous  statement, which is tangent to all
         sides of the triangle, is called  \index{včrtana krožnica!trikotnika} \pojem{the inscribed circle of the triangle}, and its center
        \index{središče!včcrtane krožnice!trikotnika}
        \pojem{the center of the inscribed circle of the triangle}.

There is one more of the four aforementioned characteristic points of the triangle.
It is related to the altitudes of the triangle.


        \bizrek \label{VisinskaTocka}
        The lines containing the altitudes of a triangle  intersect at a single point.
        \eizrek

\begin{figure}[!htb]
\centering
\input{sl.skl.3.7.4.pic}
\caption{} \label{sl.skl.3.7.4.pic}
\end{figure}

\textbf{\textit{Proof.}}
  Let $p$, $q$ and $r$ be the altitude lines
    $AA'$, $BB'$ and $CC'$
   of the triangle $ABC$ (Figure \ref{sl.skl.3.7.4.pic}).
We denote by $a$, $b$ and $c$ the lines that are perpendicular to the respective heights in points $A$, $B$ and $C$. Because the lines $a$, $b$ and $c$ are parallel to the sides of the triangle $ABC$, each two of them intersect. We denote by $P$, $Q$ and $R$ the intersections of the lines $b$ and $c$, $a$ and $c$, and $a$ and $b$, in order. The quadrilateral $ABCQ$ and $RBCA$ are parallelograms, which means that $RA \cong BC \cong AQ$, or the point $A$ is the center of the line $RQ$. The line $AA'$ is therefore the perpendicular bisector of the side $RQ$ of the triangle $PQR$. Similarly, $BB'$ and $CC'$ are the perpendicular bisectors of the sides $PR$ and $PQ$ of the same triangle. By the theorem \ref{SredOcrtaneKrozn}, the perpendicular bisectors $AA'$, $BB'$ and $CC'$ of the triangle $PQR$ intersect in some point $V$. The point $V$ is therefore the intersection of the altitude lines
    $AA'$, $BB'$ and $CC'$
   of the triangle $ABC$.
   \kdokaz

The point from the previous theorem, in which the altitude lines intersect, is called
\index{višinska točka trikotnika}
         \pojem{the altitude point of the triangle}. The triangle $A'B'C'$, which is determined by the altitudes of the triangle $ABC$, is called
\index{trikotnik!pedalni} \pojem{the pedal triangle} of the triangle
$ABC$.

We have found that every triangle has four characteristic points,
namely: the centroid, the center of the circumscribed circle, the center of the inscribed circle and the altitude point. But these are not the only characteristic points
of the triangle. We will mention some of them later. For
a point in the plane of the triangle in general, we say that it is its
\pojem{characteristic point}, if its definition is symmetrical with respect to
the vertices of this triangle.


\begin{figure}[!htb]
\centering
\input{sl.skl.3.7.5.pic}
\caption{} \label{sl.skl.3.7.5.pic}
\end{figure}


 It is clear that in any triangle the four characteristic points differ
 (Figure \ref{sl.skl.3.7.5.pic}).

In an isosceles triangle, the centroid,
the altitude, the line of symmetry of the base, and the line of symmetry of the internal angle opposite the base
have the same carrier. If $A_1$ is the center of the base
$BC$ of the isosceles triangle $ABC$, the triangles $ABA_1$ and
$ACA_1$ are congruent, which means that the angle at the vertex $A_1$ is a right angle
and the angles $BAA_1$ and $CAA_1$ are congruent. Therefore, the distance
$AA_1$ is both the centroid and the altitude, and the line $AA_1$ is both
the line of symmetry of the side $BC$ and the line of symmetry of the internal angle at the vertex $A$
of the triangle $ABC$. It follows that all four characteristic points of this
triangle lie on one line $AA_1$ (Figure
\ref{sl.skl.3.7.6.pic}).

If we use the already proven property of an isosceles triangle for
an equilateral triangle, we find that all the appropriate
centroids, altitudes, side lines of symmetry, and internal angle lines of symmetry
have the same carrier. This means that in an equilateral triangle
all four characteristic points coincide (Figure \ref{sl.skl.3.7.6.pic}).
This is actually already defined (section \ref{odd3PravilniVeck})
as the center of this equilateral (or regular) triangle.


\begin{figure}[!htb]
\centering
\input{sl.skl.3.7.6.pic}
\caption{} \label{sl.skl.3.7.6.pic}
\end{figure}

We will also show the position of the characteristic points with respect to the type of
triangle.

\begin{figure}[!htb]
\centering
\input{sl.skl.3.7.7.pic}
\caption{} \label{sl.skl.3.7.7.pic}
\end{figure}

The centroids are always inside the triangle. Therefore, the center
is an internal point of every triangle (Figure \ref{sl.skl.3.7.7.pic}).
The same conclusion applies to the center of the inscribed circle (Figure
\ref{sl.skl.3.7.7s.pic}).

\begin{figure}[!htb]
\centering
\input{sl.skl.3.7.7s.pic}
\caption{} \label{sl.skl.3.7.7s.pic}
\end{figure}

In an acute triangle, the vertices of the perpendiculars from its vertices
lie on the sides of this triangle, which means (Pasch's axiom
\ref{AksPascheva}), that its altitudes intersect in the interior. So in
an acute triangle, the altitude point lies in its interior
(Figure \ref{sl.skl.3.7.7v.pic}). In a right-angled triangle, the altitude point is the vertex at the right angle. This is because its
catheti are at the same time altitudes of the triangle. The altitude point of an obtuse triangle lies in its exterior, because
 not all of the altitudes are in its interior. The corresponding vertices
belong to the sides' supports, not to the sides themselves.

\begin{figure}[!htb]
\centering
\input{sl.skl.3.7.7v.pic}
\caption{} \label{sl.skl.3.7.7v.pic}
\end{figure}

The centre of the circumscribed circle is an interior or exterior point of the triangle, depending on whether the triangle is acute or obtuse (Figure
\ref{sl.skl.3.7.7o.pic}). We will omit the formal proof of this fact. We will only prove the following statement, which refers to right-angled triangles.

\begin{figure}[!htb]
\centering
\input{sl.skl.3.7.7o.pic}
\caption{} \label{sl.skl.3.7.7o.pic}
\end{figure}


        \bizrek The circumcentre of a right-angled triangle is at the same time the midpoint of
        its hypotenuse.
        \eizrek

\begin{figure}[!htb]
\centering
\input{sl.skl.3.7.8.pic}
\caption{} \label{sl.skl.3.7.8.pic}
\end{figure}


\textbf{\textit{Proof.}} We mark with $O$ the centre of the hypotenuse $AB$ and
with $P$ the centre of the cathetus $AC$ of the right-angled triangle $ABC$ (Figure
\ref{sl.skl.3.7.8.pic}). The distance $OP$ is the median of this triangle,
which corresponds to the cathetus $BC$, so $OP\parallel BC$. From this it follows that $OP
\perp AC$. Therefore, the triangles $OPC$ and $OPA$  are congruent (the statement
\textit{SAS} \ref{SKS}) and then $OC \cong OA$.
Since $OB \cong OA$ as well,
 the point $O$ is the centre of the circumscribed circle of this triangle.
 \kdokaz

The line $OC$ from the previous theorem is the median of the triangle. This means that the median of a right angled triangle is equal to the radius of the inscribed circle of that triangle, and also to half of its hypotenuse.

The previous theorem is also connected to Thales’ theorem for a circle \ref{TalesovIzrKroz} and its converse \ref{TalesovIzrKrozObrat}. We will now state all of these theorems in one theorem.


          \bizrek Thales’ theorem for a circle (several forms - Figure
          \ref{sl.skl.3.7.9.pic}):
         \index{theorem!Thales’ for a circle}
           \label{TalesovIzrKroz2}
           \begin{enumerate}
            \item The circumcentre of a right-angled triangle is at the same time the midpoint of
                 its hypotenuse.
             \item If $t_c$ is the median of a right-angled triangle for its hypotenuse $c$ and $R$
               the circumradius of that triangle, then
            $R=t_c=\frac{c}{2}$.
               \item If $AB$ is a diameter of a circle $k$, then for any point $X\in k$  ($X\neq A$ and $X\neq B$) is
               $\angle AXB=90^0$.
              \item If $A$, $B$ in $X$ are three non-collinear points, such that $\angle AXB=90^0$, then the point $X$ lies on a circle with the diameter $AB$.
                 \end{enumerate}
            \index{theorem!Thales’ for a circle}
             \eizrek

\begin{figure}[!htb]
\centering
\input{sl.skl.3.7.9.pic}
\caption{} \label{sl.skl.3.7.9.pic}
\end{figure}

Knowing the characteristic points of a triangle and the properties of a median of a triangle allows us to prove various other properties of both triangles and quadrilaterals and $n$-gons.


              \bzgled
              Let $CD$ be the altitude  at the hypotenuse $AB$ of a right-angled triangle $ABC$.
            If $M$ and $N$ are the midpoints of the line segments $CD$ and $BD$, then $AM \perp CN$.
           \ezgled

\begin{figure}[!htb]
\centering
\input{sl.skl.3.7.10.pic}
\caption{} \label{sl.skl.3.7.10.pic}
\end{figure}

\textbf{\textit{Proof.}}  The line $NM$ is the median of the triangle
$BCD$, so by the \ref{srednjicaTrik} $NM \parallel BC$
(Figure \ref{sl.skl.3.7.10.pic}). Because the angle at the vertex $C$ is a right
angle, $NM \perp AC$ as well. Because of this, the line $NM$ is the altitude of the triangle $ANC$. Since $CD$ is also the altitude of this triangle, $M$ is its height point. Therefore, the line $AM$ is the altitude of the third triangle of this triangle and $AM \perp CN$ applies.
 \kdokaz


        \bzgled
        If $P$ and $Q$ are the midpoints of the sides $BC$ and $CD$ of a parallelogram $ABCD$,
          then the lines $AP$ and $AQ$
         divide the diagonal $BD$ of this parallelogram into three congruent line segments.
        \ezgled

\begin{figure}[!htb]
\centering
\input{sl.skl.3.7.11.pic}
\caption{} \label{sl.skl.3.7.11.pic}
\end{figure}

\textbf{\textit{Proof.}} We mark with $E$ and $F$ the intersections of the lines
$AP$ and $AQ$ with the diagonal $BD$ of the parallelogram $ABCD$ and with $S$
the intersection of its diagonals $AC$ and $BD$ (Figure
\ref{sl.skl.3.7.11.pic}). The diagonals of the parallelogram are divided
(from \ref{paralelogram}), so the point $S$ is the common center of the lines
$AC$ and $BD$. This means that the points $E$ and $F$ are the centers of the triangles
$ACB$ and $ACD$, so they divide the altitudes $SB$ and $SD$ of these triangles
in the ratio $2:1$ (from \ref{tezisce}). Therefore:
 \begin{eqnarray*}
     BE &=& \frac{2}{3}BS = \frac{2}{3}DS = FD,\\
     EF&=& ES+ SF= \frac{1}{3}SB+ \frac{1}{3}SD=
     \frac{1}{3}(SB +SD)=\frac{1}{3} BD,
 \end{eqnarray*}
  which is what we wanted to prove.  \kdokaz


           \bzgled
            Let $BAKL$ and $ACPQ$ be positively oriented squares
        in the same plane. Prove that the lines $BP$ and $CL$ intersect at a point lying
        on the line containing the altitude $AA'$ of the triangle $ABC$.
           \ezgled

\begin{figure}[!htb]
\centering
\input{sl.skl.3.7.12.pic}
\caption{} \label{sl.skl.3.7.12.pic}
\end{figure}

\textbf{\textit{Proof.}}
 Let $AA'$
 be the altitude of the triangle $ABC$ (Figure \ref{sl.skl.3.7.12.pic}). We mark with $X$
  and $Y$ the intersections of the line $AA'$ with
the rectangles on the line $CL$ through the point $B$ and on the line $BP$ through the point $C$.
We prove that $X = Y$. The triangle $BLC$ and $ABX$ are similar according to the \textit{ASA} theorem \ref{KSK} because: $BL \cong AB$,
$\angle BLC\cong\angle ABX$ and $\angle BCL\cong\angle AXB$ (the angle with perpendicular sides -
theorem \ref{KotaPravokKraki}). Therefore $AX \cong BC$. Similarly,
the triangle $CPB$ and $ACY$ are similar, so $AY \cong BC$.
Therefore $AX \cong AY$ or $X = Y$. This means that the lines
$AA'$, $BP$ and $CL$ are the altitudes of the triangle $XBC$, so they intersect at
one point.
 \kdokaz


          \bzgled \label{zgledPravokotnik}
            Let $K$ be the midpoint of the side $CD$ of a rectangle $ABCD$.
          A point $L$ is the foot of the perpendicular from the vertex $B$ on the diagonal
         $AC$ and $S$ is the midpoint of
            the line segment $AL$. Prove that $\angle KSB$ is a right angle.
         \ezgled

\begin{figure}[!htb]
\centering
\input{sl.skl.3.7.13.pic}
\caption{} \label{sl.skl.3.7.13.pic}
\end{figure}

\textbf{\textit{Proof.}} Let $V$ be the center of the line $BL$ (Figure \ref{sl.skl.3.7.13.pic}). The line $SV$ is the median of the triangle $ABL$ for the base $AB$, so $SV\parallel AB$ and $SV =\frac{1}{2} AB$ (statement \ref{srednjicaTrik}). From the first relation and $BC\perp AB$ it follows that $SV\perp BC$ (statement \ref{KotiTransverzala}). This means that $BL$ and $SV$ are the altitude of the triangle $CSB$. Therefore, $V$  is the altitude point of this triangle, so $CV$ is the altitude of its third altitude (statement \ref{VisinskaTocka}) or $CV\perp SB$ is true. From $SV\parallel AB$ and $SV =\frac{1}{2} AB=KC$ it follows that the quadrilateral $SVCK$ is a parallelogram or $CV\parallel SK$ is true. From this and $CV\perp SB$ it finally follows (statement \ref{KotiTransverzala})  $SK\perp SB$, so $\angle KSB$ is a right angle.
 \kdokaz



              \bzgled
              Let $AP$, $BQ$ and $CR$ be the altitude, the median
            and the bisector of the angle $ACB$ ($R\in AB$) of a triangle $ABC$. Prove that if
            the triangle $PQR$ is regular, then the triangle $ABC$ is also regular.
             \ezgled


\begin{figure}[!htb]
\centering
\input{sl.skl.3.7.14.pic}
\caption{} \label{sl.skl.3.7.14.pic}
\end{figure}

\textbf{\textit{Proof.}} Let $PQR$ be a right triangle or $PQ\cong QR\cong RP$ (Figure \ref{sl.skl.3.7.14.pic}). The point $Q$ is the center of the hypotenuse $AC$ of the right triangle $APC$, so $QA \cong QC \cong QP$ (statement \ref{TalesovIzrKroz2}. Because in this case $QR\cong QC\cong QP$, from the same statement it follows that $\angle ARC$ is a right angle. From the similarity of the triangles $ACR$ and $BCR$ (statement \ref{KSK}) we get that the point $R$ is the center of the side $AB$ and that $AC\cong BC$ is true. Because the point $R$ is the center of the hypotenuse $AB$ of the right triangle $APB$, $AB = 2RP = 2PQ = 2AQ = AC$. Therefore, $AB\cong AC\cong BC$ is true, which means that $ABC$ is a right triangle.
 \kdokaz

It is not difficult to prove that a triangle is equilateral if and only if the corresponding medians are concurrent. The same is true for altitudes. But is something similar true for so-called angle bisectors?
 The line segments $BB'$ and $CC'$, where $BB'$ and $CC'$ are angle bisectors of the triangle $ABC$ and $B'\in AC$ and $C'\in AB$, are called
 \index{angle bisector} \pojem{angle bisectors}. Angle bisectors are denoted by $l_a$, $l_b$ and $l_c$.
 The aforementioned
  statement  is also true in this case, but
the proof is not so simple. This is the subject of the following well-known theorem.



            \bizrek \index{theorem!Steiner-Lemus}
            (Steiner-Lehmus\footnote{\textit{D. C. L. Lehmus} (1780--1863),\index{Lehmus, D. C. L.} French
            mathematician, who in 1840 sent this, at first glance simple
            statement, to the famous Swiss geometer \index{Steiner, J.} \textit{J. Steiner} (1796--1863),
             who  derived a very extensive proof of this theorem. Then followed
             several different solutions to this problem and one of them was published in 1908
              by
             the French mathematician \index{Poincar\'{e}, J. H.} \textit{J. H. Poincar\'{e}} (1854--1912).})
            Let  $BB'$ ($B' \in AC$) and $CC'$ ($C' \in AB$) be the bisectors
              of the interior angles of a triangle $ABC$. Then:
             $$AB \cong AC \Leftrightarrow BB'\cong CC'.$$
            \eizrek


\begin{figure}[!htb]
\centering
\input{sl.skl.3.7.15.pic}
\caption{} \label{sl.skl.3.7.15.pic}
\end{figure}

\textbf{\textit{Proof.}} (Figure \ref{sl.skl.3.7.15.pic})

($\Rightarrow$) From $AB \cong AC$ it follows that $\angle ABC \cong \angle
ACB$ (theorem \ref{enakokraki}) or $\angle B'BC \cong \angle C'CB$.
By the \textit{ASA} theorem \ref{KSK}, the triangles $B'BC$ and $C'CB$
are congruent, so $BB'\cong CC'$.

($\Leftarrow$) Let $BB'\cong CC'$.
 We assume that $AB\not\cong AC$. Without loss of generality
 let $AB < AC$. In this case $\angle ACB < \angle ABC$
 (by \ref{vecstrveckot}) or
$\angle ACC'< \angle ABB'$. This means that inside the angle $ABB'$
there is a segment $p$ with endpoint $B$, which intersects the side
$AC$ in such a point $D$, that both $\mathcal{B}(A,D,B')$ and $\angle
DBB'\cong \angle ACC'$ are true. In the triangle $BCD$ is $\angle ACB <
\angle DBC$ and because of that also $BD < CD$ (by \ref{vecstrveckot}).
Therefore there is such a point $E$, which is between the points $C$
and $D$, so that $BD \cong CE$. By the \textit{SAS} \ref{SKS} the
triangles $BDB'$ and $CEC'$ are similar, therefore the angles $BDB'$
and $CEC'$ are similar. We prove that this is not possible. Because
of Pasch's axiom \ref{AksPascheva} (used for the triangle $AC'E$ and
the line $BD$) the line $BD$ intersects the line $C'E$ in some point
$S$. In the triangle $SDE$ is the angle $SEC$ (or the angle $CEC'$)
external and by \ref{zunanjiNotrNotrVecji} it can not be similar to
the adjacent internal angle $SDE$ (or the angle $BDB'$). This means
that the assumption $AB < AC$ (analogously $AB > AC$) is not possible.
Therefore $AB\cong AC$.
 \kdokaz


         \bzgled
          Let $A_1$ be the midpoint of the side $BC$ of a triangle $ABC$.
         Calculate the measure of the angle $AA_1C$, if
           $\angle BAC=45^0$ and $\angle ABC=30^0$.
         \ezgled

\begin{figure}[!htb]
\centering
\input{sl.skl.3.7.1a.pic}
\caption{} \label{sl.skl.3.7.1a.pic}
\end{figure}

From $\angle CC'B = 90^0$ it follows first
$\angle C'CB=60^0$, then that the point $C'$ lies on the circle above
the diameter $CB$ and with the center $A_1$ (by \ref{TalesovIzrKroz}), so
$A_1C'\cong A_1C\cong A_1B$. Therefore, the triangle $CC'A_1$
is isosceles, or by \ref{enakokraki} it holds $\angle CC'A_1
\cong\angle C'CB=60^0$. This means that the triangle $CC'A_1$
is equilateral and $C'C\cong C'A_1$. From the fact that $AC'C$ is an isosceles triangle ($\angle CAC'=\angle ACC'=45^0$), it follows that $AC'\cong C'C$. If we connect this with the previous relation, we get $AC'\cong C'A_1$, which
means that the triangle $AC'A_1$ is also isosceles. Therefore, it is (by \ref{enakokraki} and \ref{zunanjiNotrNotr}):
 $$\angle C'A_1A\cong\angle C'AA_1=\frac{1}{2}\angle A_1C'B=
 \frac{1}{2}\angle C'BA_1=\frac{1}{2}\cdot 30^0=15^0.$$
 In the end there is also:
  $$\angle AA_1C=\angle C'A_1C-\angle C'A_1A =60^0-15^0=45^0,$$ which needed to be calculated. \kdokaz


        \bzgled
        Let $P$ be the midpoint of the side $BC$ of an isosceles triangle $ABC$
            and $Q$ the foot of the perpendicular from the point $P$ on the leg
        $AC$ of that triangle. Let $S$ be the midpoint of the line segment $PQ$. Prove that
        $AS \perp BQ$.
        \ezgled


\begin{figure}[!htb]
\centering
\input{sl.skl.3.7.16.pic}
\caption{} \label{sl.skl.3.7.16.pic}
\end{figure}

\textbf{\textit{Proof.}} From the congruence of the triangles $ABP$ and $ACP$
(from the statement \textit{SSS} \ref{SSS}) it follows that the congruence of the sides $APB$ and
$APC$ or $AP\perp BC$ (Figure \ref{sl.skl.3.7.16.pic}). We mark with
$R$ the center of the line $QC$. The line $SR$ is the median of the triangle
$QPC$ for the base $PC$, so according to the statement \ref{srednjicaTrik}
$SR\parallel CP$. From this and $AP\perp BC$ it follows that $SR\perp AP$. So
$S$ is the altitude point of the triangle $APR$, so according to the statement
\ref{VisinskaTocka} also $AS\perp PR$. But the line $PR$ is
the median of the triangle $BQC$ for the base $BQ$, so $PR\parallel BQ$
(from the statement \ref{srednjicaTrik}). From $AS\perp PR$ and $PR\parallel BQ$
we get $AS\perp BQ$.
 \kdokaz


       \bzgled \label{kotBSC}
      If $S$ is the incentre and $\alpha$, $\beta$,
       $\gamma$ the interior angles  at the vertices
        $A$, $B$, $C$  of a triangle
       $ABC$, then
       $$\angle BSC=90^0+\frac{1}{2}\cdot\alpha.$$
       \ezgled


\begin{figure}[!htb]
\centering
\input{sl.skl.3.7.1c.pic}
\caption{} \label{sl.skl.3.7.1c.pic}
\end{figure}

\textbf{\textit{Proof.}}  According to the statement \ref{SredVcrtaneKrozn} the simetrals of the internal angles of the triangle $ABC$ intersect in the center of the inscribed circle - in the point $S$ (Figure \ref{sl.skl.3.7.1c.pic}). So
$\angle SBC =\frac{1}{2}\cdot \beta$ and $\angle SCB
=\frac{1}{2}\cdot \gamma$. Because according to the statement \ref{VsotKotTrik} in every
triangle the sum of the internal angles is equal to $180^0$, it follows:
 $$\angle BSC = 180^0-\frac{1}{2}\cdot\left( \beta+
  \gamma\right)=180^0-\frac{1}{2}\cdot\left( 180^0-
 \alpha\right)=90^0+\frac{1}{2}\cdot\alpha,$$ which had to be proven. \kdokaz


        \bzgled
        Construct a triangle with given $a$, $t_a$, $R$ (see the labels  in section \ref{odd3Stirik}).
        \ezgled


\begin{figure}[!htb]
\centering
\input{sl.skl.3.7.16a.pic}
\caption{} \label{sl.skl.3.7.16a.pic}
\end{figure}

\textbf{\textit{Proof.}} The construction can be carried out by first drawing the circumscribed circle $k(O,R)$, choosing an arbitrary point $B\in k$, planning the cord $BC\cong a$ of the circle, the center $A_1$ of the cord $BC$ and finally the point $A$ as the intersection of the circles $k(O,R)$ and $k_1(A_1,t_a)$ (Figure \ref{sl.skl.3.7.16a.pic}). It is clear that the task of the solution is exactly when $a\leq 2R$ and the intersection of the circles $k(O,R)$ and $k_1(A_1,t_a)$ is not an empty set. The number of solutions in this case depends on the number of intersections of the circles $k(O,R)$ and $k_1(A_1,t_a)$.
 \kdokaz

        \bzgled
        Construct a right-angled triangle if the hypotenuse and the altitude to that hypotenuse
         are congruent to the given line segments $c$ and $v_c$.
        \ezgled



\begin{figure}[!htb]
\centering
\input{sl.skl.3.7.16b.pic}
\caption{} \label{sl.skl.3.7.16b.pic}
\end{figure}

\textbf{\textit{Analysis.}}
Let $ABC$ be a right-angled triangle with a right angle at the vertex $C$,
in which the hypotenuse $AB$ and the altitude $CC'$ are congruent to the line segments $c$ and $v_c$ (Figure \ref{sl.skl.3.7.16b.pic}).
By the \ref{TalesovIzrKroz2} theorem, the center $O$ of the hypotenuse $AB$ is also the center of the circumscribed circle of the triangle $ABC$.
Therefore, the vertex $C$ lies on the circle $k$ with diameter $AB$. Since $CC'\cong v_c$, the vertex $C$ also lies on the parallel $p$ to the line $AB$ at a distance $v_c$. The point $C$ is then the intersection of this parallel and the circle $k$.


\textbf{\textit{Construction.}}
First, we plan the line $AB$, which is congruent to the given line segment $c$, then the center $O$ of the line $AB$ and the circle $k(O,OA)$. Then we plan the parallel $p$ to the line $AB$ at a distance $v_c$. One of the intersections of the line $p$ and the circle $k(O,OA)$ is denoted by $C$. We prove that $ABC$ is the desired triangle.

\textbf{\textit{Proof.}}
 By construction, point $C$ lies on the circle with radius $AB$, so by \ref{TalesovIzrKroz2} $\angle ACB=90^0$, which means that $ABC$ is a right triangle with hypotenuse $AB$. By construction, it is consistent with the distance $c$. Let $CC'$ be the altitude of the triangle $ABC$. By construction, point $C$ lies on the line $p$, which is from the line $AB$ at a distance $v_c$, so also $|CC'|=d(C,AB)=d(p,AB)$ or $CC'\cong v_c$.


\textbf{\textit{Discussion.}}
 The number of solutions is dependent on the number of intersections of the line $p$ and the circle $k(O,OA)$.
 \kdokaz



        \bnaloga\footnote{2. IMO Romania - 1960, Problem 4.}
         Construct triangle ABC, given $v_a$, $v_b$ (the altitudes from $A$ and $B$) and $t_a$,
         the median from vertex $A$.
         \enaloga


\begin{figure}[!htb]
\centering
\input{sl.skl.3.7.IMO1.pic}
\caption{} \label{sl.skl.3.7.IMO1.pic}
\end{figure}

\textbf{\textit{Solution.}} Let $ABC$ be a triangle, such that
$AA' \cong v_a$ and $BB' \cong v_b$ are its altitudes and $AA_1\cong
t_a$ is its median (Figure \ref{sl.skl.3.7.IMO1.pic}). We mark with
$A'_1$ the orthogonal projection of point $A_1$ on the line $AC$.
The distance $A_1A'_1$ is the median of the triangle $BB'C$ for the
base $BB'$, so by \ref{srednjicaTrik}:
 $$|A_1A'_1|=\frac{1}{2}\cdot|BB'|=\frac{1}{2}\cdot v_b
 \hspace*{1mm} \textrm{ in }  \hspace*{1mm} A_1A'_1
\parallel BB'.$$ From this it follows that the lines
$A_1A'_1$ and $AC$ are perpendicular in point $A'_1$, thus the line $AC$
is tangent to the circle $k(A_1,\frac{1}{2} v_b)$ \ref{TangPogoj}.
The proven properties allow us to construct it.

First, we can plan the rectangular triangle $AA'A_1$ ($AA'\cong v_a$,
 $AA_1\cong t_a$ and $\angle AA'A_1 = 90^0$), then the circle
 $k(A_1,\frac{1}{2} v_b)$. From the point $A$ we plan the tangents to
 the circle $k(A_1,\frac{1}{2} v_b)$. The intersection of one of the tangents
 with the line $A'A_1$ is denoted by $C$. In the end, we plan such a point
  $B$, that
  $BA_1 \cong CA_1$ and $\mathcal{B}(C,A_1,B)$.

 We prove that the triangle $ABC$ satisfies the given conditions. From
 the construction it is $AA'\cong v_a$ the height and $AA_1 \cong t_a$
 the centroid (because $A_1$ is the center of the line $BC$) of the triangle
 $ABC$. Let $BB'$ be the height of this triangle. We prove that $BB' \cong
 v_b$.
 The line $AC$ is by construction the tangent of the circle $k(A_1,\frac{1}{2}
 v_b)$. Their point of contact is denoted by $A'_1$. By the theorem
 \ref{TangPogoj} the lines
$A_1A'_1$ and $AC$ are perpendicular in the point $A'_1$, therefore the line
$A_1A'_1$ is the median of the triangle $BB'C$ for the base $BB'$ and it is valid
$|BB'|= 2\cdot |A_1A'_1|=2\cdot\frac{1}{2}\cdot v_b=v_b$.

 The task has no solution when $v_a>t_a$. If  $v_a\leq
 t_a$, the number of solutions depends on the number of tangents, which we
 can plan from the point $A$ on the circle $k(A_1,\frac{1}{2}
 v_b)$. In this case, the tangent must intersect the line $A'A_1$.
 If $\frac{1}{2} v_b<t_a$ and $\frac{1}{2} v_b\neq v_a$,
  the task has two solutions, in the case $\frac{1}{2} v_b<t_a$ and
  $\frac{1}{2} v_b = v_a$ there is only one solution,  in the case
  $\frac{1}{2} v_b\geq t_a$ there are no solutions.
 \kdokaz


%________________________________________________________________________________
 \poglavje{Euler's Circle. Eight Point Circle}
 \label{odd3EulKroz}

We will now look at an interesting property that relates to
\index{trikotnik!pedalni}pedalni trikotnik, which we have already defined
as a triangle that is determined by the points of intersection of the heights of some triangle, and i.e.
\index{trikotnik!središčni} \pojem{središčni trikotnik}, which is determined by
the midpoints of the sides of this triangle. We will prove that
the aforementioned triangles have a common circumscribed circle (Figure
\ref{sl.skl.3.8.1.pic}). But before we do that, we prove the following lemma.


\begin{figure}[!htb]
\centering
\input{sl.skl.3.8.1.pic}
\caption{} \label{sl.skl.3.8.1.pic}
\end{figure}



        \bizrek \label{EulerKroznicaLema}
        Let $V$ be the orthocentre of a triangle $ABC$. If $K$, $L$, $M$
         and $N$ are the midpoints of the line segments $AB$, $AC$, $VC$
        and $VB$, respectively, then the quadrilateral $KLMN$ is a rectangle.
        \eizrek


\begin{figure}[!htb]
\centering
\input{sl.skl.3.8.2.pic}
\caption{} \label{sl.skl.3.8.2.pic}
\end{figure}

\textbf{\textit{Proof.}}
 Let $AA'$ be the height of this triangle. The line segments
$KN$
and $LM$ are the midpoints of the triangles $ABA'$ and $CAA'$ for the common
base $AA'$, so according to izrek \ref{srednjicaTrik}
$KN=\frac{1}{2}AA'=LM$ and $KN\parallel AA'\parallel LM$ (Figure
\ref{sl.skl.3.8.2.pic}). Therefore, the quadrilateral $KLMN$  is a parallelogram.
It is enough to prove that it has at least one internal angle that is a right angle.
The line segment $KL$ is the midpoint of the triangle $ABC$, so $KL\parallel
BC$. Since $KN\parallel AA'$ and $AA'\perp BC$, it follows that
$KL\perp KN$ or $\angle LKN=90^0$, which means that
the parallelogram $KLMN$ is also a rectangle.
 \kdokaz

 We are now ready to prove the main theorem.

\bizrek \label{EulerKroznica}
        For any given triangle
        the midpoints of the sides, the foots of the altitudes and
         the midpoints of the line segments from each vertex of the triangle to the orthocentre
         lie on the common circle - so-called  \index{krožnica!Eulerjeva}
        \pojem{Euler’s circle}\color{blue}\footnote{Krožnico imenujemo po
        švicarskem matematiku \index{Euler, L.} \textit{L. Eulerju}
        (1707--1783), ki je že leta 1765 dokazal, da imata pedalni in središčni
        trikotnik  skupno očrtano krožnico.
        Ostale lastnosti te krožnice so  leta 1821
        najprej obravnavali francoski matematiki \index{Poncelet, J. V.}
        \index{Brianchon, C. J.} \index{Terquem, O.} \index{Feuerbach, K. W.}
        \textit{J. V.
        Poncelet} (1788--1867) \textit{C. Brianchon}  (1783--1864) in
        \textit{O. Terquem}
        (1782--1862), kasneje pa še nemški matematik
         \textit{K. W. Feuerbach} (1800--1834).} of that triangle.
        \eizrek

\begin{figure}[!htb]
\centering
\input{sl.skl.3.8.3.pic}
\caption{} \label{sl.skl.3.8.3.pic}
\end{figure}

\textbf{\textit{Proof.}}
We will use the previous claim \ref{EulerKroznica}. If we keep the same labels, we have already proven that the quadrilateral $KLMN$ is a rectangle (Figure \ref{sl.skl.3.8.3.pic}). If we denote with $P$ the midpoint of the side $BC$ and with $Q$ the midpoint of the line $AV$, then also $PMQK$ is a rectangle. Because $KM$ is the common diagonal of these two rectangles, it is the diameter of their common circumscribed circle $e$. Therefore, the midpoints of the sides and the midpoints of the lines connecting the altitude point and the vertex belong to the same circle $e$. We will also prove that the intersection points of the altitudes lie on this circle. The point $A'$, which is the intersection point of the altitude from the vertex $A$, lies on the circle $e$, because $\angle QA'P$ is a right angle and $PQ$ is the diameter of the circle $e$ (Tales' theorem \ref{TalesovIzrKroz2}). Analogously, on this circle also lie the intersection points $B'$ and $C'$ of the altitudes $BB'$ and $CC'$.
 \kdokaz

 We call the Euler circle also the \index{circle!nine points} \index{circle!Feuerbach's} \pojem{Feuerbach circle} and the \pojem{circle of nine points}. We will discuss some more properties of the Euler circle in sections \ref{odd5EulPrem} and \ref{odd7SredRazteg}. Now we prove the analogous claim for the quadrilaterals called the \index{circle!eight points} \pojem{circle of eight points}.



        \bizrek
        Let $ABCD$ be a quadrilateral with the perpendicular diagonals.
        Then the midpoints of the sides
         and the foots of the perpendiculars from these midpoints to the line
         containing the opposite sides of that quadrilateral lie on the same circle.
        \eizrek

\begin{figure}[!htb]
\centering
\input{sl.skl.3.8.4.pic}
\caption{} \label{sl.skl.3.8.4.pic}
\end{figure}

\textbf{\textit{Proof.}} Let $ABCD$ be a quadrilateral with
perpendicular diagonals $AC$ and $BD$ (Figure \ref{sl.skl.3.8.4.pic}).
Let $A_1$, $B_1$, $C_1$ and $D_1$ be the centers of its sides $AB$,
$BC$, $CD$ and $DA$, and let $A'$, $B'$, $C'$ and $D'$ be the
perpendicular projections of these centers onto the lines of the
opposite sides of this quadrilateral. The quadrilateral $A_1B_1C_1D_1$
is a parallelogram (Varignon's parallelogram - Theorem
\ref{Varignon}). Because the diagonals $AC$ and $BD$ are
perpendicular, this parallelogram is a rectangle (Theorem
\ref{VarignonPoslPravRomb}), so its vertices lie on the same circle.
The diagonals $A_1C_1$ and $B_1D_1$ are the diameters of this circle.
Because $\angle C_1C'A_1=\angle C_1A'A_1=\angle B_1D'D_1=\angle
B_1B'D_1=90^0$, it follows (Tales' Theorem \ref{TalesovIzrKroz2}) that
the points $A'$, $B'$, $C'$ and $D'$ also lie on this circle.
 \kdokaz

 We mention that for any triangle $ABC$ with altitude point $V$
  its Euler circle  can be seen as
the circle of eight points of the quadrilateral $ABVC$ (its diagonals
$AV$ and $BC$ are perpendicular - Figure \ref{sl.skl.3.8.5.pic}), but
in this case two pairs of points overlap and we actually get only six
points\footnote{This fact was proved by the American mathematician
\index{Brand, L.} \textit{L. Brand} (1885--1971) in 1944}. To prove
that the statement (for the Euler circle) is true for the other three
points, we use the circle of eight points for the quadrilateral
$CAVB$. Because the two circles (for the quadrilateral $ABVC$ and
$CAVB$) have at least three common points, the circles overlap - they
are the same Euler circle of the triangle $ABC$.

\begin{figure}[!htb]
\centering
\input{sl.skl.3.8.5.pic}
\caption{} \label{sl.skl.3.8.5.pic}
\end{figure}

 %_______________________________________________________________________________
 \poglavje{Tessellations} \label{odd3Tlakovanja}

In this section we will deal with the covering of the plane with
congruent figures. Such a covering is called a
\index{tlakovanja} \pojem{tessellation} or a
\index{teselacija} \pojem{tessellation} of the plane. The figure
with which we cover the plane in this way is called a
\index{celica tlakovanja} \pojem{tessellation cell}. The most
famous tessellation is of course the covering of the plane with
congruent squares. We will also consider tessellations with other
figures. First we will answer the question of which tessellations
with regular polygons are possible.

\bizrek \label{pravilnaTlakovanja}
All possible tessellations $(n,m)$ of the plane with a regular $n$-gons,
$m$ of them around each vertex, are
(Figure \ref{sl.skl.3.9.1.pic})\footnote{This problem was solved by
the famous Greek philosopher and mathematician \index{Pitagora}
\textit{Pitagora from the island of Samos}
 (582--497 BC).}:
        $$(4,4),\hspace*{1mm} (6,3)\textrm{ in }
        (3,6).$$
        \eizrek

\begin{figure}[!htb]
\centering
\input{sl.skl.3.9.1aa.pic}
\input{sl.skl.3.9.1bb.pic}
\input{sl.skl.3.9.1.pic}
\caption{} \label{sl.skl.3.9.1.pic}
\end{figure}

\textbf{\textit{Proof.}} Let $O$ be the center and $AB$ one side of
the tessellation cell $(n,m)$ - a regular $n$-gon (Figure
\ref{sl.skl.3.9.2.pic}). With $S$ we denote the center of the side
$AB$. Because the initial $n$-gon is regular and there are $m$ such
around the vertex $B$, the internal angles of the triangle $OSB$ at
the vertices $O$, $B$ and $S$ measure $\frac{360^0}{2n}$,
$\frac{360^0}{2m}$ and $90^0$ in order. By \ref{VsotKotTrik} the
$\frac{360^0}{2n}+\frac{360^0}{2m}+90^0=180^0$. If we simplify the
equality, we get the equivalent equality:
$$\frac{1}{n}+\frac{1}{m}=\frac{1}{2},$$
or $nm-2n-2m=0$ and finally:
 \begin{eqnarray}
(n-2)(m-2)=4. \label{teselRelEvk}
\end{eqnarray}

Since $n$ and $m$ are natural numbers and greater than $2$,
the only solutions of the last equation are: $(n,m)\in \{(4,4), (3,6), (6,3)\}$.
 \kdokaz

\begin{figure}[!htb]
\centering
\input{sl.skl.3.9.2.pic}
\caption{} \label{sl.skl.3.9.2.pic}
\end{figure}

Tiling of the plane with regular polygons is called
\index{tlakovanja!pravilna} \pojem{pravilna tlakovanja} ravnine. In
Euclidean plane  there are therefore three regular tilings.

Since in hyperbolic geometry the sum of the interior angles of a triangle
is always less than $180^0$, the relation for the triangle $OSB$ from the previous
\ref{pravilnaTlakovanja} izreka becomes:
$\frac{360^0}{2n}+\frac{360^0}{2m}+90^0<180^0$ and then instead of
relation \ref{teselRelEvk} we get:
 \begin{eqnarray}
(n-2)(m-2)>4. \label{teselRelHyp}
\end{eqnarray}
 This inequality has infinitely many solutions in the set $\mathbb{N}^2$, which means
that we have in hyperbolic geometry  infinitely many regular tilings.
Two of them are for example $(3,7)$ and $(4,5)$ (Figure
\ref{sl.skl.3.9.2H.pic}\footnote{http://math.slu.edu/escher/index.php/Category:Hyperbolic-Tessellations}).
In the latter, five squares touch around one vertex. This is
possible because in hyperbolic geometry the interior angle of a square is always
sharp and is not constant. It turns out that the square with the longer side
has a smaller interior angle. It is possible to choose such a side of the square
that the interior angle is equal to $\frac{360^0}{5} =72^0$, which just
corresponds to the tiling $(4,5)$.

\begin{figure}[!htb]
\centering
\includegraphics[width=0.413\textwidth]{whyptess1.eps}\hspace*{4mm}
 \includegraphics[width=0.387\textwidth]{whyptess.eps}
\caption{} \label{sl.skl.3.9.2H.pic}
\end{figure}

In elliptic geometry, where the sum of the angles in a triangle is always
greater than $180^0$, the aforementioned relation for triangle $OSB$ becomes:
$\frac{360^0}{2n}+\frac{360^0}{2m}+90^0>180^0$, or:
 \begin{eqnarray}
(n-2)(m-2)<4. \label{teselRelElipt}
\end{eqnarray}
This equation has solutions $(3,3)$, $(4,3)$,
$(3,4$), $(5,3)$ and $(3,5)$ in the set $\mathbb{N}^2$. Because elliptic geometry is realized
as a model on a sphere, these solutions represent tessellations of the sphere with spherical
polygons. The sides of these polygons are arcs of great circles of the sphere.
If in Euclidean space with distances we connect the appropriate vertices of these
tessellations, we get the so-called \index{pravilni!poliedri} \pojem{regular
polyhedra} (Figure
\ref{sl.skl.3.9.2E.pic}\footnote{http://www.upc.edu/ea-smi/personal/claudi/web3d/}):
\pojem{regular tetrahedron}, \pojem{cube} (or \pojem{regular
hexahedron}), \pojem{regular octahedron}, \pojem{regular dodecahedron}
and \pojem{regular icohedron}. For example, $(4,3)$ would represent a cube,
in which three squares (regular 4-gon) meet at one point.


\begin{figure}[!htb]
\centering
 \includegraphics[bb=0 0 11cm 6cm]{wpoliedri.eps}
\caption{} \label{sl.skl.3.9.2E.pic}
\end{figure}


Let's go back to the Euclidean plane. If we allow the possibility that in
covering the plane we use more (finitely many) types of regular
polygons or more types of cells that are arranged the same way at each vertex, then in addition to the three regular
tessellations from the statement \ref{pravilnaTlakovanja} there are eight so-called
\index{tlakovanja!Arhimedova} \pojem{Archimedean tessellations}
\footnote{The proof of this statement was carried out by the German astronomer, mathematician and
physicist \index{Kepler, J.} \textit{J. Kepler} (1571--1630).} (Figure
\ref{sl.skl.3.9.2A.pic}\footnote{http://commons.wikimedia.org/wiki/File\%3AArchimedean-Lattice.png}).



\begin{figure}[!htb]
\centering
 \includegraphics[bb=0 0 10cm 7.7cm]{Archimedean.eps}
\caption{} \label{sl.skl.3.9.2A.pic}
\end{figure}

In addition to regular and Archimedean tilings, there are also other
tilings with polygons that are not regular. The simplest example
is the tiling with congruent parallelograms (Figure
\ref{sl.skl.3.9.3a.pic}), which we get if we deform the regular
tiling $(4,4)$ so that instead of squares in one vertex, four
parallelograms meet. This tiling is determined by the grid
of two sets of parallel lines.

\begin{figure}[!htb]
\centering
\input{sl.skl.3.9.3aa.pic}
\input{sl.skl.3.9.3aaa.pic}
\caption{} \label{sl.skl.3.9.3a.pic}
\end{figure}

\begin{figure}[!htb]
\centering
\input{sl.skl.3.9.3bb.pic}
\input{sl.skl.3.9.3cc.pic}
\caption{} \label{sl.skl.3.9.3bc.pic}
\end{figure}

If we divide all parallelograms into
two triangles with diagonals that have the same direction, we get a tiling of the plane with congruent triangles
(Figure \ref{sl.skl.3.9.3a.pic}). The triangle (basic cell) can be
arbitrary, because two such (congruent) triangles can always be connected by a common
side into a parallelogram and thus get a tiling with
parallelograms. Tiling with arbitrary congruent triangles is
a generalization of the regular tiling $(3,6)$ with regular triangles.


%\vspace*{-1mm}

As special cases of tiling with parallelograms, we get tiling with
rectangles and tiling with rhombuses (Figure \ref{sl.skl.3.9.3bc.pic}).
We will prove an even more general
statement, which may not be so obvious. It is possible that
there is also a tiling with arbitrary congruent quadrilaterals.

%\vspace*{-1mm}



        \bzgled
        Let $ABCD$ be an arbitrary quadrilateral. Prove that it is
        possible to tessellate the plane with the cell $ABCD$ so that each vertex
         is surrounded by four such quadrilaterals.
        \ezgled


\begin{figure}[!htb]
\centering
\input{sl.skl.3.9.4.pic}
\caption{} \label{sl.skl.3.9.4.pic}
\end{figure}

\textbf{\textit{Proof.}}  (Figure
\ref{sl.skl.3.9.4.pic})
 Let $ABCD$ be an arbitrary quadrilateral in
the plane with internal angles $\alpha$, $\beta$, $\gamma$ and $\delta$.
Points $O$ and $S$ are the centers of its sides $AB$ and $BC$. With central
reflections
 $\mathcal{S}_O$ and $\mathcal{S}_S$ (for the definition of central reflection
 see section \ref{odd6SredZrc})
 the  quadrilateral  $ABCD$ is mapped into quadrilateral
$BAC_1D_1$ and $A_2CBD_2$. Here $\angle ABD_1\cong\alpha$ and
$CBD_2\cong\gamma$, therefore $\angle D_1BD_2\cong\delta$. Because
$BD_1\cong AD$ and $BD_2\cong CD$, there exists a point $E$,  such
that quadrilaterals $D_1ED_2B$ and $ABCD$ are congruent. So around point
$B$ four quadrilaterals intersect, all of which are congruent to quadrilateral
$ABCD$. We can continue the process of tiling the plane, if
we use central symmetry with respect to the centers of sides
of newly formed quadrilaterals.
 \kdokaz

 Therefore there exist tilings of the plane with any triangle and any
  quadrilateral. It is clear that for any pentagon, hexagon,...
  this property does not hold.
For a regular hexagon there exists a regular tiling $(6,3)$. If in
the previous statement we put together two appropriate adjacent quadrilaterals,
we get a tiling of the plane with congruent hexagons, which are not necessarily
regular, but are always centrally symmetric.


 %_______________________________________________________________________________
 \poglavje{Sets of Points in a Plane. Sylvester's Problem}
 \label{odd3Silvester}

In this section we will investigate problems related to sets of points in the plane and the lines determined by these points. At the beginning we will consider some consequences of the first two groups of axioms (incidence and order). First we will define new concepts. Let $\mathfrak{T}$ be a set of $n$ ($n>2$) points in the plane. With $\mathcal{P}(\mathfrak{T})$ we denote the set of all lines, each of which goes through at least two points from the set $\mathfrak{T}$ (Figure \ref{sl.skl.3.10.1.pic}). Because the set $\mathfrak{T}$ contains at least two points, from the axioms of incidence it follows that the set $\mathcal{P}(\mathfrak{T})$ is not empty. The question arises, how many lines are in the set $\mathcal{P}(\mathfrak{T})$.

\begin{figure}[!htb]
\centering
\input{sl.skl.3.10.1.pic}
\caption{} \label{sl.skl.3.10.1.pic}
\end{figure}



        \bizrek \label{stevPremic}
        Let $\mathfrak{T}$ be a set of $n$ ($n>2$) points in the plane such that
        no three of them are collinear. Then the number of lines of the set $\mathcal{P}(\mathfrak{T})$ is equal to
         $$\frac{n(n-1)}{2}.$$
         \eizrek

\begin{figure}[!htb]
\centering
\input{sl.skl.3.10.2.pic}
\caption{} \label{sl.skl.3.10.2.pic}
\end{figure}

\textbf{\textit{Proof.}} Through each of the $n$ points from the set $\mathfrak{T}$ there are exactly $n -1$ lines from the set $\mathcal{P}(\mathfrak{T})$. Because we count each line in this way twice (Figure \ref{sl.skl.3.10.2.pic}), we have to divide by 2. So there are exactly $\frac{n(n-1)}{2}$ lines in the set $\mathcal{P}(\mathfrak{T})$.
 \kdokaz

The previous statement can also be solved in the following way: through the first point there are $n -1$ lines, through the second point $n - 2$ lines (one less, because the line determined by these two points is not counted twice), $n - 3$ lines through the third point and so on until one line through the penultimate point. This is a total of $(n -1) + (n - 2) +\cdots+1$ lines. Of course, this is again equal to $\frac{n(n-1)}{2}$. If we take $n-1= k$, we get the known formula for the sum of the first $k$ natural numbers:
 $$1+ 2+ \cdots + k=\frac{k(k+1)}{2}.$$

 From the previous statement we can derive the formula for the number of diagonals of an arbitrary $n$-gon.



            \bizrek
            If $D_n$ is the number of diagonals of an arbitrary $n$-gon,
            then
             $$D_n=\frac{n( n-3)}{2}.$$
            \eizrek

\begin{figure}[!htb]
\centering
\input{sl.skl.3.10.3.pic}
\caption{} \label{sl.skl.3.10.3.pic}
\end{figure}

\textbf{\textit{Proof.}} We mark with $\mathfrak{O}$ the set of all vertices of an arbitrary $n$-gon.
  The number of diagonals is equal to the number of all lines
from the set $\mathcal{P}(\mathfrak{O})$ (statement \ref{stevPremic})
decreased by the number of its sides (Figure
\ref{sl.skl.3.10.3.pic}). Therefore:
 $$D_n=\frac{n(n- 1)}{2}-n=\frac{n^2-3n}{2}=\frac{n(n-3)}{2},$$ which was to be proven. \kdokaz

We mention that the formula for the number of diagonals of an $n$-gon could also be derived directly - with a similar treatment as in the proof of statement \ref{stevPremic}. From each of the $n$ vertices of an $n$-gon we can draw $n - 3$ diagonals. In this way we count each diagonal twice, so we have to divide by 2 and we get the previous formula.

Statement \ref{stevPremic} referred to the number of lines determined by a set of points in a plane that are in such a position that no three of them are collinear. In the next example we will check what happens if we add new conditions.

\bzgled
            Let $\mathfrak{T}$ be a set of $n$ ($n > 2$) points in the plane which are
            in such a position that $m$ ($m < n$) of them lies on the same line, but otherwise
            no other three points are collinear. What is the number of lines
            in the set $\mathcal{P}(\mathfrak{T})$?
            \ezgled

\begin{figure}[!htb]
\centering
\input{sl.skl.3.10.4.pic}
\caption{} \label{sl.skl.3.10.4.pic}
\end{figure}

\textbf{\textit{Solution.}}
 (Figure \ref{sl.skl.3.10.4.pic})

 Without the additional condition, there would be $\frac{n(n -1)}{2}$ lines in the set $\mathcal{P}(\mathfrak{T})$
  (statement \ref{stevPremic}).
  The additional condition in the problem reduces this
number by one less than the number of lines determined
by the set of $m$ points in general position. This is because otherwise
these lines would be counted multiple times. So the number of lines in the set
$\mathcal{P}(\mathfrak{T})$ is equal to:
$$\frac{(n-1)}{2}-\frac{(m -1)}{2}+1.$$
 \kdokaz

The next simple example will be an introduction to the very interesting problem of the relationship
between the set of points $\mathfrak{T}$ and the set of all lines
$\mathcal{P}(\mathfrak{T})$, which this set of points determines.


            \bzgled
            Construct nine points lying on ten lines
            in such a way, that each of those ten lines contains exactly three of these nine points\footnote{\index{Newton, I.}
            \textit{I. Newton} (1642--1727), znani angleški
            matematik
            in fizik, ki je zastavil ta problem v obliki:
            ‘‘How can you plant 9 trees in a garden with 10 rows and each row containing exactly 3 trees?’’}.
            \ezgled

\begin{figure}[!htb]
\centering
\input{sl.skl.3.10.5.pic}
\caption{} \label{sl.skl.3.10.5.pic}
\end{figure}

\textbf{\textit{Proof.}} One of the possibilities is the following. Let $ABCD$
be an arbitrary rectangle, $P$ and $Q$ be the centers of sides $AB$ and $CD$, $S$
be the intersection of the diagonals, $K$ be the intersection of lines $AP$ and $DQ$, and $L$
be the intersection of lines $BP$ and $CQ$ (Figure \ref{sl.skl.3.10.5.pic}). Because
the rectangle $ABCD$ is a rectangle, the points $P$, $Q$ and $S$  are collinear. So are
the points $K$, $L$ and $S$ (points $K$ and $L$ are the centers of rectangles
$AQPD$ and $QBCP$). So we have nine points $A$, $B$, $C$, $D$, $P$,
$Q$, $S$, $K$ and $L$, of which three lie on each of the ten lines $AB$,
$CD$, $PQ$, $KL$, $AC$, $BD$,
$PA$, $PB$, $QC$ and $QD$.
 \kdokaz

 If in the previous example we denote the set of nine points with $\mathfrak{T}$, we see that the set of ten lines is not the set $\mathcal{P}(\mathfrak{T})$.
  In the set $\mathcal{P}(\mathfrak{T})$ we would have  sixteen
  lines - our ten and also the additional lines $AD$, $BC$ $DL$, $AL$, $CK$ and $BK$.
  But each of these six lines contains
only two points. So the condition that they contain
exactly three points of the initial set is not fulfilled. It is now logical to ask
the following question: Is it possible in a plane to set a finite
set of non-collinear points $\mathfrak{T}$ so that each line from the set
$\mathcal{P}(\mathfrak{T})$ contains exactly three points from the set
$\mathfrak{T}$? It is clear that it is possible if we require
that each line from $\mathcal{P}(\mathfrak{T})$ contains exactly two
points from $\mathfrak{T}$. The most simple example for this are the vertices
of a triangle and its altitudes, or any set of points from
the statement \ref{stevPremic}. The aforementioned problem for three points is not so
simple, and we will find the answer in the continuation. We mention that
the answer is negative. Even more - the answer is negative even if we require that each line from $\mathcal{P}(\mathfrak{T})$ contains
at least three points from  $\mathfrak{T}$. First, we prove one lemma
(an auxiliary statement).

\bizrek \label{SylvesterLema}
           Let $\mathfrak{T}$ be a finite set of points in the plane which do not all lie on the same line
            and $\mathcal{P}=\mathcal{P}(\mathfrak{T})$. If $T_0\in \mathfrak{T}$ and $p_0\in \mathcal{P}$ ($T_0\notin p_0$)
             are such that they determine the minimum distance, i.e.
            $$(\forall T\in \mathfrak{T})(\forall p\in \mathcal{P})
              ( T\notin p\Rightarrow d(T,p)\geq d(T_0,p_0)),$$
            then the line $p_0$ contains exactly two points from the set $\mathfrak{T}$.
            \eizrek

\begin{figure}[!htb]
\centering
\input{sl.skl.3.10.6.pic}
\caption{} \label{sl.skl.3.10.6.pic}
\end{figure}

\textbf{\textit{Proof.}} We assume the opposite. Let the line $p_0$
contain at least three different points $A$, $B$ and $C$ from the set
$\mathfrak{T}$ (Figure \ref{sl.skl.3.10.6.pic}). We mark with $T'_0$ the orthogonal projection of the point $T_0$
on the line $p_0$. If the point $T'_0$ differs from the points $A$, $B$ and
$C$, then at least two of these three points (let it be $B$ and $C$) are on the line $p_0$ on the same
side of the point $T'_0$. Without loss of generality, let
$\mathcal{B}(T'_0,B,C)$. Because $T_0,C \in \mathfrak{T}$, then the line
$q= CT_0$  ($q \neq p_0$, because $T_0\notin p_0$) belongs to the set
$\mathcal{P}$. Let $B'$ be the orthogonal projection of the point  $B$ on
the line $q$. It is not difficult to prove that in this case it holds:
 $$d(B,q)=|BB'|<T_0T'_0=d(T_0,p_0).$$
 The last relation is
in contradiction with the assumption, therefore the line $p_0$ contains exactly two
points.

If  $T_0$ lies on one of the points $A$, $B$ or $C$, the proof
 is similar.
 \kdokaz

 Now we will prove the predicted statement.

\bizrek
Let $\mathfrak{T}$ be a finite set of points in the plane which do not all lie on the same line.
Then there is a line that contains exactly two points from the set
$\mathfrak{T}$
(\textit{Sylvester's\footnote{\index{Sylvester, J. J.} \index{Karamata, J.}
\index{Erdös, P.} \index{Gallai, T.} \index{Kelly, L. M.}
\index{Coxeter, H. S. M.}
Angleški matematik \textit{J. J. Sylvester} (1814--1897)
je zastavil ta problem že leta 1893,
ki takrat ni bil rešen in so nanj pozabili.
Šele čez štirideset let (leta 1933) sta sta ga ponovno ‘‘obudila’’ Madžarski matematik
\textit{P. Erdös} (1913--1996) in srbski matematik \textit{J.
Karamata} (1902--1967), madžarski matematik
\textit{T. Gallai} (1912–-1992) pa ga je istega leta rešil. Nato je bilo
objavljenih več različnih dokazov rešitve tega problema - najbolj eleganten
iz leta 1948, ki ga je podal ameriški matematik \textit{L. M. Kelly}
(1914-–2002), je prikazan tukaj. Leta 1961 je veliki kanadski
geometer \textit{H. S. M. Coxeter} (1907--2003) dokazal to
trditev brez uporabe skladnosti - le kot posledico aksiomov
incidence in urejenosti.} problem}).
\eizrek

\begin{figure}[!htb]
\centering
\input{sl.skl.3.10.7.pic}
\caption{} \label{sl.skl.3.10.7.pic}
\end{figure}

\textbf{\textit{Proof.}}
Let $\mathcal{P}=\mathcal{P}(\mathfrak{T})$. Because the set
$\mathfrak{T}$ is finite, the set $\mathcal{P}$ is also finite, and then
the set of all distances (Figure \ref{sl.skl.3.10.7.pic}) is finite:
$$\mathcal{D} = \{d(T,p);\hspace*{1mm}
T\in \mathfrak{T},\hspace*{1mm} p\in \mathcal{P},\hspace*{1mm} T\notin p\}.$$
 Because $\mathcal{D}$ is a finite set of positive real numbers, it has its
minimum element $d(T_0,p_0)$ (no distance from
$\mathcal{D}$ is smaller), which is achieved for some point $T_0\in
\mathfrak{T}$ and some line $p_0\in \mathcal{P}$. From the definition
of the set $\mathcal{D}$ it follows that $T_0\notin p_0$. By the previous lemma
\ref{SylvesterLema} there are exactly two points from the set
$\mathfrak{T}$ on the line $p_0$.
 \kdokaz

 We will finish this section with two interesting examples.



            \bizrek
            Let $\mathfrak{T}$ be a finite set of points,
            such that distances between two points of this set are all different.
            If we connect each point with a line segment to its nearest point, then none of the points will be
            directly connected to more than five points from this set\footnote{Polish mathematician, astronomer and physicist
              \index{Steinhaus, H.}\textit{H. Steinhaus} (1887--1972)
              wrote this problem in the form:
              ‘‘Every city in the map of Europe is connected... ’’}.
            \eizrek

\begin{figure}[!htb]
\centering
\input{sl.skl.3.10.8.pic}
\caption{} \label{sl.skl.3.10.8.pic}
\end{figure}

\textbf{\textit{Proof.}}
First, we determine that two arbitrary points $X$ and $Y$ can be connected
in one of three different
ways (Figure \ref{sl.skl.3.10.8.pic}):

 \textit{1)} the point $X$ is the nearest point to $Y$, but not vice versa;

 \textit{2)} the
point $Y$ is the nearest point to $X$, but not vice versa;

\textit{3)} the point $X$ is the nearest point to $Y$ and vice versa - the point $Y$
is the nearest point to $X$.

Every point has only one point closest to it, but that point can be connected to multiple points, which are closest to it. It needs to be proven that there are no more than five such points. We assume the opposite. Let $P$ be a point of the given set $\mathfrak{T}$ and is connected to at least six points $A$, $B$, $C$, $D$, $E$, and $F$ by distance. Without loss of generality, we can assume that they are arranged (or we can label them) so that $ABCDEF$ is a hexagon (Figure \ref{sl.skl.3.10.8.pic}). One of the points $A$, $B$, $C$, $D$, $E$, or $F$ is closest to the point $P$, let it be the point $A$. So it holds: $PA < PB, PC, PD, PE, PF$. Because the points $B$, $C$, $D$, $E$, and $F$ are also connected to the point $P$, it means that the point $P$ is closest to each of them (and not the other way around). Based on this, first $BA > BP > PA$, so $\angle APB$ is the largest angle in the triangle $APB$ and is therefore larger than $60^0$. Similarly, $CB > BP,BC$, so $\angle BPC > 60^0$. The same would hold for the angles $CPD$, $DPE$, $EPF$, and $EPA$, but that is not possible, because their sum is always equal to or even greater than $360^0$, regardless of whether $P$ is an inner or outer point of the hexagon $ABCDEF$. Therefore, the point $P$ is connected to no more than five points.
 \kdokaz


            \bzgled
            What is the maximum number of regions in the plane that can be
            divided by $n$ lines\footnote{Ta problem je rešil švicarski geometer
            \index{Steiner, J.} \textit{J. Steiner} (1796--1863).}?
            \ezgled

\begin{figure}[!htb]
\centering
\input{sl.skl.3.10.9.pic}
\caption{} \label{sl.skl.3.10.9.pic}
\end{figure}

\textbf{\textit{Solution.}}  We will describe the process of designing such
$n$ lines (Figure \ref{sl.skl.3.10.9.pic}). If $n = 1$, or
if there is only one line, the plane is divided into two areas. Two
lines, if they are not parallel, divide the plane into four areas. If
we add a third line $p_3$, which is not parallel to them and does not go through
their common point, it intersects the initial two lines in two points. These
two points divide the line $p_3$ into three parts, each of which is in
one of the three previous four areas of the plane. So with line
$p_3$ we get three more parts of the plane, a total of seven. We continue the process. If $n -1$ lines divide the plane into $k$ parts, by
adding the $n$-th line $p_n$ (which is not parallel to any of the previous $n -1$ lines and does not contain any of their intersections),
we first get $n -1$ intersections on that line, then $n$ of its
parts or $n$ new areas of the given plane. So the maximum possible
number of areas for $n$ lines is equal to:
\begin{eqnarray*}
2+2+3+\cdots+n&=&1+1+2+3+\cdots+n=\\
&=&1+\frac{n(n+1)}{2}=\\&=&\frac{n^2+n+2}{2}.
\end{eqnarray*}
A formal proof of this fact could be derived by mathematical induction.
 \kdokaz


 %_______________________________________________________________________________
 \poglavje{Helly's Theorem}
\label{odd3Helly}

The next important statement is a consequence of only the first two groups of axioms
or the axioms of incidence and the axioms of order. We will also use it
in tasks related to consistency.

\bizrek \label{Helly}
            Let $\Phi_1$, $\Phi_2$, ... , $\Phi_n$ ($n \geq 4$) be convex sets in the plane.
            If every three of these sets have a common point, then all $n$ sets have a common point
            \index{izrek!Hellyjev}(Helly's theorem\footnote{Austrian
            mathematician  \index{Helly, E.} \textit{E. Helly} (1884--1943)
             discovered this statement in the general case of $n$-dimensional space $\mathbb{E}^n$
              in 1913, but published it only in 1923. Alternative proofs were
              meanwhile given by Austrian mathematician \index{Radon, J. K. A.}
              \textit{J. K. A. Radon} (1887-–1956) in 1921 and
              Hungarian mathematician \index{Kőnig, D.}
              \textit{D. Kőnig} (1884–-1944) in 1922.}).
            \eizrek

\begin{figure}[!htb]
\centering
\hspace*{10mm}
\input{sl.skl.3.11.1.pic}
\caption{} \label{sl.skl.3.11.1.pic}
\end{figure}

\textbf{\textit{Proof.}} We will prove this by induction on $n$.

\textit{(A)} Let $n = 4$ and (Figure
\ref{sl.skl.3.11.1.pic}):
\begin{itemize}
  \item $P_4\in \Phi_1 \cap \Phi_2 \cap \Phi_3$,
  \item $P_3\in \Phi_1 \cap \Phi_2 \cap \Phi_4$,
  \item $P_2\in \Phi_1 \cap \Phi_3 \cap \Phi_4$,
  \item $P_1\in \Phi_2 \cap \Phi_3 \cap \Phi_4$.
\end{itemize}
We will prove that there exists a point that lies in each of the figures $\Phi_1$,
$\Phi_2$, $\Phi_3$ and $\Phi_4$. Based on the mutual position
of points $P_1$, $P_2$, $P_3$ and $P_4$ we will consider only two most
general cases (the proof in the other cases
is similar).

\textit{1)} A quadrilateral, determined by points $P_1$, $P_2$, $P_3$ and
$P_4$, is non-convex. In this case, one of the points $P_1$, $P_2$,
$P_3$ and $P_4$ is an inner point of the triangle determined by the remaining
three points. Without loss of generality, let $P_4$ be the inner point
of the triangle $P_1P_2P_3$. The vertices of this triangle lie in the shape
$\Phi_4$. Because $\Phi_4$ is a convex shape, all sides and inner points of the triangle $P_1P_2P_3$, as well as the point $P_4$, lie in it. In this case, the point $P_4$ is the common point of shapes
$\Phi_1$, $\Phi_2$, $\Phi_3$ and $\Phi_4$.

\textit{2)} The quadrilateral determined by points $P_1$, $P_2$, $P_3$ and $P_4$ is convex. Without loss of generality, let its diagonals be $P_1P_2$ and $P_3P_4$. Because the quadrilateral is convex, its diagonals intersect in a point $S$. Given shapes are convex, so from $P_1, P_2\in \Phi_3,\Phi_4$ it follows that the diagonal $P_1P_2$ lies entirely in shapes $\Phi_3$ and $\Phi_4$. Analogously, from $P_3, P_4\in \Phi_1,\Phi_2$ it follows that the diagonal $P_3P_4$ lies entirely in shapes $\Phi_1$ and $\Phi_2$. The point $S$, which lies on both diagonals
$P_1P_2$ and $P_3P_4$, lies in all four shapes $\Phi_1$, $\Phi_2$,
$\Phi_3$ and $\Phi_4$.

With this we have proven that the statement is true for $n=4$.

\textit{(B)} Let's now assume that the statement is true for $n = k$ ($k\in
\mathbb{N}$ and $k>4$).
 We shall prove that the statement is also true for
$n = k +1$. Let $\Phi_1$, $\Phi_2$, $\ldots$ , $\Phi_{k-1}$,
$\Phi_k$ and $\Phi_{k+1}$ be such figures that every triplet of these figures
has at least one common point. Let $\Phi'=\Phi_k\cap\Phi_{k+1}$. We shall first prove that every triplet of figures $\Phi_1$, $\Phi_2$ ,$\ldots$,
$\Phi_{k-1}$, $\Phi'$ has a common point. For triplets of figures from
$\Phi_1$, $\Phi_2$, $\ldots$, $\Phi_{k-1}$ this is already fulfilled according to the assumption. Without loss of generality, it is enough to prove that
figures $\Phi_1$, $\Phi_2$ and $\Phi'=\Phi_k\cap\Phi_{k+1}$
have a common point. This is true  (based on the proven example for $n = 4$),
because every triplet of figures from $\Phi_1$, $\Phi_2$,  $\Phi_k$ and
$\Phi_{k+1}$  has a common point. From the induction assumption
(for $n=k$) it follows that figures $\Phi_1$, $\Phi_2$, $\ldots$,
$\Phi_{k-1}$, $\Phi'$  have a common point, which at the same time lies in
each of figures $\Phi_1$, $\Phi_2$, $\ldots$,  $\Phi_{k-1}$,
$\Phi_k$, $\Phi_{k+1}$.
 \kdokaz

In the continuation we shall consider some consequences of Helly's theorem.



            \bzgled
           Let $\alpha_1$, $\alpha_2$, $\cdots$, $\alpha_n$
            ($n > 3$) be half-planes covering a plane $\alpha$.
            Prove that there are three of these half-planes that also cover the plane $\alpha$.
            \ezgled

\textbf{\textit{Proof.}} Let $\beta_1$, $\beta_2$, $\cdots$,
$\beta_n$ be open half-planes, which are determined by the half-planes
$\alpha_1$, $\alpha_2$, $\cdots$, $\alpha_n$ as complementary
half-planes with respect to the plane $\alpha$ or
$\beta_k=\alpha\setminus\alpha_k$, $k\in\{1,2,\ldots,n\}$.
For every point $X$ of the plane $\alpha$ and every
$k\in\{1,2,\ldots,n\}$ the equivalence holds:
 $$X\in \alpha_k \Leftrightarrow X\notin \beta_k.$$
Assume the contrary, that none of the three half-planes $\alpha_1$,
$\alpha_2$, $\cdots$, $\alpha_n$ covers the plane $\alpha$. This
means that for every triple of them there is a point in the plane $\alpha$,
which does not lie on any of them, or for every triple of the half-planes
$\beta_1$, $\beta_2$, $\cdots$, $\beta_n$ there is a point of the plane
$\alpha$, which lies on each of them. Because the half-planes are convex
shapes, from Helly's theorem \ref{Helly} it follows that there is a point
$X$, which lies  on each of the half-planes $\beta_1$, $\beta_2$, $\cdots$,
$\beta_n$. This point therefore lies in  the plane $\alpha$, but does not lie in
any of the half-planes $\alpha_1$, $\alpha_2$, $\cdots$, $\alpha_n$,
which is in contradiction with the basic assumption that the half-planes $\alpha_1$,
$\alpha_2$, $\cdots$, $\alpha_n$ cover the plane $\alpha$.
 Therefore, there is at least one triple of the
half-planes $\alpha_1$, $\alpha_2$, $\cdots$, $\alpha_n$, which
covers the plane $\alpha$.
 \kdokaz



        \bzgled \label{lemaJung}
        If for every three of $n$ ($n > 3$) points of a plane
        there is such a circle with radius $r$ containing these three points, then it exists
        a circle of equal radius containing all these $n$ points.
        \ezgled

\begin{figure}[!htb]
\centering
\input{sl.skl.3.11.2.pic}
\caption{} \label{sl.skl.3.11.2.pic}
\end{figure}

\textbf{\textit{Proof.}} (Figure \ref{sl.skl.3.11.2.pic})

Let $A_1$, $A_2$,$\ldots$, $A_n$ be points with given properties. With
$\mathcal{K}_i$ ($i\in\{1,2,\ldots,n\}$) we denote circles with centers $A_i$ and with
radius $r$. Let $A_p$, $A_q$ and $A_l$ be any points from
the set $\{A_1, A_2,\ldots, A_n\}$. By assumption, there exists a circle with
radius $r$, which contains these three points. We denote the center of this circle
with $O$. From this it follows that $|OA_p|, |OA_q|, |OA_l|\leq r$, which means that
point $O$ lies in each of the circles $\mathcal{K}_p$, $\mathcal{K}_q$ and $\mathcal{K}_l$. Therefore,
each three of the circles $\mathcal{K}_1$, $\mathcal{K}_2$,$\ldots$, $\mathcal{K}_n$ have at least one
common point. Because the circles are convex figures (statement \ref{KrogKonv}), by
Helly's theorem there exists a point $S$, which lies in each of the circles
$\mathcal{K}_1$, $\mathcal{K}_2$,$\ldots$, $\mathcal{K}_n$. From this it follows that $\mathcal{K}(S, r)$ is the desired
circle, since $|SA_1|, |SA_2|,\ldots |SA_n|\leq r$.
 \kdokaz

 An interesting consequence of the last statement \ref{lemaJung} will be given in section
 \ref{odd7Pitagora} (statement \ref{Jung}).

%________________________________________________________________________________
\naloge{Exercises}

\begin{enumerate}

 \item Let $S$ be a point, which lies in the angle $pOq$, and to point $A$ and $B$ the orthogonal projection of
 point $S$ on the sides $p$ and $q$ of this
angle. Prove that $SA\cong SB$ if and only when the line
$OS$ is the angle bisector of the angle $pOq$.

\item Prove that the sum of the diagonals of a convex quadrilateral is greater than
the sum of its opposite sides.

  \item Prove that in every triangle
  there is at most one side shorter than the corresponding altitude.

  \item Let $AA_1$ be the altitude of triangle $ABC$. Prove
   that of the two angles, which altitude $AA_1$ determines with
sides $AB$ and $AC$, the larger one is the one, which altitude $AA_1$ determines with
the shorter side.

  \item Let $BB_1$ and $CC_1$ be the altitudes of triangle $ABC$ and
  $AB<AC$.
  Prove that $BB_1<CC_1$.

\item Let $a$, $b$ and $c$ be the sides, $t_a$, $t_b$ and $t_c$
   the corresponding centroids, and $s$ the semi-perimeter of any triangle.
Prove that:
 \begin{enumerate}
  \item $s <  t_a  + t_b +  t_c  < 2s$;
  \item $t_a + t_b + t_c  >  \frac{3}{4}(a + b + c)$.
 \end{enumerate}

\item Let $p$ be a line that is parallel to a circle $k$. Prove that
all points of this circle are on the same side of the line $p$.

\item If the circle $k$ lies in a convex figure $\Phi$, then the
circle determined by this circle also lies in this figure. Prove it.

\item Let $p$ and $q$ be two different tangents to the circle $k$ that
touch it in points $P$ and $Q$. Prove the equivalence: $p \parallel q$
exactly when $AB$ is the diameter of the circle $k$.

\item If $AB$ is the chord of the circle $k$, then the intersection of
the line $AB$ and the circle determined by the circle $k$ is equal to
this chord. Prove it.

\item Let $S'$ be the orthogonal projection of the center $S$ of the
circle $k$ onto the line $p$. Prove that $S'$ is an external point of
this circle exactly when the line $p$ does not intersect the circle.

\item Let $V$ be the altitude of the triangle $ABC$, for which $CV \cong
AB$. Determine the size of the angle $ACB$.

\item Let $CC'$ be the altitude of the right triangle $ABC$ ($\angle
ACB = 90^0$). If $O$ and $S$ are the centers of the inscribed circles
of the triangles $ACC'$ and $BCC'$, then the altitude of the internal
angle $ACB$ is perpendicular to the line $OS$. Prove it.

\item Let $ABC$ be a triangle in which $\angle ABC = 15^0$ and $\angle
ACB = 30^0$. Let $D$ be such a point of the side $BC$ that $\angle BAD
= 90^0$. Prove that $BD = 2AC$.

\item Prove that there exists a pentagon that can be covered with
such pentagons that are congruent to it.

\item Prove that there exists a decagon that can be covered with
such decagons that are congruent to it.

\item In a plane, each point is painted red or black. Prove that
there exists a right triangle that has all its vertices the same
color.

\item Let $l_1,l_2,\ldots, l_n$ ($n > 3$) be arcs, which all lie on the same
circle. The central angle of each arc is at most $180^0$.
 Prove that there exists a point, which lies on each arc,
 if every three arcs have at least one common point.

%drugi del

\item
Let $p$ and $q$ be rectangles, which intersect in the point $A$. If
$B, B'\in p$, $C, C'\in q$, $AB\cong AC'$, $AB'\cong AC$,
$\mathcal{B}(B,A,B')$ and $\mathcal{B}(C,A,C')$, then
the rectangle on the line $BC$ through the point $A$ goes through the center
of the line $B'C'$. Prove.

\item
Prove that the altitudes of an inner angle of a rectangle, which is not
a square, intersect in points, which are the vertices of a square.

\item
 Prove that the altitudes of an inner angle of a parallelogram, which is not
a rhombus, intersect in points, which are the vertices of a rectangle. Prove also that the diagonals
of this rectangle are parallel to the sides of the parallelogram and are
equal to the difference of the adjacent sides of this parallelogram.

\item
Prove that the altitudes of two sides of a triangle
are perpendicular to each other.

\item Let $B'$ and $C'$ be the points of intersection of the altitudes from the vertices $B$ and $C$ of the triangle
$ABC$. Prove the equivalence $AB\cong AC \Leftrightarrow BB'\cong
CC'$.

\item Prove that a triangle is right,
if the center of the circle drawn through the triangle and its altitude point coincide.
Is a similar statement true for any two characteristic
points of this triangle?

\item Prove that a right triangle $ABC$ and a right triangle $A'B'C'$ are congruent
exactly when they have congruent altitudes $CD$ and $C'D'$, sides $AB$ and
$A'B'$ and angle $ACD$ and $A'C'D'$.

\item If $ABCD$ is a rectangle and $AQB$ and $APD$ are right triangles with the same orientation, then the line $PQ$
is congruent with the diagonal of this rectangle. Prove.

\item Let $BB'$ and $CC'$ be the altitudes of the triangle $ABC$ ($AC>AB$) and
 $D$ is such a point on the line segment $AB$, that $AD\cong AC$. The point
$E$ is the intersection of the line $BB'$ with the line, which goes through the point $D$ and is
parallel to the line $AC$. Prove that $BE=CC'-BB'$.

\item Let $ABCD$ be a convex quadrilateral, for which it holds that
 $AB\cong BC\cong CD$ and $AC\perp BD$. Prove that $ABCD$
 is a rhombus.

\item Let $BC$ be the base of an isosceles triangle $ABC$. If $K$ and
$L$ are such points, that $\mathcal{B}(A,K,B)$, $\mathcal{B}(A,C,L)$ and $KB\cong LC$, then
the center of the line $KL$ lies on the base $BC$. Prove.

\item Let $S$ be the center of the triangle $ABC$ of an inscribed circle.
A line, which goes through the point $S$ and is parallel to the side $BC$
of this triangle, intersects the sides $AB$ and $AC$ in succession in the points
$M$ and $N$. Prove that $BM+NC=NM$.

\item Let $ABCDEFG$ be a convex heptagon. Calculate the sum
of the convex angles, which are determined by the broken line $ACEGBDFA$.

\item Prove that the centers of the sides and the vertex of an altitude of an arbitrary triangle,
in which no two sides are congruent, are the vertices
of an isosceles trapezoid.

 \item Let $ABC$ be a right triangle with a right angle at the vertex $C$.
The points $E$ and $F$ shall be the intersections of the internal angle bisectors at
the vertices $A$ and $B$ with the opposite sides,  $K$ and $L$ shall
be the orthogonal projections of the points $E$ and $F$ on the hypotenuse of this
triangle. Prove that $\angle LCK=45^0$.


\item Let $M$ be the center of the side $CD$ of a square $ABCD$ and $P$ such a point
 on the diagonal $AC$, for which it holds that $3AP=PC$. Prove that $\angle BPM$
is a right angle.

 \item Let $P$, $Q$ and $R$ be the centers of the sides $AB$, $BC$ and $CD$
  of a parallelogram $ABCD$. The lines $DP$ and $BR$ shall intersect the line
$AQ$ in the points $K$ and $L$. Prove that $KL= \frac{2}{5} AQ$.

 \item  Let $D$ be the center of the hypotenuse $AB$ of a right triangle $ABC$ ($AC>BC$).
The points $E$ and $F$ shall be the intersections of the angle trisectors of
the sides $CA$ and $CB$ with a line, which goes through $D$ and is orthogonal
to the line $CD$. The point $M$ shall be the center of the line $EF$. Prove that
$CM\perp AB$.

\item Let $A_1$ and $C_1$ be the centers of sides $BC$ and $AB$ of triangle $ABC$.
 The internal angle bisector at point $A$ intersects the line
$A_1C_1$ at point $P$. Prove that $\angle APB$ is a right angle.

 \item Let $P$ and $Q$ be points on sides $BC$ and $CD$ of square $ABCD$,
  such that the line $PA$ is the angle bisector of angle $BPQ$. Determine the size of the angle
  $PAQ$.

\item Prove that the center of the circumscribed circle lies closest to the longest side
of the triangle.

 \item Prove that the center of the inscribed circle is closest to the vertex of
the largest internal angle of the triangle.

\item Let $ABCD$ be a convex quadrilateral. Find a point $P$, such that
the sum $AP+BP+CP+DP$ is minimal.

 \item The diagonals $AC$ and $BD$ of the trapezoid $ABCD$ with the base $AB$
intersect at point $O$ and $\angle AOB=60^0$. Points $P$, $Q$
and $R$ are in order the centers of the lines $OA$, $OD$ and $BC$. Prove that $PQR$ is a right triangle.

\item Let $P$ be an arbitrary internal point of triangle $ABC$, for which
 $\angle PBA\cong \angle PCA$. Points $M$ and $L$ are the orthogonal
projections of point $P$ on sides $AB$ and $AC$, point $N$ is
the center of side $BC$. Prove that $NM\cong
NL$\footnote{Predlog za MMO 1982 (SL 9.)).}.

\item Let $AD$ be the internal angle bisector at point $A$ ($D\in BC$) of triangle
$ABC$ and $E$ a point on side $AB$, such that $\angle
BDE\cong\angle BAC$. Prove that $DE\cong DC$.

\item Let $O$ be the center of square $ABCD$ and $P$, $Q$ and $R$ points,
 which divide its perimeter into three equal parts. Prove that the minimum of the sum $|OP|+|OQ|+|OR|$ is achieved, when one of these points
is the center of a side of the square.

\item There is a given finite number of lines that divide the plane into areas.
Prove that the plane can be colored with two colors, so that each area is colored with one color, and adjacent areas are always colored with different colors.


\item Draw a triangle $ABC$, if the following data are given (see labels in section \ref{odd3Stirik}):

 (\textit{a}) $\alpha$, $\beta$, $s$; \hspace*{2mm}
 (\textit{b}) $a-b$, $c$, $\gamma$; \hspace*{2mm}
 (\textit{c}) $a$, $\beta-\gamma$, $b-c$; \hspace*{2mm}

 (\textit{d}) $a$, $\beta-\gamma$, $b+c$; \hspace*{2mm}
 (\textit{e}) $b$, $c$, $v_a$; \hspace*{2mm}
 (\textit{f}) $b$, $v_a$, $v_b$; \hspace*{2mm}

(\textit{g}) $\alpha$, $v_a$, $v_b$; \hspace*{2mm}
 (\textit{h}) $c$, $a+b$, $\gamma$;
 (\textit{i}) $v_a$, $\alpha$, $\beta$; \hspace*{2mm}

 (\textit{j}) $b$, $a+c$, $v_c$; \hspace*{2mm}
 (\textit{k}) $b-c$, $v_b$, $\alpha$; \hspace*{2mm}
 (\textit{l}) $a$, $t_b$, $t_c$;\hspace*{2mm}
 (\textit{m}) $b$, $c$, $t_a$; \hspace*{2mm}

 (\textit{n}) $t_a$, $t_b$, $t_c$; \hspace*{2mm}
(\textit{o}) $c$, $v_a$, $l_a$; \hspace*{2mm}
 (\textit{p}) $c$, $v_a$, $t_b$;
 (\textit{r}) $b$, $l_a$, $\alpha$; \hspace*{2mm}

 (\textit{s}) $v_a$, $v_b$, $t_a$; \hspace*{2mm}
(\textit{t}) $t_a$, $v_b$, $b+c$; \hspace*{2mm}
 (\textit{u}) $a$, $b$, $\alpha-\beta$;

 \item Draw an isosceles triangle $ABC$, if the following are given:
        \begin{enumerate}
        \item the base and the sum of the leg and the height on the base,
        \item the circumference and the height on the base,
        \item both heights,
        \item the angle at the base and the segment of its altitude,
        \item the leg and the point of intersection of the corresponding altitude on it,
        \item the leg and the corresponding altitude.
        \end{enumerate}

 \item Draw a right triangle $ABC$ with a right angle at point $C$, if the following data are given:

(\textit{a}) $\alpha$, $a+b$, \hspace*{2mm}
 (\textit{b}) $\alpha$, $a-b$, \hspace*{2mm}
 (\textit{c}) $a$, $b+c$,

 (\textit{d}) $c$, $a+b$,\hspace*{2mm}
 (\textit{e}) $t_a$, $t_c$, \hspace*{2mm}
 (\textit{f}) $a$, $c-b$,

(\textit{g}) $a+v_c$, $\alpha$, \hspace*{2mm}
 (\textit{h}) $t_c$, $v_c$,\hspace*{2mm}
 (\textit{i}) $a$, $t_a$,\hspace*{2mm}
 (\textit{j}) $v_c$, $l_c$.

 \item Draw a rectangle $ABCD$, if given:
        \begin{enumerate}
        \item the diagonal and one side,
        \item the diagonal and the perimeter,
        \item one side and the angle between the diagonals,
        \item the perimeter and the angle between the diagonals.
        \end{enumerate}

 \item Draw a rhombus $ABCD$, if given:
        \begin{enumerate}
        \item a side and the sum of the diagonals,
        \item a side and the difference of the diagonals,
        \item one angle and the sum of the diagonals,
        \item one angle and the difference of the diagonals.
        \end{enumerate}

 \item Draw a parallelogram $ABCD$, if given:
        \begin{enumerate}
        \item one side and the diagonals,
        \item one side and the altitude,
        \item one diagonal and the altitude,
        \item side $AB$, the angle at point $A$ and the sum $BC+AC$.
        \end{enumerate}

 \item Draw a trapezoid $ABCD$, if given:
        \begin{enumerate}
        \item the bases, the leg and the smaller angle that is not adjacent to this leg,
        \item the bases and the diagonals,
        \item the bases and the angle at the longer base,
        \item the sum of the bases, the altitude and the angle at the longer base.
        \end{enumerate}

 \item Draw a deltoid $ABCD$, if given: the diagonal $AC$, which lies on the slanted side of the deltoid, $\angle CAD$ and the sum $AD+DC$."

\item Draw a quadrilateral $ABCD$, if it is given:
        \begin{enumerate}
        \item four sides and one angle,
        \item four sides and an angle between the opposite sides,
        \item three sides and an angle at the fourth side,
        \item the centers of three sides and a distance that is consistent and parallel to the fourth side.
        \end{enumerate}


\end{enumerate}



%%% Do tu pregledala tudi Ana.

% DEL 4 - - - - - - - - - - - - - - - - - - - - - - - - - - - - - - - - - - - - - - -
%________________________________________________________________________________
% SKLADNOST TRIKOTNIKOV IN KROŽNICA
%________________________________________________________________________________

  \del{Congruence and Circle} \label{pogSKK}


We have already looked at some of the properties of circles in the previous two chapters - certain properties of the radius, diameter, cord, the relationship between the circle and the line, and the properties of the tangent to the circle. We saw that for every triangle there is an inscribed and circumscribed circle. We proved that for regular polygons and some quadrilaterals (rectangle, square) there is a circumscribed circle, and for some there is also an inscribed circle. In this chapter we will look further into the properties of the circle that are a result of the congruence of triangles.

%________________________________________________________________________________
 \poglavje{Two Circles} \label{odd4DveKroz}

 We will carry out a similar analysis of the position of the circles and lines that we did in section \ref{odd3KrozPrem}, but this time with two circles in the same plane. We will assume that the two circles we are dealing with are in the same plane from now on.

 First, let's define some terms that relate to two circles. The line that goes through the centers of two circles is the
 \index{centrala dveh krožnic}\pojem{centrala} of those two circles.
 The distance between the centers of two circles is called the
 \index{središčna razdalja dveh krožnic}
  \pojem{središčna razdalja dveh krožnic}
 (Figure \ref{sl.skk.4.1.1.pic}).

\begin{figure}[!htb]
\centering
\input{sl.skk.4.1.1.pic}
\caption{} \label{sl.skk.4.1.1.pic}
\end{figure}

Circles in the same plane with the same
center (their central distance is equal to $0$)
we call
\index{circles!concentric}
 \pojem{concentric circles}
 (Figure \ref{sl.skk.4.1.1.pic}). If concentric circles have at least
 one common point, the circles are identical (coincide). If $X$ is a common point
 of concentric circles $k_1(S,r_1)$ and $k_2(S,r_2)$, it holds
 $|SX|=r_1=r_2$ or $r_1=r_2$, which means that the circles
 are identical. Concentric circles are either identical or have
 no common points. Similarly to the circle and the line, here we raise the question of how many common points different circles can have and what is their mutual position. We will begin
 with the following statement.


            \bizrek
            Two different circles lying in the same plane
            have at most two common points.
            \eizrek


\begin{figure}[!htb]
\centering
\input{sl.skk.4.1.2.pic}
\caption{} \label{sl.skk.4.1.2.pic}
\end{figure}

\textbf{\textit{Proof.}} We assume the opposite. Let $A$, $B$
and $C$ be three different common points of two circles $k(O,r_1)$ and
$l(S,r_2)$ (Figure \ref{sl.skk.4.1.2.pic}). These three points are not
collinear, which would mean that the line $AB$ intersects the circle (e.g.
$k$) in three different points, which according to izrek \ref{KroznPremPresek} is not
possible. But if  $A$, $B$ and $C$ are non-collinear points, they determine
the triangle $ABC$, which according to izrek \ref{SredOcrtaneKrozn} means that $O=S$, or both points are
located in the intersection of the perpendiculars of this
triangle. The radii are also equal, because $r_1=|OA|=|SA|=r_2$, so
the circles  $k(O,r_1)$ and $l(S,r_2)$ are identical - both represent
the circle circumscribed around the triangle $ABC$.
 \kdokaz

So, two different circles in the same plane can have two common points, one common point, or no common points. In the first case, we say that the \index{circles!intersect} \pojem{circles intersect}, in the second case the \index{circles!touch} \pojem{circles touch} in their \index{touching point!of two circles} \pojem{touching point}, and in the third case they are \index{circles!are non-intersecting}\pojem{non-intersecting circles} (Figure \ref{sl.skk.4.1.3.pic}).


\begin{figure}[!htb]
\centering
\input{sl.skk.4.1.3.pic}
\caption{} \label{sl.skk.4.1.3.pic}
\end{figure}


If the circles do not intersect, then the interior of at least one of these two circles is either in the interior or in the exterior of the other circle. This is a consequence of Theorem \ref{DedPoslKrozKroz}.

The line that is determined by the intersection
 of two circles that
intersect is called the \pojem{secant, which is the tangent of their common chord}. In connection with this, we prove the following theorem.



            \bizrek \label{KroznPresABpravokOS}
            If two circles intersect at two points $A$ and $B$,
            then the line containing the centres of the two circles is perpendicular to the line $AB$.
             \eizrek


\begin{figure}[!htb]
\centering
\input{sl.skk.4.1.4.pic}
\caption{} \label{sl.skk.4.1.4.pic}
\end{figure}

\textbf{\textit{Proof.}} Let $A$ and $B$ be the intersection points of the circles $k_1(S_1,r_1)$ and  $k_2(S_2,r_2)$ (Figure \ref{sl.skk.4.1.4.pic}).
 Because $S_1A\cong S_1B\cong r_1$ and $S_2A\cong S_2B\cong r_2$,
 the line $S_1S_2$  is the perpendicular bisector of the line segment $AB$ (Theorem \ref{simetrala}),
 so $S_1S_2\perp
 AB$.
  \kdokaz


 We will prove that the circles that touch have a common tangent at
their touching point.



            \bizrek \label{tangSkupnaDotikKrozn}
            Let $k_1$ and $k_2$ be different circles with centres $S_1$ and
            $S_2$ touching at a point $T$. Then:

            (i) $S_1$, $S_2$ and $T$  are collinear points;

(ii) the tangent of the circle $k_1$ at the point $T$ is at the same time the tangent of the circle $k_2$ at
            the same point.
            \eizrek



\begin{figure}[!htb]
\centering
\input{sl.skk.4.1.5.pic}
\caption{} \label{sl.skk.4.1.5.pic}
\end{figure}

\textbf{\textit{Proof.}}  (Figure \ref{sl.skk.4.1.5.pic}).

(\textit{i}) We assume that the points $S_1$, $S_2$ and $T$ are not
collinear. The line $S_1S_2$ divides the plane in which the circles
lie, into two half-planes. The half-plane that contains the point $T$,
we denote by $\pi_1$, the other by $\pi_2$. From the statement \ref{izomEnaC'}
it follows that in the half-plane $\pi_2$ there exists (one and only one) point $T'$, for
which $S_1T'\cong S_1T$ and $S_2T'\cong ST_2$. This would mean that even the point $T'$, which is different from the point $T$, lies on the circles
$k_1$ and $k_2$, which is not possible. Therefore, the points $S_1$, $S_2$ and $T$
are collinear.

 (\textit{ii}) By the statement \ref{TangPogoj} the tangent
of the circle $k_1$ at the point $T$ is perpendicular to the radius $S_1T$. Similarly
the tangent of the circle $k_2$ at the same point $T$ is perpendicular to the radius $S_2T$. Because of (\textit{i}) the lines $S_1T$ and $S_2T$ coincide,
so do both perpendiculars or tangents.
 \kdokaz

By the statement \ref{tangKrozEnaStr} all points of the circle are on the same side
of each of its tangents - on the side where its center is. This
means that the circles $k_1$ and $k_2$ from the previous statement are either
 on the same side or on different sides of their common tangent. When $B(S_1,T,S_2)$, the circles are on different sides
of their common tangent and we say that the circles $k_1$
and $k_2$ \pojem{touch each other from the outside}. Otherwise the circles are on the same side
of this tangent and we say that they \pojem{touch each other from the inside}. In
the first case, the interior of one of these two circles is in the exterior of the other, in
the second case, however, it is in the interior of the other circle (Figure
\ref{sl.skk.4.1.6.pic}).

\begin{figure}[!htb]
\centering
\input{sl.skk.4.1.6.pic}
\caption{} \label{sl.skk.4.1.6.pic}
\end{figure}

When the circles $k_1(S_1,r_1)$ and $k_2(S_2,r_2)$ touch each other from the outside,
it follows from the previous statement that $|S_1S_2| = r_1 + r_2$.
If the circles touch each other from the inside, then $|S_1S_2| = |r_1 - r_2|$.
It is clear that the converse is also true. The condition $|S_1S_2| = r_1 +
r_2$ or $|S_1S_2| = |r_1 - r_2|$ is sufficient for the circles
to touch each other from the outside or from the inside. In a similar way, we obtain the other
criteria for the mutual position of two circles.

\begin{figure}[!htb]
\centering
\input{sl.skk.4.1.7.pic}
\caption{} \label{sl.skk.4.1.7.pic}
\end{figure}

            \bizrek
            Let $k_1(S_1,r_1)$
            and $k_2(S_2,r_2)$ ($r_1\geq r_2$) be two circles. Then
            (Figure \ref{sl.skk.4.1.7.pic}):

            (i) the circles $k_1(S_1,r_1)$
            and $k_2(S_2,r_2)$ are lying outside each other
             if and only if $|S_1S_2|>r_1+r_2$;

             (ii) the circles $k_1(S_1,r_1)$
            and $k_2(S_2,r_2)$ are touching each other externally
            if and only if $|S_1S_2|=r_1+r_2$;

             (iii) the circles $k_1(S_1,r_1)$
            and $k_2(S_2,r_2)$ are intersecting each other at two points
            if and only if $r_1-r_2<|S_1S_2|<r_1+r_2$;

            (iv) the circles $k_1(S_1,r_1)$
            and $k_2(S_2,r_2)$ are touching each other internally
             if and only if $|S_1S_2|=r_1-r_2$;

(v) one of the circles $k_1(S_1,r_1)$
            and $k_2(S_2,r_2)$ is lying inside another
             if and only if $|S_1S_2|<r_1-r_2$.
            \eizrek


Now we will define some concepts that relate to two
circles.

\pojem{Kot med krožnicama}, which intersect, is the angle that is determined
by the tangents of these two circles at their common point. It is not difficult to prove
that this angle is not dependent on the choice of the common point, or that the angles between
the tangents in each of the two common points are consistent (Figure
\ref{sl.skk.4.1.8.pic}).

When the circles touch, we say that they determine the angle $0^0$.


\begin{figure}[!htb]
\centering
\input{sl.skk.4.1.8.pic}
\caption{} \label{sl.skk.4.1.8.pic}
\end{figure}

The circles are \index{pravokotni!krožnici} \pojem{pravokotni}, if
they determine the angle $90^0$, or if their tangents in the common point
are perpendicular (Figure \ref{sl.skk.4.1.8.pic}).

A direct consequence of izrek \ref{TangPogoj} is the following condition
of pravokotnosti of two circles.



            \bizrek \label{pravokotniKroznici}
            Two circles are perpendicular if and only if
            the tangent of one of the circles at the points of intersection
            contains the centre of  another circle.
             \eizrek

In zgled \ref{tangKrozKonstr} we have determined how to draw
the tangents of a circle from its arbitrary external point. In izrek
\ref{tangSkupnaDotikKrozn} we have  proved that the circles, which touch,  have at least one common tangent. In the next example we will
construct \index{skupna tangenta}\pojem{common tangents} of two
circles in general position.

            \bzgled \label{tang2ehkroz}
            Construct a common tangent of two given circles
            lying in the same plane.
            \ezgled

\textbf{\textit{Solution.}} Let $k_1(S_1,r_1)$ and $k_2(S_2,r_2)$
($r_1\geq r_2$) be any two circles in the same plane and $t$ their
common tangent, which touches the circles $k_1$ and $k_2$ in
succession in the points $T_1$ and $T_2$.

We will consider two cases:

\textit{1)} First, let us assume that the points $T_1$ and $T_2$ are
on the same side of the central line $S_1S_2$ (Figure
\ref{sl.skk.4.1.9.pic}). We mark $S'_2=pr_{\perp S_1T_1}(S_2)$. By
\ref{TangPogoj} we have $\angle S_1T_1T_2\cong\angle
S_2T_2T_1=90^0$. This means that the quadrilateral $S_2T_2T_1S'_2$
is a rectangle, so $\angle S_2S'_2T_1=90^0$ and $|S'_2T_1|=|S_2T_2|=r_2$.
 Since by assumption $r_1\geq r_2$ and $|S_1T_1|=r_1$, we have
 $|S_1S'_2|=r_1-r_2$. We mark with $k$ the circle with center $S_1$ and
 radius $r_1-r_2$. The circle $k$ passes through the point $S'_2$.
 If $S_2$ is an external point of the circle $k$,
 from $\angle S_2S'_2T_1=90^0$
 it follows that the line $S_2S_2'$ is tangent
 to this circle.

\begin{figure}[!htb]
\centering
\input{sl.skk.4.1.9.pic}
\caption{} \label{sl.skk.4.1.9.pic}
\end{figure}

 The previous analysis allows us to construct it. First, we plan
 the circle $k(S_1,r_1-r_2)$, then its tangent $S_1S'_2$ in
 the point of contact $S'_2$ (example \ref{tangKrozKonstr}), the point $T_1$ as
 the intersection of the segment $S'_2T_1$ and the circle $k_1$, the fourth vertex
 $T_2$ of the rectangle $T_1S'_2S_2T_2$ (because from the construction $\angle S_2S'_2T_1=90^0$) and finally the common tangent $t=T_1T_2$.

We will prove that $t$ is indeed the common tangent. Because by
construction $T_1\in k_1$ and $\angle S_1T_1T_2\cong
\angle S_2T_2T_1=90^0$, it is enough to prove that $T_2\in
k_2$. This follows from the fact that the quadrilateral $T_1S'_2S_2T_2$
is a rectangle or $|S_2T_2|=|S'_2T_1|=r_1-(r_1-r_2)=r_2$.

In addition to the planned tangent $t$, we also get the tangent $t_1$,
which is symmetrical to the tangent $t$ with respect to the center $S_1S_2$. For
the tangents $t$ and $t_1$, we say that they are \index{skupna
tangenta!zunanja}\pojem{external tangents}.

\textit{2)} If we assume that $T_1$ and $T_2$ are on different
banks of the center $S_1S_2$, we get two more t. i. \index{skupna tangenta!notranja}\pojem{internal tangents} in
certain cases, by the same procedure, only that we replace the circle $k(S_1,r_1-r_2)$ with the circle
$k'(S_1, r_1+r_2)$.

\begin{figure}[!htb]
\centering
\input{sl.skk.4.1.10.pic}
\caption{} \label{sl.skk.4.1.10.pic}
\end{figure}

Let us also consider the number of solutions to the task (Figure
\ref{sl.skk.4.1.10.pic}). When the circles are non-intersecting and
none of them is inside the other, the circles have all four described
common tangents - two external and two internal tangents. If
the circles touch from the outside, the task has three solutions, because the internal
tangents overlap and we get a common tangent, which is mentioned in
the statement \ref{tangSkupnaDotikKrozn}. When the circles intersect,
we have only two solutions - two external tangents. When the circles touch from
the inside, there is only one common tangent (that from the statement
\ref{tangSkupnaDotikKrozn}). And finally, if the circles are non-intersecting
and one of them is inside the other, they don't have
common tangents.
\kdokaz

\bzgled
            Let $A$, $B$, $C$ and $D$ be points in the plane such that they do not all lie
            on the same circle nor all on the same line. Prove that there are two such circles
            $k$ and $l$, which have no common points, the first of them passes
            through the points $A$ and $B$, and the other one
             through the points $C$ and $D$.
            \ezgled

\begin{figure}[!htb]
\centering
\input{sl.skk.4.1.1a.pic}
\caption{} \label{sl.skk.4.1.1a.pic}
\end{figure}

\textbf{\textit{Proof.}}
 Let
$m$ and $n$ be the perpendicular lines of $AB$ and $CD$. We will consider two
cases (Figure \ref{sl.skk.4.1.1a.pic}):

 \textit{1)}
If the lines $m$ and $n$ intersect at point $O$, the desired circles
$k(O,OA)$ and $l(O,OC)$ are concentric and different (by assumption).

\textit{2)} If the lines $m$ and $n$ are parallel, the lines $AB$ and $CD$ are also
parallel (and different by assumption). Let $p$ be any parallel line of $AB$ and $CD$ such that $AB$ and $CD$
are on different sides of $p$. Let $M$ and $N$ be the intersections of $m$ and $n$ with $p$. If $M \neq N$, the desired circles are the circumscribed circles of the triangles $ABM$ and $CDN$ (by Theorem \ref{SredOcrtaneKrozn}), because they lie on different sides of the line $p$. If $M = N$ or $m = n$, the desired circles are concentric again $k(M,MA)$ and $l(M,MC)$.
 \kdokaz



%________________________________________________________________________________
 \poglavje{Center Angle and Circumferential Angle} \label{odd4SredObod}


 Let us first define the concepts of central and circumferential angle
  (Figure \ref{sl.skk.4.2.1a.pic}).

\pojem{Central angle}
  \index{angle!central}  of a circle $k(S,r)$ is any angle,
  that lies in the plane of this circle  and has its vertex in the point $S$.
   \pojem{Circumferential angle}
  \index{angle!circumferential} of this circle is any angle with its vertex
  on the circle $k$, and its legs contain two chords of
this circle. The intersection of the circle and its central or circumferential angle
is an arc, which we call the \pojem{arc corresponding} to this angle. In this
case, for the central or circumferential angle we say that the angle is \pojem{over this arc}.
 We know that any chord $PQ$ on the circle $k(S,r)$ determines two arcs.
 If we know
for which
of the two arcs it is, we sometimes for the angle over the corresponding arc $PQ$
say that the angle is \pojem{over the chord} $PQ$.

In a special case, when the chord is a diameter, the corresponding central angle
over this chord is equal to $180^0$. Tales's theorem for a circle (theorem
\ref{TalesovIzrKroz}) can be written in terms of circumferential angles in the
following way:



             \bizrek \label{TalesovIzrKroz2oblika}
            All inscribed angles subtending a diameter of a circle are right angles.
            \eizrek

\begin{figure}[!htb]
\centering
\input{sl.skk.4.2.2a.pic}
\caption{} \label{sl.skk.4.2.2a.pic}
\end{figure}

Therefore, the central angle over a diameter is twice as big as
the circumferential angle over this diameter (Figure \ref{sl.skk.4.2.2a.pic}).
We will prove that this statement is also true for the circumferential and central angle over
any chord.


\begin{figure}[!htb]
\centering
\input{sl.skk.4.2.3.pic}
\caption{} \label{sl.skk.4.2.3.pic}
\end{figure}



         \bizrek
            \label{SredObodKot}
            In any circle, a central angle is twice of the measure
            of the circumferential angle subtending the
            same arc\footnote{Predpostavlja se, da je to trditev prvi dokazal
           \index{Tales}\textit{Tales} iz Mileta (7.--6. stol. pr. n. š.).} (Figure \ref{sl.skk.4.2.3.pic}).
          \eizrek

\begin{figure}[!htb]
\centering
\input{sl.skk.4.2.4.pic}
\caption{} \label{sl.skk.4.2.4.pic}
\end{figure}

\textbf{\textit{Proof.}} Let $PQ=l$ be a chord of the circle $k(S, r)$ in
$V$ and let $P$ be an arbitrary point of this circle which does not lie on this chord. We will
prove that for the central angle $PSQ$ and the inscribed angle $PVQ$ it holds:
$$\angle PSQ = 2\angle PVQ.$$ We will consider three different cases
(Figure \ref{sl.skk.4.2.4.pic}):

\textit{1)} The center $S$ of the circle $k$ lies on one of the sides
of the inscribed angle $PVQ$. Without loss of generality, let this be
the side $VQ$. In this case, the triangle $PSV$ is an isosceles triangle
($SP \cong SV = r$), therefore $\angle SPV \cong \angle PVS$ (by the isosceles triangle theorem). In
the triangle $PSQ$, the exterior angle $PSQ$ is equal to the sum of the non-adjacent interior
angles (by the theorem about the sum of the interior angles of a triangle). Therefore:
 $$ \angle PSQ =
\angle SPV + \angle PVS = 2\angle PVS\\ = 2\angle PVQ.$$

\textit{2)} The center $S$ of the circle $k$ lies in the interior of the inscribed angle $PVQ$. In
this case, the other intersection of the circle $k$ with the line $VS$ – point $V'$ – lies on the
chord $l$, because this chord represents the intersection of the inscribed angle  $PVQ$ and the
circle $k$. If we use the result from \textit{1)} twice, we get:
  \begin{eqnarray*}
\angle PSQ &=& \angle PSV'+\angle V'SQ =
2\angle PVV'+2\angle V'VQ=\\ &=& 2(\angle PVV'+\angle V'VQ) = 2\angle PVQ.
  \end{eqnarray*}
\textit{3)} The center $S$ of the circle $k$ is an exterior point of the inscribed angle. In the
same way as in \textit{2)}, we define the point $V'$. In this case, the point $V'$ does not lie on
the chord $l$. Without loss of generality, let $P$ be an interior point of the inscribed angle $\angle V'VQ$.
Again, we use the result from \textit{1)}:
  \begin{eqnarray*}
  \angle PSQ &=& \angle V'SQ
-\angle V'SP = 2\angle V'VQ - 2\angle V'VP=\\ &=& 2(\angle V'VQ -\angle
V'VP) = 2\angle PVQ,
  \end{eqnarray*}
 which is what needed to be proven.  \kdokaz

The two most important consequences of the previous theorem are the following.

\bizrek
        \label{ObodObodKot}
        In a circle, different circumferential angles
        subtending the same arc are congruent.
        \eizrek


\begin{figure}[!htb]
\centering
\input{sl.skk.4.2.5.pic}
\caption{} \label{sl.skk.4.2.5.pic}
\end{figure}

\textbf{\textit{Proof.}} From the previous statement \ref{SredObodKot}
it follows that all circumferential angles over the same line are equal to half
of the central angle over that line (Figure \ref{sl.skk.4.2.5.pic}).
  Because of this, all
of the aforementioned circumferential angles are congruent to each other.
 \kdokaz



        \bizrek
        \label{ObodObodKotNaspr}
        Two circumferential angles
        subtending the same chord of a circle, with the vertices lying
        on different sides of the line containing this chord,
        are supplementary.
        \eizrek


\textbf{\textit{Proof.}} The chord of a circle determines two lines on it which are complementary to each other (Figure \ref{sl.skk.4.2.5.pic}).
Each of the aforementioned circumferential angles corresponds to one of these two lines, which means that the sum of the corresponding central angles is equal
to $360^0$. Because the sum of circumferential angles according to statement \ref{SredObodKot}
is equal to half of the sum of the corresponding central angles, the circumferential angles are supplementary.
 \kdokaz

As we have already mentioned, we will often talk about circumferential and central
angles over the same chord, if we only know to which of the two corresponding lines this
chord belongs. In this sense, let us formulate the following statement.



        \bizrek
          \label{SklTetSklObKot}
          Two circumferential angles
        subtending the congruent chords of a circle
        are congruent.
         \eizrek

\begin{figure}[!htb]
\centering
\input{sl.skk.4.2.4a.pic}
\caption{} \label{sl.skk.4.2.4a.pic}
\end{figure}

\textbf{\textit{Proof.}}  According to the \textit{SSS} statement \ref{SSS} congruent
chords correspond to congruent central angles (Figure
\ref{sl.skk.4.2.4a.pic}). The statement is then a direct consequence of statement
\ref{SredObodKot}.  \kdokaz

It is clear that the following statement is also true  (Figure
\ref{sl.skk.4.2.4b.pic}).

\bizrek
          \label{SklTetSklObKot2}
          Two circumferential angles
        subtending the congruent chords of congruent circles
        are congruent.
         \eizrek


\begin{figure}[!htb]
\centering
\input{sl.skk.4.2.4b.pic}
\caption{} \label{sl.skk.4.2.4b.pic}
\end{figure}

The following consequence is also very interesting and useful.



         \bizrek
          \label{ObodKotTang}
          The angle determined by a chord of a circle and the tangent of that circle
            in one of the endpoints of this chord is congruent to the circumferential
            angle subtending this chord.
         \eizrek

\begin{figure}[!htb]
\centering
\input{sl.skk.4.2.6.pic}
\caption{} \label{sl.skk.4.2.6.pic}
\end{figure}

\textbf{\textit{Proof.}}  Let $PAQ$ be the circumferential angle of some circle over the arc $PQ$, $LP$ such a tangent of this circle in the point $P$, that the points $L$ and $A$ are on different sides of the line $PQ$, and $PB$ the diameter of this circle (Figure \ref{sl.skk.4.2.6.pic}). Because $BP \perp PL$ (izrek
\ref{TangPogoj}) and $BQ \perp PQ$ (Talesov izrek
\ref{TalesovIzrKroz}), the angles $LPQ$ and $PBQ$ are congruent (izrek
\ref{KotaPravokKraki}). But according to izrek \ref{ObodObodKot}, the angles $PAQ$ and $PBQ$ are congruent, therefore $\angle LPQ \cong PAQ$.
 \kdokaz

The previous result allows the realization of one very important construction
-- the design of the geometric location of points in the plane, from which the angle of view of the line segment $AB$ is
$\omega$, i.e. $\angle AXB\cong \omega$, is the union of two open circular arcs, which are
symmetric with respect to the line $AB$.
This is actually the following problem.

%angle of view of a line segment!!!


        \bizrek
          \label{ObodKotGMT}
          Let $A$ and $B$ be two different points and $\omega$ a given angle.
        The set of all points $X$ in the plane from which the angle of view of the line segment $AB$ is
        $\omega$, i.e. $\angle AXB\cong \omega$, is the union of two open circular arcs, which are
        symmetric with respect to the line $AB$.
        \eizrek


\begin{figure}[!htb]
\centering
\input{sl.skk.4.2.7.pic}
\caption{} \label{sl.skk.4.2.7.pic}
\end{figure}

\textbf{\textit{Proof.}} Let $p$ be a chord with the endpoint $A$, so that
$\angle p,BA \cong \omega$, line $n$ is the perpendicular to this
chord at point $A$ and $s$ is the line symmetrical to the line segment $AB$ (Figure
\ref{sl.skk.4.2.7.pic}). The intersection of lines $n$ and $s$ is denoted by
$S$, the circle with the center $S$ and the radius $SA$ is denoted by $k$. With $l$
we denote the arc, which is the intersection of circle $k$ and the plane not containing the chord $p$. In the complementary plane we determine the arc $l'$ in a similar way as the arc $l$. We prove that the desired geometric location of points is the set $l \cup l'\setminus\{A,B\}$.

The fact that from each point of the arc $l$ (or $l'$), which is different from
the points $A$ and $B$, the line segment $AB$ is seen under the angle $\omega$, follows from
theorems \ref{TangPogoj} and \ref{ObodKotTang}. For an arbitrary point
$P$ of the open arc $l$ it holds: $\angle APB \cong \angle p,AB \cong
\omega$.

Assume that the point $M$ does not belong to the set $l \cup
l'\setminus\{A,B\}$. In the case when $M$ is one of the points $A$ or
$B$, the angle $AMB$ does not even exist. Let $M \neq A,B$ and without loss of generality assume that the point $M$ is in the same plane as the arc $l$.
We denote by $N$ the other intersection of the chord $AM$ and the arc $l$ ($N\neq A$).
If $\mathcal{B}(A,M,N)$, then in the triangle $NMB$ the external angle $AMB$
is greater than the adjacent internal angle $MNB$ (theorem
\ref{zunanjiNotrNotrVecji}), which, according to what has already been proven, is equal to $\omega$,
so $\angle AMB>\omega$. If $\mathcal{B}(A,N,M)$ holds, we can, by using a similar reasoning,  conclude that in this case $\angle
AMB<\omega$, which means that for no point $M\notin l \cup
l'\setminus\{A,B\}$ it holds that $\angle AMB \cong \omega$.
 \kdokaz

From the proof of the previous theorem we also get the following conclusion.


            \bizrek \label{obodKotGMTZunNotr}
             Let $AB$ be an arc of a circle $k$, $\omega$ the corresponding circumferential angle
          of this arc and $M$ a point of the half-plane with the edge $AB$ not containing this arc (Figure
            \ref{sl.skk.4.2.8.pic}). Then:

(i) $\angle AMB >\omega$ if and only if the point $M$ is an interior point of the circle $k$;

             (ii) $\angle AMB \cong\omega$ if and only if the point $M$ lies on the circle $k$;

             (iii) $\angle AMB <\omega$ if and only if the point $M$ is an exterior point of the circle $k$.
            \eizrek


\begin{figure}[!htb]
\centering
\input{sl.skk.4.2.8.pic}
\caption{} \label{sl.skk.4.2.8.pic}
\end{figure}

In the following examples we will see the use of the theorem about the peripheral and central angle and its consequences.



         \bzgled
         Construct a triangle with given $a$, $\alpha$, $v_a$. \label{konstr_aalphava}
         \ezgled

\begin{figure}[!htb]
\centering
\input{sl.skk.4.3.1d.pic}
\caption{} \label{sl.skk.4.3.1d.pic}
\end{figure}

   \textbf{\textit{Analysis.}} The point $A$ lies at the same time at the geometric location of the points from which the line $BC$ is seen at an angle $\alpha$ (the union of two circular arcs - Theorem \ref{ObodKotGMT}) and at the parallel of the line $BC$, which is $v_a$ away from it (Figure \ref{sl.skk.4.3.1d.pic}). So the vertex $A$ is the intersection of this parallel and the aforementioned geometric location of points.

\textbf{\textit{Construction.}} First, let's draw the line $BC\cong a$, then the geometric location of points $\mathcal{L}$, from which this line is seen at an angle $\alpha$ (Theorem \ref{ObodKotGMT}). Then let's draw the parallel $p$ of the line $BC$ at a distance $v_a$. With $A$ we mark the intersection of the line $p$ and the aforementioned geometric location of points $\mathcal{L}$. We will prove that the triangle $ABC$ is the desired triangle.

\textbf{\textit{Proof.}} By construction, it is clear that $BC\cong a$. By construction, the point $A$ lies on the geometric location of points from which the line $BC$ is seen at an angle $\alpha$, so $BAC\cong\alpha$. The altitude of the triangle $ABC$ from the vertex $A$ is consistent with the line $v_a$, because the point $A$ by construction lies on the line $p$, which is $v_a$ away from the line $BC$.

\textbf{\textit{Discussion.}} The necessary condition is of course $\alpha<180^0$. The number of solutions to the task is equal to the number of intersections
of the line $p$ and the set $\mathcal{L}$.
 \kdokaz



            \bzgled
           Let $p$, $q$ and $r$ be lines in the plane intersecting
            at one point and divide this plane into six congruent angles. Suppose that $P$, $Q$ and $R$
            are the foots of the perpendiculars from an arbitrary point $X$ of this plane on the lines $p$, $q$ and
            $r$, respectively. Prove that $PQR$ is a regular triangle.
            \ezgled



\begin{figure}[!htb]
\centering
\input{sl.skk.4.2.9.pic}
\caption{} \label{sl.skk.4.2.9.pic}
\end{figure}

\textbf{\textit{Proof.}}  (Figure \ref{sl.skk.4.2.9.pic}).
 Let $S$
be the intersection of the lines $p$, $q$ and $r$. It is clear that the lines determine
the angles $60^0$. Because $\angle XPS \cong \angle XQS \cong \angle XRS =
90^0$, by  \ref{TalesovIzrKroz2} the points $S$, $X$, $P$, $Q$ and
$R$ lie on the circle $k$ with diameter $SX$.
 If we use \ref{ObodObodKot} for the appropriate arcs $PQ$ and $QR$,
 we have $\angle PRQ \cong \angle PSQ =
60^0$ and $\angle QPR \cong \angle QSR = 60^0$. Because all angles
are equal to $60^0$, $PQR$ is a regular triangle.
 \kdokaz

Let $k(O,R)$ and $l(S,r)$ ($R = 2r$) be circles touching each other internally
and $P$ an arbitrary point on the circle $l$. Which curve
is described by the point $P$ if the circle $l$ rolls without slipping around the circle $k$\footnote{This problem was solved by the Polish astronomer
 \index{Copernicus, N.} \textit{N. Copernicus} (1473--1543).
 In the general case, when it is not necessarily $R = 2r$, the curve
 which is described by the point $P$, is called \index{hipocikloida}
  \pojem{hipocikloida}. In the case of the outer rolling of the circle
  around the other circle, the curve is called \index{epicikloida}
  \pojem{epicikloida}, in the case
  of the rolling of the circle around the straight line, it is called \index{cikloida}
  \pojem{cikloida}. The cikloida was first investigated by the German mathematician and
  philosopher \index{Kuzanski, N.} \textit{N. Kuzanski} (1401--1464)
  and later by the French mathematician and philosopher \index{Mersenne, M.}
  \textit{M. Mersenne} (1588--1648). It was named by the Italian physicist,
  mathematician, astronomer and philosopher \index{Galilei, G.}
   \textit{G. Galilei} (1564--1642) in 1599.}?

\begin{figure}[!htb]
\centering
\input{sl.skk.4.2.10.pic}
\caption{} \label{sl.skk.4.2.10.pic}
\end{figure}

\textbf{\textit{Solution.}}
 Let
$P_0$ be the position of the point $P$ at the moment when it lies on the circle $k$ (Figure
\ref{sl.skk.4.2.10.pic}). At the moment when the point $P$ is in position $P_i$,
the circle $l$, which is in position $l_i$, touches the circle $k$ at some
point $T_i$. Since this is a "movement without slipping", the lengths
of the corresponding arcs $P_0T_i$ and $P_iT_i$ of the circles $k$ and $l$ are equal to each other. The radius of the circle $k$ is twice as large as the radius of the circle $l$,
so for the corresponding central angles of the aforementioned arcs we have $\angle
T_iS_iP_i= 2\angle T_iOP_0$. But $\angle T_iOP_i$ is the corresponding arc angle of the circle $l$ for the same arc $P_iT_i$, so $2\angle
T_iOP_i = \angle T_iS_iP_i$ (statement \ref{ObodObodKot}). From the two previous
relations we obtain $\angle T_iOP_i= \angle T_iOP_0$, which means that the point $P_i$ is collinear with the points $O$ and $P_0$, so the desired path
of the point $P$ represents the diameter $P_0P'_0$  of the circle $k$.
 \kdokaz



            \bzgled
            Let $C$ be the midpoint of an arc $AB$ and $D$ an arbitrary point of this
            arc other than $C$. Prove that
                $$AC + BC > AD + BD.$$
            \ezgled

\begin{figure}[!htb]
\centering
\input{sl.skk.4.2.11.pic}
\caption{} \label{sl.skk.4.2.11.pic}
\end{figure}

\textbf{\textit{Proof.}}
  (Figure
\ref{sl.skk.4.2.11.pic}).
 Let $C'$ and $D'$ be such points on the line segments $AC$ and $AD$,
 that: $CC'\cong CB \cong CA$, $DD'\cong DB$,
$\mathcal{B}(A,C,C')$ and $\mathcal{B}(A,D,D')$. The triangles $C'CB$ and
$D'DB$ are isosceles, therefore by  \ref{enakokraki} and \ref{zunanjiNotrNotr},
it follows:
 $$\angle CC'B \cong \angle CBC' = \frac{1}{2}\angle ACB
\hspace*{2mm}\textit{ and }\hspace*{2mm}
 \angle DD'B \cong \angle DBD'
= \frac{1}{2}\angle ADB.$$
 Because the angles $ACB$ and $ADB$ are supplementary
  (by \ref{ObodObodKot}).
The angles $AC'B$ and $AD'B$ are supplementary as well,  and the points $C'$ and $D'$ lie
on the corresponding arc $l'$ over the segment $AB$ (by \ref{ObodKotGMT}). Because $CC'\cong CB \cong CA$, the distance $AC'$
is the diameter of the circle that contains the arc $l'$, therefore the $\angle AD'C'$ is a right angle (by \ref{TalesovIzrKroz2}). The distance $AC'$ is therefore
the hypotenuse of the right triangle $AD'C'$ and by  \ref{vecstrveckot} it follows:
$$AC + CB = AC'> AD'= AD + DB,$$ which was to be proven. \kdokaz



      \bnaloga\footnote{35. IMO Hong Kong - 1994, Problem 2.}
      $ABC$ is an isosceles triangle with $BC \cong AC$. Suppose that:

        (i) $D$ is the midpoint of $AB$ and $E$ is the point on the line $CD$ such that
            $EA\perp AC$;

        (ii) $F$ is an arbitrary point on the segment $AB$ different from $A$ and $B$;

        (iii) $G$ lies on the line $CA$ and $H$ lies on the line $CB$ such that $G$, $F$, $H$ are
              distinct and collinear.

        Prove that $EF$ is perpendicular to $HG$ if and only if $GF\cong FH$.
        \enaloga


\begin{figure}[!htb]
\centering
\input{sl.skk.4.2.IMO1.pic}
\caption{} \label{sl.skk.4.2.IMO1.pic}
\end{figure}


\textbf{\textit{Proof.}} First, from the similarity of the triangles $CAE$ and
$CBE$ (by \textit{SAS} \ref{SKS}) it follows that $\angle EBC\cong\angle
EAC=90^0$ and $EA\cong EB$ (Figure \ref{sl.skk.4.2.IMO1.pic}).
We prove the equivalence $EF\perp HG \Leftrightarrow GF\cong FH$ in both
directions.

($\Rightarrow$) Let's assume that the lines $EF$ and $HG$ are
perpendicular, i.e. $\angle EFG\cong\angle EFH=90^0$. Let $k$
and $l$ be circles with diameters $EG$ and $EH$. Because
$\angle EAG\cong\angle EFG=90^0$ and $\angle EBH\cong\angle
EFH=90^0$, by \ref{TalesovIzrKroz} we have $A,F\in k$ and $B,F\in
l$. First, from $EA\cong EB$ it follows that $\angle EAB\cong
\angle EBA$. From this and \ref{ObodObodKot} it follows:
 $$\angle EGF\cong\angle EAF\cong\angle EBF\cong\angle EHF.$$
 Therefore, the triangle $EGH$ is equilateral, so its height $EF$
 is also the altitude (congruence of triangles $EFG$ and $EFH$,
 \textit{ASA} \ref{KSK}) or $GF\cong FH$.

($\Leftarrow$) Now let $GF\cong FH$, i.e. the point $F$ is the
center of the line $GH$. Let $k$ be a circle with diameter $EG$.
In addition to the point $A$ we mark with $\widehat{F}$ the
other intersection of this circle with the line $AB$. If the
circle $k$ touches the line $AC$, it follows that $G=A$ or $F=D$
and $H=B$, so in this case $GF\cong FH$ is already fulfilled.

Assume that $\widehat{F}\neq F$. Let $\widehat{H}$ be the
intersection of the lines $G\widehat{F}$ and $CB$. Because the
point $\widehat{F}$ lies on the circle $k$ with diameter $EG$,
$\angle G\widehat{F}E=90^0$, i.e. $E\widehat{F}\perp
G\widehat{H}$. Therefore, for the points $G$, $\widehat{F}$ and
$\widehat{H}$ the assumptions of the left side of the
equivalence are fulfilled, so from the already proven first part
of the statement ($\Rightarrow$) it follows that
$G\widehat{F}\cong \widehat{F}\widehat{H}$, i.e. the point
$\widehat{F}$ is the center of the line $G\widehat{H}$. In the
triangle $GH\widehat{H}$ the $\widehat{F}F$ is the median, so
$\widehat{F}F\parallel \widehat{H}H$ and $AB\parallel BC$, which
is not possible. Thus, the assumption $\widehat{F}\neq F$
disappears, so $\widehat{F}= F$, so $\widehat{H}=H$. In the end,
from $E\widehat{F}\perp G\widehat{H}$ it follows that $EF\perp
GH$.
 \kdokaz





%________________________________________________________________________________
 \poglavje{More About Circumcircle and Incircle of a Triangle}
 \label{odd4OcrtVcrt}


First, we will consider some important points that lie on the
circumcircle of a triangle.

\bizrek \label{TockaN}
        The bisector of the side $BC$ and the bisector of the interior angle $BAC$ of a
        triangle $ABC$ ($AB\neq AC$) intersect at the Circumcircle $l(O,R)$
        of that triangle.
         \eizrek

\begin{figure}[!htb]
\centering
\input{sl.skk.4.3.1.pic}
\caption{} \label{sl.skk.4.3.1.pic}
\end{figure}


\textbf{\textit{Proof.}}
   (Figure \ref{sl.skk.4.3.1.pic}).

 Let the point $N$ be one of the intersections of the Circumcircle $l(O,R)$ of the triangle $ABC$ and
 the bisector of the side $BC$
(such that $A,N\perp BC$). Because the point $N$ lies on the bisector of the side
$BC$, it follows that $NB \cong NC$. Therefore, the triangle $BNC$ is an isosceles triangle,
so the angles $\angle NBC$ and $\angle NCB$ are congruent angles (by Theorem \ref{enakokraki}). Because the point $N$ also lies on the Circumcircle $l(O,R)$
of the triangle $ABC$, by Theorem \ref{ObodObodKot} it follows that:
 \begin{eqnarray*}
 \angle BAN
\cong \angle BCN \textrm{ (obodna kota za krajši lok }BN \textrm{) }\\
 \angle NAC \cong \angle NBC \textrm{ (obodna kota za krajši lok }
 CN \textrm{).}
 \end{eqnarray*}
 Therefore, $\angle BAN \cong \angle NAC$ or the line $AN$ is the bisector of the angle
$BAC$, thus the theorem is proven.
 \kdokaz


 The point $N$ from the previous theorem is the center of that arc $BC$ of the Circumcircle $l(O,R)$ of the triangle $ABC$ which does not contain the vertex $A$.
 We shall now prove another important property of the point $N$.



        \bizrek \label{TockaN.NBNC}
        For the point $N$ from the previous theorem is $NB\cong NS\cong NC$,
        where $S$ is the incentre of the triangle $ABC$.
         \eizrek


\begin{figure}[!htb]
\centering
\input{sl.skk.4.3.2.pic}
\caption{} \label{sl.skk.4.3.2.pic}
\end{figure}

\textbf{\textit{Proof.}} Let's mark with $\alpha$ and $\beta$ the internal
angles of the triangle $ABC$ at the vertices $A$ and $B$ (Figure
\ref{sl.skk.4.3.2.pic}). $BNS$ is an isosceles triangle (statement
\ref{enakokraki}), because the angles at the vertices $B$ and $S$ are equal.
If we use the statement \ref{zunanjiNotrNotr} and \ref{ObodObodKot},
we get:
 \begin{eqnarray*}
 \angle BSN &=& \angle ABS + \angle BAS =\frac{1}{2}\alpha+\frac{1}{2}\beta,\\
\angle SBN &=& \angle SBC +\angle CBN  = \angle SBC + \angle CAN =
\frac{1}{2}\beta+\frac{1}{2}\alpha.
 \end{eqnarray*}
Therefore $NB \cong NS$  and similarly $NC \cong NS$.
\kdokaz



           \bizrek \label{TockaNbetagama}
            Let $AA'$ be the altitude from the vertices $A$ and $AE$ bisector of the interior
           angle $BAC$ of a triangle $ABC$ ($A',E\in BC$). Suppose that $l(O,R)$ is the
            circumcircle of that triangle. If $\angle CBA=\beta\geq\angle ACB=\gamma$,
          then
           $$\angle A'AE\cong \angle EAO=\frac{1}{2}\left( \beta-\gamma\right).$$
           \eizrek

\begin{figure}[!htb]
\centering
\input{sl.skk.4.3.1b.pic}
\caption{} \label{sl.skk.4.3.1b.pic}
\end{figure}


\textbf{\textit{Proof.}}
 Let $N$ be the point defined as in the previous statements
  (Figure \ref{sl.skk.4.3.1b.pic}). The lines
 $AA'$ and $ON$ are parallel, because they are both perpendicular to the line $BC$.
 Because $OA\cong ON=R$, $AON$ is an isosceles triangle. Therefore, first of all:
 $$\angle A' AE \cong \angle ANO \cong \angle NAO =
\angle EAO,$$
 and then:
 $$\angle A'AE=\frac{1}{2}\alpha-\left(90^0-\beta \right)=
 \frac{1}{2}\alpha-\left(\frac{\alpha+\beta+\gamma}{2}-\beta \right)=
 \frac{1}{2}\left( \beta-\gamma\right),$$ which was to be proven. \kdokaz

\bzgled \label{tockaNtockePQR}
 Let $P$, $Q$ and $R$ be the midpoints of those arcs $BC$, $AC$ and $AB$ of
 the circumcircle of a triangle $ABC$ not containing the vertices $A$, $B$ and $C$
 of that triangle.
 If $E$ and $F$ are intersections of the line $QR$ with sides $AB$ and $AC$, respectively
  and $S$ the incentre of this triangle, then:

  (i) $AP \perp QR$,

   (ii) the quadrilateral $AESF$ is a rhombus.
  \ezgled


\begin{figure}[!htb]
\centering
\input{sl.skk.4.3.2a.pic}
\caption{} \label{sl.skk.4.3.2a.pic}
\end{figure}


\textbf{\textit{Proof.}} (Figure \ref{sl.skk.4.3.2a.pic})

(\textit{i}) Let $L$ be the intersection of the lines $AP$ and $QR$. By
\ref{TockaN} points $P$, $Q$ and $R$ lie on the simetrals $AS$, $BS$
and $CS$ of the internal angles of the triangle $ABC$ ($S$ is the center of the triangle
$ABC$ of the inscribed circle). If we denote with $\alpha$, $\beta$ and $\gamma$ the internal angles of the triangle $ABC$, then due to the similarity of the corresponding external angles (\ref{ObodObodKot}) we get:
 \begin{eqnarray*}
\angle RPL &=& \angle RPA = \angle RCA =\frac{1}{2}\gamma,\\
 \angle PRL &=& \angle PRQ = \angle PRC + \angle CRQ =
\angle PAC + \angle CBQ = \frac{1}{2}\alpha+ \frac{1}{2}\beta.
 \end{eqnarray*}
Therefore, the sum of the angles in the triangle $PRL$  (\ref{VsotKotTrik})
$180^0 =\frac{1}{2}\alpha+ \frac{1}{2}\beta +\frac{1}{2}\gamma
+\angle RLP = 90° + \angle RLP$. So $\angle RLP = 90°$ or $AP
\perp QR$.

(\textit{ii}) By
the statement \ref{TockaN.NBNC} it is $RA \cong RS$, therefore from the similarity of
right-angled triangles $ALR$ and $SLR$ (statement \textit{SSA}
\ref{SSK}) it follows that the point $L$ is the center of the diagonal $AS$
of the quadrilateral $AESF$. From the similarity of triangles $AEL$ and $AFL$ (statement
\textit{ASA} \ref{KSK}) it follows that the point $L$ is also the center
of the diagonal $EF$, therefore the $AESF$ is a parallelogram (statement
\ref{paralelogram}). Because the diagonals $AS$ and $EF$ are also perpendicular, the
quadrilateral $AESF$ is a rhombus (statement \ref{RombPravKvadr}).
 \kdokaz

From the previous statement we directly get a consequence.


          \bzgled \label{PedalniLemasPQR}
           Let $P$, $Q$ and $R$ be the midpoints of those arcs $BC$, $AC$ and $AB$ of
           the circumcircle of a triangle $ABC$ not containing the vertices $A$, $B$ and $C$
            of that triangle.
            Prove that the incentre of the  triangle $ABC$ is at the same time
            orthocentre  of the triangle $PQR$ (Figure \ref{sl.skk.4.3.2b.pic}).
          \ezgled


\begin{figure}[!htb]
\centering
\input{sl.skk.4.3.2b.pic}
\caption{} \label{sl.skk.4.3.2b.pic}
\end{figure}



We prove some consequences of statement \ref{ObodObodKot}, which are related to
the altitude point.



         \bizrek \label{TockaV'}
         Points that are symmetric to the orthocentre of an acute triangle
            with respect to its sides lie on the circumcircle of this triangle.
         \eizrek


\begin{figure}[!htb]
\centering
\input{sl.skk.4.3.3.pic}
\caption{} \label{sl.skk.4.3.3.pic}
\end{figure}

\textbf{\textit{Proof.}} Let $V$ be the orthocentre of a triangle
$ABC$, $l$ the circumcircle of the triangle $ABC$ and $V_a$ the other
intersection point of the circle $l$ with the altitude $AA'$ (Figure
\ref{sl.skk.4.3.3.pic}). We prove that the point $V_a$ is symmetric to
the point $V$ with respect to the line $BC$. It is enough to prove
that $VA'\cong V_aA'$. The angles $V_aBC$ and $V_aAC$ are complementary
(circumscribed angle for the chord $V_aC$ - izrek \ref{ObodObodKot}),
the angles $V_aAC$ and $CBV$  are complementary angles with
perpendicular legs (izrek \ref{KotaPravokKraki}). Therefore, the
angles $V_aBC$ and $CBV$ are also complementary, and so are the
triangles $V_aBA'$ and $VBA'$ or $VA'\cong V_aA'$. A similar statement
holds for the other two altitudes.
 \kdokaz

            \bizrek \label{TockaV'a}
            Let $V$ be the orthocentre of a triangle
            $ABC$. The circumcircles of the triangles $VBC$, $AVB$ and $ACV$ are congruent to
            the circumcircle of the triangle $ABC$.
             \eizrek


\begin{figure}[!htb]
\centering
\input{sl.skk.4.3.3a.pic}
\caption{} \label{sl.skk.4.3.3a.pic}
\end{figure}


\textbf{\textit{Proof.}} A direct consequence of the previous izrek
\ref{TockaV'}, because the three aforementioned circumcircles are
symmetric to the circumcircle of the triangle $ABC$ with respect to
the altitude of its sides (Figure \ref{sl.skk.4.3.3a.pic}).



         \bizrek \label{TockaV1}
          Points that are symmetric to the orthocentre of an acute triangle
            with respect to the midpoints of its sides lie on the circumcircle of this triangle.
         \eizrek


\begin{figure}[!htb]
\centering
\input{sl.skk.4.3.3b.pic}
\caption{} \label{sl.skk.4.3.3b.pic}
\end{figure}

\textbf{\textit{Proof.}} Let $V$ be the altitude point of the triangle $ABC$,
$l$ the circumscribed circle (with the center $O$) of the triangle $ABC$ and $V_{A_1}$ the point that is symmetrical to the point $A$ with respect to the point $O$ (Figure \ref{sl.skk.4.3.3.pic}). From the very definition of the point $V_{A_1}$ it is clear that it lies on the circle $l$. We shall prove that $V_{A_1}$ is symmetrical to the point $V$ with respect to the point $A_1$, which is the center of the line $BC$. Because $AV_{A_1}$ is the diameter of the circle $l$, according to the Theorem \ref{TalesovIzrKroz2} $\angle ACV_{A_1}=90^0$ or $V_{A_1}C\perp AC$.  The line $BV$ is the altitude of the triangle $ABC$, therefore $BV\perp AC$. From the last two relations it follows that $V_{A_1}C\parallel BV$. Similarly
$V_{A_1}B\parallel CV$. Therefore the quadrilateral $V_{A_1}CVB$ is a parallelogram, thus its diagonals $VV_{A_1}$ and $BC$ have a common center. The center of the line $BC$ is the point $A_1$, which means that  $V_{A_1}$ is symmetrical to the point $V$ with respect to the point $A_1$. The point $V_{A_1}$ but, according to the construction, lies on the circle $l$.
\kdokaz


We shall also prove some consequences of the Theorem \ref{ObodObodKot}, which are connected with the circumscribed
circle of a triangle.


        \bzgled \label{zgledTrikABCocrkrozP}
        Let  $k$ be the circumcircle of a regular triangle $ABC$.
         If $P$ is an arbitrary point lying
        on the shorter arc $BC$ of the circle $k$, then
         $$|PA|=|PB|+|PC|.$$
         \ezgled


\begin{figure}[!htb]
\centering
\input{sl.skk.4.3.4.pic}
\caption{} \label{sl.skk.4.3.4.pic}
\end{figure}

\textbf{\textit{Proof.}} Because $\angle ACP>60^0>\angle PAC$, by
the statement \ref{vecstrveckot} $AP>PC$ (Figure
\ref{sl.skk.4.3.4.pic}). Therefore, on the line $AP$ there exists such
a point $Q$, that $PQ\cong PC$. By the statement \ref{ObodObodKot}
$\angle CPQ=\angle CPA\cong\angle CBA=60^0$, which means that
$PCQ$ is an isosceles triangle, therefore also $CQ\cong CP$ and
$\angle PCQ=60^0$. From this it follows that $\angle ACQ=\angle
ACP-60^0=\angle BCP$. By the statement \textit{SAS} (statement
\ref{SKS}) the triangles $ACQ$ and $BCP$ are similar, therefore
also $AQ\cong BP$.

In the end: $|PB|+|PC|=|AQ|+|PQ|=|AP|$.
\kdokaz



            \bzgled
            Three circles of equal radii $r$ intersect at point $O$.
            Furthermore each two of them intersect at one more point: $A$, $B$, and $C$.
            Prove that the radius of the circumcircle of the triangle $ABC$ is also $r$.
            \ezgled

\begin{figure}[!htb]
\centering
\input{sl.skk.4.3.5.pic}
\caption{} \label{sl.skk.4.3.5.pic}
\end{figure}


\textbf{\textit{Proof.}}
   (Figure \ref{sl.skk.4.3.5.pic})

We mark with $k$, $l$, $j$ and $o$ the circumcircles of the triangles
$OBC$, $OAC$, $OAB$ and $ABC$ and with $P$ an arbitrary point of the
circle $k$ so that the points $O$ and $P$ are on different sides of
the line $BC$. By the assumption the circles $k$, $l$ and $j$ are
similar. The angles $BAO$ and $BCO$ are also similar, because they
are the angles between the similar circumcircles $k$ and $j$ over
the chord $BO$ (statement \ref{SklTetSklObKot2}). Analogously the
angles $CAO$ and $CBO$ are similar. Because of this:
 $$\angle BAC = \angle BAO + \angle CAO = \angle BCO
+ \angle CBO = 180° - \angle BOC = \angle BPC.$$
 Therefore the circles $k$ and $o$ have a similar
circumferential angle over the common chord $BC$, therefore they are
similar.
   \kdokaz



             \bzgled \label{KvadratKonstr4tocke}
            Construct a square $ABCD$ such that given points $P$, $Q$, $R$ and $S$
             lyes on the sides $AB$, $BC$, $CD$ and $DA$ of this square, respectively.
             \ezgled

\begin{figure}[!htb]
\centering
\input{sl.skk.4.3.1c.pic}
\caption{} \label{sl.skk.4.3.1c.pic}
\end{figure}


\textbf{\textit{Solution.}}
Since $\angle PAS$ and $\angle QCR$ are right angles, points $A$ and $B$ lie on the circles $k$ and $l$ with diameters
$PS$ and $QR$ (Figure \ref{sl.skk.4.3.1c.pic}). The carrier of the diagonal $AC$ of the square $ABCD$ is also the symmetry of the internal angles $BAD$ and $BCD$, so it goes through
the centers $N$ and $M$ of the corresponding arcs, which are determined by $k$ and $l$ (statement \ref{TockaN}).
The construction can therefore be carried out by first planning the circles $k$ and $l$, then the line $NM$, the points $A$ and $C$ and finally
the points $B$ and $D$.
 \kdokaz




         \bzgled
         Construct a triangle $v_a$, $t_a$, $l_a$.
         \ezgled

\begin{figure}[!htb]
\centering
\input{sl.skk.4.3.1e.pic}
\caption{} \label{sl.skk.4.3.1e.pic}
\end{figure}

   \textbf{\textit{Solution.}}
   Let $ABC$ be a triangle in which the altitude $AA'$,
the median $AA_1$ and the section of the symmetry $AE$ of the internal angle $BAC$ are consistent
with the distances $v_a$, $t_a$ and $l_a$. With $O$ we mark the center of the triangle $ABC$ of the drawn
circle $k$. By statement \ref{TockaN}
the lines $AE$ and $OA_1$ intersect in the point $N$, which lies on
the circle $k$ (Figure \ref{sl.skk.4.3.1e.pic}).

So we can first plan
a right triangle $AA'E$ with the leg $v_a$ and the hypotenuse $l_a$ and the point $A_1$ from the condition $AA_1\cong t_a$. Then
plan the point $N$ as the intersection of the line $AE$ and the perpendicular of the line $A'E$ through
the point $A_1$. The center $O$ is the intersection of the line $A_1N$ and
the symmetry of the distance $AN$ (because $AN$ is the chord of the circle $k$). The points $B$ and $C$ are the intersections
of the circle $k(O,OA)$ with the line $A'E$.
   \kdokaz




         \bzgled
         Construct a triangle  $R$, $r$, $a$. \label{konstr_Rra}
         \ezgled

\begin{figure}[!htb]
\centering
\input{sl.skk.4.3.1a.pic}
\caption{} \label{sl.skk.4.3.1a.pic}
\end{figure}


\textbf{\textit{Solution.}} Let $ABC$ be a triangle such that $BC\cong a$ and $l(O,R)$ and $k(S,r)$ are its circumscribed and inscribed circle
   (Figure \ref{sl.skk.4.3.1a.pic}). We denote with $\alpha$, $\beta$ and $\gamma$ its internal angles at vertices $A$, $B$ and $C$. By 
\ref{SredObodKot} we have $\alpha = \angle BAC = \frac{1}{2}\cdot\angle BOC$.
From \ref{kotBSC} it follows that $\angle BSC=90^0+\frac{1}{2}\cdot\alpha$.
From two relations we obtain the equality
$\angle BSC=90^0+\frac{1}{4}\cdot\angle BOC$, which allows the construction.

First we draw an isosceles triangle $BOC$ ($BC\cong a$ and $OB\cong
OC\cong R$). We obtain point $S$ as one of the intersections of the arc with the chord
$BC$ and the external angle $90^0+\frac{1}{4}\cdot\angle BOC$ and the line,
which is at a distance $r$ parallel to the line $BC$. Then we draw
the inscribed circle $k(S,r)$ and point $A$ as the intersection of the other two tangents
of this circle from points $B$ and $C$.
 \kdokaz



        \bnaloga\footnote{47. IMO Slovenia - 2006, Problem 1.}
        Let $ABC$ be a triangle with incentre $I$. A point $P$ in the interior of the
        triangle satisfies
        $$\angle PBA + \angle PCA = \angle PBC + \angle PCB.$$
        Show that $|AP| \geq |AI|$, and that equality holds if and only if $P = I$.
         \enaloga


\begin{figure}[!htb]
\centering
\input{sl.skl.4.3.IMO1.pic}
\caption{} \label{sl.skl.4.3.IMO1.pic}
\end{figure}

\textbf{\textit{Proof.}} We mark with $\alpha$, $\beta$ and $\gamma$
the internal angles of the triangle $ABC$ at the vertices $A$, $B$ and $C$
(Figure \ref{sl.skl.4.3.IMO1.pic}). The condition $\angle PBA + \angle PCA
= \angle PBC + \angle PCB$ can be rewritten in the form
$\beta-\angle PBC + \gamma-\angle PCB = \angle PBC + \angle PCB$
or:
 $$\angle PBC + \angle PCB=\frac{1}{2}\left( \beta+\gamma\right).$$
From this and the fact that the sum of the internal angles of each of the
triangles $BPC$ and $ABC$ is equal to $180^0$ (\izrekref{VsotKotTrik}),
it follows:
 $$\angle BPC =180^0-\frac{1}{2}\left( \beta+\gamma\right)=90^0+
 \frac{1}{2} \alpha.$$
But from \kotBSC it follows $\angle BIC =90^0+
 \frac{1}{2}\cdot \alpha$, so $\angle BPC\cong \angle BIC$.
 Therefore, the points $P$ and $I$ lie on the same curve $\mathcal{L}$ with
 the chord $BC$ and the central angle $90^0+
 \frac{1}{2} \alpha$. Let the point $N$ be the intersection
 of the perpendicular bisector of the side $BC$ and the perpendicular
 bisector of the internal angle $BAC$ of the triangle $ABC$. By
 \izrekref{TockaN}, the point $N$ lies on the circumscribed circle
 of the triangle $ABC$ and $NB\cong NI\cong NC$ (\izrekref{TockaN.NBNC}). This means that $N$ is the center of the curve
 $\mathcal{L}$, so $NP\cong NI$ or $\angle NIP\cong\angle
 NPI<90^0$. Because the points $A$, $I$ and $N$ are collinear (they lie on
 the perpendicular bisector of the internal angle $BAC$), $\angle AIP =180^0-\angle
 NIP>90^0$. From \izrekref{vecstrveckot} (for the triangle $API$)
 it now follows that $|AP| \geq |AI|$ and the equality holds exactly when
 the triangle $API$ is not, i.e. when $P=I$.
  \kdokaz


        \bnaloga\footnote{43. IMO United Kingdom - 2002, Problem 2.}
          $BC$ is a diameter of a circle center $O$. $A$ is any point on
        the circle with $\angle AOC>60^0$. $EF$ is the chord which is the perpendicular
        bisector of $AO$. $D$ is the midpoint of the minor arc $AB$. The line through
        $O$ parallel to $AD$ meets $AC$ at $J$. Show that $J$ is the incenter of triangle
        $CEF$.
         \enaloga


\begin{figure}[!htb]
\centering
\input{sl.skk.4.3.IMO4.pic}
\caption{} \label{sl.skk.4.3.IMO4.pic}
\end{figure}

\textbf{\textit{Proof.}} Because
the points $E$ and $F$ lie on the line of symmetry $EF$, and also on
the circle with center $O$, we have $$AF\cong FO\cong AO\cong EO \cong EA.$$ This
means that the quadrilateral $EOFA$ is a rhombus, which is made up of two congruent triangles $AOF$ and $AEO$.

Without loss of generality, we
assume that
$\angle COE>\angle COF$ (Figure \ref{sl.skk.4.3.IMO4.pic}).
First, from the condition $\angle
        AOC>60^0$ it follows that $\angle COF=60^0- \angle
        AOC>0^0$, so the points $A$ and $F$ are on the same side of the line
        $BC$.

 Because $AOC$ is an isosceles triangle with the base $AC$, by the
 \ref{enakokraki} and \ref{zunanjiNotrNotr} we have
  $\angle ACO =\frac{1}{2}\angle AOB$. The point $D$ is the center
  of the arc $BD$, so $\angle AOD\cong\angle DOB$
  or $\angle DOB=\frac{1}{2}\angle AOB$. This means that
   $\angle ACO\cong\angle DOB$ and the lines $AC$ and $DO$ are parallel by
   \ref{KotiTransverzala}. Because $AD\parallel JO$ by assumption, the
   quadrilateral $ADOJ$ is a parallelogram, so $AJ\cong OD$. Therefore
   $$AJ\cong OD\cong OE\cong AF\cong AE.$$
   From $AF\cong AE$ it follows that
  $AJ$ is the line of symmetry of the internal angle at the vertex $C$ of the triangle $CEF$
   (\ref{SklTetSklObKot}). Because $AJ\cong  AF\cong AE$,
  by \ref{TockaN.NBNC} the point $J$ is the center
        of the inscribed circle of this triangle.
\kdokaz


%________________________________________________________________________________
 \poglavje{Cyclic Quadrilateral} \label{odd4Tetivni}

We say that a
\index{štirikotnik!tetiven}\index{večkotnik!tetiven}\pojem{tetiven},
if there exists a circumscribed circle, or if there is a circle that
contains all of its vertices (Figure \ref{sl.skk.4.5.10.pic}). For
vertices in this case we say that they are \index{konciklične
točke}\pojem{konciklične točke}. We have already seen that every
triangle is tetiven (izrek \ref{SredOcrtaneKrozn}) and also that every
regular polygon is tetiven (izrek \ref{sredOcrtaneKrozVeck}). On the
other hand, it is clear that not all polygons are cyclic. For example,
a diamond is a quadrilateral that does not have a circumscribed
circle. In this section we will therefore deal with cyclic
quadrilaterals.

\begin{figure}[!htb]
\centering
\input{sl.skk.4.5.10.pic}
\caption{} \label{sl.skk.4.5.10.pic}
\end{figure}

Since a square is a regular polygon, it is also a cyclic
quadrilateral. It is not difficult to prove that a rectangle is also a
type of cyclic quadrilateral - the center of the circumscribed circle
is the intersection of its diagonals, which are consistent and
bisect each other. But how would we generally determine if a
quadrilateral is cyclic? It is clear that in a cyclic quadrilateral
(generally also in a polygon) the altitudes of all its sides intersect
in one point (Figure \ref{sl.skk.4.5.10.pic}). This condition is
sufficient for the quadrilateral to be cyclic, but it is not
sufficiently operative in specific cases. For the cyclicity of
quadrilaterals there is a necessary and sufficient condition that is
more useful.



             \bizrek \label{TetivniPogoj}
               A convex quadrilateral is cyclic if and only if
            its opposite interior angles are supplementary.
            Thus, if $\alpha$, $\beta$, $\gamma$ and $\delta$ are
            the interior angles of a convex quadrilateral $ABCD$,
            it is cyclic if and only if
                $$\alpha+\gamma=180^0.$$
            \eizrek

\begin{figure}[!htb]
\centering
\input{sl.skk.4.5.11.pic}
\caption{} \label{sl.skk.4.5.11.pic}
\end{figure}

\textbf{\textit{Proof.}}
 (Figure \ref{sl.skk.4.5.11.pic})

($\Rightarrow$) First, let's assume that the quadrilateral $ABCD$
is cyclic. Because it is convex, the vertices $A$ and $C$ are on
different sides of the line $BD$. By \ref{ObodObodKotNaspr}
$\alpha+\gamma=180^0$.

($\Leftarrow)$ Now let's assume that the opposite angles of the
quadrilateral $ABCD$ are supplementary, i.e. $\alpha+\gamma=180^0$.
Let $k$ be the circle drawn through the triangle $ABD$. In this
case, the fourth vertex $C$ of the line $BD$ is seen under the
angle complementary to the angle at the vertex $A$, which means
that the point $C$ also lies on the circle $k$ (\ref{ObodKotGMT}).
\kdokaz

  A direct consequence is the following theorem.



             \bizrek \label{TetivniPogojZunanji}
              A convex quadrilateral is cyclic if and only if
            one of its interior angles is congruent to the opposite exterior angle.
            Thus, if $\alpha$, $\beta$, $\gamma$ and $\delta$ are
            the interior angles and
            $\alpha'$, $\beta'$, $\gamma'$ in $\delta'$ the exterior angles
             of a convex quadrilateral $ABCD$,
            it is cyclic if and only if
                $$\alpha\cong\gamma'.$$
            \eizrek


\begin{figure}[!htb]
\centering
\input{sl.skk.4.5.12.pic}
\caption{} \label{sl.skk.4.5.12.pic}
\end{figure}

We use the criterion from \ref{TetivniPogoj} for a parallelogram
and a trapezoid.

            \bizrek \label{paralelogramTetivEnakokr}
            A parallelogram is cyclic if and only if it is a rectangle.
            \eizrek


\textbf{\textit{Proof.}} Let $\alpha$, $\beta$, $\gamma$ and $\delta$
be the interior angles of the parallelogram $ABCD$
 (Figure \ref{sl.skk.4.5.13.pic}).

($\Leftarrow$) If the parallelogram is a rectangle,
$\alpha+\gamma=90^0+90^0=180^0$, which means that $ABCD$ is a cyclic
quadrilateral (\ref{TetivniPogoj}).

($\Rightarrow$) Let's assume that $ABCD$ is a trapezoidal parallelogram.
 Because $ABCD$ is a parallelogram, according to Theorem \ref{paralelogram}
 $\alpha\cong\gamma$. Because it is also a trapezoid, according to Theorem \ref{TetivniPogoj}
$\alpha+\gamma=180^0$. So $\alpha\cong\gamma=90^0$, therefore
  $ABCD$ is a rectangle.
 \kdokaz

\begin{figure}[!htb]
\centering
\input{sl.skk.4.5.13.pic}
\caption{} \label{sl.skk.4.5.13.pic}
\end{figure}



            \bizrek \label{trapezTetivEnakokr}
            A trapezium is cyclic if and only if it is isosceles.
            \eizrek

\textbf{\textit{Proof.}} Let $ABCD$ be a trapezoid with a base $AB$ and with
internal angles $\alpha$, $\beta$, $\gamma$ and $\delta$
 (Figure \ref{sl.skk.4.5.13.pic}). In any trapezoid
  it holds that $\alpha+\delta=180^0$ and $\beta+\gamma=180^0$.


($\Leftarrow$) Let's assume that trapezoid $ABCD$ is isosceles, i.e. $AD
\cong BC$. According to Theorem \ref{trapezEnakokraki} in this case
$\alpha\cong\beta$. So $\alpha+\gamma=\beta+\gamma=180^0$, therefore
according to Theorem \ref{TetivniPogoj} $ABCD$ is a cyclic quadrilateral.

($\Rightarrow$) Let trapezoid $ABCD$ be a cyclic quadrilateral and $k$
its circumscribed circle. The bases $AB$ and $CD$ are parallel
chords of this circle, so they have a common perpendicular, which goes through
the center $S$ of the circle $k$ and is perpendicular to the chords $AB$ and
$CD$. This means that the legs $AD$ and $BC$ are symmetrical with respect to this perpendicular, so they are congruent and trapezoid $ABCD$  is isosceles.
 \kdokaz

 Particularly interesting are cyclic quadrilaterals with perpendicular
 diagonals\footnote{\index{Brahmagupta}\textit{Brahmagupta} (598--660), Indian mathematician, who
 studied such quadrilaterals.}.
 The following example applies to
such quadrilaterals.


            \bzgled \label{TetivniLemaBrahm}
            Suppose that the diagonals of a cyclic quadrilateral $ABCD$ are perpendicular and intersect
            at a point $S$. Prove that the foot of the perpendicular from the point $S$ on the line $AB$
            contains the midpoint of the line $CD$.
            \ezgled

\begin{figure}[!htb]
\centering
\input{sl.skk.4.5.14.pic}
\caption{} \label{sl.skk.4.5.14.pic}
\end{figure}

\textbf{\textit{Proof.}}
 (Figure \ref{sl.skk.4.5.14.pic})

Let $N$ and $M$ be the intersections of the rectangle with the line $AB$ through the point $S$ with the sides $AB$ and $CD$ of the quadrilateral $ABCD$. Then it holds:
 \begin{eqnarray*}
 \angle CDB &\cong& \angle CAB \hspace*{3mm}
 \textrm{(external angle for the appropriate locus } CB
 \textrm{ - izrek \ref{ObodObodKot}})\\
      &\cong& \angle NSB  \hspace*{3mm}
 \textrm{ (angle with
perpendicular arms - izrek \ref{KotaPravokKraki})}\\
     &\cong& \angle MSD  \hspace*{3mm}
 \textrm{(perfect angle)}
 \end{eqnarray*}
 Because $\angle CDB\cong \angle MSD$, $MD \cong MS$ (izrek \ref{enakokraki}).
 Similarly, $MC \cong MS$. Therefore,
 $MD \cong MC$, which means that $M$ is the center of the side $CD$.
 \kdokaz

We will consider one property of cyclic quadrilaterals with perpendicular diagonals
in the example \ref{HamiltonPoslTetiv}.



             \bzgled \label{TetŠtirZgl0}
             Let $k$ be the circumcircle of a cyclic quadrilateral $ABCD$
            and $N$, $M$, $L$ and $P$ the midpoints of those arcs $AB$, $B$C, $CD$ and $AD$
            of the circle $k$, not containing the third vertices of this quadrilateral.
            Prove that $NL\perp PM$.
            \ezgled


\begin{figure}[!htb]
\centering
\input{sl.skk.4.5.0.pic}
\caption{} \label{sl.skk.4.5.0.pic}
\end{figure}

\textbf{\textit{Proof.}} Let $S$ be the intersection of the lines $NL$ and $PM$
(Figure \ref{sl.skk.4.5.0.pic}).
 If we use the izrek \ref{ObodObodKot} and \ref{TockaN} twice, we get:

 \begin{eqnarray*}
  \angle PNS &=& \angle PND +\angle DNL =
\angle PBD +\angle DBL =\\
 &=& \frac{1}{2} \angle ABD +\frac{1}{2}\angle CBD = \frac{1}{2}\angle
 ABC.
 \end{eqnarray*}

We similarly prove that $\angle NPS = \frac{1}{2}\angle
 ADC$. Therefore, according to the statement \ref{TetivniPogoj}:
 $$\angle PNS +\angle NPS = \frac{1}{2} \left(\angle ABC+\angle
 ADC\right)=90^0.$$
 If we use the statement \ref{VsotKotTrik} for the triangle $PSN$, we get
 $\angle PSN = 90^0$.
  \kdokaz


              \bzgled \label{TetivniVcrtana}
             Let $ABCD$ be a cyclic quadrilateral.
             Prove that incentres of the triangles $BCD$, $ACD$, $ABD$ and $ABC$
             are the vertices of a rectangle.
             \ezgled

\begin{figure}[!htb]
\centering
\input{sl.skk.4.5.1.pic}
\caption{} \label{sl.skk.4.5.1.pic}
\end{figure}

\textbf{\textit{Proof.}} We mark with $A_1$, $B_1$, $C_1$ and $D_1$
the incentres of the triangles $BCD$, $ACD$, $ABD$ and $ABC$
and with $N$, $M$, $L$ and $P$ the incentres of those arcs $AB$, $BC$, $CD$ and
$AD$ of the cyclic quadrilateral $ABCD$, which do not contain the other
vertices of this quadrilateral (Figure \ref{sl.skk.4.5.1.pic}). From the statement
\ref{TockaN} it follows that $BL$ and $DM$ are the angle bisectors of the angles $CBD$ and
$BDC$, therefore the point $A_1$ is the intersection of the lines $BL$ and $DM$. Similarly,
the point $B_1$ is the intersection of the lines $CP$ and $AL$. According to the statement
\ref{TockaN.NBNC},
 $LC\cong LA_1\cong LB_1\cong LD$, therefore $A_1LB_1$ is an isosceles triangle
 with the base $A_1B_1$. From the statement \ref{TockaN} it also follows that
 $LN$ is the angle bisector of the angle $ALB$ or $B_1LA_1$. In an isosceles
 triangle $A_1LB_1$ the angle bisector of the angle $B_1LA_1$ contains the altitude
 of this triangle from the point $L$. This means that $LN\perp
 A_1B_1$ holds. Similarly,
 $LN\perp C_1D_1$,  $PM\perp A_1D_1$
 and
 $PM\perp C_1B_1$ hold. From the previous statement \ref{TetŠtirZgl0} we know that $LN\perp PM$,
 therefore the quadrilateral $A_1B_1C_1D_1$ is a rectangle.
  \kdokaz

We have already mentioned that a rectangle is a right-angled quadrilateral. Now we will
prove an interesting property of rectangles that relates to points
that lie on its circumscribed circle.



            \bzgled
             Let $P$ be an arbitrary point on the shorter arc $AB$ of the circumcircle of a rectangle $ABCD$.
            Suppose that $L$ and $M$ are the foots of the perpendiculars from the point $P$ on the
            diagonals $AC$ and $BD$, respectively. Prove that the length of the line segment $LM$ does not depend
            on the position of the point $P$.
            \ezgled


\begin{figure}[!htb]
\centering
\input{sl.skk.4.5.15.pic}
\caption{} \label{sl.skk.4.5.15.pic}
\end{figure}

\textbf{\textit{Proof.}}
 Let $O$ be the center of the circle $k$ (Figure \ref{sl.skk.4.5.15.pic}).
The quadrilateral $PMOL$ is a right-angled one, because $\angle OLP + \angle OMP
=90^0+90^0= 180^0$ (statement \ref{TetivniPogoj}). We mark with $l$
the circumscribed circle of this quadrilateral. Because the angles $OLP$ and $OMP$ are both
right, the distance $OP$ (or the radius of the circle $k$) is the radius of the
circle $l$. Then $LM$ is the chord of the circle $l$, which belongs to the peripheral angle
$\angle LOM =\angle AOB$, which is constant. Regardless of the choice
of the point $P$, the distance $LM$ is the chord of the circle with a constant radius
$OA$, which belongs to a constant peripheral angle $AOB$ (or the corresponding
constant central angle of this circle). The chords, which belong
to the corresponding central angles of the corresponding circles, are proportional to each other,
so the length of the distance $LM$ does not depend on the position of the point $P$.
 \kdokaz

  The properties of the right-angled quadrilateral are often used to prove
  various
properties of triangles.


            \bzgled \label{PedalniVS}
            The orthocentre of an acute triangle is the incentre of its \index{trikotnik!pedalni} pedal triangle.
            \ezgled

\begin{figure}[!htb]
\centering
\input{sl.skk.4.5.16.pic}
\caption{} \label{sl.skk.4.5.16.pic}
\end{figure}

\textbf{\textit{Proof.}}
 Let $AA'$, $BB'$ and $CC'$ be the altitudes of the triangle $ABC$, which intersect
 in the point $V$ of
the altitude of this
triangle (Figure \ref{sl.skk.4.5.16.pic}). If $A_1$ is the midpoint
of the side $BC$, the points $B'$ and $C'$ lie on the circle $k(A_1,A_1B)$
(statement \ref{TalesovIzrKroz2}). Therefore, the quadrilateral $BC'B'C$
is cyclic, so according to \ref{TetivniPogojZunanji} $\angle
AC'B'\cong \angle ACB = \gamma$. Similarly, we prove that the quadrilateral $AC'A'C$ is cyclic, so $\angle BC'A'\cong
\angle ACB = \gamma$. Therefore, the angles $AC'B'$ and $BC'A'$ are congruent. Because
$CC'\perp AB$, the angles $CC'B'$ and $CC'A'$ are also congruent. This
means that the line $C'C$ is perpendicular to the angle $A'C'B'$. Similarly, the lines $A'A$ and $B'B$ are perpendicular to the corresponding internal angles
of the triangle $A'B'C'$, so the point $V$ is the centre of the triangle $A'B'C'$
of the inscribed circle.
 \kdokaz

 From the proof of the previous statement (\ref{PedalniVS}) we can conclude that the angles, which are determined by the sides
of the pedal triangle $A'B'C'$  with the sides of the triangle
$ABC$, are equal to the corresponding angles of the triangle $ABC$. We will use this fact in the following example.



            \bzgled \label{PedalniLemaOcrtana}
             Let $O$ be the circumcentre of a triangle $ABC$.
            Prove that the lines $OA$, $OB$ and $OC$ are perpendicular to the corresponding sides of the
            pedal triangle $A'B'C'$.
            \ezgled


\begin{figure}[!htb]
\centering
\input{sl.skk.4.5.17.pic}
\caption{} \label{sl.skk.4.5.17.pic}
\end{figure}

\textbf{\textit{Proof.}} (Figure \ref{sl.skk.4.5.17.pic}).

Let $L$ denote the intersection of the lines $OA$ and $B'C'$. It is enough to prove that the internal angle at the vertex $L$ of the triangle $C'LA$ is a right angle. Let us calculate the other two angles of this triangle. From the previous example \ref{PedalniVS} the angle at the vertex $C'$ is equal to $\gamma$. The triangle $AOB$ is isosceles and $\angle AOB=2\gamma$ (statement \ref{SredObodKot}). Therefore (statements \ref{enakokraki} and \ref{VsotKotTrik}) $\angle C' AL=\angle BAO =90^0-\gamma$, which implies that $\angle ALC'=90^0$.
 \kdokaz

A direct consequence of the statement \ref{PedalniVS} and \ref{PedalniLemasPQR} is the following statement.



            \bzgled \label{PedalniLemasLMN}
            Let $P$, $Q$ and $R$ be the midpoints of those arcs $BC$, $AC$ and $AB$
            of the circumcircle of a triangle $ABC$ not containing the vertices $A$, $B$ and $C$.
            Suppose that the point $S$ is the incentre of the triangle $ABC$
             and $L=SA\cap QR$, $M=SB\cap PR$ and $N=SC\cap PQ$. Then the triangles $LMN$ and $ABC$ have
            the common incentre.
             (Figure \ref{sl.skk.4.5.18.pic}).
            \ezgled


\begin{figure}[!htb]
\centering
\input{sl.skk.4.5.18.pic}
\caption{} \label{sl.skk.4.5.18.pic}
\end{figure}

\bzgled \label{Miquelova točka}
             Let $P$, $Q$ and $R$ be an arbitrary points on the sides $BC$, $AC$ and $AB$
             of the triangle $ABC$, respectively. Prove that the circumcircles of triangles
              $AQR$, $PBR$ and $PQC$ intersect at in one point (so-called \index{točka!Miquelova}
            \pojem{Miquel point}\color{green1}\footnote{The point is named after
            the French mathematician \index{Miquel, A.} \textit{A. Miquel} (1816–-1851), who published
            this statement in 1838 as an article in Liouville's
            (\index{Liouville, J.}\textit{J. Liouville} (1809–-1882), French
            mathematician) journal. But, as is often the case in mathematics, Miquel
            was not the first to prove this statement. Ten years before him, this fact was
            discovered and published by the famous Swiss mathematician
            \index{Steiner, J.} \textit{J. Steiner} (1769--1863).}).
            \ezgled


\begin{figure}[!htb]
\centering
\input{sl.skk.4.5.2.pic}
\caption{} \label{sl.skk.4.5.2.pic}
\end{figure}

\textbf{\textit{Proof.}} (Figure \ref{sl.skk.4.5.2.pic})

We denote with $k_A$, $k_B$ and $k_C$ the circumcircles of triangles
$AQR$, $PBR$ and $PQC$ and the internal angles of triangle $ABC$ in
order with $\alpha$, $\beta$ and $\gamma$. Let $S$ be the other
intersection of the circles $k_B$ and $k_C$ (the proof is similar in
the case when $S = P$). Quadrilateral $BPSR$ and $PCQS$ are
tangential, so $\angle RSP = 180^0 - \beta$ and $\angle QSP = 180^0
-\gamma$ (statement \ref{TetivniPogoj}). From this it follows that
$\angle RSQ = \beta +\gamma$ and then also $\angle RAQ + \angle RSQ
=\alpha + \beta +\gamma = 180^0$. Quadrilateral $ARSQ$ is also
tangential (statement \ref{TetivniPogoj}) or it has its own
circumcircle, which is actually the circle $k_A$, circumscribed to
triangle $AQR$. This means that the circles $k_A$, $k_B$ and $k_C$
intersect in point $S$.
 \kdokaz

In this chapter \ref{pogINV} we will prove one generalization of the previous
statement (see example \ref{MiquelKroznice}).



        \bnaloga\footnote{45. IMO Greece - 2004, Problem 1.}
          Let $ABC$ be an acute-angled triangle with $AB\neq AC$. The
circle with diameter $BC$ intersects the sides $AB$ and $AC$ at $M$ and $N$,
respectively. Denote by $O$ the midpoint of the side $BC$. The bisectors of
the angles $\angle BAC$ and $\angle MON$ intersect at $L$. Prove that the circumcircles
of the triangles $BML$ and $CNL$ have a common point lying on the side
$BC$.
         \enaloga


\begin{figure}[!htb]
\centering
\input{sl.skk.4.4.IMO1.pic}
\caption{} \label{sl.skk.4.4.IMO1.pic}
\end{figure}


\textbf{\textit{Proof.}} We mark with $E$ the intersection of the angle bisector
of $BAC$ with the side $BC$ of the triangle $ABC$ (Figure
\ref{sl.skk.4.4.IMO1.pic}). We prove that $E$ is the desired point  i.e.
that it lies on the circumcircles
of both triangles $BML$
        and $CNL$.

Because from the construction of the points $M$ and $N$ it follows that $OM\cong ON$, the triangle $OMN$
is isosceles. This means that the bisector of $OL$ is also the bisector of the side $MN$ (follows from the similarity of the triangles $MSO$ and $NSO$, where $S$ is the center
of the segment $MN$). Therefore, the point $L$ lies on the bisector of the segment $MN$
 of the triangle, so by  \ref{TockaN} it lies on the circumcircle
 $k$
 of the triangle $AMN$. The condition $AB\neq AC$ tells us that the angle bisectors of $BAC$ and
 the side $MN$ (or the angle bisector of $MON$) are different, so their intersection is a point.

  If we use  \ref{TetivniPogojZunanji} and \ref{ObodObodKot},
 we get:
  \begin{eqnarray*}
   \angle BCA &\cong& AMN \cong\angle ALN,\\
   \angle ABC &\cong& ANM \cong\angle ALM.
  \end{eqnarray*}
From these relations and  \ref{TetivniPogojZunanji} it follows that $NLEC$ and $LMBE$ are tangential
quadrilateral. Therefore, the point $E$ lies on the circumcircles
of both triangles $BML$
        and $CNL$.
 \kdokaz

We say that a
\index{tetragon!tangential}\index{polygon!tangential}\pojem{tangential}
\index{tetragon!tangential}\index{polygon!tangential}\pojem{tangential},
if there exists an inscribed circle, or if there is such a
circle that the normals of all the sides of the polygon are its tangents
(Figure \ref{sl.skk.4.6.1.pic}). We have already seen that every
triangle is tangential (\ref{SredVcrtaneKrozn}) and also
a regular polygon is tangential (\ref{sredVcrtaneKrozVeck}). On the other hand, it is clear that not all
polygons are tangential. For example, a rectangle is a tetragon which does not have
an inscribed circle. In this section we will focus on
tangential tetragons.

\begin{figure}[!htb]
\centering
\input{sl.skk.4.6.1.pic}
\caption{} \label{sl.skk.4.6.1.pic}
\end{figure}

Since a square is a regular polygon, it is also a tangential tetragon. But how would we in general determine whether a tetragon is tangential? It is clear that in a tangential tetragon (in general also in a polygon) the simetrals of all of its internal angles intersect in one point (Figure
\ref{sl.skk.4.6.1.pic}). This condition is sufficient for the tangentiality of a polygon.
Unfortunately, this condition is not very useful in
specific cases. There is a more useful condition that is necessary and
sufficient for the tangentiality of tetragons.



             \bizrek \label{TangentniPogoj}
              A quadrilateral $ABCD$ is tangential if and only if
               $$|AB| + |CD| = |BC| + |AD|.$$
            \eizrek

\begin{figure}[!htb]
\centering
\input{sl.skk.4.6.2.pic}
\caption{} \label{sl.skk.4.6.2.pic}
\end{figure}


\textbf{\textit{Proof.}}  (Figure \ref{sl.skk.4.6.2.pic})

($\Rightarrow$) First, let's assume that the quadrilateral $ABCD$ is tangent and
  $k$ is its inscribed circle.
 Let $P$, $Q$, $R$ and $S$
be the points of tangency of sides $AB$, $BC$, $CD$ and $DA$ with the circle $k$. Because
the appropriate tangent lines are concurrent (by Theorem \ref{TangOdsek}), it holds:
$AP \cong AS$, $BP \cong BQ$, $CQ \cong CR$ and $DR \cong DS$. Therefore
 \begin{eqnarray*}
|AB| + |CD|&=&|AP| + |PB| + |CR| + |RD| \\&=& |AS| + |SD| + |BQ| +
|QC|\\&=&|AD| + |BC|.
 \end{eqnarray*}

 ($\Leftarrow$) We prove the converse statement. Let's assume that in the quadrilateral
 $ABCD$ the sums of the pairs of opposite sides are equal, i.e.
 $|AB| + |CD| = |BC| + |AD|$. There exists a circle $k$, which touches
sides $AB$, $BC$ and $DA$ of this quadrilateral (its center is
the intersection of the internal angle bisectors at vertices $A$ and $B$ of this
quadrilateral). We prove that this circle also touches side
$CD$ of the quadrilateral $ABCD$. Let $D'$ be the intersection of the other tangent from
point $C$ of the circle $k$ and the line $AD$. Let's assume that $D'\neq
D$. Without loss of generality, let $\mathcal{B}(A,D',D)$. Because the quadrilateral $ABCD'$ is tangent to the circle $k$, by the already proven part of the theorem it holds $|AB| + |CD'| = |AD'|+|BC|$. But since by the assumption also $|AB| + |CD| = |AD| + |BC|$, it also holds $|CD|-|CD'| = |DA| - |D'A|=|DD'|$ i.e.  $|CD|= |CD'| + |DD'|$. But this is not possible due to the triangle inequality \ref{neenaktrik} (points $C$,
$D$ and $D'$ cannot be collinear, because otherwise it would hold $C\in
AD$). In a similar way we arrive at a contradiction also in the case when $\mathcal{B}(A,D,D')$. Therefore it holds $D'= D$, thus $ABCD$ is a tangential quadrilateral.
 \kdokaz

 The following theorems are a direct consequence of the previous criterion.

        \bizrek \label{TangDeltoidRomb}
        A rhombus, a deltoid, and a square are
         tangential quadrilaterals (Figure \ref{sl.skk.4.6.3.pic}).
        \eizrek

\begin{figure}[!htb]
\centering
\input{sl.skk.4.6.3.pic}
\caption{} \label{sl.skk.4.6.3.pic}
\end{figure}

         \bizrek \label{TangParalelogram}
          A parallelogram is a tangential quadrilateral
          if and only if it is a rhombus (Figure \ref{sl.skk.4.6.3.pic}).
        \eizrek

In the next two examples we will consider
tension and tangent quadrilaterals at the same time.



        \bzgled Let $k_A$, $k_B$, $k_C$ and $k_D$ circles with centres $A$,
        $B$, $C$ and $D$, such that two in a row (also $k_A$ and $k_D$)
        are touching each other externally. Prove that the quadrilateral defined by
        the touching points of circles is cyclic, and the quadrilateral $ABCD$ is tangential.
        \ezgled

\begin{figure}[!htb]
\centering
\input{sl.skk.4.6.4.pic}
\caption{} \label{sl.skk.4.6.4.pic}
\end{figure}


\textbf{\textit{Proof.}}
 Let $P$, $Q$, $R$ and $S$ be the touching points of
the given circles in order, and $p$ and $r$ the common tangents of
the corresponding circles at points $P$ and $R$ (Figure \ref{sl.skk.4.6.4.pic}).

First, we have:
 \begin{eqnarray*}
 |AD| + |BC| &=& |AP| + |PD| + |BR| + |RC| =\\
  &=& |AQ| + |SD| + |QB| + |SC| =\\
  &=&  |AB| +
|CD|.
 \end{eqnarray*}
  Therefore, $ABCD$ is a tangent quadrilateral (statement \ref{TangentniPogoj}).

Tangents $p$ and $r$ divide the internal angle at vertices $P$ and $R$
of the quadrilateral $PQRS$ into angles, each of which is equal to half
of the corresponding central angle (statement \ref{ObodKotTang}). The aforementioned
central angles are the internal angles of the quadrilateral $ABCD$. We denote
them with $\alpha$, $\beta$, $\gamma$ and $\delta$. Therefore, (statement
\ref{VsotKotVeck}):
 \begin{eqnarray*}
 \angle QPS+ \angle SRQ&=& \angle QP,p+\angle p,PS+ \angle SR,r+\angle r,RQ=\\
  &=& \frac{1}{2}\alpha+\frac{1}{2}\delta+\frac{1}{2}\gamma+\frac{1}{2}\beta=\\
  &=& \frac{1}{2}\left(\alpha+\delta+\gamma+\beta\right)=\\
  &=& \frac{1}{2}\cdot360^0=180^0,
 \end{eqnarray*}
which means that $PQRS$ is a tension quadrilateral.
  \kdokaz

It is clear that the inscribed
 circle of the quadrilateral $ABCD$ is also the circumscribed circle of the quadrilateral $PQRS$. Because according to the assumption
 $PAQ$, $QBR$, $RCS$ and $SDP$ are equilateral triangles with bases
 $PQ$, $QR$, $RS$ and $SP$, the simetrals of the internal angles of the quadrilateral $ABCD$
 are also the simetrals of the sides of the quadrilateral $PQRS$ (Figure
 \ref{sl.skk.4.6.4a.pic}). This is also the second (simpler)
 way to prove the second part of the previous example - the assertion that $PQRS$
 is a tangential quadrilateral.

\begin{figure}[!htb]
\centering
\input{sl.skk.4.6.4a.pic}
\caption{} \label{sl.skk.4.6.4a.pic}
\end{figure}



            \bzgled \label{tetivTangLema}
            Let $L$ be the intersection of the diagonals of a cyclic quadrilateral  $ABCD$.
            Prove that the foots of the perpendiculars from the point $L$ on the sides of
            this quadrilateral are the vertices of a tangential quadrilateral.
            \ezgled


\begin{figure}[!htb]
\centering
\input{sl.skk.4.6.5.pic}
\caption{} \label{sl.skk.4.6.5.pic}
\end{figure}


        \textbf{\textit{Proof.}}
  Let $P$, $Q$, $R$ and $S$ be the perpendicular projections from the point $L$ on the sides
   $AB$, $BC$, $CD$ and $DA$
of the quadrilateral $ABCD$ (Figure \ref{sl.skk.4.6.5.pic}).
Because of the appropriate right angles, $PBQL$ and $APLS$
are tangential quadrilaterals (statement \ref{TetivniPogoj}). According to the assumption, $ABCD$ is also tangential. If we
use this, we get the equality of the appropriate external angles (statement \ref{ObodObodKot}). Therefore:
$$\angle SPL \cong \angle SAL = \angle DAC \cong \angle DBC
= \angle LBQ \cong \angle LPQ.$$
From this it follows that the line $PL$ is the simetral of the internal angle at
the vertex $P$ of the quadrilateral $PQRS$. Similarly, the lines $QL$, $RL$ and $SL$
are the simetrals of the other three internal angles of this quadrilateral. Therefore, $L$ is the center of the inscribed circle of the quadrilateral $PQRS$,
so this is tangential.
        \kdokaz

We will now prove another interesting property of tangential quadrilaterals.

\bzgled
            Prove that the incircles of triangles $ABC$ and $ACD$ touch each
            other if and only if $ABCD$ is a tangential quadrilateral.
            \ezgled


\begin{figure}[!htb]
\centering
\input{sl.skk.4.6.6.pic}
\caption{} \label{sl.skk.4.6.6.pic}
\end{figure}


        \textbf{\textit{Proof.}}
   (Figure \ref{sl.skk.4.6.6.pic}).

   First, we prove some relations that hold for any convex quadrilateral $ABCD$.
Let $P$, $Q$ and $X$ be the points in which the inscribed circle $k$ of triangle $ABC$ touches its sides
$AB$, $BC$ and $CA$, and $R$, $S$ and $Y$  be the points in which the inscribed circle $l$ of triangle $ACD$ touches its sides $CD$, $DA$ and $AC$. First, it holds (from \ref{TangOdsek}):
 \begin{eqnarray*}
 |AX| &=& |AP|=\frac{1}{2}\left(|AX|+|AP|\right)=\frac{1}{2}\left(|AC|-|CX|+|AB|-|BP|\right)\\
  &=& \frac{1}{2}\left(|AC|-|CQ|+|AB|-|BQ|\right)=
  \frac{1}{2}\left(|AC|+|AB|-|BC|\right),
 \end{eqnarray*}
 therefore it holds:
  $$|AX|=\frac{1}{2}\left(|AC|+|AB|-|BC|\right).$$
  In the same way, we prove that it also holds:
  $$|AY|=\frac{1}{2}\left(|AC|+|AD|-|DC|\right).$$
  Now we can start with proving the equivalence.

The circles $k$ and $l$ touch each other exactly when $X = Y$ or $|AX| = |AY|$. The last equality holds exactly when: $$\frac{1}{2}\left(|AC|+|AB|-|BC|\right)=\frac{1}{2}\left(|AC|+|AD|-|DC|\right)$$
or $|AB| + |DC| = |AD| + |BC|$, which is fulfilled exactly when $ABCD$ is a tangential quadrilateral (from \ref{TangentniPogoj}).
   \kdokaz

The consequence of this is the following claim.

\bzgled
            Let $ABCD$ be a tangential quadrilateral.
            Then the  incircles of the triangles $ABC$ and $ACD$ touch each other
            if and only if
            the  incircles of the triangles $ABD$ and $CBD$ touch each
            other (Figure \ref{sl.skk.4.6.6a.pic}).
            \ezgled


\begin{figure}[!htb]
\centering
\input{sl.skk.4.6.6a.pic}
\caption{} \label{sl.skk.4.6.6a.pic}
\end{figure}


        \textbf{\textit{Proof.}} The statements that the incircles of the triangles $ABC$ and $ACD$ or the triangles $ABD$ and $CBD$ touch, are
        equivalent to the statement that the quadrilateral $ABCD$ is tangent. This means that the initial statements are equivalent.
         \kdokaz

%_______________________________________________________________________________
 \poglavje{Bicentric Quadrilateral} \label{odd4TetivniTangentni}

  Some quadrilaterals are both tangential and chordal. We call them \index{štirikotnik!tetivnotangentni}\pojem{tetivnotangentni} or \index{štirikotnik!bicentrični}\pojem{bicentrični} quadrilaterals. Which
 quadrilaterals are these? The square is certainly one of them. Is it the only one? The answer is negative. The quadrilateral we get from the example \ref{tetivTangLema} is always tangent. In a certain case it will be chordal as well.
 Namely, the following statement is true.



                \bizrek \label{tetivTangIzrek}
                If $L$ is the intersection of the perpendicular diagonals of a cyclic quadrilateral
            $ABCD$, then the foots of the perpendiculars from the point $L$ on the sides of
            this quadrilateral are the vertices of a bicentric quadrilateral.
                \eizrek



\begin{figure}[!htb]
\centering
\input{sl.skk.4.7.2.pic}
\caption{} \label{sl.skk.4.7.2.pic}
\end{figure}


        \textbf{\textit{Proof.}}
   (Figure \ref{sl.skk.4.7.2.pic})

Let's use the same notation as in the example \ref{tetivTangLema}. We have already proven that $PQRS$ is a tangent quadrilateral. We will now prove that it is also a bicentric quadrilateral. From the proof of the statement in the aforementioned example \ref{tetivTangLema} it follows:
\begin{eqnarray*}
        \angle SPQ &=& \angle SPL+\angle LPQ = \angle SAL+\angle LBQ =\\
          &=& \angle DAC + \angle DBC
        = 2\cdot\angle DBC
\end{eqnarray*}
or $\angle SPQ= 2\cdot\angle DBC$. Similarly, $\angle SRQ= 2\cdot\angle ACB$.
   Because, by assumption, $AC\perp BD$, $CLB$ is a right angled triangle, therefore:
  $$\angle SRQ+\angle SRQ=2\cdot(\angle DBC+\angle ACB)=2\cdot 90^0=180^0.$$
   By  \ref{TetivniPogoj} $PQRS$ is a bicentric quadrilateral.
   \kdokaz

 It is not difficult to convince oneself that the converse statement is also true.



        \bizrek
        Let $PQRS$ be a bicentric quadrilateral. Suppose that point $L$ is
        the incentre of this quadrilateral and at the same time the intersection of the diagonals
        of a cyclic quadrilateral $ABCD$. If $P$, $Q$, $R$ and $S$ the foots of the perpendiculars
        from the point $L$ on the sides of quadrilateral $ABCD$, then $AC\perp BD$.
        \eizrek

In the next exercise we will see that for three non-collinear points $A$, $B$ and $C$ there is only one point $D$, such that $ABCD$ is a bicentric quadrilateral.


            \bnaloga\footnote{4.
            IMO Czechoslovakia - 1962, Problem 5.}
             On the circle $k$ there are given three distinct points $A$, $B$, $C$. Construct (using
            only straightedge and compasses) a fourth point $D$ on $k$ such that a circle
            can be inscribed in the quadrilateral thus obtained.
            \enaloga

\begin{figure}[!htb]
\centering
\input{sl.skk.4.5.IMO1.pic}
\caption{} \label{sl.skk.4.5.IMO1.pic}
\end{figure}


\textbf{\textit{Solution.}} Without loss of generality we can assume
that $AB\geq BC$.

Let $D$ be a point that satisfies the conditions of the task, or such that $ABCD$ is a tangent-chord quadrilateral (Figure \ref {sl.skk.4.5.IMO1.pic}). We denote by $a$, $b$, $c$ and $d$ the sides of $AB$, $BC$, $CD$ and $DA$ of this quadrilateral, and by $\alpha$, $\beta$, $\gamma$ and $\delta$ its internal angles at vertices $A$, $B$, $C$ and $D$. From the condition of the tangency of the quadrilateral $ABCD$ (formula \ref {TetivniPogoj}) it follows that $\delta = 180^0 - \beta$, and from its tangency (formula \ref {TangentniPogoj}) that $a + c = b + d$ or $d-c = a-b$. In this way, the task is reduced to the design of the third vertex $D$ of the triangle $ACD$, where the sides $AC$, the angle $\angle ADC = 180^0 - \beta$ and the difference of sides $AD-CD = AB-BC = a-b$ are given. Let $E$ be a point on the line $AD$, for which $DE \cong DC$. Then $AE = AD-DE = AD-CD = a-b$. The triangle $ECD$ is isosceles, so by formula \ref {enakokraki} it follows that $\angle CED \cong \angle DCE$. Therefore, $\angle AEC = 180^0 - \angle DEC = 180^0 - \frac {1}{2} \beta$. This allows us to construct the triangle $ACE$ or the point $E$.

First, we plan the point $E$ as the intersection of the arc $l$ (see the construction described in formula \ref {ObodKotGMT}) $180^0 - \frac {1}{2} \angle ABC$) and the circle $j (A, AB-BC)$ (if $AB \cong AC$, we assume $E = A$). The point $D$ can then be designed as the intersection of the strip $AE$ and the similitude $s_{EC}$ of the line $EC$ (in the case $AB \cong AC$, or $E = A$, $D$ is the second intersection of the similitude $s_{EC} = s_{AC}$ with the circle $k$).

We prove that the point $D$ satisfies the conditions of the task, or that $ABCD$ is a tangent-chord quadrilateral. First, we will consider the case when $AB>BC$.

By construction, the point $D$ lies on the line of symmetry $EC$, so $DE\cong DC$ and also (by statement \ref{enakokraki}) $\angle DEC\cong\angle DCE$. We have drawn the point $E$ so that it lies on the arc $l$ with the string $AC$ and the angular measure $180^0-\frac{1}{2}\angle ABC$, so $\angle AEC=180^0-\frac{1}{2}\angle ABC$. Because, by construction, $\mathcal{B}(A,E,D)$, we have $\angle DCE\cong\angle DEC=\frac{1}{2}\angle ABC$. From the isosceles triangle $EDC$ by statement \ref{VsotKotTrik} it follows that $\angle ADC=\angle EDC=180^0-\angle ABC$. Therefore, $\angle EDC+\angle ABC=180^0$, so by statement \ref{TetivniPogoj} the quadrilateral $ABCD$ is a string quadrilateral, or $D\in k$.

 We will now prove that the quadrilateral $ABCD$ is a tangent quadrilateral. In the first part of the proof (the string property) we have already seen that $DE\cong DC$. The point $E$ by construction lies on the circle $j(A,|AB-BC|)$, so $|AE|=|AB|-|BC|$. Because $\mathcal{B}(A,E,D)$ also holds, we have $|AD|-|CD|=|AD|-|DE|=|AE|=|AB|-|BC|$. From this it follows that $|AD|+|BC|=|AB|+|CD|$ and by statement \ref{TangPogoj} the quadrilateral $ABCD$ is tangent.

 If $AB\cong BC$, the point $D$ by construction already lies on the circle $k$. Because both points $B$ and $D$ lie on the line of symmetry $AC$, the quadrilateral $ABCD$ is a deltoid, so it is also tangent (by statement \ref{TangDeltoidRomb}).

 We will now investigate the number of solutions to the problem. The circle $k(A,AB-AC)$ and the arc $l$ always intersect in one point $E$. Because $ABC<180^0$, we have $\angle AEC=180^0-\frac{1}{2}\beta>90^0$. This means that the line of symmetry $s_{EC}$ always intersects the half-line $AE$ in one point $D$ and $\mathcal{B}(A,E,D)$ holds. This means that the problem always has one and only one solution.
  \kdokaz




  %______________________________________________________________________________
 \poglavje{Simson Line} \label{odd4Simson}

We will first prove the basic statement.

\bizrek \label{SimpsPrem}
The foots of the perpendiculars from an arbitrary point lying on the circumcircle of a triangle to the  lines containing the sides of this triangle  are three collinear points. The line containing these points is the so-called \pojem{Simson\footnote{Premico imenujemo po škotskem matematiku \index{Simson, R.} \textit{R. Simsonu} (1687--1768), čeprav je to lastnost prvi objavil škotski matematik \index{Wallace, W.}  \textit{W. Wallace} (1768--1843) šele leta 1799.} line}\color{blue}.
\eizrek

\begin{figure}[!htb]
\centering
\input{sl.skk.4.7.1a.pic}
\caption{} \label{sl.skk.4.7.1a.pic}
\end{figure}

\textbf{\textit{Proof.}} Let $S$ be an arbitrary point of the circumcircle $k$ of the triangle $ABC$ and $P$, $Q$ and $R$ the orthogonal projections of the point $S$ on the lines containing the sides $BC$, $AC$ and $AB$ (Figure \ref{sl.skk.4.7.1a.pic}). Without loss of generality, we assume that $\mathcal{B}(B,P,C)$, $\mathcal{B}(A,Q,C)$ and $\mathcal{B}(A,B,R)$ hold. In this case, the points $Q$ and $R$ are on different sides of the line $BC$, so it is enough to prove $\angle BPR \cong \angle CPQ$. Because of the appropriate right angles and the position of the point $S$, the quadrilaterals $BRSP$, $ABSC$, $ARSQ$ and $SPQC$ are cyclic, so (from izrek \ref{TetivniPogoj}:
\begin{eqnarray*}
 \angle BPR &=& \angle BSR = \angle RSC - \angle BSC=\\
&=& \angle RSC - (180° - \angle BAC) =\\
&=& \angle RSC - \angle RSQ = \angle CSQ = \angle CPQ,
 \end{eqnarray*}
which means that the points $P$, $Q$ and $R$ are collinear. \kdokaz

In the following we will consider further interesting properties of Simson's line. Because each point $X$, which lies on the circumcircle of some triangle, determines the Simson line, we will denote this line by $x$. In this way, each triangle determines one mapping $X\mapsto x$.

\bzgled \label{SimsZgled1}
            Let $P$ be an arbitrary point of the circumcircle $k$ of a triangle
            $ABC$. Suppose that $P_A$ is the intersection of the perpendicular line of
            the line $BC$ through the point $P$ with the circle $k$.
            Prove that the line $AP_A$ is parallel to the Simson line $p$
            of the triangle at the point $P$.
            \ezgled


\begin{figure}[!htb]
\centering
\input{sl.skk.4.7.1b.pic}
\caption{} \label{sl.skk.4.7.1b.pic}
\end{figure}

 \textbf{\textit{Proof.}}
 Let $X$, $Y$ and $Z$ be the orthogonal projections of the point $P$ on the lines
  $BC$, $AC$ and $AB$ (Figure \ref{sl.skk.4.7.1b.pic}). By izreku \ref{SimpsPrem} the Simson line
  $p$ is determined by the points $X$, $Y$ and $Z$. Similarly to izreku
  \ref{SimpsPrem},
the quadrilateral $PYXC$ is a trapezoid, therefore $\angle YXP \cong \angle ACP$.
By izreku \ref{ObodObodKot} the angles $ACP$ and $AP_AP$ above
the trapezoid $AP$ are supplementary, therefore $\angle YXP \cong \angle AP_AP$
or $XY\parallel AP$ (izrek \ref{KotiTransverzala}).
  \kdokaz



            \bzgled \label{SimsZgled2}
            Let $P$ and $Q$ be arbitrary points lying on the circumcircle
            $k(O,r)$ of a triangle $ABC$ and $p$ and $q$ their Simson lines. Prove that
             $$\angle pq = \frac{1}{2}\angle POQ.$$
           \ezgled

\begin{figure}[!htb]
\centering
\input{sl.skk.4.7.1c.pic}
\caption{} \label{sl.skk.4.7.1c.pic}
\end{figure}

\textbf{\textit{Proof.}}
  Let $X_P$ and $X_Q$ be the feet of the perpendiculars from points $P$ and $Q$
   on the line $BC$ and $P_A$ and $Q_A$
the intersections of these perpendiculars with the circle $k$ (Figure
\ref{sl.skk.4.7.1c.pic}). The Simson lines $p$ and $q$ are parallel to the lines $AP_A$ and $AQ_A$ (example \ref{SimsZgled1}).
Therefore, the angle determined by the lines $p$ and $q$ is equal to the inscribed angle $Q_AAP_A$, which is equal to half of the central angle $Q_AOP_A$
(by statement \ref{SredObodKot}) or half of the angle $QOP$ (because the trapezoid $PP_AQ_AQ$ is isosceles and by statement \ref{trapezTetivEnakokr} it is also equilateral, i.e. $PQ \cong P_AQ_A$).
 \kdokaz


            \bzgled \label{SimsZgled3}
            Let $P$ be an arbitrary point of the circumcircle $k$ of a triangle
            $ABC$, $p$ its Simson line and $V$ the orthocentre of this triangle.
             Prove that the line $p$ bisects the line segment $PV$.
            \ezgled

\begin{figure}[!htb]
\centering
\input{sl.skk.4.7.1d.pic}
\caption{} \label{sl.skk.4.7.1d.pic}
\end{figure}

 \textbf{\textit{Proof.}}
Let $P_A$ and $X$ be the points defined as in example
\ref{SimsZgled1} (Figure \ref{sl.skk.4.7.1d.pic}). Let $V'$
and $P'$ be the points that are symmetric to the points $V$ and $P$ with respect to the line $BC$. The point $V'$ lies on the circumscribed circle $k$ of the triangle $ABC$
(by statement \ref{TockaV'}). Because of the properties of symmetry
or the axis of reflection (see subsection \ref{odd6OsnZrc}), statement
\ref{KotiTransverzala}, statement \ref{ObodObodKot} and example
\ref{SimsZgled1} it holds:
   $$\angle VP'P\cong\angle V'PP'\cong\angle AV'P
   \cong\angle AP_AP\cong\angle p,PP'.$$
 Therefore, $\angle VP'P \cong \angle p,PP'$, so by statement
 \ref{KotiTransverzala} the lines $VP'$ and $p$ are parallel. Because
the point $X$ is the midpoint of the segment $PP'$, the line $p$ contains the midpoint
of the triangle $PVP'$ (by statement \ref{srednjicaTrik}) or the midpoint of its
side $PV$.
 \kdokaz

In sections \ref{odd5Hamilton} and \ref{odd7SredRazteg} we will prove two more properties
of Simson lines (see \ref{HamiltonSimson} and \ref{SimsEuler}), which are related to Hamilton's theorem or
Euler's circle of a triangle.




%________________________________________________________________________________
 \poglavje{Torricelli Point} \label{odd4Torricelli}

In this section we will give another famous point of a triangle.


             \bizrek \label{izrekTorichelijev}
             On each side of a triangle $ABC$ the equilateral triangles $BEC$, $CFA$ and $AGB$
             are externally erected. Prove:
            \begin{enumerate}
              \item $AE$, $BF$ and $CG$ are congruent line segments;
              \item the lines $AE$, $BF$ and $CG$ intersect at one point
              (so-called \pojem{Torricelli\footnote{Problem was first posed by French mathematician \index{Fermat, P.} \textit{P.
            Fermat} (1601--1665) as a challenge to Italian mathematician and physicist \index{Torricelli, E.} \textit{E.
            Torricelli} (1608--1647). Torricelli's solution was published by his student - Italian mathematician and physicist \textit{V. Viviani} (1622–-1703) - in 1659. We also call this point \index{point!Fermat's}\pojem{Fermat
            point}.}
             point} \color{blue}of this triangle) and every two of them determine an angle with measure $60^0$.
            \end{enumerate}
             \index{point!Torricelli's}
            \eizrek


\begin{figure}[!htb]
\centering
\input{sl.skk.4.7.1.pic}
\caption{} \label{sl.skk.4.7.1.pic}
\end{figure}

 \textbf{\textit{Proof.}}
  (Figure \ref{sl.skk.4.7.1.pic})

 (\textit{i}) Triangles $AEC$ and $FBC$ are congruent by \textit{SAS} \ref{SKS} theorem ($AC \cong FC$ , $CE \cong CB$ and
$\angle ACE \cong \angle FCB = \angle ACB + 60°$), so $AE\cong BF$. Analogously is $AE\cong CG$.

(\textit{ii}) Let $k$, $l$ and $j$ be the circumscribed circles of the triangles $BEC$, $CFA$ and $AGB$. We shall first prove that these circles intersect in one point. With $T$ we denote the second intersection point of the circles $k$ and $l$ ($T\neq C$). The quadrilaterals $BECT$ and $CFAT$ are cyclic, therefore (by the statement \ref{TetivniPogoj}) both the angles $BTC$ and $ATC$ measure $120^0$. Thus, also the angle $ATB$ measures $120^0$, which means that the quadrilateral $AGBT$ is cyclic (by the statement \ref{TetivniPogoj}) or that the point $T$ also lies on the circle $j$.

We shall prove that each of the lines $AE$, $BF$, $CG$ goes through the point $T$. From the equality of the corresponding
circumscribed angles (by the statement \ref{ObodObodKot}) we obtain:

\begin{eqnarray*}
\angle ATE&=&\angle ATF+\angle FTC+\angle CTE=\\
&=&\angle ACF+\angle FAC+\angle CBE
=3\cdot 60^0=180^0.
\end{eqnarray*}

Thus, $A$, $T$ and $E$ are collinear points, or the point $T$ lies on the line $AE$. Analogously, the point $T$ also
lies on the lines $BF$ and $CG$. It is also clear that:
$\angle AE,BF\cong\angle ATF\cong\angle ACF=60^0$.
 \kdokaz

  In the section \ref{odd9MetrInv} (by the statement \ref{izrekToricheliFerma}) we shall prove another interesting property of the Torricelli's point.


%________________________________________________________________________________
 \poglavje{Excircles of a Triangle} \label{odd4Pricrt}

 We have already proved that for any triangle there exist
 circumscribed and inscribed circle. The first one contains all the vertices
of the triangle, the second one touches all its sides. Now we shall
show that there also exist circles which touch one side
and two lines containing the sides of the triangle.


            \bizrek
            The bisector of the interior angle at vertex $A$ and the bisectors of
            the exterior angles at vertices $B$ and $C$ of a triangle $ABC$ intersect at one point,
            which is the centre of the circle touching the side $BC$ and the lines containing the sides $AB$ and
            $AC$. It is so-called \index{pričrtane krožnice trikotnika} \pojem{excircle of the triangle}\color{blue}.
            \eizrek



\begin{figure}[!htb]
\centering
\input{sl.27.1.94_veliki_zadatak_lema.pic}
\caption{} \label{sl.27.1.94_veliki_zadatak_lema.pic}
\end{figure}

\textbf{\textit{Proof.}} We prove the statement similarly to the inscribed circle of a triangle. The simetrali
of the external angles at the vertices $B$ and $C$ are not parallel and they
intersect at some point - we mark it with $S_a$ (Figure
\ref{sl.27.1.94_veliki_zadatak_lema.pic}). Because the point $S_a$ lies on
these two simetrali, it holds $A,S_a\div BC$ and $S_a$ is equally distant from the lines $AB$, $BC$ and
$AC$. Therefore, $S_a$ also belongs to the simetral of the internal angle at
the vertex $A$ and is the center of the circle that touches the side $BC$ and
the lines $AB$ and $AC$.
 \kdokaz

 Now we are ready to prove the so-called \index{velika naloga}
  \pojem{‘‘velika naloga’’}, which is very useful in
 designing triangles.




              \bizrek \label{velikaNaloga}
              Let $P$, $Q$, $R$ be the touching points of the incircle $k(S,r)$
               of a triangle $ABC$ with the sides $BC=a$, $AC=b$, $AB=c$ ($b>c$) and $P_i$, $Q_i$, $R_i$
                ($i\in \{a,b,c\}$) the touching points of the excircles
              $k_i(S_i,r_i)$ with lines $BC$, $AC$ in $AB$. Let $l(O,R)$
               be the circumcircle of this triangle with the semiperimeter
                $s=\frac{a+b+c}{2}$, $A_1$ the midpoint of the line segments $BC$, $M$ and $N$,
                intersections of the line $OA_1$ with the circle $l$ ($N,A\div BC$) and
                $M’$, $N’$ the foots of the perpendiculars  from these points on the line $AB$ (Figure
                \ref{sl.27.1.94_veliki_zadatak.pic}). Then:
                \vspace*{2mm}

                (\textit{i}) $AQ_a\cong AR_a=s$, \hspace*{0.4mm} (\textit{ii})
                $AQ\cong AR=s-a$, \hspace*{0.4mm} (\textit{iii}) $QQa\cong RRa\cong a$,
                % \vspace*{1mm}

(\textit{iv}) $PPa=b-c$,\hspace*{1mm}
                (\textit{v}) $P_bP_c=b+c$,
                % \vspace*{1mm}

                 (\textit{vi})
                  $A_1$ is the midpoint of the line segment $PP_a$ in $P_bP_c$,
                %\vspace*{1mm}

                 (\textit{vii}) $A_1N= \frac{r_a- r}{2}$,\hspace*{2mm}
               (\textit{viii}) $A_1M= \frac{r_b + r_c }{2} $,\hspace*{1mm}


              (\textit{ix}) $r_a +r_b +r_c =4R+r$,\footnote{To lastnost
               trikotnika je leta 1790 odkril francoski matematik \index{L'Huilier, S. A. J.}
                \textit{S. A. J. L'Huilier}
               (1750--1840).}
             %\vspace*{1mm}

                (\textit{x}) $NN’= \frac{r_a +r}{2}$, \hspace*{1mm} (\textit{xi})
                $MM’= \frac{r_b -r_c }{2 }$,\hspace*{1mm} (\textit{xii})
                $N’B\cong AM’=\frac{ b - c}{2}$,
                % \vspace*{1mm}

                (\textit{xiii}) $AN’\cong BM’= \frac{b + c}{2} $, \hspace*{1mm} (\textit{xiv}) $M’N’\cong b$.
                 \eizrek


\begin{figure}[!htb]
\centering
\input{sl.27.1.94_veliki_zadatak.pic}
\caption{} \label{sl.27.1.94_veliki_zadatak.pic}
\end{figure}

 \textbf{\textit{Proof.}}
  Preden začnemo z dokazovanjem, omenimo, da po izreku \ref{TockaN}
  točka $N$ leži na simetrali notranjega kota $BAC$ trikotnika
  $ABC$.

(\textit{i}) If we use the equality of tangent lines
  (statement \ref{TangOdsek}), we get:
 \begin{eqnarray*}
 AQ_a&\cong &AR_a= \frac{1}{2}\left(AQ_a+AR_a\right)=
 \frac{1}{2}\left(AB+BR_a+AC+CQ_a\right)\\
 &=& \frac{1}{2}\left(AB+BP_a+AC+CP_a\right)=
\frac{1}{2}\left(AB+BC+AC\right)=s.
 \end{eqnarray*}

 (\textit{ii}) In a similar way, we get:
 \begin{eqnarray*}
 AQ&\cong &AR= \frac{1}{2}\left(AQ+AR\right)=
 \frac{1}{2}\left(AB-BR+AC-CQ\right)\\
 &=& \frac{1}{2}\left(AB-BP+AC-CP\right)=
\frac{1}{2}\left(AB+AC-BC\right)=s-a.
 \end{eqnarray*}

 (\textit{iii}) $QQ_a\cong AQ_a-AQ=a$.

 (\textit{iv}) We prove the equality by first calculating:\\ $BP$ and $CP_a\cong CQ_a$.

 (\textit{v}) $P_bP_c\cong CP_c+BP_b-a=2s-a=b+c$.

 (\textit{vi}) It follows from $BP\cong CP_a=s-b$.

 (\textit{vii}) Points $A_1$ and $N$ are the centers of the diagonal of trapezoid
 $SPS_aP_a$ with
 bases $SP\cong r$ and
$S_aP_a\cong r_a$. The equality follows
from statement \ref{srednjTrapez}.

 (\textit{viii}) Lines $NA$ and $S_cS_b$
  are perpendicular (the internal
 and external angles at point
$A$ are symmetrical), so point $M$ lies on line $S_cS_b$. The desired equality follows
from the fact that line $A_1M$ is the median of trapezoid
$S_cP_cP_bS_b$ (statement \ref{srednjTrapez}).

 (\textit{ix}) It follows directly from $2\cdot R=NM=NA_1+A_1M$ and
 (\textit{vii}) and (\textit{viii}).

(\textit{x}) It follows from statement \ref{srednjTrapez} and the fact that 
$NN'$ is the median of trapezoid $SRR_aS_a$.

(\textit{xi}) Points $M$ and $M'$ are the centers of the diagonal of trapezoid
$R_cS_cR_bS_b$ with bases $R_cS_c\cong r_c$ and $R_bR_b\cong r_b$. The equality
follows from statement \ref{srednjTrapez}.

 (\textit{xii}) Point $N$ is the center of line $RR_a$, so:
  $$N'B=AN'-AB= \frac{1}{2}\left(AR_a+AR\right)-c=
  \frac{1}{2}\left(s+(s-a)\right)-c= \frac{1}{2}\left(b-c\right).$$

(\textit{xiii}) and (\textit{xiv}) follow
  directly from the proven equality (\textit{xii}).
 \kdokaz

%KONSTRUKCIJA (VN)

        \bzgled
         Construct a triangle $ABC$,  with given:

         (\textit{a}) $a$, $b-c$, $r$, \hspace*{3mm} (\textit{b})
        $b-c$, $r$, $v_b$, \hspace*{3mm} (\textit{c})  $a$, $b+c$, $r$,
        \hspace*{3mm}(\textit{č}) $R$, $r$, $r_a$.
         \ezgled



\textbf{\textit{Solution.}}
 For each construction we will use the big task - \ref{velikaNaloga}.
 We will use
  the same labels.

\begin{figure}[!htb]
\centering
\input{sl.27.1.94_veliki_zadatak_konstr.pic}
\caption{} \label{sl.27.1.94_veliki_zadatak_konstr.pic}
\end{figure}

\begin{figure}[!htb]
\centering
\input{sl.27.1.94_veliki_zadatak_konstr2.pic}
\caption{} \label{sl.27.1.94_veliki_zadatak_konstr2.pic}
\end{figure}



(\textit{a}) Because $PP_a=b-c$ and point $A_1$ is the common center
of side
 $BC$ and
 the line $PP_a$, it also holds that $PA_1 = \frac{1}{2}( b - c)$ (Figure
\ref{sl.27.1.94_veliki_zadatak_konstr.pic}).
First, we plan the side $BC$, then its center
$A_1$, point $P$, the inscribed circle of the triangle, tangents from
the vertices $B$ and $C$, and finally the vertices $A$.



(\textit{b}) Similarly to the previous example. First, we plan
the line $PA_1$, then the inscribed circle of the triangle $ABC$ (Figure
\ref{sl.27.1.94_veliki_zadatak_konstr.pic}).
We also need to use the condition of the height from the vertex $B$. With $L$
we mark the orthogonal projection from the point $A_1$ onto the line $AC$.
The line $A_1L$ is the median of the triangle $CBB'$, so $A_1L
=\frac{1}{2}BB'=\frac{1}{2} v_b$. Therefore, the line $AC$ can
be constructed as the common tangent of the inscribed circle and the circle
$k(A_1, \frac{1}{2}v_b)$. Thus we get the vertex $C$, then the vertices $B$ and $A$.

(\textit{c}) We know that $RR_a\cong a$ and $AN'=\frac{1}{2}(b+c)$, and that $N'$ is the center of the line $RR_a$ (Figure \ref{sl.27.1.94_veliki_zadatak_konstr2.pic}). So, from the given
data,
 we first construct the points $A$, $N'$, $R$ and $R_a$, and then also $S$, $N$
  and
$S_a$. In the end, we draw the dotted and the dashed circle -
the sides of the triangle lie on their common tangents.


(\textit{č}) Because $A_1N =\frac{1}{2}(r_a -r)$ and $MN = 2R$, we can
first plan the points $N$, $A_1$ and $M$, and then also the dotted
circle of the triangle $ABC$ and the side $BC$ (Figure \ref{sl.27.1.94_veliki_zadatak_konstr2.pic}). The construction can
be finished in two ways. In the first case, we translate the task into a construction of a triangle that we already know: $a$, $R$, $r$ (example \ref{konstr_Rra}), in
the second case, we use the equality $RR_a\cong a$.
 \kdokaz




%________________________________________________________________________________
\naloge{Exercises}

\begin{enumerate}

\item   The sides of a triangle are $6$, $7$ and $9$. Let $k_1$, $k_2$ and $k_3$ be the circles with centers
in the vertices of this triangle. The circles touch each other so that
 the circle with the center in the vertex of the smallest angle of the triangle touches the other two circles from the inside,
while the remaining two circles touch from the outside. Calculate the lengths of the radii of these three circles.

\item Prove that the angle formed by the secants of a circle that intersect each other outside the circle is equal to half the difference of the central angles corresponding to the arcs that lie between the arms of this angle.

\item   The apex of the angle $\alpha$ is an external point of the circle $k$. Between the arms of this angle, on the circle, there are two
arcs, which are in the ratio $3:10$. The larger of these arcs corresponds to the central angle $40^0$. Determine
the measure of the angle $\alpha$.

\item  Prove that the angle formed by the tangents of a circle is equal to half the difference of the central angles corresponding to the arcs that lie between the arms of this angle.

\item Let $L$ be the orthogonal projection of an arbitrary point $K$ of the circle $k$
on its tangent $t$ through the point $T\in k$ and $X$ the point that is
symmetric to the point $L$ with respect to the line $KT$. Determine the geometric
position of the points $X$.

\item Let $BB'$ and $CC'$ be the altitudes of the triangle $ABC$ and $t$
the tangent of this triangle at the point $A$. Prove that
 $B'C'\parallel t$.

\item In the right triangle $ABC$ is above the cathetus $AC$ as the diameter
drawn circle that intersects the hypotenuse $AB$ at the point $E$. The tangent of this
circle at the point $E$ intersects the other cathetus $BC$ at the point $D$. Prove that $BDE$
is an isosceles triangle.

\item In the right angle with the vertex $A$ is drawn a circle that touches the sides of this angle
at the points $B$ and $C$. Any tangent of this circle intersects the lines $AB$ and $AC$, in order
in the points $M$ and $N$ (so that $\mathcal{B}(A,M,B)$). Prove that:
$$\frac{1}{3}\left(|AB|+|AC|\right) < |MB|+|NC| <
\frac{1}{2}\left(|AB|+|AC|\right).$$

\item Prove that in the right triangle the sum of the sides is equal to the sum
of the diameters of the inscribed and drawn circle.

\item Let the similitudes of the internal angles of the convex quadrilateral intersect in six different points.
Prove that four of these points are the vertices of the pedal quadrilateral.

\item Let: $c$ be the length of the hypotenuse, $a$ and $b$ the lengths
of the catheti and $r$ the radius of the inscribed circle of the right triangle. Prove that:
\begin{enumerate}
 \item $2r + c \geq 2 \sqrt{ab}$, \item $a + b + c > 8r$.
\end{enumerate}

\item Let $P$ and $Q$ be the centers of the shorter arcs $AB$ and $AC$
of the regular triangle $ABC$ of the drawn circle. Prove that the sides $AB$ and $AC$ of this triangle divide
the chord $PQ$ into three proportional segments.

\item Let $k_1$, $k_2$, $k_3$, $k_4$ be four circles, each of which from the outside touches one side   and two sides of an arbitrary convex
quadrilateral. Prove that the centers of these circles are concircular points.

\item Circles $k$ and $l$ touch each other from the outside in point $A$. Points $B$ and $C$ are the points of contact of the common external tangent of these two circles. Prove that $\angle BAC$ is a right angle.

\item Let $ABCD$ be a deltoid ($AB\cong AD$ and $CB\cong CD$). Prove:
\begin{enumerate}
 \item $ABCD$ is a tangent quadrilateral,
 \item  $ABCD$ is a parallelogram exactly when $AB\perp BC$.
\end{enumerate}

\item Circles $k$ and $k_1$ touch each other from the outside in point $T$, where they intersect lines $p$ and $q$. Line $p$ has two more intersections with the circles, $P$ and  $P_1$, line $q$ has $Q$ and $Q_1$.  Prove that $PQ\parallel P_1Q_1$.

\item Let $MN$ be the common tangent of circles $k$ and $l$ ($M$ and $N$ are the points of contact), which intersect in points $A$ and $B$. Calculate the measure of the sum $\angle MAN+\angle MBN$.

\item Let $t$ be the tangent of triangle $ABC$ of the circumscribed circle in point $A$. A line parallel to the tangent $t$ intersects sides $AB$ and $AC$ in points $D$ and $E$.
Prove that points $B$, $C$, $D$ and $E$ are concyclic.

\item Let $D$ and $E$ be any points of the semicircle drawn over diameter $AB$. Let $AD\cap BE= \{F\}$ and $AE\cap BD= \{G\}$.
Prove that $FG\perp AB$.

\item Let $M$ be a point of circle $k(O,r)$. Determine the geometric location of the centers of all the tangents of this circle that have one endpoint in point $M$.

\item Let $M$ and $N$ be points that are symmetric to the vertex $A'$ of altitude $AA'$ of triangle $ABC$ with respect to side $AB$ and $AC$, and let $K$ be the intersection of lines $AB$ and $MN$. Prove that points $A$, $K$, $A'$, $C$ and $N$ are concyclic.

\item Let $ABCD$ be a parallelogram, $E$ the altitude point of triangle $ABD$, and $F$ the altitude point of triangle $ABC$. Prove that quadrilateral $CDEF$
is a parallelogram.

\item Circles with centers $O_1$ and $O_2$ intersect in points $A$ and $B$. The line $p$,
which goes through point $A$, intersects these two circles in points $M_1$ and $M_2$. Prove that
$\angle O_1M_1B\cong\angle O_2M_2B$.

\item The circle with center $O$ is drawn over the diameter $AB$.
Let $C$ and $D$ be such points on the line $AB$, that $CO\cong OD$. Parallel lines through points $C$ and $D$ intersect the circle in points $E$ and $F$.
Prove that lines $CE$ and $DF$ are perpendicular to the line $EF$.

\item On the string $AB$ of the circle $k$ with center $O$ lies the point $C$, point $D$ is the other intersection of the circle $k$ with the drawn circle of the triangle $ACO$. Prove that
$CD\cong CB$.

\item Let $AB$ be the transversal of the circle $k$. Lines $AC$ and $BD$ are tangents
to the circle $k$ in points $C$ and $D$.
Prove that:
 $$||AC|-|BD||< |AB| < |AC|+|BD|.$$

\item Let $S$ be the intersection of the sides $AD$ and $BC$
of the trapezoid $ABCD$ with the base $AB$. Prove that the drawn circles of the triangles $SAB$ and
$SCD$ touch in point $S$.

\item Lines $PB$ and $PD$ touch the circle $k(O,r)$ in points $B$ and $D$.
Line $PO$ intersects the circle $k$ in points $A$ and $C$ ($\mathcal{B}(P,A,C)$). Prove that the
line $BA$ is the angle bisector of the angle $PBD$.

\item Quadrilateral $ABCD$ is inscribed in the circle with center $O$. Diagonals $AC$ and
$BD$ are perpendicular. Let $M$ be the perpendicular projection of the center $O$
on the line $AD$. Prove that
 $$|OM|=\frac{1}{2}|BC|.$$

\item Lines $AB$ and $BC$ are adjacent sides of a regular nonagon, which is inscribed in the circle $k$ with center $O$.
Point $M$ is the center of the side $AB$, point $N$ is the center
of the radius $OX$ of the circle $k$, which is perpendicular to the line $BC$. Prove that
$\angle OMN=30^0$.

\item Circles $k_1$ and $k_2$ intersect in points $A$ and $B$. Let $p$ be a line that goes through point $A$, circle $k_1$ intersects also in point $C$, circle $k_2$ intersects also in point $D$, and $q$ be a line that goes through point $B$, circle $k_1$ intersects also in point $E$, circle $k_2$ intersects also in point $F$. Prove that $\angle CBD\cong\angle EAF$.

\item Circles $k_1$ and $k_2$ intersect in points $A$ and $B$. Draw a line $p$, that goes through point $A$, so that the length of the line $MN$, where $M$ and $N$ are the intersections of line $p$ with circles $k_1$ and $k_2$, is maximal.

\item Let $L$ be the orthogonal projection of an arbitrary point $K$ of the circle $k$ on its tangent through the point $T\in k$ and $X$ be the point that is symmetric to the point $L$ with respect to the line $KT$. Determine the geometric position of the points $X$.

\item Prove that the string polygon with an even number of vertices, that has all the internal angles congruent, is a regular polygon.

\item Two circles touch each other from the inside in the point $A$. The line $AB$ is the diameter of the larger circle, the string $BK$ of the larger circle touches the smaller circle in the point $C$. Prove that the line $AC$ is the bisector of the angle $BAK$.

\item Let $BC$ be the string of the circle $k$. Determine the geometric position of the altitude points of all the triangles $ABC$, where $A$ is an arbitrary point that lies on the circle $k$.

\item We have a quadrilateral with three acute internal angles. Prove that the longer diagonal goes through the vertex that belongs to the acute angle.

\item Let $ABCDEF$ be a string hexagon, $AB\cong DE$ and $BC\cong EF$. Prove that $CD\parallel AF$.

\item Let $ABCD$ be a convex quadrilateral, where $\angle ABD=50^0$, $\angle ADB=80^0$, $\angle ACB=40^0$ and $\angle DBC=\angle BDC +30^0$. Calculate the measure of the angle $\angle DBC$.

\item Let $M$ be an arbitrary internal point of the angle with the vertex $A$, points $P$ and $Q$ the orthogonal projections of the point $M$ on the sides of this angle, and point $K$ the orthogonal projection of the vertex $A$ on the line $PQ$. Prove that $\angle MAP\cong \angle QAK$.

\item In the archery octagon $A_1A_2\ldots A_8$ it holds that $A_1A_2\parallel A_5A_6$, $A_2A_3\parallel A_6A_7$,
$A_3A_4\parallel A_7A_8$. Prove that $A_8A_1\cong A_4A_5$.

\item A circle intersects each side of a quadrilateral in two points and thus on all sides of the quadrilateral it determines the consistent tautologies.
Prove that this quadrilateral is tangent.

\item The lengths of the sides of the tangent pentagon $ABCDE$ are natural numbers and at the same time $|AB|=|CD|=1$.
The inscribed circle of the pentagon touches the side $BC$ in the point $K$.
Calculate the length of the line $BK$.

\item Prove that the circle that passes through the adjacent vertices $A$ and $B$ of the regular
pentagon $ABCDE$ and its center $O$, also passes through the intersection
of its diagonals $AD$ and $BE$.

\item Let $H$ be the altitude point of the triangle $ABC$, $l$ the circle above  the diameter $AH$
and $P$ and $Q$ the intersections of this circle with the sides $AB$ and $AC$. Prove that the
tangents of the circle $k$ through the points $P$ and $Q$ intersect on the side $BC$.

\item The circle $l$ touches the circle $k$ from the inside in the point $C$. Let $M$ be any point
of the circle $l$ (different from $C$). The tangent of the circle $l$ in the point $M$ intersects the circle $k$ in
the points $A$ and $B$. Prove that $\angle ACM \cong \angle MCB$.

\item Let $k$ be the inscribed circle of the triangle $ABC$ and $R$ the center of that arc $AB$ of this circle,
which does not contain the point $C$. The lines $RP$ and $RQ$ are the tautologies of this circle. The first one
is parallel, the second one is perpendicular to the internal angle  $\angle BAC$. Prove:
\begin{enumerate}
\item The line $BQ$ is the internal angle $\angle CBA$,
\item  The triangle, which is determined by the lines $AB$, $AC$ and $PR$, is a right triangle.
 \end{enumerate}

\item Let $X$ be such an internal point of the triangle $ABC$, that it holds:
 $\angle BXC =\angle BAC+60^0$,  $\angle AXC =\angle ABC+60^0$ and  $\angle AXB =\angle AC B+60^0$. Let
$P$, $Q$ and $R$ be the other intersections of the lines $AX$, $BX$ and $CX$ with the inscribed circle of the triangle $ABC$. Prove that the  triangle $PQR$ is a right triangle.

\item Prove that the tangent points of an inscribed circle of a triangle $ABC$ divide its sides into segments of lengths $s-a$, $s-b$ and $s-c$ ($a$, $b$ and $c$ are the lengths of the sides, $s$ is the semi-perimeter of the triangle).


 \item Circles $k$, $l$ and $j$ touch each other from the outside in non-linear points $A$, $B$ and $C$. Prove that the circumscribed circle of the triangle $ABC$ is perpendicular to the circles $k$, $l$ and $j$.


 \item Let $ABCD$ be a square with the center of the circumscribed circle in the point $O$. With $E$ we denote the intersection of its diagonals $AC$ and $BD$ and with $F$, $M$ and $N$ the centers of the lines $OE$, $AD$ and $BC$. If $F$, $M$ and $N$ are collinear points, then $AC\perp BD$ or $AB\cong CD$. Prove.



\item Draw a triangle $ABC$ (see the labels in section \ref{odd3Stirik}):

 (\textit{a}) $a$, $\alpha$, $r$, \hspace*{2mm}
 (\textit{b}) $a$, $\alpha$, $r_a$, \hspace*{2mm}
 (\textit{c}) $a$, $v_b$, $v_c$, \hspace*{2mm}

 (\textit{d}) $\alpha$, $v_a$, $s$, \hspace*{2mm}
 (\textit{e}) $v_a$, $l_a$, $r$, \hspace*{2mm}
(\textit{f}) $\alpha$, $v_a$, $l_a$, \hspace*{2mm}

 (\textit{g}) $\alpha$, $\beta$, $R$, \hspace*{2mm}
 (\textit{h}) $c$, $r$, $R$, \hspace*{2mm}
 (\textit{i}) $a$, $v_b$, $R$, \hspace*{2mm}

 \item Draw a circle $k$ so that:

  \begin{enumerate}
    \item it touches two given non-parallel lines $p$ and $q$, and the tangent that determines the touch points is consistent with the given distance $t$,
    \item its center is the given point $S$, and the given line $p$ determines on it the tangent that is consistent with the given distance $t$,
    \item it passes through the given points $A$ and $B$, and its center lies on the given circle $l$,
    \item it has the given radius $r$ and it touches the two given circles $l$ and $j$,
    \item it touches the line $p$ in the point $P$ and passes through the given point $A$.
  \end{enumerate}

  \item Draw a square $ABCD$, if the given vertex $B$ and two points $E$ and $F$ that lie on the sides $AD$ and $CD$ are given.

\item Given is a line $CD$ and points $A$ and $B$ ($A,B\notin CD$). On the line $CD$ draw a point $M$, such that $\angle AMC\cong2\angle BMD$.


  \item Draw a triangle $ABC$ with the following data:

   (\textit{a}) $v_a$, $t_a$, $\beta-\gamma$, \hspace*{2mm}
   (\textit{b}) $v_a$, $l_a$, $R$, \hspace*{2mm}
   (\textit{c}) $R$, $\beta-\gamma$, $t_a$, \hspace*{2mm}

   (\textit{d}) $R$, $\beta-\gamma$, $v_a$, \hspace*{2mm}
   (\textit{e}) $R$, $\beta-\gamma$, $a$. \hspace*{2mm}


\item On the beam of the side $AB$ of the rectangle $ABCD$ draw a point $E$, from which the sides $AD$ and $DC$ are seen under the same angle. When does the task have a solution?

    \item In the convex quadrilateral $ABCD$ it holds that $BC\cong CD$. Draw this quadrilateral, if the sides $AB$ and $AD$ and the internal angle at the vertices $B$ and $D$ are given.

    \item In the given circle $k$ draw a triangle $ABC$, if the vertex $A$, the line $p$, which is parallel to the altitude $AA'$, and the intersection point $B_2$ of the altitude beam $BB'$ and this circle are given.

    \item Draw a regular triangle $ABC$, if its side $BC$ is congruent to the distance $a$, the altitude beams of the sides $AB$ and $AC$ and the line of symmetry of the internal angle $BAC$ go through the given points $M$, $N$ and $P$ one after another.

\item Draw a triangle $ABC$, if the following are given:
  \begin{enumerate}
    \item the vertex $A$, the center of the circumscribed circle $O$ and the center of the inscribed circle $S$,
    \item the center of the circumscribed circle $O$, the center of the inscribed circle $S$ and the center of the escribed circle $S_a$,
    \item the vertex $A$, the center of the circumscribed circle $O$ and the altitude point $V$,
    \item the vertices $B$ and $C$ and the internal angle bisector of the internal angle $BAC$,
    \item the vertex $A$, the center of the circumscribed circle $O$ and the intersection point $E$ of the side $BC$ with the internal angle bisector of the internal angle $BAC$,
    \item the points $M$, $P$ and $N$, in which the altitude and the centroid from the vertex $A$ and the internal angle bisector of the internal angle $BAC$ intersect the circumscribed circle of the triangle,
    \item the vertex $A$, the center of the circumscribed circle $O$, the point $N$, in which the internal angle bisector of the internal angle $BAC$ intersects the circumscribed circle of the triangle, and the distance $a$, which is parallel to the side $BC$.
  \end{enumerate}

  \item Draw a triangle $ABC$ with the following data:

   (\textit{a}) $a$, $b$, $\alpha=3\beta$, \hspace*{3mm}
   (\textit{b}) $t_a$, $t_c$, $v_b$.\hspace*{3mm}

\item Through the point $M$, which lies inside the given circle $k$, draw such a cord that the difference of its segments (from the point $M$) is equal to the given distance $a$.

\item Draw a triangle $ABC$, if you know:

 (\textit{a}) $b-c$, $r$, $r_a$, \hspace*{1.8mm}
 (\textit{b}) $a$, $r$, $r_a$, \hspace*{1.8mm}
 (\textit{c}) $a$, $r_b+r_c$, $v_a$, \hspace*{1.8mm}
 (\textit{d}) $b+c$, $r_b$, $r_c$,

 (\textit{e}) $R$, $r_b$, $r_c$, \hspace*{1.8mm}
 (\textit{f}) $b$, $R$, $r+r_a$, \hspace*{1.8mm}
(\textit{g}) $a$, $v_a$, $r_a-r$, \hspace*{1.8mm}
 (\textit{h}) $\alpha$, $r$, $b+c$.


 \item The following are given: the circle $k$, its diameter $AB$ and the point $M\notin k$. With only a straightedge, draw a rectangle from the point $M$ to the line $AB$.

\item The given are: square $ABCD$ and such points $M$ and $N$ on sides $BC$ and $CD$, that $\angle MAN=45^0$.
 With only a straightedge draw a rectangle from point $A$ to line $MN$.

\end{enumerate}





% DEL 5 - - - - - - - - - - - - - - - - - - - - - - - - - - - - - - - - - - - - - - -
%________________________________________________________________________________
% VEKTORJI
%________________________________________________________________________________

  \del{Vectors} \label{pogVEKT}


%________________________________________________________________________________
\poglavje{Vector Definition. The Sum of Vectors} \label{odd5DefVekt}

Intuitively, a vector is a directed line segment that can be moved parallel\footnote{The concept of a vector was known to the ancient Greeks. The modern concept of vectors, associated with linear algebra and analytic geometry, began to develop in the 19th century as a generalization of complex numbers. In this sense, the English mathematician \index{Hamilton, W. R.}\textit{W. R. Hamilton} (1805--1865) defined the so-called \index{quaternions}\textit{quaternions} $q = w + ix + jy + kz$, $i^2 = j^2 = k^2 = -ijk = -1$ as a generalization of complex numbers in four-dimensional space}. In this sense, the vector $\overrightarrow{AB}$ would represent the entire set of line segments that are consistent, parallel and have the same direction as a given line segment $AB$. We could also write $\overrightarrow{CD}=\overrightarrow{AB}$ for each line segment $CD$ from this set of line segments (Figure \ref{sl.vek.5.1.1.pic}).

\begin{figure}[!htb]
\centering
\input{sl.vek.5.1.1.pic}
\caption{} \label{sl.vek.5.1.1.pic}
\end{figure}

In this way, we get an idea for a formal definition of vectors. First, we introduce the relation $\varrho$ on the set of pairs of points. Let $A$, $B$, $C$ and $D$ be points in the same plane
(Figure \ref{sl.vek.5.1.2.pic}). We say that $(A,B)\varrho (C,D)$, if one of the three conditions\footnote{In this way - in \index{geometry!affine}affine geometry (without the axioms of congruence) - the vector was defined by the Serbian mathematician \index{Veljković, M.}\textit{M. Veljković} (1954--2008), professor at the Mathematical Gymnasium in Belgrade.}\index{relation!$\varrho$} is fulfilled:

\begin{enumerate}
  \item The quadrilateral $ABDC$ is a parallelogram,
  \item There exist points $P$ and $Q$, such that the quadrilaterals $ABQP$ and $CDQP$ are parallelograms,
  \item $A=B$ and $C=D$.
\end{enumerate}

\begin{figure}[!htb]
\centering
\input{sl.vek.5.1.2.pic}
\caption{} \label{sl.vek.5.1.2.pic}
\end{figure}

It is intuitively clear that the second condition needs to be added due to the example when points $A$, $B$, $C$ and $D$ are collinear. The third condition will apply to the so-called vector of zero.

From the definition of the relation $\varrho$ itself, we get the following proposition.



                \bizrek \label{vektRelRo}
                If $(A,B)\varrho (C,D)$ and $A\neq B$, then the line segments $AB$ and $CD$
                 are congruent, parallel, and have the same direction.
                \eizrek

\textbf{\textit{Proof.}} Because $A\neq B$, only the first two conditions from the definition remain. In this case, the relations $AB\parallel CD$ and $AB\cong CD$ are direct consequences of the definition of a parallelogram and proposition \ref{paralelogram}.

Regarding the orientation of the line segments, we would use the definition: parallel line segments $XY$ and $UV$ are \pojem{equally oriented}, if one of the conditions is fulfilled:
\begin{itemize}
  \item $Y,V\ddot{-} XU$;
  \item There exist points $S$ and $T$, such that $ST\parallel XY$,  $Y,T\ddot{-} XS$ and $V,T\ddot{-} US$.
\end{itemize}
\kdokaz

The next proposition is needed, so that we can define vectors.


                \bizrek
                The relation $\varrho$ on the set of pairs of points in the plane is an equivalence relation.
                \eizrek


\textbf{\textit{Proof.}} The reflexivity and symmetry are direct consequences of the definition of the relation $\varrho$. For transitivity, however, it is necessary to check all possible combinations of conditions 1--3 from the definition.
\kdokaz

\pojem{Vector} \index{vector} is now defined as the class of the equivalence relation $\varrho$:

$$\overrightarrow{AB}=[AB]_{\varrho}=\{(X,Y);\hspace*{1mm}(X,Y)\varrho(A,B)\}.$$

\begin{figure}[!htb]
\centering
\input{sl.vek.5.1.3.pic}
\caption{} \label{sl.vek.5.1.3.pic}
\end{figure}


We will call point $A$ the \index{starting point of a vector}\pojem{starting point}, point $B$ the \index{end point of a vector}\pojem{end point} of the vector $\overrightarrow{AB}$.

It is clear that in the case of $(A,B)\varrho (C,D)$, the pairs $(A,B)$ and $(C,D)$ are from the same class, which means that then $\overrightarrow{AB}=\overrightarrow{CD}$. The converse is also true: the relation $\overrightarrow{AB}=\overrightarrow{CD}$ means that $(A,B)\varrho (C,D)$
(Figure \ref{sl.vek.5.1.3.pic}).

If it is not necessary to emphasize which representative of the class it is, we will also denote vectors with $\overrightarrow{v}$, $\overrightarrow{u}$, $\overrightarrow{w}$, ...

The vector $\overrightarrow{AA}$, which represents the set of coinciding points, we will call the \index{vector!zero}\pojem{zero vector}. The zero vector will also be denoted by $\overrightarrow{0}$
(Figure \ref{sl.vek.5.1.4.pic}).

\begin{figure}[!htb]
\centering
\input{sl.vek.5.1.4.pic}
\caption{} \label{sl.vek.5.1.4.pic}
\end{figure}

We will say that the vector $\overrightarrow{BA}$ is the \index{vector!opposite}\pojem{opposite vector} of the vector $\overrightarrow{AB}$ and denote it by $-\overrightarrow{AB}$. It is clear that the definition of the opposite vector is not dependent on the choice of the representative of the class; the opposite vector is obtained if we exchange the starting and end points of the vector (or of each pair of points of the corresponding class). So if we denote $\overrightarrow{v}=\overrightarrow{AB}$ then
$-\overrightarrow{v}=-\overrightarrow{AB}=\overrightarrow{BA}$
(Figure \ref{sl.vek.5.1.4.pic}).


We denote the set of all vectors of the plane with $\mathcal{V}$.

A direct consequence of \ref{vektRelRo} is the following assertion.


                \bizrek \label{vektVzpSkl}
                If $\overrightarrow{AB}=\overrightarrow{CD}$, then the line segments $AB$ and $CD$
                 are congruent, parallel, and have the same direction.
                \eizrek

Based on the previous statement, we can now correctly define the following concepts:
\begin{itemize}
  \item \index{smer vektorja}\pojem{smer} of vector $\overrightarrow{AB}\neq\overrightarrow{0}$ is determined by any line $p\parallel AB$,
  \item \index{dolžina!vektorja}\pojem{length} or \index{intenziteta vektorja}\pojem{intensity} of vector $\overrightarrow{AB}$ is $|\overrightarrow{AB}|=|AB|$ ($|\overrightarrow{0}|=0$),
  \item \index{usmerjenost vektorja}\pojem{orientation} or \index{orientacija!vektorja}\pojem{orientation} of vector $\overrightarrow{AB}\neq\overrightarrow{0}$ is determined by the ordered pair $(A,B)$, where $A$ is the starting point, and $B$ is the end point of this vector.
\end{itemize}

If vectors $\overrightarrow{v}$ and $\overrightarrow{u}$ have the same direction, we also say that they are \index{vektorja!vzporedna}\pojem{parallel} or \index{vektorja!kolinearna}\pojem{colinear} and denote it with $\overrightarrow{v}\parallel\overrightarrow{u}$ (in this case we also say that $\overrightarrow{0}$ is colinear with every vector), otherwise the vectors are \index{vektorja!nekolinearna}\pojem{non-colinear}.

If colinear vectors $\overrightarrow{v}$ and $\overrightarrow{u}$ satisfy $\overrightarrow{v}=\overrightarrow{SA}$ and $\overrightarrow{u}=\overrightarrow{SB}$, we say that vectors $\overrightarrow{v}$ and $\overrightarrow{u}$ are \pojem{equally oriented} (notation $\overrightarrow{u}\rightrightarrows \overrightarrow{v}$), if $A,B\ddot{-} S$, or \pojem{oppositely oriented} (notation $\overrightarrow{u}\rightleftarrows \overrightarrow{v}$), if $A,B\div S$.

It is clear that colinear vectors $\overrightarrow{AB}$ and $\overrightarrow{CD}$ are equally oriented exactly when lines $AB$ and $CD$ are equally oriented.


From these definitions it directly follows that $|-\overrightarrow{v}|=|\overrightarrow{v}|$ and $-\overrightarrow{v}\parallel\overrightarrow{v}$; vectors $\overrightarrow{v}$ and $-\overrightarrow{v}$ are oppositely oriented (if $\overrightarrow{v}=\overrightarrow{AB}$, then $-\overrightarrow{v}=\overrightarrow{BA}$).

The following statement is very important.

\bizrek \label{vektABCObst1TockaD}
                For every three points there $A$, $B$ and $C$ there exists exactly one point $D$,
                 such that $\overrightarrow{AB}=\overrightarrow{CD}$, i.e.
                $$(\forall A, B, C)(\exists_1 D)\hspace*{1mm}\overrightarrow{AB}=\overrightarrow{CD}.$$
                \eizrek


\textbf{\textit{Proof.}} We will consider three different options (Figure \ref{sl.vek.5.1.5.pic}).

\begin{figure}[!htb]
\centering
\input{sl.vek.5.1.5.pic}
\caption{} \label{sl.vek.5.1.5.pic}
\end{figure}

\textit{1)} If $A=B$, then $D=C$ is the only point for which $(A,B)\varrho (C,D)$ or $\overrightarrow{CD}=\overrightarrow{AB}=\overrightarrow{0}$.

\textit{2)} Let $A\neq B$ and $C\notin AB$. Then $D$ is the fourth vertex of the parallelogram $ABDC$ (which exists and is only one due to
Playfair's axiom \ref{Playfair1}) and the only point for which $(A,B)\varrho (C,D)$ or $\overrightarrow{CD}=\overrightarrow{AB}$.

\textit{3)} Let $A\neq B$ and $C\in AB$. Let $P$ be any point that does not lie on the line $AB$. By the proven part \textit{2)} there is exactly one point $Q$, for which $\overrightarrow{PQ}=\overrightarrow{AB}$, and then also exactly one point $D$, for which $\overrightarrow{CD}=\overrightarrow{PQ}=\overrightarrow{AB}$.
\kdokaz

The previous statement can also be written in another way (Figure \ref{sl.vek.5.1.6.pic}).



                \bizrek \label{vektAvObst1TockaB}
                For every point $X$ and every vector $\overrightarrow{v}$ there exists exactly one point $Y$,
                 such that $\overrightarrow{XY}=\overrightarrow{v}$, i.e.
                $$(\forall X)(\forall \overrightarrow{v})(\exists_1 Y)\hspace*{1mm}\overrightarrow{XY}=\overrightarrow{v}.$$
                \eizrek

\begin{figure}[!htb]
\centering
\input{sl.vek.5.1.6.pic}
\caption{} \label{sl.vek.5.1.6.pic}
\end{figure}

As it is evident from the proof, for every vector $\overrightarrow{v}$ we can choose an arbitrary representative from its class with an arbitrary starting point. In a similar way, we could have proven that this is also possible regarding the end point. Intuitively, this means that we can move the vector parallel in only one way to its starting or end point.

We can therefore put two arbitrary vectors so that they have the same starting point or so that the starting point of the second vector is at the same time the end point of the first vector.
This fact allows us to define the sum of two vectors.

If $\overrightarrow{v}=\overrightarrow{AB}$ and $\overrightarrow{u}=\overrightarrow{BC}$, the \index{vsota!vektorjev}\pojem{sum of vectors} $\overrightarrow{v}$ and $\overrightarrow{u}$ is the vector $\overrightarrow{v}+\overrightarrow{u}=\overrightarrow{AC}$ (Figure \ref{asl.vek.5.1.7.pic}). So $\overrightarrow{AB}+\overrightarrow{BC}=\overrightarrow{AC}$ is true.


\begin{figure}[!htb]
\centering
\input{sl.vek.5.1.7.pic}
\caption{} \label{asl.vek.5.1.7.pic}
\end{figure}


It is necessary to prove the correctness of the definition - that the sum of vectors is not dependent on the choice of representatives of two classes ($\overrightarrow{v}$ and $\overrightarrow{u}$) or on the choice of point $A$.

                \bizrek \label{vektKorektDefSest}
                If $A$, $B$, $C$, $A'$, $B'$ and $C'$ are points such that $\overrightarrow{AB}=\overrightarrow{A'B'}$ and
                 $\overrightarrow{BC}=\overrightarrow{B'C'}$, then there is also $\overrightarrow{AC}=\overrightarrow{A'C'}$.
                \eizrek

\begin{figure}[!htb]
\centering
\input{sl.vek.5.1.7a.pic}
\caption{} \label{sl.vek.5.1.7a.pic}
\end{figure}


\textbf{\textit{Proof.}}  (Figure \ref{sl.vek.5.1.7.pic}).

In the formal proof, we would consider different possibilities regarding the relation $\varrho$.
  \kdokaz

 The rule of adding vectors, for which we use the aforementioned equality:

\begin{itemize}
  \item $\overrightarrow{AB}+\overrightarrow{BC}=\overrightarrow{AC}$,
\end{itemize}

We call it the \index{pravilo!trikotniško}\pojem{triangle rule} or the \index{Chaslesova identiteta}\pojem{Chaslesova\footnote{\index{Chasles, M.} \textit{M. Chasles} (1793–-1880),  French mathematician.} identity}.
The rule is named this way even though it is also valid when $A$, $B$ and $C$ are collinear points (or $\overrightarrow{v}$ and $\overrightarrow{u}$ are collinear vectors); in this case, it is not a triangle (Figure \ref{sl.vek.5.1.8.pic}).

\begin{figure}[!htb]
\centering
\input{sl.vek.5.1.8.pic}
\caption{} \label{sl.vek.5.1.8.pic}
\end{figure}



It turns out that the set of all vectors $\mathcal{V}$ with respect to the operation
of adding vectors represents the structure of a so-called \index{group!commutative}\pojem{commutative group} (or \index{group!Abelian}\pojem{Abelian\footnote{\index{Abel, N. H.}\textit{N. H. Abel} (1802--1829), Norwegian mathematician.} group}). The concept of a group (which is not always commutative) was already mentioned in section \ref{odd2AKSSKL} in the context of isometries; we will deal with this structure in more detail in section \ref{odd6Grupe}. The properties of this structure (commutative groups) are given in the statement
of the following theorem.

\bizrek \label{vektKomGrupa} The ordered pair $(\mathcal{V},+)$ form a commutative  group, which means that:
                \begin{enumerate}
                  \item $(\forall \overrightarrow{v}\in\mathcal{V})(\forall \overrightarrow{v}\in\mathcal{V})\hspace*{1mm}
                      \overrightarrow{v}+\overrightarrow{v}\in\mathcal{V}$
                   (\index{grupoidnost}\textit{closure}),
                  \item $(\forall \overrightarrow{u}\in\mathcal{V})
                        (\forall \overrightarrow{v}\in\mathcal{V})
                        (\forall \overrightarrow{w}\in\mathcal{V})
                        \hspace*{1mm}
                      \left(\overrightarrow{u}+\overrightarrow{v}\right)+\overrightarrow{w}
                      = \overrightarrow{u}+\left(\overrightarrow{v}+\overrightarrow{w}\right)
                      $
                   (\index{asociativnost}\textit{associativity}),
                  \item $(\exists \overrightarrow{e}\in\mathcal{V})
                        (\forall \overrightarrow{v}\in\mathcal{V})
                        \hspace*{1mm}
                     \overrightarrow{v}+\overrightarrow{e}
                      = \overrightarrow{e}+\overrightarrow{v}
                      $
                   (\index{nevtralni element}\textit{identity element}),
                  \item  $(\forall \overrightarrow{v}\in\mathcal{V})
                        (\exists \overrightarrow{u}\in\mathcal{V})
                        \hspace*{1mm}
                     \overrightarrow{v}+\overrightarrow{u}
                      = \overrightarrow{u}+\overrightarrow{v}=\overrightarrow{e}
                      $
                   (\index{inverzni element}\textit{inverse element}),
                  \item   $(\forall \overrightarrow{v}\in\mathcal{V})
                        (\forall \overrightarrow{u}\in\mathcal{V})
                        \hspace*{1mm}
                     \overrightarrow{v}+\overrightarrow{u}
                      = \overrightarrow{u}+\overrightarrow{v}
                      $
                   (\index{komutativnost}\textit{commutativity}).
                \end{enumerate}
                \eizrek

\textbf{\textit{Proof.}}

 \textit{1)} The sum of two arbitrary vectors is a vector, which follows from the definition of the sum of two vectors.

 \textit{2)} Let $\overrightarrow{u}$, $\overrightarrow{v}$ and $\overrightarrow{w}$ be arbitrary vectors and $A$, $B$, $C$ and $D$ such points that $\overrightarrow{u}=\overrightarrow{AB}$, $\overrightarrow{v}=\overrightarrow{BC}$ and $\overrightarrow{w}=\overrightarrow{CD}$ (statement \ref{vektAvObst1TockaB}) (Figure \ref{sl.vek.5.1.9a.pic}).


\begin{figure}[!htb]
\centering
\input{sl.vek.5.1.9a.pic}
\caption{} \label{sl.vek.5.1.9a.pic}
\end{figure}


  By the definition of vector addition (and the correctness of this definition - statement \ref{vektKorektDefSest}) it is:
 \begin{eqnarray*}
 & & \left(\overrightarrow{u}+\overrightarrow{v}\right)+\overrightarrow{w}=
     \left(\overrightarrow{AB}+\overrightarrow{BC}\right)+\overrightarrow{CD}=
     \overrightarrow{AC}+\overrightarrow{CD}=\overrightarrow{AD}\\
  & & \overrightarrow{u}+\left(\overrightarrow{v}+\overrightarrow{w}\right)                     =\overrightarrow{AB}+\left(\overrightarrow{BC}+\overrightarrow{CD}\right)=
     \overrightarrow{AB}+\overrightarrow{BD}=\overrightarrow{AD}
 \end{eqnarray*}
  From this it follows that the addition of vectors is associative.

 \textit{3)}  In the case of vector addition, the neutral element is the vector of zero or $\overrightarrow{e}=\overrightarrow{0}$, because for every vector $\overrightarrow{v}=\overrightarrow{AB}$ it is true:

 \begin{eqnarray*}
 & & \overrightarrow{v}+\overrightarrow{0}=
     \overrightarrow{AB}+\overrightarrow{BB}=
     \overrightarrow{AB}=\overrightarrow{v}\\
  & & \overrightarrow{0}+\overrightarrow{v}=
     \overrightarrow{AA}+\overrightarrow{AB}=
     \overrightarrow{AB}=\overrightarrow{v}.
 \end{eqnarray*}


 \textit{4)}  In the case of vector addition, for an arbitrary vector $\overrightarrow{v}=\overrightarrow{AB}$ the inverse element is the opposite vector $\overrightarrow{u}=-\overrightarrow{v}=\overrightarrow{BA}$:

\begin{eqnarray*}
 & & \overrightarrow{v}+\left(-\overrightarrow{v}\right)=
     \overrightarrow{AB}+\overrightarrow{BA}=
     \overrightarrow{AA}=\overrightarrow{0}\\
  & & -\overrightarrow{v}+\overrightarrow{v}=
     \overrightarrow{BA}+\overrightarrow{AB}=
     \overrightarrow{BB}=\overrightarrow{0}.
 \end{eqnarray*}

 \textit{5)} Let $v$ and $u$ be any vectors and $A$, $B$ and $C$ such points that $\overrightarrow{v}=\overrightarrow{AB}$ and $\overrightarrow{u}=\overrightarrow{BC}$ (statement \ref{vektAvObst1TockaB}). We will consider two examples (Figure \ref{sl.vek.5.1.9b.pic}).


\begin{figure}[!htb]
\centering
\input{sl.vek.5.1.9b.pic}
\caption{} \label{sl.vek.5.1.9b.pic}
\end{figure}

 \textit{a)} We assume that points $A$, $B$ and $C$ are nonlinear or $\overrightarrow{v}$ and $\overrightarrow{u}$ are nonlinear vectors. We mark with $D$ the fourth vertex of the parallelogram $ABCD$. By definition of the relation $\varrho$ it is $(A,B)\varrho (D,C)$ and $(B,C)\varrho (A,D)$, by definition of vectors it then follows that $\overrightarrow{AB}=\overrightarrow{DC}$ and $\overrightarrow{BC}=\overrightarrow{AD}$. So:
  $$\overrightarrow{v}+\overrightarrow{u}=
  \overrightarrow{AB}+\overrightarrow{BC}=
  \overrightarrow{AC}=
  \overrightarrow{AD}+\overrightarrow{DC}=
  \overrightarrow{BC}+\overrightarrow{AB}
                      = \overrightarrow{u}+\overrightarrow{v}.$$

\textit{a)} Let $A$, $B$ and $C$ be collinear points or $\overrightarrow{v}$ and $\overrightarrow{u}$ be collinear vectors. Express the vectors $\overrightarrow{v}$ and $\overrightarrow{u}$ as a sum of $\overrightarrow{v}=\overrightarrow{v}_1+\overrightarrow{v}_2$ and $\overrightarrow{u}=\overrightarrow{u}_1+\overrightarrow{u}_2$, so that none of the vectors $\overrightarrow{v}_1$,  $\overrightarrow{v}_2$,  $\overrightarrow{u}_1$ and $\overrightarrow{u}_2$ are collinear. If we now use what was proven in case \textit{a)}, we get:
 $$\overrightarrow{v}+\overrightarrow{u}=
  \overrightarrow{v}_1+\overrightarrow{v}_2+
  \overrightarrow{u}_1+\overrightarrow{u}_2=
   \overrightarrow{u}_1+\overrightarrow{u}_2+
  \overrightarrow{v}_1+\overrightarrow{v}_2=
                       \overrightarrow{u}+\overrightarrow{v},$$ which is what needed to be proven. \kdokaz

The next statement is a consequence of the previous expression.

                \bzgled \label{vektABCD_ACBD}
                For arbitrary points $A$, $B$, $C$ and $D$ is
                $$\overrightarrow{AB}=\overrightarrow{CD}\Rightarrow \overrightarrow{AC}=\overrightarrow{BD}.$$
                \ezgled

\begin{figure}[!htb]
\centering
\input{sl.vek.5.1.10.pic}
\caption{} \label{sl.vek.5.1.10.pic}
\end{figure}

\textbf{\textit{Proof.}} By the definition of vector addition and the expression \ref{vektKomGrupa} (commutativity) is (Figure \ref{sl.vek.5.1.10.pic}):
$\overrightarrow{AC}=
\overrightarrow{AB}+ \overrightarrow{BC}=
\overrightarrow{CD}+ \overrightarrow{BC}=
\overrightarrow{BC}+\overrightarrow{CD}=
\overrightarrow{BD}.$
  \kdokaz

  Another consequence of the commutativity of vector addition (expression \ref{vektKomGrupa}) is the following rule of addition for non-collinear vectors.

  \begin{itemize}
    \item \textit{For every three non-collinear points $A$, $B$ and $C$ is $\overrightarrow{AB}+\overrightarrow{AC}=\overrightarrow{AD}$ exactly when $ABDC$ is a parallelogram.}
  \end{itemize}


\begin{figure}[!htb]
\centering
\input{sl.vek.5.1.11.pic}
\caption{} \label{sl.vek.5.1.11.pic}
\end{figure}

This rule is called the \index{pravilo!paralelogramsko}\pojem{paralelogramsko pravilo}\footnote{Paralelogramsko pravilo was probably known to the ancient Greeks. It is assumed that the ancient Greek mathematician and philosopher \index{Aristotel} \textit{Aristotel} (384--322 BC) mentioned it.} (Figure \ref{sl.vek.5.1.11.pic}).

The consequence of the associativity from the izrek \ref{vektKomGrupa} is the following rule for the addition of vectors:
  \begin{itemize}
    \item $\overrightarrow{A_1A_2}+\overrightarrow{A_2A_3}+\cdots +\overrightarrow{A_{n-1}A_n}=\overrightarrow{A_1A_n}$,
  \end{itemize}

which is called the \index{pravilo!poligonsko}\pojem{poligonsko pravilo} (Figure \ref{asl.vek.5.1.12.pic}).

\begin{figure}[!htb]
\centering
\input{sl.vek.5.1.12.pic}
\caption{} \label{asl.vek.5.1.12.pic}
\end{figure}

The direct consequence of this rule is the following assertion (Figure \ref{sl.vek.5.1.13.pic}).

                \bzgled
                For arbitrary points $A_1$, $A_2$,..., $A_n$ in a plane is
                $$\overrightarrow{A_1A_2}+\overrightarrow{A_2A_3}+\cdots +\overrightarrow{A_{n-1}A_n}+\overrightarrow{A_nA_1}=\overrightarrow{0}.$$
                \ezgled

\begin{figure}[!htb]
\centering
\input{sl.vek.5.1.13.pic}
\caption{} \label{sl.vek.5.1.13.pic}
\end{figure}

With the following izrek we will give an equivalent definition of the center of a line.

                \bzgled \label{vektSredDalj}
                A point  $S$ is the midpoint of a line segment  $AB$ if and only if $\overrightarrow{SA}=-\overrightarrow{SB}$ i.e. $\overrightarrow{SA}+\overrightarrow{SB}=\overrightarrow{0}$.
                \ezgled

\begin{figure}[!htb]
\centering
\input{sl.vek.5.1.14.pic}
\caption{} \label{sl.vek.5.1.14.pic}
\end{figure}

 \textbf{\textit{Solution.}} (Figure \ref{sl.vek.5.1.14.pic})

($\Leftarrow$) Let's assume that $\overrightarrow{SA}=-\overrightarrow{SB}$ or $\overrightarrow{AS}=\overrightarrow{SB}$. This means that $AS$ and $SB$ are parallel, have the same direction and are consistent lines (statement \ref{vektVzpSkl}) or $SA\cong BS$ and $\mathcal{B}(A,S,B)$. So $S$ is the center of the line $AB$.

($\Rightarrow$) Now let $S$ be the center of the line $AB$. It is enough to prove that $\overrightarrow{SA}=\overrightarrow{BS}$ or $\overrightarrow{BS}=\overrightarrow{SA}$.
 Let's assume that $\overrightarrow{BS}=\overrightarrow{SA'}$.
 But in this case $SA'\cong BS$ and $\mathcal{B}(A',S,B)$. By statement \ref{ABnaPoltrakCX} we have $A=A'$, so $\overrightarrow{BS}=\overrightarrow{SA'}=\overrightarrow{SA}$.
\kdokaz

Let's also define the operation of subtracting two vectors. \pojem{The difference of vectors} \index{razlika!vektorjev} $\overrightarrow{v}$ and $\overrightarrow{u}$ is the sum of vectors $\overrightarrow{v}$ and $-\overrightarrow{u}$ or
$$\overrightarrow{v}-\overrightarrow{u}=\overrightarrow{v}+(-\overrightarrow{u}).$$

            \bzgled \label{vektOdsev}
            For arbitrary three points $O$, $A$ and $B$ is
            $$\overrightarrow{OB}-\overrightarrow{OA}=\overrightarrow{AB}.$$
            \ezgled


\begin{figure}[!htb]
\centering
\input{sl.vek.5.1.15.pic}
\caption{} \label{sl.vek.5.1.15.pic}
\end{figure}

 \textbf{\textit{Solution.}} (Figure \ref{sl.vek.5.1.15.pic})

 From the definition of subtraction and addition of vectors and commutativity of addition
 (statement \ref{vektKomGrupa}) it follows:
 $$\overrightarrow{OB}-\overrightarrow{OA}=
 \overrightarrow{OB}+(-\overrightarrow{OA})=
 \overrightarrow{OB}+\overrightarrow{AO}=
\overrightarrow{AO}+\overrightarrow{OB}=
 \overrightarrow{AB},$$ which had to be proven. \kdokaz

\bzgled \label{vektOPi}
           Suppose that a point $O$ lies on a line $p$ and let $P_1, P_2, \cdots, P_n$ be points
            lying in the same half-plane $\alpha$ with the edge $p$.
            If:
            $$\overrightarrow{OS_n}=\overrightarrow{OP_1}+\overrightarrow{OP_2}+
               \cdots+\overrightarrow{OP_n},$$
            then the point $S_n$ also lies in the half-plane $\alpha$.
           \ezgled

\begin{figure}[!htb]
\centering
\input{sl.vek.5.2.1a.pic}
\caption{} \label{sl.vek.5.2.1a.pic}
\end{figure}

 \textbf{\textit{Solution.}} We will prove the statement by induction
 according to $n$ (Figure \ref{sl.vek.5.2.1a.pic}).

 (\textit{i}) For $n=1$ it is clear that $\overrightarrow{OS_1}=\overrightarrow{OP_1}$,
 and $P_1\in\alpha\Rightarrow S_1=P_1\in\alpha$.

(\textit{ii}) We assume that the statement is true for $n=k$. We will prove
 that it is then also true for $n=k+1$. Let $P_1, P_2, \cdots, P_k,
 P_{k+1}\in\alpha$. We will prove that then also $S_{k+1}\in
 \alpha$. It holds:
   $$\overrightarrow{OS_{k+1}}=\overrightarrow{OP_1}+\overrightarrow{OP_2}+
         \cdots+\overrightarrow{OP_k}+\overrightarrow{OP_{k+1}}=
         \overrightarrow{OS_k}+\overrightarrow{OP_{k+1}}.$$
The point $S_k$ by the induction assumption lies in the half-plane
$\alpha$. Because  $S_k,P_{k+1}\in \alpha$ and $O\in p$, the angle $\angle
S_kOP_{k+1}$ is convex and the whole angle lies in the half-plane $\alpha$. This
means that also the point $S_{n+1}$ (as the fourth vertex of the parallelogram $P_{k+1}OP_kS_{k+1}$) lies in the interior of the angle $\angle
S_kOP_{k+1}$ or in the half-plane $\alpha$.

 \kdokaz

With the concept of vector addition, we can define the concept of multiplying a vector by an arbitrary natural number:
$$1\cdot \overrightarrow{v} =\overrightarrow{v},\hspace*{2mm} (n+1)\cdot \overrightarrow{v}=n\cdot\overrightarrow{v}+\overrightarrow{v},\hspace*{2mm} (n\in \mathbb{N}).$$
We can extend this definition to whole numbers: $0\cdot \overrightarrow{v}=\overrightarrow{0}$ and $-n\cdot \overrightarrow{v}=n\cdot(-\overrightarrow{v})$.
It is clear that in this case, the vectors $\overrightarrow{v}$ and $l\cdot \overrightarrow{v}$ ($l\in \mathbb{Z}$) are always collinear and $|l\cdot \overrightarrow{v}|=|l|\cdot |\overrightarrow{v}|$.
This gives us an idea for the definition of multiplying a vector by an arbitrary real number $\lambda\in \mathbb{R}$.

First, for $\lambda=0$ let us define $0\cdot \overrightarrow{v}=\overrightarrow{0}$. If
$\lambda\neq 0$ and $\overrightarrow{v}=\overrightarrow{AB}$, then $\lambda\cdot\overrightarrow{v}=\overrightarrow{AC}$, where $C$ is such a point on the line $AB$ that (Figure \ref{sl.vek.5.2.1.pic}):

 \begin{itemize}
   \item $|AC|=|\lambda| \cdot|AB|$,
   \item $C,B\ddot{-} A$, if $\lambda>0$,
   \item $C,B\div A$, if $\lambda<0$.
 \end{itemize}


\begin{figure}[!htb]
\centering
\input{sl.vek.5.2.1.pic}
\caption{} \label{sl.vek.5.2.1.pic}
\end{figure}


 It is clear that the point $C$ is uniquely determined for each $\lambda\neq 0$, and the vector $\lambda\cdot \overrightarrow{v}$ does not depend on the choice of point $A$. This means that the definition of multiplying a vector by a real number is correct.

From the definition itself it follows that for every vector $\overrightarrow{v}$ and every real number $\lambda\in \mathbb{R}$ it holds that $|\lambda \cdot \overrightarrow{v}|=|\lambda| \cdot |\overrightarrow{v}|$.

Multiplying the vector $\overrightarrow{v}$ by the real number $\lambda$ will also be called \index{množenje vektorja s skalarjem}\pojem{multiplying the vector $\overrightarrow{v}$ by the scalar $\lambda$}.

Similarly to how we omit the multiplication sign $\cdot$ when multiplying algebraic expressions, we will also omit the multiplication sign $\cdot$ when multiplying a vector by a real number, or rather than writing $\lambda\cdot\overrightarrow{v}$ we will just write $\lambda\overrightarrow{v}$.

The next theorem gives us the necessary and sufficient condition for the collinearity of two vectors.


            \bizrek \label{vektKriterijKolin}
            Vectors $\overrightarrow{v}$ and $\overrightarrow{u}$ ($\overrightarrow{v},\overrightarrow{u}\neq \overrightarrow{0}$) are collinear if and only if there is such $\lambda\in \mathbb{R}$, that is $\overrightarrow{u}=\lambda\cdot\overrightarrow{v}$.
            \eizrek


\begin{figure}[!htb]
\centering
\input{sl.vek.5.2.2.pic}
\caption{} \label{sl.vek.5.2.2.pic}
\end{figure}

 \textbf{\textit{Proof.}} (Figure \ref{sl.vek.5.2.2.pic})

  Let $\overrightarrow{v}$ and $\overrightarrow{u}$ be any vectors ($\overrightarrow{v},\overrightarrow{u}\neq \overrightarrow{0}$), $A$ any point, and $B$ and $C$ such points that $\overrightarrow{AB}=\overrightarrow{v}$ and $\overrightarrow{AC}=\overrightarrow{u}$ (statement \ref{vektAvObst1TockaB}).


 ($\Leftarrow$) First, let us assume that $\overrightarrow{u}=\lambda\cdot\overrightarrow{v}$ for some $\lambda\in \mathbb{R}$. By definition, point $C$ lies on the line $AB$, which means that vectors $\overrightarrow{AB}$ and $\overrightarrow{AC}$, or $\overrightarrow{v}$ and $\overrightarrow{u}$, are collinear.

 ($\Rightarrow$) Now let us assume that vectors $\overrightarrow{v}$ and $\overrightarrow{u}$, or $\overrightarrow{AB}$ and $\overrightarrow{AC}$, are collinear.
  In this case:
  \begin{eqnarray*}
  \overrightarrow{u}&=&\overrightarrow{AC}=\frac{|AC|}{|AB|}\cdot \overrightarrow{AB}=
  \frac{|\overrightarrow{u}|}{|\overrightarrow{v}|}\cdot \overrightarrow{v},
  \hspace*{2mm} \textrm{ if } C,B\ddot{-} A;\\
  \overrightarrow{u}&=&\overrightarrow{AC}=-\frac{|AC|}{|AB|}\cdot \overrightarrow{AB}=
  -\frac{|\overrightarrow{u}|}{|\overrightarrow{v}|}\cdot \overrightarrow{v},
  \hspace*{2mm} \textrm{ if } C,B\div A,
  \end{eqnarray*}
  which is what needed to be proven. \kdokaz

If we combine the operation of adding vectors and multiplying a vector by a scalar, we get what is called a linear combination of vectors. More precisely - for any $n$-tuple of vectors $(\overrightarrow{a_1},\overrightarrow{a_2},\ldots,\overrightarrow{a_n})$ and any $n$-tuple of real numbers $(\alpha_1,\alpha_2,\ldots,\alpha_n)\in \mathbb{R}^n$ the vector
 $$\overrightarrow{v}=\alpha_1\cdot \overrightarrow{a_1}+
 \alpha_2\cdot \overrightarrow{a_2}
 +\cdots +\alpha_n\cdot \overrightarrow{a_n}$$
 \index{linearna kombinacija vektorjev}\pojem{linearna kombinacija vektorjev} $\overrightarrow{a_1}$, $\overrightarrow{a_2}$, $\ldots$, $\overrightarrow{a_n}$.



                \bizrek \label{vektLinKombNicLema}
                Let $\overrightarrow{a}$ and $\overrightarrow{b}$ be two non-collinear non-zero vectors
                and  $(\alpha,\beta)\in \mathbb{R}^2$ two real numbers.
                If $\alpha\cdot\overrightarrow{a}=
                \beta\cdot\overrightarrow{b}$, then $\alpha=\beta=0$.
                \eizrek

 \textbf{\textit{Proof.}} We assume the opposite - without loss of generality, that $\alpha\neq0$. Then $\overrightarrow{a}=
 \frac{\beta}{\alpha}\cdot\overrightarrow{b}$, which means (statement \ref{vektKriterijKolin}), that the vector $\overrightarrow{b}$ is collinear with the vector $\overrightarrow{a}$. This is in contradiction with the assumption, so $\alpha=\beta=0$.
\kdokaz

A direct consequence of the previous statement is the following claim.



                 \bizrek \label{vektLinKombNic}
                If for the linear combination of two non-collinear non-zero vectors $\overrightarrow{a}$ and $\overrightarrow{b}$ holds
                $$\alpha\cdot\overrightarrow{a}+
                \beta\cdot\overrightarrow{b}=\overrightarrow{0},$$
                 then $\alpha=\beta=0$.
                \eizrek

\textbf{\textit{Proof.}} If we write the relation $\alpha\cdot\overrightarrow{a}+
 \beta\cdot\overrightarrow{b}=\overrightarrow{0}$ in the form $\alpha\cdot\overrightarrow{a}=
 -\beta\cdot\overrightarrow{b}$, we see that the statement is a direct consequence of the previous expression.
\kdokaz

 The following expression is very important, which refers to the representation of any vector as a linear combination of two nonlinear non-zero vectors.


                \bizrek \label{vektLinKomb1Razcep}
                Let $\overrightarrow{a}$ and $\overrightarrow{b}$ be two non-collinear non-zero vectors in the same plane.
                Each vector $\overrightarrow{v}$ in this plane can be express  in a single way as a linear combination
                of vectors  $\overrightarrow{a}$ and $\overrightarrow{b}$, i.e. there is exactly
                one pair of real numbers $(\alpha,\beta)\in \mathbb{R}^2$, such that
                $$\overrightarrow{v}=\alpha\cdot\overrightarrow{a}+
                \beta\cdot\overrightarrow{b}.$$
                \eizrek



\begin{figure}[!htb]
\centering
\input{sl.vek.5.2.3.pic}
\caption{} \label{sl.vek.5.2.3.pic}
\end{figure}

 \textbf{\textit{Proof.}}  (Figure \ref{sl.vek.5.2.3.pic})

 Let $O$ be an arbitrary point and $A$, $B$ and $V$ such points that $\overrightarrow{OA}=\overrightarrow{a}$, $\overrightarrow{OB}=\overrightarrow{b}$  and $\overrightarrow{OV}=\overrightarrow{v}$ (expression \ref{vektAvObst1TockaB}). We mark with $A'$ the intersection of the line $OA$ and the parallel of the line $OB$ through the point $V$ and with $B'$ the intersection of the line $OB$ and the parallel of the line $OA$ through the point $V$. So the quadrilateral $OA'VB'$ is a parallelogram. By the expression \ref{vektKriterijKolin} there exist such $\alpha,\beta\in \mathbb{R}$, that $\overrightarrow{OA'}=
 \alpha\cdot\overrightarrow{OA}=
 \alpha\cdot\overrightarrow{a}$ and
 $\overrightarrow{OB'}=
 \beta\cdot\overrightarrow{OB}=
 \beta\cdot\overrightarrow{b}$. By the parallelogram rule  it is:

 $$\overrightarrow{v}=
 \overrightarrow{OV}=\overrightarrow{OA'}+\overrightarrow{OB'}=
 \alpha\cdot\overrightarrow{a}+\beta\cdot\overrightarrow{b}.$$

We will prove that $(\alpha,\beta)\in \mathbb{R}^2$ is the only such pair. We assume that for
$(\alpha',\beta')\in \mathbb{R}^2$ it is true that $\overrightarrow{v}=
 \alpha'\cdot\overrightarrow{a}+\beta'\cdot\overrightarrow{b}$. From
 $\overrightarrow{v}=
 \alpha'\cdot\overrightarrow{a}+\beta'\cdot\overrightarrow{b}=
 \alpha\cdot\overrightarrow{a}+\beta\cdot\overrightarrow{b}$ it follows that
 $(\alpha'-\alpha)\cdot\overrightarrow{a}+(\beta'-\beta)\cdot\overrightarrow{b}=
\overrightarrow{0}$. According to the previous statement \ref{vektLinKombNic} we have that
  $\alpha'-\alpha=\beta'-\beta=0$ which means $\alpha'=\alpha$ and $\beta'=\beta$.
\kdokaz

\bizrek \label{vektVektorskiProstor}
                The set $\mathcal{V}$ of all vectors in the plane form so-called
                 \index{vector space} vector space \footnote{ The first one to, in a modern way, define the concept of a vector space in 1888 was the Italian mathematician \index{Peano, G.}\textit{G. Peano} (1858–-1932).} over the field $\mathbb{R}$, which means that:
                \begin{enumerate}
                  \item the ordered pair $(\mathcal{V},+)$ form a commutative  group,
                  \item $(\forall \alpha \in \mathbb{R})(\forall \overrightarrow{v}\in \mathcal{V})\hspace*{1mm} \alpha\cdot \overrightarrow{v} \in \mathcal{V}$,
                  \item $(\forall \alpha,\beta \in \mathbb{R})(\forall \overrightarrow{v}\in \mathcal{V})\hspace*{1mm} \alpha\cdot (\beta\cdot\overrightarrow{v})=(\alpha\beta)\cdot \overrightarrow{v}$,
                  \item $(\forall \overrightarrow{v}\in \mathcal{V})\hspace*{1mm} 1\cdot \overrightarrow{v}=\overrightarrow{v}$,
                  \item $(\forall \alpha,\beta \in \mathbb{R})(\forall \overrightarrow{v}\in \mathcal{V})\hspace*{1mm} (\alpha+ \beta)\cdot\overrightarrow{v}=\alpha\cdot \overrightarrow{v}+\beta\cdot \overrightarrow{v}$,
                  \item $(\forall \alpha \in \mathbb{R})(\forall \overrightarrow{v},\overrightarrow{u}\in \mathcal{V})\hspace*{1mm} \alpha\cdot(\overrightarrow{v}+\overrightarrow{u})=
                      \alpha\cdot \overrightarrow{v}+\alpha\cdot \overrightarrow{u}$.
                \end{enumerate}
                \eizrek



 \textbf{\textit{Proof.}} The claim $\textit{1}$ is, in a sense, already proven in \ref{vektKomGrupa}. The claims $\textit{2}-\textit{5}$ are a direct consequence of the definition of multiplication of a vector with a scalar. The proof of claim $\textit{6}$ is not as simple - in the proof we would have to use Dedekind's axiom of continuity \ref{aksDed}.
 \kdokaz

We will use the notation $\mathcal{\overrightarrow{V}}^2$ for the vector space structure on the set of all vectors of the plane $\mathcal{V}$ over the field $\mathbb{R}$. If we were to consider vectors in Euclidean space, we would also get a vector space. But we can also consider a vector space more abstractly as an arbitrary ordered pair $(\mathcal{\overrightarrow{V}},\mathcal{F})$, because $(\mathcal{\overrightarrow{V}},+)$ represents a commutative group, $(\mathcal{F},+,\cdot)$ is a field, and all conditions $\textit{2}-\textit{6}$ from the previous theorem are satisfied.

Because every vector in the plane can be expressed as a linear combination of two non-zero vectors $\overrightarrow{a}$ and $\overrightarrow{b}$ of that plane, we say that such two vectors $\overrightarrow{a}$ and $\overrightarrow{b}$ are called the \index{baza vektorskega prostora}\pojem{baza} (base) of the vector space $\mathcal{\overrightarrow{V}}^2$. If for some vector $\overrightarrow{v}\in \mathcal{V}$ it holds that $\overrightarrow{v}=\alpha\cdot\overrightarrow{a}+
                \beta\cdot\overrightarrow{b}$, we say that the pair $(\alpha,\beta)$ represents the \index{koordinate vektorja}\pojem{koordinate} (coordinates) of the vector $\overrightarrow{v}$ in the base $(\overrightarrow{a},\overrightarrow{b})$. From theorem \ref{vektLinKomb1Razcep} it follows that coordinates of every vector are uniquely determined in any base.



Therefore, there are infinitely many bases in the vector space $\mathcal{\overrightarrow{V}}^2$, the number of vectors in any base is always 2, so we say that the vector space $\mathcal{\overrightarrow{V}}^2$ has \pojem{dimension} 2 or is \pojem{two-dimensional}.

It is clear that the dimension of the vector space $\mathcal{\overrightarrow{V}}^3$, which is determined by the vectors in Euclidean space, is equal to 3. Euclidean space is therefore \pojem{three-dimensional}, each vector in some base is determined by a triplet of real numbers, which represent the coordinates of that vector.

In any vector space we determine the base and dimension in a similar way. Because in general the dimension of space is not limited to 3, this allows us to explore \pojem{$n$-dimensional} Euclidean spaces (for any $n\in \mathbb{N}$).

The following examples are very useful.

\bzgled \label{vektSredOSOAOB}
                Let $O$, $A$ and $B$ be arbitrary points  and $S$ the midpoint of the line segment $AB$. Then
                $$\overrightarrow{OS}=\frac{1}{2}\cdot\left(\overrightarrow{OA}+
                \overrightarrow{OB}\right).$$
                \ezgled


\begin{figure}[!htb]
\centering
\input{sl.vek.5.2.4.pic}
\caption{} \label{sl.vek.5.2.4.pic}
\end{figure}

 \textbf{\textit{Proof.}}  (Figure \ref{sl.vek.5.2.4.pic})

  By \ref{vektSredDalj} we have $\overrightarrow{SA}=-\overrightarrow{SB}$ or $\overrightarrow{AS}=-\overrightarrow{BS}$. By the triangle rule for vector addition we have
  $\overrightarrow{OS}=\overrightarrow{OA}+\overrightarrow{AS}$ and
  $\overrightarrow{OS}=\overrightarrow{OB}+\overrightarrow{BS}$. If we add the equalities and take into account $\overrightarrow{AS}=-\overrightarrow{BS}$, we get: $2\cdot\overrightarrow{OS}=\overrightarrow{OA}+\overrightarrow{AS}
  +\overrightarrow{OB}+\overrightarrow{BS}=\overrightarrow{OA}
  +\overrightarrow{OB}$
 or
$\overrightarrow{OS}=\frac{1}{2}\cdot\left(\overrightarrow{OA}+
                \overrightarrow{OB}\right)$.
\kdokaz

                \bzgled \label{vektDelitDaljice}
                Let $O$, $A$ and $B$ be arbitrary points  and  $P$ a point in the line segment $AB$ such that $|AP|:|PB|=n:m$. Then
                $$\overrightarrow{OP}=\frac{1}{n+m}\left(m\cdot\overrightarrow{OA}+
                n\cdot\overrightarrow{OB}\right).$$
                \ezgled


\begin{figure}[!htb]
\centering
\input{sl.vek.5.2.5.pic}
\caption{} \label{sl.vek.5.2.5.pic}
\end{figure}

 \textbf{\textit{Proof.}}  (Figure \ref{sl.vek.5.2.5.pic})

First, from $|AP|:|PB|=n:m$ it follows that $|AP|=\frac{n}{n+m}\cdot|AB|$ and $|BP|=\frac{m}{n+m}\cdot|AB|$ or $\overrightarrow{AP}=\frac{n}{n+m}\cdot\overrightarrow{AB}$ and $\overrightarrow{BP}=\frac{m}{n+m}\cdot\overrightarrow{BA}$ (because $\mathcal{B}(A,P,B)$).
Then:
\begin{eqnarray*}
  \overrightarrow{OP}&=&\overrightarrow{OA}+\overrightarrow{AP}=\overrightarrow{OA}
  +\frac{n}{n+m}\cdot\overrightarrow{AB}\hspace*{3mm} \textrm{ in}\\
  \overrightarrow{OP}&=&\overrightarrow{OB}+\overrightarrow{BP}=\overrightarrow{OB}
  +\frac{m}{n+m}\cdot\overrightarrow{BA}.
\end{eqnarray*}
  If we multiply the first equality by $m$, the second by $n$ and then add them, we get:
\begin{eqnarray*}
(n+m)\cdot\overrightarrow{OP}&=&m\cdot\overrightarrow{OA}
  +n\cdot\overrightarrow{OB}+\frac{nm}{n+m}\cdot\left(\overrightarrow{AB}+
  \overrightarrow{BA}\right)=\\
  &=&m\cdot\overrightarrow{OA}
  +n\cdot\overrightarrow{OB}.
\end{eqnarray*}
  If we divide the obtained equality by $n+m$, we get the desired relation.
\kdokaz


                \bzgled \label{vektParamDaljica}
                Let $O$, $A$ and $B$ be arbitrary points. A point $X$ lies on the line segment $AB$ if and only if for some scalar $0\leq\lambda\leq1$ is
                $$\overrightarrow{OX}=(1-\lambda)\cdot\overrightarrow{OA}+
                \lambda\cdot\overrightarrow{OB}.$$
                \ezgled


 \textbf{\textit{Proof.}}  (Figure \ref{sl.vek.5.2.6.pic})

 We first assume that the point $X$ lies on the line $AB$. Then for some  $0\leq\lambda\leq1$ it holds that $\overrightarrow{AX}=\lambda\cdot \overrightarrow{AB}$.

 \begin{eqnarray*}
  \overrightarrow{OX}&=&\overrightarrow{OA}+\overrightarrow{AX}=\\
  &=& \overrightarrow{OA}+\lambda\cdot \overrightarrow{AB}=\\
  &=& \overrightarrow{OA}+\lambda\cdot \left(\overrightarrow{OB}-\overrightarrow{OA}\right)=\\
  &=& (1-\lambda)\cdot\overrightarrow{OA}+
                \lambda\cdot\overrightarrow{OB}.
\end{eqnarray*}

Assume now that for some $0\leq\lambda\leq1$ we have
                $\overrightarrow{OX}=(1-\lambda)\cdot\overrightarrow{OA}+
                \lambda\cdot\overrightarrow{OB}$. Then:

 \begin{eqnarray*}
  \overrightarrow{AX}&=&\overrightarrow{AO}+\overrightarrow{OX}=\\
  &=&\overrightarrow{AO}+(1-\lambda)\cdot\overrightarrow{OA}+
                \lambda\cdot\overrightarrow{OB}=\\
 &=&-\lambda\cdot\overrightarrow{OA}+
                \lambda\cdot\overrightarrow{OB}=\\
 &=&\lambda\cdot\left(\overrightarrow{AO}+
                \overrightarrow{OB}\right)=\\
 &=&\lambda\cdot\overrightarrow{AB}.
\end{eqnarray*}
Therefore, for some  $0\leq\lambda\leq1$ we have $\overrightarrow{AX}=\lambda\cdot \overrightarrow{AB}$,
 which means that the point $X$ lies on the line $AB$.
\kdokaz

\begin{figure}[!htb]
\centering
\input{sl.vek.5.2.6.pic}
\caption{} \label{sl.vek.5.2.6.pic}
\end{figure}

A direct consequence is the following theorem.

  $O$, $A$ in $B$ . A point $X$ lies on the line segment $AB$ if and only if for some

                \bzgled
                Let $O$, $A$ and $B$ be arbitrary points. A point $X$ lies on the line segment $AB$ if and only if for some scalars $\alpha, \beta\in [0,1]$ ($\alpha+\beta=1$) is
                $$\overrightarrow{OX}=\alpha\cdot\overrightarrow{OA}+
                \beta\cdot\overrightarrow{OB}.$$
                \ezgled

In the next theorem we will give the \index{vector equation of a line} \pojem{vector equation of a line}.

                \bzgled \label{vektParamPremica}
                Let $O$, $A$ in $B$ be arbitrary points. A point $X$ lies on the line $AB$ if and only if for some scalar $\lambda\in \mathbb{R}$ is
               $$\overrightarrow{OX}=(1-\lambda)\cdot\overrightarrow{OA}+
                \lambda\cdot\overrightarrow{OB}.$$
                \ezgled


 \textbf{\textit{Proof.}}  (Figure \ref{sl.vek.5.2.6.pic})

The proof is the same as in the previous statement, only that we use the fact that the point $X$ lies on the line $AB$ exactly when the vectors $\overrightarrow{AX}$ and $\overrightarrow{AB}$ are collinear, or for some
$\lambda\in \mathbb{R}$ it holds that $\overrightarrow{AX}=\lambda\cdot \overrightarrow{AB}$ (statement \ref{vektKriterijKolin}).
\kdokaz

\index{količnik kolinearnih vektorjev}\index{razmerje!kolinearnih vektorjev}
If for collinear vectors $\overrightarrow{v}$ and $\overrightarrow{u}$ it holds that $\overrightarrow{v}=\lambda \overrightarrow{u}$ ($\lambda\in\mathbb{R}$) and $\overrightarrow{u}\neq \overrightarrow{0}$, then we can define the \pojem{količnik kolinearnih vektorjev} or the \pojem{razmerje kolinearnih vektorjev}  (Figure \ref{sl.vek.5.2.7.pic}):
$$\overrightarrow{v}:\overrightarrow{u}
=\frac{\overrightarrow{v}}{\overrightarrow{u}}=\lambda.$$


\begin{figure}[!htb]
\centering
\input{sl.vek.5.2.7.pic}
\caption{} \label{sl.vek.5.2.7.pic}
\end{figure}

From the definition of multiplication of a vector with a real number we get the following statement.

                    \bizrek \label{vektKolicnDolz}
                    Let $\overrightarrow{v}$ and $\overrightarrow{u}$ be collinear vectors and $\overrightarrow{u}\neq \overrightarrow{0}$. Then
                    $$\frac{\overrightarrow{v}}{\overrightarrow{u}}=
                    \left\{ \begin{array}{lll}
                    \frac{|\overrightarrow{v}|}{|\overrightarrow{u}|},   &
                    \textrm{ if } \overrightarrow{u}\rightrightarrows \overrightarrow{v}\\
                     -\frac{|\overrightarrow{v}|}{|\overrightarrow{u}|},      &
                     \textrm{ if } \overrightarrow{u}\rightleftarrows\overrightarrow{v}\\
                     0,  & \textrm{ if } \overrightarrow{v}=\overrightarrow{0}
                    \end{array}\right.$$
                    \eizrek


The following statement also refers to the newly defined concept.

\bizrek
                     \label{izrekEnaDelitevDaljiceVekt}
                     For any line segment $AB$ and any $\lambda\in\mathbb{R}\setminus\{-1\}$ there exists
                     exactly one such point $P$ on the line $AB$, that is
                     $$\frac{\overrightarrow{AP}}{\overrightarrow{PB}}=\lambda.$$
                    \eizrek

\begin{figure}[!htb]
\centering
\input{sl.vek.5.2.8.pic}
\caption{} \label{sl.vek.5.2.8.pic}
\end{figure}

 \textbf{\textit{Proof.}}  (Figure \ref{sl.vek.5.2.8.pic})

 Let $P$ be such a point that $$\overrightarrow{AP}=\frac{\lambda}{1+\lambda}\cdot\overrightarrow{AB}.$$
Since $\lambda\neq-1$, such a point always exists. Because $$\overrightarrow{PB}=\overrightarrow{PA}+\overrightarrow{AB}=
-\frac{\lambda}{1+\lambda}\cdot\overrightarrow{AB}+\overrightarrow{AB}=
\frac{1}{1+\lambda}\overrightarrow{AB},$$
it also holds that $\frac{\overrightarrow{AP}}{\overrightarrow{PB}}=\lambda$.

If for another point $P'$ it holds that $\frac{\overrightarrow{AP'}}{\overrightarrow{P'B}}=\lambda$, from
$\overrightarrow{AP'}=\lambda\cdot\overrightarrow{P'B}$ and
$\overrightarrow{P'B}=\overrightarrow{P'A}+\overrightarrow{AB}$ we get $\overrightarrow{AP'}=\frac{\lambda}{1+\lambda}\cdot\overrightarrow{AB}$. Therefore
 $\overrightarrow{AP'}=\overrightarrow{AP}$ or
 $\overrightarrow{P'P}=\overrightarrow{P'A}+\overrightarrow{AP}=
 -\overrightarrow{AP}+\overrightarrow{AP}=\overrightarrow{0}$. So $P'=P$, which means that there is only one point $P$, for which $\frac{\overrightarrow{AP}}{\overrightarrow{PB}}=\lambda$.
 \kdokaz

 We say that the point $P$ from the previous statement divides the line segment $AB$ in the \pojem{ratio} $\lambda$.



                    \bzgled
                    Let $ABCD$ be a trapezium with the base $AB$. Calculate the ratio in which the line
                    segment $PD$ divides the diagonal $AC$, if $|AB| = 3\cdot |CD|$ and $P$ is the midpoint of the line segment $AB$.
                    \ezgled

\begin{figure}[!htb]
\centering
\input{sl.vek.5.2.9.pic}
\caption{} \label{sl.vek.5.2.9.pic}
\end{figure}

 \textbf{\textit{Solution.}}  The intersection of the line $DP$ and the diagonal $AC$ is denoted by $S$ (Figure \ref{sl.vek.5.2.9.pic})
and let
 $\overrightarrow{u}=\overrightarrow{AP}$ and $\overrightarrow{v}=\overrightarrow{AD}$.
The vector $\overrightarrow{AS}$ and $\overrightarrow{AC}$ are collinear, so by \ref{vektKriterijKolin} $\overrightarrow{AS}=\lambda\overrightarrow{AC}$, for some $\lambda\in \mathbb{R}$. The vectors $\overrightarrow{PS}$ and $\overrightarrow{PD}$ are also collinear, i.e. for some $\mu\in \mathbb{R}$ we have $\overrightarrow{PS}=\mu\overrightarrow{PD}$. The vector $\overrightarrow{AC}$ and $\overrightarrow{AS}$ are written as a linear combination of the vectors $\overrightarrow{u}$ and $\overrightarrow{v}$:
 \begin{eqnarray*}
\hspace*{-1.8mm} \overrightarrow{AC}&=&\overrightarrow{AD}+\overrightarrow{DC}
 =\overrightarrow{v}+\frac{2}{3}\overrightarrow{u}
 =\frac{2}{3}\overrightarrow{u}+\overrightarrow{v},\\
 \hspace*{-1.8mm}\overrightarrow{AS}&=&\overrightarrow{AP}+\overrightarrow{PS}
 =\overrightarrow{u}+\mu\overrightarrow{PD}
 =\overrightarrow{u}+\mu(-\overrightarrow{u}+\overrightarrow{v})
 =(1-\mu)\overrightarrow{u}+\mu\overrightarrow{v}.
 \end{eqnarray*}
Since $\overrightarrow{AS}=\lambda\overrightarrow{AC}$, we get:
\begin{eqnarray*}
 \overrightarrow{AS}&=&
 \frac{2}{3}\lambda\overrightarrow{u}+\lambda\overrightarrow{v};\\
 \overrightarrow{AS}&=&
 (1-\mu)\overrightarrow{u}+\mu\overrightarrow{v}.
 \end{eqnarray*}
 Because $\overrightarrow{u}$ and $\overrightarrow{v}$ are collinear vectors, by \ref{vektLinKomb1Razcep}  $\frac{2}{3}\lambda=1-\mu$ and $\lambda=\mu$. If we solve this simple system of equations, we get $\lambda=\mu=\frac{3}{5}$. Therefore $\overrightarrow{AS}=\frac{3}{5}\overrightarrow{AC}$, i.e. $AS:SC=3:2$.
 \kdokaz

%________________________________________________________________________________
\poglavje{Vector Length}  \label{odd5DolzVekt}

We have already defined the length of a vector in section \ref{odd5DefVekt}. From the definition of multiplication of a vector with a scalar in the previous section \ref{odd5LinKombVekt} we saw that for every vector $\overrightarrow{v}$ and every real number $\lambda$ it holds that $|\lambda \cdot \overrightarrow{v}|=|\lambda| \cdot |\overrightarrow{v}|$. In this section we will discuss some more properties of the length of a vector.

First, from the definition of addition of vectors and the triangle inequality we get the following claim.

            \bizrek \label{neenakTrikVekt}
            For any two vectors $\overrightarrow{v}$ and $\overrightarrow{u}$ it holds that
            $$|\overrightarrow{v}+\overrightarrow{u}|\leq |\overrightarrow{v}|+|\overrightarrow{u}|.$$
            \eizrek

\begin{figure}[!htb]
\centering
\input{sl.vek.5.1.7.pic}
\caption{} \label{sl.vek.5.1.7.pic}
\end{figure}

 \textbf{\textit{Proof.}}  (Figure \ref{sl.vek.5.1.7.pic})

  Let $A$ be an arbitrary point and let $B$ and $C$ be such points that $\overrightarrow{AB}=\overrightarrow{v}$ and $\overrightarrow{BC}=\overrightarrow{u}$ (claim \ref{vektAvObst1TockaB}). From the definition of addition of vectors and the length of a vector and the triangle inequality (claim \ref{neenaktrik}) we get:
  \begin{eqnarray*}
  |\overrightarrow{v}+\overrightarrow{u}|&=&|\overrightarrow{AB}+\overrightarrow{BC}|=\\
  &=&|\overrightarrow{AC}|=|AC|\leq
  |AB|+|AC|=
  |\overrightarrow{AB}|+|\overrightarrow{AC}|=\\
  &=&|\overrightarrow{v}|+|\overrightarrow{u}|,
   \end{eqnarray*}
 which was to be proven.  \kdokaz

In a similar way as we can define a fraction, we can also define the product of collinear vectors. If $\overrightarrow{v}$ and $\overrightarrow{u}$ are two collinear vectors, we define the operation \pojem{multiplication of two collinear vectors}\footnote{This is a special case of the so-called \textit{dot product}, which in linear algebra is defined for any two vectors. In Euclidean space, $\overrightarrow{v}\cdot\overrightarrow{u}=
|\overrightarrow{v}|\cdot|\overrightarrow{u}|\cdot\cos
\angle \overrightarrow{v},\overrightarrow{u}$.}:
 \begin{eqnarray*}
 \overrightarrow{v}\cdot \overrightarrow{u}=
\left\{
  \begin{array}{ll}
    |\overrightarrow{v}|\cdot|\overrightarrow{u}|, &
\overrightarrow{v},\overrightarrow{u}\rightrightarrows ; \\
    -|\overrightarrow{v}|\cdot|\overrightarrow{u}|, & \overrightarrow{v},\overrightarrow{u}\rightleftarrows.
  \end{array}
\right.
\end{eqnarray*}
 The result of this operation, which is a real number, is called  \index{product of collinear vectors} \pojem{product of collinear vectors}.
It is clear from the definition that:
     \begin{eqnarray} \label{eqnMnozVektDolzina}
     \overrightarrow{v}\cdot \overrightarrow{v}=|\overrightarrow{v}|^2
     \end{eqnarray}
 It is also clear that for three collinear points $A$, $B$ and $L$ the following equivalence holds (Figure \ref{sl.vek.5.3.1.pic}):

 \begin{eqnarray} \label{eqnMnozVektRelacijaB}
     \overrightarrow{LA}\cdot \overrightarrow{LB}<0
\hspace*{1mm}\Leftrightarrow\hspace*{1mm} \mathcal{B}(A,L,B)
\hspace*{3mm} \textrm{(}A,B,L \textrm{ are collinear)}.
     \end{eqnarray}


\begin{figure}[!htb]
\centering
\input{sl.vek.5.3.1.pic}
\caption{} \label{sl.vek.5.3.1.pic}
\end{figure}

\bnaloga\footnote{15. IMO USSR - 1973, Problem 1.}
Point $O$ lies on line $g$;
 $\overrightarrow{OP_1},\overrightarrow{OP_2},
 \cdots,\overrightarrow{OP_n}$
are unit vectors such that points $P_1, P_2, \cdots, P_n$
all lie in a plane containing $g$ and on one side of $g$. Prove that
if $n$ is odd,
 $$|\overrightarrow{OP_1}+\overrightarrow{OP_2}+
 \cdots+\overrightarrow{OP_n}|\geq 1.$$ \label{OlimpVekt15}
 \enaloga

\begin{figure}[!htb]
\centering
\input{sl.vek.5.3.IMO1.pic}
\caption{} \label{sl.vek.5.3.IMO1.pic}
\end{figure}

 \textbf{\textit{Solution.}} Let $k$ be a unit circle with center
 at point $O$, which intersects the line $g$ at points $A$ and $B$. All
 points $P_1, P_2, \cdots, P_n$ lie on the corresponding
 arc, which on the circle $k$ is determined by points $A$ and $B$.
 Without loss of generality
 we can assume that the vectors are labeled so that it holds:
 $\angle BOP_1\leq \angle BOP_2\leq\cdots\leq \angle BOP_n$
 (Figure \ref{sl.vek.5.3.IMO1.pic}).
 Let $s$ be the bisector of angle $\angle P_1OP_n$ and $g'$
 the perpendicular of the line $s$ at point $O$. Because $\angle P_1OP_n$ is a convex
 angle, the whole lies on the same side of the line $g'$. So are also all
 points $P_i$ ($i\in \{1,2,\cdots,n\}$ on the same side of the line $g'$,
 because they lie in this angle. So we can sharpen our assumption from the task, that all points $P_1, P_2, \cdots, P_n$ are on the same side of some line
 $g$ or $g'$,
 with the requirement (which follows from this assumption),
  that the first and last vector of the sequence
 are symmetric with respect to the perpendicular $s$  of the line $g'$ through point $O$. We will use this fact in the proof.

 By assumption $n$ is an odd number. Let $n=2k-1$, $k\in \mathbb{N}$.
 We will prove by induction on $k$.

 (\textit{i}) If $k=1$ or $n=1$, then $|\overrightarrow{OP_1}|=1$
 or $|\overrightarrow{OP_1}|\geq1$ and the statement
 is fulfilled.

(\textit{ii}) Let's assume that the statement is true for $k=l$ or for every sequence
 $n=2l-1$ of vectors. We will prove that the statement is also true for $k=l+1$ or in
 the case of a sequence $n=2l+1$ of vectors. Let
 $\overrightarrow{OP_1},\overrightarrow{OP_2},
 \cdots,\overrightarrow{OP_{2l+1}}$
  be unit vectors, where $P_1, P_2, \cdots, P_{2l+1}$
 are points that
 lie in the same plane with the edge  $g'$ (in the plane $\alpha$)
  and the vectors $\overrightarrow{OP_1}$
 and $\overrightarrow{OP_{2l+1}}$ are symmetrical with respect
  to the line $s$. From this it follows that the vector
  $\overrightarrow{OU}=\overrightarrow{OP_1}+\overrightarrow{OP_{2l+1}}$
  lies on the line $s$ or $U\in s$. Let
  $\overrightarrow{OV}=\overrightarrow{OP_2}+\overrightarrow{OP_3}+
         \cdots+\overrightarrow{OP_{2l}}$. By the induction assumption
  $|\overrightarrow{OV}|=|\overrightarrow{OP_2}+\overrightarrow{OP_3}+
         \cdots+\overrightarrow{OP_{2l}}|\geq 1$. Therefore:
 \begin{eqnarray*}
  |\overrightarrow{OP_1}+\overbrace{\overrightarrow{OP_2}+
         \cdots+\overrightarrow{OP_{2l}}}+\overrightarrow{OP_{2l+1}}|=
         |\overrightarrow{OU}+\overrightarrow{OV}|.
 \end{eqnarray*}
 Let
 $\overrightarrow{OW}=\overrightarrow{OU}+\overrightarrow{OV}$.  The quadrilateral $VOUW$ is a parallelogram (or the points $O$, $U$, $V$ and $W$
 are collinear).
 By the example \ref{vektOPi} the points $U$ and $V$ are in the plane $\alpha$. The point
 $U$ lies on the line $s$, which is the line of symmetry of the obtuse angle, which
 is determined by the line $g'$ and the point $O$. Therefore $0\leq\angle
 UOV\leq 90^0$. From the parallelogram $VOUW$ it then follows that $\angle OVW >
 90^0$. From the triangle inequality \ref{neenaktrik} (for the triangle
 $OVW$) it follows that $|\overrightarrow{OW}|>|\overrightarrow{OV}|$.
 The inequality is also valid in the case when the points $O$, $U$, $V$ and $W$
 are collinear. Now we have:
 \begin{eqnarray*}
  |\overrightarrow{OP_1}+\overrightarrow{OP_2}+
         \cdots+\overrightarrow{OP_{2l}}+\overrightarrow{OP_{2l+1}}|=
         |\overrightarrow{OU}+\overrightarrow{OV}|=
         |\overrightarrow{OW}|>|\overrightarrow{OV}|\geq 1,
 \end{eqnarray*}
 which had to be proven. \kdokaz

%________________________________________________________________________________
 \poglavje{Further Use of Vectors}  \label{odd5UporabVekt}

 First, let's look at a statement that
 is a direct consequence of the definition of vectors (section \ref{odd5DefVekt}).

              \bizrek \label{vektParalelogram}
              A quadrilateral $ABCD$ is a parallelogram if and only if
               $\overrightarrow{AB}=\overrightarrow{DC}$.
               \eizrek


 We should also mention that we can already
prove the property of the median of a triangle - statement \ref{srednjicaTrik} -
now in the following form.

                \bizrek
                \label{srednjicaTrikVekt}
                 Let $PQ$ be the midsegment of a triangle  $ABC$, corresponding
                to the side
                $BC$. Then
                $$ \overrightarrow{PQ} = \frac{1}{2} \overrightarrow{AB}.$$
                 \eizrek

\begin{figure}[!htb]
\centering
\input{sl.vek.5.4.1c.pic}
\caption{} \label{sl.vek.5.4.1c.pic}
\end{figure}

 \textbf{\textit{Proof.}}  (Figure \ref{sl.vek.5.4.1c.pic})
 The statement is a direct consequence of  statement \ref{srednjicaTrik}
 \kdokaz

We have seen that we can add vectors according to the parallelogram rule
(when they have a common starting point) or according to the triangle rule (when the
starting point of the second one is at the end of the first one). Now we will introduce another
rule for adding two vectors that are in any position. This is called the \index{pravilo!splošno za seštevanje vektorjev}\pojem{general vector addition rule}.

              \bizrek \label{vektSestSplosno}
              If $S_1$ and $S_2$
               are  the midpoints of line segments $A_1B_1$ and $A_2B_2$, then
              $$\overrightarrow{A_1A_2}+\overrightarrow{B_1B_2}=
              2\overrightarrow{S_1S_2}.$$
              \eizrek

\begin{figure}[!htb]
\centering
\input{sl.vek.5.4.1b.pic}
\caption{} \label{sl.vek.5.4.1b.pic}
\end{figure}

 \textbf{\textit{Proof.}}  (Figure \ref{sl.vek.5.4.1b.pic})

Since $S_1$ and $S_2$
    are the midpoints of the lines $A_1B_1$ and $A_2B_2$,
    $\overrightarrow{S_1A_1}=-\overrightarrow{S_1B_1}$ and $\overrightarrow{S_2A_2}=-\overrightarrow{S_2B_2}$ or $\overrightarrow{S_1A_1}+\overrightarrow{S_1B_1}=\overrightarrow{0}$ and $\overrightarrow{S_2A_2}+\overrightarrow{S_2B_2}=\overrightarrow{0}$.

    If we decompose the vector $\overrightarrow{S_1S_2}$ in two ways using the polygonal rule for adding vectors, we first get:
    \begin{eqnarray*}
     \overrightarrow{S_1S_2}&=&\overrightarrow{S_1A_1}+\overrightarrow{A_1A_2}+
     \overrightarrow{A_2S_2}\\
     \overrightarrow{S_1S_2}&=&\overrightarrow{S_1B_1}+\overrightarrow{B_1B_2}+
     \overrightarrow{B_2S_2},
    \end{eqnarray*}
    then by adding these:
  \begin{eqnarray*}
  2\cdot\overrightarrow{S_1S_2}&=&\overrightarrow{S_1A_1}+\overrightarrow{A_1A_2}+
     \overrightarrow{A_2S_2}+\\
     &+&\overrightarrow{S_1B_1}+\overrightarrow{B_1B_2}+
     \overrightarrow{B_2S_2}=\\
     &=& \overrightarrow{A_1A_2}+\overrightarrow{B_1B_2},
    \end{eqnarray*}
  which had to be proven. \kdokaz

We will often use the relation from the previous statement in
the form:
 $$\overrightarrow{S_1S_2}=
 \frac{1}{2}(\overrightarrow{A_1A_2}+\overrightarrow{B_1B_2}),$$
    which is a generalization of the property of the median of a trapezoid and a triangle.



    At this point, we will repeat all three rules for adding two vectors. In the first case, we therefore choose representatives of the vectors so that the beginning of the second is at the end of the first, in the second case the vectors have a common starting point, in the third case the representatives are in general position (Figure \ref{sl.vek.5.4.1a.pic}):

\begin{itemize}
\item \textit{for every three points $A$, $B$ and $C$ it holds that $\overrightarrow{AB}+\overrightarrow{BC}=\overrightarrow{AC}$ (the triangle rule),}
\item \textit{for every three non-collinear points $A$, $B$ and $C$ it holds that $\overrightarrow{AB}+\overrightarrow{AC}=\overrightarrow{AD}$ exactly when $ABDC$ is a parallelogram (the parallelogram rule),}
\item \textit{if $S_1$ and $S_2$ are the midpoints of the line segments $A_1B_1$ and $A_2B_2$, then
$\overrightarrow{A_1A_2}+\overrightarrow{B_1B_2}=
2\overrightarrow{S_1S_2}$ (the general rule).}
\end{itemize}

\begin{figure}[!htb]
\centering
\input{sl.vek.5.4.1a.pic}
\caption{} \label{sl.vek.5.4.1a.pic}
\end{figure}

Except for these three, we also have the polygon rule, which is related to the addition of multiple vectors (Figure \ref{sl.vek.5.1.12.pic}):
\begin{itemize}
\item \textit{$\overrightarrow{A_1A_2}+\overrightarrow{A_2A_3}+\cdots +\overrightarrow{A_{n-1}A_n}=\overrightarrow{A_1A_n}$ (the polygon rule).}
\end{itemize}

\begin{figure}[!htb]
\centering
\input{sl.vek.5.1.12.pic}
\caption{} \label{sl.vek.5.1.12.pic}
\end{figure}

\bzgled \label{vektPetkoinikZgled}
Points $M$, $N$, $P$ and $Q$ are the midpoints of the sides $AB$, $BC$, $CD$ and $DE$ of a pentagon $ABCDE$. Prove
that the line segment $XY$,  determined by the midpoints of the line segments $MP$ and $NQ$, is parallel to the line $AE$
and calculate $\frac{\overrightarrow{XY}}{\overrightarrow{AE}}$.
\ezgled

\begin{figure}[!htb]
\centering
\input{sl.vek.5.4.2.pic}
\caption{} \label{sl.vek.5.4.2.pic}
\end{figure}

\textbf{\textit{Solution.}} (Figure \ref{sl.vek.5.4.2.pic})
 If we use the expressions \ref{vektSestSplosno} and \ref{srednjicaTrikVekt} we get:
 \begin{eqnarray*}
 \overrightarrow{XY}&=&\frac{1}{2}\left(\overrightarrow{MQ}+\overrightarrow{PN} \right)=\\
 &=&\frac{1}{2}\left(\frac{1}{2}\left(\overrightarrow{AE}+\overrightarrow{BD}\right)
 +\frac{1}{2}\overrightarrow{DB} \right)=\\
 &=&\frac{1}{4}\overrightarrow{AE}
 \end{eqnarray*}
 So the vectors $\overrightarrow{XY}$ and $\overrightarrow{AE}$  are collinear and $\frac{\overrightarrow{XY}}{\overrightarrow{AE}}=\frac{1}{4}$.
 \kdokaz


                    \bzgled
                    Let $O$ be an arbitrary point in the plane of a triangle $ABC$ and
                    $D$ and $E$ points of the sides $AB$ and $BC$ such that
                     $$\frac{\overrightarrow{AD}}{\overrightarrow{DB}}=
                    \frac{\overrightarrow{BE}}{\overrightarrow{EC}}=\frac{m}{n}.$$
                    Let $F$ be the intersection of the line segments $AE$ and $CD$. Express $\overrightarrow{OF}$ as
                    a function of  $\overrightarrow{OA}$, $\overrightarrow{OB}$, $\overrightarrow{OC}$, $m$ and $n$.
                    \ezgled

\begin{figure}[!htb]
\centering
\input{sl.vek.5.4.3.pic}
\caption{} \label{sl.vek.5.4.3.pic}
\end{figure}

 \textbf{\textit{Solution.}} (Figure \ref{sl.vek.5.4.3.pic})

It is enough to express the vector $\overrightarrow{BF}$ as a linear combination of the vectors $\overrightarrow{BA}$ and $\overrightarrow{BC}$, because:
\begin{eqnarray*}
 \overrightarrow{OF}&=&\overrightarrow{OB}+\overrightarrow{BF}\\
 \overrightarrow{BA}&=&\overrightarrow{OA}-\overrightarrow{OB}\\
 \overrightarrow{BC}&=&\overrightarrow{OC}-\overrightarrow{OB}.
 \end{eqnarray*}
By \ref{vektParamPremica} we have:
\begin{eqnarray*}
 \overrightarrow{BF}&=&\lambda\overrightarrow{BD}+(1-\lambda)\overrightarrow{BC}=
 \lambda\frac{n}{n+m}\overrightarrow{BA}+(1-\lambda)\overrightarrow{BC};\\
 \overrightarrow{BF}&=&\mu\overrightarrow{BA}+(1-\mu)\overrightarrow{BE}=
 \mu\overrightarrow{BA}+(1-\mu)\frac{m}{n+m}\overrightarrow{BC}
 \end{eqnarray*}
for some numbers $\lambda,\mu\in \mathbb{R}$. Because the vectors $\overrightarrow{BA}$ and $\overrightarrow{BC}$ are nonlinear, by \ref{vektLinKomb1Razcep} we obtain the system:
\begin{eqnarray*}
 & & \lambda\frac{n}{n+m}=\mu\\
 & & 1-\lambda=(1-\mu)\frac{m}{n+m},
 \end{eqnarray*}
 which we solve for $\lambda$ and $\mu$ in terms of $m$ and $n$. It is enough to calculate only $\lambda$ and plug it in  $\overrightarrow{BF}=
 \lambda\frac{n}{n+m}\overrightarrow{BA}+(1-\lambda)\overrightarrow{BC}$.
 \kdokaz


                    \bzgled
                    Let $A_1$, $A_2$, ..., $A_n$ be points on a line $p$ and $B_1$, $B_2$, ..., $B_n$  points on a line $q$ ($n\geq 3$), such that
                    $$\overrightarrow{A_1A_2}:
                    \overrightarrow{A_2A_3}:\cdots:
                    \overrightarrow{A_{n-1}A_n}=
                    \overrightarrow{B_1B_2}:
                    \overrightarrow{B_2B_3}:\cdots:
                    \overrightarrow{B_{n-1}B_n}.$$
                    Prove that the midpoints of line segments
                    $A_1B_1$, $A_2B_2$, ..., $A_nB_n$ lie on one line.
                    \ezgled

\begin{figure}[!htb]
\centering
\input{sl.vek.5.4.4.pic}
\caption{} \label{sl.vek.5.4.4.pic}
\end{figure}


 \textbf{\textit{Proof.}} (Figure \ref{sl.vek.5.4.4.pic})

 It is enough to prove that three arbitrary centers are collinear points.
Without loss of generality, we prove only that points $S_1$, $S_2$, $S_3$ are collinear. Let:
$$\frac{\overrightarrow{A_1A_2}}{\overrightarrow{A_2A_3}}=
\frac{\overrightarrow{B_1B_2}}{\overrightarrow{B_2B_3}}=\lambda.$$
 By  \ref{vektSestSplosno} is:
  \begin{eqnarray*}
 \overrightarrow{S_1S_2}&=&\frac{1}{2}\left(\overrightarrow{A_1A_2}+
 \overrightarrow{B_1B_2}\right)=\\
 &=&\frac{1}{2}\left(\lambda\overrightarrow{A_2A_3}+
 \lambda\overrightarrow{B_2B_3}\right)=\\
 &=&\frac{\lambda}{2}\overrightarrow{S_2S_3},
 \end{eqnarray*}
 which means that $\overrightarrow{S_1S_2}$ and $\overrightarrow{S_2S_3}$ are collinear vectors (\ref{vektKriterijKolin}), so $S_1$, $S_2$ and $S_3$ are collinear points.
 \kdokaz



%________________________________________________________________________________
 \poglavje{Centroid of a Polygon With Respect to Its Vertices}  \label{odd5TezVeck}

Now we will generalize the concept of the centroid of a triangle to an arbitrary polygon. In section \ref{odd3ZnamTock} we defined the concept of the centroid of a triangle. The next theorem relates to one additional property of this concept, which is related to the concept of a vector.


            \bizrek \label{tezTrikVekt}
            If $T$ is the centroid of a triangle $ABC$, then
            $$\overrightarrow{TA}+\overrightarrow{TB}+
            \overrightarrow{TC}=\overrightarrow{0}.$$
            \eizrek


\begin{figure}[!htb]
\centering
\input{sl.vek.5.5.1.pic}
\caption{} \label{sl.vek.5.5.1.pic}
\end{figure}

\textbf{\textit{Proof.}} We mark with $A_1$ the center of the side $BC$ (Figure \ref{sl.vek.5.5.1.pic}). By izreku \ref{vektSredOSOAOB} is $\overrightarrow{TA_1}=
 \frac{1}{2}\left(\overrightarrow{TB}+\overrightarrow{TC}\right)$, by izreku \ref{tezisce} for the centroid $T$ of the triangle $ABC$ is valid  $|AT|:|TA_1|=2:1$ or $\overrightarrow{TA}=-2\cdot\overrightarrow{TA_1}$. So:
$$\overrightarrow{TA}+\overrightarrow{TB}+
            \overrightarrow{TC}=\overrightarrow{TA}+2\cdot\overrightarrow{TA_1}=
            \overrightarrow{TA}-\overrightarrow{TA}=
            \overrightarrow{0},$$ which had to be proven. \kdokaz

It is also valid the converse statement:



            \bizrek \label{tezTrikVektObr}
            Let $A$, $B$ and $C$ be three non-collinear points. If $X$ is a point such that
            $$\overrightarrow{XA}+\overrightarrow{XB}+
            \overrightarrow{XC}=\overrightarrow{0},$$
            then
            $X$ is the centroid of the triangle $ABC$.
            \eizrek

\textbf{\textit{Proof.}} Let $T$ be the centroid of the triangle. By
the previous izreku \ref{tezTrikVekt} is
$\overrightarrow{TA}+\overrightarrow{TB}+
\overrightarrow{TC}=\overrightarrow{0}$. By the assumption
is also $\overrightarrow{XA}+\overrightarrow{XB}+
\overrightarrow{XC}=\overrightarrow{0}$. If we subtract
the two equalities, we get
$3 \cdot \overrightarrow{TX}=\overrightarrow{0}$ or $X=T$.
 \kdokaz

The proven property of the centroid of a triangle gives us an idea for the definition of the centroid of an arbitrary polygon (Figure \ref{sl.vek.5.5.2.pic}).

\begin{figure}[!htb]
\centering
\input{sl.vek.5.5.2.pic}
\caption{} \label{sl.vek.5.5.2.pic}
\end{figure}

A point $T$ is the \index{center of mass!polygon}\pojem{center of mass of a polygon $A_1A_2\ldots A_n$ with respect to its vertices}, if the following is true:
$$\overrightarrow{TA_1}+\overrightarrow{TA_2}+\cdots +\overrightarrow{TA_n}=\overrightarrow{0}.$$
The previous relation can also be written as:
$$\sum_{k=1}^n\overrightarrow{TA_k}=\overrightarrow{0}.$$

We also mention that the center of mass of any figure $\Phi$ in a plane is defined as a point $T$, for which the following is true:
$$\sum_{X\in \Phi}\overrightarrow{TX}=\overrightarrow{0}.$$

In the case of our polygon $A_1A_2\ldots A_n$, we actually found the center of mass of a figure that represents the union of all the vertices of this polygon $\{A_1,A_2,\ldots, A_n\}$. That is why we emphasized that this is the center of mass of a polygon with respect to its vertices. In addition, we could talk about the \pojem{center of mass of a polygon with respect to all its points}, which would more accurately match the general definition of the center of mass of any figure.

If we consider the center of mass of a figure in a physical sense as - the center of mass - the first variant of the center of mass represents the center of mass, where all the mass is in the vertices and each vertex has the same mass. In the second case, this is the center of mass of the polygon, where the mass is evenly distributed throughout its interior\footnote{\index{Archimedes}
        \textit{Archimedes
       of Syracuse} (3rd century BC) was a Greek mathematician
      who first created the concept of the center of mass,
which he used in many of his writings on mechanics, but
we can only guess what he had in mind when
he considered the center of mass, because none of his surviving writings contain
a clear definition of the concept. The center of mass as the center of mass in a physical sense played an important role in Newton's (\index{Newton,
I.}\textit{I. Newton} (1643-1727), English physicist and mathematician) mechanics, where larger bodies are often considered as points with a certain mass.}.

If we talk about a general polygon, the aforementioned centroids are only equal in any triangle, already in any quadrilateral the centroids differ.
In the following we will only consider the centroid of the polygon with respect to its vertices, so we will just call it the \pojem{centroid of the polygon}.

First we will consider the centroid of a quadrilateral.

                \bizrek
                The centroid of a parallelogram $ABCD$ is its circumcentre $S$, i.e.
                the intersection of its diagonals.
                \eizrek

\begin{figure}[!htb]
\centering
\input{sl.vek.5.5.3.pic}
\caption{} \label{sl.vek.5.5.3.pic}
\end{figure}

 \textbf{\textit{Proof.}} (Figure \ref{sl.vek.5.5.3.pic})

 By izreku \ref{paralelogram} is the point $S$ the common centre of its diagonals $AC$ and $BD$, so from izreku \ref{vektSredDalj} it follows that $\overrightarrow{SA}+\overrightarrow{SC}=\overrightarrow{0}$ and $\overrightarrow{SB}+\overrightarrow{SD}=\overrightarrow{0}$.
 Therefore it holds:
 $$\overrightarrow{SA}+\overrightarrow{SB}+
 \overrightarrow{SC}+\overrightarrow{SD}=\overrightarrow{0},$$
which means that $S$ is the centroid of the parallelogram $ABCD$.
\kdokaz


                \bizrek \label{vektVarignon}
                The centroid of a quadrilateral $ABCD$ is the centroid  of its Varignon
                parallelogram (see theorem \ref{Varignon}).
                \eizrek

\begin{figure}[!htb]
\centering
\input{sl.vek.5.5.1a.pic}
\caption{} \label{sl.vek.5.5.1a.pic}
\end{figure}

 \textbf{\textit{Proof.}}  (Figure \ref{sl.vek.5.5.1a.pic})

Let $P$, $K$, $Q$ and $L$ be the midpoints of sides $AB$, $BC$, $CD$ and $DA$, or $PKQL$ the Varignon parallelogram of the quadrilateral $ABCD$ (the quadrilateral $PKQL$ is a parallelogram by the statement \ref{Varignon}). According to the previous statement \ref{vektVarignon}, the intersection of diagonals $PQ$ and $LK$ (denoted by $T$) is at the same time the centroid of the parallelogram $PKQL$. Then it is (statements \ref{vektSredOSOAOB}) and \ref{paralelogram}):
 \begin{eqnarray*}
 \overrightarrow{TA}+\overrightarrow{TB}+
 \overrightarrow{TC}+\overrightarrow{TD}&=&
 2\cdot\overrightarrow{TP}+
 2\cdot\overrightarrow{TQ}=\\
 &=&
 2\cdot\left(\overrightarrow{TP}+
\overrightarrow{TQ}\right)=\\
 &=&\overrightarrow{0},
 \end{eqnarray*}
which means that $T$ is the centroid of the quadrilateral $ABCD$.
\kdokaz



                \bizrek \label{vektTezVeckXT}
                If $X$  is an arbitrary point and $T$ the centroid  of a polygon $A_1A_2\ldots A_n$, then
                $$\overrightarrow{XT}=\frac{1}{n}\left(\overrightarrow{XA_1}
                +\overrightarrow{XA_2}+\cdots +\overrightarrow{XA_n}\right).$$
                \eizrek


\begin{figure}[!htb]
\centering
\input{sl.vek.5.5.4.pic}
\caption{} \label{sl.vek.5.5.4.pic}
\end{figure}

 \textbf{\textit{Proof.}}  (Figure \ref{sl.vek.5.5.4.pic})

First, $\overrightarrow{XT}=\overrightarrow{XA_k}+\overrightarrow{A_kT}$ (for every $k\in\{1,2,\ldots,n\}$). If we add all $n$ relations, we get:

\begin{eqnarray*}
 n\cdot\overrightarrow{XT}
 &=&\sum_{k=1}^{n}\left(\overrightarrow{XA_k}+\overrightarrow{A_kT}\right)=\\
 &=&\sum_{k=1}^{n}\overrightarrow{XA_k}+\sum_{k=1}^{n}\overrightarrow{A_kT}=\\
 &=&
 \sum_{k=1}^{n}\overrightarrow{XA_k}-\sum_{k=1}^{n}\overrightarrow{TA_k}=\\
 &=&
 \sum_{k=1}^{n}\overrightarrow{XA_k}-\overrightarrow{0}=\\
 &=&\sum_{k=1}^{n}\overrightarrow{XA_k}.
 \end{eqnarray*}

Therefore:
$$\overrightarrow{XT}=\frac{1}{n}\cdot\sum_{k=1}^{n}\overrightarrow{XA_k},$$ which was to be proven. \kdokaz

As a result of the previous equation, a particularly useful relation holds for the centroid of an arbitrary triangle.



                \bizrek \label{vektTezTrikXT}
                 If $X$  is an arbitrary point and $T$ the centroid  of a triangle $ABC$, then
                $$\overrightarrow{XT}=\frac{1}{3}\left(\overrightarrow{XA}
                +\overrightarrow{XB}+\overrightarrow{XC}\right).$$
                \eizrek


\begin{figure}[!htb]
\centering
\input{sl.vek.5.5.5.pic}
\caption{} \label{sl.vek.5.5.5.pic}
\end{figure}

 \textbf{\textit{Proof.}}  (Figure \ref{sl.vek.5.5.5.pic})

A direct consequence of the previous equation for $n=3$
\kdokaz



% The process for an n-gon (n-1)-gon ...

 So far we have had an effective process for determining the centroids of triangles and quadrilaterals. For an arbitrary quadrilateral, we actually do not know whether the centroid exists at all. We will address this issue in the following.

 We will look for the idea in the following facts. If we consider the distance $AB$ as a degenerate $2$-gon, its centroid is the center of the distance $AB$; we denote it with $T_2$. For the centroid $T_3$ of the triangle $ABC$, it then holds that
 $\overrightarrow{CT_3}=\frac{2}{3}\cdot \overrightarrow{CT_2}$ (equation \ref{tezisce}). We will generalize this idea in the following equation.


                \bizrek \label{vektTezVeck}
                Every polygon $A_1A_2\ldots A_n$ has exactly one centroid. If
                $T_{n-1}$ is the centroid of the polygon $A_1A_2\ldots A_{n-1}$ and $T_n$ a point such that $$\overrightarrow{A_nT_n}=\frac{n-1}{n}\cdot\overrightarrow{A_nT_{n-1}},$$
                then $T_n$ is the centroid of the polygon $A_1A_2\ldots A_n$.
                \eizrek



\begin{figure}[!htb]
\centering
\input{sl.vek.5.5.2a.pic}
\caption{} \label{sl.vek.5.5.2a.pic}
\end{figure}

 \textbf{\textit{Proof.}}  (Figure \ref{sl.vek.5.5.2a.pic})

First, from the relation $\overrightarrow{A_nT_n}=\frac{n-1}{n}\cdot\overrightarrow{A_nT_{n-1}}$ it follows that
 $\overrightarrow{T_nA_n}=-\frac{n-1}{n}\cdot\overrightarrow{A_nT_{n-1}}$ and $\overrightarrow{T_nT_{n-1}}=\frac{1}{n}\cdot\overrightarrow{A_nT_{n-1}}$.
  Because $T_{n-1}$ is the centroid of the polygon $A_1A_2\ldots A_{n-1}$, by \ref{vektTezVeckXT} it holds that $\overrightarrow{T_nA_1}+\overrightarrow{T_nA_2}+\cdots +\overrightarrow{T_nA_{n-1}}=\left(n-1\right)\cdot \overrightarrow{T_nT_{n-1}}$.

 So:
 \begin{eqnarray*}
 & & \overrightarrow{T_nA_1}+\overrightarrow{T_nA_2}+\cdots
 +\overrightarrow{T_nA_{n-1}}+\overrightarrow{T_nA_n}=\\
 &=&\left(n-1\right)\cdot \overrightarrow{T_nT_{n-1}} +\overrightarrow{T_nA_n}=\\
 &=&\frac{n-1}{n}\cdot\overrightarrow{A_nT_{n-1}} -\frac{n-1}{n}\cdot\overrightarrow{A_nT_{n-1}}=\\
 &=&\overrightarrow{0},
 \end{eqnarray*}

 which means that $T_n$ is the centroid of the polygon $A_1A_2\ldots A_n$.

 Assume that the polygon $A_1A_2\ldots A_n$ has another centroid $T'$. But in this case it holds (\ref{vektTezVeckXT}):
 \begin{eqnarray*}
 \overrightarrow{0}=
 \overrightarrow{T'A_1}+\overrightarrow{T'A_2}+\cdots
 \overrightarrow{T'A_n}=
 n\cdot\overrightarrow{T'T_n}.
 \end{eqnarray*}
 So $\overrightarrow{T'T_n}=\overrightarrow{0}$ or $T'=T_n$, which means that the polygon $A_1A_2\ldots A_n$ has only one centroid.
 \kdokaz

The previous statement \ref{vektTezVeck} allows us to effectively construct the centroid of a polygon $A_1A_2\ldots A_n$, so that we first construct the centroids of polygons $A_1A_2$, $A_1A_2A_3$,$A_1A_2A_3A_4$, ... and finally $A_1A_2\ldots A_n$ in order (Figure \ref{sl.vek.5.5.2a.pic}):
 \begin{itemize}
   \item point $T_2$ is the center of the line segment $A_1A_2$,
   \item $T_3$ is such a point that it holds: $\overrightarrow{A_3T_3}=\frac{2}{3}\cdot \overrightarrow{A_3T_2}$,
   \item $T_4$ is such a point that it holds: $\overrightarrow{A_4T_4}=\frac{3}{4}\cdot \overrightarrow{A_4T_3}$,\\
    $\vdots$
   \item $T_n$ is such a point that it holds $\overrightarrow{A_nT_n}=\frac{n-1}{n}\cdot \overrightarrow{A_nT_{n-1}}$.
 \end{itemize}


 We get an even easier procedure for determining the centroid of a polygon if we use the relation
 $$\overrightarrow{XT}=\frac{1}{n}\left(\overrightarrow{XA_1}
                +\overrightarrow{XA_2}+\cdots +\overrightarrow{XA_n}\right)$$
     from statement \ref{vektTezVeckXT}. So for any point $X$ we simply plot the vector $\overrightarrow{XT}$ and get the point $T$.

     We emphasize that in both cases of the centroid construction of a polygon we need the process of planning a point that divides a given line segment in a certain ratio. We will discuss this process in section \ref{odd5TalesVekt} (see \ref{izrekEnaDelitevDaljice} and \ref{izrekEnaDelitevDaljiceNan}).



 In the proof of statement \ref{vektTezVeck} we did not use
the fact that the points $A_1$, $A_2$, ..., $A_n$ are in the same plane.
The statement is also true in the case where $ABCD$ ($n=4$) is a so-called \pojem{tetrahedron}
\index{tetrahedron}. The aforementioned point is then called
\index{težišče!tetraedra} \pojem{the centroid of the tetrahedron}. It is
possible to
prove an analogous statement for a tetrahedron; the line segments determined by the vertices of the tetrahedron and the centroids of the opposite faces pass through the centroid of this tetrahedron, which it divides in the ratio $3 :1$.

In the general case, if $n \in \mathbb{N}$, we can talk about the so-called \index{simplex}\pojem{simplex} (generalization: point, line, triangle, tetrahedron, ...), which lies in the $(n-1)$-dimensional Euclidean space.

                    \bzgled
                    Let $A$, $B$ and $C$ be the centroids of a triangles $OMN$,
                    $ONP$ and $OPM$, then the centroid  $T$ of the triangle $MNP$,
                    the centroid $T_1$ of the triangle $ABC$ and the point $O$ are
                    three collinear points. Furthermore, it is $OT_1:T_1T=2:1$.
                    \ezgled

\begin{figure}[!htb]
\centering
\input{sl.vek.5.5.6.pic}
\caption{} \label{sl.vek.5.5.6.pic}
\end{figure}

 \textbf{\textit{Proof.}} We mark with $P_1$, $M_1$ and $N_1$ the centroids of the sides $MN$, $NP$ and $PM$ of the triangle $PMN$ (Figure \ref{sl.vek.5.5.6.pic}).
  If we use the formulas \ref{vektTezTrikXT}, \ref{tezisce} and \ref{vektSredOSOAOB}, we get:

  \begin{eqnarray*}
  \overrightarrow{OT_1}&=&\frac{1}{3}\left(
  \overrightarrow{OA}+\overrightarrow{OB}+\overrightarrow{OC} \right)=\\
  &=&\frac{1}{3}\left(
  \frac{2}{3}\overrightarrow{OP_1}+\frac{2}{3}\overrightarrow{OM_1}+
  \frac{2}{3}\overrightarrow{ON_1} \right)=\\
  &=&\frac{1}{3}\left(
  \frac{1}{3}\left(\overrightarrow{OM}+\overrightarrow{ON}\right)
  +\frac{1}{3}\left(\overrightarrow{ON}+\overrightarrow{OP}\right)+
  \frac{1}{3}\left(\overrightarrow{OP}+\overrightarrow{OM}\right) \right)=\\
  &=&\frac{2}{9}\left(
  \overrightarrow{OM}+\overrightarrow{ON}+
  \overrightarrow{OP} \right)=\\
  &=&\frac{2}{3}\overrightarrow{OT}
  \end{eqnarray*}

From $\overrightarrow{OT_1}=\frac{2}{3}\overrightarrow{OT}$ it follows that $\overrightarrow{OT_1}=2\overrightarrow{T_1T}$, which means that the vectors $\overrightarrow{OT_1}$ and $\overrightarrow{T_1T}$ are collinear (also the points $O$, $T_1$ and $T$) and $OT_1:T_1T=2:1$.
\kdokaz

\bzgled
                    The centroid of a regular $n$-gon $A_1A_2...A_n$ is its centre (i.e. incentre and circumcentre).
                    \ezgled

\begin{figure}[!htb]
\centering
\input{sl.vek.5.5.7.pic}
\caption{} \label{sl.vek.5.5.7.pic}
\end{figure}

 \textbf{\textit{Proof.}} Let $S$ be the centroid of a regular $n$-gon $A_1A_2...A_n$ (Figure \ref{sl.vek.5.5.7.pic}).
It is enough to prove that:
$$\overrightarrow{SA_1}+\overrightarrow{SA_2}+
\cdots+\overrightarrow{SA_n}=\overrightarrow{0}.$$
Although in the case when $n$ is an even number, the statement is trivial, we will prove it for a general value of $n$ (even and odd).
Assume that $\overrightarrow{SA_1}+\overrightarrow{SA_2}+\cdots+\overrightarrow{SA_n}=\overrightarrow{SX}$, where $X\neq S$.
But if we rotate the polygon around the centroid $S$ by an angle $\theta=\frac{360}{n}$ (see the section on rotation \ref{odd6Rotac}), the sum of vectors on the left side of the equality does not change, the result on the right side becomes the vector $\overrightarrow{SX'}$, where $X'$ is the point we get from $X$ with the same rotation. Because the right side of the equality must remain unchanged, we get $\overrightarrow{SX'}=\overrightarrow{SX}$ or $X'=X$. This is possible only in the case when $X=S$ or $\overrightarrow{SX}=\overrightarrow{0}$.
\kdokaz






%________________________________________________________________________________
 \poglavje{Hamilton's Theorem}  \label{odd5Hamilton}

Sedaj bomo nadaljevali z lastnostmi, ki se nanašajo na značilne točke trikotnika (razdelek \ref{odd3ZnamTock}).

        \bizrek \label{HamiltonLema}
        If $O$,  $V$ and $A_1$ are the circumcentre, the orthocentre and the midpoint
        of the side $BC$ of a triangle $ABC$, respectively, then
        $$\overrightarrow{AV}=2\cdot \overrightarrow{OA_1}.$$
        \eizrek


 \textbf{\textit{Proof.}}
Let $B_1$ be the midpoint of the side $AC$  (Figure \ref{sl.vek.5.6.2.pic}).

The vectors $\overrightarrow{OA_1}$ and $\overrightarrow{AV}$ are collinear, so $\overrightarrow{OA_1}=\alpha \cdot \overrightarrow{AV}$ for some $\alpha \in \mathbb{R}$ (statement \ref{vektKriterijKolin}). Similarly, for the same reasons $\overrightarrow{OB_1}=\beta \cdot \overrightarrow{BV}$ (or $\overrightarrow{B_1O}=\beta \cdot \overrightarrow{VB}$) for some $\beta \in \mathbb{R}$. By statement \ref{srednjicaTrikVekt} (the median of a triangle) we have:
 $$\overrightarrow{B_1A_1}=\frac{1}{2}\overrightarrow{AB}=
 \frac{1}{2}\left(\overrightarrow{AV}+\overrightarrow{VB}\right)=
 \frac{1}{2}\overrightarrow{AV}+\frac{1}{2}\overrightarrow{VB}.$$
 At the same time we also have:
 $$\overrightarrow{B_1A_1}=\overrightarrow{B_1O}+\overrightarrow{OA_1}=
\beta\overrightarrow{VB}+\alpha\overrightarrow{AV}=
 \alpha\overrightarrow{AV}+\beta\overrightarrow{VB}.$$

Since the vectors $\overrightarrow{AV}$ and $\overrightarrow{VB}$ are non-collinear, from statement \ref{vektLinKomb1Razcep} it follows that $\alpha=\frac{1}{2}$ and $\beta=\frac{1}{2}$. Therefore $\overrightarrow{OA_1}=\alpha \overrightarrow{AV}=\frac{1}{2}\overrightarrow{AV}$ or $\overrightarrow{AV}=2\cdot \overrightarrow{OA_1}$.
\kdokaz

\begin{figure}[!htb]
\centering
\input{sl.vek.5.6.2.pic}
\caption{} \label{sl.vek.5.6.2.pic}
\end{figure}

 The following statement is very useful.

             \bizrek \label{Hamilton}\index{izrek!Hamiltonov}
             (Hamilton's\footnote{\index{Hamilton, W. R.}\textit{W. R. Hamilton} (1805--1865), angleški matematik.} theorem)
              If $O$ and $V$ are circumcentre and orthocentre of a triangle $ABC$, respectively then
             $$\overrightarrow{OA}+\overrightarrow{OB}
             +\overrightarrow{OC}=\overrightarrow{OV}.$$

             \eizrek



 \textbf{\textit{Proof.}} We mark with $A_1$ the centre of the side $BC$  (Figure \ref{sl.vek.5.6.2.pic}). If we use statement \ref{vektSredOSOAOB} and \ref{HamiltonLema}, we get:

$$\overrightarrow{OA}+\overrightarrow{OB}
        +\overrightarrow{OC}
        =\overrightarrow{OA}+2\cdot \overrightarrow{OA_1}=
        \overrightarrow{OA}+\overrightarrow{AV}=
        \overrightarrow{OV},$$ which had to be proven. \kdokaz

 We will continue with the consequences of the previous two equations.


             \bzgled \label{HamiltonPoslTetiv}
            A quadrilateral $ABCD$ is inscribed in a circle with a centre $O$.
            The diagonals $AC$ and $BD$ are perpendicular.
            If $M$ is the foot of the perpendicular from the centre $O$  on the  line $CD$, then
             $$|OM|=\frac{1}{2}|AB|.$$
               \ezgled

\begin{figure}[!htb]
\centering
\input{sl.vek.5.6.3a.pic}
\caption{} \label{sl.vek.5.6.3a.pic}
\end{figure}

 \textbf{\textit{Proof.}} Let $V$ be the altitude point of the triangle $BCD$
 (Figure \ref{sl.vek.5.6.3a.pic}).
 Because $AC\perp BD$, the point $V$ lies on the diagonal $AC$. By 
 \ref{HamiltonLema}, $\overrightarrow{OM}=\frac{1}{2} \overrightarrow{BV}$,
 so $|OM|=\frac{1}{2}|BV|$. Because there is also (\ref{ObodObodKot} and
 \ref{KotaPravokKraki}):
  $$\angle BAV=\angle BAC\cong\angle BDC\cong\angle AVB,$$
it follows that $BV\cong AB$ (\ref{enakokraki}) or
$|OM|=\frac{1}{2}|AB|$.
 \kdokaz


        \bzgled Let $V$ be the orthocentre and $O$ the circumcentre of a triangle $ABC$ and $AV\cong AO$.
            Prove that $\angle BAC=60^0$.
        \ezgled

\begin{figure}[!htb]
\centering
\input{sl.vek.5.6.1a.pic}
\caption{} \label{sl.vek.5.6.1a.pic}
\end{figure}

\textbf{\textit{Proof.}} Let $A_1$ be the center of the side $BC$ of the triangle $ABC$ (Figure \ref{sl.vek.5.6.1a.pic}). By the statement \ref{HamiltonLema} it follows that
$|AV|=2\cdot|OA_1|$. Since $OA\cong OC$, it follows that in the right triangle $OA_1C$ it holds that $|OC|=2\cdot|OA_1|$. With $O'$ we mark the point which is symmetrical to the point $O$ with respect to the point $A_1$. From $\triangle OA_1C\cong \triangle O'A_1C$ (the \textit{SAS} statement \ref{SKS}) it follows that $OC\cong O'C\cong OO'$, which means that the $\triangle OO'C$ is an isosceles triangle or $\angle A_1OC=60^0$. From the statement \ref{SredObodKot} and the congruence of the triangles $BOA_1$ and $COA_1$ (the \textit{SSS} statement \ref{SSS}) at the end it follows that
$\angle BAC=\frac{1}{2}\angle BOC=\angle A_1OC=60^0$.
  \kdokaz

        \bzgled \label{TetivniVisinska}
        Let $ABCD$ be a cyclic quadrilateral and:
         $V_A$ the orthocentre of the triangle $BCD$,
          $V_B$  the orthocentre of the triangle $ACD$,
           $V_C$  the orthocentre of the triangle $ABD$ and
           $V_D$  the orthocentre of the triangle  $ABC$.
           Prove that:\\
         a) the line segments $AV_A$,
         $BV_B$, $CV_C$ and $DV_D$
         has a common midpoint,\\
         b) the quadrilateral $V_AV_BV_CV_D$ is congruent
          to the quadrilateral $ABCD$.
        \ezgled


\begin{figure}[htp]
\centering
\input{sl.vek.5.6.3b.pic}
\caption{} \label{sl.vek.5.6.3b.pic}
\end{figure}


\begin{figure}[htp]
\centering
\input{sl.vek.5.6.3.pic}
\caption{} \label{sl.vek.5.6.3.pic}
\end{figure}


 \textbf{\textit{Solution.}} Let $O$ be the center of the circumscribed circle of the cyclic quadrilateral $ABCD$ (Figure \ref{sl.vek.5.6.3.pic}). It is clear that the point $O$ is at the same time the center of the circumscribed circle of the triangles $BCD$, $ACD$, $ABD$ and $ABC$.
 By Hamilton's statement \ref{Hamilton} it follows:

\begin{eqnarray*}
 \overrightarrow{OV_A}&=&\overrightarrow{OB}+\overrightarrow{OC}+\overrightarrow{OD}\\
\overrightarrow{OV_B}&=&\overrightarrow{OA}+\overrightarrow{OC}+\overrightarrow{OD}
 \end{eqnarray*}
 We then get:
\begin{eqnarray*}
 \overrightarrow{V_BV_A}&=&\overrightarrow{V_BO}+\overrightarrow{OV_A}=\\
&=&\overrightarrow{OV_A}-\overrightarrow{OV_B}=\\
 &=&\overrightarrow{OB}+\overrightarrow{OC}+\overrightarrow{OD}
-(\overrightarrow{OA}+\overrightarrow{OC}+\overrightarrow{OD})=\\
&=&\overrightarrow{OB}-\overrightarrow{OA}=\\
&=&\overrightarrow{AB}.
 \end{eqnarray*}


Therefore
 $\overrightarrow{V_BV_A}=\overrightarrow{AB}$. By the statement \ref{vektParalelogram} the quadrilateral $ABV_AV_B$ is a parallelogram, by the statement \ref{paralelogram} its diagonals $AV_A$ and $BV_B$ have a common center - we mark it with $S$
 (Figure \ref{sl.vek.5.6.3b.pic}). In a similar way each of the pairs of lines $AV_A$ and $CV_C$ or $AV_A$ and $DV_D$ have a common center. Because it is the center of the line $AV_A$, it follows that all four lines $AV_A$,
         $BV_B$, $CV_C$ and $DV_D$
         have a common center - the point $S$.

 In a similar way as $\overrightarrow{V_BV_A}=\overrightarrow{AB}$ it also follows that $\overrightarrow{V_CV_B}=\overrightarrow{BC}$, $\overrightarrow{V_DV_C}=\overrightarrow{CD}$ and $\overrightarrow{V_DV_A}=\overrightarrow{DA}$.
  This means that the quadrilateral $V_AV_BV_CV_D$ and $ABCD$ (statement \ref{vektVzpSkl} and \ref{KotaVzporKraki}) have all the congruent sides and interior angles. Therefore $V_AV_BV_CV_D\cong ABCD$. For a formal proof of this we can use the isometry $\mathcal{I}:A,B,C\mapsto V_A,V_B,V_C$ and prove $\mathcal{I}(D)=V_D$.
 \kdokaz

\bzgled \label{HamiltonSimson}\index{premica!Simsonova}
Let $ABCD$ be a cyclic quadrilateral and:  $a$ is the Simson
line with respect to the triangle $BCD$ and the point $A$, $b$ is the Simson
line with respect to the triangle  $ACD$  and the point  $B$, $c$  is the Simson
line with respect to the triangle  $ABD$  and the point  $C$ ter $d$ S is the Simson
line with respect to the triangle  $ABC$  and the point  $D$.
Prove that the lines $a$, $b$, $c$ and
$d$  intersect at a single point.
\ezgled

\begin{figure}[htp]
\centering
\input{sl.vek.5.6.4.pic}
\caption{} \label{sl.vek.5.6.4.pic}
\end{figure}

\textbf{\textit{Solution.}} (Figure \ref{sl.vek.5.6.4.pic}).

The claim is a direct consequence of theorems
\ref{SimsZgled3} and \ref{TetivniVisinska}  - the lines $a$, $b$, $c$ and
$d$ intersect at the point $S$ (from  theorem \ref{TetivniVisinska}).
\kdokaz



%________________________________________________________________________________
\poglavje{Euler Line}  \label{odd5EulPrem}

We will now prove an important property related to the three characteristic points of a triangle.

                \bizrek \label{EulerjevaPremica}
                The circumcentre $O$, the centroid $T$ and the orthocentre $V$
                of an arbitrary triangle lies on the same line. Besides that it is
                $$|OT|:|TV|=1:2.$$
                \eizrek

\begin{figure}[!htb]
\centering
\input{sl.vek.5.7.1.pic}
\caption{} \label{sl.vek.5.7.1.pic}
\end{figure}

 \textbf{\textit{Proof.}}  (Figure \ref{sl.vek.5.7.1.pic})

 If we use theorem \ref{vektTezTrikXT} and Hamilton's theorem \ref{Hamilton}, we get:
 $$\overrightarrow{OT}=\frac{1}{3}\left(\overrightarrow{OA}
                +\overrightarrow{OB}+\overrightarrow{OC}\right)=
                \frac{1}{3}\overrightarrow{OV}.$$

The vectors $\overrightarrow{OT}$ and $\overrightarrow{OV}$ are therefore collinear and it holds that $\overrightarrow{OT}:\overrightarrow{OV}=1:3$. This means that the points $O$, $T$ and $V$ are collinear and it holds that $|OT|:|TV|=1:2$.
 \kdokaz

The line from the previous theorem, on which three characteristic points lie, is called the \index{line!Euler's} \pojem{Euler line}.

 In the next theorem we will see the connection between the Euler line and the Euler circle, which we discussed in section \ref{odd3EulKroz}.




                    \bizrek \label{EulerKrozPrem1}\index{circle!Euler's}
                    The centre of Euler of an arbitrary triangle lies on
                    The Euler line of this triangle.
                    Furthermore, it is the midpoint of the line segment determined by
                     the orthocentre and the circumcentre of this triangle.
                    \eizrek

\begin{figure}[!htb]
\centering
\input{sl.vek.5.7.2.pic}
\caption{} \label{sl.vek.5.7.2.pic}
\end{figure}

 \textbf{\textit{Proof.}}
 Let $AA'$, $BB'$ and $CC'$ be altitudes, and $A_1$, $B_1$ and $C_1$ be the centres of sides $BC$, $AC$ and $AB$ of triangle $ABC$. We mark the centre of the circumscribed circle with $O$, the point of altitude of this triangle with $V$, and the centres of lines $VA$, $VB$ and $VC$ with $V_A$, $V_B$ and $V_C$ (Figure \ref{sl.vek.5.7.2.pic}).

 By theorem \ref{EulerKroznica}, points $A'$, $B'$, $C'$, $A_1$, $B_1$, $C_1$, $V_A$, $V_B$ and $V_C$ lie on one circle - i.e. the Euler circle. We mark the centre of this circle with $E$. Because $\angle V_AA'A_1\cong \angle AA'C=90^0$, by theorem \ref{TalesovIzrKroz2} line $V_AA_1$ is the diameter of this circle, or point $E$ is the centre of line $V_AA_1$.

According to the statement \ref{HamiltonLema} it holds:
$$\overrightarrow{OA_1}=\frac{1}{2}\cdot \overrightarrow{AV}=\overrightarrow{V_AV}.$$
Therefore $\overrightarrow{OA_1}=\overrightarrow{V_AV}$, which means that the quadrilateral $A_1OV_AV$ is a parallelogram (\ref{vektParalelogram}). Its diagonals $VO$ and $V_AA_1$ intersect (\ref{paralelogram}), so the point $E$ is the center of the line $OV$ and lies on Euler's line of the triangle $ABC$ (\ref{EulerjevaPremica}).
 \kdokaz

In section \ref{odd7SredRazteg} (\ref{EulerKroznicaHomot}) we will see the continuation of the previous statement, which refers to Euler's circle.


%________________________________________________________________________________
 \poglavje{Thales' Theorem - Basic Proportionality Theorem}  \label{odd5TalesVekt}

 We have already seen in section \ref{odd5LinKombVekt} that we can talk about the ratio of two collinear vectors.
 For collinear vectors $\overrightarrow{v}$ and $\overrightarrow{u}$ ($\overrightarrow{u}\neq \overrightarrow{0}$) we defined their ratio or quotient
 $$\overrightarrow{v}:\overrightarrow{u}
=\frac{\overrightarrow{v}}{\overrightarrow{u}}=\lambda,$$
 if for some $\lambda\in\mathbb{R}$ it holds $\overrightarrow{v}=\lambda \overrightarrow{u}$.

In a similar way as with numbers, we can define the ratio of two pairs of collinear vectors. If $\overrightarrow{a}$ and $\overrightarrow{b}$ ($\overrightarrow{b}\neq \overrightarrow{0}$) are a pair of collinear vectors or $\overrightarrow{c}$ and $\overrightarrow{d}$ ($\overrightarrow{d}\neq \overrightarrow{0}$) are a pair of collinear vectors, we say that the pairs of vectors \index{sorazmerje kolinearnih vektorjev}\pojem{sorazmerna}, if it holds:

        $$\frac{\overrightarrow{a}}{\overrightarrow{b}}
        =\frac{\overrightarrow{c}}{\overrightarrow{d}}.$$

 The next very important statement refers to the defined concept of ratio.

\bizrek
            \label{TalesovIzrek}(Thales'\footnote{Starogrški filozof in matematik \textit{Tales}
            \index{Tales} iz Mileta (640--546 pr. n. š.) je obravnaval sorazmerje ustreznih daljic,
            ki jih dobimo, če dve premici presekamo z dvema vzporednicama, pri tem pa ni omenjal
            vektorske oblike.} theorem - Basic Proportionality Theorem)\\
            Let $a$, $b$, $p$ and $p'$ be lines in the same plane, and $O=a\cap b$, $A=a\cap p$, $A'=a\cap p'$, $B=b\cap p$ and $B'=b\cap p'$.\\
             If $p\parallel p'$, then
            $$\frac{\overrightarrow{OA'}}{\overrightarrow{OA}}=
            \frac{\overrightarrow{OB'}}{\overrightarrow{OB}}=
            \frac{\overrightarrow{A'B'}}{\overrightarrow{AB}}.$$
            \index{izrek!Talesov o sorazmerju}
            \eizrek

\begin{figure}[!htb]
\centering
\input{sl.vek.5.8.1.pic}
\caption{} \label{sl.vek.5.8.1.pic}
\end{figure}

 \textbf{\textit{Proof.}}  (Figure \ref{sl.vek.5.8.1.pic})

Since by assumption $p\parallel p'$, the vectors $\overrightarrow{A'B'}$ and $\overrightarrow{AB}$ are collinear. By \ref{vektKriterijKolin} $\overrightarrow{A'B'}=\lambda\overrightarrow{AB}$ for some $\lambda\in \mathbb{R}$. In a similar way, from the collinearity of the vectors $\overrightarrow{OA'}$ and $\overrightarrow{OA}$ or $\overrightarrow{OB'}$ and $\overrightarrow{OB}$ it follows that $\overrightarrow{OA'}=\alpha\overrightarrow{OA}$ for some $\alpha\in \mathbb{R}$ or $\overrightarrow{OB'}=\beta\overrightarrow{OB}$ for some $\beta\in \mathbb{R}$. From this we obtain:
  $$\frac{\overrightarrow{A'B'}}{\overrightarrow{AB}}=\lambda,\hspace*{2mm}
  \frac{\overrightarrow{OA'}}{\overrightarrow{OA}}=\alpha,\hspace*{2mm}
  \frac{\overrightarrow{OB'}}{\overrightarrow{OB}}=\beta.$$
 It is enough to prove that $\alpha=\beta=\lambda$.
 If we use the rule for subtracting vectors \ref{vektOdsev} and \ref{vektVektorskiProstor} (point $\textit{6}$), we get:
 \begin{eqnarray*}
 \overrightarrow{A'B'}&=&\overrightarrow{OB'}-\overrightarrow{OA'}
 =\beta\overrightarrow{OB}-\alpha \overrightarrow{OA};\\
  \overrightarrow{A'B'}&=&\lambda\overrightarrow{AB}=
  \lambda\left(\overrightarrow{OB}-\overrightarrow{OA}\right)
 =\lambda\overrightarrow{OB}-\lambda\overrightarrow{OA}.
 \end{eqnarray*}
 Since $\overrightarrow{OA}$ and $\overrightarrow{OB}$ are non-collinear vectors, by \ref{vektLinKomb1Razcep}
 $\alpha=\beta=\lambda$ or $\frac{\overrightarrow{OA'}}{\overrightarrow{OA}}=
            \frac{\overrightarrow{OB'}}{\overrightarrow{OB}}=
            \frac{\overrightarrow{A'B'}}{\overrightarrow{AB}}$.
\kdokaz


  A direct consequence is the Tales theorem in the form of a ratio of distances, which is not in vector form (Figure \ref{sl.vek.5.8.2.pic}).

\bizrek \label{TalesovIzrekDolzine}
Let $a$, $b$, $p$ and $p'$ be lines in the same plane, and
$O=a\cap b$, $A=a\cap p$, $A'=a\cap p'$, $B=b\cap p$ and $B'=b\cap p'$.\\
If $p\parallel p'$, then
$$\frac{OA'}{OA}=
\frac{OB'}{OB}=
\frac{A'B'}{AB}$$
and also
$$\frac{OA'}{OB'}=
\frac{OA}{OB}.$$
\eizrek

\begin{figure}[!htb]
\centering
\input{sl.vek.5.8.2.pic}
\caption{} \label{sl.vek.5.8.2.pic}
\end{figure}

\textbf{\textit{Proof.}} The claim directly follows from theorems \ref{TalesovIzrek} and \ref{vektKolicnDolz}.
\kdokaz

We will prove that the converse is also true, i.e. that from the appropriate proportion follows the parallelism of the corresponding lines.

(Converse Thales' proportionality theorem)\\
Let $a$, $b$, $p$ and $p'$ be lines in the same plane, and $O=a\cap b$, $A=a\cap p$, $A'=a\cap p'$, $B=b\cap p$ and $B'=b\cap p'$.\\ If
$$\frac{\overrightarrow{OA'}}{\overrightarrow{OA}}=
\frac{\overrightarrow{OB'}}{\overrightarrow{OB}},$$
then $p\parallel p'$ and also
$$\frac{\overrightarrow{A'B'}}{\overrightarrow{AB}}=
\frac{\overrightarrow{OA'}}{\overrightarrow{OA}}=
\frac{\overrightarrow{OB'}}{\overrightarrow{OB}}.$$
\index{izrek!Talesov obratni o sorazmerju}

\textbf{\textit{Proof.}} We mark $\frac{\overrightarrow{OA'}}{\overrightarrow{OA}}=
\frac{\overrightarrow{OB'}}{\overrightarrow{OB}}=\lambda$. In this case, first $\overrightarrow{OA'}=\lambda\overrightarrow{OA}$ and $\overrightarrow{OB'}=\lambda\overrightarrow{OB}$. Therefore, (statement \ref{vektOdsev} and \ref{vektVektorskiProstor}):
$$\overrightarrow{A'B'}=\overrightarrow{OB'}-\overrightarrow{OA'}
=\lambda\overrightarrow{OB}-\lambda\overrightarrow{OA}
=\lambda\left(\overrightarrow{OB}-\overrightarrow{OA}\right)
=\lambda\overrightarrow{AB}.$$
Since $\overrightarrow{A'B'}=\lambda\overrightarrow{AB}$, according to statement \ref{vektKriterijKolin} vectors $\overrightarrow{A'B'}$ and $\overrightarrow{AB}$ are collinear. This means that $AB\parallel A'B'$ or $p\parallel p'$.

Finally, from statement \ref{TalesovIzrek} follows the relation
$\frac{\overrightarrow{A'B'}}{\overrightarrow{AB}}=
\frac{\overrightarrow{OA'}}{\overrightarrow{OA}}=
\frac{\overrightarrow{OB'}}{\overrightarrow{OB}}$.
\kdokaz

We also mention some consequences of Tales' statement \ref{TalesovIzrek}.

\bizrek \label{TalesPosl1}
                    If parallel lines $p_1$, $p_2$, $p_3$ intersect a line $a$ at points $A_1$, $A_2$,
                    $A_3$ and a line $b$ at points $B_1$, $B_2$,
                    $B_3$, then
                     $$\frac{A_1A_2}{B_1B_2}=\frac{A_2A_3}{B_2B_3}
                    \hspace*{1mm}
                    \textrm{ and } \hspace*{1mm}
                    \frac{A_1A_2}{A_2A_3}=\frac{B_1B_2}{B_2B_3}.$$
                    \eizrek

\begin{figure}[!htb]
\centering
\input{sl.vek.5.8.4.pic}
\caption{} \label{sl.vek.5.8.4.pic}
\end{figure}

 \textbf{\textit{Proof.}}  (Figure \ref{sl.vek.5.8.4.pic})

 Without loss of generality, let $\mathcal{B}(A_1,A_2,A_3)$ and $\mathcal{B}(B_1,B_2,B_3)$.

We mark with $c$ the parallel to the line $b$ through the point $A_1$ and with $C_2$ and $C_3$ the intersections of the line $c$ with the lines $p_2$ and $p_3$. The quadrilateral $B_1B_3C_3A_1$ and $B_2B_3C_3C_2$ are parallelograms, therefore $\overrightarrow{B_1B_3}=\overrightarrow{A_1C_3}$ and $\overrightarrow{B_2B_3}=\overrightarrow{C_2C_3}$.

If we use the statement \ref{TalesovIzrekDolzine}, we get:
 \begin{eqnarray*}
 \frac{|A_1A_2|}{|A_2A_3|}&=&
 \frac{|A_1A_3|-|A_2A_3|}{|A_2A_3|}=
\frac{|A_1A_3|}{|A_2A_3|}-1=\\
 &=&\frac{|A_1C_3|}{|C_2C_3|}-1=
 \frac{|B_1B_3|}{|B_2B_3|}-1=\\
 &=&\frac{|B_1B_3|-|B_2B_3|}{|B_2A_3|}=
\frac{|B_1B_2|}{|B_2B_3|},
 \end{eqnarray*}
or $\frac{A_1A_2}{A_2A_3}=\frac{B_1B_2}{B_2B_3}$ and
$\frac{A_1A_2}{B_1B_2}=\frac{A_2A_3}{B_2B_3}$.
 \kdokaz


                    \bizrek \label{TalesPosl2}
                    If parallel lines $p_1$, $p_2$,..., $p_n$ intersect a line $a$ at points $A_1$, $A_2$,...,
                    $A_n$ and a line $b$ at points $B_1$, $B_2$,...,
                    $B_n$, then
                    $$\frac{A_1A_2}{B_1B_2}=\frac{A_2A_3}{B_2B_3}=\cdots=
                    \frac{A_{n-1}A_n}{B_{n-1}B_n}\hspace*{1mm}
                    \textrm{ and } \hspace*{1mm}$$
                    $$A_1A_2:A_2A_3:\cdots :A_{n-1}A_n= B_1B_2:B_2B_3:\cdots :B_{n-1}B_n.$$
                    \eizrek


\begin{figure}[!htb]
\centering
\input{sl.vek.5.8.5.pic}
\caption{} \label{sl.vek.5.8.5.pic}
\end{figure}

 \textbf{\textit{Proof.}}  (Figure \ref{sl.vek.5.8.5.pic})

The statement is a direct consequence of the statement \ref{TalesPosl1}.
 \kdokaz

\bizrek \label{TalesPosl3}
Let $p_1$, $p_2$ and $p_3$ be lines that intersect at a point $O$. If $a$ and $b$ are
parallel lines that intersect the line $p_1$ at points $A_1$ and $B_1$,
the line $p_2$ at points $A_2$ and $B_2$, and
the line $p_3$ at points $A_3$ and $B_3$, then
$$\frac{A_1A_2}{B_1B_2}=\frac{A_2A_3}{B_2B_3}
\hspace*{1mm}
\textrm{ and } \hspace*{1mm}
\frac{A_1A_2}{A_2A_3}=\frac{B_1B_2}{B_2B_3}.$$
\eizrek

\begin{figure}[!htb]
\centering
\input{sl.vek.5.8.6.pic}
\caption{} \label{sl.vek.5.8.6.pic}
\end{figure}

 \textbf{\textit{Proof.}}  (Figure \ref{sl.vek.5.8.6.pic})

 By izreku \ref{TalesovIzrekDolzine} we get:
  $$\frac{A_1A_2}{B_1B_2}=
  \frac{OA_2}{OB_2}=\frac{A_2A_3}{B_2B_3},$$ which was to be proven. \kdokaz

\bizrek \label{TalesPosl4}
                    Let $p_1$, $p_2$,..., $p_n$  be  lines that intersect at a point $O$. If $a$ and $b$ are
                    parallel lines that intersect the line $p_1$ at points $A_1$ and $B_1$,
                    the line $p_2$ at points $A_2$ and $B_2$,...,
                    the line $p_n$ at points $A_n$ and $B_n$, then
                     $$\frac{A_1A_2}{B_1B_2}=\frac{A_2A_3}{B_2B_3}=\cdots=
                    \frac{A_{n-1}A_n}{B_{n-1}B_n}\hspace*{1mm}
                    \textrm{ and } \hspace*{1mm}$$
                    $$A_1A_2:A_2A_3:\cdots :A_{n-1}A_n= B_1B_2:B_2B_3:\cdots :B_{n-1}B_n.$$
                    \eizrek



\begin{figure}[!htb]
\centering
\input{sl.vek.5.8.8.pic}
\caption{} \label{sl.vek.5.8.8.pic}
\end{figure}

 \textbf{\textit{Proof.}}  (Figure \ref{sl.vek.5.8.8.pic})

The statement is a direct consequence of  \ref{TalesPosl3}.
 \kdokaz

 The following very well-known and useful planning tasks are next.



                     \bzgled
                     \label{izrekEnaDelitevDaljiceNan}
                     \index{delitev daljice!na enake dele}
                      Construct points that divide a line segment $AB$
                      into $n$ congruent line segments.
                    \ezgled

\begin{figure}[!htb]
\centering
\input{sl.vek.5.8.9.pic}
\caption{} \label{sl.vek.5.8.9.pic}
\end{figure}


\textbf{\textit{Solution.}}  (Figure \ref{sl.vek.5.8.9.pic})

Let $X$ be an arbitrary point that does not lie on the line $AB$ and $Q_1$, $Q_2$, ..., $Q_n$ such points on the line segment $AX$, so that
$\overrightarrow{AQ_1}=\overrightarrow{Q_1Q_2}=\cdots =\overrightarrow{Q_{n-1}Q_n}$.
With $P_1$, $P_2$, ..., $P_{n-1}$ we denote the intersection of the line $AB$ with the parallels of the line $BQ_n$ through the points $Q_1$, $Q_2$, ..., $Q_{n-1}$.

We prove that $P_1$, $P_2$, ..., $P_{n-1}$ are the desired points. By \ref{TalesPosl2} we have:
$$\frac{AP_1}{AQ_1}=\frac{P_1P_2}{Q_1Q_2}=\dots=\frac{P_{n-1}B}{Q_{n-1}Q_n}.$$
Since, by assumption, $|\overrightarrow{AQ_1}|=|\overrightarrow{Q_1Q_2}|=\cdots =|\overrightarrow{Q_{n-1}Q_n}|$, also
$|AP_1|=|P_1P_2|=\cdots =|P_{n-1}B|$.
 \kdokaz


                     \bzgled
                     \label{izrekEnaDelitevDaljice}
                     \index{delitev daljice!v razmerju}
                     Divide a given line segment $AB$ in the ratio $n:m$ ($n,m\in \mathbb{N}$),
                    i.e. determine such point $T$ on the line $AB$ that
                     $$\frac{\overrightarrow{AT}}{\overrightarrow{TB}}=\frac{n}{m}.$$
                     Prove that there is the only one solution for such point $T$.
                    \ezgled


\begin{figure}[!htb]
\centering
\input{sl.vek.5.8.10.pic}
\caption{} \label{sl.vek.5.8.10.pic}
\end{figure}


\textbf{\textit{Solution.}}  (Figure \ref{sl.vek.5.8.10.pic})


  By \ref{izrekEnaDelitevDaljiceVekt} there is only one point $T$, for which $\frac{\overrightarrow{AT}}{\overrightarrow{TB}}=\frac{n}{m}$.

Now we will describe the process of constructing point $T$. Let $\overrightarrow{v}$ be an arbitrary vector that is not collinear with vector $\overrightarrow{AB}$, and let $P$ and $Q$ be such points that $\overrightarrow{AP}=n\overrightarrow{v}$ and $\overrightarrow{BQ}=-m\overrightarrow{v}$. We denote the intersection of lines $AB$ and $PQ$ by $T$ (the intersection exists because $P,Q\div AB$). Then it holds:
   \begin{eqnarray*}
   \frac{\overrightarrow{AT}}{\overrightarrow{TB}}=
   -\frac{\overrightarrow{TA}}{\overrightarrow{TB}}=
   -\frac{\overrightarrow{AP}}{\overrightarrow{BQ}}=
   -\frac{n\overrightarrow{v}}{-m\overrightarrow{v}}=\frac{n}{m},
   \end{eqnarray*}
 which had to be proven.  \kdokaz

 In section \ref{odd7Harm} we will further investigate the question of dividing a line segment into a given ratio.



                    \bzgled \label{vektTrapezZgled}
                    A line parallel to bases of a trapezium intersects its legs and diagonals in four points
                    and determine three line segments. Prove that two of them are congruent. \\
                    After that, construct a line parallel to the bases of that trapezium  such that all
                    three mentioned line segments are congruent.
                    \ezgled

\begin{figure}[!htb]
\centering
\input{sl.vek.5.8.11.pic}
\caption{} \label{sl.vek.5.8.11.pic}
\end{figure}

\textbf{\textit{Proof.}}
Let $l$ be a line parallel to the bases
$AB$ of trapezium $ABCD$, that intersects
sides $AD$ and $BC$ as well as diagonals $AC$ and $BD$
in points $M$, $Q$, $N$ and $P$ (Figure \ref{sl.vek.5.8.11.pic}). By Tales' theorem \ref{TalesovIzrekDolzine} it holds:
$MN:DC=MA:DA$ or
$PQ:DC=BQ:BC$. Because by
the consequence \ref{TalesPosl1} of Tales' theorem $MA:AD=BQ:BC$, it also holds
$MN:DC=PQ:DC$ or $MN\cong PQ$.

If $E$ is the center of the base $AB$ and $N_0$ is the intersection of the lines $DE$ and $AC$, the desired line $l_0$ passes through the point $N_0$ and is parallel to the line $AB$. If $M_0$, $P_0$ and $Q_0$ are the intersections of the line $l_0$ with the lines $AD$, $BD$ and $CB$, then by \ref{TalesPosl3} of Tales' theorem it follows that $M_0N_0\cong N_0P_0$.
\kdokaz


                    \bzgled If $r$ is the inradius and $r_a$, $r_b$ and $r_c$ exradii
                    of an arbitrary triangle, then $$\frac{1}{r_a}
                    +\frac{1}{r_b} +\frac{1}{r_c}= \frac{1}{r}.$$
                    \ezgled


\begin{figure}[!htb]
\centering
\input{sl.vek.5.8.12.pic}
\caption{} \label{sl.vek.5.8.12.pic}
\end{figure}

\textbf{\textit{Proof.}} If we use the labels from the big task
\ref{velikaNaloga} (Figure \ref{sl.vek.5.8.12.pic}), from Tales' theorem \ref{TalesovIzrek} it follows that:
$$ \frac{r}{r_a} = \frac{SQ}{S_aQ_a} =  \frac{AQ}{AQ_a}  =  \frac{s-a}{s}.$$
 Similarly, it is also:
 $$ \frac{r}{r_b} =  \frac{s-b}{s}\hspace*{2mm} \textrm{ and }\hspace*{2mm}
 \frac{r}{r_c} =  \frac{s-c}{s}.$$
 By adding the three equalities we get the desired relation.
  \kdokaz



                \bizrek \label{velNalTockP'}
                Suppose that the incircle and the excircle of a triangle $ABC$ touch its side $BC$ in  points $P$ and $P_a$.
                If $PP'$ is a diameter of the incircle $k(S,r)$ ($P'\in k$), then $Pa$, $P'$ and $A$ are collinear
                points.
                \eizrek



\begin{figure}[!htb]
\centering
\input{sl.vek.5.8.13.pic}
\caption{} \label{sl.vek.5.8.13.pic}
\end{figure}


 \textbf{\textit{Proof.}} Use the labels from the big task
\ref{velikaNaloga} (Figure \ref{sl.vek.5.8.13.pic}). Because the distance $PP'$
  is a diameter of the inscribed circle, the point $S$ is its center.

Let $\widehat{P'}$ be the intersection of the line segment $PS$ with the line $AP_a$.
 We will prove that
  $\widehat{P'}=P'$, or that
$S\widehat{P'}=r$. From Tales' theorem it follows that:
 $$\frac{S\widehat{P'}}{S_aP_a} = \frac{AS}{AS_a} =\frac{SQ}{S_aQ_a}.$$
 Because $S_a P_a  = S_aQ_a  = r_a$,
 it is also true that $S\widehat{P'}=SQ=r$.
 \kdokaz

In a similar way to \ref{velNalTockP'} we also prove the following theorem.




                \bizrek \label{velNalTockP'1}
                 Suppose that the incircle and the excircle of a triangle $ABC$ touch its side $BC$ in  points $P$ and $P_a$.
                If $P_aP_a'$ is a diameter of the excircle $k_a(S_a,r_a)$ ($P_a'\in k_a$), then $P_a'$, $P$ and $A$ are collinear
                points
                  (Figure \ref{sl.vek.5.8.14.pic}).
                \eizrek




\begin{figure}[htp]
\centering
\input{sl.vek.5.8.14.pic}
\caption{} \label{sl.vek.5.8.14.pic}
\end{figure}

%\vspace*{10mm}



             \bizrek \label{velNalTockP'2}
              Suppose that the excircles $k_b(S_b,r_b)$ and $k_c(S_c,r_c)$ of a triangle $ABC$
                touch the line $BC$ in  points $P_b$ and $P_c$.
                If $P_bP_b'$ and $P_cP_c'$ are diameters of the excircles
                 $k_b$ ($P_b'\in k_b$) and $k_c$ ($P_c'\in k_c$), then $P_c'$, $P_b$ and $A$ are collinear
                points and also $P_b'$, $P_c$ and $A$ are collinear
                points (Figure \ref{sl.vek.5.8.15.pic}).
             \eizrek


\begin{figure}[htp]
\centering
\input{sl.vek.5.8.15.pic}
\caption{} \label{sl.vek.5.8.15.pic}
\end{figure}






The last three statements can be used in the construction of triangles,
which we will illustrate in the following task.

\bzgled
                Construct a triangle $ABC$, with given $r$, $b-c$, $t_a$.
                 \ezgled


\begin{figure}[htp]
\centering
\input{sl.vek.5.8.16.pic}
\caption{} \label{sl.vek.5.8.16.pic}
\end{figure}

 \textbf{\textit{Solution.}}
 Let $ABC$ be the desired triangle or the triangle, in which the
radius of the inscribed circle is $r$, the centroid $AA_1$ is consistent with
$t_a$ and the difference of sides $AC$ and $AB$ is equal to $b-c$. Use
the notation from the big task \ref{velikaNaloga} and the formula
\ref{velNalTockP'}  (Figure \ref{sl.vek.5.8.16.pic}). We know that $PP_a=b-c$, point $A_1$ is
the common center of side $BC$ and distance $PP_a$. From the formula
\ref{velNalTockP'} it follows that the points $P_a$, $P'$ and $A$
are collinear.

So first we construct the right triangle $P'PP_a$, then the
circle $k$ with diameter $PP'$ or the inscribed circle
 of the triangle $ABC$ and the point $A_1$ as
the center of the line segment $PP_a$. The point $A$ is the intersection of the line segment
$P_aP'$ with the circle $k_1(A_1,t_a)$,  points $B$ and $C$ are
the intersections of the line $BC$ with the tangents of the circle $k$  from the point
$A$.
 \kdokaz

            \bzgled
             Let $A_1$ be the midpoint of the line segment $BC$, $k(S,r)$
            the incircle and $AA'$ the altitude of a triangle $ABC$.
            Suppose that $L$ is the intersection of the lines $A_1S$ and $AA'$.
            Prove that $AL\cong
             r$.
            \ezgled


\begin{figure}[htp]
\centering
\input{sl.vek.5.8.3.pic}
\caption{} \label{sl.vek.5.8.3.pic}
\end{figure}

\textbf{\textit{Solution.}} Let $P$, $Q$ and $R$ be the points of intersection of the circle $k$ with the sides $BC$, $AC$ and $AB$. Let $k(S_a,r_a)$ be the drawn circle that touches the side $BC$ in the point $P_a$, the lines $AB$ and $AC$ in the points $R_a$ and $Q_a$. Let $P'$ be the intersection of the lines $AP_a$ and $PS$ (Figure \ref{sl.vek.5.8.3.pic}). From the statement \ref{velNalTockP'} and the big exercise \ref{velikaNaloga} it follows:
\begin{itemize}
  \item the point $P'$ lies on the circle $k$,
  \item the point $A_1$ is the center of the segment $PP_a$.
\end{itemize}
We prove that $AL\cong
 r$. The segment $SA_1$ is the median of the triangle
$PP'P_a$, therefore $SA_1\parallel P'P_a$ or $LS\parallel AP'$.
Since $AL\parallel P'S$, the quadrilateral $AP'SL$ is a parallelogram
and $AL\cong P'S\cong r$.
 \kdokaz



%________________________________________________________________________________
\naloge{Exercises}

\begin{enumerate}

  %Vsota  in razlika vektorjev
    %_____________________________________

  \item Draw any vectors $\overrightarrow{a}$, $\overrightarrow{b}$, $\overrightarrow{c}$ and $\overrightarrow{d}$ so that their sum is equal to:

  (\textit{a}) one of these four vectors,\\
  (\textit{b}) the difference of two of these four vectors.

   \item Let $ABCDE$ be a pentagon in some plane. Prove that in this plane there exists
a pentagon with sides that determine the same vectors as the
 diagonals of the pentagon $ABCDE$.

  \item Let $A$, $B$, $C$ and $D$ be any points in a plane. Is it generally true that:

  (\textit{a}) $\overrightarrow{AB}+\overrightarrow{BD}=\overrightarrow{AD}+\overrightarrow{BC}$?\\
  (\textit{b}) $\overrightarrow{AB}=\overrightarrow{DC}\hspace*{1mm}\Rightarrow \hspace*{1mm} \overrightarrow{AC}+\overrightarrow{BD}=2\overrightarrow{BC}$?

  \item Given is a segment $AB$. Using only a ruler (constructions in affine geometry) draw a point $C$ so that:

    (\textit{a}) $\overrightarrow{AC}=-\overrightarrow{AB}$, \hspace*{3mm}
   (\textit{b}) $\overrightarrow{AC}=5\overrightarrow{AB}$, \hspace*{3mm}
   (\textit{c}) $\overrightarrow{AC}=-3\overrightarrow{AB}$.

\item Let $ABCD$ be a quadrilateral and $O$ any point in the plane of this quadrilateral. Express the vectors of sides and
diagonals of this quadrilateral with vectors $\overrightarrow{a}=\overrightarrow{OA}$, $\overrightarrow{b}=\overrightarrow{OB}$, $\overrightarrow{c}=\overrightarrow{OC}$ and $\overrightarrow{d}=\overrightarrow{OD}$.

   \item Let $ABCD$ be a quadrilateral and $O$ any point in the plane of this quadrilateral. Is the equivalence true that the quadrilateral $ABCD$ is a parallelogram exactly when $\overrightarrow{OA}+\overrightarrow{OC}=
    \overrightarrow{OB}+\overrightarrow{OD}$?

 \item Let $ABCD$ be a parallelogram, $S$
the intersection of its diagonals and $M$ any point in the plane of this parallelogram. Prove that:
        $$\overrightarrow{MS} = \frac{1}{4}\cdot
        \left( \overrightarrow{MA}+\overrightarrow{MB}
         +\overrightarrow{MC} + \overrightarrow{MD} \right).$$

 \item Let $ABB_1A_2$,
$BCC_1B_2$ and $CAA_1C_2$ be parallelograms, which are drawn over the sides of triangle $ABC$. Prove that:
$$\overrightarrow{A_1A_2}+\overrightarrow{B_1B_2}+
\overrightarrow{C_1C_2}=\overrightarrow{0}.$$

  \item The perpendicular lines $p$ and $q$, which intersect in point $M$, intersect the circle $k$ with center $O$ in
points $A$, $B$, $C$ and $D$. Prove that:
$$\overrightarrow{OA}+ \overrightarrow{OB} + \overrightarrow{OC} + \overrightarrow{OD} = 2\overrightarrow{OM}.$$

\item Let $A$, $B$, $C$ and $D$ be any points in some plane. Can all six lines, which are determined by these points, be oriented so that the sum of the corresponding six vectors is equal to the vector of nothing?

\item Let $A_1$, $B_1$ and $C_1$ be the centers of sides $BC$, $AC$ and $AB$ of triangle $ABC$ and $M$ any point. Prove:

(\textit{a}) $\overrightarrow{AA_1}+\overrightarrow{BB_1}+
\overrightarrow{CC_1}=\overrightarrow{0}$,\\
   (\textit{b}) $\overrightarrow{MA}+\overrightarrow{MB}+\overrightarrow{MC}=
   \overrightarrow{MA_1}+\overrightarrow{MB_1}+\overrightarrow{MC_1}$,\\
   (\textit{c}) There exists a triangle $PQR$, such that for its vertices it holds:\\ \hspace*{7mm} $\overrightarrow{PQ}=\overrightarrow{CC_1}$, $\overrightarrow{PR}=\overrightarrow{BB_1}$ and $\overrightarrow{RQ}=\overrightarrow{AA_1}$.

 \item Let $M$, $N$, $P$, $Q$, $R$ and  $S$ be in order the centers of the sides of an arbitrary hexagon.
Prove that it holds:
$$\overrightarrow{MN}+\overrightarrow{PQ}+
\overrightarrow{RS}=\overrightarrow{0}.$$


  %Linearna kombinacija vektorjev
    %_____________________________________

  \item Let $ABCDEF$ be a convex hexagon, for which $AB\parallel DE$, and points $K$ and $L$ are the centers of the lines determined by the centers of the remaining pairs of opposite sides. Prove that $K=L$ exactly when $AB\cong DE$.

 \item Let $P$ and $Q$ be such points of sides $BC$ and $CD$ of a parallelogram $ABCD$, that $BP:PC=2:3$ and
$CQ:QD=2:5$. Point $X$ is the intersection of lines $AP$ and $BQ$. Calculate the ratios in which point $X$ divides
line $AP$ and $BQ$.

\item Let $A$, $B$, $C$ and $D$ be arbitrary points of a plane. Point $E$ is the center of line $AB$, $F$ and
$G$ are such points, that $\overrightarrow{EF} = \overrightarrow{BC}$ and $\overrightarrow{EG} = \overrightarrow{AD}$,  and $S$ is the center of line $CD$. Prove that $G$, $S$ and $F$
are collinear points.

\item Let $K$ and $L$  be such points of side $AD$ and diagonal $AC$ of a parallelogram $ABCD$, that $\frac{\overrightarrow{AK}}{\overrightarrow{KD}}=\frac{1}{3}$ and
    $\frac{\overrightarrow{AL}}{\overrightarrow{LC}}=\frac{1}{4}$. Prove that $K$, $L$ and $B$ are collinear points.

\item Let $X_n$ and $Y_n$ ($n\in \mathbb{N}$) be such points of sides $AB$ and $AC$ of triangle $ABC$, that $\overrightarrow{AX_n}=\frac{1}{n+1}\cdot \overrightarrow{AB}$ and
    $\overrightarrow{AY_n}=\frac{1}{n}\cdot \overrightarrow{AC}$. Prove that there exists a point, which lies on all lines
$X_nY_n$  ($n\in \mathbb{N}$).

 \item Let $M$, $N$, $P$ and $Q$ be the centers of sides $AB$, $BC$, $CD$ and $DA$
     of quadrilateral
     $ABCD$. Is the following equivalence true, that the quadrilateral $ABCD$ is a parallelogram exactly when:

    (\textit{a}) $2\overrightarrow{MP}=\overrightarrow{BC}+\overrightarrow{AD}$ and
    $2\overrightarrow{NQ}=\overrightarrow{BA}+\overrightarrow{CD}$?\\
   (\textit{b}) $2\overrightarrow{MP}+2\overrightarrow{NQ}=
   \overrightarrow{AB}+\overrightarrow{BC}+\overrightarrow{CD}+\overrightarrow{DA}$?

 \item Let $E$, $F$ and $G$ be the centers of sides $AB$, $BC$ and $CD$ of parallelogram $ABCD $, lines $BG$ and $DE$ intersect line $AF$ in points $N$ and $M$. Express $\overrightarrow{AF}$, $\overrightarrow{AM}$ and $\overrightarrow{AN}$ as a linear combination of vectors $\overrightarrow{AB}$ and $\overrightarrow{AD}$. Prove that points $M$ and $N$ divide the distance $AF$ in the ratio $2:2:1$.

 \item Points $K$, $L$, $M$ and $N$ lie on sides $AB$, $BC$, $CD$ and $DA$
of quadrilateral $ABCD$. If the quadrilateral $KLMN$ is a parallelogram and it is true that
$$\frac{\overrightarrow{AK}}{\overrightarrow{KB}}=
\frac{\overrightarrow{BL}}{\overrightarrow{LC}}
=\frac{\overrightarrow{CM}}{\overrightarrow{MD}}=
\frac{\overrightarrow{DN}}{\overrightarrow{NA}}=\lambda$$ for some $\lambda\neq\pm 1$,
then the quadrilateral $ABCD$
is a parallelogram. Prove it.

\item Let $M$ be the center of side $DE$ of regular hexagon $ABCDEF$. Point
$N$ is the center of line $AM$, point $P$ is the center of side $BC$. Express $\overrightarrow{NP}$ as a linear combination of vectors
$\overrightarrow{AB}$ and $\overrightarrow{AF}$.


  %The length of a vector
    %_____________________________________

  \item Prove that for any points $A$, $B$ and $C$ it is true that:

(\textit{a}) $|\overrightarrow{AC}|\leq|\overrightarrow{AB}|+|\overrightarrow{BC}|$ \hspace*{6mm}
   (\textit{b}) $|\overrightarrow{AC}|\geq|\overrightarrow{AB}|-|\overrightarrow{BC}|$\\
  Under which conditions is the equality true?

  \item Let $M$ and $N$ be points that lie on the lines $AD$ and $BC$, respectively, such that $\frac{\overrightarrow{AM}}{\overrightarrow{MD}}\cdot \frac{\overrightarrow{CN}}{\overrightarrow{NB}}=1$. Prove that:
      $$|MN|\leq\max\{|AB|, |CD|\}.$$

  %Težišče
    %_____________________________________

  \item Points $T$ and $T'$ are the centroids of $n$-sided polygons $A_1A_2...A_n$ and $A'_1A'_2...A'_n$. Calculate:
$$\overrightarrow{A_1A'_1}+\overrightarrow{A_2A'_2}+\cdots+\overrightarrow{A_nA'_n}.$$

    \item Prove that quadrilaterals $ABCD$ and  $A'B'C'D'$ have a common centroid exactly when:
$$\overrightarrow{AA'}+\overrightarrow{BB'}+\overrightarrow{CC'}+
\overrightarrow{DD'}=\overrightarrow{0}.$$

    \item Let $P$, $Q$, $R$ and $S$ be the centroids of triangles $ABD$, $BCA$, $CDB$ and $DAC$. Prove that quadrilaterals $PQRS$ and $ABCD$ have a common centroid.

  \item Let $A_1A_2A_3A_4A_5A_6$ be an arbitrary hexagon and $B_1$, $B_2$, $B_3$, $B_4$, $B_5$ and $B_6$ are the centroids of triangles $A_1A_2A_3$, $A_2A_3A_4$, $A_3A_4A_5$, $A_4A_5A_6$, $A_5A_6A_1$ and $A_6A_1A_2$, respectively.
Prove that these centroids determine a hexagon with three pairs of parallel sides.

 \item Let $A$, $B$, $C$ and $D$ be four different points. Points $T_A$, $T_B$, $T_C$ and $T_D$
        are the centroids of triangles $BCD$, $ACD$, $ABD$ and $ABC$, respectively. Prove that the lines $AT_A$, $BT_B$, $CT_C$ and $DT_D$ intersect at one point $T$. In what ratio does point $T$ divide these lines?

\item Let $CC_1$ be the centroid of the triangle $ABC$ and $P$ any point on the side
$AB$ of this triangle. The parallel $l$ of the line $CC_1$ through the point $P$ intersects the line $AC$ and $BC$ in points $M$ and $N$. Prove that:
$$\overrightarrow{PM} + \overrightarrow{PN}= \overrightarrow{AC} + \overrightarrow{BC}.$$


 %Hamilton and Euler
 %______________________________________________________


 \item Let $A$, $B$, $C$ be points of a plane that lie on the same side of the line $p$, and $O$ a point on
the line $p$, for which  $|\overrightarrow{OA}| = |\overrightarrow{OB}| = |\overrightarrow{OC}| =1$. Prove that then it also holds: $$|\overrightarrow{OA} + \overrightarrow{OB} + \overrightarrow{OC}| \geq 1.$$

\item Calculate the angles determined by the vectors $\overrightarrow{OA}$, $\overrightarrow{OB}$ and $\overrightarrow{OC}$, if the points $A$, $B$ and $C$ lie
on a circle with center
$O$ and in addition it holds:
$$\overrightarrow{OA} + \overrightarrow{OB} + \overrightarrow{OC} = \overrightarrow{0}.$$

 \item Let $A$, $B$, $C$ and $D$ be points that lie
on a circle with center
$O$, and it holds
$$\overrightarrow{OA} + \overrightarrow{OB} + \overrightarrow{OC} + \overrightarrow{OD} = \overrightarrow{0}.$$
Prove that $ABCD$ is a rectangle.

\item Let $\overrightarrow{a}$, $\overrightarrow{b}$ and $\overrightarrow{c}$ be vectors of a plane, for which $|\overrightarrow{a}| = |\overrightarrow{b}| = |\overrightarrow{c}| =x$. Investigate, in which case it also holds $|\overrightarrow{a} + \overrightarrow{b} + \overrightarrow{c}| = x$.


 %Tales
 %______________________________________________________


  \item Divide the distance $AB$:

  (\textit{a}) into five equal distances,\\
  (\textit{b}) in the ratio $2:5$,\\
  (\textit{c}) into three distances, which are in the ratio $2:\frac{1}{2}:1$.


  \item The distance $AB$ is given. Only by using the rule with the possibility of drawing parallels (constructions in affine geometry) plot the point $C$, if it is:

(\textit{a}) $\overrightarrow{AC}=\frac{1}{3}\overrightarrow{AB}$  \hspace*{3mm}
   (\textit{b}) $\overrightarrow{AC}=\frac{3}{5}\overrightarrow{AB}$  \hspace*{3mm}
   (\textit{c}) $\overrightarrow{AC}=-\frac{4}{7}\overrightarrow{AB}$

  \item The distances $a$, $b$ and $c$ are given. Plot the distance $x$, so that it will:

(\textit{a}) $a:b=c:x$ \hspace*{3mm}
    (\textit{b}) $x=\frac{a\cdot b}{c}$ \hspace*{3mm}
   (\textit{c}) $x=\frac{a^2}{c}$\hspace*{3mm}\\
   (\textit{č}) $x=\frac{2ab}{3c}$\hspace*{3mm}
   (\textit{č}) $(x+c):(x-c)=7:2$

 \item Let $M$ and $N$ be points on the arm $OX$,  $P$ point on the arm $OY$ angle
$XOY$ and $NQ\parallel MP$ and $PN\parallel QS$ ($Q\in OY$, $S\in OX$). Prove that
$|ON|^2=|OM|\cdot |OS|$ (for the distance $ON$ in this case we say that it is \index{geometrijska sredina daljic}\pojem{geometrijska sredina} of the distances $OM$
and $OS$).


\item Let  $ABC$ be a triangle and $Q$, $K$, $L$, $M$, $N$ and $P$ such points on the sides $AB$, $AC$, $BC$,
$BA$, $CA$ and $CB$, that $AQ\cong CP\cong AC$, $AK\cong BL\cong AB$ and $BM\cong CN\cong BC$.
Prove that $MN$, $PQ$ and $LK$ are three parallel.

\item Let $P$ be the center of the centroid $AA_1$ of the triangle $ABC$. The point $Q$ is the intersection of the line $BP$
with the side $AC$. Calculate the ratios $AQ:QC$ and $BP:PQ$.

\item The points $P$ and $Q$ lie on the sides $AB$ and $AC$ of the triangle $ABC$, and $\frac{|\overrightarrow{PB}|}{|\overrightarrow{AP}|}
    +\frac{|\overrightarrow{QC}|}{|\overrightarrow{AQ}|}=1$. Prove that the centroid of this triangle lies on the line $PQ$.

 \item Let $a$, $b$ and $c$ be three sides with a common  starting point $S$ and $M$
point on the side $a$. If the point $M$ "moves" along the side $a$,
 the ratio of the distance of this point from the lines $b$ and $c$ is constant. Prove it.

 \item Let $D$ be a point that lies on the side $BC$ of the triangle $ABC$ and
 $F$ and $G$ points in which the line passing through the point $D$ and parallel  to the centroid $AA_1$, intersects the lines $AB$ and $AC$.
Prove that the sum $|DF|+|DG|$ is constant if the point $D$ "moves" along the side
$BC$.


   \item
   Draw a triangle with the following data:

   (\textit{a}) $v_a$, $r$, $b-c$  \hspace*{3mm}
   (\textit{b}) $\beta$, $r$, $b-c$

\end{enumerate}






% DEL 6 - - - - - - - - - - - - - - - - - - - - - - - - - - - - - - - - - - - - - - -
%________________________________________________________________________________
% IZOMETRIJE
%________________________________________________________________________________

\del{Isometries} \label{pogIZO}

%________________________________________________________________________________
\poglavje{Isometries. Identity Map}  \label{odd6Ident}

We have formally defined isometries or isometric transformations of a plane $\mathcal{I}:\mathbb{E}^2\rightarrow \mathbb{E}^2$ in section \ref{odd2AKSSKL} as transformations of a plane that preserve the relation of congruence of pairs of points. We later used them to define the relation of congruence of figures. We intuitively identify them as movements of a plane. Some of them we already know from before (we have not formally introduced them in this book) - rotation and translation (Figure \ref{sl.izo.6.1.1.pic}). Even reflection over a line is an isometry. But this differs from the aforementioned two isometries because it is not a true movement (in a plane). In order to see it as a movement, we need to go to a space where we can rotate the plane by $180^0$. This changes the orientation of the plane.

\begin{figure}[!htb]
\centering
\input{sl.izo.6.1.1.pic}
\caption{} \label{sl.izo.6.1.1.pic}
\end{figure}

We call such isometries, which change the orientation of the plane, \index{izometrija!indirektna} \pojem{indirect}.
Reflection over a line is therefore an indirect isometry. For those isometries that preserve the orientation of the plane, we say that they are \index{izometrija!direktna}\pojem{direct}. Rotation and translation are examples of direct isometries. At this point, we will not prove the fact that every isometry of a plane is either direct or indirect, or that if one isometry preserves the orientation of one figure, it also preserves the orientation of all other figures.

It is clear that the composition of two direct or two indirect isometries represents a direct isometry. Similarly, the composition of one direct and one indirect isometry is an indirect isometry.

Intuitively, with direct isometries we can bring the figure and its image with movement in the plane to the point of overlap. For indirect isometries this is not possible with free movement in the plane - it is necessary to use movement in three-dimensional space.

Except for the conditions of directness and indirectness, another very important property of isometries is the number of fixed points. We recall that we call a point fixed under an isometry if it is mapped to itself by that isometry (section \ref{odd2AKSSKL}). Intuitively, a rotation has exactly one fixed point – its center. A translation has no fixed points. A reflection over a line has infinitely many, but they are all fixed points on the axis of that reflection.
Is it possible for an isometry to have three non-collinear fixed points? Intuitively, it is clear (and we have also formally proven it in Theorem \ref{IizrekABCident}) that all other points in the plane are also fixed under such an isometry. We have called such an isometry an \index{isometry!identical} identical isometry or the identity, and we have denoted it with $\mathcal{E}$. Obviously, the identity is also a direct isometry, since it preserves all figures.
Because of its importance, we will write Theorem \ref{IizrekABCident} once again, but in a slightly different form.

\bizrek \label{IizrekABC2}
The identity map $\mathcal{E}$ is the only isometry of a plane having three non-collinear fixed points.
 \eizrek

With the following theorem, we will provide another criterion for the identity.

\bizrek \label{izo2ftIdent}
A direct isometry of a plane that has at least two fixed points is the identity map.
\eizrek

\begin{figure}[!htb]
\centering
\input{sl.izo.6.1.2.pic}
\caption{} \label{sl.izo.6.1.2.pic}
\end{figure}

\textbf{\textit{Proof.}}
  We assume that the direct isometry
plane $\mathcal{I}$ has at least two fixed points $A$ and $B$ or
$A'=\mathcal{I}(A)=A$ and $B'=\mathcal{I}(B)=B$ (Figure
\ref{sl.izo.6.1.2.pic}). Let $X$ be an arbitrary point that does not lie on the line $AB$. Let $X'=\mathcal{I}(X)$. Because $\mathcal{I}$
is a direct isometry, the triangles $AXB$ and $AX'B$ are oriented the same, so the point $X'$ lies in the plane with the edge $AB$ and
with the point $X$. From $\mathcal{I}:A,B,X\mapsto A,B,X'$ it follows
 $XA\cong X'A$, $XB\cong X'B$ and $AB\cong AB$. Because
also $XA\cong XA$ and $XB\cong XB$,  by
 \ref{izomEnaC'} $X=X'$. Therefore, $X$ is also a fixed point. Because $A$, $B$ and $X$ are three
 non-linear
 points of the isometry  $\mathcal{I}$, $\mathcal{I}=\mathcal{E}$
 (statement \ref{IizrekABC2}).
 \kdokaz

In the following we will formally introduce and consider different
types of isometries.

%________________________________________________________________________________
 \poglavje{Reflections} \label{odd6OsnZrc}

Although the basic reflection is an intuitively already known transformation, we will
first give a formal definition.

Let $s$ be a line of some plane. The transformation of this plane, in which each point of the line $s$ is fixed and in which each point $X$, which does not lie on the line $s$, is mapped to such a point $X'$,
 that  $s$ is the perpendicular bisector of the segment $XX'$, is called
  \index{reflection!over a line} \pojem{basic reflection}
  or \index{reflection!basic}\pojem{reflection over a line}
  $s$ and is denoted by $\mathcal{S}_s$ (Figure \ref{sl.izo.6.2.1.pic}).
   The line $s$ is
 \index{axis!of reflections} \pojem{the axis of reflections}.

 Because  from the notation $\mathcal{S}_s$
 it is already clear that it is a reflection over the line $s$, we will call it
 shorter: reflection $\mathcal{S}_s$.

\begin{figure}[!htb]
\centering
\input{sl.izo.6.2.1.pic}
\caption{} \label{sl.izo.6.2.1.pic}
\end{figure}

If the figure $\phi$ is mapped onto the figure $\phi'$ by the reflection $\mathcal{S}_s$, or $\mathcal{S}_s: \phi \rightarrow \phi'$, we will say that the figures $\phi$ and $\phi'$ are \pojem{symmetric} with respect to the axis $s$ (Figure \ref{sl.izo.6.2.2.pic}). The axis $s$ is the \pojem{axis of symmetry} of the figures $\phi$ and $\phi'$.

 If $\phi=\phi'$ or $\mathcal{S}_s: \phi \rightarrow \phi$, we will say that the figure $\phi$ is \index{figure!axis-symmetric} \pojem{axis-symmetric} or \pojem{axis-centered}. The line $s$ is the \index{axis!symmetry of a figure}\pojem{axis of symmetry} or \index{centered line} \pojem{centered line} of this figure (Figure \ref{sl.izo.6.2.2.pic}).

\begin{figure}[!htb]
\centering
\input{sl.izo.6.2.2.pic}
\caption{} \label{sl.izo.6.2.2.pic}
\end{figure}

 From the definition it is already clear that all the fixed points of the reflection $\mathcal{S}_s$ lie on the axis $s$ of this reflection.

 Although it is intuitively clear, we need to prove the following property of the defined mapping.



    \bizrek \label{izozrIndIzo}
    A reflection is an opposite isometry.
    \eizrek

\begin{figure}[!htb]
\centering
\input{sl.izo.6.2.3.pic}
\caption{} \label{sl.izo.6.2.3.pic}
\end{figure}


\textbf{\textit{Proof.}}
  Let $\mathcal{S}_s$ be a reflection over the line $s$ (Figure
\ref{sl.izo.6.2.3.pic}). From the definition it is clear that it represents a bijective mapping. It remains to
  prove that for any points
   $X$ and $Y$, which are mapped by $\mathcal{S}_s$ into the points  $X'$ and $Y'$, it holds that
   $XY\cong X'Y'$. We will consider several cases.

   If $X=X'$ and $Y=Y'$, this relation
    is automatically fulfilled (statement \ref{sklRelEkv}).

   Let $X\neq
   X'$. We will mark with $X_s$ the center of the line segment $XX'$. Because $s$ is the symmetry of
   the line segment $XX'$, by definition $X_s\in s$ and $XX'\perp s$.
   If $Y=Y'$, the triangles $XX_sY$ and $X'X_sY$ (or $X'X_sY'$) are
    congruent (statement
   \textit{SAS} \ref{SKS}), therefore $XY\cong X'Y'$.

We will now look at an example of $X\neq X'$ and $Y\neq Y'$. Similarly, for
   the center
   $Y_s$ of the line $YY'$ it holds that $Y_s\in s$ and $YY'\perp s$.
   The triangles $XX_sY_s$ and $X'X_sY_s$
    are
    congruent (by the
   theorem
   \textit{SAS}, so there exists an isometry $\mathcal{I}$, such that
   $\mathcal{I}: X, X_s,Y_s\mapsto X', X_s,Y_s$. This
   maps the line segment $X_sY_sX$ onto the line segment $X_sY_sX'$
   (by axiom \ref{aksIII2}). Because  the isometry $\mathcal{I}$ preserves angles, the line segment $Y_sY$ is mapped onto the line segment $Y_sY'$,
   the point $Y$ is mapped onto the point $Y'$. From
   $\mathcal{I}: X, Y\mapsto X', Y'$ it follows at the end that $XY\cong X'Y'$.

    Let $A,B\in s$ and $C\notin s$. Then $\mathcal{S}_s(A)=A$, $\mathcal{S}_s(B)=A$ and $\mathcal{S}_s(C)=C'\neq C$. This means that the triangle $ABC$ is mapped onto the triangle $ABC'$ by the basic reflection $\mathcal{S}_s$. Because $C,C' \div s$ or $C,C' \div AB$, the two triangles are differently oriented, so the basic reflection $\mathcal{S}_s$ is an indirect isometry.
    \kdokaz

It is clear that in the proof of the previous theorem it also holds that
$\mathcal{I}=\mathcal{S}_s$. We can only determine this at the end, when
we prove that $\mathcal{S}_s$ is an isometry and use theorem
\ref{IizrekABC}.

We will now prove some simple properties of reflections across a line.


    \bizrek \label{izoZrcPrInvol}
     For an arbitrary line $p$ it holds that $$\mathcal{S}_p^2=\mathcal{E}\hspace*{1mm}\textrm{ i.e. }
    \hspace*{1mm}\mathcal{S}_p^{-1}=\mathcal{S}_p.$$
    \eizrek


\textbf{\textit{Proof.}} It is enough to prove that $\mathcal{S}_p^2(X)=X$ for every point $X$ of the plane. Let
$\mathcal{S}_p(X)=X'$. If $X\in p$ or $X=X'$, the relation
$\mathcal{S}_p^2(X)=X$ is automatically fulfilled. If $X\notin p$, by definition the line $p$ is the perpendicular bisector of the line segments $XX'$ (and also $X'X$),
so $\mathcal{S}_p(X')=X$ or $\mathcal{S}_p^2(X)=X$.
 \kdokaz

Every isometry (as well as every mapping) $f:\mathbb{E}^2\rightarrow \mathbb{E}^2$,
    for which $f^2=\mathcal{E}$, is called
     \pojem{involution}\index{involution}.
     The basic reflection is therefore an involution.



        \bizrek \label{zrcFiksKroz}
         Let $l$ be  an arbitrary circle
        with the centre $S$ and $p$ a line in the plane.
         If $S\in p$, then $\mathcal{S}_p(l)=l$.
        \eizrek


\begin{figure}[!htb]
\centering
\input{sl.izo.6.2.4.pic}
\caption{} \label{sl.izo.6.2.4.pic}
\end{figure}


\textbf{\textit{Proof.}} Let $X\in l$ be an arbitrary point of the
circle $l$ and $\mathcal{S}_p(X)=X'$ (Figure \ref{sl.izo.6.2.4.pic}).
Because $\mathcal{S}_p:S,X\mapsto S, X'$, we have $SX\cong SX'$ or
$X'\in l$. Therefore $\mathcal{S}_p(l)\subseteq l$. Similarly, an
arbitrary point $Y$ of the circle $l$ is the image of the point
$Y'=\mathcal{S}_p^{-1}(Y)=\mathcal{S}_p(Y)$, which lies on this
circle. Therefore $\mathcal{S}_p(l)\supseteq l$. Therefore
$\mathcal{S}_p(l)=l$.
 \kdokaz

        \bzgled \label{izoSimVekt}
        If
        $\mathcal{S}_s:A, B\mapsto A', B'$, the vector
        $\overrightarrow{v}=\overrightarrow{AB}+\overrightarrow{A'B'}$
        is parallel to the line $s$.
        \ezgled

\begin{figure}[!htb]
\centering
\input{sl.izo.6.2.5.pic}
\caption{} \label{sl.izo.6.2.5.pic}
\end{figure}


\textbf{\textit{Proof.}} Let $A_s$ and $B_s$ be the centres of the
lines $AA'$ and $BB'$ (Figure \ref{sl.izo.6.2.5.pic}). Because $p$
is the symmetry of these lines, $A_s, B_s\in p$. Therefore the
vector
 \begin{eqnarray*}
 \overrightarrow{v}&=&\overrightarrow{AB}+\overrightarrow{A'B'}=\\
 &=&(\overrightarrow{AA_s}+\overrightarrow{A_sB_s}+\overrightarrow{B_sB})+
 (\overrightarrow{A'A_s}+\overrightarrow{A_sB_s}+\overrightarrow{B_sB'})=\\
 &=& 2\overrightarrow{A_sB_s}
 \end{eqnarray*}
is parallel to the line $s$.
 \kdokaz

We will continue with the properties of the basic reflection in the next section,
  but now let's look at the use of this isometry.



             \bzgled \label{HeronProbl}
             (Heron's\footnote{This problem was
            posed by \index{Heron}\textit{Heron of Alexandria} (20--100). In
            his work 'Catoprica' he formulated a law that says that a beam that goes from
            point $A$ and is reflected from line $p$ through point $B$, passes
            the shortest possible path.} problem) \index{problem!Heron's}
             Two points $A$ and $B$ are given on the same side of a line $p$.
             Find a point $X$ on the line $p$ such that the sum $|AX|+|XB|$ is minimal.
            \ezgled

\begin{figure}[!htb]
\centering
\input{sl.izo.6.2.6.pic}
\caption{} \label{sl.izo.6.2.6.pic}
\end{figure}


\textbf{\textit{Solution.}}
 Let $A'$ be the image of point $A$ under
reflection $\mathcal{S}_p$ (Figure \ref{sl.izo.6.2.6.pic}). With $X$
we mark the intersection of line $p$ and $A'B$ (points $A'$ and $B$ are on
different sides of line $p$). We will prove that $X$ is the desired point.

Let $Y\neq X$ be an arbitrary point on line $p$. Because reflection
$\mathcal{S}_p$ is an isometry that maps line segment $AX$ to line segment $A'X$ and line segment $AY$ to line segment $A'Y$, we have $AX\cong A'X$ and $AY\cong A'Y$. If
we also use the triangle inequality - \ref{neenaktrik} (for triangle $A'YB$), we get: $$|AX| + |XB| = |A'X| + |XB| = |A'B| <
|A'Y| + |YB| = |AY| + |YB|,$$ which was to be proven. \kdokaz



          \bzgled
           Let $k$ and $l$ be circles on the same side of a line $p$ in the same plane.
             Construct a point $S$ on the line $p$ such that
            the tangents from this point to the circles $k$ and $l$
            determine the congruent angles with the line $p$.
              \ezgled

\begin{figure}[!htb]
\centering
\input{sl.izo.6.2.7.pic}
\caption{} \label{sl.izo.6.2.7.pic}
\end{figure}

\textbf{\textit{Solution.}} Let $q$ and $r$ be the tangents in that order of the circles $k$ and $l$, which intersect on the line $p$ in the point $S$ and with it determine the corresponding angle (Figure \ref{sl.izo.6.2.7.pic}). The line $p$ is the angle bisector of the angle determined by the tangents $q$ and $r$, therefore $\mathcal{S}_p(r)=q$. The line $q$ is therefore also a tangent of the circle $l'=\mathcal{S}_p(l)$. This means that we can plan the line $q$ as the common tangent of the circles $k$ and $l'$ (see example \ref{tang2ehkroz}). Then $S=q\cap p$ and $r=\mathcal{S}_p(q)$.
 \kdokaz


            \bzgled \label{FagnanLema}
            Let $AP$, $BQ$ and $CR$ be the altitudes of a triangle $ABC$.
            If $P'=\mathcal{S}_{AB}(P)$ and
             $P''=\mathcal{S}_{AC}(P)$, prove that $P'$, $R$, $Q$ and $P''$
             are collinear points.
             \ezgled

\begin{figure}[!htb]
\centering
\input{sl.izo.6.2.9a.pic}
\caption{} \label{sl.izo.6.2.9a.pic}
\end{figure}

\textbf{\textit{Solution.}} (Figure \ref{sl.izo.6.2.9a.pic}) First,
from $\angle BRC=90^0$ and $\angle BQC=90^0$ by Tales' theorem
\ref{TalesovIzrKroz} it follows that points $R$ and $Q$ lie on a
circle with diameter $BC$. Therefore $BRQC$ is a chordal quadrilateral
and by theorem \ref{TetivniPogojZunanji} it holds that $\angle
ARQ\cong \angle BCQ = \gamma$. Similarly (from the chordality of
$CARP$ quadrilateral) it also holds that $\angle BRP\cong\gamma$, so
$\angle ARQ\cong \angle BRP$. Because $\mathcal{S}_{AB}:P, R, B\mapsto
P', R, B$, it holds that $\angle BRP\cong \angle BRP'$. From the
previous relations it follows that $\angle ARQ\cong \angle BRP'$.
Triangle $ABC$ is acute, which means that points $P$ and $Q$ are on
the same side of line $AB$. From this it follows that points $Q$ and
$P'$ are on different sides of line $AB$. From the proven relation
$\angle ARQ\cong \angle BRP'$ it now follows that $P'$, $R$ and $Q$
are collinear points. Similarly, points $P''$, $R$ and $Q$ are
collinear, which means that all four points $P'$, $R$, $Q$ and $P''$
lie on the same line.
 \kdokaz



             \bizrek \label{FagnanLema1}
            Let $A$ be a point not lying on lines $p$ and $q$ in the same plane.
             Construct points $B\in p$ and $C\in q$ such that
             the perimeter of the triangle $ABC$ is minimal.
             \eizrek

\begin{figure}[!htb]
\centering
\input{sl.izo.6.2.8.pic}
\caption{} \label{sl.izo.6.2.8.pic}
\end{figure}


\textbf{\textit{Solution.}}
 Let $A'=\mathcal{S}_p(A)$, $A''=\mathcal{S}_q(A)$ and $B$ and $C$
  be the intersection points of line $A'A''$ with lines $p$ and $q$
   (Figure \ref{sl.izo.6.2.8.pic}).

We will prove that $ABC$ is the desired triangle. If $B_1$ and $C_1$ are any points (where $B_1\neq B$ or $C_1\neq C$) that lie on the lines $p$ and $q$, the perimeter of the triangle $AB_1C_1$ is equal to the length of the broken line $A'B_1C_1A''$ ($AB_1\cong A'B_1$ and $AC_1\cong A'' C_1$), and the perimeter of the triangle $ABC$ is equal to the length of the line segment $A'A''$ ($AB\cong A'B$ and $AC\cong A'' C$). From Theorem \ref{neenakIzlLin} it follows that the first length is greater than the second, so the perimeter of the triangle $ABC$ is less than the perimeter of the triangle $AB_1C_1$.
 \kdokaz


            \bzgled
            For a given acute triangle $ABC$ determine the inscribed triangle of minimal perimeter
            \index{problem!Fagnano}(Fagnano's problem\footnote{\textit{G. F. Fagnano}
            \index{Fagnano, G. F.} (1717--1797), Italian mathematician,
             posed this problem in 1775 and solved
            it
            with the methods
            of differential calculus. A more
            elementary solution to this problem was later given by the Hungarian
            mathematician \index{Fejer, L.} \textit{L. Fejer} (1880--1959).}).
            \ezgled

\begin{figure}[!htb]
\centering
\input{sl.izo.6.2.9.pic}
\caption{} \label{sl.izo.6.2.9.pic}
\end{figure}


\textbf{\textit{Solution.}} We mark for $AP$, $BQ$ and $CR$ the altitudes of the triangle $ABC$ (Figure \ref{sl.izo.6.2.9.pic}).
 If $X$
is any point on the side $BC$, then the minimal perimeter of the triangle
$XYZ$, where the vertices $Y$ and $Z$ lie on the sides $AC$ and $AB$,
is obtained in the way described in the previous Theorem
\ref{FagnanLema1}. The points $Y$ and $Z$ are therefore the intersections of the sides
$AC$ and $AB$ with the line $X'X''$ ($X'=\mathcal{S}_{AB}(X)$,
$X''=\mathcal{S}_{AC}(X)$). In this case, the perimeter of the triangle $XYZ$ is equal
to the length of the line segment $X'X''$.

So it remains for us to determine for which point $X$ of side
$BC$ is the perimeter of triangle $XYZ$ the smallest or the distance $X'X''$
the shortest. From the properties of reflection it follows that $AX'\cong AX\cong AX''$,
$\angle X'AB\cong \angle BAX$ and $\angle X''AC\cong \angle CAX$
or $\angle X'AX''=2\angle BAC$. As $X'AX''$ is an isosceles triangle $X'AX''$ is constant (depending on the point $X$), therefore
its base $X'X''$ is the shortest when its shortest leg $AX'$ (by \ref{SkladTrikLema}) or the distance
$AX\cong AX'\cong AX''$. This is fulfilled when $X$ is the foot of the altitude of triangle $ABC$ from point $A$ or $X=P$. From \ref{FagnanLema} it follows that $Y=Q$ and $Z=R$. So the inscribed triangle with the smallest perimeter is actually the pedal triangle of triangle
$ABC$.
 \kdokaz



            \bzgled
            Let $P$ and $Q$ be interior points of an angle $aOb$.
            Construct a point $X$ on the side $a$ of this angle such that
            the rays $XP$ and $XQ$ intersect the side $b$ at points $Y$ and $Z$ such that $XY\cong XZ$.
            \ezgled

\begin{figure}[!htb]
\centering
\input{sl.izo.6.2.10.pic}
\caption{} \label{sl.izo.6.2.10.pic}
\end{figure}

\textbf{\textit{Solution.}}
 Let $X$, $Y$ and $Z$ be the sought points (Figure \ref{sl.izo.6.2.10.pic}).
 Let $S$ be the center of the segment $YZ$ and $P'=\mathcal{S}_a(P)$.
 The measure of the angle
$aOb$ we denote by $\omega$. The triangle $XYZ$ is equilateral, therefore the
circumcenter $XS$ is at the same time the altitude and the symmetry of the internal angle $XYZ$
 of this
triangle. So it holds that $\angle OSX=90^0$ and $\angle YXS\cong
\angle ZXS$. From $\mathcal{S}_a:P, X, O\mapsto P', X, O$ it follows
that $\angle PXO\cong \angle P'XO$. Now we can calculate the measure of the angle
$P'XQ$:
 \begin{eqnarray*}
\angle P' XQ &=& \angle P' XP + \angle PXQ = 2\angle OXP + 2\angle
PXS\\ &=& 2\angle OXS = 2(90° - \omega) =180° - 2\omega.
 \end{eqnarray*}
Since the points $P'$ and $Q$ are known to us, the point $X$ is the
intersection of the segment $a$ and the arc with the chord $P'Q$ and the central angle $180°
- 2\omega$ (the theorem \ref{ObodKotGMT}).
 \kdokaz



             \bzgled
             Let $ABCD$ be a rectangle such that $|AB|=3|BC|$. Suppose that $E$ and $F$ are
              points on the side $AB$ such that
               $AE\cong EF\cong FB$.
               Prove $$\angle AED+\angle AFD+\angle
            ABD=90^0.$$
            \ezgled

\begin{figure}[!htb]
\centering
\input{sl.izo.6.2.11.pic}
\caption{} \label{sl.izo.6.2.11.pic}
\end{figure}

\textbf{\textit{Solution.}} Let $F'=\mathcal{S}_{CD}(F)$ and
$B'=\mathcal{S}_{CD}(B)$
  (Figure \ref{sl.izo.6.2.11.pic}).
 Then $DF'\cong DF$ and $\angle DF'F\cong \angle DFF'$. From the similarity of
triangles $DAF$ and $F'B'B$ (the \textit{SAS} theorem \ref{SKS}) it follows that
$DF\cong F'B$ and $\angle DFA\cong \angle B'BF'$. It is also true that
$\angle FF'B\cong \angle B'BF'$ (the \ref{KotiTransverzala} theorem).
Therefore:
 $$\angle DF'B=\angle DF'F+\angle FF'B=\angle DFF'+\angle DFA=\angle AFF'=90^0.$$
  Because $DF'\cong F'B$, $DF'B$ is an isosceles right triangle with the hypotenuse $BD$, therefore (the \ref{enakokraki} theorem) $\angle DBF'=\angle BDF'=45^0$. Because $DAE$ is also an isosceles right triangle with the hypotenuse $DE$, $\angle AED=45^0$ or $\angle AED=\angle DBF'$. Therefore:
$$\angle AED+\angle AFD+\angle ABD=\angle DBF'+\angle F'BB'+
\angle ABD=\angle ABC=90^0,$$ which was to be proven. \kdokaz

 At the next example we will describe the standard procedure of using
  isometries in design tasks. We assume that it is necessary to construct points $X\in k$ and $Y\in l$, where $k$ and $l$ are given circles
 (it can also be lines or a circle and a line), but we know that for some
  isometry $\mathcal{I}$ it is true that  $\mathcal{I}(X)=Y$. In this case
  from $X\in k$ it follows that $Y=\mathcal{I}(X)\in \mathcal{I}(k)$. Because
 $Y$ is also in $l$, we can construct the point $Y$ from the condition $Y\in
 l\cap \mathcal{I}(k)$.



            \bzgled
            Two circles $k$ and $l$ on different sides of a line $p$ are given.
            Construct a square $ABCD$ such that $A\in k$, $C\in l$ and $B,D\in p$.
            \ezgled

\begin{figure}[!htb]
\centering
\input{sl.izo.6.2.12.pic}
\caption{} \label{sl.izo.6.2.12.pic}
\end{figure}


\textbf{\textit{Solution.}}
  (Figure \ref{sl.izo.6.2.12.pic})

Let $ABCD$ be the sought square. Then $\mathcal{S}_p(C)=A$. The point $C$ lies on the circle $l$, therefore its image -- point $A$ -- after reflection $\mathcal{S}_p$ lies on the image of the circle $l$ -- circle $l'$, which we can plot. But the point $A$ also lies on the circle $k$, therefore $A\in l'\cap k$. In this way, we first find the vertex $A$, and then $C=\mathcal{S}_p(A)$. The center of the square $S$ is found as the center of the diagonal $AC$. In the end, the vertices $B$ and $D$ are the intersections of the line $p$ and the circle with center $S$ and radius $SA$.

The number of solutions depends on the number of intersections of the circles $l'$ and $k$. \kdokaz



        \bnaloga\footnote{38. IMO Argentina - 1997, Problem 2.}
        The angle at $A$ is the smallest angle of triangle $ABC$. The points $B$ and $C$
        divide the circumcircle of the triangle into two arcs. Let $U$ be an interior point
        of the arc between $B$ and $C$ which does not contain $A$. The perpendicular
        bisectors of $AB$ and $AC$ meet the line $AU$ at $V$ and $W$, respectively. The lines
        $BV$ and $CW$ meet at $T$. Show that $$|AU| =|TB| + |TC|.$$
        \enaloga

\begin{figure}[!htb]
\centering
\input{sl.izo.6.2.IMO1.pic}
\caption{} \label{sl.izo.6.2.IMO1.pic}
\end{figure}

\textbf{\textit{Solution.}} Let's mark with $q$ and $p$ the
 sides $AB$ and $AC$ of the triangle $ABC$
 (Figure \ref{sl.izo.6.2.IMO1.pic}). Their intersection -- point
 $O$ -- is the center of the circumscribed circle $l$ of this triangle. According to
 the assumption, $V=AU\cap q$ and $W=AU\cap p$. Let $D$ be the other
 intersection of the line segment $CW$ with the circle $l$. With $\mathcal{S}_p$
 we mark the reflection over
 the line $p$. $\mathcal{S}_p$ maps the points $A$, $C$, $W$ and $O$
 in order, into the points $C$, $A$, $W$ and $O$ and the line segment $AW$
 into the line segment $CW$. Because $O\in p$, by the theorem \ref{zrcFiksKroz}
 $\mathcal{S}_p(l) =l$. From this follows:
  $$\mathcal{S}_p(U) =\mathcal{S}_p(AW\cap l)=
  \mathcal{S}_p(AW)\cap \mathcal{S}_p(l)=
  CW\cap l=D.$$
   Therefore $\mathcal{S}_p:\hspace*{1mm} A,U\mapsto C,D$, so $AU\cong CD$ and $AD\cong CU$. From the similarity of the line segments $AD$
   and $CU$ follows the similarity of the corresponding angles
   (theorem \ref{SklTetSklObKot}) or equivalently
   $\angle ABD\cong\angle UAC$. But the angles $BDC$ and $BAC$ are also similar over the line segment $BC$ (theorem\ref{ObodObodKot}).

   Because the point $V$ lies on the line $q$, which is the symmetry of the line segment $AB$,
   $\mathcal{S}_q:\hspace*{1mm} A,B,V\mapsto B,A,V$. This means that
   $\angle BAV\cong\angle ABV$.

   From the similarity of the corresponding angles it follows:
    \begin{eqnarray*} \angle BDT&=&\angle BDC\cong\angle BAC=\\
    &=&\angle BAU+
   \angle UAC=\\
   &=&\angle ABV+\angle ABD=\\
    &=&\angle DBT.
   \end{eqnarray*}
 Therefore the triangle $DTB$ is isosceles and by the theorem \ref{enakokraki}
 it holds that $TD\cong TB$. In the end we get:
  $$|AU|= |CD|=|CT|+|TD|=|CT|+|TB|,$$ which was to be proven. \kdokaz

\bnaloga  \footnote{50. IMO Germany - 2009, Problem 4.}
Let $ABC$ be a triangle with $|AB|=|AC|$. The angle bisectors of $\angle CAB$ and $\angle ABC$
meet the sides $BC$ and $CA$ at $D$ and $E$, respectively. Let $K$ be the incentre of triangle $ADC$.
Suppose that $\angle BEK=45^0$. Find all possible values of $\angle CAB$.
\enaloga



 \textbf{\textit{Solution.}} We denote by $S$ the centre of the inscribed circle
  of triangle $ABC$.
  This means that $S$ is the intersection of the altitudes $AD$ and $BE$
  of angles $\alpha=\angle CAB$ and $\beta=\angle ABC$. The point $S$ also lies on
  the altitude
$CF$ ($F\in AB$) of angle $\gamma=\angle ACB$. From $|AB|=|AC|$ it follows
$\beta\cong\gamma$, therefore $\angle SBC\cong\angle SCB$.

By assumption, the point $K$ is the centre of the inscribed circle
  of triangle $ADC$, therefore the point $K$ lies on the altitude $CS$ of
  angle $\angle ACD=\angle ACB$. The segment $DK$ is the altitude of
  angle $ADC$. From the similarity of triangles $ADB$ and $ADC$ (the \textit{SAS} theorem \ref{SKS})
  it follows that $\angle
  ADC\cong\angle ADB=90^0$. From this it follows that $\angle SDK=45^0$.

Let $E'=\mathcal{S}_{SC}(E)$. Because $CS$ is the altitude of angle $ACB$,
the segment $CA$ is mapped onto the segment $CB$ by the reflection $\mathcal{S}_{SC}$. Therefore $E'\in CB$. Because $S,K,C\in SC$, we have
$\mathcal{S}_{SC}:\hspace*{1mm}S,K,C\mapsto S,K,C$.


\begin{figure}[!htb]
\centering
\input{sl.izo.6.2.IMO2a.pic}
\caption{} \label{sl.izo.6.2.IMO2a.pic}
\end{figure}

We will consider two cases. If $E'=D$ (Figure
\ref{sl.izo.6.2.IMO2a.pic}), then $\angle SEC\cong \angle SDC=90^0$.
Triangles $ABE$ and $CBE$ are similar in this case (the \textit{ASA}
theorem \ref{KSK}), therefore $AB\cong BC$. Triangle $ABC$ is
isosceles or $\angle BAC=60^0$.


\begin{figure}[!htb]
\centering
\input{sl.izo.6.2.IMO2.pic}
\caption{} \label{sl.izo.6.2.IMO2.pic}
\end{figure}

Let $E'\neq D$ (Figure \ref{sl.izo.6.2.IMO2.pic}). From
$\mathcal{S}_{SC}:\hspace*{1mm}S,K\mapsto S,K$ it follows that $\angle SE'K
\cong \angle SEK=45^0$. Because $\angle SDK \cong \angle
SE'K=45^0$, by izreku \ref{ObodKotGMT} the points $S$, $K$, $D$ and
$E'$ are concircle. From izreku \ref{TalesovIzrKroz} it then follows
that $\angle SKE'\cong \angle SDE'=90^0$. Because the reflection
$\mathcal{S}_{SC}$ maps the angle $SKE$ into the angle $SKE'$, $\angle
SKE\cong \angle SKE'=90^0$ (the points $E$, $K$ and $E'$ are
therefore collinear). From this and izreku \ref{VsotKotTrik} for the triangle
$SKE$ it follows that $\angle KSE=45^0$. It is the outer angle of the triangle
$SBC$, so by izreku \ref{zunanjiNotrNotr} $\angle SBC +\angle
SCB=45^0$ or $\beta+\gamma=90^0$. From this relation it follows that
$\angle BAC=90^0$.

 The possible values of the angle
                $\angle CAB$ are therefore $60^0$ and $90^0$.
\kdokaz


%________________________________________________________________________________
 \poglavje{More on Reflections} \label{odd6Sopi}

In the initial research on isometries, we found that every direct isometry with at least two fixed points represents the identity. It is intuitively clear that an opposite isometry of the plane with at least one fixed point is a reflection. We will prove an even stronger statement.



            \bizrek \label{izo1ftIndZrc}
            An opposite isometry of the plane with at least one fixed point is a reflection.
            The axis of this reflection passes through this point.
            \eizrek

\begin{figure}[!htb]
\centering
\input{sl.izo.6.3.1.pic}
\caption{} \label{sl.izo.6.3.1.pic}
\end{figure}

 \textbf{\textit{Proof.}}
 (Figure \ref{sl.izo.6.3.1.pic})

Let $A$ be a fixed point of an indirect isometry $\mathcal{I}$ of
some plane. Because it is indirect, the isometry $\mathcal{I}$ is
not the identity, so there is such a point $B$ of this plane that
$\mathcal{I}(B)=B'\neq B$. Let $p$ be the perpendicular bisector of
the segment $BB'$. Therefore, $\mathcal{I}$ is an isometry that
maps the point $A$ and $B$ to the point $A$ and $B'$, so $AB\cong
AB'$. This means that the point $A$ lies on the perpendicular
bisector $p$ of the segment $BB'$. We prove that the isometry
$\mathcal{I}$ represents the reflection over the line $p$ or
$\mathcal{I} = \mathcal{S}_p$. The composition $\mathcal{S}_p \circ
\mathcal{I}$ (of two indirect isometries) is a direct isometry
with two fixed points $A$ and $B$, so based on the aforementioned
formula \ref{izo2ftIdent} it represents the identity. Therefore,
$\mathcal{S}_p \circ \mathcal{I}=\mathcal{E}$. From this it follows
that $ \mathcal{I}=\mathcal{S}_p^{-1} \circ\mathcal{E}$ or
$\mathcal{I} = \mathcal{S}_p$.

The following formula is very useful.

 \bizrek \label{izoTransmutacija}
 Let $\mathcal{I}$ be an isometry and $p$ an arbitrary line in the plane.
 Suppose that this isometry maps the line $p$ to a line $p'$. If $\mathcal{S}_p$ is
  a reflection with axis $p$, then
 $$\mathcal{I}\circ \mathcal{S}_p\circ \mathcal{I}^{-1} = \mathcal{S}_{p'}.$$
  \eizrek

\begin{figure}[!htb]
\centering
\input{sl.izo.6.3.2.pic}
\caption{} \label{sl.izo.6.3.2.pic}
\end{figure}

 \textbf{\textit{Proof.}}
 Let $Y$ be an arbitrary point of the line $p'$ and $X = \mathcal{I}^{-1}(Y)$
 (Figure \ref{sl.izo.6.3.2.pic}).

Since $\mathcal{I}(p)=p'$ or $\mathcal{I}^{-1}(p')=p$, the point $X$ lies on the line $p$ and is $\mathcal{S}_p(X)=X$. Therefore, it holds:
$$\mathcal{I}\circ\mathcal{S}_p\circ\mathcal{I}^{-1}(Y)=
 \mathcal{I}\circ\mathcal{S}_p(X)=
 \mathcal{I}(X)=Y.$$
 Thus, the composition $\mathcal{I}\circ\mathcal{S}_p\circ\mathcal{I}^{-1}$ is an indirect isometry with a fixed point $Y$, therefore, according to the previous statement \ref{izo1ftIndZrc}, it represents the central reflection. But $Y$ is an arbitrary point of the line $p'$, which means that the axis of this reflection is just the line $p'$ or $\mathcal{I}\circ \mathcal{S}_p\circ \mathcal{I}^{-1} = \mathcal{S}_{p'}$.
 \kdokaz

The transformation
$\mathcal{I}\circ\mathcal{S}_p\circ\mathcal{I}^{-1}$ from the previous
statement is called the \index{transmutation!reflection across a line}
\pojem{transmutation} of the reflection $\mathcal{S}_p$ with the isometry $\mathcal{I}$.

We prove a simple consequence of the previous statement.



             \bzgled \label{izoZrcKomut}
             Two  reflections in the plane commute if and only if
              their axis are perpendicular or coincident, i.e.
             $$\mathcal{S}_p \circ \mathcal{S}_q
            = \mathcal{S}_q \circ \mathcal{S}_p \Leftrightarrow
             (p\perp q \vee p=q).$$
                \ezgled

\begin{figure}[!htb]
\centering
\input{sl.izo.6.3.3.pic}
\caption{} \label{sl.izo.6.3.3.pic}
\end{figure}

 \textbf{\textit{Proof.}}
 (Figure \ref{sl.izo.6.3.3.pic})

First, $\mathcal{S}_p \circ \mathcal{S}_q
= \mathcal{S}_q \circ \mathcal{S}_p
\Leftrightarrow  \mathcal{S}_p=  \mathcal{S}_q \circ
 \mathcal{S}_p \circ \mathcal{S}_q^{-1}$. If we use the previous 
statement
  \ref{izoTransmutacija} for $\mathcal{I} = \mathcal{S}_q$,
we get that the last equality is equivalent to $\mathcal{S}_p =
\mathcal{S}_{p'}$, where $p' = \mathcal{S}_q(p)$. This is exactly the case
when $p=p'=\mathcal{S}_q(p)$ i.e. when $p\perp q$ or $p=q$.
 \kdokaz

The composite of two basic reflections $\mathcal{I}=\mathcal{S}_p \circ
\mathcal{S}_q$ will be very important in further research
isometries. We immediately notice that the composite of two
indirect isometries represents a direct isometry.
If the axes of these reflections coincide, we have
already determined that it is the identity (statement \ref{izoZrcPrInvol}).
It is interesting to ask what $\mathcal{I}$ represents in the general case,
when the axes of $p$ and $q$ of these reflections are different and
coplanar; especially in the case when they intersect or are parallel.
We will find the answer to this question in the following sections. In
the next statement we will find out something about the fixed points of the composite
of two basic reflections.

        \bizrek \label{izoKomppqX}
        Let $p$ and $q$ be two distinct coplanar lines.
        Then
         $$\mathcal{S}_q\circ\mathcal{S}_p(X)=X \Leftrightarrow p\cap q=\{X\}.$$
         \eizrek

\textbf{\textit{Proof.}}

($\Leftarrow$) Trivial, because from $X\in p$ and $X\in q$ it follows
$\mathcal{S}_p(X)= \mathcal{S}_q(X)=X$ i.e.
$\mathcal{S}_q\circ\mathcal{S}_p(X)=X$.

($\Rightarrow$) Let $\mathcal{S}_q\circ\mathcal{S}_p(X)=X$ in
$\mathcal{S}_p(X)=X'$. From this it follows that $\mathcal{S}_q(X')=X$.
Assume that $X\neq X'$. In this case, according to the definition of
the axis of reflection $p$ and $q$ are the symmetrals of the line
segment $XX'$. But this is not possible, because the line segment has
one symmetral, while $p$ and $q$ are different by assumption.
Therefore, $X=X'$. From this it follows that
$\mathcal{S}_p(X)=\mathcal{S}_q(X)=X$, which means that $X\in p$ and
$X\in q$ or $X\in p\cap q$. Because $p$ and $q$ are different lines,
by Axiom \ref{AksI1} $p\cap q=\{X\}$.
 \kdokaz

 If the lines $p$ and $q$ intersect, from the previous statement it follows that the composite
  $\mathcal{S}_q\circ\mathcal{S}_p$
 has only one fixed point, which is their intersection point. If $p$ and
$q$ are parallel, the composite $\mathcal{S}_q\circ\mathcal{S}_p$
has no fixed points.

For further research of isometries, the following concept is very
characteristic.

The set of all lines of a plane that pass through one point $S$ of
this plane, is called the \index{šop!konkurentnih premic}
 \pojem{set of concurrent lines} (or \index{šop!eliptični}\pojem{elliptic set}) with center $S$ and is denoted by
 $\mathcal{X}_S$.

The set of all lines of a plane that are parallel to one line $s$ of
this plane, is called the \index{šop!vzporednic} \pojem{set of
parallel lines} (or \index{šop!parabolični}\pojem{parabolic set} or \index{snop premic}\pojem{set of lines}) and is denoted by
 $\mathcal{X}_s$.

\begin{figure}[!htb]
\centering
\input{sl.izo.6.3.4.pic}
\caption{} \label{sl.izo.6.3.4.pic}
\end{figure}



 In both cases we will talk about the \pojem{set of lines}
  (Figure \ref{sl.izo.6.3.4.pic}). If we will not specifically emphasize whether it is a set
  of concurrent lines or a set of parallel lines, we will denote this set
  only by $\mathcal{X}$.

In connection with the introduced concepts, the following statement
is valid.

\bizrek \label{izoSop}
Let $p$, $q$ and $r$ be lines in the plane.
The product $\mathcal{S}_r \circ \mathcal{S}_q \circ \mathcal{S}_p$
is a reflection if and only if
the lines $p$, $q$ and $r$ belong to the same family of lines.
The axis of the new reflection also belongs to this family.
\eizrek

\begin{figure}[!htb]
\centering
\input{sl.izo.6.3.5.pic}
\caption{} \label{sl.izo.6.3.5.pic}
\end{figure}

 \textbf{\textit{Proof.}}
 (Figure \ref{sl.izo.6.3.5.pic})

 ($\Leftarrow$) First, let us assume that the lines $p$, $q$ and $r$
 belong to the same family of lines $\mathcal{X}$. We will consider two
 cases.

\textit{1)} Let $\mathcal{X}$ be a family of concurrent lines with
a center $S$ or $p, q, r\in \mathcal{X}_S$. In this case,
the composition $\mathcal{I}=\mathcal{S}_r \circ \mathcal{S}_q \circ
\mathcal{S}_p$ is an indirect isometry with a fixed point $S$, so according to
the izorek \ref{izo1ftIndZrc} it represents a reflection $\mathcal{S}_t$ with an axis that goes through the point $S$ or $t\in
\mathcal{X}_S$.

\textit{2)} Let $\mathcal{X}$ be a set of parallel lines and $n$ any common
 rectangle of lines $p$, $q$ and $r$. In this case, the composite
  $\mathcal{I}=\mathcal{S}_r \circ \mathcal{S}_q \circ
\mathcal{S}_p$ is an indirect isometry  and it holds that $\mathcal{I}(n)=n$.
We will now prove that there exists
 a fixed point on line $n$ of
 this isometry. Let $A$ be any point
 on line
 $n$ and $\mathcal{I}(A)=A'$. If $A$ is not a fixed point of isometry
 $\mathcal{I}$, with $O$
 we denote the center of line $AA'$ and with $O'=\mathcal{I}(O)$.
  $\mathcal{I}$ is an indirect isometry, therefore vectors
   $AO$ and $A'O'$ are congruent and have different orientation on line $n$.
   From this and from the fact that $O$ is the center of line $AA'$, it follows that
 $\overrightarrow{A'O}=\overrightarrow{OA}=\overrightarrow{A'O'}$.
   Therefore $O=O'$ or $O$ is a fixed
 point of isometry $\mathcal{I}$, which represents some basic reflection $\mathcal{S}_t$ (statement \ref{izo1ftIndZrc}). In this case
  $\mathcal{S}_t(n)=\mathcal{I}(n)=n$. This means that $t=n$ or
  $t\perp n$. The first option is not possible, because from $\mathcal{S}_r
  \circ \mathcal{S}_q \circ
\mathcal{S}_p=\mathcal{S}_n$ it follows that $\mathcal{S}_q \circ
\mathcal{S}_p=\mathcal{S}_r\circ \mathcal{S}_n$. The last relation
is not possible, because lines $p$ and $q$ are parallel, lines $r$ and $n$
are perpendicular and they intersect (statement \ref{izoKomppqX}). Therefore $t\perp n$, which means that $t$ is also a line from the set of parallel lines $\mathcal{X}$.


 ($\Rightarrow$) We will now assume that
 $\mathcal{S}_r \circ \mathcal{S}_q \circ \mathcal{S}_p=\mathcal{S}_t$
  for some line $t$. In this case it holds that
 $\mathcal{S}_q \circ \mathcal{S}_p=\mathcal{S}_r
 \circ\mathcal{S}_t$. We will again consider two cases.

 \textit{1)} We will assume that $p$ and $q$ intersect in some point $S$.
 In this case, it holds that $\mathcal{S}_r
 \circ\mathcal{S}_t(S) = \mathcal{S}_q \circ \mathcal{S}_p(S)=S$. According to
  statement \ref{izoKomppqX}, lines $r$ and $t$ also intersect in point $S$, therefore $p, q, r, t\in \mathcal{X}_S$.

\textit{2)} Let's assume that the lines $p$ and $q$ are parallel.
  From the aforementioned equation \ref{izoKomppqX} it follows that
  the composite  $\mathcal{S}_q \circ \mathcal{S}_p$ has no fixed points.
  Then neither does the composite $\mathcal{S}_r \circ \mathcal{S}_t$,
  so $r\parallel t$. Let $n$
  be the common perpendicular of the lines $p$ and $q$. Then $\mathcal{S}_r
 \circ\mathcal{S}_t(n) = \mathcal{S}_q \circ \mathcal{S}_p(n)=n$.
  From this it follows that $\mathcal{S}_t(n)=\mathcal{S}_r(n)=n'$.
  If $n\neq n'$, then both parallels $p$ and $q$ are axes
  of symmetry of the lines $n$ and $n'$, which is not possible. Therefore $n=n'$ and
   $\mathcal{S}_t(n)=\mathcal{S}_r(n)=n$.
  So $r,t\perp n$, which means that the lines $p$, $q$, $r$ and
  $t$
  belong to the family of lines that are all perpendicular to the line $n$.
 \kdokaz

The consequence of the proven statement is the following equation.



        \bizrek \label{izoSop2n+1}
        The product of an odd number of reflections with the axes
        from the same family of lines is also a reflection.
        \eizrek

 \textbf{\textit{Proof.}} We will carry out the proof by induction on
 the number $n\geq 1$, where $m=2n+1$ (odd) is the number of lines.

 For $n=1$ or $m=3$, the statement is a direct consequence of the previous equation
 \ref{izoSop}.

 We assume that for every $k\in \mathbb{N}$ ($k\geq 1$)
  and every $(2 k+1)$-tuple
 of lines
 $p_1,p_2,\ldots ,p_{2k+1}$, which are all from the same family, the composite
 $\mathcal{S}_{p_{2k+1}}\circ
 \cdots \circ\mathcal{S}_{p_2}\circ\mathcal{S}_{p_1}$ represents
 some basic reflection $\mathcal{S}_p$ (inductive assumption).

It is necessary to prove that then for $k+1$
and every $(2(k+1)+1)$-terica
of lines
$p_1,p_2,\ldots ,p_{2(k+1)+1}$, which are all from the same family of lines, the composition
$\mathcal{I}=\mathcal{S}_{p_{2(k+1)+1}}\circ
 \cdots \circ\mathcal{S}_{p_2}\circ\mathcal{S}_{p_1}$ represents
 a certain basic reflection $\mathcal{S}_{p'}$. But:
   \begin{eqnarray*}
   \mathcal{I}&=&\mathcal{S}_{p_{2(k+1)+1}}\circ
 \cdots \circ\mathcal{S}_{p_2}\circ\mathcal{S}_{p_1}=\\
   &=&\mathcal{S}_{p_{2k+3}}\circ\mathcal{S}_{p_{2k+2}}\circ
   \mathcal{S}_{p_{2k+1}}\circ
 \cdots \circ\mathcal{S}_{p_2}\circ\mathcal{S}_{p_1}=\\
 &=& \mathcal{S}_{p_{2k+3}}\circ\mathcal{S}_{p_{2k+2}}\circ
   \mathcal{S}_p= \hspace*{14mm} \textrm{ (by the induction hypothesis)}\\
   &=&  \mathcal{S}_{p'} \hspace*{48mm} \textrm{ (by  \ref{izoSop}),}
   \end{eqnarray*}
  which had to be proven.  \kdokaz


 We shall also prove some direct consequences of  \ref{izoSop}.

        \bzgled \label{iropqrrqp}
        If $p$, $q$ and $r$ are lines from the same family of lines $\mathcal{X}$, then
        $$\mathcal{S}_r \circ \mathcal{S}_q \circ \mathcal{S}_p =
        \mathcal{S}_p \circ \mathcal{S}_q \circ \mathcal{S}_r.$$
        \ezgled

\textbf{\textit{Proof.}} Because $p$, $q$ and $r$ are lines from the same
family of lines $\mathcal{X}$, by the previous  \ref{izoSop} it follows that
$\mathcal{S}_r \circ \mathcal{S}_q \circ \mathcal{S}_p
=\mathcal{S}_t$ for some line $t\in\mathcal{X}$. Therefore:
 $$(\mathcal{S}_r \circ
\mathcal{S}_q \circ \mathcal{S}_p)^2
=\mathcal{S}_t^2=\mathcal{E},\hspace*{3mm}\textrm{or,}$$
  $$\mathcal{S}_r \circ
\mathcal{S}_q \circ \mathcal{S}_p \circ\mathcal{S}_r \circ
\mathcal{S}_q \circ \mathcal{S}_p=\mathcal{E}.$$
 If we multiply the last
relation from the left by $\mathcal{S}_r$,
$\mathcal{S}_q$ and $\mathcal{S}_p$, we get the desired equality.
 \kdokaz

\bzgled \label{izoSopabc}
         If $a$, $b$ and $c$ are lines such that $\mathcal{S}_a(b)=c$,
         then those lines are from the same family of lines.
         \ezgled

\begin{figure}[!htb]
\centering
\input{sl.izo.6.3.6a.pic}
\caption{} \label{sl.izo.6.3.6a.pic}
\end{figure}

 \textbf{\textit{Proof.}}
 (Figure \ref{sl.izo.6.3.6a.pic})

From the transmutation theorem \ref{izoTransmutacija} it follows that
$\mathcal{S}_a\circ\mathcal{S}_b\circ\mathcal{S}_a=
\mathcal{S}_{\mathcal{S}_a(b)}= \mathcal{S}_c$. By multiplying
the obtained relation
$\mathcal{S}_a\circ\mathcal{S}_b\circ\mathcal{S}_a= \mathcal{S}_c$
first from the right with $\mathcal{S}_a$, and then from the left with
$\mathcal{S}_c$, we get
$\mathcal{S}_c\circ\mathcal{S}_a\circ\mathcal{S}_b=\mathcal{S}_a$.
By theorem \ref{izoSop} the lines $a$, $b$ and $c$ belong to the same
family of lines.
 \kdokaz

 It is clear that the line $a$ from the previous theorem is the line of symmetry of the lines $b$ and $c$.

 With the following theorem we will determine the line $t$ from the relation
 $\mathcal{S}_r \circ
\mathcal{S}_q \circ \mathcal{S}_p =\mathcal{S}_t$ for
 lines $p$, $q$ and $r$ from the same family of lines.


        \bizrek \label{izoSoppqrt}
        If $p$, $q$, $r$ and $t$ are lines such that
        $\mathcal{S}_r \circ
        \mathcal{S}_q \circ \mathcal{S}_p =\mathcal{S}_t$, then the axis of symmetry of the lines
         $p$ and $r$ is also
        the axis of symmetry of the lines $q$ and $t$.
        \eizrek


\begin{figure}[!htb]
\centering
\input{sl.izo.6.3.6.pic}
\caption{} \label{sl.izo.6.3.6.pic}
\end{figure}

\textbf{\textit{Proof.}}
 From $\mathcal{S}_r \circ
\mathcal{S}_q \circ \mathcal{S}_p =\mathcal{S}_t$ it follows that
the lines $p$, $q$, $r$ and $t$
 belong to the same set $\mathcal{X}$ (statement \ref{izoSop}).
 Let $s$ be the line of symmetry of the lines $p$ and $r$ (Figure \ref{sl.izo.6.3.6.pic}).
 This means that $\mathcal{S}_s(p)=r$,
 so according to the transmutation statement \ref{izoTransmutacija} also
  $\mathcal{S}_s\circ\mathcal{S}_p\circ\mathcal{S}_s
  =\mathcal{S}_{\mathcal{S}_s(p)}
  =\mathcal{S}_r$. From this it further follows:
 $$\mathcal{S}_t = \mathcal{S}_r \circ\mathcal{S}_q \circ \mathcal{S}_p
 = (\mathcal{S}_s\circ\mathcal{S}_p\circ\mathcal{S}_s)\circ
 \mathcal{S}_q\circ\mathcal{S}_p =
\mathcal{S}_s\circ(\mathcal{S}_p\circ \mathcal{S}_s\circ
\mathcal{S}_q)\circ \mathcal{S}_p.$$
 Because $\mathcal{S}_s(p)=r$, from the previous example \ref{izoSopabc} it follows that
 the lines $s$, $p$ and $r$
belong to the same set, so $s\in \mathcal{X}$. From statement
\ref{iropqrrqp} it follows $\mathcal{S}_p\circ \mathcal{S}_s\circ
\mathcal{S}_q = \mathcal{S}_q\circ \mathcal{S}_s\circ
\mathcal{S}_p$. If we use the transmutation statement
\ref{izoTransmutacija} again, we get:
$$\mathcal{S}_t =
\mathcal{S}_s\circ\mathcal{S}_p\circ \mathcal{S}_s\circ
\mathcal{S}_q\circ \mathcal{S}_p=
\mathcal{S}_s\circ\mathcal{S}_q\circ \mathcal{S}_s\circ
\mathcal{S}_p\circ \mathcal{S}_p=
\mathcal{S}_s\circ\mathcal{S}_q\circ \mathcal{S}_s=
\mathcal{S}_{\mathcal{S}_s(q)}.$$
  From $\mathcal{S}_t =\mathcal{S}_{\mathcal{S}_s(q)}$ it follows
  $\mathcal{S}_s(q)=t$, which means that the line $s$ is the line of symmetry of the lines $q$ and $t$.
 \kdokaz

 In the next planning task we will illustrate two methods
 of solving.

\bizrek
Let $p$, $q$ and $r$ be lines in the plane.
Construct a triangle $ABC$ such that its vertices $B$ and $C$
lie on the line $p$ and the lines $q$ and $r$ are perpendicular
bisectors of the sides $AB$ and $AC$.
\eizrek


\begin{figure}[!htb]
\centering
\input{sl.izo.6.3.7.pic}
\caption{} \label{sl.izo.6.3.7.pic}
\end{figure}

 \textbf{\textit{Solution.}} (first way)

 Let $q$ and $r$ be the perpendicular bisectors of the sides $AB$ and $AC$
 (Figure \ref{sl.izo.6.3.7.pic}\textit{a}). Then
 $\mathcal{S}_q(B)=A$ and $\mathcal{S}_r(C)=A$.
The points $B$ and $C$ lie on the line $p$, so the point $A$ lies on
the images of this line with respect to the reflections $\mathcal{S}_q$ and
$\mathcal{S}_r$. Therefore, we can construct the point $A$ as the
intersection of the lines $p'_q=\mathcal{S}_q(p)$ and
$p'_r=\mathcal{S}_r(p)$.

 \textbf{\textit{Solution.}} (second way)


 The intersection of the lines $q$ and $r$ is the center of the circumscribed circle of this triangle
 (see \ref{SredOcrtaneKrozn});
we denote it by $O$ (Figure \ref{sl.izo.6.3.7.pic}\textit{b}). Therefore, we can also construct the third perpendicular bisector $s$ of the side $BC$, which is
perpendicular to the line $p$ at the point $O$. The composition $\mathcal{S}_q
\circ \mathcal{S}_r \circ \mathcal{S}_s$ by \ref{izoSop}
is a certain basic reflection $\mathcal{S}_l$ (because $q, r, s\in
\mathcal{X}_O$). The fixed points of this composition are $O$ and $B$,
so $\mathcal{S}_q \circ \mathcal{S}_r \circ \mathcal{S}_s =
\mathcal{S}_{OB}$. Therefore, we can construct the line $OB=l$ as the
perpendicular bisector of the segment $XX'$, where $X$ is an arbitrary point and
$X'=\mathcal{S}_q \circ \mathcal{S}_r \circ \mathcal{S}_s(X)$. In
the case when $X=X'$, we have $l=OX$. Finally, we obtain the point $B$ as the
intersection of the lines $p$ and $l$.
 \kdokaz

\bzgled
               Let $ABCDE$ be a pentagon with a right angle
               at the vertex  $A$.
            The perpendicular bisectors of the sides $AE$, $BC$,
            and $CD$ (lines $p$, $q$, and $r$) intersect
            at a point $O$ and the perpendicular bisectors of the sides $AB$ and $DE$
            (lines $x$ and $y$) intersect at a point $S$
            ($S\neq O$). Prove that one of the vertices of a triangle $KLM$,
            such that $p$, $q$
            and $r$ are the perpendicular bisectors of its sides,
            belongs to the line $OS$.
             \ezgled

\begin{figure}[!htb]
\centering
\input{sl.izo.6.3.8.pic}
\caption{} \label{sl.izo.6.3.8.pic}
\end{figure}

 \textbf{\textit{Solution.}}
 (Figure \ref{sl.izo.6.3.8.pic})

Let $\mathcal{I} = \mathcal{S}_y \circ\mathcal{S}_r
\circ\mathcal{S}_q \circ\mathcal{S}_x \circ\mathcal{S}_p$.
$\mathcal{I}$ is an indirect isometry, with a fixed point $E$,
so by \ref{izo1ftIndZrc} it represents some reflection $S_l$. Because $\angle EAB$ is a right angle, simetrali $p$ and $x$
are perpendicular, so the reflections $\mathcal{S}_p$ and $\mathcal{S}_x$
commute (\ref{izoZrcKomut}). So we have:
 $$\mathcal{I} =\mathcal{S}_l=
  \mathcal{S}_y \circ\mathcal{S}_r
\circ\mathcal{S}_q \circ\mathcal{S}_x \circ\mathcal{S}_p=
  \mathcal{S}_y \circ (\mathcal{S}_r
\circ\mathcal{S}_q \circ\mathcal{S}_p) \circ\mathcal{S}_x.$$
 Lines $p$, $q$ and $r$ belong to the same
$\mathcal{X}_O$, so the composite $\mathcal{S}_r \circ\mathcal{S}_q
\circ\mathcal{S}_p$ represents some reflection $\mathcal{S}_t$,
where it also holds that $t\in\mathcal{X}_O$ or the $t$-axis goes through the point
$O$ (\ref{izoSop}).
 So we have $\mathcal{S}_l=
  \mathcal{S}_y
\circ\mathcal{S}_t  \circ\mathcal{S}_x$, so lines $x$, $t$ and
$y$ belong to the same, or the $t$-line goes through the point $S=x\cap
y$. Because  the $t$-line goes through different points $O$ and $S$,  $t=OS$. So $\mathcal{S}_{OS}=\mathcal{S}_t=\mathcal{S}_r
\circ\mathcal{S}_q \circ\mathcal{S}_p$. Let $p$, $q$ and $r$
be the simetrali of sides $KL$, $LM$ and $MK$ of triangle $KLM$. In this
case $\mathcal{S}_{OS}(K)=\mathcal{S}_t(K)=\mathcal{S}_r
\circ\mathcal{S}_q \circ\mathcal{S}_p(K)=K$, so point $K$
lies on the line $OS$.
 \kdokaz



%________________________________________________________________________________
 \poglavje{Rotations} \label{odd6Rotac}

So far we have learned about isometries that have more than one fixed point.
We have proven that there are two such types of isometries, namely identity as a direct isometry with at least two
fixed points (Theorem \ref{izo2ftIdent}) and reflection as an indirect isometry with at least one
fixed point (Theorem \ref{izo1ftIndZrc}). In this section we will
learn about a type of direct isometries that have exactly one
fixed point.

Let $S$ be an arbitrary point and $\omega\neq 0$ be an oriented plane. The transformation of this plane, in which the point
$S$ is fixed, each other point $X\neq S$ of this plane is mapped to
such a point $X'$, that $\measuredangle XSX'\cong \omega$ and
$SX'\cong SX$, is called a rotation with center $S$ for angle $\omega$;
we denote it with $\mathcal{R}_{S,\omega}$ (Figure
\ref{sl.izo.6.4.1.pic}).

\begin{figure}[!htb]
\centering
\input{sl.izo.6.4.1.pic}
\caption{} \label{sl.izo.6.4.1.pic}
\end{figure}

From the definition itself, the following theorems directly follow.



        \bizrek \label{RotacFiksT}
        The only fixed point of a rotation $\mathcal{R}_{S,\omega}$
        is its center $S$ i.e.
        $$\mathcal{R}_{S,\omega}(X)=X
        \hspace*{1mm} \Leftrightarrow  \hspace*{1mm} X=S.$$
        \eizrek


             \bizrek \label{rotacEnaki}
         Rotations $\mathcal{R}_{S,\alpha}$ and $\mathcal{R}_{B,\beta}$
          are equal if and only if $A=B$ and $\alpha=\beta$ i.e.
          $$\mathcal{R}_{S,\alpha}=\mathcal{R}_{B,\beta}
          \hspace*{1mm} \Leftrightarrow  \hspace*{1mm} A=B \hspace*{1mm}
         \wedge\hspace*{1mm} \alpha=\beta.$$
            \eizrek



         \bizrek
         The inverse transformation of a rotation is a rotation with the same center
        and congruent angle with opposite orientation, i.e.
        $$\mathcal{R}_{S,\omega}^{-1}=\mathcal{R}_{S,-\omega}.$$
         \eizrek



 Although it is intuitively clear, we need to prove that rotation
  is an isometry.

\bizrek
             Rotations are isometries of the plane.
              \eizrek

\begin{figure}[!htb]
\centering
\input{sl.izo.6.4.2.pic}
\caption{} \label{sl.izo.6.4.2.pic}
\end{figure}

 \textbf{\textit{Proof.}} Let $\mathcal{R}_{S,\omega}$ be an arbitrary
 rotation. From the definition it is clear that it represents a bijective mapping. It remains to be proven that for any points $X$ and $Y$
 and their images $X'$ and $Y'$ ($\mathcal{R}_{S,\omega}:X,Y\mapsto X', Y'$) it holds
 $X'Y'\cong XY$ (Figure \ref{sl.izo.6.4.2.pic}).

If one of the points $X$ or $Y$ is equal to the center of the rotation $S$ (for example
$X=S$), the relation $X'Y'\cong XY$ is automatically fulfilled, since in
this case, according to the definition of the rotation, it holds $SY'\cong SY$.

Assume that none of the points $X$ and $Y$ is the center of the rotation
$S$. First, according to the definition of the rotation $SX'\cong SX$ and $SY'\cong
SY$. Then (from relation \ref{orientKotVsota}):
 \begin{eqnarray*}
 \measuredangle Y'SX'&=& \measuredangle Y'SX+\measuredangle XSX'=
 \measuredangle Y'SX + \omega=\\
 &=& \measuredangle Y'SX + \measuredangle YSY'=
 \measuredangle YSY'+ \measuredangle Y'SX=\\ &=&\measuredangle YSX
 \end{eqnarray*}
 This means that the triangle $X'SY'$ and $XSY$ are congruent (\textit{SAS} theorem \ref{SKS}),
  therefore
 $X'Y'\cong XY$.
 \kdokaz

   As every isometry, also the rotation maps a line into a line.
   In the next theorem we will see the relation between a line and
   its image under the rotation.



        \bizrek \label{rotacPremPremKot}
        A line and its rotated image determine an angle congruent
         to the angle of this rotation.
        If the angle of the rotation measures $180^0$,
        the line and its image are parallel.
        \eizrek

\begin{figure}[!htb]
\centering
\input{sl.izo.6.4.3.pic}
\caption{} \label{sl.izo.6.4.3.pic}
\end{figure}

 \textbf{\textit{Proof.}} Let $p'$ be the image of the line $p$ under the rotation
  $\mathcal{R}_{S,\omega}$ (Figure \ref{sl.izo.6.4.3.pic}).

If $S\in p$, the proof is trivial. Assume that $S\notin p$.
  We denote with $P$ the orthogonal projection of the center $S$ onto
   the line $p$ and  its image $P'=\mathcal{R}_{S,\omega}(P)$. Because $S\notin
   p$,
   it follows that $P\neq S$ or $P'\neq P$. From
   $P\in p$ it follows that $P'\in p'$. Because  the isometry preserves the angles, from
   $SP\perp p$ it follows that $SP'\perp p'$.

   If $\omega=180^0$, the points $P$, $S$ and $P'$ are collinear and
   the lines $p$ and $p'$ have a common perpendicular
   $PP'$, which means that $p\parallel p'$.

   In the case $\omega\neq 180^0$, the lines intersect (in the opposite
    case, from $p\parallel p'$ it would follow that $SP\parallel SP'$, which is not possible).
    Let $V$ be their intersection point. The angle, determined by
    the lines
    $p$ and $p'$, is equal to the angle determined by the lines $SP$ and
    $SP'$ (the angle with perpendicular legs - statement \ref{KotaPravokKraki}),
     which is just the angle of rotation $\omega$.
    \kdokaz


   In the proof of the previous statement, the process of designing the image
   $p'$
   of the line $p$ under the rotation $\mathcal{R}_{S,\omega}$ is also described.
   First, we draw the orthogonal projection $P$ of the center $S$ onto
   the line $p$, and then its image $P=\mathcal{R}_{S,\omega}(P')$.
   The line $p'$ is obtained as the perpendicular of the line $SP'$ in the point
   $P'$.
   Of course, another way would be to rotate two arbitrary points
   of the line $p$.

  In the next important statement, we will see how to express
  the rotation using the basic reflections.

\bizrek \label{rotacKom2Zrc}
Any rotation
$\mathcal{R}_{S,\omega}$  can be expressed as the product of two reflections
$\mathcal{S}_p$ and $\mathcal{S}_q$ where
$p$ and $q$  are arbitrary lines, such that $S=p\cap q$
and $\measuredangle pq=\frac{1}{2}\omega$.\\ The reverse is also true -
the product of two reflections $\mathcal{S}_p$ and $\mathcal{S}_q$ ($S=p\cap q$)
is a rotation  with the centre $S$ and the angle
$\omega=2\cdot\measuredangle pq$, i.e.\\
$$\mathcal{R}_{S,\omega}=\mathcal{S}_q\circ\mathcal{S}_p
\hspace*{1mm} \Leftrightarrow  \hspace*{1mm} S=p\cap q \hspace*{1mm}
\wedge\hspace*{1mm} \measuredangle pq=\frac{1}{2}\omega.$$
\eizrek

\begin{figure}[!htb]
\centering
\input{sl.izo.6.4.4.pic}
\caption{} \label{sl.izo.6.4.4.pic}
\end{figure}

 \textbf{\textit{Proof.}}

 ($\Leftarrow$) Assume that $p$
 and $q$ are lines that intersect in the point $S$. Let $\omega
  =2\cdot\measuredangle pq$
 (Figure \ref{sl.izo.6.4.4.pic}). We will prove that
  $\mathcal{R}_{S,\omega}=\mathcal{S}_q\circ\mathcal{S}_p$,
  or $\mathcal{R}_{S,\omega}(X)=\mathcal{S}_q\circ\mathcal{S}_p(X)$ for every
point $X$ of the plane. We will consider two cases.

    \textit{1)} If $X=S$, then $\mathcal{S}_p(X)=\mathcal{S}_q(X)=X$ (because
the lines $p$ and $q$ intersect in the point $S$). So 
$\mathcal{R}_{S,\omega}(S)=S=\mathcal{S}_q\circ\mathcal{S}_p(S)$.

\textit{2)} Let $X\neq S$ and $\mathcal{S}_q\circ\mathcal{S}_p(X)=X'$.
     Prove that  also $\mathcal{R}_{S,\omega}(X)=X'$. Let
$\mathcal{S}_p(X)=X_1$. Then  $\mathcal{S}_q(X_1)=X'$. Because
of
this, first $SX\cong SX_1\cong SX'$, and then:
 \begin{eqnarray*}
  \measuredangle XSX'&=& \measuredangle XSX_1+ \measuredangle
  X_1SX'=\\
   &=& 2\cdot\measuredangle p,SX_1+2\cdot\measuredangle SX_1,q=\\
   &=& 2\cdot(\measuredangle p,SX_1+ \measuredangle SX_1,q) =\\
   &=& 2\cdot\measuredangle pq=\omega.
 \end{eqnarray*}
 By the definition of rotation, $\mathcal{R}_{S,\omega}(X)=X'$.

  ($\Rightarrow$) Let
  $\mathcal{R}_{S,\omega}=\mathcal{S}_q\circ\mathcal{S}_p$. Then
  it
  is
  $\mathcal{S}_q\circ\mathcal{S}_p(S)=\mathcal{R}_{S,\omega}(S)=S$.
  By izrek \ref{izoKomppqX} $S=p\cap q$. It remains to prove
  $\measuredangle pq=\frac{1}{2}\omega$ or $\omega
  =2\cdot\measuredangle pq$. We denote $\widehat{\omega}
  =2\cdot\measuredangle pq$. By the first part of the proof ($\Leftarrow$)
  it is
  $\mathcal{S}_q\circ\mathcal{S}_p=\mathcal{R}_{S,\widehat{\omega}}$.
  Therefore $\mathcal{R}_{S,\omega}=\mathcal{S}_q\circ\mathcal{S}_p=
  \mathcal{R}_{S,\widehat{\omega}}$. From izrek \ref{rotacEnaki}
  it follows $\omega =\widehat{\omega}
  =2\cdot\measuredangle pq$.
   \kdokaz



         \bizrek \label{RotacDirekt}
        A rotation is a direct isometry.
        \eizrek

 \textbf{\textit{Proof.}} A direct consequence of the previous izrek
 \ref{rotacKom2Zrc}, because the composition of two basic reflections is an indirect isometry.
 \kdokaz

Similarly to the basic reflection, also for the rotation, there is an izrek about
transmutation\index{transmutation!rotations}.

\bizrek \label{izoTransmRotac}
For an arbitrary rotation $\mathcal{R}_{O,\alpha}$
and an arbitrary isometry $\mathcal{I}$ is
$$\mathcal{I}\circ
\mathcal{R}_{O,\alpha}\circ\mathcal{I}^{-1}=
\mathcal{R}_{\mathcal{I}(O),\alpha'},$$
 where: $\alpha'=\alpha$, if $\mathcal{I}$ a direct isometry, or
$\alpha'=-\alpha$, if $\mathcal{I}$ is an opposite isometry.
\eizrek

\textbf{\textit{Proof.}}  By izrek \ref{rotacKom2Zrc} we can write the rotation
   $\mathcal{R}_{O,\alpha}$
  as a composite of two basic
reflections, namely
$\mathcal{R}_{O,\omega}=\mathcal{S}_q\circ\mathcal{S}_p$, where
 $O=p\cap q$ and $\measuredangle pq=\frac{1}{2}\alpha$.
If we use the transmutation theorem for basic reflections, we get:
 \begin{eqnarray*}
  \mathcal{I}\circ
        \mathcal{R}_{O,\alpha}\circ\mathcal{I}^{-1}&=&
        \mathcal{I}\circ\mathcal{S}_q\circ\mathcal{S}_p\circ\mathcal{I}=\\
        &=&
        \mathcal{I}\circ\mathcal{S}_q\circ\mathcal{I}^{-1}
        \circ\mathcal{I}\circ\mathcal{S}_p\circ\mathcal{I}^{-1}=\\
        &=&\mathcal{S}_{q'}\circ\mathcal{S}_{p'}=\\
        &=&\mathcal{R}_{O',\alpha'},
 \end{eqnarray*}
where $p'$ and $q'$ are the images of lines $p$ and $q$ under the isometry
$\mathcal{I}$, $O'=p'\cap q'$ and $\alpha'=2\cdot\measuredangle
p'q'$. Because $O=p\cap q$, we have $\mathcal{I}(O)=
\mathcal{I}(p\cap q)=\mathcal{I}(p)\cap \mathcal{I}(q)=p'\cap
q'=O'$. Because isometries preserve the relation of congruence of angles, we have
 $\alpha'=2\cdot\measuredangle p'q'=2\cdot\measuredangle pq=\alpha$, if
 $\mathcal{I}$ is direct, or
 $\alpha'=2\cdot\measuredangle p'q'=-2\cdot\measuredangle pq=-\alpha$, if
 $\mathcal{I}$ is indirect.
 \kdokaz

 In the next theorem we will prove the important fact that for two non-parallel lines there exists a rotation that transforms the first line into the second.

\bizrek
       Let $AB$ and $A'B'$ be congruent line segments that are not parallel.
        There is exactly one rotation that maps the line segment $AB$ to the line segment $A'B'$,
        such that the point $A$ maps to the point $A'$ and the point $B$ to the point $B'$.
        \eizrek

\begin{figure}[!htb]
\centering
\input{sl.skl.3.1.10Rotac.pic}
\caption{} \label{sl.skl.3.1.10Rotac.pic}
\end{figure}

 \textbf{\textit{Proof.}} Let $S$ be the intersection of the line segments $AA'$ and $BB'$, and $\omega=\measuredangle AB, A'B'$. We will prove that $\mathcal{R}_{S,\omega}$ is the desired rotation.
 Because the point $S$ lies on the line segments $AA'$ and $BB'$, we have $SA\cong SA'$ and $SB\cong SB'$ (Figure \ref{sl.skl.3.1.10Rotac.pic}). Then from $AB\cong A'B'$ by the \textit{SSS} \ref{SSS} it follows that $\triangle SAB\cong \triangle SA'B'$ (see also example \ref{načrt1odd3}). Therefore, we have $\measuredangle ASB\cong \measuredangle A'SB'$, or:
 $$\measuredangle ASA'=\measuredangle ASB+\measuredangle BSA'=
 \measuredangle A'SB'+\measuredangle BSA'= \measuredangle BSB'.$$
 By the definition of rotation, we have $\mathcal{R}_{S,\measuredangle ASA'}:A,B\mapsto A',B'$. By izreku \ref{rotacPremPremKot} we have $\measuredangle ASA'=\measuredangle AB, A'B'=\omega$, so $\mathcal{R}_{S,\omega}:A,B\mapsto A',B'$.
 Because rotation as an isometry preserves the relation $\mathcal{B}$, the line segment $AB$ is mapped by this rotation to the line segment $A'B'$.

Let $\mathcal{R}_{\widehat{S},\widehat{\omega}}$ be another rotation, for which we have $\mathcal{R}_{\widehat{S},\widehat{\omega}}:A,B\mapsto A',B'$. Then
$\mathcal{R}^{-1}_{S,\omega} \circ \mathcal{R}_{\widehat{S},\widehat{\omega}}$ is a direct isometry with two fixed points $A$ and $B$, so by izreku \ref{izo2ftIdent} it represents the identity $\mathcal{E}$, or
$\mathcal{R}_{\widehat{S},\widehat{\omega}} = \mathcal{R}_{S,\omega}$.
 \kdokaz


 In the following examples we will use the fact that a triangle
 $ABC$ is equilateral if and only if
 $\mathcal{R}_{A,60^0}(B)=C$.

\bzgled
            Let $P$ be an interior point of an equilateral triangle
            $ABC$.\\
          a) Prove that
          $|PA|+|PB|\geq |PC|.$\\
          b) Suppose that $\angle BPA=\mu$,
            $\angle CPA= \nu$ and $\angle BPC= \xi$. Calculate the interior angles of a triangle,
            with sides that are congruent to the line segments $PA$, $PB$ in $PC$.
           \ezgled

\begin{figure}[!htb]
\centering
\input{sl.izo.6.4.5.pic}
\caption{} \label{sl.izo.6.4.5.pic}
\end{figure}

 \textbf{\textit{Proof.}} If $P'=\mathcal{R}_{A,60^0}(P)$,
 the triangle $APP'$ is equilateral (Figure \ref{sl.izo.6.4.5.pic}).
 Because $\mathcal{R}_{A,60^0}(B)=C$,
 the line segment $BP$ is mapped to
the corresponding line segment $CP'$ by this rotation. Therefore, the triangle $PP'C$ has sides that are
congruent to the line segments $PA$, $PB$ and $PC$. By the triangle inequality
- statement \ref{neenaktrik} - we have $|PA|+|PB|=|PP'|+|P'C|\geq PC.$

Let us calculate the angles of the triangle $PP'C$ as well. The rotation
$\mathcal{R}_{A,60^0}$ maps the triangle $ABP$ to the triangle
$ACP'$, so the triangles are congruent and $\angle BPA=\angle CP'A$. From
this and from statement \ref{VsotKotTrik} it follows that:
 \begin{eqnarray*}
  \angle CP'P &=& \angle CP'A-\angle PP'A=\angle
  BPA-60^0=\mu-60^0,\\
  \angle CPP' &=& \angle CPA-60^0=\nu-60^0,\\
  \angle PCP' &=& 180^0
  -(\mu-60^0)-(\nu-60^0)=300^0-(\mu+\nu)=\\
  &=& 300^0-(360^0-\xi)=\xi-60^0,
 \end{eqnarray*}
  which had to be calculated.  \kdokaz



        \bzgled
        Let $ABCD$ be a rhombus, such that the interior angle at the vertex $A$
         is equal to $60^0$.
         If a line  $l$ intersects the sides $AB$ and $BC$  of this rhombus at  points $P$ and
        $Q$ such that  $|BP|+|BQ|=|AB|$, then $PQD$ is a regular triangle.
        \ezgled

\begin{figure}[!htb]
\centering
\input{sl.izo.6.4.6.pic}
\caption{} \label{sl.izo.6.4.6.pic}
\end{figure}

\textbf{\textit{Proof.}} From $AB\cong AD$ and $\angle DAB=60^0$
 it follows that the triangle $ABD$ is regular (Figure \ref{sl.izo.6.4.6.pic}).
 Similarly, the
 triangle $BCD$ is also regular. This means that $\mathcal{R}_{D,60^0}:A,B\mapsto
 B,C$. Because $|BP|+|BQ|=|AB|$ and $|BP|+|AP|=|AB|$, we have $AP\cong BQ$.
 But the rotation $\mathcal{R}_{D,60^0}$ maps the segment $AB$
to the segment $BC$, so because of the condition $AP\cong BQ$ it also
maps the point $P$ to the point $Q$. Therefore, $\mathcal{R}_{D,60^0}(P)=Q$, which
means that the triangle $DPQ$ is regular.
 \kdokaz

At the next example we will illustrate the use of rotation in
design tasks.



        \bzgled
        Let $A$ be an interior point of an angle $pOq$. Construct points
        $B$ and $C$ on the sides $p$ and $q$
        such that $ABC$ is a regular triangle.
        \ezgled

\begin{figure}[!htb]
\centering
\input{sl.izo.6.4.7.pic}
\caption{} \label{sl.izo.6.4.7.pic}
\end{figure}

 \textbf{\textit{Solution.}}  (Figure \ref{sl.izo.6.4.7.pic}).
Let $ABC$ be such a regular triangle that its vertices $B$ and
$C$ lie on the sides $p$ and $q$. Then
$\mathcal{R}_{A,60^0}(B)=C$. The point $B$ lies on the segment $p$, so
its image - the point $C$ - lies on the segment
$p'=\mathcal{R}_{A,60^0}(p)$. The vertices $C$ can be obtained as
the intersection of the segments $p'$ and $q$, and then the vertex $B$ as
$B=\mathcal{R}_{A,-60^0}(C)$. If the segments $p'$ and $q$ do not
intersect, there is no solution. But if they lie on the same line, the task
 has infinitely many solutions.
  \kdokaz

If in the previous task the condition was that the triangle $ABC$
was isosceles and right-angled at the vertex $A$, we would
use the rotation with center $A$ for the angle $45^0$. Similarly, if we needed to plan the square $PQRS$ with center at the point
$A$ under the condition $P\in p$, $Q\in q$ (Figure \ref{sl.izo.6.4.8.pic}).

\bnaloga\footnote{1. IMO Romania - 1959, Problem 5.}
        An arbitrary point $M$ is selected in the interior of the segment $AB$. The
        squares $AMCD$ and $MBEF$ are constructed on the same side of $AB$, with
        the segments $AM$ and $MB$ as their respective bases. The circles circumscribed
        about these squares, with centres $P$ and $Q$, intersect at $M$ and also
        at another point $N$. Let $N'$ denote the point of intersection of the straight
        lines $AF$ and $BC$.

        (a) Prove that the points $N$ and $N'$ coincide.

        (b) Prove that the straight lines $MN$ pass through a fixed point $S$ independent
            of the choice of $M$.

        (c) Find the locus of the midpoints of the segments $PQ$ as $M$ varies between
            $A$ and $B$
        \enaloga



\begin{figure}[!htb]
\centering
\input{sl.izo.6.4.IMO1.pic}
\caption{} \label{sl.izo.6.4.IMO1.pic}
\end{figure}

 \textbf{\textit{Solution.}} Let $k$ and $l$
 denote the circumscribed circles
  of the squares
  $AMCD$ and $MBEF$
 (Figure \ref{sl.izo.6.4.IMO1.pic}).

 (\textit{i}) The rotation $\mathcal{R}_{M,-90^0}$ maps the points
 $A$ and $F$ in order to the points $C$ and $B$ or the line $AF$ to the line $CB$. By the statement
   of \ref{rotacPremPremKot}
   these two lines determine the angle of rotation,
 so $AF\perp CB$. Therefore $\angle AN'C\cong\angle BN'F =90^0$, which means
 (by the statement
  \ref{TalesovIzrKroz}) that the point $N'$ lies on both circles
  $k$ and $l$
  with diameters $AC$ and $BF$. Because of this,
  $N'=N$.

 (\textit{ii}) Let $j$ be a circle with diameter $AB$ and $S$ the center
 of that arc (semicircle) $AB$, which is on the opposite side of the line
 $AB$ with respect to the squares $AMCD$ and $MBEF$. The point $S$ is not dependent
  on the choice of point $M$. We prove that all lines $MN$ go through the point $S$.
  Because $\angle ANB=90^0$, by the statement
  \ref{TalesovIzrKroz} also the point $N$ lies
  on the circle $j$. From the statement
  \ref{ObodObodKot}
  it follows that $\angle ANM =\angle ADM=45^0$ and $\angle MNB =\angle
  MEB=45^0$, so $NM$ is the perpendicular bisector of the angle $ANB$. By the statement
  \ref{TockaN} the perpendicular bisector $NM$ goes through the point $S$.

\begin{figure}[!htb]
\centering
\input{sl.izo.6.4.IMO1a.pic}
\caption{} \label{sl.izo.6.4.IMO1a.pic}
\end{figure}

 (\textit{iii}) Let $O$ be the center of the line segment $PQ$.
 We denote with $P'$, $Q'$ and $O'$ the orthogonal projections of points
 $P$, $Q$ and $O$ on the line $AB$ (Figure
 \ref{sl.izo.6.4.IMO1a.pic}). The line $OO'$ is the median of the rectangle
 $P'Q'QP$, so by Theorem \ref{srednjTrapez}:
  $$|OO'|=\frac{1}{2}\left(|PP'|+|QQ'| \right)=
   \frac{1}{2}\left(\frac{1}{2}|AM|+\frac{1}{2}|MB| \right)=\frac{1}{4}|AB|.$$
 Therefore, the distance of the point $O$ from the line $AB$ is constant or
 independent of the choice of the point $M$.

  Let $L$ be the center of the square $ABGH$, which is on the same side
  of the line $AB$ as the square $AMCD$ and $MBEF$. With $O_A$ and $O_B$
  we denote the centers of the line segments $LA$ and $LB$. Because $O_AO_B$
  is the median of the triangle $ALB$ with the base $AB$, $O_AO_B \parallel
  AB$.
  From $|AO_A|=\frac{1}{2}|AL|=\frac{1}{4}|AG|$ it follows that
  the distance of the parallels $O_AO_B$ and $AB$ is equal to $\frac{1}{4}|AB|$. From
  the already proven fact $d(O,AB)=\frac{1}{4}|AB|$ it follows that the point
  $O$ lies on the line $O_AO_B$. But, when the point $M$ moves inside the line segment $AB$, the points
  $P$ and $Q$ move inside the line segments $AL$ and $BL$. This means that the point $O$ is
  inside the triangle $ALB$ or $O$ lies on the open line
  segment $O_AO_B$.

  We will also prove that any point $O$ of the line segment $O_AO_B$
  is the center of a line segment $PQ$ for a certain choice of the squares
   $AMCD$ and $MBEF$ or the points  $M$. In this case, we get the point  $M$
   from the condition $|AO'|=\frac{1}{2}|AM|+\frac{1}{4}|AB|$ or
   $|AM|=2|AO'|-\frac{1}{2}|AB|$. Such a point $M$ always
   exists, if $\frac{1}{4}|AB|<|AO'|<\frac{3}{4}|AB|$, or
   when the point $O$ lies on the open line segment $O_AO_B$.

The desired geometric location of points $M$ is therefore the line $O_AO_B$.
   \kdokaz


%________________________________________________________________________________
  \poglavje{Half-Turn} \label{odd6SredZrc}

We will consider the rotation by the angle $180^0$ as
a special type of isometry.

The rotation of the plane $\mathcal{R}_{S,\omega}$ by the angle $\omega=180^0$
we will call the \index{zrcaljenje!središčno}\pojem{central reflection}
or \index{zrcaljenje!čez točko}\pojem{reflection over a point} with
\index{središče!središčnega zrcaljenja}\pojem{center} $S$ and denote it by $\mathcal{S}_S$ (Figure \ref{sl.izo.6.5.1.pic}). So
$\mathcal{S}_S=\mathcal{R}_{S,180^0}$.


\begin{figure}[!htb]
\centering
\input{sl.izo.6.5.1.pic}
\caption{} \label{sl.izo.6.5.1.pic}
\end{figure}

From the definition it is clear that for any point $X\neq S$ it holds
$\mathcal{S}_S(X)=X'$ exactly when $S$ is the center of the line
$XX'$.

Directly from the definition it also follows the next statement.

           \bizrek \label{izoSredZrcInv}
           A half-turn is an involution, i.e.
           $\mathcal{S}^2_S=\mathcal{E}$. \eizrek

  Because the central reflection is a type of rotation, it also has all
  the properties of rotation. The only fixed point
  of the central reflection is therefore the center of this reflection.

 Just like the rotation, the central reflection
 $\mathcal{S}_S$
 can be represented as the composition of two basic reflections. The angle, which is determined by the axes, is equal to half the angle of the rotation (statement \ref{rotacKom2Zrc}) -
 in our case
 this is $\frac{180^0}{2}=90^0$. This means that the axes of the two
 reflections are perpendicular in the case of central reflection. In this case, according to
  \ref{izoZrcKomut}, the basic reflections commute
  (Figure \ref{sl.izo.6.5.2.pic}).
 So the following statement holds.

\begin{figure}[!htb]
\centering
\input{sl.izo.6.5.2.pic}
\caption{} \label{sl.izo.6.5.2.pic}
\end{figure}

\bizrek \label{izoSrZrcKom2Zrc}
        Any half-turn around a point  $S$
         can be expressed as the product of two reflections
          $\mathcal{S}_p$ and $\mathcal{S}_q$ where
          $p$ and $q$  are arbitrary perpendicular lines intersecting at the point $S$.\\
          The reverse is also true -
         the product of two reflections
          $\mathcal{S}_p$ and $\mathcal{S}_q$ ($S=p\cap q$ and
         $p\perp q$)
         is the half-turn around the point $S$, i.e.\\
         $$\mathcal{S}_S=\mathcal{S}_q\circ\mathcal{S}_p=
         \mathcal{S}_p\circ\mathcal{S}_q
         \hspace*{1mm} \Leftrightarrow  \hspace*{1mm} S=p\cap q \hspace*{1mm}
         \wedge\hspace*{1mm} p\perp q.$$
        \eizrek



        \bizrek \label{izoKomp3SredZrc}
        The product of three half-turns is a half-turn.
       If the centres of these half-turns are three non-collinear points,
        then the centre of the new half-turn is the fourth vertex of a parallelogram.
        \eizrek


\begin{figure}[!htb]
\centering
\input{sl.izo.6.5.3.pic}
\caption{} \label{sl.izo.6.5.3.pic}
\end{figure}

\textbf{\textit{Proof.}}
 Let $\mathcal{S}_A$, $\mathcal{S}_B$ and $\mathcal{S}_C$ be three central
symmetries with centers $A$, $B$ and $C$ (Figure
\ref{sl.izo.6.5.3.pic}). We denote by $p$ the line $AB$, by $a$, $b$
and $c$ the rectangles of the line $p$ through the points $A$, $B$ and
$C$, and by $c'$ the rectangle of the line $c$ through the point $C$. Then by izreku \ref{izoSrZrcKom2Zrc}:
 $$\mathcal{S}_C \circ \mathcal{S}_B \circ \mathcal{S}_A=
  \mathcal{S}_{c'} \circ \mathcal{S}_c \circ \mathcal{S}_b
  \circ \mathcal{S}_p \circ \mathcal{S}_p \circ \mathcal{S}_a
  = \mathcal{S}_{c'} \circ \mathcal{S}_c \circ \mathcal{S}_b \circ \mathcal{S}_a.$$
   The axes $a$, $b$ and $c$ belong to the same set of parallels $\mathcal{X}_a$,
because they are all perpendicular to the line $p$. The composite
$\mathcal{S}_c \circ \mathcal{S}_b \circ \mathcal{S}_a$ therefore
represents the basic reflection $S_d$, where the axis $d$ belongs to the same set
$\mathcal{X}_a$ (izrek \ref{izoSop}). Then both lines $c'$
and $d$ are perpendicular, so:
 $$\mathcal{S}_C \circ \mathcal{S}_B \circ \mathcal{S}_A=
 \mathcal{S}_{c'} \circ \mathcal{S}_d=\mathcal{S}_D,$$
where $D=d\cap c'$. By izreku \ref{izoSoppqrt} the pairs of lines $a$,
$c$ and $b$, $d$ have a common somernica. This means that if
$A$, $B$ and $C$ are three non-collinear points, the quadrilateral $ABCD$
is a parallelogram.
 \kdokaz

The consequence of the proven statement is the following izrek.




        \bizrek \label{izoKomp2n+1SredZrc}
        The product of an odd number of half-turns is a half-turn.
        \eizrek

 \textbf{\textit{Proof.}} We will prove the statement by induction on
 the number $n\geq 1$, where $m=2n+1$ (odd) is the number of points.

 For $n=1$, i.e. $m=3$, the statement is a direct consequence of the previous izrek
 \ref{izoKomp3SredZrc}.

Let's assume that for every $k\in \mathbb{N}$ ($k\geq 1$)
 and every $(2k+1)$-tuple
 of coplanar points
 $A_1,A_2,\ldots ,A_{2k+1}$ the composite
 $\mathcal{S}_{A_{2k+1}}\circ
 \cdots \circ\mathcal{S}_{A_2}\circ\mathcal{S}_{A_1}$ represents
 some central reflection $\mathcal{S}_A$ (inductive assumption).

 It is necessary to prove that  then also for $k+1$
 and every $(2(k+1)+1)$-tuple
 of coplanar points
 $A_1,A_2,\ldots ,A_{(2(k+1)+1)}$ the composite
 $\mathcal{I}=\mathcal{S}_{A_{2(k+1)+1}}\circ
 \cdots \circ\mathcal{S}_{A_2}\circ\mathcal{S}_{A_1}$ represents
 some central reflection $\mathcal{S}_{A'}$. So:

   \begin{eqnarray*}
   \mathcal{I}&=&\mathcal{S}_{A_{2(k+1)+1}}\circ
 \cdots \circ\mathcal{S}_{A_2}\circ\mathcal{S}_{A_1}=\\
   &=&\mathcal{S}_{A_{2k+3}}\circ\mathcal{S}_{A_{2k+2}}\circ
   \mathcal{S}_{A_{2k+1}}\circ
 \cdots \circ\mathcal{S}_{A_2}\circ\mathcal{S}_{A_1}=\\
 &=& \mathcal{S}_{A_{2k+3}}\circ\mathcal{S}_{A_{2k+2}}\circ
   \mathcal{S}_A= \hspace*{14mm} \textrm{ (by the inductive assumption)}\\
   &=&  \mathcal{S}_{A'} \hspace*{48mm} \textrm{ (by  \ref{izoKomp3SredZrc}),}
   \end{eqnarray*}
  which was to be proven.  \kdokaz

A direct consequence of \ref{izoTransmRotac}, which concerns rotation, is the following theorem or izrek about transmutation of central reflection\index{transmutation!of central reflection}.


        \bizrek \label{izoTransmSredZrc}
        For an arbitrary half-turn
        $\mathcal{S}_{O}$
         and an arbitrary isometry $\mathcal{I}$ is
        $$\mathcal{I}\circ
        \mathcal{S}_{O}\circ\mathcal{I}^{-1}=
        \mathcal{R}_{\mathcal{I}(O)}.$$
        \eizrek


 Just like the basic reflection, also the central
 reflection is often used in planning
 tasks.

\bzgled
        Let $S$ be an interior point of an angle $pOq$. Construct a square
         $ABCD$ with the centre $S$ such that the vertices
        $A$ and $C$ lie on the sides $p$ and $q$.
        \ezgled

\begin{figure}[!htb]
\centering
\input{sl.izo.6.5.4.pic}
\caption{} \label{sl.izo.6.5.4.pic}
\end{figure}

 \textbf{\textit{Proof.}} Let $ABCD$ be a square that satisfies the given conditions -
 $S$ is its center, $A\in p$ and $C\in q$ (Figure \ref{sl.izo.6.5.4.pic}).
   Because the point $S$ is the center of the diagonal
  $AC$, $\mathcal{S}_S(A)=C$. From $A\in p$ it follows that $C\in
  \mathcal{S}_S(p)$.
  We therefore obtain the point $C$ from the condition $C\in q\cap\mathcal{S}_S(p)$.
  Finally, $A=\mathcal{S}_S(C)$, $D=\mathcal{R}_{S,90^0}(C)$ and
  $B=\mathcal{R}_{S,-90^0}(C)$.
 \kdokaz

            \bzgled
              Let $A$ be one of the two intersections of circles $k$ and $l$. Construct
            a common secant of these two circles passing through the point $A$ and determine
            a two congruent chord with these circles.
           \ezgled


\begin{figure}[!htb]
\centering
\input{sl.izo.6.5.5.pic}
\caption{} \label{sl.izo.6.5.5.pic}
\end{figure}

 \textbf{\textit{Solution.}}
   Let $B$ be the other intersection of circles $k$ and $l$
    (Figure \ref{sl.izo.6.5.5.pic}). Then one solution to the task
    is the line $AB$, since the distance $AB$
   is the common secant of both circles.

  Let $s\ni A$ be a common secant of circles $k$ and $l$, which also intersects
  in points $K$ and $L$, so that $AK\cong AL$. In the case $K=L$
  we have $K=L=B$, which is the already mentioned solution. If $K\neq L$, 
  from the condition $A, K, L\in s$ it follows that $A$ is the center of the line $KL$.
  Therefore $\mathcal{S}_A(K)=L$, which means that we obtain the point $L$
  from the condition $L\in l\cap \mathcal{S}_A(k)$. Finally,
  $K=\mathcal{S}_A(L)$.
   \kdokaz

\bzgled
Construct a quadrilateral $ABCD$ with the sides that are congruent to
the four given line segments $a$, $b$, $c$, and $d$ and
the line segment defined by the midpoints of
the two opposite sides $AD$ and $BC$ that is congruent to the given line segment $l$.
\ezgled

\begin{figure}[!htb]
\centering
\input{sl.izo.6.5.6.pic}
\caption{} \label{sl.izo.6.5.6.pic}
\end{figure}

 \textbf{\textit{Solution.}}
Let $P$ and $Q$ be the midpoints of sides $AD\cong d$ and $BC\cong b$
of quadrilateral $ABCD$, so that $PQ\cong l$, $AB\cong a$ and
$CD\cong c$ still hold (Figure \ref{sl.izo.6.5.6.pic}). Let $S$
be the midpoint of diagonal $AC$. We can construct triangle $PQS$ since $PQ\cong l$, $QS=\frac{1}{2}a$ and $PS =\frac{1}{2}c$ (by Theorem \ref{srednjicaTrik}). Points $C$ and $A$ lie on
circles $k_d=k(P,\frac{1}{2}d)$ and $k_b=k(Q,\frac{1}{2}b)$, respectively. Since
$C=\mathcal{S}_S(A)$, point $C$ also lies on the image $k'_d$
of circle $k_d$ under central reflection $\mathcal{S}_S$.
Point $C$ is therefore one of the intersections of circles $k'_d$ and $k_b$. Then $A=\mathcal{S}_S(C)$, $D=\mathcal{S}_P(A)$ and
$B=\mathcal{S}_Q(C)$.
 \kdokaz


        \bzgled
        Let $P$, $Q$, $R$, $S$ and $T$ be points in the plane. Construct
        a pentagon $ABCDE$ such that the points $P$, $Q$, $R$, $S$ and $T$ are
        the midpoints of its sides $AB$, $BC$, $CD$, $DE$ and $EA$, respectively.
        \ezgled


\begin{figure}[!htb]
\centering
\input{sl.izo.6.5.7.pic}
\caption{} \label{sl.izo.6.5.7.pic}
\end{figure}

\textbf{\textit{Solution.}}
Let $ABCDE$ be the desired pentagon
    (Figure \ref{sl.izo.6.5.7.pic}) and:
 $$\mathcal{I} = \mathcal{S}_T \circ \mathcal{S}_S
  \circ \mathcal{S}_R \circ \mathcal{S}_Q \circ \mathcal{S}_P.$$
  From  \ref{izoKomp2n+1SredZrc} it follows that the isometry
  $\mathcal{I}$ is a central
reflection $\mathcal{S}_O$. Since
$\mathcal{S}_O(A)=\mathcal{I}(A)=A$, we have $O=A$  or $\mathcal{I}=\mathcal{S}_A$. The point
$A$ can therefore be constructed as the center of the line $XX'$, where
$X$ is any point on the given plane and $X'= \mathcal{I}(X)$.
  \kdokaz


        \bzgled
        Let $A_iB_i$ ($i\in \{1,2,3\}$)
         be parallel chords of a circle $k$. Suppose that $S_1$, $S_2$ and $S_3$
          are the midpoints of the line segments $A_2A_3$, $A_1A_3$ and $A_1A_2$,
           respectively. Prove that
         $C_i=\mathcal{S}_{S_i}(B_i)$ are three collinear points.
        \ezgled

\begin{figure}[!htb]
\centering
\input{sl.izo.6.5.4a.pic}
\caption{} \label{sl.izo.6.5.4a.pic}
\end{figure}

\textbf{\textit{Proof.}} Let $S$ be the center of the circle $k$
(Figure \ref{sl.izo.6.5.4a.pic}). Because $SA_i\cong SB_i$ ($i\in
\{1,2,3\}$), all the simetrals $s_i$ of the chords $A_iB_i$ ($i\in
\{1,2,3\}$) pass through the point $S$. From $s_i\perp A_iB_i$ and the fact that
the chords are parallel to each other, it follows that all the simetrals $s_i$
are equal to each other - we denote them with $s$. Therefore, $s$ is the common simetral of
the chords $A_iB_i$, so $\mathcal{S}_s(A_i)=B_i$  ($i\in
\{1,2,3\}$).

Since $S_1$, $S_2$, and $S_3$ are the centers of the lines $C_1B_1$, $C_2B_2$, and $C_3B_3$, and the sides $A_2A_3$, $A_1A_3$, and $A_1A_2$ of the triangle $A_1A_2A_3$, according to the theorem \ref{vektSestSplosno},
$\overrightarrow{S_1S_2}=
\frac{1}{2}(\overrightarrow{C_1C_2}+\overrightarrow{B_1B_2})$ or
$\overrightarrow{C_1C_2}=
2\overrightarrow{S_1S_2}-\overrightarrow{B_1B_2}
=\overrightarrow{A_2A_1}-\overrightarrow{B_1B_2}=\overrightarrow{A_2A_1}
+\overrightarrow{B_2B_1}$.
Since the points $A_2$ and $A_1$ are reflected across the line $s$ at the points $B_2$ and $B_1$, the vector
$\overrightarrow{C_1C_2}=\overrightarrow{A_2A_1}
+\overrightarrow{B_2B_1}$ is collinear with the line $s$ (see
\ref{izoSimVekt}). Similarly, the vector $\overrightarrow{C_2C_3}$
is collinear with the line $s$, so the points $C_1$, $C_2$, and $C_3$
lie on the same line parallel to the line $s$.
 \kdokaz



%________________________________________________________________________________
 \poglavje{Translations} \label{odd6Transl}

 So far we have considered isometries that have at least one fixed point. Of the direct isometries we had the identity and rotation, and of the indirect ones, the basic reflection. Now we will introduce a new type of isometries that have no fixed points.

  Let $\overrightarrow{v}$
 be an arbitrary vector of the Euclidean plane ($\overrightarrow{v}\neq \overrightarrow{0}$). The transformation of this plane, in which the
point $X$ is mapped to such a point $X'$, that $\overrightarrow{XX'}=\overrightarrow{v}$, is called the \index{translacija} \pojem{translation} or the \index{vzporedni premik} \pojem{parallel shift} for
the vector $\overrightarrow{v}$, and is denoted by $\mathcal{T}_{\overrightarrow{v}}$ (Figure \ref{sl.izo.6.6.1.pic}). The vector  $\overrightarrow{v}$ is\index{vektor!translacije} \pojem{the vector of translation}.

\begin{figure}[!htb]
\centering
\input{sl.izo.6.6.1.pic}
\caption{} \label{sl.izo.6.6.1.pic}
\end{figure}

We will now prove the first basic properties of translation.

\bizrek \label{translEnaki}
        Two translations are equal if and only
        the vectors of these translations are equal, i.e.
        $$\mathcal{T}_{\overrightarrow{v}}=
        \mathcal{T}_{\overrightarrow{u}}\Leftrightarrow \overrightarrow{v}=\overrightarrow{u}.$$
        \eizrek

 \textbf{\textit{Proof.}} The part ($\Leftarrow$) is trivial. We prove part ($\Rightarrow$). For an arbitrary point $X$ we have
 $\mathcal{T}_{\overrightarrow{v}}(X)=
 \mathcal{T}_{\overrightarrow{u}}(X)=X'$. In this case
 $\overrightarrow{v}=\overrightarrow{u}=\overrightarrow{XX'}$.
  \kdokaz


        \bizrek
        A translation has no fixed points.
        \eizrek

\textbf{\textit{Proof.}} If $X$ is a fixed point of the translation  $\mathcal{T}_{\overrightarrow{v}}$ or $\mathcal{T}_{\overrightarrow{v}}(X)=X$, then $\overrightarrow{v}=\overrightarrow{XX}=\overrightarrow{0}$, which according to the definition of translation is not possible.
\kdokaz


        \bizrek
        The inverse transformation of a translation is a translation
        with the opposite vector, i.e.
         $\mathcal{T}^{-1}_{\overrightarrow{v}}=
         \mathcal{T}_{-\overrightarrow{v}}$
        \eizrek

\textbf{\textit{Proof.}} Let $Y$ be an arbitrary point and $X$ such a point that $\overrightarrow{XY}=\overrightarrow{v}$. From this it follows that $\mathcal{T}_{\overrightarrow{v}}(X)=Y$. From $\overrightarrow{YX}=-\overrightarrow{v}$ it follows that $\mathcal{T}_{-\overrightarrow{v}}(Y)=X$. Because $\mathcal{T}^{-1}_{\overrightarrow{v}}(Y)=X$, it follows that $\mathcal{T}^{-1}_{\overrightarrow{v}}(Y)=
 \mathcal{T}_{-\overrightarrow{v}}(Y)$. Because this is true for an arbitrary point $Y$, we have $\mathcal{T}^{-1}_{\overrightarrow{v}}=
 \mathcal{T}_{-\overrightarrow{v}}$.
  \kdokaz

        \bizrek
         Translations are isometries of the plane.
         \eizrek


\begin{figure}[!htb]
\centering
\input{sl.izo.6.6.2.pic}
\caption{} \label{sl.izo.6.6.2.pic}
\end{figure}

\textbf{\textit{Proof.}} Let $\mathcal{T}_{\overrightarrow{v}}$ be an arbitrary translation. From the definition it is clear that it represents a bijective mapping. It remains to be proven that for any points $X$ and $Y$
 and their images $\mathcal{T}_{\overrightarrow{v}}:X,Y\mapsto X', Y'$ it holds that
 $X'Y'\cong XY$ (Figure \ref{sl.izo.6.6.2.pic}).
 From $\overrightarrow{XX'}=\overrightarrow{v}$ and $\overrightarrow{YY'}=\overrightarrow{v}$ it follows that $\overrightarrow{XX'}=\overrightarrow{YY'}$. By \ref{vektABCD_ACBD} it is then also true that $\overrightarrow{XY}=\overrightarrow{X'Y'}$ or $X'Y'\cong XY$.
  \kdokaz

We will now show that, just like a rotation, any translation can be expressed as
the composition of two basic reflections.



        \bizrek \label{translKom2Zrc}
          Any translation $\mathcal{T}_{\overrightarrow{v}}$
         can be expressed as the product of two reflections
          $\mathcal{S}_p$ and $\mathcal{S}_q$ where
          $p$ and $q$  are arbitrary parallel lines with a common
          perpendicular line ($P\in p$ and $Q\in q$)
          such that $\overrightarrow{v}
         =2\overrightarrow{PQ}$, i.e.
         $$\mathcal{T}_{\overrightarrow{v}}=\mathcal{S}_q\circ\mathcal{S}_p
         \hspace*{1mm} \Leftrightarrow  \hspace*{1mm} p\parallel q \hspace*{1mm}
         \wedge\hspace*{1mm} \overrightarrow{v}
         =2\overrightarrow{PQ}\hspace*{1mm} (PQ\perp p,\hspace*{1mm}
          P\in p,\hspace*{1mm} Q\in q).$$
        \eizrek


\begin{figure}[!htb]
\centering
\input{sl.izo.6.6.3.pic}
\caption{} \label{sl.izo.6.6.3.pic}
\end{figure}

 \textbf{\textit{Proof.}}  (Figure \ref{sl.izo.6.6.3.pic})

($\Leftarrow$) Let us assume that $p\parallel q$, $\overrightarrow{v}=2\overrightarrow{PQ}$, $PQ\perp p$,
         $P\in p$ and $Q\in q$.
 We need to prove that
     $\mathcal{T}_{\overrightarrow{v}}(X)=
     \mathcal{S}_q\circ\mathcal{S}_p(X)$ for any point $X$ of this plane.
 Let $\mathcal{S}_q\circ\mathcal{S}_p(X)=X'$. We will prove that also $\mathcal{T}_{\overrightarrow{v}}(X)=X'$ or $\overrightarrow{XX'}= \overrightarrow{v}$.
 We will mark
$\mathcal{S}_p(X)=X_1$. In this case it is clear that $\mathcal{S}_q(X_1)=X'$. Because of this, first the line $XX'$ is the perpendicular
bisector of the parallels $p$ and $q$, which means that the vector $\overrightarrow{XX'}$ is parallel to the vector $\overrightarrow{v}$. No matter the position of the point $X$, in
every case the length of the line $XX'$ is twice as long as the distance between
the parallels $p$ and $q$, which is equal to the length of the line $PQ$. So $\overrightarrow{XX'} =2\overrightarrow{PQ}= \overrightarrow{v}$. With this we have proven that
$\mathcal{T}_{\overrightarrow{v}}(X)=X'$.

($\Rightarrow$) Let now $\mathcal{T}_{\overrightarrow{v}}=\mathcal{S}_q\circ\mathcal{S}_p$.
The parallels $p$ and $q$ are in this case parallel, because otherwise the intersection point of these two lines would be a fixed point of the composition $\mathcal{S}_q\circ\mathcal{S}_p$, but the translation $\mathcal{T}_{\overrightarrow{v}}$ has no fixed points.
From the first part of the proof ($\Leftarrow$) it follows that the composition $\mathcal{S}_q\circ\mathcal{S}_p$ is equal to the translation $\mathcal{T}_{\overrightarrow{v_1}}$ for the vector $\overrightarrow{v_1}$, where $\overrightarrow{v_1}=2\overrightarrow{PQ}$, $PQ\perp p$,
         $P\in p$ and $Q\in q$. But from
$\mathcal{T}_{\overrightarrow{v}}=
\mathcal{S}_q\circ\mathcal{S}_p=\mathcal{T}_{\overrightarrow{v_1}}$
 it follows that $\overrightarrow{v}=\overrightarrow{v_1}$ (from statement \ref{translEnaki}) or $\overrightarrow{v}=2\overrightarrow{PQ}$.
  \kdokaz

From the previous statement it follows that the translation as the composition
of two indirect isometries is a direct isometry.


        \bizrek
         A translation is a direct isometry.
        \eizrek

We will prove that we can represent a translation as
the composition of two central reflections.

\bizrek \label{transl2sred}
The product of two half-turns is a translation, such that
$$\mathcal{S}_Q\circ\mathcal{S}_P=
\mathcal{T}_{2\overrightarrow{PQ}}.$$
\eizrek

\begin{figure}[!htb]
\centering
\input{sl.izo.6.6.4.pic}
\caption{} \label{sl.izo.6.6.4.pic}
\end{figure}

\textbf{\textit{Proof.}} Let $X$ be an arbitrary point, $\mathcal{S}_Q\circ\mathcal{S}_P(X)=X'$ and $\mathcal{S}_P(X)=X_1$ (Figure \ref{sl.izo.6.6.4.pic}). Then $\mathcal{S}_Q(X_1)=X'$. By the definition of central reflections, points $P$ and $Q$ are the centers of the lines $XX_1$ and $X_1X'$, which means that the line $PQ$ is the median of the triangle $XX_1X'$ for the base $XX'$. By Theorem \ref{srednjicaTrikVekt},
$\overrightarrow{XX'}=2\overrightarrow{PQ}$, so $\mathcal{T}_{2\overrightarrow{PQ}}(X)=X'=
\mathcal{S}_Q\circ\mathcal{S}_P(X)$. Since this is true for every point $X$, $\mathcal{S}_Q\circ\mathcal{S}_P=
\mathcal{T}_{2\overrightarrow{PQ}}$.
\kdokaz

\bizrek \label{translKomp}
The product of two translations is a translation for the vector
that is the sum of the vectors of these two translations, i.e.
$$\mathcal{T}_{\overrightarrow{u}}\circ
\mathcal{T}_{\overrightarrow{v}}=
\mathcal{T}_{\overrightarrow{v}+\overrightarrow{u}}.$$
\eizrek

\begin{figure}[!htb]
\centering
\input{sl.izo.6.6.5.pic}
\caption{} \label{sl.izo.6.6.5.pic}
\end{figure}

\textbf{\textit{Proof.}}  (Figure \ref{sl.izo.6.6.5.pic}).
 Let $P$ be an arbitrary point, $Q$ such a point that
 $\overrightarrow{PQ} = \frac{1}{2}\overrightarrow{v}$,
and $R$ a point for which
$\overrightarrow{QR} = \frac{1}{2}\overrightarrow{u}$.
 Because $\overrightarrow{v} + \overrightarrow{u}
 = 2\overrightarrow{PQ} + 2\overrightarrow{QR} = 2\overrightarrow{PR}$,
 by the previous statement \ref{transl2sred} it follows:
 $$\mathcal{T}_{\overrightarrow{u}}\circ
        \mathcal{T}_{\overrightarrow{v}}=
        \mathcal{T}_{2\overrightarrow{QR}}\circ
        \mathcal{T}_{2\overrightarrow{PQ}}=
        \mathcal{S}_R\circ\mathcal{S}_Q\circ
        \mathcal{S}_Q\circ\mathcal{S}_P=
        \mathcal{S}_R\circ\mathcal{S}_P=
        \mathcal{T}_{2\overrightarrow{PR}}=
        \mathcal{T}_{\overrightarrow{v}+\overrightarrow{u}},$$ which was to be proven. \kdokaz

 The consequence of the proven is the following statement.

        \bizrek
       The product of translations is commutative, i.e.
        $$\mathcal{T}_{\overrightarrow{u}}\circ
        \mathcal{T}_{\overrightarrow{v}}=
        \mathcal{T}_{\overrightarrow{v}}\circ
        \mathcal{T}_{\overrightarrow{u}}$$
        \eizrek

  \textbf{\textit{Proof.}} By the previous statement \ref{translKomp} it is:
  $$\mathcal{T}_{\overrightarrow{u}}\circ
        \mathcal{T}_{\overrightarrow{v}}=
        \mathcal{T}_{\overrightarrow{v}+\overrightarrow{u}}=
        \mathcal{T}_{\overrightarrow{u}+\overrightarrow{v}}=
        \mathcal{T}_{\overrightarrow{v}}\circ
        \mathcal{T}_{\overrightarrow{u}},$$ which was to be proven. \kdokaz


Similarly to the basic reflection, rotation and central reflection, also for translation the statement about
transmutation\index{transmutation!of translation} holds.

\bizrek \label{izoTransmTrans}
For an arbitrary  translation
$\mathcal{T}_{2\overrightarrow{PQ}}$
 and an arbitrary isometry $\mathcal{I}$ is
$$\mathcal{I}\circ
\mathcal{T}_{2\overrightarrow{PQ}}\circ\mathcal{I}^{-1}=
\mathcal{T}_{2\overrightarrow{\mathcal{I}(P)\mathcal{I}(Q)}}.$$
\eizrek

\textbf{\textit{Proof.}}  By  \ref{transl2sred} we can write the translation
 $\mathcal{T}_{2\overrightarrow{PQ}}$
 as the composite of two central
reflections, namely
$\mathcal{T}_{2\overrightarrow{PQ}}=\mathcal{S}_Q\circ\mathcal{S}_P$.
If we use  the transmutation theorem for central reflections \ref{izoTransmSredZrc}, we get:
 \begin{eqnarray*}
  \mathcal{I}\circ
        \mathcal{T}_{2\overrightarrow{PQ}}\circ\mathcal{I}^{-1}&=&
        \mathcal{I}\circ\mathcal{S}_Q\circ\mathcal{S}_P\circ\mathcal{I}=\\
        &=&
        \mathcal{I}\circ\mathcal{S}_Q\circ\mathcal{I}^{-1}
        \circ\mathcal{I}\circ\mathcal{S}_P\circ\mathcal{I}^{-1}=\\
        &=&\mathcal{S}_{\mathcal{I}(Q)}\circ\mathcal{S}_{\mathcal{I}(P)}=\\
        &=&\mathcal{T}_{2\overrightarrow{\mathcal{I}(P)\mathcal{I}(Q)}},
 \end{eqnarray*}
  which is what needed to be proven. \kdokaz



        \bizrek \label{izoKompTranslRot}
        The product of a translation and a rotation (also a rotation and a translation)
        is a rotation with the same angle.
        \eizrek

\begin{figure}[!htb]
\centering
\input{sl.izo.6.6.5a.pic}
\caption{} \label{sl.izo.6.6.5a.pic}
\end{figure}

\textbf{\textit{Proof.}} Let $\mathcal{I}=\mathcal{R}_{S,\omega}\circ
 \mathcal{T}_{\overrightarrow{v}}$ be the composition of the translation
 $\mathcal{T}_{\overrightarrow{v}}$ and the rotation
 $\mathcal{R}_{S,\omega}$. Let $b$ be a line passing through the point
 $S$ and perpendicular to the vector $\overrightarrow{v}$
 (Figure \ref{sl.izo.6.6.5a.pic}). According to
 \ref{translKom2Zrc} for a given line $a$, $a\parallel b$, it holds that
 $\mathcal{T}_{\overrightarrow{v}}=\mathcal{S}_b\circ
 \mathcal{S}_a$. Let $c$ be a line passing through the point $S$, for which
 $\measuredangle b,c=\frac{1}{2}\omega$. According to
 \ref{rotacKom2Zrc} is $\mathcal{R}_{S,\omega}=\mathcal{S}_c\circ
 \mathcal{S}_b$. Therefore:
  $$\mathcal{I}=\mathcal{R}_{S,\omega}\circ
 \mathcal{T}_{\overrightarrow{v}}=\mathcal{S}_c\circ
 \mathcal{S}_b\circ\mathcal{S}_b\circ
 \mathcal{S}_a=\mathcal{S}_c\circ
 \mathcal{S}_a.$$
 By Playfair's axiom, the lines $a$ and $c$ intersect in some point
 $S_1$, according to \ref{KotiTransverzala} is
 $\measuredangle a,c=\measuredangle b,c=\frac{1}{2}\omega$.
 If we use \ref{rotacKom2Zrc} once again, we get:
$$\mathcal{R}_{S,\omega}\circ
 \mathcal{T}_{\overrightarrow{v}}=\mathcal{S}_c\circ
 \mathcal{S}_a=\mathcal{R}_{S_1,\omega}.$$
 In a similar way it can be proven that the composition
  of the rotation and
        translation $\mathcal{T}_{\overrightarrow{v}}
        \circ\mathcal{R}_{S,\omega}$ is a rotation for the same angle.
 \kdokaz

In the following we will consider the use of translation. First, we
will see how we can use translation in planning tasks.


             \bzgled
            Let $A$ and $B$ be interior points of an angle $pOq$.
            Construct points $C$ and $D$ on the sides $p$ and $q$
            such that the quadrilateral $ABCD$ is a parallelogram.
           \ezgled

\begin{figure}[!htb]
\centering
\input{sl.izo.6.6.6.pic}
\caption{} \label{sl.izo.6.6.6.pic}
\end{figure}

\textbf{\textit{Solution.}}  (Figure \ref{sl.izo.6.6.6.pic})
Since $ABCD$ is a parallelogram, $\overrightarrow{AB}= \overrightarrow{DC} = \overrightarrow{v}$.
 Therefore, $\mathcal{T}_{\overrightarrow{v}}(D)= C$. From the given
conditions, the point $D$ lies on the line segment $p$, so its image $C$ under
the translation $\mathcal{T}_{\overrightarrow{v}}$ lies on the image $p'$ of the line segment $p$ under this translation (the image $p'$ can be obtained by drawing the image $O'$ of the point $O$, and then the parallel line segment of the line segment $p$ from the point $O'$). Since the point $C$ also lies on the line segment $q$, we can draw it from the condition $C\in p'\cap q$. The number of solutions to the task depends on whether the line segments $p'$ and $q$ have any common points. Since the line segments $p$ and $q$ are not parallel (it is the angle between them), the line segments $p'$ and $q$ do not have more than one common point. Therefore, the task has one or no solution.
 \kdokaz



        \bzgled
          Let $AB$ be a chord of a circle $k$, $P$ and $Q$ points of this circle lying on the
        same side of the line $AB$ and $d$ a line segment in the plane.
        Construct a point $L$ on the circle
         $k$ such that $XY\cong d$, where $X$ and $Y$
         are intersections of the lines $LP$ and $LQ$ with the chord $AB$.
        \ezgled


\begin{figure}[!htb]
\centering
\input{sl.izo.6.6.7.pic}
\caption{} \label{sl.izo.6.6.7.pic}
\end{figure}

\textbf{\textit{Solution.}}  (Figure \ref{sl.izo.6.6.7.pic})
Although the points $X$ and $Y$ are unknown, the vector
$\overrightarrow{v}=\overrightarrow{XY}$ is known, which has the same length
as the distance $d$ and is parallel and has the same direction as the vector
$\overrightarrow{AB}$. Also, as $\omega=\angle PLQ$ is known,
because it is the central angle for the chord $PQ$ (statement \ref{ObodKotGMT}). Let
$P'=\mathcal{T}_{\overrightarrow{v}}(P)$. Because
$\overrightarrow{PP'}=\overrightarrow{v}=\overrightarrow{XY}$, the quadrilateral $PP'YX$ is a parallelogram, so the angles $P'YQ$ and $PLQ$
are complementary (statement \ref{KotaVzporKraki}). We can therefore
construct the point $Y$ as the intersection of the chord $AB$ with the appropriate chord
$P'Q$ and central angle $\omega$. Then
$X=\mathcal{T}^{-1}_{\overrightarrow{v}}(Y)$.
 \kdokaz



        \bzgled \footnote{Predlog za MMO 1996. (SL 10.)}
        Let $H$ be the orthocentre of a triangle $ABC$ and $P$ the point lying on
        the circumcircle of this triangle different from its vertices.
        $E$ is the foot of the altitude from the vertex $B$ of the triangle $ABC$.
        Suppose that the quadrilaterals $PAQB$ and $PARC$ are parallelograms. The lines
        $AQ$ and $HR$ intersect at a point $X$. Prove that $EX\parallel AP$.
        \ezgled

\begin{figure}[!htb]
\centering
\input{sl.izo.6.6.8.pic}
\caption{} \label{sl.izo.6.6.8.pic}
\end{figure}

 \textbf{\textit{Solution.}} We mark with $O$ the center of the circumscribed
 circle of the triangle $ABC$ and with $\mathcal{T}_{\overrightarrow{PA}}$ the translation for the vector $\overrightarrow{PA}$ (Figure \ref{sl.izo.6.6.8.pic}).

Since $PAQB$ and $PARC$ are parallelograms, $\overrightarrow{BQ}=\overrightarrow{PA}=\overrightarrow{CR}$.
This means that the translation $\mathcal{T}_{\overrightarrow{PA}}$ maps the triangle $BPC$
into the triangle $QAR$, the altitude point $H_1$ of the triangle $BPC$ into the altitude point $H'_1$
of the triangle $QAR$. We prove that $H'_1=H$ or $H=\mathcal{T}_{\overrightarrow{PA}}(H_1)$
or $\overrightarrow{H_1H}=\overrightarrow{PA}$. If we use Hamilton's theorem \ref{Hamilton}
for the triangle $ABC$ and $PBC$, which have a common center of the inscribed circle, we get
$\overrightarrow{OA}+\overrightarrow{OB}+\overrightarrow{OC}
=\overrightarrow{OH}$ and $\overrightarrow{OP}+\overrightarrow{OB}+\overrightarrow{OC}
=\overrightarrow{OH_1}$. From this it follows first $\overrightarrow{OH}-\overrightarrow{OA}=
\overrightarrow{OH_1}-\overrightarrow{OP}$, then $\overrightarrow{OH}-\overrightarrow{OH_1}=
\overrightarrow{OA}-\overrightarrow{OP}$ or $\overrightarrow{H_1H}=\overrightarrow{PA}$.

So $H$ is the altitude point of the triangle $ARQ$, which means that $RH\perp AQ$ or $\angle AXH=90^0$.
Since $\angle AEH=90^0$ as well, the points $X$ and $E$ lie on the circle
with diameter $AH$ (theorem \ref{TalesovIzrKrozObrat}), so $AXEH$ is a trapezoid. So:
\begin{eqnarray*}
\angle AXE&=&180^0-\angle AHE \hspace*{4mm} \textrm{(theorem \ref{TetivniPogoj})}\\
&=&180^0-\angle ACB \hspace*{4mm} \textrm{(theorem \ref{KotaPravokKraki})}\\
&=&180^0-\angle APB \hspace*{4mm} \textrm{(theorem \ref{ObodObodKot})}\\
&=&\angle QAP \hspace*{4mm} \textrm{(by  theorem \ref{paralelogram}, because }PAQB\textrm{ is a parallelogram)}
\end{eqnarray*}

By theorem \ref{KotiTransverzala}, $EX\parallel AP$.
\kdokaz

%________________________________________________________________________________
 \poglavje{Composition of Two Rotations} \label{odd6KompRotac}

In the previous sections, we have already considered the composition of some mappings.
We have found that the composition of two basic reflections is a direct isometry, namely the identity or a rotation or a translation, depending on whether the lines are equal or intersect or are parallel. In the previous section, we proved that the composition of two central reflections and also the composition of two translations is a translation.

In this section, we will investigate the composition of two rotations. We start with the following statement.


        \bizrek \label{rotacKomp2rotac}
        Let $R_{A,\alpha}$ and $R_{B,\beta}$ be rotations of the plane
         and $\mathcal{I}=R_{B,\beta}\circ R_{A,\alpha}$.
        Then:
        \begin{enumerate}
            \item if $\alpha+\beta\notin\{k\cdot 360^0;k\in \mathbb{Z}\}$ and $A=B$,
          the product $\mathcal{I}$ is a rotation for the angle $\alpha+\beta$ with the centre $A$,
            \item if $\alpha+\beta\notin\{k\cdot 360^0;k\in \mathbb{Z}\}$ and $A\neq B$,
          the product $\mathcal{I}$ is a rotation for the angle $\alpha+\beta$
          with the centre $C$,\\ where $\measuredangle CAB=\frac{1}{2}\alpha$
          and $\measuredangle ABC=\frac{1}{2}\beta$,
            \item if $\alpha+\beta\in\{k\cdot 360^0;k\in \mathbb{Z}\}$ and $A=B$,
          the product $\mathcal{I}$ is the identity map,
            \item if $\alpha+\beta\in\{k\cdot 360^0;k\in \mathbb{Z}\}$ and $A\neq B$,
          the product $\mathcal{I}$ is a translation.
        \end{enumerate}
        \eizrek

\begin{figure}[!htb]
\centering
\input{sl.izo.6.7.1.pic}
\caption{} \label{sl.izo.6.7.1.pic}
\end{figure}

 \textbf{\textit{Proof.}}  (Figure \ref{sl.izo.6.7.1.pic})

Let us assume that the centres $A$ and $B$ are different (the case
$A=B$ is trivial).

Let $p$ and $q$ be such lines and planes, that
$\measuredangle p,AB=\frac{1}{2}\alpha$ and $\measuredangle
AB,q=\frac{1}{2}\beta$. If we decompose each of the rotations with the help of
reflections on the axes (statement \ref{rotacKom2Zrc}), we get:
$$\mathcal{I}=R_{B,\beta}\circ R_{A,\alpha}=
\mathcal{S}_q\circ\mathcal{S}_{AB}\circ\mathcal{S}_{AB}\circ
\mathcal{S}_p=\mathcal{S}_q\circ
\mathcal{S}_p.$$

If $\alpha+\beta\in\{0^0,360^0,-360^0\}$,
then $\frac{1}{2}\left(\alpha+\beta\right)\in\{0^0,180^0,-180^0\}$,
which means that the lines $p$ and $q$ are parallel (statement \ref{KotiTransverzala}).
 In this case, the composition $\mathcal{I}=\mathcal{S}_q\circ\mathcal{S}_p$
 is a translation (statement \ref{translKom2Zrc}).

If $\alpha+\beta\notin\{0^0,360^0,-360^0\}$,
then $\frac{1}{2}\left(\alpha+\beta\right)\notin\{0^0,180^0,-180^0\}$
and the lines $p$ and $q$ intersect in some point $C$ (statement \ref{KotiTransverzala}).
As $\measuredangle pCq$ is the exterior angle of the triangle $ABC$, it is equal to the sum
of the angles $\measuredangle p,AB$ and $\measuredangle AB,q$ (statement \ref{zunanjiNotrNotr}).
Therefore $\measuredangle pCq=\frac{1}{2}\left(\alpha+\beta\right)$.
The composition
$\mathcal{I}=\mathcal{S}_q\circ\mathcal{S}_p$ in this case represents
a rotation (statement \ref{rotacKom2Zrc}) $\mathcal{R}_{C,2\measuredangle pCq}=\mathcal{R}_{C,\alpha+\beta}$.
 \kdokaz

 In the following we will see the use of the previous statement in various tasks.


        \bzgled
        Let $P$, $Q$ and $R$ be non-collinear points.
        Construct a triangle $ABC$ such that $ARB$, $BPC$ and $CQA$ are regular triangles
         with the same orientation.
        \ezgled

\begin{figure}[!htb]
\centering
\input{sl.izo.6.7.2.pic}
\caption{} \label{sl.izo.6.7.2.pic}
\end{figure}

 \textbf{\textit{Solution.}}  (Figure \ref{sl.izo.6.7.2.pic})

If $A$, $B$ and $C$ are points that satisfy the given
conditions, it holds:
$$\mathcal{R}_{Q,60^0} \circ \mathcal{R}_{R,60^0} \circ \mathcal{R}_{P,60^0}(C)=C.$$
But according to the previous statement \ref{rotacKomp2rotac}
the composite $\mathcal{R}_{Q,60^0} \circ \mathcal{R}_{R,60^0} \circ \mathcal{R}_{P,60^0}$
is a central symmetry (because $60^0+60^0+60^0=180^0$) with a
fixed
point $C$, so:
$$\mathcal{R}_{Q,60^0} \circ \mathcal{R}_{R,60^0} \circ \mathcal{R}_{P,60^0}=\mathcal{S}_C.$$
The vertex $C$ can therefore be constructed as the center of the line segment $XX'$,
where $X'$ is the image of an arbitrary point $X$ under the composite
$\mathcal{R}_{Q,60^0} \circ \mathcal{R}_{R,60^0} \circ
\mathcal{R}_{P,60^0}$. Then $B=\mathcal{R}_{P,60^0}(C)$ and
$A=\mathcal{R}_{R,60^0}(B)$.
 \kdokaz



        \bzgled \label{rotKompZlato}
        Let $BALK$ and $ACPQ$ be squares with the same orientation and
        $Z$ the midpoint of the line segment $PK$.
        Prove that $BZC$ is an isosceles right triangle with the hypotenuse $BC$.
        \ezgled

\begin{figure}[!htb]
\centering
\input{sl.izo.6.7.3.pic}
\caption{} \label{sl.izo.6.7.3.pic}
\end{figure}

 \textbf{\textit{Solution.}}  (Figure \ref{sl.izo.6.7.3.pic})

 Let $\mathcal{I}=\mathcal{R}_{B,90^0}\circ \mathcal{R}_{C,90^0}$. Because
$90°+90°= 180^0$, according to statement \ref{rotacKomp2rotac}
the isometry $\mathcal{I}$ represents the central symmetry $\mathcal{S}_Y$,
where $\measuredangle YCB=\frac{1}{2}\cdot 90^0=45^0$ and $\measuredangle CBY=\frac{1}{2}\cdot 90^0=45^0$.
Therefore, the point $Y$ is the vertex of the isosceles right triangle $BYC$ with the hypotenuse $BC$. But since also
$\mathcal{S}_Y(P)=\mathcal{I}(P)=K$, it follows that $Y=Z$.
 \kdokaz

\bzgled \label{rotZgl1}
On each side of a quadrilateral $ABCD$ squares
 $BALK$, $CBMN$, $DCSR$ and $ADPQ$
are externally erected.
Let $E$, $F$, $G$ and $H$ be the midpoints of the line segments
$PK$, $MR$, $LN$ and $SQ$.
  Prove that the quadrilaterals $BFDE$ and $AGCH$ are also squares.
\ezgled

\begin{figure}[!htb]
\centering
\input{sl.izo.6.7.4.pic}
\caption{} \label{sl.izo.6.7.4.pic}
\end{figure}

 \textbf{\textit{Solution.}} By the previous statement \ref{rotKompZlato}
 the triangles $DEB$ and $BFD$ are isosceles and right with a common
 hypotenuse $BD$ (Figure \ref{sl.izo.6.7.4.pic}). From this it follows that the quadrilateral
        $BFDE$  is a square. In the same way we get that the quadrilateral $AGCH$ is a square.
 \kdokaz



        \bzgled \label{rotZgl2}
        Let $A_1$, $B_1$ and $C_1$ be the centres of squares
        externally erected on the sides $BC$, $AC$ and $AB$ of an arbitrary
         triangle $ABC$
         and $P$ the midpoint of the side $AC$. Prove that:
        \begin{enumerate}
          \item  $C_1PA_1$ is an isosceles right triangle,
          \item $C_1B_1$ and $A_1A$ are perpendicular and congruent line segments,
          \item the line segments $AA_1$, $BB_1$ and $CC_1$  intersect at
          a single point\footnote{To je drugi od treh planimetričnih
          izrekov, ki jih je objavil italijanski matematik
          \index{Bellavitis, G.} \textit{G.
            Bellavitis} (1803--1880) na kongresu v Milanu 1844.}.
        \end{enumerate}
        \ezgled


\textbf{\textit{Solution.}}

\textit{1)} Let $\mathcal{I}= \mathcal{R}_{C_1,90^0}
\mathcal{R}_{A_1,90^0}$ (Figure \ref{sl.izo.6.7.5.pic}). Because $90^0+90^0= 180^0$, the isometry $\mathcal{I}$ represents the central reflection $\mathcal{S}_Y$, where the point $Y$ is the vertex of an isosceles right triangle $C_1YA_1$ with the hypotenuse $C_1A_1$ (statement \ref{rotacKomp2rotac}). Because $\mathcal{S}_Y(C)=\mathcal{I}(C)=A$, it follows that $Y=P$.


\begin{figure}[!htb]
\centering
\input{sl.izo.6.7.5.pic}
\caption{} \label{sl.izo.6.7.5.pic}
\end{figure}



\textit{2)} With the rotation $\mathcal{R}_{P,90^0}$, the line segment $B_1C_1$ is mapped to the line segment $AA_1$ (Figure \ref{sl.izo.6.7.5a.pic}). Therefore, the line segments are congruent and perpendicular (statement \ref{rotacPremPremKot}).

\begin{figure}[!htb]
\centering
\input{sl.izo.6.7.5a.pic}
\caption{} \label{sl.izo.6.7.5a.pic}
\end{figure}

\textit{3)} From the proven part \textit{(2)} it follows that the lines $AA_1$, $BB_1$, $CC_1$ are the altitudes of the triangle $A_1B_1C_1$ (Figure \ref{sl.izo.6.7.5a.pic}), so they intersect in its altitude point (statement \ref{VisinskaTocka}).
 \kdokaz



        \bzgled \label{rotZgl3}
        Let $ABCD$ and $AB_1C_1D_1$ be squares with the same orientation,
         $P$ and $Q$  the midpoints of the line segments $BD_1$ and $B_1D$.
        Suppose that $O$ and $S$ are the centres of these squares.
        Prove that the quadrilateral $POQS$ is also a square.
        \ezgled



\begin{figure}[!htb]
\centering
\input{sl.izo.6.7.6.pic}
\caption{} \label{sl.izo.6.7.6.pic}
\end{figure}

\textbf{\textit{Solution.}}
 The statement is obvious if we use part \textit{(1)}
  from the previous example \ref{rotZgl2} twice (Figure \ref{sl.izo.6.7.6.pic}).
 \kdokaz



 If we connect the facts from examples \ref{rotZgl1} and \ref{rotZgl3},
 we get the following statement (Figure \ref{sl.izo.6.7.4a.pic}).

\begin{figure}[!htb]
\centering
\input{sl.izo.6.7.4a.pic}
\caption{} \label{sl.izo.6.7.4a.pic}
\end{figure}



        \bzgled \label{rotZgl4}
        Let $BALK$, $CBMN$, $DCSR$ and $ADPQ$ be the
        squares externally erected on the four sides of an arbitrary quadrilateral $ABCD$.
         Then  all quadrilaterals defined by the following vertices are also squares:
        \begin{enumerate}
          \item the point $B$, the midpoint of the line segment $MR$, the point $D$ and the midpoint of the line segment $PK$,
          \item the point $A$, the midpoint of the line segment $LN$, the point $C$ and the midpoint of the line segment $SQ$,
          \item the midpoints of the line segments $QL$, $LB$, $BD$ and $DQ$,
          \item the midpoints of the line segments $KM$, $MC$, $CA$ and $AK$,
          \item the midpoints of the line segments $NS$, $SD$, $DB$ and $BN$,
          \item the midpoints of the line segments $RP$, $PA$, $AC$ and $CR$.
        \end{enumerate}

        \ezgled

      In the next example we will use the composition of rotations in a situation where the sum of the angles of these rotations is equal to $360^0$.


        \bzgled \label{rotZgl5}
        Let $ABC$ and $A'B'C'$ be isosceles triangles of the same orientation with the bases $BC$ and
        $B'C'$ and $\angle BAC\cong\angle B'A'C'=\alpha$.
        Suppose that $A_0$, $B_0$ and $C_0$ are the midpoints of the line segments $AA'$, $BB'$ and $CC'$.
        Prove that $A_0B_0C_0$ is also an isosceles triangle
        with the base $B_0C_0$ and $\angle B_0A_0C_0\cong\alpha$.
        \ezgled

\begin{figure}[!htb]
\centering
\input{sl.izo.6.7.7.pic}
\caption{} \label{sl.izo.6.7.7.pic}
\end{figure}

 \textbf{\textit{Solution.}}  (Figure \ref{sl.izo.6.7.7.pic}).

Let $\mathcal{I}= \mathcal{S}_{C_0}
\circ \mathcal{R}_{A',\alpha} \circ \mathcal{S}_{B_0} \circ
\mathcal{R}_{A,-\alpha}$. The composite $\mathcal{I}$ is a direct isometry. Because the sum of the corresponding
angles of rotation is equal to $360^0$ and $\mathcal{I}(C)=C$, it must be
$\mathcal{I}=\mathcal{E}$ (statement \ref{rotacKomp2rotac}). If we use the  statement about
the transmutation of rotation \ref{izoTransmRotac}, we get
 $\mathcal{S}_{A_0} \circ\mathcal{R}_{A,\alpha} \circ \mathcal{S}_{A_0} =\mathcal{R}_{A',\alpha}$.
 From what has been proven and from the statement \ref{transl2sred} it follows:
 \begin{eqnarray*}
 \mathcal{E}&=&\mathcal{I}= \mathcal{S}_{C_0}
 \circ \mathcal{R}_{A',\alpha} \circ \mathcal{S}_{B_0}
 \circ \mathcal{R}_{A,-\alpha}=\\
 &=& \mathcal{S}_{C_0} \circ (\mathcal{S}_{A_0} \circ\mathcal{R}_{A,\alpha}
 \circ \mathcal{S}_{A_0}) \circ \mathcal{S}_{B_0} \circ \mathcal{R}_{A,-\alpha}=\\
 &=& (\mathcal{S}_{C_0} \circ \mathcal{S}_{A_0}) \circ\mathcal{R}_{A,\alpha}
 \circ (\mathcal{S}_{A_0} \circ \mathcal{S}_{B_0}) \circ \mathcal{R}_{A,-\alpha}=\\
 &=& \mathcal{T}_{2\overrightarrow{A_0C_0}} \circ\mathcal{R}_{A,\alpha}
 \circ \mathcal{T}_{2\overrightarrow{B_0A_0}} \circ \mathcal{R}_{A,-\alpha}
 \end{eqnarray*}
 Therefore $\mathcal{E}=\mathcal{T}_{2\overrightarrow{A_0C_0}}
 \circ\mathcal{R}_{A,\alpha} \circ \mathcal{T}_{2\overrightarrow{B_0A_0}}
 \circ \mathcal{R}_{A,-\alpha}$, or $\mathcal{T}_{2\overrightarrow{C_0A_0}}=
 \mathcal{R}_{A,\alpha} \circ \mathcal{T}_{2\overrightarrow{B_0A_0}} \circ
 \mathcal{R}_{A,-\alpha}$. If we use the statement about the transmutation of translation
 \ref{izoTransmTrans}, we get:
 $$\mathcal{T}_{2\overrightarrow{C_0A_0}}= \mathcal{R}_{A,\alpha}
 \circ \mathcal{T}_{2\overrightarrow{B_0A_0}} \circ
 \mathcal{R}_{A,-\alpha}=\mathcal{T}_{2\overrightarrow{B'_0A'_0}},$$
  where $\mathcal{R}_{A,\alpha}:A_0, B_0\mapsto A'_0, B'_0$. Therefore
  $\overrightarrow{C_0A_0}=\overrightarrow{B'_0A_0'}$ or
  $\overrightarrow{A_0C_0}=\overrightarrow{A'_0B'_0}$.
  The vector $\overrightarrow{A_0B_0}$ is transformed into the vector $\overrightarrow{A'_0B'_0}=\overrightarrow{A_0C_0}$ by the rotation $\mathcal{R}_{A,\alpha}$,
  so $|A_0B_0|=|A_0C_0|$ and $\measuredangle B_0A_0C_0=\measuredangle\overrightarrow{A_0B_0},
  \overrightarrow{A_0C_0}=\alpha$.
  \kdokaz

\bzgled \label{RotacZglVeck}
Let $A_1A_2...A_n$ and $B_1B_2...B_n$ be regular
 $n$-gons with the same orientation. Suppose that
 $S_1$, $S_2$, ..., $S_n$ are the midpoints of the line segments
$A_1B_1$, $A_2B_2$, ..., $A_nB_n$. Prove that $S_1S_2...S_n$ is also a regular $n$-gon.
\ezgled

\begin{figure}[!htb]
\centering
\input{sl.izo.6.7.8.pic}
\caption{} \label{sl.izo.6.7.8.pic}
\end{figure}

 \textbf{\textit{Solution.}}
Since $n$-gons $A_1A_2...A_n$ and $B_1B_2...B_n$
are regular and have the same orientation, $A_1A_2A_3$ and $B_1B_2B_3$
are equilateral and have the same orientation
as the triangle with the bases $A_1A_3$ and $B_1B_3$ (Figure \ref{sl.izo.6.7.8.pic}).
In addition, $\angle A_1A_2A_3\cong \angle B_1B_2B_3=\frac{(n - 2)\cdot 180^0}{n}$ (\ref{pravVeckNotrKot}).

From the previous example \ref{rotZgl5} it follows that
also  $S_1S_2S_3$ is an equilateral triangle and
$\angle S_1S_2S_3=\frac{(n - 2)\cdot 180^0}{n}$.
Therefore, the sides $S_1S_2$ and $S_2S_3$ are congruent, the interior angle at
the vertex $S_2$ is congruent to the angle of the regular $n$-gon.
We prove analogously that all
sides of the polygon $S_1S_2...S_n$ are congruent and all its
interior angles are congruent, which means that this polygon is regular.
 \kdokaz

We will consider one interesting consequence of the composition of rotations in section \ref{odd7Napoleon}.


%________________________________________________________________________________
 \poglavje{Glide Reflections} \label{odd6ZrcDrs}

So far we have learned about some types of isometries - three direct isometries
(identity, rotation and translation) and one indirect isometry (reflection in a point).
Now we will define another indirect isometry that has no fixed points.

Let $\mathcal{T}_{\overrightarrow{v}}$ be a translation
 for vector $\overrightarrow{v} = 2\overrightarrow{PQ}$
and $\mathcal{S}_{PQ}$ be a reflection over the line $PQ$. The composite
$\mathcal{S}_{PQ}\circ \mathcal{T}_{\overrightarrow{v}}$ is called
\index{glide reflection}\pojem{glide reflection} with the axis $PQ$ for vector $\overrightarrow{v} = 2\overrightarrow{PQ}$
 (Figure \ref{sl.izo.6.8.1.pic}). We denote it by
$\mathcal{G}_{2\overrightarrow{PQ}}$.


 From the definition it is clear that a glide reflection is determined by its axis and vector.

 We prove the basic properties of glide reflection.



        \bizrek \label{IzoZrcDrs1}
        Any glide reflection can be expressed as the product of three reflections.\\
        The reflection and translation in the product, as the presentation of glide reflection, commute, i.e.
         $$\mathcal{G}_{2\overrightarrow{PQ}}=
         \mathcal{S}_{PQ}\circ \mathcal{T}_{2\overrightarrow{PQ}}=
         \mathcal{T}_{2\overrightarrow{PQ}}\circ \mathcal{S}_{PQ}.$$
        \eizrek

\begin{figure}[!htb]
\centering
\input{sl.izo.6.8.2.pic}
\caption{} \label{sl.izo.6.8.2.pic}
\end{figure}

 \textbf{\textit{Proof.}} Let $\mathcal{G}_{2\overrightarrow{PQ}}=
 \mathcal{S}_{PQ}\circ \mathcal{T}_{2\overrightarrow{PQ}}$ with the axis $PQ$
 and vector $2\overrightarrow{PQ}$. Let $p$ and $q$ be the perpendiculars
  of the line $PQ$ at points $P$ and $Q$ (Figure \ref{sl.izo.6.8.2.pic}).
  Because $p,q\perp PQ$ and $p\parallel q$, we can represent the translation
  $\mathcal{T}_{2\overrightarrow{PQ}}$  as
  the composite $\mathcal{S}_q\circ\mathcal{S}_p$. In this case, the reflection $\mathcal{S}_{PQ}$
  commutes with reflections $\mathcal{S}_p$ and $\mathcal{S}_q$ (statement \ref{izoZrcKomut}). Therefore:
 $$\mathcal{G}_{2\overrightarrow{PQ}}=
 \mathcal{S}_{PQ}\circ \mathcal{T}_{2\overrightarrow{PQ}}=
 \mathcal{S}_{PQ}\circ\mathcal{S}_q\circ\mathcal{S}_p=
 \mathcal{S}_q\circ\mathcal{S}_p\circ\mathcal{S}_{PQ}=
 \mathcal{T}_{2\overrightarrow{PQ}}\circ\mathcal{S}_{PQ},$$ which had to be proven. \kdokaz

\bizrek
\label{izoZrcdrsZrcdrs}
The product of a glide reflection with itself is a translation (Figure \ref{sl.izo.6.8.3.pic}).
\eizrek

\begin{figure}[!htb]
\centering
\input{sl.izo.6.8.3.pic}
\caption{} \label{sl.izo.6.8.3.pic}
\end{figure}

 \textbf{\textit{Proof.}} We use the previous statement \ref{IzoZrcDrs1}
 and the statement \ref{translKomp}:
 $$\mathcal{G}^2_{2\overrightarrow{PQ}}=
 \mathcal{T}_{2\overrightarrow{PQ}}\circ \mathcal{S}_{PQ}\circ
 \mathcal{S}_{PQ}\circ \mathcal{T}_{2\overrightarrow{PQ}}=
 \mathcal{T}^2_{2\overrightarrow{PQ}}=
 \mathcal{T}_{4\overrightarrow{PQ}},$$ which had to be proven. \kdokaz

  From the proof of statement \ref{IzoZrcDrs1} it follows that every glide reflection can
   be represented as the composition of three basic
reflections, where the axis of one reflection is perpendicular to the axes of the other two.
A glide reflection is therefore an indirect
isometry as the composition of three
basic reflections - indirect isometries. The same fact also follows from
the definition of glide reflection, because it is the composition of one direct and one indirect isometry.

        \bizrek
        A glide reflection has no fixed points.
        \eizrek

\textbf{\textit{Proof.}} By statement \ref{IzoZrcDrs1} a glide reflection
 can be represented as the composition of three basic reflections:
$$\mathcal{G}_{2\overrightarrow{PQ}}=
 \mathcal{S}_{PQ}\circ\mathcal{S}_q\circ\mathcal{S}_p,$$
 where $p$ and $q$ are perpendicular to the line $PQ$ in points $P$ and $Q$.
 If the glide reflection $\mathcal{G}_{2\overrightarrow{PQ}}$ or
 the composition $\mathcal{S}_{PQ}\circ\mathcal{S}_q\circ\mathcal{S}_p$
 had a fixed point, $\mathcal{S}_{PQ}\circ\mathcal{S}_q\circ\mathcal{S}_p$
 as an indirect isometry would represent the basic reflection (statement \ref{izo1ftIndZrc}).
 By statement \ref{izoSop} in this case the lines $p$, $q$ and $PQ$ would belong to the same
 bundle, which is not possible. The glide reflection $\mathcal{G}_{2\overrightarrow{PQ}}$ therefore has no fixed points.
 \kdokaz

We have already used the fact from the statement \ref{IzoZrcDrs1},
that we can always represent a glide reflection as a composite of three basic
reflections, where the axes of these reflections are not in the same family. But does the converse hold
- that the composite of three basic reflections, whose axes are not from the same family, is always a glide reflection?
In connection with this, the following statement, which is also very important for the classification
of isometries, which we will carry out in the next section, will hold.


        \bizrek \label{izoZrcdrsprq}
        If lines $p$, $q$ and $r$ in the plane
        are not from the same family of lines, then product
        $\mathcal{S}_p \circ
        \mathcal{S}_q\circ \mathcal{S}_r$ is a glide reflection.
        \eizrek

\begin{figure}[!htb]
\centering
\input{sl.izo.6.8.4.pic}
\caption{} \label{sl.izo.6.8.4.pic}
\end{figure}

\textbf{\textit{Proof.}}  (Figure \ref{sl.izo.6.8.4.pic})
 The line $q$ intersects either the line $p$ or the line
$r$, because otherwise all three lines would belong to one set of parallel
lines. Without loss of generality, let $q\cap r=\{A\}$. The lines $p$,
$q$ and $r$ are not from the same set, so $A\notin p$. Let $s$
be a line that goes through the point $A$ and is perpendicular to the line $p$ at
the point $B$. The lines $s$, $q$ and $r$ belong to the same set
$\mathcal{X}_A$, so by izrek \ref{izoSop} $\mathcal{S}_r \circ
\mathcal{S}_q \circ \mathcal{S}_ s = \mathcal{S}_t$, where $t$ is also
$t\in \mathcal{X}_A$. From this it follows that $\mathcal{S}_r \circ
\mathcal{S}_q = \mathcal{S}_t \circ \mathcal{S}_s$ or (if we use
izrek \ref{izoSrZrcKom2Zrc}):
$$\mathcal{S}_r \circ \mathcal{S}_q \circ \mathcal{S}_p = \mathcal{S}_t
\circ \mathcal{S}_s \circ \mathcal{S}_p = \mathcal{S}_t \circ \mathcal{S}_B.$$
Let $C$ be the perpendicular projection of the point $B$ onto the line $t$
(in this case $B\neq C$, because otherwise the lines $s$
and $t$, and consequently $r$ and $q$, would overlap) and $b$ the line that is perpendicular to the line $BC$ at
the point $B$.
Then, by izrek \ref{izoSrZrcKom2Zrc} and \ref{translKom2Zrc}:
$$\mathcal{S}_r \circ \mathcal{S}_q \circ \mathcal{S}_p = \mathcal{S}_t \circ \mathcal{S}_B=
\mathcal{S}_t \circ \mathcal{S}_b \circ \mathcal{S}_{BC}=
\mathcal{T}_{2\overrightarrow{BC}} \circ \mathcal{S}_{BC}=
\mathcal{G}_{2\overrightarrow{BC}},$$ which was to be proven. \kdokaz

        \bzgled \label{izoZrcDrsKompSrOsn}
        Any glide reflection can be expressed as the product of a reflection and a half-turn, specifically
        $$\mathcal{G}_{2\overrightarrow{PQ}}=\mathcal{S}_q\circ \mathcal{S}_P,$$
        where $q$ is perpendicular to the line  $PQ$ in the point $Q$.
        \ezgled

By definition,
    $\mathcal{G}_{2\overrightarrow{PQ}}=\mathcal{S}_{PQ}
    \circ \mathcal{T}_{2\overrightarrow{PQ}}$. If
    we use the expression \ref{transl2sred} and \ref{izoSrZrcKom2Zrc}, it is:
 $$\mathcal{G}_{2\overrightarrow{PQ}}=\mathcal{S}_{PQ}\circ \mathcal{T}_{2\overrightarrow{PQ}}=
 \mathcal{S}_{PQ}\circ\mathcal{S}_Q\circ\mathcal{S}_P=
 \mathcal{S}_{PQ}\circ\mathcal{S}_{PQ}\circ
 \mathcal{S}_q\circ\mathcal{S}_P=
 \mathcal{S}_q\circ\mathcal{S}_P,$$ which had to be proven. \kdokaz


%________________________________________________________________________________
 \poglavje{Classification of Plane Isometries. Chasles' Theorem} \label{odd6KlasifIzo}

In the previous sections we investigated certain types of isometries.
The question arises as to whether these are the only isometries
or whether there is another type of isometry that we have not
encountered yet. In this section we will prove that the answer to this
question is negative and make a final classification of all
isometries of the plane.

First, we will consider direct isometries. So far we have mentioned
the identity, rotation and translation. We will prove that these are the only
types of direct isometries. First, let us recall that each of
these isometries can be represented as the composition of two reflections over a line. Depending on whether the axes overlap, intersect or are
parallel, we got the identity, rotation and translation. These are also the only possibilities for the mutual position of two lines in
the plane. Can all direct isometries be
represented as the composition of two reflections over a line? In that case, the three
aforementioned direct isometries would really be the only ones. We will use this idea in the following theorem.


       \bizrek \label{Chaslesov+}
      Any direct isometry can be expressed as the product of two reflections.
        The only direct isometries are
      identity map, rotations and
       translations.
      \eizrek


\textbf{\textit{Proof.}} (Figure \ref{sl.izo.6.10.1.pic})

 Let $\mathcal{I} : \mathbb{E}^2\rightarrow \mathbb{E}^2$
 be a direct isometry of the plane. We will carry out the proof according to the number
of fixed points.

\textit{1)} If a direct isometry $\mathcal{I}$ has at least two fixed points,
it is, according to \ref{izo2ftIdent}, the identity. We can represent it as the composite $\mathcal{I} = \mathcal{S}_p \circ
\mathcal{S}_p$ (\ref{izoZrcPrInvol}) for an arbitrary line
$p$.

\begin{figure}[!htb]
\centering
\input{sl.izo.6.10.1.pic}
\caption{} \label{sl.izo.6.10.1.pic}
\end{figure}

 \textit{2)} Let's assume that the isometry
$\mathcal{I}$ has exactly one fixed point $S$. Let $p$
be an arbitrary line passing
 through the point $S$. The composite $\mathcal{S}_p\circ \mathcal{I}$
 is an indirect isometry with a fixed
point $S$, therefore, according to \ref{izo1ftIndZrc}, it represents a central
reflection - for example $\mathcal{S}_q$, where the axis $q$ passes through the point
$S$. So we have $\mathcal{S}_p\circ \mathcal{I} = \mathcal{S}_q$
or $\mathcal{I}
=\mathcal{S}_p^{-1}\circ\mathcal{S}_q=\mathcal{S}_p\circ\mathcal{S}_q$
(\ref{izoZrcPrInvol}). The lines $p$ and $q$ intersect in the point
$S$ (from $p=q$ we get $\mathcal{I}=\mathcal{S}_p \circ
\mathcal{S}_p=\mathcal{E}$), therefore, according to \ref{rotacKom2Zrc},
$\mathcal{I}$ represents a rotation with the center in the point $S$.

\textit{3)} Let $\mathcal{I}$ be an isometry without fixed points.
Then for any point $A$ of this plane $\mathcal{I}(A) = A'\neq
A$. Let $p$ be the perpendicular bisector of the line segment $AA'$ and $S_p$ the reflection over the line $p$. In this case $\mathcal{S}_p\circ \mathcal{I}(A)=
\mathcal{S}_p(A')=A$. The composition $\mathcal{S} \circ
\mathcal{I}_p$ is therefore an indirect isometry with a fixed point $A$, so
according to theorem \ref{izo1ftIndZrc} it represents the reflection over some line $q$ - $\mathcal{S}_q$, where the axis of $q$ goes through the point $A$.
As in the previous example,
$\mathcal{I}=\mathcal{S}_p\circ\mathcal{S}_q$. The lines $p$ and $q$
do not intersect and are not equal, because otherwise  from the already proven
$\mathcal{I}$ it would represent the identity or a rotation and would have
at least one fixed point. So the only possibility left is that the lines
$p$ and $q$ are parallel. In this case $\mathcal{I}$ is a translation
(theorem \ref{translKom2Zrc}).
 \kdokaz

We are left with indirect isometries. The only ones mentioned so far
are the basic reflection and the glide reflection.
Are they also the only ones? We will answer this
    in the next
    theorem.



             \bizrek \label{Chaslesov-}
             Any opposite isometry is either a reflection
             either it can be represented as the product of three reflections.
               The only opposite isometries are
            reflections and
            glide reflections.
             \eizrek

\textbf{\textit{Proof.}} (Figure \ref{sl.izo.6.10.2.pic})

Let $\mathcal{I} : \mathbb{E}^2\rightarrow \mathbb{E}^2$
be an indirect isometry of the plane. We will again carry out the proof
depending on the number of fixed points.

\textit{1)} If the  isometry $\mathcal{I}$ has at least one fixed
point, $\mathcal{I}$ according to theorem \ref{izo1ftIndZrc} is the reflection over a line.

\textit{2)} Let's assume that $\mathcal{I}$ is an isometry without any fixed points.
     In this case, for any point $A$ on this plane, we have
$\mathcal{I}(A) = A'\neq A$. Let $p$ be the line of symmetry of the line segment $AA'$ and $\mathcal{S}_p$ be the reflection over the line $p$. Then
$\mathcal{S}_p\circ \mathcal{I}(A)=  \mathcal{S}_p(A')=A$. Therefore, the composition $\mathcal{S}_p \circ \mathcal{I}$ is a direct isometry with a
fixed point at $A$, which, according to the previous statement, represents a rotation or the identity (a translation has no fixed points). The other case is not possible,
because otherwise $\mathcal{I} = \mathcal{S}_p$ and the isometry $\mathcal{I}$ would have fixed points, which contradicts
the initial assumption. So the composition $\mathcal{S} \circ
\mathcal{I}_p$ is a rotation with center at $A$, which we can represent
as the composition of two reflections over the lines $q$ and $r$, which intersect
at the point $A$. Therefore, $\mathcal{S}_p\circ \mathcal{I}=\mathcal{S}_q\circ
\mathcal{S}_r$ or $\mathcal{I}=\mathcal{S}_p\circ \mathcal{S}_q\circ
\mathcal{S}_r$. Since  $A'\neq A$ and $p$ is the line of symmetry of the line segment $AA'$,
the point $A$ does not lie on the line $p$. This means that the lines $p$, $q$
and $r$ are not in the same plane. Therefore, according to \ref{izoZrcdrsprq},
$\mathcal{I}=\mathcal{S}_p\circ \mathcal{S}_q\circ \mathcal{S}_r$
is a glide reflection.
 \kdokaz

     Because of the previous two statements,
      we can say
      that these isometries are the only type of isometries.
We can also say that we can represent any isometry of the plane as the composition of
the basic isometries, where we can choose the axes so that
there are no more than three in the composition. From the proof of the aforementioned statements, it is clear that we can determine any isometry only by the number
of fixed points and whether the isometry is direct or indirect.
We will formulate these facts in the next two statements.



            \bizrek \label{IzoKompZrc}
            Any isometry of the plane can be expressed as the product of
            one, two or three reflections.
            \eizrek

\bizrek \label{Chaslesov} \index{izrek!Chaslesov}
            (Chasles’\footnote{\index{Chasles, M.}
            \textit{M. Chasles} (1793--1880),
            French geometer, who derived this classification in 1831.})
            The only isometries of the plane
             $\mathcal{I} : E^2 \rightarrow E^2$ are:
              identity map, reflections, rotations,
            translations and glide reflections. Specifically:\\
                \hspace*{3mm}(i) if $\mathcal{I}$ is
            a direct isometry and has at least two fixed points, then $\mathcal{I}$
            is the identity map,\\
                 \hspace*{3mm}(ii) if $\mathcal{I}$ is a direct isometry and has exactly one fixed point,
             then $\mathcal{I}$ is a rotation (or specially a half-turn),\\
                 \hspace*{3mm}(iii) if $\mathcal{I}$ is
            a direct isometry and has no fixed points,
              then $\mathcal{I}$ is a translation,\\
                 \hspace*{3mm}(iv) if $\mathcal{I}$ is
            an opposite isometry and has at least one fixed point,
             then $\mathcal{I}$ is a reflection,\\
                \hspace*{3mm}(v)
             if $\mathcal{I}$ is
            an opposite isometry and has no fixed points,
              then $\mathcal{I}$ is a glide reflection.
             \eizrek

  All that we have said in this section about the classification of isometries is
illustrated in the following table (Figure \ref{IzoKlas.eps}):

\vspace*{-2mm}

%\begin{figure}[!htb]
%\centering
%\input{sl.izo.6.10.3.pic}
%\caption{} \label{sl.izo.6.10.3.pic}
%\end{figure}

\begin{figure}[!htb]
\centering
 \includegraphics[width=0.85\textwidth]{IzoKlas.eps}
\caption{} \label{IzoKlas.eps}
\end{figure}

%________________________________________________________________________________
 \poglavje{Hjelmslev's Theorem} \label{odd6Hjelmslev}

The following theorem, which refers to opposite isometries, is very useful.

(Hjelmslev's\footnote{
\index{Hjelmslev, J. T.}
\textit{J. T. Hjelmslev} (1873--1950), Danish
mathematician.}) \index{theorem!Hjelmslev's}
The midpoints of all line segments defined by corresponding pairs
of points of an arbitrary indirect isometry lie on the same line.

\begin{figure}[!htb]
\centering
\input{sl.izo.6.11.1.pic}
\caption{} \label{sl.izo.6.11.1.pic}
\end{figure}

\textbf{\textit{Proof.}} (Figure \ref{sl.izo.6.11.1.pic})

Let $\mathcal{I}$ be an indirect isometry and $X'=\mathcal{I}(X)$ for
an arbitrary point $X$ and $X_s$ the center of the line segment $XX'$. From Chasles'
theorem \ref{Chaslesov} it follows that the only types of indirect isometries of a plane
are reflection and glide reflection. We prove that in both cases
the sought line is exactly the line of reflection or glide reflection. In
the first case, where $\mathcal{I}=\mathcal{S}_s$, it is trivial that $X_s\in s$. Let $\mathcal{I}=\mathcal{G}_{2\overrightarrow{PQ}}=
\mathcal{S}_s\circ \mathcal{T}_{2\overrightarrow{PQ}}$, where $s=PQ$. If we denote by $X_1=\mathcal{T}_{2\overrightarrow{PQ}}(X)$ and
$X_2$ the center of the line segment $X_1X'$, then the line segment $X_sX_2$ is the median of the triangle $XX_1X'$ with the base $XX_1$. Therefore, (from theorem \ref{srednjicaTrik}) $X_sX_2\parallel XX_1\parallel s$ or $X_s\in
s$ (Playfair's axiom \ref{Playfair}).
 \kdokaz

The proven theorem can be used if we have congruent figures
or if we find an indirect isometry that maps one set
of points to another. We will illustrate this with the following examples.

         \bzgled
         Let $ABC$ and $A'B'C'$ be congruent triangles with the opposite orientation.
         Prove that the midpoints of the line segments
             $AA$', $BB'$ and $CC'$ lie on the same line.
          \ezgled

\begin{figure}[!htb]
\centering
\input{sl.izo.6.11.2.pic}
\caption{} \label{sl.izo.6.11.2.pic}
\end{figure}

\textbf{\textit{Proof.}} (Figure \ref{sl.izo.6.11.2.pic})
Since $ABC$ and $A'B'C'$ are congruent triangles with different orientations, there exists an indirect isometry $\mathcal{I}$, which maps triangle $ABC$ to triangle $A'B'C'$. In this case, $A$ and $A'$, $B$ and $B'$, and $C$ and $C'$ are pairs of this isometry, so by \ref{Chasles-Hjelmsleva} the midpoints of the line segments $AA'$, $BB'$, and $CC'$ lie on the same line.
 \kdokaz

        \bzgled
          Let $A$ be a point and $p$ a line in the plane. Suppose that points $X_i$
         lie on the line $p$ and $AX_iY_i$ are regular triangles with the same orientation.
          Prove that the midpoints of the line segments
          $X_iY_i$ lie on the same line.
         \ezgled


\begin{figure}[!htb]
\centering
\input{sl.izo.6.11.3.pic}
\caption{} \label{sl.izo.6.11.3.pic}
\end{figure}

\textbf{\textit{Proof.}} (Figure \ref{sl.izo.6.11.3.pic})
The isometry $\mathcal{R}_{A,60^0}\circ\mathcal{S}_p$ is an indirect isometry, which maps points $X_i$ to points $Y_i$,
 so by \ref{Chasles-Hjelmsleva} the midpoints
 of the line segments
$X_iY_i$ lie
on the same line.
  \kdokaz


%________________________________________________________________________________
 \poglavje{Isometry Groups. Symmetries of Figures} \label{odd6Grupe}

In section \ref{odd2AKSSKL} we found that the set
 $\mathfrak{I}$ of all isometries
of a plane together with the operation of the composition of
mappings represents the so-called
structure \index{group}\pojem{group}\footnote{The theory of groups
was discovered by the brilliant young French mathematician \index{Galois, E.}
\textit{E. Galois} (1811--1832).}.
 This means that the following properties are fulfilled:
\begin{enumerate}
  \item $(\forall f\in \mathfrak{I})(\forall g\in \mathfrak{I})
  \hspace*{1mm}f\circ g\in \mathfrak{I}$,
  \item $(\forall f\in \mathfrak{I})(\forall g\in \mathfrak{I})
  (\forall h\in \mathfrak{I})
  \hspace*{1mm}(f\circ g)\circ h=f\circ (g\circ h)$,
  \item $(\exists e\in \mathfrak{I})(\forall f\in \mathfrak{I})
  \hspace*{1mm}f\circ e=e\circ f=f$,
  \item $(\forall f\in \mathfrak{I})(\exists g\in \mathfrak{I})
  \hspace*{1mm}f\circ g=g\circ f=e$.
\end{enumerate}

  Property (2) is valid in general for
the composition of mappings. Properties (1), (3) and (4) were introduced with
the axiom \ref{aksIII4}. Property (1) means that the composition of two
isometries is again an isometry, (3) and (4) refer to the identity
and the inverse isometry. The group mentioned, which is determined by the set
$\mathfrak{I}$ of all isometries of a plane with respect to the operation of
the composition of mappings, is called the \index{group!isometries}
\pojem{group of all isometries of the plane}. We will also denote it by  $\mathfrak{I}$.

There are also other groups of isometries, which we obtain
if we take an appropriate subset of all isometries of the plane. Property (1)
tells us that this set cannot be arbitrary. For example, the set of all
rotations of the plane is not a group, because the composition of two
rotations is not always a rotation (it can also be a translation)."

From the properties of translation it follows that the set of all translations
together with the identity $\mathcal{E}$ forms a group - i.e.
\index{group!translation}\pojem{translation group} with the designation
$\mathfrak{T}$. We call it a \index{subgroup}
\pojem{subgroup} of the group $\mathfrak{I}$ of all isometries of the plane (this
fact is denoted by $\mathfrak{T}<\mathfrak{I}$). In fact, every
group of isometries of the plane is a subgroup of the group $\mathfrak{I}$ of all isometries of this plane.
However, the group of translations is not the only subgroup of the group
$\mathfrak{I}$. All direct isometries of the plane in fact form one
such subgroup; we denote it by $\mathfrak{I}^+$. This means that
$\mathfrak{I^+}<\mathfrak{I}$. Since translations are direct
isometries, $\mathfrak{T}<\mathfrak{I^+}$ is also true. It is clear that
the set of all indirect isometries does not determine a group, since the composition
of two indirect isometries is a direct isometry (even the identity
is not an indirect isometry).

There are also finite subgroups of the group $\mathfrak{I}$ (which have
a finite number of isometries). An example of such a subgroup is the so-called
Klein group
$\mathfrak{K}$ ($\mathfrak{K}<\mathfrak{I}$) (or \textit{Klein's quadrilateral}), which is determined by
the set of isometries $\{\mathcal{E}, \mathcal{S}_p, \mathcal{S}_q,
\mathcal{S}_O\}$, where $p$ and $q$ are rectangles that intersect at
the point $O$. This group can also be represented by the following table:

\vspace*{5mm}

\hspace*{22mm}\begin{tabular}{|c||c|c|c|c|} \hline
  % after \\ : \hline or \cline{col1-col2} \cline{col3-col4} ...
  $\circ$ & $\mathcal{E}$ & $\mathcal{S}_p$ & $\mathcal{S}_q$& $\mathcal{S}_O$
   \\ \hline \hline
  $\mathcal{E}$ & $\mathcal{E}$& $\mathcal{S}_p$& $\mathcal{S}_q$& $\mathcal{S}_O$
  \\ \hline
  $\mathcal{S}_p$ & $\mathcal{S}_p$ & $\mathcal{E}$ & $\mathcal{S}_O$& $\mathcal{S}_q$
  \\ \hline
  $\mathcal{S}_q$ & $\mathcal{S}_q$ & $\mathcal{S}_O$ & $\mathcal{E}$& $\mathcal{S}_p$
  \\ \hline
  $\mathcal{S}_O$ & $\mathcal{S}_O$ & $\mathcal{S}_q$ &$\mathcal{S}_p$ & $\mathcal{E}$
  \\ \hline
\end{tabular}


\vspace*{5mm}

Since Klein's group $\mathfrak{K}$ contains the glide reflection, which is an indirect isometry, $\mathfrak{K}$ is not a subgroup of the group $\mathfrak{I}^+$.

 We will call the group of isometries consisting only of
 the identity $\{\mathcal{E}\}$ the \index{group!trivial}
  \pojem{trivial group} with the notation $\mathfrak{E}$. This group is obviously a subgroup
   of every group
  of isometries; e.g. $\mathfrak{E}<\mathfrak{T}<\mathfrak{I^+}<\mathfrak{I}$ or
  $\mathfrak{E}<\mathfrak{K}<\mathfrak{I}$.

If we take only the identity $\{\mathcal{E}$ and
one glide reflection $\mathcal{S}_p\}$ we get another finite group of isometries.
We get the same structure if instead of the glide reflection over the line we take
the glide reflection over the point, i.e. $\{\mathcal{E}$ and $\mathcal{S}_O\}$. Although the
sets are different, the structure of the group is the same, as we illustrate with tables.
We say that in this case the groups
\index{group!isomorphic}\pojem{isomorphic}.

\vspace*{5mm}

\hspace*{12mm}\begin{tabular}{|c||c|c|} \hline
  % after \\ : \hline or \cline{col1-col2} \cline{col3-col4} ...
  $\circ$ & $\mathcal{E}$ & $\mathcal{S}_p$
   \\ \hline \hline
  $\mathcal{E}$ & $\mathcal{E}$& $\mathcal{S}_p$
  \\ \hline
  $\mathcal{S}_p$ & $\mathcal{S}_p$ & $\mathcal{E}$
  \\ \hline
\end{tabular}
\hspace*{22mm}
\begin{tabular}{|c||c|c|} \hline
  % after \\ : \hline or \cline{col1-col2} \cline{col3-col4} ...
  $\circ$ & $\mathcal{E}$ & $\mathcal{S}_O$
   \\ \hline \hline
  $\mathcal{E}$ & $\mathcal{E}$& $\mathcal{S}_O$
  \\ \hline
  $\mathcal{S}_O$ & $\mathcal{S}_O$ & $\mathcal{E}$
  \\ \hline
\end{tabular}

\vspace*{5mm}

We have already shown that every isometry of the plane can be represented
as a composite of a finite number of reflections over a line (we can always
choose at most three reflections) - the statement
\ref{IzoKompZrc}. We say that reflections over a line
\pojem{generate} the group $\mathfrak{I}$ of all isometries of that plane or that they are  \pojem{generators} of the group  $\mathcal{I}$.
 We recall
that the composite of an even number of reflections over a line represents a direct isometry,
while in the case of an indirect isometry, the number of reflections over a line is odd.

 There is another important type of groups of isometries that we
 will present in the following statement.




        \bizrek
        Let $\phi$ be a figure in the plane. The set of all isometries
        of that plane that map the figure $\phi$ to itself forms a group.
          \eizrek

\textbf{\textit{Proof.}}
 Let $\mathfrak{G}$ be the set of all isometries
    of the plane that map the figure $\phi$ to itself. We will prove
    that this set determines a group. For any isometries
     $f,g\in \mathfrak{G}$  we have $f(\phi)=\phi$ and
$g(\phi)=\phi$. But in this case we also have $g \circ f (\phi) =
\phi$  and $f^{-1}(\phi) = \phi$. So conditions
(1) and (4) are satisfied. Regarding condition (2), we have
already said that the operation of composition of mappings is always satisfied.
Condition (3) is also satisfied in our case, since
we have $\mathcal{E}(\phi)=\phi$.
 \kdokaz

The group from the previous statement - the set of all isometries
of the plane that map the figure $\phi$ onto itself - is called the
\index{group!symmetry}\emph{symmetry group} of the figure $\phi$ and is
denoted by $\mathfrak{G}(\phi)$. It is clear that for every figure $\phi$
it holds that $\mathfrak{G}(\phi)<\mathfrak{I}$. We mention that in the
proof that a subset of a group (in our case the group $\mathfrak{I}$)
represents a group or its subgroup, it is enough to check only the
conditions (\textit{i}) and (\textit{iv}). The condition (\textit{ii})
is always satisfied, while the condition (\textit{iii}) directly
follows from the conditions (\textit{i}) and (\textit{iv}).

There is another group that is a subgroup of the symmetry group
$\mathfrak{G}(\phi)$. This group consists of the set of all direct
isometries from $\mathfrak{G}(\phi)$. We call it the
\index{group!rotation}\emph{rotation group} of the figure $\phi$ and
denote it by $\mathfrak{G}^+(\phi)$. It is clear that for every figure
its rotation group is a subgroup of its symmetry group, i.e.
$\mathfrak{G}^+(\phi)<\mathfrak{G}(\phi)$.

In the next example we will determine the symmetry groups of
different figures.

\bzgled \label{grupeSimPrimeri}
Determine the symmetry group  and the rotation group of:
(i) a square, (ii) a rectangle, (iii) a trapezium,
(iv) a line, (v) a ray, (vi) a circle.
\ezgled

\textbf{\textit{Solution.}} (Figure \ref{sl.izo.6.12.1.pic})

(\textit{i}) A square with the center $O$ has four sides that intersect
in the point $O$ and four rotations (including the identity), therefore:

$\mathfrak{G}(\phi)=\{\mathcal{S}_p, \mathcal{S}_q,
\mathcal{S}_r, \mathcal{S}_s, \mathcal{E}, \mathcal{R}_{O,90^0} ,
\mathcal{R}_{O,180^0} , \mathcal{R}_{O,270^0} \}$ and

$\mathfrak{G}^+(\phi)=\{\mathcal{E}, \mathcal{R}_{O,90^0} ,
\mathcal{R}_{O,180^0} , \mathcal{R}_{O,270^0} \}$.

\begin{figure}[!htb]
\centering
\input{sl.izo.6.12.1.pic}
\caption{} \label{sl.izo.6.12.1.pic}
\end{figure}

(\textit{ii}) A rectangle with center $O$ has only two diagonals,
so $\mathfrak{G}(\phi)=\{\mathcal{S}_p, \mathcal{S}_q,
\mathcal{E},  \mathcal{S}_O \}$ and
$\mathfrak{G}^+(\phi)=\{\mathcal{E},  \mathcal{S}_O \}$. It is
interesting to note that the first group of rectangles
$\mathfrak{G}(\phi)$ is actually Klein's group $\mathfrak{K}$.

 (\textit{iii}) If $\phi$ is an arbitrary trapezoid, the identity
  is the only one that maps it onto itself. This means that
   $\mathfrak{G}(\phi)=\mathfrak{G}^+(\phi)
  =\{\mathcal{E}\}$  is a group that we have already called the trivial group.

(\textit{iv}) The groups of symmetry and rotation of the line $p$
are infinite groups and specifically:
 \begin{eqnarray*}
 \mathfrak{G}(p)&=&\{\mathcal{S}_l;l\perp p\}\cup
 \{\mathcal{S}_p\}\cup
 \{\mathcal{S}_O;O\in p\}\cup
 \{ \mathcal{T}_{\overrightarrow{v}};\overrightarrow{v} \parallel p\}\cup
 \{\mathcal{E}\}\\
 \mathfrak{G}^+(p)&=&
 \{\mathcal{S}_O;O\in p\}\cup
 \{ \mathcal{T}_{\overrightarrow{v}};\overrightarrow{v} \parallel p\}\cup
 \{\mathcal{E}\}
  \end{eqnarray*}

(\textit{v}) For the line segment $h=OA$ we have
$\mathfrak{G}(h)=\{\mathcal{E}, \mathcal{S}_{h}\}$ and
$\mathfrak{G}^+(h)=\{\mathcal{E}\}$.

(\textit{vi}) For the circle $k(O,r)$ all reflections through the
point $O$ and all rotations with center $O$ (including the identity)
belong to the group of symmetry. The group of rotations is
composed only of the aforementioned rotations. So:
 \begin{eqnarray*}
 \mathfrak{G}(k)&=&\{\mathcal{S}_l;l\ni O\}\cup
 \{\mathcal{R}_{O,\alpha};\alpha \textrm{ any angle}\}\cup
 \{\mathcal{E}\}\\
 \mathfrak{G}^+(k)&=&\{\mathcal{R}_{O,\alpha};\alpha \textrm{ any angle}\}\cup
 \{\mathcal{E}\},
  \end{eqnarray*}
 which was to be proven. \kdokaz

We have seen that a rectangle has exactly two diagonals and is also
centrally symmetric. Now we will prove a general statement.

\bzgled
         If a figure in the plane has exactly two axes of symmetry,
          then it also has a centre of symmetry.
        \ezgled

\begin{figure}[!htb]
\centering
\input{sl.izo.6.12.2.pic}
\caption{} \label{sl.izo.6.12.2.pic}
\end{figure}

\textbf{\textit{Proof.}} Let $p$ and $q$ be the only axes of symmetry of the figure $\phi$. We shall prove that $p\perp q$ (Figure \ref{sl.izo.6.12.2.pic}).
Since $\mathcal{S}_p(\phi)=\mathcal{S}_q(\phi)=\phi$, by the
transmutability theorem \ref{izoTransmutacija} it is also
$$\mathcal{S}_{\mathcal{S}_p(q)}(\phi)=
\mathcal{S}_p\circ\mathcal{S}_q\circ\mathcal{S}_p^{-1}(\phi)=\phi.$$
The lines $p$ and $q$ are the only axes of symmetry of the figure $\phi$, therefore either $\mathcal{S}_p(q)=p$ or $\mathcal{S}_p(q)=q$. In both cases
it follows that $p=q$ or $p\perp q$. However, the first possibility is excluded, since $p$ and $q$ are different by assumption. Therefore the lines $p$ and $q$ are perpendicular and intersect in a point $O$, so:
$$\mathcal{S}_O(\phi)=
\mathcal{S}_q\circ\mathcal{S}_p(\phi)=\mathcal{S}_q(\mathcal{S}_p(\phi))=
\phi,$$ which means that the figure is centrally symmetric.
 \kdokaz

In group terms we can write the previous theorem in the following way.



            \bzgled If the symmetry group of some figure contains exactly two reflections
          then it also contains a half-turn.
           \ezgled

First of all, we notice that different figures can have the same symmetry group. For example, the symmetry group of a rectangle and a line is the Klein group $\mathfrak{K}$. Also, an isosceles triangle and an isosceles trapezoid have the same symmetry group $\{\mathcal{E}, \mathcal{S}_{p}\}$.

Also, the first three and the fifth symmetry groups from the example
\ref{grupeSimPrimeri} are finite groups, while the fourth and the sixth are infinite. From this example we can also see that the infinity of the symmetry group of a figure is not related to the boundedness of this figure. The symmetry group of a bounded figure can be finite, but it can also be infinite. The same goes for unbounded figures. For further use, we shall define the concept of boundedness more precisely.

The figure $\phi$ is \index{figure!bounded} \pojem{bounded}, if there exists such a
distance $AB$,  that for any points $X$ and $Y$ of this figure it holds
$XY<AB$.

What are the groups of symmetry of bounded figures then? We have already found out,
that they can be finite (square) or  infinite (circle). In
the next few izreki we will explore the groups of symmetry
of bounded figures.



       \bizrek \label{GrupaSomer} If $\phi$ is a bounded figure, then $\mathfrak{G}(\phi)$ does not contain any
        translations.
        \eizrek

\begin{figure}[!htb]
\centering
\input{sl.izo.6.12.3.pic}
\caption{} \label{sl.izo.6.12.3.pic}
\end{figure}

\textbf{\textit{Proof.}} Let $X$ be an arbitrary point of the figure $\phi$
(Figure \ref{sl.izo.6.12.3.pic}). Because $\phi$ is bounded, there exists a distance $AB$, which is longer than any distance $XY$, where $Y$ is also
$\phi$. We assume the opposite, that some translation
$\mathcal{T}_{\overrightarrow{v}}$ belongs to the group
$\mathcal{G}(\phi)$. Because in this case
$\mathcal{T}_{\overrightarrow{v}}(\phi) = \phi$, then also
$\mathcal{T}_{\overrightarrow{v}}^n(\phi) = \phi$ or
$\mathcal{T}_{n\overrightarrow{v}}(\phi) = \phi$ for any
natural number $n\in \mathbb{N}$. So
$\mathcal{T}_{n\overrightarrow{v}}(X) = X_n\in \phi$. From the condition
of boundedness of the figure $\phi$ it follows $|AB|>|XX_n|=|n\overrightarrow{v}|$.
This should hold for every $n\in \mathbb{N}$, which of course is not
the case. The last relation therefore leads us to a contradiction, which
means that in the group $\mathfrak{G}(\phi)$  there are no translations.
 \kdokaz

   From the previous izrek it follows, that the group of symmetry
   of a bounded figure also does not contain a glide reflection, since
its square is a translation (example \ref{izoZrcdrsZrcdrs}). But how is it with reflections and rotations?



          \bizrek \label{GrupaSomer1} If a bounded figure has more axes of symmetry,
           then they all intersect at one point.
          \eizrek

\textbf{\textit{Proof.}}  Let $p$ and $q$ be the axes of symmetry of a
bounded figure $\phi$. The lines $p$ and $q$ are not parallel, because otherwise the composition of the appropriate reflections $\mathcal{S}_p \circ \mathcal{S}_q$
would be a translation. So $p$ and $q$ intersect in some point $S$. If
$r$ is a third arbitrary axe, it contains the point $S$, because otherwise (from \ref{izoZrcdrsprq}) the composition $\mathcal{S}_p \circ
\mathcal{S}_q\circ \mathcal{S}_r$ would be a mirror glide.
 \kdokaz



          \bizrek \label{GrupaSomerRot} If a bounded figure has at least one axe of symmetry
           and at least one center of rotation, then
          this center lies on the axe of symmetry.
         \eizrek

    \textbf{\textit{Proof.}}
    We assume the opposite - let $p$ be the axe of symmetry of a bounded figure $\phi$ and
    $S$
    the center of rotation
$\mathcal{R}_{S,\alpha}$, which maps this figure into itself, but
$S\notin p$. The rotation $\mathcal{R}_{S,\alpha}$ can
be represented as the composition of two reflections over the lines with the points, which
go through the point $S$. So $\mathcal{S}_p \circ
\mathcal{R}_{S,\alpha} = \mathcal{S}_p \circ \mathcal{S}_q\circ
\mathcal{S}_r$. The lines $p$, $q$ and $r$ are not in the same bundle, so
(from \ref{izoZrcdrsprq}) the composition $\mathcal{S}_p
\circ \mathcal{R}_{S,\alpha}$ is a mirror glide, which cannot
be in the group $\mathfrak{G}(\phi)$, which means that the center $S$ lies
on the axe $p$.
 \kdokaz



        \bizrek \label{GrupaRot} All rotations of a bounded figure
        have the same center.
         \eizrek

\begin{figure}[!htb]
\centering
\input{sl.izo.6.12.4.pic}
\caption{} \label{sl.izo.6.12.4.pic}
\end{figure}

\textbf{\textit{Proof.}} (Figure \ref{sl.izo.6.12.4.pic}).
 Let's assume the opposite - that $\mathcal{R}_{O,\alpha}$ and
 $\mathcal{R}_{S,\beta}$
are rotations that map the bounded figure $\phi$ into itself and $O\neq
S$. Then $O'=\mathcal{R}_{S,\beta}(O)\neq S$. By \ref{izoTransmRotac}
$\mathcal{R}_{O',\alpha}=\mathcal{R}_{S,\beta}\circ
\mathcal{R}_{O,\alpha}\circ\mathcal{R}_{S,\beta}^{-1}$, so
  $\mathcal{R}_{O',\alpha}\in \mathfrak{G}(\phi)$ and then also
  $\mathcal{I}=\mathcal{R}_{O',\alpha}^{-1}\circ
  \mathcal{R}_{O,\alpha}\in \mathfrak{G}(\phi)$.
  Let $p$ be a line that with the line $OO'$ in the point
$O$ determines the angle $\frac{1}{2}\alpha$, and $q$ its parallel through the point $O'$. By \ref{rotacKom2Zrc} $\mathcal{I} =
\mathcal{S}_q \circ \mathcal{S}_{OO'} \circ \mathcal{S}_{OO'}
\circ \mathcal{S}_p = \mathcal{S}_q \circ \mathcal{S}_p$, so
$\mathcal{I}$ is a translation, which is not possible. This means that
$O\neq S$ is not true, so all rotations of a bounded figure
have the same centre.
      \kdokaz

    From the previous equations we get the following statement.




         \bizrek \label{GrupaOmejenLik}  If symmetry group of a bounded figure $\phi$ is not trivial,
        then there is a point $S$ such that the only possible isometries in this group are:
          \\
          (i) the identity map,\\
            (ii) rotations with the centre $S$,\\
         (iii) reflections with axes,
             which all contains the point $S$.
         \eizrek

    In this way we have determined all possible groups of bounded figures.
    These groups are not necessarily
finite. What are then the finite groups? Let's explore
some examples of symmetry groups.


           \bzgled \label{GrupaDiederska} Determine the symmetry group and the rotation group
            of a regular $n$-gon.
          \ezgled

\begin{figure}[!htb]
\centering
\input{sl.izo.6.12.5.pic}
\caption{} \label{sl.izo.6.12.5.pic}
\end{figure}

\textbf{\textit{Solution.}} (Figure \ref{sl.izo.6.12.5.pic})
 No matter if $n$ is an even or odd number, a regular $n$-gon has
 exactly $n$
diagonals (see section \ref{odd3PravilniVeck}). The basic angle
of rotation that transforms a $n$-gon into itself is $\theta = \frac{360^0}{n}$.
Therefore, if we denote the group of symmetries and the group of rotations
 of a regular $n$-gon with $\mathcal{D}_n$ and $\mathcal{C}_n$, and its center with $O$, we get:
 \begin{eqnarray*}
  \mathfrak{D}_n&=&\{ \mathcal{S}_{p_1}, \mathcal{S}_{p_2},\ldots,
  \mathcal{S}_{p_n},
   \mathcal{E}, \mathcal{R}_{O,\theta}, \mathcal{R}_{O,2\theta},\ldots
   \mathcal{R}_{O,(n-1)\theta}\}\\
   \mathfrak{C}_n&=&\{ \mathcal{E}, \mathcal{R}_{O,\theta},
   \mathcal{R}_{O,2\theta},\ldots \mathcal{R}_{O,(n-1)\theta}\},
 \end{eqnarray*}
  which is what needed to be determined. \kdokaz

    The group of symmetries $\mathfrak{D}_n$ of a regular $n$-gon from
    the previous example
    is called
    \index{group!diedral} \pojem{diedral group}, the group of rotations
    of a regular $n$-gon $\mathfrak{C}_n$ is
     \index{group!cyclic}  \pojem{cyclic group}. It is clear that
     $\mathfrak{C}_n<\mathfrak{D}_n$.

  Groups $\mathfrak{D}_n$ and $\mathfrak{C}_n$ ($n\geq 3$)
   can also be generalized for the cases $n < 3$, even though in this case we are no longer
   talking about an $n$-gon. $\mathfrak{D}_2$ is thus
actually Klein's group $\mathfrak{K}$ (the group of symmetries of a line
or some sort of a "2-gon"). The group $\mathfrak{C}_2$
consists of the identity and one central symmetry. This group can be
represented as the group of rotations of a line. The group $\mathfrak{D}_1$
contains the identity and one reflection, but it is isomorphic to the group
$\mathfrak{C}_2$ (an example we have already mentioned at the beginning of this
section). The group $\mathfrak{C}_1$ is the trivial group $\mathfrak{E}$.

\begin{figure}[!htb]
\centering
\input{sl.izo.6.12.5a.pic}
\caption{} \label{sl.izo.6.12.5a.pic}
\end{figure}

We mention that the groups $\mathfrak{D}_n$ and $\mathfrak{C}_n$ ($n\in
\mathbb{N}$) are not symmetry groups only for regular $n$-sided polygons (Figure
\ref{sl.izo.6.12.5a.pic}). They can also be symmetry groups of unlimited
shapes. For example, Klein's group $\mathfrak{D}_2$ is also a symmetry group
of the hyperbola with equation $x\cdot y= 1$ in the Cartesian coordinate system
$O_{xy}$. The asymptotes are then determined by the equations $y = x$ and $y =
-x$, and the center of symmetry is the origin $O$. $\mathfrak{D}_2$ is also a symmetry group of the ellipse, given by
the equation $\frac{x^2}{a^2}+\frac{y^2}{b^2}=1$.

It is clear that all groups $\mathfrak{D}_n$ and $\mathfrak{C}_n$
($n\in \mathbb{N}$) are finite groups of isometries. We will show that these are the only finite groups of isometries!



           \bizrek \label{GrupaLeonardo} \index{izrek!Leonarda da Vincija}
           (Leonardo da Vinci\footnote{Famous Italian painter, architect and
           inventor \index{Leonardo da Vinci}
              \textit{Leonardo da Vinci} (1452--1519) studied all possible
            symmetries of the central building and possible arrangements of
            chapels around
            it, which preserve the basic symmetry.}) The only finite groups of isometries are $\mathfrak{D}_n$ and $\mathfrak{C}_n$.
              \eizrek

\textbf{\textit{Proof.}} Let $\mathfrak{G}$ be a finite group of
isometries. If it contained a translation
$\mathcal{T}_{\overrightarrow{v}}$, it would have an infinite
subgroup $\{ \mathcal{E}, \mathcal{T}_{\overrightarrow{v}},
\mathcal{T}_{\overrightarrow{v}}^2,
\mathcal{T}_{\overrightarrow{v}}^3,\ldots \}$, which is not possible.
Therefore, $\mathfrak{G}$ is a group without translations. In a
similar way as before (statement \ref{GrupaOmejenLik}), we can
prove that the only possible isometries in this group (except for
the identical mapping) are rotations with the same center $S$ and
reflections over a line with eight, which go through the point $S$.
If the group has no rotations, at most one basic reflection is
possible (already two would generate a rotation). In this case,
the only possible groups are $\mathfrak{D}_1$ and $\mathfrak{C}_1$.

Without loss of generality, we assume that all angles of rotation
are positive. Because the group $\mathfrak{G}$ is finite, there
exists a rotation $\mathcal{R}_{O,\theta}$ with the smallest
angle $\theta$. For the same reason, there exists a natural
number $n$, for which $\mathcal{R}_{O,\theta}^n = \mathcal{E}$
(otherwise, the finite group $\mathfrak{G}$ would have an infinite
subgroup $\{ \mathcal{E}, \mathcal{R}_{O,\theta},
\mathcal{R}_{O,\theta}^2, \mathcal{R}_{O,\theta}^3,\ldots \}$).
Therefore, $\theta = \frac{360^0}{n}$.

Let $\mathcal{R}_{O,\delta}$ be an arbitrary rotation from
$\mathfrak{G}$. We prove that the angle $\delta$ is a multiple of
the angle $\theta$ or $\delta=k\cdot \theta$ for some $k\in
\mathbb{N}$. We assume the opposite, that such $k$ does not exist.
Because $\delta>\theta$, for some $l\in \mathbb{N}$ we have
$l\theta<\delta<(l+1)\theta$. If we rearrange, we get
$0<\delta-l\theta<\theta$. But
$\mathcal{R}_{O,\delta-l\theta}=\mathcal{R}_{O,\delta}\circ
\left(\mathcal{R}_{O,\theta}^{-1}\right)^l\in \mathfrak{G}$, which
is contradictory to how we defined the angle $\theta$. Therefore,
all rotations from $\mathfrak{G}$ are of the form
$\mathcal{R}_{O,\theta}^k$, $k\in \{ 1,2,\ldots, n\}$.

If there are no basic reflections in the group, then $\mathfrak{G}$ is a cyclic
group or $\mathfrak{G}=\mathfrak{C}_n$ for some $n\in \mathbb{N}$. If there are
also reflections over a line in the group, their composites
(of each two reflections) are rotations, so each pair of axes determines an angle, which is
half of some angle of rotation. In this case,
$\mathfrak{G}$ is a dihedral group or $\mathfrak{G}=\mathfrak{D}_n$ for some $n\in \mathbb{N}$.
 \kdokaz

    We emphasize once again that the final group of symmetries and the group of symmetries
    of a limited figure are not the same concept.
    The group of symmetries of limited figures is not necessarily final
    (for example, the group of symmetries of a circle). Similarly, unlimited
figures can have a finite group of symmetries (for example, an equilateral triangle) and
unlimited figures with an infinite group of symmetries (for example, a line). We will illustrate these facts with the following diagram (Figure
\ref{sl.izo.6.12.6.pic}).


\begin{figure}[!htb]
\centering
\input{sl.izo.6.12.6.pic}
\caption{} \label{sl.izo.6.12.6.pic}
\end{figure}


    We will also mention another type of infinite groups of isometries that preserve
     certain tessellations of the plane (see section
     \ref{odd3Tlakovanja}). So it's not just about the groups of symmetries
     of these tessellations, but also about all subgroups of these groups.

For example, if we choose
      tiling
     a plane with an arbitrary parallelogram $ABCD$ (Figure
    \ref{sl.izo.6.12.7.pic}), the
     infinite group of symmetries of this tiling is generated by
    reflections in the vertex $A$ and the centers of the sides $AB$ and $AD$. In this
    group are then also reflections in the vertices, centers of sides and
    intersections
    of diagonals of all
    parallelograms of this tiling and all translations that are composites
     of two of
    these reflections.
     But one of its subgroups, which also preserves the same tiling, is the group
     of all mentioned translations. It is generated by
     two
     translations $\mathcal{T}_{\overrightarrow{AB}}$
     and $\mathcal{T}_{\overrightarrow{AD}}$.

\begin{figure}[!htb]
\centering
\input{sl.izo.6.12.7aa.pic}
\input{sl.izo.6.12.7bb.pic}
\caption{} \label{sl.izo.6.12.7.pic}
\end{figure}

     The group of symmetries in
     the case of tiling with a regular triangle $ABC$ (Figure
    \ref{sl.izo.6.12.7.pic})
     or tiling $(3,6)$ contains
      rotations in the vertices of the grid at angles $k\cdot 60^0$,
      reflections over lines, which are
     determined by the sides and heights of these triangles, and translations,
      generated by translations $\mathcal{T}_{\overrightarrow{AB}}$
     and $\mathcal{T}_{\overrightarrow{AC}}$. This group has several
     different subgroups, all preserving tiling $(3,6)$.

     All such groups that preserve certain tilings are called
    \index{group!discrete} \pojem{discrete groups of isometries}.
     We formally define them as follows: A group of isometries $\mathfrak{G}$ is a discrete group of isometries,
     if there exists such
    $\varepsilon>0$, that the lengths of vectors of all translations and
    measures of angles of all rotations from this group are greater than $\varepsilon$.

There are two types of discrete groups of isometries: \index{group!frieze}
 \pojem{frieze
 groups}\footnote{The term \textit{frieze}, which
  was used by the Ancient Greeks,
 meant a repeating border pattern.} and \index{group!wallpaper}\pojem{wallpaper groups}. In frieze
 groups, the subgroup
 of all translations is generated by one single translation, in wallpaper
 groups by two translations, determined by two non-linear
 vectors. It has been proven that there are exactly 7 frieze groups and 17
 wallpaper groups\footnote{All 17 groups or the corresponding types of ornamentation were
 known to the Egyptians,
 and were often used by Muslim artists. In the Alhambra palace (Spain), the Moors
 painted all 17 types of ornamentation in the 14th century. The formal proof that there
 are
 exactly
 17 wallpaper groups was first given by the Russian mathematician,
 crystallographer and mineralogist \index{Fedorov, E.}
 \textit{E. Fedorov}
  (1853–-1919) in 1891,
 and then completed and continued by the Hungarian mathematician
  \index{Pólya, G.}\textit{G. Pólya} (1887–-1985) in 1924, who is more famous for his
  famous book  ‘‘How to Solve Mathematical Problems’’.}.
  Frieze groups determine 7 different types of \index{border}
 \pojem{borders} -
 strips with repeating patterns (Figure \ref{sl.izo.6.12.7}a\footnote{http://mathworld.wolfram.com/WallpaperGroups.html}), wallpaper
 groups 17 different types of \pojem{ornamentation} - covering a plane
 with identical patterns (Figure \ref{sl.izo.6.12.7}b\footnote{http://www.quadibloc.com/math/tilint.htm}).

\begin{figure}[!h]
\centering
 \includegraphics[bb=0 0 7cm 7cm]{bands.eps}\\
\vspace*{15mm}
\includegraphics[width=1\textwidth]{wall17_phpbtUJbf.eps}
\caption{} \label{sl.izo.6.12.7}
\end{figure}

%v bmp
%\begin{figure}[!h]
%\centering
 %\includegraphics[bb=0 0 7cm 7cm]{bands.bmp}\\
%\vspace*{8mm}
%\includegraphics[bb=0 0 12cm 9.85cm]{wall17_phpbtUJbf.bmp}
%\caption{} \label{sl.izo.6.12.7}
%\end{figure}

\bnaloga\footnote{40. IMO Romania - 1999, Problem 1.}
           Determine all finite sets $\mathbf {S}$ of at least three points in the plane which satisfy the
           following condition:\\
            for any two distinct points $A$ and $B$ in $\mathbf {S}$, the perpendicular bisector
            of the line segment $AB$ is an axis of symmetry for $\mathbf {S}$.
            \enaloga


\textbf{\textit{Solution.}}
  Let $\mathbf {S}=\{A_1,A_2,\ldots
A_n \}$ in $\mathcal{M}=\{s_{A_iA_j};\hspace*{1mm}A_i,A_j\in
\mathcal{S}\}$ be a set of all symmetries. Because  $\mathbf {S}$
is a finite set (or figure), by Theorem \ref{GrupaSomer1} all
perpendiculars of this set or symmetries from the set $\mathcal{M}$ go through
one point - we denote it with $O$. It is clear that $O$
is the intersection of any two symmetries from $\mathcal{M}$, so $O=s_{A_1A_2}\cap s_{A_2A_3}$.

Let $k$ be a circle with center $O$ that goes through the point $A_1$.
We first prove that all points of the set $\mathbf {S}$ lie on this
circle. Let $A_i\in \mathbf {S}$ be an arbitrary point. From
the proven $O\in s_{A_1A_i}$ and $A_1 \in k$ it follows
$A_i=\mathcal{S}_{s_{A_1A_i}}(A_1)\in
\mathcal{S}_{s_{A_1A_i}}(k)=k$ (Figure \ref{sl.izo.6.12.IMO1.pic}).

Without loss of generality, we can assume that the points $\mathbf
{S}$ are ordered in sequence on the circle $k$, so that $\angle
A_1OA_2<\angle A_1OA_3<\cdots \angle A_1OA_n$ (otherwise we can
make a new labeling of these points). This means that $A_1A_2\ldots
A_n$ represents a polygon, which is inscribed in the circle $k$. We
also prove that this polygon is regular. Because this polygon is
concyclic, it is enough to
prove that all sides are congruent. Because $s_{A_1A_3}$
is the altitude of the set $\mathbf {S}$ (and therefore also of the
polygon $A_1A_2\ldots
A_n$), $A_1A_2\cong A_2A_3$. Similarly, $s_{A_iA_{i+2}}$
($i\in \{2,3,\ldots n-1\}$) is the altitude of the polygon $A_1A_2\ldots A_n$
and $A_iA_{i+1}\cong A_{i+1}A_{i+2}$, which means that
$A_1A_2\ldots A_n$ is a regular polygon.

The set $\mathbf {S}$ therefore represents the vertices of a regular
polygon.
 \kdokaz

%\vspace*{31mm}

\newpage

\naloge{Exercises}

\begin{enumerate}

  \item Given a line $p$ and points $A$ and $B$, which lie on
  opposite sides of the line $p$. Construct a point
  $X$, which lies on
the line $p$, so that the difference $|AX|-|XB|$ is maximal.

  \item In the plane are given lines $p$, $q$ and $r$. Construct
  an equilateral triangle $ABC$, so that
   the vertex $B$ lies on the line $p$, $C$ on $q$,
and the altitude from the vertex $A$ lies on the line $r$.

\item Given a quadrilateral $ABCD$ and a point $S$. Draw a parallelogram
with the center in the point $S$, so
that its vertices lie on the lines of
the sides of the given quadrilateral.

\item Let $\mathcal{I}$ be an indirect isometry of the plane, which
maps the point $A$ to the point $B$,
$B$ to $A$. Prove that $\mathcal{I}$ is a central symmetry.

\item Let $K$ and $L$ be points, which are symmetric to the vertex
$A$ of the triangle $ABC$ with respect to
the lines of symmetry of the internal angles at the vertices  $B$ and $C$. Let $P$ be the point of tangency of the inscribed circle of this triangle and the sides $BC$.
Prove that $P$ is the center of the segment $KL$.

\item Let $k$ and $l$ be two circles on different sides of the line
$p$. Draw an isosceles triangle $ABC$, so that its altitude $AA'$ lies on the line $p$, the vertex $B$ lies on the circle $k$, and the vertex $C$ lies on the circle $l$.

\item Let $k$ be a circle and $a$, $b$, and $c$ be lines in the same plane. Draw a triangle $ABC$ inscribed in the circle $k$, so that its sides $BC$, $AC$, and $AB$ are parallel to the lines $a$, $b$, and $c$, respectively.

\item Let $ABCDE$ be a quadrilateral with $BC\parallel DE$ and $CD\parallel EA$. Prove that the vertex $D$ lies on the perpendicular bisector of the line segment $AB$.

\item The lines $p$, $q$, and $r$ lie in the same plane. Prove that the composition $\mathcal{S}_r\circ\mathcal{S}_q\circ\mathcal{S}_p
 =\mathcal{S}_p\circ\mathcal{S}_q\circ \mathcal{S}_r$ is equivalent to the condition that the lines $p$, $q$, and $r$ belong to the same bundle.

\item Let $O$, $P$, and $Q$ be three non-collinear points. Construct a square $ABCD$ (in the plane $OPQ$) with the center at the point $O$, so that the points $P$ and $Q$ lie on the lines $AB$ and $BC$, respectively.

\item Let $\mathcal{R}_{S,\alpha}$ be a rotation and $\mathcal{S}_p$ be a reflection in the same plane and $S\in p$. Prove that the composition $\mathcal{R}_{S,\alpha}\circ\mathcal{S}_p$ and $\mathcal{S}_p\circ\mathcal{R}_{S,\alpha}$ represent a reflection.

\item Given a point $A$ and a circle $k$ in the same plane. Draw a square $ABCD$, so that the midpoints of the diagonals $BD$ lie on the circle $k$.

\item Let $ABC$ be an arbitrary triangle. Prove:
 $$\mathcal{R}_{C,2\measuredangle BCA}\circ
 \mathcal{R}_{B,2\measuredangle ABC}\circ
 \mathcal{R}_{A,2\measuredangle CAB}=\mathcal{E}.$$

\item Prove that the composition of a reflection $\mathcal{S}_p$
and a central reflection $\mathcal{S}_S$ ($S\in p$)
represents a reflection.

\item Let $O$, $P$, and $Q$ be three non-collinear points. Construct a square $ABCD$ (in the plane $OPQ$) with the center at the point $O$, so that the points $P$ and $Q$ lie on the lines $AB$ and $CD$, respectively.

\item What does the composition of a translation and a central reflection represent?

\item Given are a line $p$ and circles $k$ and $l$, which
lie in the same plane. Draw a line that is parallel to
the line $p$, so that it determines the congruent tangents on the circles $k$ and $l$.

\item Let $c$ be a line that intersects the parallels $a$ and $b$, and $l$
 be a distance. Draw an equilateral triangle $ABC$, so that $A\in a$, $B\in b$, $C\in c$ and $AB\cong l$.

\item Prove that the composition of a rotation and an axial reflection
of a plane
represents a glide reflection exactly when the center
of rotation does not lie on the axis of the axial reflection.

\item Let $ABC$ be an equilateral triangle. Prove
that the composition $\mathcal{S}_{AB}
    \circ\mathcal{S}_{CA}
    \circ\mathcal{S}_{BC}$
represents a glide reflection. Also determine the vector and the axis of this glide reflection.

\item  Given are  points $A$ and $B$ on the same side of a line
$p$.
Draw a line  $XY$, which lies on the line $p$ and is congruent
to the given distance $l$, so that the sum
$|AX|+|XY|+|YB|$ is minimal.

\item  Let $ABC$ be an isosceles right triangle with the right angle at the vertex $A$. What does the composition:
$\mathcal{G}_{\overrightarrow{AB}}\circ \mathcal{G}_{\overrightarrow{CA}}$ represent?

\item In the same plane are given lines  $a$, $b$ and $c$.
Draw points $A\in a$ and $B\in b$
so that $\mathcal{S}_c(A)=B$.

\item  Given are lines $p$ and $q$ and a point $A$ in the same plane.
Draw points $B$ and $C$ so
that the lines $p$ and $q$ will be the internal angle bisectors at
the vertices $B$ and $C$ of the triangle $ABC$.

\item  Let $s$ be the angle bisector of one of the angles determined by the lines $p$ and $q$. Prove that $\mathcal{S}_s\circ\mathcal{S}_p =
\mathcal{S}_q\circ\mathcal{S}_s$.

\item Let $S$ be the center of the triangle $ABC$ inscribed in the circle and $P$ be the point where this circle touches the side $BC$. Prove: $$\mathcal{S}_{SC} \circ\mathcal{S}_{SA}\circ\mathcal{S}_{SB} =\mathcal{S}_{SP}.$$

\item The lines $p$, $q$ and $r$ of a plane go through the center $S$ of the circle $k$. Draw a triangle $ABC$, which is inscribed in this circle, so that the lines $p$, $q$ and $r$ are the altitudes of the vertices $A$, $B$ and $C$ of this triangle.

\item The lines $p$, $q$, $r$, $s$ and $t$ of a plane intersect in the point $O$, and the point $M$ lies on the line $p$. Draw a pentagon so that $M$ is the center of one of its sides, and the lines $p$, $q$, $r$, $s$ and $t$ are the altitudes.

\item The point $P$ lies in the plane of the triangle $ABC$. Prove that the lines which are symmetric to the lines $AP$, $BP$ and $CP$ with respect to the altitudes at the vertices $A$, $B$ and $C$ of this triangle, belong to the same family.

\item Calculate the angle determined by the lines $p$ and $q$, if it holds that $\mathcal{S}_p\circ\mathcal{S}_q\circ\mathcal{S}_p = \mathcal{S}_q\circ\mathcal{S}_p\circ\mathcal{S}_q$.

\item Let $\mathcal{R}_{A,\alpha}$ and $\mathcal{R}_{B,\beta}$ be rotations in the same plane. Determine all points $X$ in this plane, for which it holds that $\mathcal{R}_{A,\alpha}(X)=\mathcal{R}_{B,\beta}(X)$.

\item The lines $p$ and $q$ intersect at an angle of $60^0$ in the center $O$ of the equilateral triangle $ABC$. Prove that the segments determined by the sides of the triangle $ABC$ on the lines $p$ and $q$ are congruent.

\item The point $S$ is the center of the regular pentagon $ABCDE$. Prove that it holds:
 $$\overrightarrow{SA} + \overrightarrow{SB} + \overrightarrow{SC}
  + \overrightarrow{SD} + \overrightarrow{SE} = \overrightarrow{0}.$$

\item Prove that the diagonals of the regular pentagon intersect in the points which are also the vertices of the regular pentagon.

\item   Let $ABP$ and $BCQ$ be two triangles with the same orientation and $\mathcal{B}(A,B,C)$. The points $K$ and $L$ are the midpoints of the lines $AQ$ and $PC$. Prove that $BLK$ is a triangle.

\item   There are three concentric circles and a line in the same plane. Draw a triangle so that its vertices are on these circles in order, and one side is parallel to the given line.

\item   The point $P$ is an internal point of the triangle $ABC$, so that $\angle APB=113^0$ and $\angle BPC=123^0$. Calculate the size of the angles of the triangle whose sides are consistent with the distances $PA$, $PB$ and $PC$.

\item   The points $P$, $Q$ and $R$ are given. Draw a triangle $ABC$ so that $P$, $Q$ and $R$ are the centers of the squares constructed over the sides $BC$, $CA$ and $AB$ of this triangle.

\item   Let $A$ and $B$ be points and $p$ a line in the same plane. Prove that the composite $\mathcal{S}_B\circ\mathcal{S}_p\circ\mathcal{S}_A$ is a reflection exactly when $AB\perp p$.

\item   Let $p$, $q$ and $r$ be tangents to the triangle $ABC$ of the inscribed circles that are parallel to its sides $BC$, $AC$ and $AB$. Prove that the lines $p$, $q$, $r$, $BC$, $AC$ and $AB$ determine such a hexagon, in which the pairs of opposite sides are consistent with the distances.

\item   Draw a triangle with the data: $\alpha$, $t_b$, $t_c$.

\item Let $ALKB$ and $ACPQ$ be two squares outside the triangle $ABC$ drawn over the sides $AB$ and $AC$, and $X$ the center of the side $BC$. Prove that $AX\perp LQ$ and $|AX|=\frac{1}{2}|QL|$.

\item Let $O$ be the center of the triangle $ABC$ and $D$ and $E$ the points of the sides $CA$ and $CB$, so that $CD\cong CE$. The point $F$ is the fourth vertex of the parallelogram $BODF$. Prove that the triangle $OEF$ is a right triangle.

\item Let $L$ be the point in which the inscribed circle of the triangle $ABC$ touches its side $BC$. Prove: $$\mathcal{R}_{C,\measuredangle ACB}\circ\mathcal{R}_{A,\measuredangle BAC} \circ\mathcal{R}_{B,\measuredangle CBA} =\mathcal{S}_L.$$

\item Points $P$ and $Q$ as well as $M$ and $N$ are the centers of two squares,
which are drawn
outside of the opposite sides of any quadrilateral. Prove that
$PQ\perp MN$ and $PQ\cong MN$.

\item  Let $APB$ and $ACQ$ be two right triangles, which are drawn
outside of the triangle $ABC$
on the sides $AB$ and $AC$. Point $S$ is the center
of the side $BC$ and $O$ is the center of the triangle $ACQ$. Prove that
$|OP|=2|OS|$.

\item Prove that the reflection in an axis and the translation of a plane
commute only when the axis of this reflection is parallel to
the vector of the translation.

\item  In the same plane, given are a line $p$, circles $k$ and $l$ and a distance $d$.
Draw a rhombus $ABCD$ with a side, which is congruent to the distance $d$, the side $AB$ lies
on the line $p$, points $C$ and $D$ in turn lie
on the circles $k$ and $l$.

\item  Let $p$ be a line, $A$ and $B$ points which lie
on the same side of the line $p$, and $d$ a distance in the same plane.
Draw points $X$ and $Y$ on the line $p$ so that $AX\cong BY$ and $XY\cong d$.

\item  Let $H$ be the altitude of the triangle $ABC$ and $R$ the radius of the circumscribed circle of this
triangle. Prove that $|AB|^2+|CH|^2=4R^2$.

\item  Let $EAB$ be a triangle, which is drawn on the side $AB$ of the square
$ABCD$. Let also $M=pr_{\perp AE}(C)$ and $N=pr_{\perp BE}(D)$ and point $P$
be the intersection of the lines $CM$ and $DN$. Prove that $PE\perp AB$.

\item  Draw an equilateral triangle $ABC$ so that its vertices in turn
lie on three parallel lines $a$, $b$ and $c$ in the same plane,
the center of this triangle lies on the line $s$, which intersects
the lines $a$, $b$ and $c$.

\item If a pentagon has at least two axes of symmetry, it is regular. Prove.

\item Let $A$, $B$ and $C$ be three collinear points. What does
the composite $\mathcal{G}_{\overrightarrow{BC}}\circ \mathcal{S}_A$ represent?

\item Let $p$, $q$ and $r$ be lines that are not in the same plane, and let $A$ be a point in the same plane. Draw a line $s$ through the point $A$ such that $\mathcal{S}_r\circ \mathcal{S}_q\circ \mathcal{S}_p(s)=s'$ and $s\parallel s'$.

%new tasks
%___________________________________

\item Let $Z$ and $K$ be inner points of the rectangle $ABCD$.
Draw points $A_1$, $B_1$, $C_1$ and $D_1$, which in turn lie on the sides $AB$, $BC$, $CD$ and $DA$ of this rectangle, so that $\angle ZA_1A\cong\angle B_1A_1B$, $\angle A_1B_1B\cong\angle C_1B_1C$, $\angle B_1C_1C\cong\angle D_1C_1D$ and $\angle C_1D_1D\cong\angle KD_1A$.

\item The point $A$ lies on the line $a$, and the point $B$ lies on the line $b$.
Determine the rotation that maps the line $a$ to the line $b$ and the point $A$ to the point $B$.

\item In the center of the square, two rectangles intersect.
Prove that these rectangles intersect the sides of the square at the vertices of a new square.

\item Given a circle $k$ and lines $a$, $b$, $c$, $d$ and $e$, which lie in the same plane. Draw a pentagon on the circle $k$ with sides that are in turn parallel to the lines $a$, $b$, $c$, $d$ and $e$.

\item The point $P$ lies inside the angle $aOb$. Draw a line $p$ through the point $P$, which with the sides $a$ and $b$ determines a triangle with the smallest area.

\item The parallelogram $PQKL$ is drawn in the parallelogram $ABCD$ (the vertices of the first lie on the sides of the second). Prove that the parallelograms have a common center.

\item The arcs $l_1, l_2,\cdots , l_n$ lie on the circle $k$ and the sum of their lengths is less than the radius of this circle. Prove that there exists such a diameter $PQ$ of the circle $k$, that none of its endpoints lies on any of the arcs $l_1, l_2,\cdots , l_n$.

\item Given is a circle $k(S,20)$. Players $\mathcal{A}$ and $\mathcal{B}$ take turns drawing circles with radii $x_i$ ($1<x_i<2$) inside the circle $k$ such that no circle has common points with any of the previously drawn circles. The player who draws the last circle wins. Does there exist a winning strategy for either player $\mathcal{A}$ or $\mathcal{B}$?

\item Let $AB$ and $CD$ be chords of the circle $k$ that have no common points, and let $P$ be an arbitrary point on the chord $CD$. Draw a point $X$ on the circle $k$ such that the chords $XA$ and $XB$ intersect the chord $CD$ at points $Y$ and $Z$ such that the point $P$ is the midpoint of the segment $ZY$.

\item Outside the parallelogram $ABCD$ are drawn equilateral triangles above its sides. Prove that the centers of these triangles are the vertices of a new parallelogram.

\item Draw a trapezoid such that the bases are congruent to the given segments $a$ and $c$, and the diagonals are congruent to the given segments $e$ and $f$.

\item Points (cities) $A$ and $B$ are on different banks of a river (the strip determined by the parallels $p$ and $q$). It is necessary to build a bridge (the segment $PQ\perp p$, $P\in P$ and $Q\in q$) over the river that will connect the cities $A$ and $B$, so that the path between the cities is the shortest ($|AP|+|PQ|+|QB|$ is minimal).

\item In the plane are given the lines $a$, $b$ and $p$ and the segment $d$. Draw a parallel $q$ to the line $p$ that with the lines $a$ and $b$ determines a segment congruent to the segment $d$.

\item The triangle $ABE$ is drawn outside the rectangle $ABCD$ above the side $AB$. The rectangles of the lines $AE$ and $BE$ through the points $C$ and $D$ intersect at the point $P$. Prove that $PE\parallel BC$.

\item The point $M$ lies inside the square $ABCD$. Prove that there exists a quadrilateral with perpendicular diagonals and sides that are congruent to the segments $MA$, $MB$, $MC$ and $MD$.

\item The congruent circles intersect in points $P$ and $Q$. The line $l$ is
parallel to the line $m$, which goes through the centers of two circles, and $l$ intersects
the circles in succession in points $A$ and $B$ and $C$ and $D$. Prove that the measure
of the angle $APC$ is not dependent on the choice of the line $l$.

\item What does the composite $\mathcal{S}_A\circ \mathcal{S}_p$ represent?

\item Let $t$ be the tangent of the inscribed circle of the triangle $ABC$ in the vertex $A$.
Prove that
$$\mathcal{G}_{\overrightarrow{CA}} \circ \mathcal{G}_{\overrightarrow{BC}}
\circ \mathcal{G}_{\overrightarrow{AB}} =\mathcal{S}_t .$$

\item What does the composite $\mathcal{S}_A\circ
\mathcal{S}_B\circ \mathcal{S}_{AB}$ represent?

\item Let $ABC$ be an equilateral triangle. Determine the axis and the vector
of the reflection glide, which is determined by the composite
$\mathcal{S}_{BC}\circ \mathcal{S}_{AB}\circ \mathcal{S}_{CA}$.


\item In the same plane, given is the heptagon $PQRSTUV$ and the circle $k$.
Draw the heptagon $ABCDEFG$, which is inscribed in the given circle,
and its sides are parallel to the sides of the given heptagon.


\item Prove that
$\mathcal{S}_A\circ\mathcal{S}_B\circ\mathcal{S}_C=
\mathcal{S}_C\circ\mathcal{S}_B\circ\mathcal{S}_A$.

\item Let $A$, $B$ and $C$ be three non-collinear points. Determine the point
$S$, for which
$$\mathcal{S}_S\circ\mathcal{S}_A\circ
\mathcal{S}_S\circ\mathcal{S}_B\circ
\mathcal{S}_S\circ\mathcal{S}_C=\mathcal{E}.$$

\item If $S$ is the center of the segment $AB$, then
$\mathcal{S}_S\circ\mathcal{S}_A\circ\mathcal{S}_S=\mathcal{S}_B$. Prove.

%olimp
%____________________________________________________

\item \footnote{Predlog za MMO 1971. (SL 12.)} Let $ABC$ and $A'B'C'$ be congruent
equilateral triangles. Prove that the centers of the segments $AA'$, $BB'$ and $CC'$
are either collinear points or the vertices of a new equilateral triangle.

\item \footnote{Proposal for MMO 1967. (LL 41.)} The line $l$ goes through the altitude point of the triangle $ABC$. We denote by $l_a$, $l_b$ and $l_c$ the images of the line $l$ with respect to the lines $BC$, $AC$ and $BC$. Prove that the lines $l_a$, $l_b$ and $l_c$ intersect in a common point that lies on the circumscribed circle of the triangle $ABC$.

\item \footnote{Proposal for MMO 1982. (SL 20.)} The isosceles triangles $BAM$, $DCP$, $BCN$ and $DAQ$ are constructed over the sides of the convex quadrilateral $ABCD$. The first two triangles are constructed outside, the other two inside this quadrilateral. What can we say about the quadrilateral $MNPQ$?

 \item Let $\mathcal{R}_{D,90^0}\circ
 \mathcal{R}_{C,90^0}\circ\mathcal{R}_{B,90^0}\circ\mathcal{R}_{A,90^0}
 =\mathcal{E}$. Prove that $AC\perp BD$ and $AC\cong BD$.


\end{enumerate}







% DEL 7 - - - - - - - - - - - - - - - - - - - - - - - - - - - - - - - - - - - - - - -
%________________________________________________________________________________
% PODOBNOST
%________________________________________________________________________________


  \del{Similarity} \label{pogPOD}


In chapters \ref{pogSKL} and \ref{pogSKK} we dealt with the relation of similarity of figures. For the definition of this relation we used isometries, which we introduced in section \ref{odd2AKSSKL} and studied in more detail in chapter \ref{pogIZO}.
Analogously, we will now first introduce similarity transformations to define the relation of similarity of figures (Figure \ref{sl.pod.7.0.1.pic}).

\begin{figure}[!htb]
\centering
\input{sl.pod.7.0.1.pic}
\caption{} \label{sl.pod.7.0.1.pic}
\end{figure}

A very important statement, which is in a certain way already related to the concept of similarity, is Tales' statement \ref{TalesovIzrek}, which we specially dealt with and proved in section \ref{odd5TalesVekt}.





%________________________________________________________________________________
\poglavje{Similarity Transformations}
\label{odd7TransfPodob}

The idea for the definition of similarity transformations comes from the concept of isometries. Intuitively, isometries represent movements. In section \ref{odd2AKSSKL} we formally defined them as bijective maps that preserve the similarity relation of pairs of points.

A bijective mapping of a plane into another plane $f:\hspace*{1mm}\mathbb{E}^2\rightarrow \mathbb{E}^2$ is called a \index{transformations of similarity}\pojem{similarity transformation} with a \pojem{similarity coefficient} \index{coefficient!of similarity} $k\in \mathbb{R}^+$, if for every two points $A,B\in \mathbb{E}^2$ and their images $A'=f(A)$ and $B'=f(B)$ it holds: $A'B':AB=k$ or $|A'B'|=k\cdot |AB|$ (Figure \ref{sl.pod.7.0.2.pic}).

\begin{figure}[!htb]
\centering
\input{sl.pod.7.0.2.pic}
\caption{} \label{sl.pod.7.0.2.pic}
\end{figure}


It is clear that isometries of a plane are a special case of similarity transformations with a coefficient $k=1$.



            \bizrek \label{TransPodB}
            Similarity transformations preserve the relation $\mathcal{B}$, i.e.
            for every three points $A$, $B$ and $C$ of the plane and their images$A'$, $B'$ and $C'$ it is:
             $$\mathcal{B}(A,B,C)\hspace*{1mm}\Rightarrow\hspace*{1mm}\mathcal{B}(A',B',C').$$
            \eizrek

\begin{figure}[!htb]
\centering
\input{sl.pod.7.0.3.pic}
\caption{} \label{sl.pod.7.0.3.pic}
\end{figure}


\textbf{\textit{Proof.}} Let $f$ be a similarity transformation with a coefficient $k$.
Assume that for points $A$, $B$ and $C$ it holds $\mathcal{B}(A,B,C)$ (Figure \ref{sl.pod.7.0.3.pic}). This means that $AB+BC=AC$.
By the definition of similarity transformations it then holds:
$$A'B' + B'C' = k\cdot AB + k\cdot BC = k\cdot (AB + BC) = k\cdot AC = A'C'.$$
From this it follows $\mathcal{B}(A',B',C')$.
\kdokaz

A direct consequence of the previous statement \ref{TransPodB} is the following claim.


            \bizrek \label{TransPodKol}
            A similarity transformation maps a line to a line, a line segment to a
            line segment, a ray to a ray, a half-plane to a half-plane, an angle to an
            angle and an $n$-gon to an $n$-gon.
            \eizrek

Also, the proof of the following claim is direct.

\bizrek \label{TransPodOhranjajoRazm}
            Similarity transformations preserve the relation of congruence and the ratio of line segments.
            It means that for every four points $A$, $B$, $C$ and $D$ of the plane and their images $A'$, $B'$, $C'$ and $D'$ it is:
            \begin{itemize}
              \item $AB\cong CD\hspace*{1mm}\Rightarrow\hspace*{1mm}A'B'\cong C'D'$;
              \item $AB:CD=A'B':C'D'$.
            \end{itemize}
            \eizrek

\begin{figure}[!htb]
\centering
\input{sl.pod.7.0.4.pic}
\caption{} \label{sl.pod.7.0.4.pic}
\end{figure}

\textbf{\textit{Proof.}} Let $f$ be a similarity transformation with coefficient $k$, for which $f:\hspace*{1mm}A,B,C,D\mapsto A',B',C',D'$ (Figure \ref{sl.pod.7.0.4.pic}). Then $A'B'=k\cdot AB$ and $C'D'=k\cdot CD$. Therefore:
$$\frac{A'B'}{C'D'}=\frac{k\cdot AB}{k\cdot CD}=\frac{AB}{CD},$$
from which we obtain both the required properties.
\kdokaz

A direct consequence is the following theorem.



            \bizrek \label{TransPodKroznKrozn}
            A similarity transformation maps a circle to a circle; the centre, the radius and the diameter
             of the first circle maps to the centre, the radius, and the diameter of
             the second circle (Figure \ref{sl.pod.7.0.5.pic}).
           \eizrek

\begin{figure}[!htb]
\centering
\input{sl.pod.7.0.5.pic}
\caption{} \label{sl.pod.7.0.5.pic}
\end{figure}


Similarly to isometries, we can also define two types of
similarity transformations.
For such similarity transformations that preserve the orientation of the plane, we call them
\index{transformacije podobnosti!direktna}\pojem{direct}.
For those
similarity transformations that reverse the orientation of the plane, we call them
\index{transformacije podobnosti!indirektna} \pojem{indirect} (Figure \ref{sl.pod.7.0.1a.pic}).
We will not formally
prove the fact that every similarity transformation of the plane is either direct
or indirect, or if one similarity transformation preserves the orientation
of one figure, it also preserves the orientation
of all other figures.

\begin{figure}[!htb]
\centering
\input{sl.pod.7.0.1a.pic}
\caption{} \label{sl.pod.7.0.1a.pic}
\end{figure}

 It is clear that the composition of two direct or two indirect
 similarity transformations represents a direct similarity transformation. Similarly,
 the composition of one direct and one indirect similarity transformation is an indirect
 similarity transformation.


The set of all similarity transformations of a plane is denoted by $\mathfrak{P}$. It is clear that $\mathfrak{I}\subset\mathfrak{P}$, where $\mathfrak{I}$ is the set of all izometrij of a plane.
In section \ref{odd2AKSSKL} we found that
 $\mathfrak{I}$ represents
 a group with respect to the operation
of composition
of mappings. For this group we used the same notation as for the set itself. $\mathfrak{I}=(\mathfrak{I},\circ)$. The properties of this group were studied in more detail in section \ref{odd6Grupe}.
At this point we will prove that $\mathfrak{P}=(\mathfrak{P},\circ)$ also represents a group structure - i.e. \index{grupa!transformacij podobnosti}\pojem{group of similarity transformations}.



            \bizrek \label{TransPodGrupa}
            The set of all similarity transformations $\mathfrak{P}$ with respect to the composition
            of mappings $\circ$ form a group, i.e.:
            \begin{enumerate}
             \item $(\forall f\in \mathfrak{P})(\forall g\in \mathfrak{P})
            \hspace*{1mm}f\circ g\in \mathfrak{P}$;
            \item $(\forall f\in \mathfrak{P})(\forall g\in \mathfrak{P})
            (\forall h\in \mathfrak{P})
            \hspace*{1mm}(f\circ g)\circ h=f\circ (g\circ h)$;
             \item $(\exists e\in \mathfrak{P})(\forall f\in \mathfrak{P})
            \hspace*{1mm}f\circ e=e\circ f=f$;
            \item $(\forall f\in \mathfrak{P})(\exists g\in \mathfrak{P})
            \hspace*{1mm}f\circ g=g\circ f=e$.
            \end{enumerate}
            \eizrek


\textbf{\textit{Proof.}}  We prove all the properties one by one.

(\textit{1}) Let $f$ and $g$ be similarity transformations with coefficients $k_2$ and $k_1$. Let $X$ and $Y$ be any two points on the plane and $g:\hspace*{1mm}X,Y\mapsto X_1,Y_1$ and $f:\hspace*{1mm}X_1,Y_1\mapsto X',Y'$. Then $f\circ g:\hspace*{1mm}X,Y\mapsto X',Y'$. Because $X_1Y_1=k_1\cdot XY$ and $X'Y'=k_2\cdot X_1Y_1$, it also follows that $X'Y'=k_1k_2\cdot XY$, which means that $f\circ g$ is a similarity transformation with coefficient $k=k_1\cdot k_2$.

(\textit{2})  The property is valid in general for the operation of the composition of mappings.

 (\textit{3}) The identity $e=\mathcal{E}$ is a similarity transformation with coefficient $k=1$.

  (\textit{4}) Let $f$ be a similarity transformation with coefficient $k$. Because $f$ is a bijective transformation, its inverse mapping $f^{-1}$ exists. Let $X$ and $Y$ be any two points on the plane and $f:\hspace*{1mm}X,Y\mapsto X',Y'$. Then  $f^{-1}:\hspace*{1mm}X',Y'\mapsto X,Y$. From $X'Y'=k\cdot XY$ it follows that $XY=\frac{1}{k}\cdot X'Y'$ ($\frac{1}{k}$ exists because $k\in \mathbb{R}^+$), which means that $f^{-1}$ is a similarity transformation with coefficient $\frac{1}{k}$.
 \kdokaz

 It is clear that the group of all isometries of the plane is a subgroup of the group of all similarity transformations of this plane. We have written this fact in the following way: $\mathfrak{I}<\mathfrak{P}$ (see section \ref{odd6Grupe}). Also, all subgroups of the group  $\mathfrak{I}$ are at the same time subgroups of the group  $\mathfrak{P}$. We get another subgroup if we take only the direct similarity transformations $\mathfrak{P}^+<\mathfrak{P}$.




%________________________________________________________________________________
\poglavje{Homothety}
\label{odd7SredRazteg}

In the previous section we had as examples of similarity transformations all isometries, if we choose the coefficient of similarity to be $k=1$. Now we will define a new type of similarity transformation.

Let $S$ be an arbitrary point in the plane and $k\in \mathbb{R}\setminus \{0\}$. The transformation of this plane, with which an arbitrary point $X$ of this plane is mapped into such a point $X'$, so that
 $$\overrightarrow{SX'} = k\cdot \overrightarrow{SX},$$
 is called the \index{središčni razteg}\pojem{središčni razteg} or \index{homotetija}\pojem{homotetija} $h_{S,k}$ with \index{središče!središčnega raztega}\pojem{središčem} $S$ and \index{koeficient!središčnega raztega}\pojem{koeficientom} $k$. All lines through point $S$ are \index{žarek središčnega raztega}\pojem{žarki} središčnega raztega. Figures are \index{lika!homotetična}\pojem{homotetična}, if
there exists a središčni razteg, which maps one figure into the other (Figure \ref{sl.pod.7.1.1.pic}).


\begin{figure}[!htb]
\centering
\input{sl.pod.7.1.1.pic}
\caption{} \label{sl.pod.7.1.1.pic}
\end{figure}


 From the definition it follows that the središčni razteg is a bijektivna transformation of the plane, which is
uniquely determined by its centre and coefficient.

From the definition itself it is also clear that in the cases $k=1$ or $k=-1$ we get the identity or the central reflection. In other words: $h_{S,1}=\mathcal{E}$ or $h_{S,-1}=\mathcal{S}_S$ (Figure \ref{sl.pod.7.1.2.pic}).


\begin{figure}[!htb]
\centering
\input{sl.pod.7.1.2.pic}
\caption{} \label{sl.pod.7.1.2.pic}
\end{figure}



            \bizrek \label{raztFiksTock}
            The only fixed point of a homothety $h_{S,k}$ (for $k\neq 1$) is its centre is $S$.
            \eizrek


\textbf{\textit{Proof.}} We first prove $h_{S,k}(S)=S$. Let $h_{S,k}(S)=S'$. By definition of the središčni razteg, $\overrightarrow{SS'}=k\cdot \overrightarrow{SS}=k\cdot \overrightarrow{0}=\overrightarrow{0}$, from which it follows that $S'=S$.

We assume that for a point $X \neq S$, we have $h_{S,k}(X)=X$. In this case, $\overrightarrow{SX}=k\cdot \overrightarrow{SX}$. Because $\overrightarrow{SX}\neq \overrightarrow{0}$, it follows that $k=\frac{\overrightarrow{SX}}{\overrightarrow{SX}}=1$, which, according to the assumption, is not possible. Therefore, $S$ is the only fixed point of the central stretch $h_{S,k}$.
\kdokaz

We have already predicted a new type of similarity transformation. In connection with this, we will first prove a lemma.



            \bizrek \label{RaztTales}
            Suppose that $h_{S,k}:\hspace*{1mm}X,Y\mapsto X',Y'$. Then:
            $$\overrightarrow{X'Y'}=k\cdot \overrightarrow{XY}.$$
            \eizrek

\begin{figure}[!htb]
\centering
\input{sl.pod.7.1.3.pic}
\caption{} \label{sl.pod.7.1.3.pic}
\end{figure}

\textbf{\textit{Proof.}} From $X'=h_{S,k}(X)$ and $Y'=h_{S,k}(Y)$ by definition of the central stretch it follows that $\overrightarrow{SX'} = k\cdot \overrightarrow{SX}$ and $\overrightarrow{SY'} = k\cdot \overrightarrow{SY}$ (Figure \ref{sl.pod.7.1.3.pic}). By formulas \ref{vektOdsev} and \ref{vektVektorskiProstor} we have:

$$\overrightarrow{X'Y'}=
\overrightarrow{SY'}-\overrightarrow{SX'}
=k\cdot \overrightarrow{SY}-k\cdot \overrightarrow{SX}=
k\cdot\left(\overrightarrow{SY}-\overrightarrow{SX} \right)
k\cdot \overrightarrow{XY},$$ which was to be proven. \kdokaz

            \bizrek \label{RaztTransPod}
            A homothety $h_{S,k}$ is a similarity transformation with the coefficient $|k|$.
            \eizrek


\textbf{\textit{Proof.}} Let $X'=h_{S,k}(X)$ and $Y'=h_{S,k}(Y)$ for arbitrary points $X$ and $Y$. By the previous formula \ref{RaztTales} we have $\overrightarrow{X'Y'}=k\cdot \overrightarrow{XY}$ or $\frac{\overrightarrow{X'Y'}}{\overrightarrow{XY}}=k$. Therefore, $\frac{X'Y'}{XY}=\frac{|\overrightarrow{X'Y'}|}{|\overrightarrow{XY}|}=|k|$, which means that $h_{S,k}$ is a similarity transformation with the coefficient $|k|$.
 \kdokaz

The central extension therefore has all the general properties that we have already proved for similarity transformations. We will especially emphasize the following statement.

            \bizrek \label{RaztKol}
            A homothety maps a line to a line, a line segment to a
            line segment, a ray to a ray, a half-plane to a half-plane, an angle to an
            angle and an $n$-gon to an $n$-gon.
            \eizrek

\textbf{\textit{Proof.}}
 The statement is a direct consequence of  \ref{RaztTransPod} and \ref{TransPodKol}.
 \kdokaz

            \bzgled \label{RaztKroznKrozn}
            A homothety
            maps a circle to a circle; the centre, the radius and the diameter
             of the first circle maps to the centre, the radius, and the diameter of
             the second circle.
            \ezgled


\textbf{\textit{Proof.}}
 The statement is a direct consequence of  \ref{RaztTransPod} and \ref{TransPodKroznKrozn}.
 \kdokaz

 We also mention two consequences of  \ref{RaztTales}.



                \bizrek  \label{RaztPremica}
                A homothety maps each line to its parallel line.
                The only lines that map to itself under a  homothety $h_{S,k}$ ($k\neq 1$), are those that contain the centre $S$.
                \eizrek

\begin{figure}[!htb]
\centering
\input{sl.pod.7.1.4.pic}
\caption{} \label{sl.pod.7.1.4.pic}
\end{figure}

\textbf{\textit{Proof.}} Let $XY$ be an arbitrary line. By  \ref{RaztKol}, its image under the central extension $h_{S,k}$ is a line $X'Y'$, where $X'=h_{S,k}(X)$ and $Y'=h_{S,k}(Y)$ (Figure \ref{sl.pod.7.1.4.pic}). From  \ref{RaztTales} it follows that $\overrightarrow{X'Y'}=k\cdot \overrightarrow{XY}$, which means that the vectors $\overrightarrow{X'Y'}$ and $\overrightarrow{XY}$ are collinear (\ref{vektKriterijKolin}) or that the lines $X'Y'\parallel XY$.

If the line $p$ goes through the center $S$ of the central extension $h_{S,k}$, then according to the statement \ref{raztFiksTock} $S\in p'=h_{S,k}(p)$. But from the proven part of this statement it follows that $p'\parallel p$. As a consequence of Playfair's axiom \ref{Playfair1} it follows that $p'=p$.

Let's assume that for some line $p$ it holds that $p=p'=h_{S,k}(p)$. Let $X\in p$ be an arbitrary point on the line $p$ and $X'=h_{S,k}(X)$.
It is then also clear that $X'\in p'=p$. If $X=X'$, then according to the statement \ref{raztFiksTock} $S=X\in p$. But if $X'\neq X$, then $S\in XX'=p$.
 \kdokaz

                \bizrek \label{homotOhranjaKote}
                A homothety maps an angle to the congruent angle.
                \eizrek

\begin{figure}[!htb]
\centering
\input{sl.pod.7.1.5.pic}
\caption{} \label{sl.pod.7.1.5.pic}
\end{figure}

\textbf{\textit{Proof.}} (Figure \ref{sl.pod.7.1.5.pic})

 The statement is a direct consequence of the statements \ref{RaztKol}, \ref{RaztPremica} and \ref{KotaVzporKraki}.
 \kdokaz

 The statement from the previous statement can also be written in the following form:
 The central extension preserves the angles (their measure). In this sense, two curves will intersect at the same angle as their images under the central extension. The mappings that have this property are called \index{konformna preslikava}\pojem{conformal mappings}.

 We have proven that the central extension maps a circle to a circle. The converse statement is also true.



            \bizrek \label{RaztKroznKrozn1}
            For any two circles of a plane, there is a homothety,
            which maps one circle to another.
            \eizrek


\begin{figure}[!htb]
\centering
\input{sl.pod.7.1.12.pic}
\caption{} \label{sl.pod.7.1.12.pic}
\end{figure}



\textbf{\textit{Proof.}} Let $k_1(S_1,r_1)$ and $k_2(S_2,r_2)$ be any two circles in the same plane (Figure \ref{sl.pod.7.1.12.pic}).

According to the statement \ref{RaztKroznKrozn} it is enough to choose the central extension $h_{S,k}$, which maps the point $S_1$ to the point $S_2$ and the radius $r_1$ to the radius $r_2$. Because $h_{S,k}$ is a similarity transformation with the coefficient $|k|$, from the last condition it follows that $r_2=|k|\cdot r_1$ or $|k|=\frac{r_2}{r_1}$. In the case of $r_1\neq r_2$ or $|k|\neq 1$ for $k$ we can choose the negative value $k=-\frac{r_2}{r_1}$. From $h_{S,k}(S_1)=S_2$ we get $\overrightarrow{SS_2}=k\overrightarrow{SS_1}$, or: $$\frac{\overrightarrow{S_2S}}{\overrightarrow{SS_1}}=
-\frac{\overrightarrow{SS_2}}{\overrightarrow{SS_1}}=-k=\frac{r_2}{r_1}.$$
According to the statement \ref{izrekEnaDelitevDaljice} there is one such point $S$.

In the case of $r_1=r_2$, we can choose the central reflection $\mathcal{S}_S$, which maps the point $S_1$ to the point $S_2$ ($S$ is the center of the line $S_1S_2$), because the central reflection is also a central extension or $\mathcal{S}_S=h_{S,-1}$.
 \kdokaz

 In the proof of the previous statement, in the case of $r_1\neq r_2$ or $|k|\neq 1$ for $k$ we used the negative value $k=-\frac{r_2}{r_1}$, as a result, the ratio $\frac{\overrightarrow{S_2S}}{\overrightarrow{SS_1}}$ was positive. The point $S$ in this case lies on the line $S_1S_2$ - this is the so-called \pojem{internal division of the line} $S_1S_2$ in the ratio $\frac{r_2}{r_1}$. If we took a positive value for $k$ as $k=\frac{r_2}{r_1}$, we would get $\frac{\overrightarrow{S_2S}}{\overrightarrow{SS_1}}=-\frac{r_2}{r_1}$ or another solution for the point $S$ and the central extension (Figure \ref{sl.pod.7.1.12.pic}). In this case we speak of the so-called \pojem{external division of the line} - we will talk more about this in section \ref{odd7Harm}.

The next statement is intuitively clear, so we will state it without a formal proof (Figure \ref{sl.pod.7.1.6.pic}).


            \bizrek \label{homotDirekt}
            A homothety $h_{S,k}$ is a direct similarity transformation.
            \eizrek

\begin{figure}[!htb]
\centering
\input{sl.pod.7.1.6.pic}
\caption{} \label{sl.pod.7.1.6.pic}
\end{figure}


                \bizrek \label{homotGrupa}
                The set of all homotheties with the same centre $S$  with respect to the composition
            of mappings $\circ$ form a commutative group. Furthermore:
                \begin{itemize}
                  \item $h_{S,k_2}\circ h_{S,k_1}=h_{S,k_1\cdot k_2}$;
                  \item $h^{-1}_{S,k}=h_{S,\frac{1}{k}}$.
                \end{itemize}
                \eizrek

\textbf{\textit{Proof.}} We will prove that all properties of a commutative group structure are fulfilled.

(\textit{1}) Let $h_{S,k_1}$ and $h_{S,k_2}$ be central dilatations of the same plane with the same center $S$ and $X$ an arbitrary point of that plane. We will prove that their composition is also a central dilatation.
We denote $h_{S,k_1}(X)=X_1$ and $h_{S,k_2}(X_1)=X'$. In this case $h_{S,k_2}\circ h_{S,k_1}(X)=X'$. But from $\overrightarrow{SX_1}=k_1\cdot \overrightarrow{SX}$ and $\overrightarrow{SX'}=k_2\cdot \overrightarrow{SX_1}$ it follows that $\overrightarrow{SX'}=k_1k_2\cdot \overrightarrow{SX}$ or $X'=h_{S,k_1\cdot k_2}(X)$.
Since this is true for every point $X$, we have: $$h_{S,k_2}\circ h_{S,k_1}=h_{S,k_1\cdot k_2}.$$


(\textit{2})  Associativity holds generally for the operation of composition of mappings.

 (\textit{3}) The identity $e=\mathcal{E}$ is a central dilatation with the coefficient $k=1$ or $\mathcal{E}=h_{S,1}$.

 (\textit{4}) Let $h_{S,k}$ be an arbitrary central dilatation with the center $S$. We denote $h^{-1}_{S,k}=h_{S,\frac{1}{k}}$. From what is proven in (\textit{1}) we get:
  $$h_{S,k}\circ h_{S,\frac{1}{k}}=h_{S,k\cdot \frac{1}{k}}=h_{S,1}=\mathcal{E}.$$

(\textit{5}) We shall also prove the commutativity, i.e. that for any central stretch $h_{S,k_1}$ and $h_{S,k_2}$ it holds that
 $h_{S,k_2}\circ h_{S,k_1}=h_{S,k_1}\circ h_{S,k_2}$. But from the proof in (\textit{1}) and the commutativity of multiplication of real numbers we get:
 $$h_{S,k_2}\circ h_{S,k_1}=h_{S,k_1\cdot k_2}
 =h_{S,k_2\cdot k_1}=h_{S,k_1}\circ h_{S,k_2},$$ which was to be proven. \kdokaz

The group from the previous statement we shall call the \index{group!homothety}\pojem{group of homotheties} and denote it with $\mathfrak{H}_S$. This group is a subgroup of the group of similarity transformations $\mathfrak{P}$ (statement \ref{RaztTransPod}) and is actually isomorphic to the group $(\mathbb{R}\setminus \{0\},\cdot)$.

 The following examples are connected with the direct construction of the image of a point or a figure under a given central stretch.



            \bzgled
           Let $S$ and $A$ be two distinct points in the plane. Construct
            the point $A'=h_{S,k}(A)$ if:

            (i) $k=2$, \hspace*{4mm}   (ii) $k=\frac{1}{3}$, \hspace*{4mm}
               (iii) $k=-3$, \hspace*{4mm}    (iv) $k=-\frac{2}{5}$.
            \ezgled

\begin{figure}[!htb]
\centering
\input{sl.pod.7.1.8.pic}
\caption{} \label{sl.pod.7.1.8.pic}
\end{figure}

\textbf{\textit{Solution.}} (Figure \ref{sl.pod.7.1.8.pic})

Examples (\textit{i}) and (\textit{iii}) are simple. In examples (\textit{ii}) and (\textit{iv}) we shall use the consequence
 of Tales' statement \ref{izrekEnaDelitevDaljiceNan}.
 \kdokaz


            \bzgled
            Construct the image of a pentagon $ABCDE$ under a homothety $h_{S,k}$, if $k=1,5$.
            \ezgled

\begin{figure}[!htb]
\centering
\input{sl.pod.7.1.7.pic}
\caption{} \label{sl.pod.7.1.7.pic}
\end{figure}

\textbf{\textit{Solution.}} (Figure \ref{sl.pod.7.1.7.pic})

First, from the condition $\overrightarrow{SX'}=1,5\cdot \overrightarrow{SX}$ we construct $A'=h_{S,k}$ in the same way as in the previous example. Then
using the formula \ref{RaztPremica} we construct the images of the remaining vertices. For example, $B'=SB\cap l$, where $l$ is a parallel to the line $AB$
through the point $A'$.
 \kdokaz

 In the next two examples we will see the use of the central extension in different constructions.


                \bzgled
                Let $A$ be one of the two intersections of a circles $k$ and $l$.
               Construct a line $s$ through the point $A$ that intersects the circles $k$ and $l$
               and defines chords that are in the ratio $3:2$.
                \ezgled

\begin{figure}[!htb]
\centering
\input{sl.pod.7.1.9.pic}
\caption{} \label{sl.pod.7.1.9.pic}
\end{figure}

\textbf{\textit{Solution.}} (Figure \ref{sl.pod.7.1.9.pic})

Let $X$ and $Y$ be the intersections of the desired line $s$ with
the circles $k$ and $l$, so that $XA:AY=3:2$. Then
$Y=h_{A,-\frac{2}{3}}$. So we can construct the point $Y$
as the second intersection of the circles $l$ and $k'=h_{A,-\frac{2}{3}}(k)$. The line $s$ is then determined by the points $A$ and $Y$,
the point $X$ is the second intersection of the line $s$ with the circle $k$.
 \kdokaz



                \bzgled \label{sredRaztegZgledKvadrat}
                Construct a square $PQRS$ into the given acute triangle $ABC$,
                so that its side $PQ$ lies on the side $BC$ of the triangle and the vertices $R$ and
                $S$ lie on the pages $AB$ and $AC$.
                \ezgled


\begin{figure}[!htb]
\centering
\input{sl.pod.7.1.10.pic}
\caption{} \label{sl.pod.7.1.10.pic}
\end{figure}

\textbf{\textit{Solution.}} (Figure \ref{sl.pod.7.1.10.pic})

If we "forget" the condition that the vertex $R$ lies on the side $AC$, there are infinitely many such squares. We can design one $P_1Q_1R_1S_1$. Choose an arbitrary point $S_1$ on the side $AB$ for the vertex, then the vertex $P_1$ is the orthogonal projection of the point $S_1$ onto the side $BC$, and finally $Q_1$ and $R_1$ are the vertices of the square. If $PQRS$ is the desired square, the squares $PQRS$ and $P_1Q_1R_1S_1$ are homothetic with the center of the central stretch (homothety) at the point $B$. Indeed, by Tales' theorem \ref{TalesovIzrek} first:
$$\frac{\overrightarrow{BP_1}}{\overrightarrow{BP}}=
\frac{\overrightarrow{BS_1}}{\overrightarrow{BS}}$$
or for some $k\in \mathbb{R}\setminus \{0\}$ it holds:
$$\overrightarrow{BP_1}=k\cdot\overrightarrow{BP}\hspace*{3mm}\textrm{ in }
\hspace*{3mm}\overrightarrow{BS_1}=k\cdot\overrightarrow{BS},$$
which means that the central stretch $h_{B,k}$ maps the points $P$ and $Q$ to the points $P_1$ and $Q_1$, as well as the square $PQRS$ to the square $P_1Q_1R_1S_1$ (because the square is mapped to the square by the central stretch, which is not difficult to prove).

Therefore, we can obtain the point $R$ as the intersection of the strip $BR_1$ with the side $AC$ (on the ray of the central stretch), and then the remaining vertices of the square $PQRS$
as the appropriate orthogonal projections.
The task always has only one solution.
\kdokaz



            \bzgled
            Let $Pp$ and $Qq$ be rays in the plane. Construct points $A\in Pp$ and $B\in Qq$ such that
            $$PA:AB:BQ=1:2:1.$$
            \ezgled

\begin{figure}[!htb]
\centering
\input{sl.pod.7.1.11.pic}
\caption{} \label{sl.pod.7.1.11.pic}
\end{figure}

\textbf{\textit{Solution.}}
From $A'$ to $B''$ we mark any points of the line segments $Pp$ and $Qq$,
so that $PA'\cong QB''\cong x$, where $x$ is any distance (Figure \ref{sl.pod.7.1.11.pic}). Let
$B'$ be one of the intersections of the circle $k(A',2x)$ and the parallel line $PQ$, which
goes through the point $B''$, and $Q'$
the fourth vertex of the parallelogram $QB''B'Q'$. The point $Q'$ therefore
lies on the line $PQ$ and $Q'B'\cong QB''\cong x$ and $A'B'\cong 2x$. Let $h_{P,k}$ be the central extension with center $P$ and coefficient $k=\frac{\overrightarrow{PQ}}{\overrightarrow{PQ'}}$.
Then $h_{P,k}(Q')=Q$. If we mark $h_{P,k}(A')=A$ and $h_{P,k}(B')=B$, the points $A$ and $B$ lie on the line segments $Pp$ and $Qq$ ($B\in Qq$ by statement \ref{RaztPremica}, because $Q'B'\parallel Qq$). By statement \ref{RaztPremica} it is also $AB\parallel A'B'$.
By Tales' statement \ref{TalesovIzrekDolzine} it is:
$\frac{PA}{PA'}=\frac{AB}{A'B'}=\frac{PB}{PB'}=\frac{BQ}{B'Q'}$,
so also $PA:AB:BQ= PA':A'B':B'Q'=x:2x:x=1:2:1$.
\kdokaz



At this point we will extend the statement from statement \ref{EulerKrozPrem1}. We will also prove the already proven part of the statement once again - with the help of the central extension.


            \bizrek \label{EulerKroznicaHomot} \index{krožnica!Eulerjeva}
           The centre of the Euler circle of a triangle lies on its Euler line.
        It is the midpoint of the segment joining the orthocentre and the circumcentre of that triangle.
         The radius of the Euler circle is half of the radius of the circumcircle.
            \eizrek

\begin{figure}[!htb]
\centering
\input{sl.pod.7.1.0e.pic}
\caption{} \label{sl.pod.7.1.0e.pic}
\end{figure}

\textbf{\textit{Proof.}}
Let $AA'$, $BB'$ and $CC'$ be the altitudes and $A_1$, $B_1$ and $C_1$ be the midpoints of the sides $BC$, $AC$ and $AB$ of the triangle $ABC$. We denote by $O$ the center of the circumscribed circle $k$, by $V$ the altitude point of this triangle and by $V_A$, $V_B$ and $V_C$ the midpoints of the segments $VA$, $VB$ and $VC$ (Figure \ref{sl.pod.7.1.0e.pic}).

 By Theorem \ref{EulerKroznica}, the points $A'$, $B'$, $C'$, $A_1$, $B_1$, $C_1$, $V_A$, $V_B$ and $V_C$ lie on one circle $e$ - i.e. Euler's circle. We denote the center of this circle by $E$.

 Let $V_a$ or $V_{A_1}$ be the points that are symmetric to the altitude point $V$ with respect to the line $BC$ or the point $A_1$. Similarly, we define the points $V_b$, $V_{B_1}$, $V_c$ and $V_{C_1}$. By the definition of the central extension, we have:
 $$\hspace*{-1.8mm} h_{V,\frac{1}{2}}:\hspace*{1mm} A, V_a, V_{A_1}, B, V_b, V_{B_1}, C, V_c, V_{C_1}
 \mapsto V_A, A', A_1, V_B, B', B_1, V_C, C', C_1.$$

 Let $h_{V,\frac{1}{2}}(O)=\widehat{E}$. The point $\widehat{E}$ is therefore the midpoint of the segment $VO$.

According to the formulas \ref{TockaV'} and \ref{TockaV1}, the points $V_a$, $V_{A_1}$, $V_b$, $V_{B_1}$, $V_c$ and $V_{C_1}$ lie on the circle $k$. This also applies to the points $A$, $B$ and $C$. The images of these points under the central extension $h_{V,\frac{1}{2}}$ lie on the image $k'=h_{V,\frac{1}{2}}(k)$ of the circle $k$, i.e. from $A$, $V_a$, $V_{A_1}$, $B$, $V_b$, $V_{B_1}$, $C$, $V_c$, $V_{C_1}$ $\in k$ it follows that $V_A$, $A'$, $A_1$, $V_B$, $B'$, $B_1$, $V_C$, $C'$, $C_1$ $\in k'=h_{V,\frac{1}{2}}(k)$. Since $V_A, A', A_1, V_B, B', B_1, V_C, C', C_1\in e$, we have $k'=e$ and $E=\widehat{E}$ (example \ref{RaztKroznKrozn}). From this it follows that the radius of the circle $e$ is equal to half the radius of the circle $k$, and its center is also the center of the line $VO$, which is the Euler line (section \ref{odd5EulPrem}).
 \kdokaz



            \bzgled \label{SimsEuler} \index{premica!Simsonova}
            Let $PQ$ be an arbitrary diameter of the circumcircle of a triangle
            $ABC$ and $p$ and $q$ Simson lines at the points $P$ and $Q$. Prove that $p$ and $q$ are
            perpendicular lines intersecting on the Euler circle of this triangle.
            \ezgled

\begin{figure}[!htb]
\centering
\input{sl.skk.4.7.1e.pic}
\caption{} \label{sl.skk.4.7.1e.pic}
\end{figure}

 \textbf{\textit{Proof.}}  (Figure \ref{sl.skk.4.7.1e.pic}).
The lines $p$ and $q$ are perpendicular, which is a consequence of the statement from
example \ref{SimsZgled2}. We denote by $L$ the intersection of the lines $p$ and
$q$. Let $k(O,R)$ be the circumcircle of the triangle $ABC$, $V$ the altitude point of this triangle, and $P_1$ and $Q_1$ the centers of the lines $VP$ and
$VQ$. The points $P_1$ and $Q_1$ lie in succession on the lines $p$ and $q$
(example \ref{SimsZgled3}). It remains to be proven that the point $L$ lies
on the Euler circle of the triangle $ABC$.

According to the statement \ref{EulerKroznicaHomot} the central extension
$h_{V,\frac{1}{2}}$ maps the circumscribed circle $k$ of the triangle $ABC$ to
the Euler circle $e(E,\frac{R}{2})$ of this triangle; in this process $h_{V,\frac{1}{2}}(O)=E$ as well.
But the same extension maps the points $P$ and $Q$ to the points $P_1$ and $Q_1$,
or the diameter $PQ$ of the circle $k$ to the diameter $P_1Q_1$ of the circle $e$
(example \ref{RaztKroznKrozn}).
Since $\angle P_1LQ_1=90^0$, by the statement
\ref{TalesovIzrKroz2} the point $L$ lies on the Euler circle $e$.
 \kdokaz



            \bnaloga\footnote{3. IMO Hungary - 1961, Problem 5.}
            Construct triangle $ABC$ if $AC = b$, $AB = c$ and $\angle AMB =\omega$, where $M$ is
            the midpoint of segment $BC$ and $\omega<90^0$.
            \enaloga

\begin{figure}[!htb]
\centering
\input{sl.pod.7.1.IMO1.pic}
\caption{} \label{sl.pod.7.1.IMO1.pic}
\end{figure}

\textbf{\textit{Solution.}} Let $ABC$ be a triangle that
satisfies the conditions  $AC = b$, $AB = c$ and $\angle AMB =\omega$,
where $M$ is the midpoint of the segment $BC$ and $\omega<90^0$ (Figure
\ref{sl.pod.7.1.IMO1.pic}). From the inequality $\omega<90^0$ it follows that $AC>AB$,
or $b>c$. Since $\angle AMB =\omega$, by the statement
\ref{ObodKotGMT} the point $M$ lies on the arc $l$ with the chord $AB$ and the central angle
$\omega$. The extension $h_{B,2}$  maps the points $B$ and $M$ in a row to
the points $B$ and $C$, and the arc $l$ to the arc $l'$. From $M\in l$ it follows that $C\in
l'$. Since $AC=b$, it follows that the point $C$ lies on
the circle $k(A,b)$ as well. So $C\in l'\cap k(A,b)$. The proven facts
allow the construction.

 First of all, we draw the segment $AB=c$ and the arc $l$ with this chord and the central angle
$\omega$, then the arc $l'=h_{B,2}(l)$, and finally the point $C$ as
one of the intersections of the arc $l'$ with the circle $k(A,b)$.

We will prove that the constructed triangle $ABC$ satisfies the conditions of the task. By construction, $AB=c$ immediately. From $C\in k(A,b)$ it follows that $AC=b$. Let $M=h^{-1}_{B,2}(C)$. Because $C\in l'=h_{B,2}(l)$, $M \in l$. Because $l$ is a line segment with this string and the obtuse angle $\omega$, $\angle AMB=\omega$. It remains to be proven that the point $M$ is the center of the line segment $BC$, which follows directly from the relation $M=h^{-1}_{B,2}(C)$.

We will find the conditions for the number of solutions to the task. We have already mentioned that, due to the condition $\omega<90^0$, $b>c$ must be true. In the case of $b\leq c$, there is no solution. The number of solutions to the task is further dependent on the number of intersections of the line segment $l'$ with the circle $k(A,b)$.
\kdokaz





%________________________________________________________________________________
 \poglavje{Classification of Similarity Transformations} \label{odd7PrezentTransPod}

In section \ref{odd6KlasifIzo} we classified isometries, and here we will do a similar classification of similarity transformations. Also in this classification,
the number of fixed points and the fact that the similarity transformation is direct or indirect will be important.

In the previous two sections we have found that all isometries and central stretches represent similarity transformations. Also, their composition is a similarity transformation (statement \ref{TransPodGrupa}). Now we will prove that the converse is also true.



                \bizrek \label{TransPodKompHomIzo}
                Each similarity transformations $f$ with coefficient $k$ can be expressed as the product
                 of one isometry and one homothety with an arbitrary centre:
                $$f=h_{S,k}\circ\mathcal{I}_1=\mathcal{I}_2\circ h_{S,k}.$$
                \eizrek

\textbf{\textit{Proof.}}

Let $f$ be an arbitrary similarity transformation with coefficient $k$.
If we denote the central stretch with arbitrary center $S$ and coefficient $\frac{1}{k}$ with $h_{S,\frac{1}{k}}$, then the composite $h_{S,\frac{1}{k}}\circ f$ represents
a similarity transformation with coefficient $k\cdot \frac{1}{k}=1$ (statement \ref{TransPodGrupa}) or isometry. So $h_{S,\frac{1}{k}}\circ f=\mathcal{I}_1$, where $\mathcal{I}_1$ is some isometry. According to statement \ref{homotGrupa}, $f=h_{S,\frac{1}{k}}^{-1}\circ\mathcal{I}_1=h_{S,k}\circ\mathcal{I}_1$.
Similarly, $f\circ h_{S,\frac{1}{k}}=\mathcal{I}_2$, where $\mathcal{I}_2$ is some isometry, so $f=\mathcal{I}_2\circ h_{S,k}$.


                \bizrek \label{TransPodOhranjaKote}
                Similarity transformations preserve the measure of angles, i.e. there map an angle to the congruent angle.
                \eizrek


\textbf{\textit{Proof.}}
 The statement is a direct consequence of statements \ref{TransPodKompHomIzo} and \ref{homotOhranjaKote}.
\kdokaz

            \bizrek \label{homotTransm}
            For each isometry  $\mathcal{I}$ and each homothety $h_{S,k}$
             it is:
            $$\mathcal{I}\circ h_{S,k}\circ \mathcal{I}^{-1}=h_{\mathcal{I}(S),k}$$
            \eizrek


\begin{figure}[!htb]
\centering
\input{sl.pod.7.1p.2.pic}
\caption{} \label{sl.pod.7.1p.2.pic}
\end{figure}

\textbf{\textit{Proof.}}  (Figure
\ref{sl.pod.7.1p.2.pic})

Let $\mathcal{I}(S)=S_1$. We denote $f=\mathcal{I}\circ h_{S,k}\circ \mathcal{I}^{-1}$. It is necessary to prove that $f=h_{S_1,k}$ or $f(X_1)=h_{S_1,k}(X_1)$ for an arbitrary point $X_1$ of this plane.

Let us also denote $X=\mathcal{I}^{-1}(X_1)$, $X'=h_{S,k}(X)$ and
$X'_1=\mathcal{I}(X_1)$. Then we have:
\begin{eqnarray*}
f(X_1)&=& \mathcal{I}\circ h_{S,k}\circ \mathcal{I}^{-1}(X_1)\\
 &=& \mathcal{I}\circ h_{S,k}(X)\\
 &=& \mathcal{I}(X')\\
 &=& X'_1
\end{eqnarray*}

 Therefore $f(X_1)=X'_1$. From $X'=h_{S,k}(X)$ it follows that $\overrightarrow{SX'}=k\cdot\overrightarrow{SX}$. Because $\mathcal{I}$ is an isometry and $\mathcal{I}:\hspace*{1mm}S,X,X'\mapsto S_1,X_1,X'_1$, also
$\overrightarrow{S_1X'_1}=k\cdot\overrightarrow{S_1X_1}$ or $h_{S_1,k}(X_1)=X'_1$.
This means that for any point $X_1$ we have $f(X_1)=X'_1=h_{S_1,k}(X_1)$, so
$f=h_{S_1,k}$.
\kdokaz



            \bizrek \label{homotIzomKom}
             An Isometry $\mathcal{I}$ and a homothety $h_{S,k}$ commute
             if and only if
             the centre of this homothety is a fixed point of the isometry $\mathcal{I}$, i.e.:
            $$\mathcal{I}\circ h_{S,k}=h_{S,k}\circ\mathcal{I}\hspace*{1mm}
            \Leftrightarrow\hspace*{1mm}\mathcal{I}(S)=S.$$
            \eizrek


\textbf{\textit{Proof.}}
 By the previous statement (\ref{homotTransm}) we have:
 \begin{eqnarray*}
\mathcal{I}\circ h_{S,k}=h_{S,k}\circ\mathcal{I}
\hspace*{1mm}
            &\Leftrightarrow& \hspace*{1mm} \mathcal{I}\circ h_{S,k}\circ\mathcal{I}^{-1}=h_{S,k}\\
\hspace*{1mm}
            &\Leftrightarrow& \hspace*{1mm} h_{\mathcal{I}(S),k}=h_{S,k}\\
\hspace*{1mm}
            &\Leftrightarrow& \hspace*{1mm} \mathcal{I}(S)=S,
\end{eqnarray*}
 which was to be proven. \kdokaz

If we choose a rotation for the isometry from statement \ref{TransPodKompHomIzo} with the same centre as the central dilation, we get a very useful type of similarity transformation.

The composition of a rotation and a central stretch with the same center is called a \index{rotational stretch} \pojem{rotational stretch} (Figure
\ref{sl.pod.7.1p.1.pic}):
$$\rho_{S,k,\omega}=h_{S,k}\circ \mathcal{R}_{S,\omega}$$
with the \index{center!of a rotational stretch}\pojem{center} $S$, the \index{coefficient!of a rotational stretch}\pojem{coefficient} $k$, and the \index{angle!of a rotational stretch}\pojem{angle} $\omega$.


\begin{figure}[!htb]
\centering
\input{sl.pod.7.1p.1.pic}
\caption{} \label{sl.pod.7.1p.1.pic}
\end{figure}

By \ref{homotIzomKom} we have:
$$\rho_{S,k,\omega}=h_{S,k}\circ \mathcal{R}_{S,\omega}=
\mathcal{R}_{S,\omega}\circ h_{S,k}.$$


Since a central stretch and a rotation are direct transformations (\ref{homotDirekt} and \ref{RotacDirekt}), a stretch rotation is also a direct transformation of similarity.

It is clear that a central stretch can also be considered as a type of stretch rotation, if we assume that the identity is a rotation by the angle $0^0$:
$$h_{S,k}=\rho_{S,k,0^0}.$$

Similarly, we can also see a rotation as a type of stretch rotation:
$$\mathcal{R}_{S,\omega}=\rho_{S,1,\omega}.$$


                \bizrek \label{rotacRaztKot}
                An arbitrary line and its image under a stretch rotation
                determine an oriented angle which is congruent to the angle of this stretch rotation:
                    $$\rho_{S,k,\omega}(p)=p'\hspace*{1mm} \Rightarrow
                    \hspace*{1mm} \angle p,p'=\omega.$$
                 \eizrek


\begin{figure}[!htb]
\centering
\input{sl.pod.7.1p.3.pic}
\caption{} \label{sl.pod.7.1p.3.pic}
\end{figure}

\textbf{\textit{Proof.}}
Let $p'=\rho_{S,k,\omega}(p)$ be the image of the line $p$ under the stretch rotation $\rho_{S,k,\omega}=h_{S,k}\circ \mathcal{R}_{S,\omega}$ (Figure \ref{sl.pod.7.1p.3.pic}) and $p_1=\mathcal{R}_{S,\omega}(p)$. Then $h_{S,k}(p_1)=p'$. By  \ref{rotacPremPremKot} we have $\measuredangle p,p_1=\omega$. By  \ref{RaztPremica} we have $p_1\parallel p'$. Therefore $\measuredangle p,p'=\angle p,p_1=\omega$ (\ref{KotiTransverzala1}).
\kdokaz



                \bizrek \label{rotacRaztKompSredZrc}
                The product of a half-turn and a stretch rotation
                with the same centre is a stretch rotation. Furthermore:
                $$\rho_{S,k,\omega}\circ \mathcal{S}_S=
                \mathcal{S}_S\circ\rho_{S,k,\omega}=\rho_{S,-k,\omega}.$$
                \eizrek

\textbf{\textit{Proof.}}  (Figure \ref{sl.pod.7.1p.3a.pic})

As we have already mentioned in section \ref{odd7SredRazteg},  $\mathcal{S}_S=h_{S,-1}$.
By  \ref{homotGrupa} we have:
\begin{eqnarray*}
\rho_{S,k,\omega}\circ \mathcal{S}_S=
\mathcal{R}_{S,\omega}\circ h_{S,k}\circ h_{S,-1}=
\mathcal{R}_{S,\omega}\circ h_{S,-k}=\rho_{S,-k,\omega}.
\end{eqnarray*}
Similarly we have:
\begin{eqnarray*}
\mathcal{S}_S\circ\rho_{S,k,\omega}=
h_{S,-1}\circ h_{S,k}\circ \mathcal{R}_{S,\omega}=
h_{S,-k}\circ\mathcal{R}_{S,\omega}=\rho_{S,-k,\omega},
\end{eqnarray*}
 which is what needed to be proven. \kdokaz


\begin{figure}[!htb]
\centering
\input{sl.pod.7.1p.3a.pic}
\caption{} \label{sl.pod.7.1p.3a.pic}
\end{figure}

                A direct consequence is the following theorem.

\bizrek \label{rotacRaztNegKoefk}
                For each stretch rotation is:
                $$\rho_{S,-k,\omega}=\rho_{S,k,180^0+\omega}.$$
                \eizrek

\textbf{\textit{Proof.}} (Figure \ref{sl.pod.7.1p.3a.pic})

By the previous statement \ref{rotacRaztKompSredZrc} and the statement \ref{rotacKomp2rotac} it is:
\begin{eqnarray*}
\rho_{S,-k,\omega}=
\mathcal{S}_S\circ\rho_{S,k,\omega}=
\mathcal{S}_S\circ\mathcal{R}_{S,\omega}\circ h_{S,k}
=\mathcal{R}_{S,180^0+\omega}\circ h_{S,k}=
\rho_{S,k,180^0+\omega},
\end{eqnarray*}
which had to be proven. \kdokaz

If for the isometry from the statement \ref{TransPodKompHomIzo} we choose the reflection over a line that goes through the center of the central stretch, we get another type of similarity transformation.

The composite of the central stretch $s$ and the central stretch $h_{S,k}$ with the center $S\in s$ is called the \index{osni razteg} \pojem{osni razteg} (Figure
\ref{sl.pod.7.1p.1a.pic}):
$$\sigma_{S,k,s}=h_{S,k}\circ \mathcal{S}_s;\hspace*{2mm} (S\in s)$$
with the \index{središče!osnega raztega}\pojem{center}, the \index{koeficient!rotacijskega raztega}\pojem{coefficient} $k$ and the \index{os!osnega raztega}\pojem{axis} $s$.


\begin{figure}[!htb]
\centering
\input{sl.pod.7.1p.1a.pic}
\caption{} \label{sl.pod.7.1p.1a.pic}
\end{figure}

By the statement \ref{homotIzomKom} it is:
$$\sigma_{S,k,s}=h_{S,k}\circ \mathcal{S}_s=
\mathcal{S}_s\circ h_{S,k}.$$

Since the central stretch is a direct and the rotation is an indirect transformation (the statement \ref{homotDirekt} and \ref{izozrIndIzo}), the axis stretch is an indirect transformation of similarity.




                \bizrek \label{transPod1FixTocLema}
                Let $f$ be a similarity transformation that is not an isometry.
                If $f$ maps each line to its parallel line, then $f$ is a homothety.
                \eizrek

\begin{figure}[!htb]
\centering
\input{sl.pod.7.1p.1bb.pic}
\caption{} \label{sl.pod.7.1p.1bb.pic}
\end{figure}

\textbf{\textit{Proof.}}  (Figure \ref{sl.pod.7.1p.1bb.pic})

Let $k$ be the coefficient of similarity of the transformation $f$. Because $f$ is not an isometry, $k \neq 1$.

By assumption, $f$ maps each line into a parallel line. We first prove that there are at least two lines that intersect and are not mapped into themselves.
Let $X$, $Y$ and $Z$ be any three non-collinear points of the plane and let $p$, $q$ and $r$ be lines determined by the points $X$, $Y$ and $Z$: $p = XY$, $q = YZ$ and $r = XZ$. Let $p' = f (p)$, $q' = f (q)$ and $r' = f (r)$. By assumption, $p \parallel p'$, $q \parallel q'$ and $r \parallel r'$. We prove that at least one of the three lines is not mapped into itself. Assume the contrary, that $p = p'$, $q = q'$ and $r = r'$. But in this case, $f (X) = f (p \cap q) = f (p) \cap f (q) = p \cap q = X$ and similarly, for example, $f (Y) = Y$. The distance $XY$ would be mapped into itself in this case, which is not possible, since $k \neq 1$. Without loss of generality, let $p \neq p'$. In the same way, by using a triangle in which no altitude is parallel to the line $p$, we can prove that there is another line that intersects the line $p$ and is not mapped into itself by $f$.

So there are lines $b$ and $c$ that intersect at point $A$ and for $b'=f(b)$ and $c'=f(c)$ it holds that $b\parallel b'$, $c\parallel c'$, $b\neq b'$ and $c\neq c'$. Let $B\in b$ and $C\in c$ be any points that are different from point $A$.
We mark $A'=b\cap c$, $B'=f(B)$ and $C'=f(C)$. First, $f(A)=f(b\cap c)=f(b)\cap f(c)=b'\cap c'=A'$, $B'\in b'$ and $C'\in c'$.
Because $b\parallel b'$, also $AB\parallel A'B'$. The distance $AB$ is mapped by transformation $f$ into the distance $A'B'$, so $A'B'=k\cdot AB$.
The lines $AA'$ and $BB'$ are not parallel. Otherwise the quadrilateral $ABB'A'$ would be a parallelogram or $AB\cong A'B'$ (by statement \ref{paralelogram}), which is not possible because $k\neq 1$. We mark with $S$ the intersection of the lines $AA'$ and $BB'$. Because $b\parallel b'$, by Tales' statement:
$$\frac{SA'}{SA}=\frac{SB'}{SB}=\frac{A'B'}{AB}=k.$$
 This means that the central stretch $h_{S,k_1}$ with center $S$ and coefficient $k_1=k$ (or $k_1=-k$) maps points $A$ and $B$ into points $A'$ and $B'$.
 Let $\widehat{C'}=h_{S,k_1}(C)$. Because by assumption $C'=f(C)$ or $A'C'=k\cdot AC$, by statement \ref{RaztPremica} $\widehat{C'}=C'$ or $h_{S,k_1}(C)=C'$.

 The mapping
$$g=f^{-1}\circ h_{S,k_1}$$
 is by statement \ref{TransPodGrupa} a similarity transformation with similarity coefficient $\frac{1}{k}\cdot |k_1|=\frac{1}{k}\cdot k=1$, so it represents an isometry.
But $g(A)=f^{-1}\circ h_{S,k_1}(A)=f^{-1}(A')=A$, or $A$ is a fixed point of isometry $g$. In a similar way we prove that also $B$ and $C$ are fixed points of isometry $g$, which means that $g=\mathcal{E}$ is the identity (by statement \ref{IizrekABC2}). So $f^{-1}\circ h_{S,k_1}=g=\mathcal{E}$ or $f=h_{S,k_1}$.
\kdokaz

 Now we are ready for the next important statement.

\bizrek \label{transPod1FixToc}
                Each similarity transformation other than isometry has exactly one fixed point.
                \eizrek


\begin{figure}[!htb]
\centering
\input{sl.pod.7.1p.1c.pic}
\caption{} \label{sl.pod.7.1p.1c.pic}
\end{figure}

\textbf{\textit{Proof.}}  (Figure \ref{sl.pod.7.1p.1c.pic})

Let $f$ be a similarity transformation with coefficient $k$. By assumption, $k\neq 1$.
Similarly to the proof of the previous izrek \ref{transPod1FixTocLema}, we quickly convince ourselves that $f$ cannot have two fixed points, because $k=1$ would be.

We will prove that $f$ has a fixed point.
According to the previous statement (\ref{transPod1FixTocLema}), we can assume that there exists at least one line $p$ that is not mapped into its parallel. Indeed, if we assume that each plane line is mapped into its parallel, then according to the statement \ref{transPod1FixTocLema} the mapping $f$ is a central stretch with a fixed point (its center).
Let $p'=f(p)$ and $ p'\nparallel p$ ($\neg p'\parallel p$). We mark the intersection of lines $p$ and $p'$ with $A$. Let $A'=f(A)$. If $A'=A$, then $A$ is a fixed point of transformation $f$ and the proof is finished. So let $A'\neq A$. From $A\in p$ it follows that $A'\in p'$. We mark the parallel to the line $p$ through the point $A'$ with $q$. Because $A'\neq A$, $q\neq p$. Let $q'=f(q)$. Because $p\parallel q$ and $f:\hspace*{1mm}p,q\rightarrow p',q'$, also $p'\parallel q'$ (if $p'$ and $q'$ would intersect in a point $T$, then also lines $p$ and $q$ would intersect in the point $f^{-1}(T)$). In the same way from $q\neq p$ it follows that $q'\neq p'$. This means that lines $p$, $q$, $p'$ and $q'$ determine the parallelogram $AA'BC$, where
 $B=q\cap q'$ and $C=p\cap q'$. Let $B'=f(B)$. From $B\in q$ it follows that $B'\in q'$. If $B'=B$, then $B$ is a fixed point and the proof is finished. So we can further assume that $B'\neq B$. From $p'\parallel q'$ it follows that $AA'\parallel BB'$. In the case $AB\parallel A'B'$ the quadrilateral $AA'B'B$ would be a parallelogram or $AB\cong A'B'$, which is not possible, because $f:\hspace*{1mm}A,B\rightarrow A',B'$ and $k\neq 1$. So lines $AB$ and $A'B'$ intersect in a point $S$. We mark $S'=f(S)$. We will prove that $S$ is a fixed point or $S'=S$.
According to Tales' statement:
\begin{eqnarray} \label{eqnTransfPod1Ft1}
 \frac{AS}{SB}=\frac{A'S}{SB'}
\end{eqnarray}
 Because $f:\hspace*{1mm}A,B,S\rightarrow A',B',S'$, also (statement \ref{TransPodOhranjajoRazm}):
\begin{eqnarray}  \label{eqnTransfPod1Ft2}
 \frac{AS}{SB}=\frac{A'S'}{S'B'}
\end{eqnarray}
Also from $S\in AB$ it follows that $S'\in A'B'$.
If we connect the last two equalities  \ref{eqnTransfPod1Ft1} and
 \ref{eqnTransfPod1Ft2}, it is:
\begin{eqnarray}  \label{eqnTransfPod1Ft3}
 \frac{A'S}{SB'}=\frac{A'S'}{S'B'},
\end{eqnarray}
where $S$ and $S'$ are points of the line $A'B'$. But similarity transformations preserve the relation $\mathcal{B}$ (statement \ref{TransPodB}), or:
\begin{eqnarray*}
 \mathcal{B}(A,S,B)\hspace*{1mm} &\Leftrightarrow& \hspace*{1mm} \mathcal{B}(A',S',B');\\
 \mathcal{B}(S,A,B)\hspace*{1mm} &\Leftrightarrow& \hspace*{1mm} \mathcal{B}(S',A',B');\\
\mathcal{B}(A,B,S)\hspace*{1mm} &\Leftrightarrow& \hspace*{1mm} \mathcal{B}(A',B',S').
\end{eqnarray*}
 From this and from the relation \ref{eqnTransfPod1Ft3} it follows:
\begin{eqnarray*}
 \frac{\overrightarrow{A'S}}{\overrightarrow{SB'}}=
\frac{\overrightarrow{A'S'}}{\overrightarrow{S'B'}},
\end{eqnarray*}
 so according to the statement \ref{izrekEnaDelitevDaljiceVekt} $S'=S$ or $S$ is a fixed point of similarity transformation $f$.
\kdokaz

Now we can make the predicted classification of similarity transformations.



                \bizrek \label{transPodKlasif}
                The only similarity transformations of the plane are:
                    \begin{itemize}
                      \item isometries,
                      \item homotheties,
                      \item stretch rotations,
                      \item stretch reflections.
                    \end{itemize}
                \eizrek

\textbf{\textit{Proof.}}
Let $f$ be an arbitrary similarity transformation with coefficient $k$.

If $k=1$, $f$ is an isometry.

Assume that $k\neq 1$ or $f$ is not an isometry. According to Theorem \ref{transPod1FixToc}, $f$ has exactly one fixed point - we denote it with $S$. So $f(S)=S$. According to Theorem  \ref{TransPodKompHomIzo}, we can represent the transformation of similarity $f$ as the composition of a central stretch with an arbitrary center (we choose the point $S$ as the center) and a coefficient $k$ and one isometry $\mathcal{I}$:
$$f=\mathcal{I}\circ h_{S,k}.$$
 Because $S$ is a fixed point of the transformation of similarity $f$ and the central stretch $h_{S,k}$,
it is:
$$S=f(S)=\mathcal{I}\circ h_{S,k}(S)=\mathcal{I}(S).$$
So $\mathcal{I}(S)=S$ or $\mathcal{I}$ is an isometry with a fixed point $S$. According to Theorem \ref{Chaslesov},  $\mathcal{I}$ can be: identity, rotation with center $S$ or reflection over a line that goes through the point $S$:
\begin{eqnarray*}
\mathcal{I}=\left\{
              \begin{array}{l}
                \mathcal{E}, \\
                \mathcal{R}_{S,\omega}, \\
                \mathcal{S}_s (S\in s)
              \end{array}
            \right.
\end{eqnarray*}
Therefore:
\begin{eqnarray*}
f=\mathcal{I}\circ h_{S,k}=\left\{
              \begin{array}{l}
                h_{S,k}, \\
                \rho_{S,k,\omega}, \\
                \sigma_{S,k,s}
              \end{array}
            \right.
\end{eqnarray*}
 which had to be proven. \kdokaz

 A direct consequence is the following theorem.

\bizrek \label{transPodKlasifDirInd}
                The only direct similarity transformations of the plane other than isometries are:
                    \begin{itemize}
                      \item homotheties,
                      \item stretch rotations.
                    \end{itemize}
                The only opposite similarity transformations of the plane other than isometries are:
                    \begin{itemize}
                      \item stretch reflections.
                    \end{itemize}
                \eizrek


It is very useful to know the property of the composite of two stretch rotations.

\bizrek \label{RotRazKomoz}
            The product of two stretch rotations $\rho_{S_1,k_1,\omega_1}$ and $\rho_{S_2,k_2,\omega_2}$
            is a direct isometry, a homothety or a stretch rotation. Furthermore (Figure \ref{sl.pod.7.1p.5a.pic}):
            \begin{eqnarray*}
            \rho_{S_2,k_2,\omega_2}\circ \rho_{S_1,k_1,\omega_1}=
            \left\{
              \begin{array}{ll}
                \mathcal{R}_{S,\omega}, & k_1k_2=1, \hspace*{2mm}
                \omega=\omega_1+\omega_2\neq n\cdot 180^0; \\
                \mathcal{T}_{\overrightarrow{v}}, & k_1k_2=1, \hspace*{2mm}
                \omega=\omega_1+\omega_2=n\cdot 180^0; \\
                h_{S,k}, & k=k_1k_2\neq 1, \hspace*{2mm}
                \omega=\omega_1+\omega_2=n\cdot 180^0; \\
                \rho_{S,k,\omega}, &  k=k_1k_2\neq 1, \hspace*{2mm}
                \omega=\omega_1+\omega_2\neq n\cdot 180^0.
              \end{array}
            \right.
            \end{eqnarray*}
            for $n\in \mathbb{Z}$.
            \eizrek


\begin{figure}[!htb]
\centering
\input{sl.pod.7.1p.5.pic}
\caption{} \label{sl.pod.7.1p.5.pic}
\end{figure}


\textbf{\textit{Proof.}} (Figure \ref{sl.pod.7.1p.5.pic})

Without loss of generality, assume that $k_1>0$ and $k_2>0$. For example, if $k_1<0$, we can by \ref{rotacRaztNegKoefk} write
 $\rho_{S,k_1,\omega}=\rho_{S,-k_1,180^0+\omega}$, where $-k_1>0$.

Let $f=\rho_{S_2,k_2,\omega_2}\circ \rho_{S_1,k_1,\omega_1}$. By the formulas \ref{RaztTransPod} and \ref{TransPodGrupa}, $f$ is a similarity transformation
with coefficient $k=k_1\cdot k_2$. It is a direct transformation as the composite of two direct transformations. By the formula  \ref{transPodKlasifDirInd}, $f$ can be an isometry, central dilation or rotational dilation.

\begin{figure}[!htb]
\centering
\input{sl.pod.7.1p.5.pic}
\caption{} \label{sl.pod.7.1p.5.pic}
\end{figure}


Let $p$ be an arbitrary line and $f(p)=p'$. Calculate the measure of the oriented angle $\measuredangle p,p'$. Let $\rho_{S_1,k_1,\omega_1}(p)=p_1$ and consequently $\rho_{S_2,k_2,\omega_2}(p_1)=p'$. By the formula \ref{rotacRaztKot}, $\angle p,p_1=\omega_1$ and $\angle p_1,p'=\omega_2$.
If $\omega=\omega_1+\omega_2\neq n\cdot 180^0$ for every $n\in \mathbb{Z}$, then
$\angle p,p'=\omega_1+\omega_2$ (formula \ref{zunanjiNotrNotr}).
If $\omega=\omega_1+\omega_2=n\cdot 180^0$ for some $n\in \mathbb{Z}$, then $p\parallel p'$ (formula \ref{KotiTransverzala}).


If $k_1\cdot k_2=1$, the composite $f$ is a similarity transformation with coefficient $k=1$ and it is a direct isometry $f=\mathcal{I}\in \mathfrak{I}^+$. It can be a translation or a rotation (in the special case of identity) (formula \ref{Chaslesov+}).
 If $\omega=\omega_1+\omega_2\neq n\cdot 180^0$ or
$\angle p,p'=\omega_1+\omega_2$, it is a rotation by the angle $\omega=\omega_1+\omega_2$ (formula \ref{rotacPremPremKot}).
 If $\omega=\omega_1+\omega_2=n\cdot 180^0$ or $p\parallel p'$, it is a translation (or identity).

Let us now assume that $k_1\cdot k_2\neq 1$. As we have already mentioned, in this case $f$ can be a central extension or a rotational extension with a coefficient $k=k_1\cdot k_2$.
 If $\omega=\omega_1+\omega_2\neq n\cdot 180^0$ or
$\angle p,p'=\omega_1+\omega_2$, it is a rotational extension for an angle $\omega=\omega_1+\omega_2$  (statement \ref{rotacRaztKot}).
 If $\omega=\omega_1+\omega_2=n\cdot 180^0$ or $p\parallel p'$, it is a central extension according to statement \ref{RaztPremica} (or a central reflection if $k_1\cdot k_2=-1$).
\kdokaz

A direct consequence is the following theorem.


            \bizrek

             The product of two homotheties
            is a direct isometry or a homothety. Furthermore:
            \begin{eqnarray*}
            h_{S_2,k_2}\circ h_{S_1,k_1}=
            \left\{
              \begin{array}{ll}
                \mathcal{T}_{\overrightarrow{v}}, & k_1k_2=1; \\
                h_{S,k}, & k=k_1k_2\neq \pm 1.
              \end{array}
            \right.
            \end{eqnarray*}
            \eizrek

\textbf{\textit{Proof.}}
The statement follows directly from the previous statement, if we write
$h_{S_1,k_1}=\rho_{S_1,k_1,0^0}$ and $h_{S_2,k_2}=\rho_{S_2,k_2,0^0}$.
\kdokaz



            \bnaloga\footnote{17. IMO Bulgaria - 1975, Problem 3.}
            On the sides of an arbitrary triangle $ABC$, triangles $ABR$, $BCP$, $CAQ$ are
            constructed externally with $\angle CBP\cong\angle CAQ=45^0$, $\angle BCP\cong\angle ACQ=35^0$,
            $\angle ABR\cong\angle BAR=15^0$. Prove that $\angle QRP=90^0$ and $QR\cong RP$.
            \enaloga


\begin{figure}[!htb]
\centering
\input{sl.pod.7.1.IMO2.pic}
\caption{} \label{sl.pod.7.1.IMO2.pic}
\end{figure}

\textbf{\textit{Solution.}} Let $S$ be the third vertex
of the isosceles triangle $BAS$, which is drawn over the side $BA$
of the triangle $ABC$ (Figure \ref{sl.pod.7.1.IMO2.pic}). Because $\angle
RBS\cong\angle SAR=45^0$ and $\angle BSR\cong\angle ASR=30^0$, the
triangles $BPC$, $AQC$, $BRS$ and $ARS$ are similar to each other.

Let $f_1$ and $f_2$ be the stretching rotations:
 \begin{eqnarray*}
  &&f_1=\rho_{B,k,45^0}=h_{B,k}\circ \mathcal{R}_{B,45^0}\\
  &&f_2=\rho_{A,\frac{1}{k},45^0}=h_{A,\frac{1}{k}}\circ
  \mathcal{R}_{A,45^0},
 \end{eqnarray*}
where:
$$k=\frac{|BC|}{|BP|}=\frac{|AC|}{|AQ|}=\frac{|BS|}{|BR|}
=\frac{|AS|}{|AR|}.$$
 Let $f=f_2\circ f_1$. By the statement \ref{RotRazKomoz} it is:
 $$f=f_2\circ f_1=\rho_{A,\frac{1}{k},45^0}\circ
 \rho_{B,k,45^0}=\rho_{T,1,90^0}=\mathcal{R}_{T,90^0}.$$
  Then it is valid:
 \begin{eqnarray*}
  &&\mathcal{R}_{T,90^0}(P)=f(P)=f_2\circ f_1(P)=f_2(C)=Q\\
  &&\mathcal{R}_{T,90^0}(R)=f(R)=f_2\circ f_1(R)=f_2(S)=R.
 \end{eqnarray*}
 From the previous (second) relation we see that $\mathcal{R}_{T,90^0}(R)=R$, which means that $R=T$ (statement \ref{RotacFiksT}) or
 $\mathcal{R}_{T,90^0}=\mathcal{R}_{R,90^0}$.
 From this and from the first relation now follows:
  $$\mathcal{R}_{R,90^0}(P)=Q,$$
  which means that $\angle QRP=90^0$
             and $QR\cong RP$.
 \kdokaz


%________________________________________________________________________________
 \poglavje{Similar Figures. Similarity of Triangles} \label{odd7PodobTrik}

As we have already announced in the introduction to this chapter, similarity transformations allow us to define the concept of similarity of figures.

We say that the figure $\Phi$ \index{figure!similar}\pojem{similar} to the figure $\Phi'$ from the same plane (the notation $\Phi\sim \Phi'$),
 if there exists a similarity transformation $f$ of this plane, which maps the figure $\Phi$ to the figure $\Phi'$ or $f(\Phi)=\Phi'$ (Figure \ref{sl.pod.7.2.1.pic}).
 The coefficient of similarity transformation $f$ is at the same time \index{coefficient!of similarity of figures}\pojem{the coefficient of similarity of figures} $\Phi$ and $\Phi'$.

\begin{figure}[!htb]
\centering
\input{sl.pod.7.2.1.pic}
\caption{} \label{sl.pod.7.2.1.pic}
\end{figure}

It is clear that for $k=1$ we get the congruence of figures as a special case of similarity. Congruent figures are therefore also similar, but the converse is not true, that is:
$$\Phi\cong \Phi' \hspace*{1mm}\Rightarrow \hspace*{1mm}\Phi\sim \Phi'.$$

We prove the most important property of the similarity of figures.

                \bizrek
                 The similarity of figures is an equivalence relation.
                \eizrek

 \textbf{\textit{Proof.}} It is necessary (and sufficient) to prove that the similarity of figures is reflexive, symmetric and transitive.


 (\textit{R}) For every figure $\Phi$ it holds that $\Phi\sim \Phi$, because the identity $\mathcal{E}$, which maps the figure $\Phi$ to itself, is a similarity transformation with the coefficient $k=1$  (Figure \ref{sl.pod.7.2.1r.pic}).


\begin{figure}[!htb]
\centering
\input{sl.pod.7.2.1r.pic}
\caption{} \label{sl.pod.7.2.1r.pic}
\end{figure}

 (\textit{S}) We assume that for the figures $\Phi_1$ and $\Phi_2$ it holds that $\Phi_1\sim \Phi_2$ (Figure \ref{sl.pod.7.2.1s.pic}). By definition there exists a similarity transformation $f$, which maps the figure $\Phi_1$ to the figure $\Phi_2$, or $f:\hspace*{1mm}\Phi_1\rightarrow \Phi_2$.
 By  \ref{TransPodGrupa} the inverse mapping $f^{-1}$, for which $f^{-1}:\hspace*{1mm}\Phi_2\rightarrow \Phi_1$, is also a similarity transformation. Therefore it holds that $\Phi_2\sim \Phi_1$.

\begin{figure}[!htb]
\centering
\input{sl.pod.7.2.1s.pic}
\caption{} \label{sl.pod.7.2.1s.pic}
\end{figure}

(\textit{T}) Let's assume that for figures $\Phi_1$, $\Phi_2$ and $\Phi_3$ it holds that $\Phi_1\sim \Phi_2$ and $\Phi_2\sim \Phi_3$ (Figure \ref{sl.pod.7.2.1t.pic}). Prove that then also $\Phi_1\sim \Phi_3$ holds. By definition, there exist similarity transformations $f$ and $g$, such that $f:\hspace*{1mm}\Phi_1\rightarrow \Phi_2$ and $g:\hspace*{1mm}\Phi_2\rightarrow \Phi_3$ hold. But by \ref{TransPodGrupa} the composition of mapping $g\circ f:\hspace*{1mm}\Phi_1\rightarrow \Phi_3$
 is also a similarity transformation, therefore $\Phi_1\sim \Phi_3$ holds.
\kdokaz

\begin{figure}[!htb]
\centering
\input{sl.pod.7.2.1t.pic}
\caption{} \label{sl.pod.7.2.1t.pic}
\end{figure}



Because the relation of similarity between figures is symmetric, in the case of $\Phi\sim \Phi'$ we will say that figures $\Phi$ and $\Phi'$ are similar.

In the following we will specifically consider the similarity between triangles. Let's assume that triangles $ABC$ and $A'B'C'$ are similar, i.e. $\triangle ABC\sim\triangle A'B'C'$. This means that there exists a similarity transformation that maps triangle $ABC$ to triangle $A'B'C'$. In the case of triangles (and also polygons) we will additionally require that the vertices are mapped in order by the similarity transformation, i.e. $f:\hspace*{1mm}A,B,C\mapsto A',B',C'$.
Since a similarity transformation maps lines to lines and angles to angles (\ref{TransPodKol}), sides $AB$, $BC$ and $CA$ of triangle $ABC$ are mapped to sides $A'B'$, $B'C'$ and $C'A'$, and the internal angles $BAC$, $ABC$ and $ACB$ of triangle $ABC$ are mapped to the internal angles $B'A'C'$, $A'B'C'$ and $A'C'B'$ of triangle $A'B'C'$. For pairs of elements in this mapping we will say that they are \pojem{corresponding} or \pojem{congruent}.

According to the statement \ref{TransPodOhranjajoRazm} similarity transformations preserve the ratio of distances, which means that the corresponding sides are proportional, that is\footnote{By using the similarity of the isosceles right triangles with the help of the appropriate ratio, \index{Tales}\textit{Tales from Miletus} (7th-6th century BC) calculated the height of the pyramid of Cheops.} (Figure \ref{sl.pod.7.2.2.pic}):
 \begin{eqnarray} \label{eqnPodTrik1}
 \triangle ABC\sim\triangle A'B'C'\hspace*{1mm} \Rightarrow \hspace*{1mm} \frac{A'B'}{AB}=\frac{A'C'}{AC}=\frac{B'C'}{BC}=k,
 \end{eqnarray}
where $k$ is the coefficient of similarity.

\begin{figure}[!htb]
\centering
\input{sl.pod.7.2.2.pic}
\caption{} \label{sl.pod.7.2.2.pic}
\end{figure}

According to the statement \ref{TransPodOhranjaKote} similarity transformations preserve angles, that is (Figure \ref{sl.pod.7.2.2.pic}):
 \begin{eqnarray} \label{eqnPodTrik2}
 \triangle ABC\sim\triangle A'B'C'\hspace*{1mm} \Rightarrow \hspace*{1mm}
 \left\{
   \begin{array}{l}
    \angle B'A'C'\cong\angle BAC; \\
    \angle A'B'C'\cong\angle ABC; \\
     \angle A'C'B'\cong\angle ACB.
   \end{array}
 \right.
 \end{eqnarray}



 From the fact that the similarity transformation preserves angles (statement \ref{TransPodOhranjaKote}) and the ratio of distances (statement \ref{TransPodOhranjajoRazm}), it also follows that the heights, centroids, ... of one triangle are mapped into the heights, centroids, ... of the other triangle, while the ratio of the corresponding elements is preserved.

In the case of similarity of triangles, we have the same problem as with congruence - it is not always easy to prove the similarity of triangles directly by definition. So, analogously to congruence, we also get the \index{statement!about similarity of triangles}\pojem{statements about similarity of triangles}.
We have already seen that from the similarity of two triangles $ABC$ and $A'B'C'$ we get the ratio of the corresponding sides \ref{eqnPodTrik1} or the congruence of the corresponding angles \ref{eqnPodTrik2}. The following statements speak about what conditions are sufficient for the triangles to be similar.

\bizrek \label{PodTrikSKS}
Triangles $ABC$ and $A'B'C'$ are similar
if two pairs of the sides of the triangles are proportional and the pair of included angles between the sides is congruent, i.e.:
 \begin{eqnarray*}
   \left.
    \begin{array}{l}
     \angle B'A'C'\cong \angle BAC\\
    \frac{A'B'}{AB}=\frac{A'C'}{AC}
   \end{array}
   \right\}\hspace*{1mm}\hspace*{1mm}\Rightarrow\triangle ABC\sim\triangle A'B'C'
   \end{eqnarray*}
   \eizrek

\begin{figure}[!htb]
\centering
\input{sl.pod.7.2.3.pic}
\caption{} \label{sl.pod.7.2.3.pic}
\end{figure}



 \textbf{\textit{Proof.}} (Figure \ref{sl.pod.7.2.3.pic})
Let it be:
 $$k=\frac{A'B'}{AB}=\frac{A'C'}{AC}.$$
Let $B''$ and $C''$ be such points on the line segments
$A'B'$ and $A'C'$, that $A'B''\cong AB$ and
$A'C''\cong AC$. Triangles
 $A'B''C''$ and $ABC$ are similar (\textit{SAS} \ref{SKS}), therefore there exists an isometry $\mathcal{I}$, which maps triangle $ABC$ to triangle
$A'B''C''$. Because
 $$\frac{A'B'}{A'B''}=\frac{A'B'}{AB}=k\hspace*{1mm}\textrm{ and }\hspace*{1mm}\frac{A'C'}{A'C''}=\frac{A'C'}{AC}=k,$$ or (because points $B''$ and $C''$ lie on the line segments
$A'B'$ and $A'C'$) $\overrightarrow{A'B'}=k\cdot \overrightarrow{A'B''}$ and $\overrightarrow{A'C'}=k\cdot \overrightarrow{A'C''}$, central stretch $h_{A',k}$ maps points $A'$, $B''$ and $C''$ to points  $A'$, $B'$ and $C'$ or triangle $A'B''C''$ to triangle $A'B'C'$. Therefore the composition $f=h_{A',k}\circ \mathcal{I}$, which is a similarity transformation, maps triangle $ABC$ to triangle $A'B'C'$, which means that $\triangle ABC\sim\triangle A'B'C'$.
 \kdokaz

\bizrek \label{PodTrikKKK}
            Triangles $ABC$ and $A'B'C'$ are similar
         if two pairs of the angles of the triangles  are congruent, i.e.:
            \begin{eqnarray*}
            \left.
             \begin{array}{l}
             \angle B'A'C'\cong\angle BAC\\
             \angle A'B'C'\cong\angle ABC
            \end{array}
            \right\}\hspace*{1mm}\hspace*{1mm}\Rightarrow\triangle ABC\sim\triangle A'B'C'
            \end{eqnarray*}
        \eizrek

\begin{figure}[!htb]
\centering
\input{sl.pod.7.2.4.pic}
\caption{} \label{sl.pod.7.2.4.pic}
\end{figure}

 \textbf{\textit{Proof.}} (Figure \ref{sl.pod.7.2.4.pic})
Let's mark:
 $$k=\frac{A'B'}{AB}.$$
Let $B''$ and $C''$ be such points on the line segments
$A'B'$ and $A'C'$, that $A'B''\cong AB$ and
$A'C''\cong AC$ holds. Then $\frac{A'B'}{A'B''}=k$ as well. Triangles
 $A'B''C''$ and $ABC$ are similar (by the \textit{SAS} theorem \ref{SKS}), so there exists an isometry $\mathcal{I}$, that maps triangle $ABC$ to triangle
$A'B''C''$. From this similarity follows that $\angle A'B''C''\cong\angle ABC$ as well. Because $\angle A'B'C'\cong\angle ABC$ by assumption, it is also true that $\angle A'B''C''\cong\angle A'B'C'$. By  \ref{KotiTransverzala} theorem $B''C''\parallel B'C'$. By Tales' \ref{TalesovIzrek} theorem
$$\frac{A'C'}{A'C''}=\frac{A'B'}{A'B''}=k.$$ Because $B''$ and $C''$ are points on the line segments
$A'B'$ and $A'C'$, it is also true that $\overrightarrow{A'B'}=k\cdot \overrightarrow{A'B''}$ and $\overrightarrow{A'C'}=k\cdot \overrightarrow{A'C''}$. This means that the central stretch $h_{A',k}$ maps points $A'$, $B''$ and $C''$ to points  $A'$, $B'$ and $C'$ or triangle $A'B''C''$ to triangle $A'B'C'$. The composition $f=h_{A',k}\circ \mathcal{I}$, which is a similarity transformation, therefore maps triangle $ABC$ to triangle $A'B'C'$, so $\triangle ABC\sim\triangle A'B'C'$.
 \kdokaz

\bizrek \label{PodTrikSSS}
Triangles $ABC$ and $A'B'C'$ are similar
if three pairs of the sides of the triangles are proportional, i.e.
\begin{eqnarray*}
\frac{A'B'}{AB}=\frac{A'C'}{AC}=\frac{B'C'}{BC}
\hspace*{1mm}\hspace*{1mm}\Rightarrow\triangle ABC\sim\triangle A'B'C'
\end{eqnarray*}
\eizrek

\begin{figure}[!htb]
\centering
\input{sl.pod.7.2.5.pic}
\caption{} \label{sl.pod.7.2.5.pic}
\end{figure}

\textbf{\textit{Proof.}} (Figure \ref{sl.pod.7.2.5.pic})
Let's mark:
$$k=\frac{A'B'}{AB}=\frac{A'C'}{AC}=\frac{B'C'}{BC}.$$
Let $B''$ and $C''$ be such points of the line segments
$A'B'$ and $A'C'$, that $A'B''\cong AB$ and
$A'C''\cong AC$ hold. Then it is also true that $$\frac{A'B'}{A'B''}=\frac{A'C'}{A'C''}=k.$$
By the converse of Tales' theorem \ref{TalesovIzrekObr} it follows that $B''C''\parallel B'C'$ and $$\frac{B'C'}{B''C''}=\frac{A'B'}{A'B''}.$$
Therefore $$\frac{B'C'}{B''C''}=\frac{A'B'}{A'B''}=\frac{A'C'}{A'C''}=k=\frac{B'C'}{BC},$$
which means that $B''C''\cong BC$ holds. From this it follows that
the triangles
 $A'B''C''$ and $ABC$ are congruent (\textit{SSS} theorem \ref{SSS}), so there exists an isometry $\mathcal{I}$, which maps the triangle $ABC$ to the triangle
$A'B''C''$. Similarly (as in the proof of the previous two theorems) the central stretch $h_{A',k}$ maps the points $A'$, $B''$ and $C''$ to the points  $A'$, $B'$ and $C'$ or the triangle $A'B''C''$ to the triangle $A'B'C'$. In the end, the composition $f=h_{A',k}\circ \mathcal{I}$, which is a similarity transformation, maps the triangle $ABC$ to the triangle $A'B'C'$, so $\triangle ABC\sim\triangle A'B'C'$.
\kdokaz

The next theorem will be given without a proof  (Figure \ref{sl.pod.7.2.6.pic}).

\bizrek \label{PodTrikSSK}
Triangles $ABC$ and $A'B'C'$ are similar
if two pairs of the sides of the triangles are proportional and the pair of the angles opposite to the longer sides is congruent.
\eizrek


\begin{figure}[!htb]
\centering
\input{sl.pod.7.2.6.pic}
\caption{} \label{sl.pod.7.2.6.pic}
\end{figure}

We will continue by using theorems about similarity of triangles.





                \bzgled
                 Let $k$ be the circumcircle of a triangle $ABC$ and $B'$ and $C'$
                the foots of the perpendiculars  from the vertices $B$ and $C$ on the tangent of the circle $k$ in the vertex $A$.
                Prove that the altitude $AD$ of this triangle is the geometric mean of the line segments $BB'$ and $CC'$, i.e.:
                $$|AD|=\sqrt{|BB'|\cdot |CC'|}.$$
                \ezgled

\begin{figure}[!htb]
\centering
\input{sl.pod.7.2.7.pic}
\caption{} \label{sl.pod.7.2.7.pic}
\end{figure}

 \textbf{\textit{Proof.}} (Figure \ref{sl.pod.7.2.7.pic})

Since $BAB'$ is congruent to
the angle subtended by the chord $AB$ at
the tangent of the circle $k$ in the vertex $A$ or $\angle BAB'\cong\angle ACD$ (theorem \ref{ObodKotTang}). Since also
$\angle AB'B\cong\angle CDA=90^0$, we have
$\triangle AB'B\sim\triangle CDA$ (theorem\ref{PodTrikKKK}). From this it follows:
\begin{eqnarray} \label{eqnPodTrikZgl1a}
AD:BB'=AC:BA.
\end{eqnarray}

In the same way we prove $\triangle AC'C\sim\triangle BDA$ or
\begin{eqnarray} \label{eqnPodTrikZgl1b}
 CC':AD=AC:BA.
\end{eqnarray}
 From relations \ref{eqnPodTrikZgl1a} and \ref{eqnPodTrikZgl1b}
it follows $AD:BB'=CC':AD$ or $|AD|=\sqrt{|BB'|\cdot |CC'|}$.
 \kdokaz

\bzgled \label{izrekSinusni}
Let $v_a$ be the length of the altitude $AD$, $k(O,R)$ the circumcircle of a triangle $ABC$
and $b=|AC|$ and $c=|AB|$. Prove that\footnote{The theorem from this example in trigonometry represents the so-called \index{izrek!Sinusni}\textit{sinusoidal theorem}, which was proved by the Arab mathematician \index{al-Biruni, A. R.}\textit{A. R. al-Biruni} (973--1048). Namely, if we insert $\frac{v_a}{c}=\sin \beta$ into the given equality, we get $\frac{b}{\sin\beta}=2R$ and similarly
$\frac{a}{\sin\alpha}=\frac{b}{\sin\beta}=\frac{c}{\sin\gamma}=2R$.}:
$$bc=2R\cdot v_a.$$
\ezgled


\begin{figure}[!htb]
\centering
\input{sl.pod.7.2.8a.pic}
\caption{} \label{sl.pod.7.2.8a.pic}
\end{figure}

 \textbf{\textit{Proof.}} We also mark $A_0=\mathcal{S}_O(A)$ (Figure \ref{sl.pod.7.2.8a.pic}). The distance $AA_0$ is the diameter of the circle $k$, so according to Theorem \ref{TalesovIzrKroz2} $\angle ACA_0=90^0=\angle ADB$. The central angles $CBA$ and $CA_0A$ over the arc $AC$ of the circle $k$ are congruent (Theorem \ref{ObodObodKot}), i.e. $\angle DBA=\angle CBA\cong\angle CA_0A$. According to Theorem \ref{PodTrikKKK}, $\triangle ABD\sim\triangle AA_0C$, so:
$$\frac{AB}{AA_0}=\frac{AD}{AC},$$
i.e.
$bc=2R\cdot v_a$.
\kdokaz



            \bzgled
            Suppose that the bisector of the interior angle $BA$C of a triangle $ABC$ intersects
            its side $BC$ at the point $E$ and the circumcircle of this triangle at the point $N$
            ($N\neq A$). Let's denote $b=|AC|$, $c=|AB|$ and $l_a=|AE|$. Prove that:
            $$|AN|=\frac{bc}{l_a}.$$
            \ezgled

\begin{figure}[!htb]
\centering
\input{sl.pod.7.2.8.pic}
\caption{} \label{sl.pod.7.2.8.pic}
\end{figure}

Suppose that the bisector of the interior angle $BA$C of a triangle $ABC$ intersects
its side $BC$ at the point $E$ and the circumcircle of this triangle at the point $N$
($N\neq A$). Let's denote $b=|AC|$, $c=|AB|$ and $l_a=|AE|$. Prove that:
$$|AN|=\frac{bc}{l_a}.$$


\begin{figure}[!htb]
\centering
\input{sl.pod.7.2.8.pic}
\caption{} \label{sl.pod.7.2.8.pic}
\end{figure}

\textbf{\textit{Proof.}} Let us also mark with $AD$ the altitude ($v_a=|AD|$), with $k(O,R)$ the circumscribed circle of the triangle $ABC$ and $M=\mathcal{S}_O(N)$ (Figure \ref{sl.pod.7.2.8.pic}). According to Theorem \ref{TockaN}, the point $N$ lies on the side $BC$'s simetrical line, which means that $NM$ is the diameter of the circle $k$ and $MN\perp BC$. According to Theorem \ref{TalesovIzrKroz2}, $\angle NAM=90^0=\angle EDA$. From $AD,MN\perp BC$ it follows that $AD\parallel MN$, therefore according to Theorem \ref{KotiTransverzala} $\angle DAE\cong\angle MNA$. The triangle $ADE$ and $NAM$ are therefore similar (Theorem \ref{PodTrikKKK}), so:
$$\frac{AD}{AN}=\frac{AE}{NM}.$$
If we use the previous Theorem \ref{izrekSinusni}, we get:
 $$|AN|=\frac{2R\cdot v_a}{l_a}=\frac{bc}{l_a},$$ which had to be proven. \kdokaz



%________________________________________________________________________________
 \poglavje{The Theorems of Ceva and Menelaus} \label{odd7MenelCeva}


 In this section we will discuss the so-called \index{theorem!theorem of the double}\emph{theorem of the double}.




            \btheorem \label{izrekCeva}\index{theorem!Ceva's}
         (Ceva\footnote{\index{Ceva, G.} \textit{G. Ceva} (1648--1734), Italian mathematician, who
         proved this theorem in 1678.})
         Let $P$, $Q$ and $R$ be points lying on the lines containing the sides $BC$, $CA$ and $AB$
          of a triangle $ABC$. Then the lines $AP$, $BQ$ and $CR$ belong
        to the same family of lines if and only if:
        \begin{eqnarray}\label{formulaCeva}
         \frac{\overrightarrow{BP}}{\overrightarrow{PC}}\cdot
        \frac{\overrightarrow{CQ}}{\overrightarrow{QA}}\cdot
         \frac{\overrightarrow{AR}}{\overrightarrow{RB}}=1.
         \end{eqnarray}
        \etheorem

\begin{figure}[!htb]
\centering
\input{sl.pod.7.5.1.pic}
\caption{} \label{sl.pod.7.5.1.pic}
\end{figure}

%slikaNova1-3-1
%\includegraphics[width=100mm]{slikaNova1-3-1.pdf}

\textbf{\textit{Proof.}}
($\Rightarrow$) We assume that the lines $AP$, $BQ$ and $CR$
belong to the same elliptic conic, that is, they intersect in a
point $S$ (we leave it to the reader to prove the statement in
the case when the lines are parallel). We denote the line that
contains point $A$ and is parallel to line $BC$ with $p$, and its
intersections with lines $BQ$ and $CR$ with $K$ and $L$ (Figure
\ref{sl.pod.7.5.1.pic}). Now from Tales' theorem \ref{TalesovIzrek} it follows that
 $\frac{\overrightarrow{BP}}{\overrightarrow{KA}}=
  \frac{\overrightarrow{SP}}{\overrightarrow{SA}}=
  \frac{\overrightarrow{PC}}{\overrightarrow{AL}}.$
From this we obtain:
 \begin{eqnarray*}
  \frac{\overrightarrow{BP}}{\overrightarrow{PC}}=
  \frac{\overrightarrow{KA}}{\overrightarrow{AL}}.
 \end{eqnarray*}
 Similarly, we also have:
 \begin{eqnarray*}
  \frac{\overrightarrow{CQ}}{\overrightarrow{QA}}=
  -\frac{\overrightarrow{QC}}{\overrightarrow{QA}}=
  -\frac{\overrightarrow{CB}}{\overrightarrow{AK}}=
  \frac{\overrightarrow{CB}}{\overrightarrow{KA}},\\
 \frac{\overrightarrow{AR}}{\overrightarrow{RB}}=
  -\frac{\overrightarrow{RA}}{\overrightarrow{RB}}=
  -\frac{\overrightarrow{AL}}{\overrightarrow{BC}}=
  \frac{\overrightarrow{AL}}{\overrightarrow{CB}},
 \end{eqnarray*}
 which means that
 $$\frac{\overrightarrow{BP}}{\overrightarrow{PC}}\cdot
  \frac{\overrightarrow{CQ}}{\overrightarrow{QA}}\cdot
  \frac{\overrightarrow{AR}}{\overrightarrow{RB}}=
 \frac{\overrightarrow{KA}}{\overrightarrow{AL}}\cdot
  \frac{\overrightarrow{CB}}{\overrightarrow{KA}}\cdot
  \frac{\overrightarrow{AL}}{\overrightarrow{CB}}=1.$$
  ($\Leftarrow$) Now we assume that relation
  (\ref{formulaCeva}) holds and that
   $S$ is the intersection of lines $BQ$ and $CR$ (the case when $BQ$ and $CR$ are parallel,
    we again leave it to the reader). If we now denote with $P'$ the intersection of line $AS$
     with line $BC$, from the first part of the proof it follows (Figure \ref{sl.pod.7.5.1.pic}):
     $$\frac{\overrightarrow{BP'}}{\overrightarrow{P'C}}\cdot
  \frac{\overrightarrow{CQ}}{\overrightarrow{QA}}\cdot
  \frac{\overrightarrow{AR}}{\overrightarrow{RB}}=1.$$
  By the assumption (\ref{formulaCeva}) we then have
  $\frac{\overrightarrow{BP}}{\overrightarrow{PC}}=
  \frac{\overrightarrow{BP'}}{\overrightarrow{P'C}}$   or $P=P'$ (statement \ref{izrekEnaDelitevDaljiceVekt}).
   Therefore, lines $AP$, $BQ$, $CR$ intersect in point $S$.
  \kdokaz

We observe that the statement is true in the case when the lines $AP$, $BQ$ and $CR$
intersect in one point (elliptical cone), as well as in the case when these lines
are parallel (parabolic cone). It seems that we could more easily prove
the statement if we introduced
points at infinity and thus generalized the concept of (elliptical) cone
also to the case of parallel lines. These are the ideas that led to the development of so-called \index{geometrija!projektivna}\pojem{projective geometry}.




            \bizrek \index{izrek!Menelajev}\label{izrekMenelaj}
            (Menelaj\footnote{\index{Menelaj} \textit{The ancient Greek mathematician Menelaus of Alexandria} (1st century)
             proved this statement in his work \textit{Spharica}, namely
             in the case of spherical triangles. He mentions the statement in the plane
            as already known. Since the previous documents about this are not preserved, we call
            the statement Menelaus's statement. Because of its similarity,
            Menelaus's and Ceva's statements are also called \textit{theorems of the twins}. The 1500-year period that
            separates these
             two discoveries is testimony to how
             geometry gradually developed during this
             period.})
             Let $P$, $Q$ and $R$ be points lying on the lines containing the sides $BC$, $CA$ and $AB$
             of a triangle $ABC$. Then the points $P$, $Q$ and $R$ are collinear if and only if:
            \begin{eqnarray}\label{formulaMenelaj}
             \frac{\overrightarrow{BP}}{\overrightarrow{PC}}\cdot
            \frac{\overrightarrow{CQ}}{\overrightarrow{QA}}\cdot
             \frac{\overrightarrow{AR}}{\overrightarrow{RB}}=-1.
             \end{eqnarray}
            \eizrek

\begin{figure}[!htb]
\centering
\input{sl.pod.7.5.2.pic}
\caption{} \label{sl.pod.7.5.2.pic}
\end{figure}

\textbf{\textit{Proof.}}
($\Rightarrow$) Let $P$, $Q$ and $R$ be points of a line $l$.
We denote with $A'$, $B'$ and $C'$ the intersection points of the perpendiculars from the vertices
$A$, $B$ and $C$ of the triangle on the line $l$ (Figure \ref{sl.pod.7.5.2.pic}).
By using Tales' theorem (\ref{TalesovIzrek}) we get:
\begin{eqnarray*}
 \frac{\overrightarrow{BP}}{\overrightarrow{PC}}\cdot
  \frac{\overrightarrow{CQ}}{\overrightarrow{QA}}\cdot
  \frac{\overrightarrow{AR}}{\overrightarrow{RB}}=
  -\frac{\overrightarrow{PB}}{\overrightarrow{PC}}\cdot
  \frac{\overrightarrow{QC}}{\overrightarrow{QA}}\cdot
  \frac{\overrightarrow{RA}}{\overrightarrow{RB}}=
  -\frac{\overrightarrow{BB'}}{\overrightarrow{CC'}}\cdot
  \frac{\overrightarrow{CC'}}{\overrightarrow{AA'}}\cdot
  \frac{\overrightarrow{AA'}}{\overrightarrow{BB'}}=
  -1.
   \end{eqnarray*}
 ($\Leftarrow$) Let now the relation (\ref{formulaMenelaj}) be satisfied. With $P'$
  we denote the intersection of the lines $QR$ and $BC$. If the lines $QR$ and $BC$
   were parallel, from this by Tales' theorem it would follow
  $\frac{\overrightarrow{CQ}}{\overrightarrow{QA}}=
  \frac{\overrightarrow{RB}}{\overrightarrow{AR}}$
   and then from (\ref{formulaMenelaj}) also
   $\frac{\overrightarrow{BP}}{\overrightarrow{PC}}=-1$ or
   $\overrightarrow{BP}=\overrightarrow{CP}$, which
   is not possible. Let therefore $P'=QR\cap BC$. Then the points
   $P'$, $Q$, and $R$ are collinear. By the proven in the first part
   of the theorem
  we have:
 \begin{eqnarray*}
  \frac{\overrightarrow{BP'}}{\overrightarrow{P'C}}\cdot
  \frac{\overrightarrow{CQ}}{\overrightarrow{QA}}\cdot
  \frac{\overrightarrow{AR}}{\overrightarrow{RB}}=-1.
  \end{eqnarray*}
  From this and the assumed relations (\ref{formulaMenelaj}) it follows
  $\frac{\overrightarrow{BP}}{\overrightarrow{PC}}=
  \frac{\overrightarrow{BP'}}{\overrightarrow{P'C}}$   or $P=P'$ (theorem \ref{izrekEnaDelitevDaljiceVekt}),
   which means that the points $P$, $Q$ and $R$ are collinear.
 \kdokaz


We continue with the use of two proven theorems.

\bizrek
                The lines, joining the vertices of a triangle to the tangent points of the incircle, intersect at one point (so-called \index{točka!Gergonova}\pojem{Gergonne\footnote{To trditev je dokazal \index{Gergonne, J. D.}\textit{J. D. Gergonne} (1771--1859), francoski matematik. V 19. stoletju je bila posvečena posebna
                pozornost metričnim lastnostim trikotnika, tako so bile odkrite tudi druge karakteristične točke trikotnika (glej naslednja zgleda).} point} \color{blue} of this triangle).
                \eizrek


\begin{figure}[!htb]
\centering
\input{sl.pod.7.5.3.pic}
\caption{} \label{sl.pod.7.5.3.pic}
\end{figure}


 \textbf{\textit{Proof.}}
Let $P$, $Q$ and $R$ be the points in which the inscribed circle of the triangle $ABC$ (see izrek \ref{SredVcrtaneKrozn}) touches its
sides $BC$, $CA$ and $AB$ (Figure \ref{sl.pod.7.5.3.pic}). We prove that the lines $AP$, $BQ$ and $CR$ intersect at one point. By izrek \ref{TangOdsek} the corresponding tangent lines are: $BP\cong BR$, $CP\cong CQ$ and
$AQ\cong AR$. Because $\mathcal{B}(B,P,C)$, $\mathcal{B}(C,Q,A)$ and $\mathcal{B}(A,R,B)$, we have $\frac{\overrightarrow{BP}}{\overrightarrow{PC}}
\cdot \frac{\overrightarrow{CQ}}{\overrightarrow{QA}}
\cdot \frac{\overrightarrow{AR}}{\overrightarrow{RB}} >0$. Therefore:
$$\frac{\overrightarrow{BP}}{\overrightarrow{PC}}
\cdot \frac{\overrightarrow{CQ}}{\overrightarrow{QA}}
\cdot \frac{\overrightarrow{AR}}{\overrightarrow{RB}}=
\frac{|BP|}{|PC|}\cdot\frac{|CQ|}{|QA|}\cdot\frac{|AR|}{|RB|}=1.$$
 By Ceva's izrek \ref{izrekCeva} the lines $AP$, $BQ$ and $CR$ belong to one family. Because of Pash's axiom \ref{PaschIzrek} the lines $AP$ and $BQ$ intersect, it is an elliptic family, which means that the lines $AP$, $BQ$ and $CR$ intersect at one point.
 \kdokaz

\bzgled
                Prove that the lines, joining the vertices of a triangle to the tangent points of the excircles to the opposite sides, intersect at one point (so-called \index{točka!Nagelova}\pojem{Nagel\footnote{\index{Nagel, C. H.}\textit{C. H. Nagel} (1803--1882), German mathematician, who published this theorem in 1836. Each of the lines in this theorem divides the area of the triangle into two equal parts, so the theorem is also called \index{izrek!o polobsegu trikotnika}\pojem{the statement about the half-perimeter of the triangle}.
                point} \color{green1} of this triangle).
                \ezgled

\begin{figure}[!htb]
\centering
\input{sl.pod.7.5.6.pic}
\caption{} \label{sl.pod.7.5.6.pic}
\end{figure}


 \textbf{\textit{Proof.}}
 We use the notation from the big task (\ref{velikaNaloga}). We prove that the lines $AP_a$, $BQ_b$ and $CR_c$ intersect at one point (Figure \ref{sl.pod.7.5.6.pic}). If we use the fact $\frac{\overrightarrow{BP_a}}{\overrightarrow{P_aC}}
\cdot \frac{\overrightarrow{CQ_b}}{\overrightarrow{Q_bA}}
\cdot \frac{\overrightarrow{AR_c}}{\overrightarrow{R_cB}} >0$ and the relations from the big task (\ref{velikaNaloga}), we get:
\begin{eqnarray*}
\frac{\overrightarrow{BP_a}}{\overrightarrow{P_aC}}
\cdot \frac{\overrightarrow{CQ_b}}{\overrightarrow{Q_bA}}
\cdot \frac{\overrightarrow{AR_c}}{\overrightarrow{R_cB}}&=&
\frac{|BP_a|}{|P_aC|}\cdot\frac{|CQ_b|}{|Q_bA|}\cdot\frac{|AR_c|}{|R_cB|}=\\&=&
\frac{s-c}{s-b}\cdot\frac{s-a}{s-c}\cdot\frac{s-b}{s-a}=
1
\end{eqnarray*}
By Ceva's theorem \ref{izrekCeva} the lines $AP_a$, $BQ_b$ and $CR_c$ belong to one pencil. Since, as a consequence of Pash's axiom \ref{PaschIzrek}, the lines $AP_a$ and $BQ_b$ intersect, it is an elliptic pencil, which means that the lines $AP_a$, $BQ_b$ and $CR_c$ intersect at one point.
 \kdokaz

\bzgled
                Prove that the lines, joining the vertices of a triangle to the
                 points dividing opposite sides in the ratio of squares of adjacent sides,
                  intersect at one point (so-called \index{točka!Lemoinova}
                \pojem{Lemoine\footnote{\index{Lemoine, E. M. H.}\textit{E. M. H. Lemoine} (1751--1816), French mathematician.} point}  \color{green1}  of this triangle).
                \ezgled



\begin{figure}[!htb]
\centering
\input{sl.pod.7.5.5.pic}
\caption{} \label{sl.pod.7.5.5.pic}
\end{figure}


 \textbf{\textit{Proof.}} Let $X$, $Y$ and $Z$ be points on sides $BC$, $AC$ and $AB$ of triangle $ABC$, such that $\frac{|BX|}{|XC|}=\frac{|BA|^2}{|AC|^2}$, $\frac{|CY|}{|YA|}=\frac{|CB|^2}{|BA|^2}$ and $\frac{|AZ|}{|ZB|}=\frac{|AC|^2}{|CB|^2}$
 (Figure \ref{sl.pod.7.5.5.pic}).
 Because $\frac{\overrightarrow{BX}}{\overrightarrow{XC}}
\cdot \frac{\overrightarrow{CY}}{\overrightarrow{YA}}
\cdot \frac{\overrightarrow{AZ}}{\overrightarrow{ZB}} >0$, we have:
\begin{eqnarray*}
\frac{\overrightarrow{BX}}{\overrightarrow{XC}}
\cdot \frac{\overrightarrow{CY}}{\overrightarrow{YA}}
\cdot \frac{\overrightarrow{AZ}}{\overrightarrow{ZB}}&=&
\frac{|BX|}{|XC|}\cdot\frac{|CY|}{|YA|}\cdot\frac{|AZ|}{|ZB|}=\\
&=&
\frac{|BA|^2}{|AC|^2}\cdot\frac{|CB|^2}{|BA|^2}\cdot\frac{|AC|^2}{|CB|^2}=
1.
\end{eqnarray*}
 By Ceva's theorem \ref{izrekCeva} lines $AX$, $BY$ and $CZ$ belong to one pencil. Because of Pash's axiom \ref{PaschIzrek} lines $AX$ and $BY$ intersect, it is an elliptic pencil, which means that lines $AX$, $BY$ and $CZ$ intersect in one point.
 \kdokaz

\bzgled
                Let $Z$ and $Y$ be points of the sides $AB$ and $AC$ of a triangle
                $ABC$ such that $AZ:ZB=1:3$ and $AY:YC=1:2$, and $X$ the point,
                in which the line $YZ$ intersects the line containing the side $BC$ of this triangle. Calculate
                 $\overrightarrow{BX}:\overrightarrow{XC}$.
                \ezgled


\begin{figure}[!htb]
\centering
\input{sl.pod.7.5.4.pic}
\caption{} \label{sl.pod.7.5.4.pic}
\end{figure}


 \textbf{\textit{Proof.}}
 (Figure \ref{sl.pod.7.5.4.pic})

Since the points $X$, $Y$ and $Z$ are collinear, by Menelaus's theorem \ref{izrekMenelaj} it follows that:
$$\frac{\overrightarrow{BX}}{\overrightarrow{XC}}
\cdot\frac{\overrightarrow{CY}}{\overrightarrow{YA}}
\cdot\frac{\overrightarrow{AZ}}{\overrightarrow{ZB}}=-1.$$
If we use the given conditions, we get:
$$\frac{\overrightarrow{BX}}{\overrightarrow{XC}}
\cdot\frac{2}{1}
\cdot\frac{1}{3}=-1$$
or $\overrightarrow{BX}:\overrightarrow{XC}=-3:2$.
\kdokaz



                \bzgled
                Prove that the tangents of the circumcircle of an arbitrary scalene triangle $ABC$ at its
                vertices $A$, $B$ and $C$ intersect the lines containing the opposite sides of this triangle
                at three collinear points\footnote{This is a special case of Pascal's theorem \ref{izrekPascalEvkl} (section \ref{odd7PappusPascal}). \index{Pascal, B.} \textit{B. Pascal} (1623--1662), French mathematician and philosopher.}.
                \ezgled


\begin{figure}[!htb]
\centering
\input{sl.pod.7.12.2a.pic}
\caption{} \label{sl.pod.7.12.2a.pic}
\end{figure}

\textbf{\textit{Solution.}}
 Let $X$, $Y$ and $Z$ be the points of intersection
 of the tangent to the circle $k$ of the triangle $ABC$ in the vertices $A$, $B$ and $C$
 with the sides $BC$, $AC$ and $AB$ of this
 triangle (Figure \ref{sl.pod.7.12.2a.pic}). We prove that $X$, $Y$ and $Z$ are collinear points. By \ref{ObodKotTang} we have $\angle XAB\cong\angle ACB$. Because the triangle $AXB$ and $CXA$ have a common internal angle at the vertex $X$, we have $\triangle AXB\sim \triangle CXA$ (\ref{PodTrikKKK}), so:
 $$\frac{|AX|}{|CX|}=\frac{|AB|}{|AC|}=\frac{|BX|}{|AX|}.$$
From $\frac{|AX|}{|CX|}=\frac{|BX|}{|AX|}$ it follows that $|XB|\cdot |XC|=|XA|^2$. If we divide this relation by $|XC|^2$, we get $\frac{|BX|}{|CX|}=\frac{|AX|^2}{|CX|^2}$. Since $\frac{|AX|}{|CX|}=\frac{|AB|}{|AC|}$, we finally get:
 \begin{eqnarray*}
&& \frac{|BX|}{|XC|}=\frac{|BA|^2}{|AC|^2}
 \end{eqnarray*}
 and similarly
 \begin{eqnarray*}
&& \frac{|CY|}{|YA|}=\frac{|CB|^2}{|BA|^2},\\
&& \frac{|AZ|}{|ZB|}=\frac{|AC|^2}{|CB|^2}.
 \end{eqnarray*}
 Because the points $X$, $Y$ and $Z$ do not lie on the sides $BC$, $AC$ and $AB$, we have $\frac{\overrightarrow{BX}}{\overrightarrow{XC}}
\cdot \frac{\overrightarrow{CY}}{\overrightarrow{YA}}
\cdot \frac{\overrightarrow{AZ}}{\overrightarrow{ZB}} <0$, so:
\begin{eqnarray*}
\frac{\overrightarrow{BX}}{\overrightarrow{XC}}
\cdot \frac{\overrightarrow{CY}}{\overrightarrow{YA}}
\cdot \frac{\overrightarrow{AZ}}{\overrightarrow{ZB}}&=&
-\frac{|BX|}{|XC|}\cdot\frac{|CY|}{|YA|}\cdot\frac{|AZ|}{|ZB|}=\\
&=&
-\frac{|BA|^2}{|AC|^2}\cdot\frac{|CB|^2}{|BA|^2}\cdot\frac{|AC|^2}{|CB|^2}=
-1.
  \end{eqnarray*}
  By Menelaus' theorem \ref{izrekMenelaj}, the points $X$, $Y$ and $Z$ are collinear.
  \kdokaz

The next property of quadrilaterals is a continuation
      of examples
       \ref{TetivniVcrtana} and \ref{TetivniVisinska}. Namely, we are dealing with the following problem.
       Given is a quadrilateral $ABCD$ with the characteristic points of the triangles $BCD$, $ACD$, $ABD$ and $ABC$ as vertices. What does the quadrilateral represent, that has the centroid of the triangle $BCD$, $ACD$, $ABD$ and $ABC$ as vertices? The answer is trivial if the centroid of the circumscribed circle is given, since in that case $ABCD$ is a quadrilateral with one point as the centroid of the circumscribed circles and the sought quadrilateral does not exist. The cases of the centroids of the circumscribed circles and the altitude points are treated in the aforementioned examples \ref{TetivniVcrtana} and \ref{TetivniVisinska}. The next example gives the answer for the case of the centroids of the triangles, in the case of an arbitrary quadrilateral.




            \bzgled \label{TetivniTezisce}
              Let $ABCD$ be an arbitrary quadrilateral and
            $T_A$ the centroid of the triangle $BCD$,
            $T_B$ the centroid of the triangle $ACD$,
            $T_C$ the centroid of the triangle $ABD$ and
            $T_D$ the centroid of the triangle $ABC$.
            Prove that:\\
              a) a) the lines $AT_A$, $BT_B$, $CT_C$ and $DT_D$ intersect at one point which is the common centroid
             of the quadrilaterals $ABCD$ and $T_AT_BT_CT_D$,\\
                b) the quadrilateral $T_AT_BT_CT_D$ is similar to the quadrilateral $ABCD$ with the coefficient
            of similarity equal to $\frac{1}{3}$.
            \ezgled

\begin{figure}[!htb]
\centering
\input{sl.pod.7.2.9.pic}
\caption{} \label{sl.pod.7.2.9.pic}
\end{figure}

 \textbf{\textit{Proof.}} (Figure \ref{sl.pod.7.2.9.pic})

The lines $AT_A$, $BT_B$, $CT_C$ and $DT_D$ intersect at one point which is the common centroid of the quadrilaterals $ABCD$ and $T_AT_BT_CT_D$. The quadrilateral $T_AT_BT_CT_D$ is similar to the quadrilateral $ABCD$ with the coefficient of similarity equal to $\frac{1}{3}$.

\textit{a)}
We mark with $P$, $K$, $Q$, $L$, $M$ and $N$ the centers of the lines $AB$, $BC$, $CD$, $DA$, $AC$ and $BD$.
Since the point $T_A$ is the center of gravity of the triangle $BCD$, the intersection of its medians $BQ$, $CN$ and $DK$, it also holds that $QT_A:T_AB=1:2$ (\ref{tezisce}) or $\overrightarrow{QT_A}=\frac{1}{3}\overrightarrow{QB}$. Similarly, from the triangle $ACD$ we get that $\overrightarrow{QT_B}=\frac{1}{3}\overrightarrow{QA}$. From the last two relations and \ref{vektVektorskiProstor} we obtain: $$\overrightarrow{T_BT_A}=\overrightarrow{T_BQ}+\overrightarrow{QT_A}
=\frac{1}{3}\overrightarrow{AQ}+\frac{1}{3}\overrightarrow{QB}=
\frac{1}{3}\left(\overrightarrow{AQ}+\overrightarrow{QB} \right)=\frac{1}{3}\overrightarrow{AB}.$$
Therefore:
\begin{eqnarray} \label{eqnCevaTez1}
\overrightarrow{T_BT_A}=\frac{1}{3}\overrightarrow{AB},
\end{eqnarray}
so $T_BT_A\parallel AB$. We mark with $T$ the intersection of the lines $AT_A$ and $BT_B$. By Tales' theorem and \ref{eqnCevaTez1} it holds that:
\begin{eqnarray} \label{eqnCevaTez2}
\frac{\overrightarrow{TT_A}}{\overrightarrow{TA}}
=\frac{\overrightarrow{TT_B}}{\overrightarrow{TB}}=
\frac{\overrightarrow{T_AT_B}}{\overrightarrow{AB}}=-\frac{1}{3}.
\end{eqnarray}
The lines $AT_A$ and $BT_B$ therefore intersect in the point $T$, which divides it in the ratio $2:1$. Similarly, the lines $AT_A$ and $CT_C$ or the lines $AT_A$ and $DT_D$ intersect in a point which divides it in the ratio $2:1$, and that is (because of the line $AT_A$) precisely the point $T$. This means that the lines $AT_A$, $BT_B$, $CT_C$ and $DT_D$ intersect in the point $T$.

We will now prove that point $T$ is the center of mass of the quadrilaterals $ABCD$ and $T_AT_BT_CT_D$.
Since
$$\frac{\overrightarrow{QT_B}}{\overrightarrow{T_BA}}\cdot
\frac{\overrightarrow{AP}}{\overrightarrow{PB}}\cdot
\frac{\overrightarrow{BT_A}}{\overrightarrow{T_AQ}}=
\frac{1}{2}\cdot\frac{1}{1}\cdot\frac{2}{1}=1,
$$
by Ceva's theorem \ref{izrekCeva} for the triangle $QAB$ the lines $AT_A$, $BT_B$ and $PQ$ intersect in one point - point $T$. Therefore point $T$ lies on the line $PQ$. Similarly, point $T$ lies on the line $KL$, which means that point $T$ is the intersection of the diagonals $PQ$ and $KL$ of the quadrilateral $PKQL$. Since this is a Varignon parallelogram of the quadrilateral $ABCD$, point $T$ is, by theorem \ref{vektVarignon}, the center of mass of the quadrilateral $ABCD$. Since from \ref{eqnCevaTez2} it also follows that:
$$\overrightarrow{TT_A}+\overrightarrow{TT_B}+\overrightarrow{TT_C}+\overrightarrow{TT_D}=
-\frac{1}{3}\cdot\left(
\overrightarrow{TA}+\overrightarrow{TB}+\overrightarrow{TC}+\overrightarrow{TD}
\right)=-\frac{1}{3}\cdot \overrightarrow{0}=\overrightarrow{0},$$
point $T$ is also the center of mass of the quadrilateral $T_AT_BT_CT_D$.

\textit{b)} From \ref{eqnCevaTez2} it follows that $\overrightarrow{TT_A}=-\frac{1}{3}\overrightarrow{TA}$,
$\overrightarrow{TT_B}=-\frac{1}{3}\overrightarrow{TB}$,
$\overrightarrow{TT_C}=-\frac{1}{3}\overrightarrow{TC}$ and
$\overrightarrow{TT_D}=-\frac{1}{3}\overrightarrow{TD}$, or
$h_{T,-\frac{1}{3}}:\hspace*{1mm}A,B,C,D\mapsto T_A,T_B,T_C,T_D$, which means that the quadrilateral $T_AT_BT_CT_D$ is similar to the quadrilateral
         $ABCD$ with the similarity coefficient $\frac{1}{3}$.
\kdokaz

%________________________________________________________________________________
 \poglavje{Harmonic Conjugate Points. Apollonius Circle}
  \label{odd7Harm}


Since it is very important for what follows, we will write theorem \ref{izrekEnaDelitevDaljiceVekt} from section \ref{odd5LinKombVekt} in another form.

\bizrek \label{HarmCetEnaSamaDelitev}
        If $A$ and $B$ are different points on the line $p$ and $\lambda\neq -1$
        an arbitrary real number, then  there is exactly one point $L$ on the line $p$ such that:
        $$ \frac{\overrightarrow{AL}}{\overrightarrow{LB}}=\lambda.$$
        \eizrek


In section \ref{odd5TalesVekt} we found out how to divide the distance $AB$ in the ratio $m:n$ (statement \ref{izrekEnaDelitevDaljice}). We will now use this idea to construct the point $L$ from the previous statement \ref{HarmCetEnaSamaDelitev}.



            \bzgled
            The points $A$ and $B$ and the real number $\lambda\neq -1$ are given.
            Construct a point $L$ on the line $AB$ such that:
        $$\frac{\overrightarrow{AL}}{\overrightarrow{LB}}=\lambda.$$
            \ezgled


\begin{figure}[!htb]
\centering
\input{sl.pod.7.4.1.pic}
\caption{} \label{sl.pod.7.4.1.pic}
\end{figure}

 \textit{\textbf{Solution.}}  (Figure \ref{sl.pod.7.4.1.pic})

In the case $\lambda=0$ the proof is direct - then $L=A$. Let $\lambda\neq 0$.
Construct
any parallel lines $a$ and $b$ through the points $A$ and $B$. With $P$ and $Q$ we mark  such points on the lines $a$ and $b$,
that $P,Q\div AB$ and $|AP|=|\lambda|$ and $|BQ|=1$. We also mark $Q'=\mathcal{S}_B(Q)$.
We will consider two cases.

\textit{1)} Let $\lambda>0$. It is clear that for the sought point $L$ it must hold that $\mathcal{B}(A,L,B)$. With $L_1$ we mark
the intersection of the lines $AB$ and $PQ$. By Tales' statement \ref{TalesovIzrek} it holds:
 $$\frac{\overrightarrow{AL_1}}{\overrightarrow{L_1B}}=
\frac{\overrightarrow{AP}}{\overrightarrow{QB}}=\lambda.$$

 \textit{2)} If $\lambda<0$, the sought point does not lie on the line $AB$. With $L_1$ we mark
the intersection of the lines $AB$ and $PQ'$. Because $\lambda\neq -1$, this intersection exists. By Tales' statement \ref{TalesovIzrek} it holds:
 $$\frac{\overrightarrow{AL_2}}{\overrightarrow{L_2B}}=
\frac{\overrightarrow{AP}}{\overrightarrow{Q'B}}
=-\frac{\overrightarrow{AP}}{\overrightarrow{BQ'}}=-\left(-\lambda\right)=\lambda,$$ which was to be proven. \kdokaz

From the previous example it is therefore clear that for each of the conditions $\frac{\overrightarrow{AL}}{\overrightarrow{LB}}=\lambda$ ($\lambda>0$) or $\frac{\overrightarrow{AL}}{\overrightarrow{LB}}=\lambda$ ($\lambda<0$ and $\lambda\neq -1$) there is exactly one solution for the point $L$ on the line $AB$. In the first case, the point $L$ lies on the line $AB$, so we say that this is the \index{delitev daljice!notranja}\pojem{inner division} of the line $AB$ in the ratio $\lambda$ ($\lambda>0$). In the second case, the point $L$ does not lie on the line $AB$, so we say that this is the \index{delitev daljice!zunanja}\pojem{outer division} of the line $AB$ in the ratio $|\lambda|$ ($\lambda<0$ and $\lambda\neq -1$).

We can also say that for the condition $\frac{AL}{LB}=\lambda$ ($\lambda>0$, $\lambda\neq 1$) there are two solutions for the point $L$ on the line - the inner or the outer division.

If $L_1$ and $L_2$ are the inner and outer division of the line (for the same $\lambda$), it holds:
$$\frac{\overrightarrow{AL_1}}{\overrightarrow{L_1B}}:
\frac{\overrightarrow{AL_2}}{\overrightarrow{L_2B}}=-1.$$
 If we define the expression $\frac{\overrightarrow{AL_1}}{\overrightarrow{L_1B}}:
\frac{\overrightarrow{AL_2}}{\overrightarrow{L_2B}}$ as the \index{dvorazmerje parov točk}\pojem{ratio of pairs of points} $(A,B)$ and $(L_1,L_2)$ and denote
$$d(A,B;L_1,L_2)=\frac{\overrightarrow{AL_1}}{\overrightarrow{L_1B}}:
\frac{\overrightarrow{AL_2}}{\overrightarrow{L_2B}},$$
 we see that for the inner and outer division of the line $AB$ with points $L_1$ and $L_2$ the corresponding ratio is $-1$, or $d(A,B;L_1,L_2)=-1$
This gives us the idea for a new definition.

We say that different collinear points $A$, $B$, $C$ and $D$ determine
 \index{harmonična četverica točk}
 \pojem{harmonično četverico točk} oz. that the pair $(A, B)$,
 \pojem{harmonično konjugiran} s parom  $(C, D)$, oznaka
 $\mathcal{H}(A,B;C,D)$, if:
  \begin{eqnarray}\label{formulaHarmEvkl}
  \frac{\overrightarrow{AC}}{\overrightarrow{CB}}=
  -\frac{\overrightarrow{AD}}{\overrightarrow{DB}},
 \end{eqnarray} or:
 \begin{eqnarray*}
d(A,B;C,D)=\frac{\overrightarrow{AC}}{\overrightarrow{CB}}:
\frac{\overrightarrow{AD}}{\overrightarrow{DB}}=-1.
 \end{eqnarray*}

We prove the basic properties of the defined relation.
 As we have already mentioned, the points $C$ and $D$, for which $\mathcal{H}(A,B;C,D)$ is true, represent
the internal and external division of the distance $AB$ in some ratio $\lambda$ ($\lambda>0$ and $\lambda\neq 1$). There are therefore an infinite number of such pairs $C$, $D$. But if one of the points $C$ or $D$ is given, the other is uniquely determined. We express this in other words in the following proposition.



        \bizrek \label{HarmCetEnaSamaTockaD}
        Let $C$ be a point on a line $AB$ different from the points $A$ and $B$
        and also from the midpoint of the line segment $AB$,
        then  there is exactly one point $D$ such that
        $\mathcal{H}(A,B;C,D)$.
        \eizrek


 \textit{\textbf{Proof.}} Let $\frac{\overrightarrow{AC}}{\overrightarrow{CB}}=\lambda$. Because $C$ is different from $A$, $B$ and
         the midpoint of the line segment $AB$, $\lambda\neq 0$ and $\lambda\neq 1$. We are looking for a point $D$, for which $\mathcal{H}(A,B;C,D)$ is true, or $d(A,B;C,D)=-1$ or equivalently $\frac{\overrightarrow{AD}}{\overrightarrow{DB}}=-\lambda$. Because $-\lambda\neq -1$, by  \ref{HarmCetEnaSamaDelitev} there is only one point $D$, for which this is fulfilled.
\kdokaz

\bizrek \label{HarmCetEF}
       Let $A$, $B$, $C$ and $D$ be four different collinear points
            on a line $p$ and $O$ a point not lying on this line. Suppose that
            a line that is parallel to the line $OA$ through the point $B$
            intersects the lines $OC$ and $OD$ at the points $E$ and $F$.
            Then:
        $$\mathcal{H}(A,B;C,D) \hspace*{1mm} \Leftrightarrow \hspace*{1mm} \mathcal{S}_B(E)=F.$$
        \eizrek


 \textit{\textbf{Proof.}}
  (Figure \ref{sl.pod.7.4.2.pic})

($\Rightarrow$) If $\mathcal{H}(A,B;C,D)$ is true, by Tales' theorem \ref{TalesovIzrek}:
$$\frac{\overrightarrow{AO}}{\overrightarrow{EB}}=
\frac{\overrightarrow{AC}}{\overrightarrow{CB}}=
-\frac{\overrightarrow{AD}}{\overrightarrow{DB}}=
-\frac{\overrightarrow{AO}}{\overrightarrow{FB}}=
\frac{\overrightarrow{AO}}{\overrightarrow{BF}},$$
 it follows that
$\overrightarrow{EB}=\overrightarrow{BF}$ or $\mathcal{S}_B(E)=F$.

($\Leftarrow$) Now suppose that $\mathcal{S}_B(E)=F$. From this it follows that $\overrightarrow{EB}=\overrightarrow{BF}$, so (again by Tales' theorem \ref{TalesovIzrek}):
$$\frac{\overrightarrow{AC}}{\overrightarrow{CB}}=
\frac{\overrightarrow{AO}}{\overrightarrow{EB}}=
\frac{\overrightarrow{AO}}{\overrightarrow{BF}}=
-\frac{\overrightarrow{AO}}{\overrightarrow{FB}}=
-\frac{\overrightarrow{AD}}{\overrightarrow{DB}},$$
so by definition $\mathcal{H}(A,B;C,D)$.
\kdokaz

\begin{figure}[!htb]
\centering
\input{sl.pod.7.4.2.pic}
\caption{} \label{sl.pod.7.4.2.pic}
\end{figure}

The previous theorem allows us to effectively construct the fourth point in a harmonic set of points.



        \bzgled \label{HarmCetEnaSamaTockaDKonstr}
        Let $C$ be a point that lies on a line $AB$. Suppose that $C$ is different from the points $A$,
        $B$ and the midpoint of the line segment $AB$.  Construct a point $D$ such that
        $\mathcal{H}(A,B;C,D)$.
        \ezgled

\textit{\textbf{Solution.}} According to the theorem \ref{HarmCetEnaSamaTockaD} there is only one such point $D$, for which $\mathcal{H}(A,B;C,D)$. Now we will construct it. Let $O$ be an arbitrary point, which does not lie on the line $AB$ (Figure \ref{sl.pod.7.4.2.pic}), and $l$ is a parallel to the line $AO$ in the point $B$. With $E$ we mark the intersection of the lines $OC$ and $l$ and $F=\mathcal{S}_B(E)$. The point $D$ is the intersection of the lines $OF$ and $AB$.
By the previous theorem \ref{HarmCetEF} it is $\mathcal{H}(A,B;C,D)$.
\kdokaz

In a similar way as in the previous example, for given collinear points $A$, $B$ and $D$ (the point $D$ does not lie on the line $AB$) we can plan such a point $C$, that $\mathcal{H}(A,B;C,D)$ is valid.

It is intuitively clear that from $\mathcal{H}(A,B;C,D)$ it follows $\mathcal{H}(A,B;D,C)$, because if the points $C$ and $D$ divide the line $AB$ in the same ratio, then the same is true for the points $D$ and $C$. It is interesting that from $\mathcal{H}(A,B;C,D)$ it also follows $\mathcal{H}(C,D;A,B)$, which means that if the points $C$ and $D$ divide the line $AB$ in the same ratio, then the points $A$ and $B$ divide the line $CD$ in the same ratio. We prove both properties formally.

\bizrek
a) $\mathcal{H}(A,B;C,D) \hspace*{1mm} \Rightarrow
\hspace*{1mm} \mathcal{H}(A,B;D,C)$; \\
\hspace*{22mm}b) $\mathcal{H}(A,B;C,D)
\hspace*{1mm} \Rightarrow
\hspace*{1mm} \mathcal{H}(C,D;A,B)$;
\eizrek


\textit{\textbf{Proof.}}

$$a)\hspace*{1mm}\mathcal{H}(A,B;C,D)
\hspace*{1mm} \Rightarrow
                 \hspace*{1mm}
 \frac{\overrightarrow{AC}}{\overrightarrow{CB}}= -\frac{\overrightarrow{AD}}{\overrightarrow{DB}}
\hspace*{1mm} \Rightarrow
                 \hspace*{1mm}
\frac{\overrightarrow{AD}}{\overrightarrow{DB}}= -\frac{\overrightarrow{AC}}{\overrightarrow{CB}}
\hspace*{1mm} \Rightarrow
                 \hspace*{1mm}
\mathcal{H}(A,B;D,C).$$


 $$b)\hspace*{1mm}\mathcal{H}(A,B;C,D)
\hspace*{1mm} \Rightarrow
                 \hspace*{1mm}
 \frac{\overrightarrow{AC}}{\overrightarrow{CB}}= -\frac{\overrightarrow{AD}}{\overrightarrow{DB}}
\hspace*{1mm} \Rightarrow
                 \hspace*{1mm}
\frac{\overrightarrow{CA}}{\overrightarrow{AD}}= -\frac{\overrightarrow{CB}}{\overrightarrow{BD}}
\hspace*{1mm} \Rightarrow
                 \hspace*{1mm}
\mathcal{H}(C,D;A,B),$$ which had to be proven. \kdokaz



             \bizrek \label{izrek 1.2.1}
            Let $A$, $B$, $C$ and $D$ be different collinear points.
            Then $\mathcal{H}(A,B;C,D)$ if and only if there exists such a quadrilateral $PQRS$,
            that:
            $$A=PQ\cap RS, \hspace*{2mm}B=QR\cap PS, \hspace*{2mm} C\in PR \hspace*{2mm}
             \textrm{and}
            \hspace*{2mm} D\in QS.$$
            \eizrek

\textit{\textbf{Proof.}} ($\Rightarrow$) Let $PQRS$ be a quadrilateral,
such that $A=PQ\cap RS$, $B=QR\cap PS$, $C\in PR$ and $D\in QS$
(Figure \ref{sl.pod.7.4.7a.pic}). By Menelaus' theorem
\ref{izrekMenelaj} for the triangle $ABP$ and the line $QS$ we get:
$$\frac{\overrightarrow{AD}}{\overrightarrow{DB}}\cdot
   \frac{\overrightarrow{BS}}{\overrightarrow{SP}}\cdot
   \frac{\overrightarrow{PQ}}{\overrightarrow{QA}}=-1.$$
 Similarly, by Ceva's theorem \ref{izrekCeva} for the same
 triangle and the point $R$ we get:
 $$\frac{\overrightarrow{AC}}{\overrightarrow{CB}}\cdot
   \frac{\overrightarrow{BS}}{\overrightarrow{SP}}\cdot
   \frac{\overrightarrow{PQ}}{\overrightarrow{QA}}=1.$$
From these two relations it follows
$\frac{\overrightarrow{AC}}{\overrightarrow{CB}}=
-\frac{\overrightarrow{AD}}{\overrightarrow{DB}}$ or
$\mathcal{H}(A,B;C,D)$.

\begin{figure}[!htb]
\centering
\input{sl.pod.7.4.7a.pic}
\caption{} \label{sl.pod.7.4.7a.pic}
\end{figure}

 ($\Leftarrow$) Now, let's assume that
 $\mathcal{H}(A,B;C,D)$ holds. Without loss of generality, let the point
 $C$ be between the points $A$ and $B$; in the other two cases the proof
 proceeds in the same way. Let $P$ be an arbitrary point outside the line $AB$ and $Q$
 an arbitrary point between the points $A$ and $P$. By Pasch's\footnote{\index{Pasch, M.}
 \textit{M. Pasch}
 (1843--1930), German mathematician, who introduced the relation of order of points in his work
  \textit{Predavanja o novejši geometriji}  (Lectures on Modern Geometry) from 1882.} \ref{PaschIzrek}
  axiom, the line $PC$
  and the line $QB$ intersect in some point $R$ and the line $AR$ and the line $PB$ intersect in some
  point $S$, therefore the line $AB$ and the line $QS$ intersect in some point $D_1$ (from $AB\parallel QS$
   it follows that
  $C$ is the center of the line segment $AB$, which is not possible due to
  $\mathcal{H}(A,B;C,D)$). Now, from the first part of the proof ($\Rightarrow$)
   it follows $\mathcal{H}(A,B;C,D_1)$. Since by the assumption we also have
   $\mathcal{H}(A,B;C,D)$, by the uniqueness of the fourth point of a harmonic
   quadrilateral of points (theorem \ref{HarmCetEnaSamaTockaD})
    it follows $D=D_1$. Therefore $PQRS$ is the desired
   quadrilateral.
   \kdokaz

We observe that the previous statement is also valid in the case when $C$ is the center of the line $AB$ and $D$ is a point at infinity (Figure \ref{sl.pod.7.4.7a.pic}).

  The statement on the right side of the equivalence from the previous statement (the existence of the appropriate quadrilateral) could be accepted as the definition of the harmonic quadruple of points in Euclidean geometry. In projective geometry, this definition is even more natural, since in such a definition we do not use metric.

  From the previous statement it also follows that the relation of harmonic quadruple is preserved in the so-called \pojem{central projection} in space. Indeed, if $A$, $B$, $C$ and $D$ are points for which $\mathcal{H}(A,B;C,D)$ holds, and $A'$,  $B'$, $C'$ and $D'$ are the central projections of these points, then from the existence of the appropriate quadrilateral for the quadruple $A$, $B$, $C$, $D$ it follows that the appropriate quadrilateral exists also for the quadruple $A'$,  $B'$, $C'$, $D'$. The latter is the central projection of the first quadrilateral, since the central projection preserves collinearity. Of course, we should also include in the previous consideration the case when some of the central projections are points at infinity.

  \index{geometrija!projektivna}In this sense, projective geometry can be described as the geometry that deals with objects and properties that are preserved in the central projection. Because "being a harmonic quadruple of points" is one of such properties, it is the subject of study in projective geometry. Even more - in this geometry, this relation is one of the basic concepts (see \cite{Mitrovic})."

We have already seen that on the line $AB$ there are exactly two points that divide the distance $AB$ in
the ratio $\lambda>0$, $\lambda\neq 1$ (in the case $\lambda= 1$ this is only one point - the center of the line). These points represent
the inner and outer division of the line in this ratio and together with points $A$ and $B$ determine a harmonic quadruple of points. The question arises: What does the set of all
such points $X$ in the plane mean, so that $\frac{AX}{XB}=\lambda$ ($\lambda\neq 1$)? The answer is given by the following theorem.








             \bizrek \label{ApolonijevaKroznica}
               Suppose that $A$ and $B$ are points in the plane and $\lambda>0$, $\lambda\neq 1$
                an arbitrary real number. If $C$ and $D$ are points of the line $AB$ such that:
                $$\frac{\overrightarrow{AC}}{\overrightarrow{CB}}=
                -\frac{\overrightarrow{AD}}{\overrightarrow{DB}}=\lambda,$$
                i.e.  $\mathcal{H}(A,B;C,D)$, then the set of all points $X$ of this plane such that:
                $$\frac{AX}{XB}=\lambda,$$
                is a circle with the diameter $CD$
               (so-called  Apollonius Circle\footnote{\index{Apolonij}
             \textit{Apolonij iz Perge} (3.-- 2. st. pr. n.
            š.), starogrški matematik.}).
             \index{krožnica!Apolonijeva}
             \eizrek

\begin{figure}[!htb]
\centering
\input{sl.pod.7.4.3.pic}
\caption{} \label{sl.pod.7.4.3.pic}
\end{figure}

 \textit{\textbf{Proof.}}
Let $k$ be a circle with diameter $CD$. Let $X$ be an arbitrary point of this plane and $p$ a parallel line through point $B$ to line $AX$. With $E$ and $F$ we mark the intersections of lines $XC$ and $XD$
with line $p$ (Figure \ref{sl.pod.7.4.3.pic}). By Theorem \ref{HarmCetEF}, point $B$ is the center
of the line $EF$, because $\mathcal{H}(A,B;C,D)$. It needs to be proven that:
$$\frac{AX}{XB}=\lambda \hspace*{1mm} \Leftrightarrow \hspace*{1mm} X\in k.$$

($\Leftarrow$) Let $X\in k$. Then $\angle EXF=\angle CXD=90^0$ and $BE\cong BF\cong BX$  (statement \ref{TalesovIzrKroz2}). From this and from Tales's statement \ref{TalesovIzrek} it follows:
$$\frac{AX}{XB}=\frac{AX}{EB}=\frac{AC}{CB}=\lambda.$$

($\Rightarrow$) Let $\frac{AX}{XB}=\lambda$. From this and from Tales's statement \ref{TalesovIzrek} it follows:
$$\frac{AX}{XB}=\lambda=\frac{AC}{CB}=\frac{AX}{EB},$$
so $XB\cong EB\cong BF$. Therefore $EXF$
is a right angled triangle (statement \ref{TalesovIzrKroz2}). Then  $\angle CXD=\angle EXF=90^0$, which means that the point $X$ lies on the circle $k$ above
the diameter $CD$ (statement \ref{TalesovIzrKroz2}).
\kdokaz


For given points $A$ and $B$ and for different values $\lambda\in \mathbb{R}$ ($\lambda>0$, $\lambda\neq 1$) we have different Apollonius's circles (Figure \ref{sl.pod.7.4.4a.pic}). We will mark them with $\mathcal{A}_{AB,\lambda}$, but if we know which distance it is, we can also write it shorter $k_{\lambda}$. In the case $\lambda=1$ we are actually looking for a set of all such points $X$ of this plane, for which:
                $\frac{AX}{XB}=1$, i.e. $AX\cong BX$. Then the sought set does not represent a circle, but the symmetry $s_{AB}$ of the distance $AB$.

\begin{figure}[!htb]
\centering
\input{sl.pod.7.4.4a.pic}
\caption{} \label{sl.pod.7.4.4a.pic}
\end{figure}



            \bzgled
            Points $A$ and $B$ and a line $l$ in the plane are given.
            Construct the point $L$ on the line $l$ such that $LA:LB=5:2$.
            \ezgled

\begin{figure}[!htb]
\centering
\input{sl.pod.7.4.4.pic}
\caption{} \label{sl.pod.7.4.4.pic}
\end{figure}

 \textit{\textbf{Solution.}}
 (Figure \ref{sl.pod.7.4.4.pic})

Let's first draw a point $C$ on the line $AB$, so that $AC:CB=5:2$
 (see example \ref{izrekEnaDelitevDaljice}), and then the fourth point of the harmonic quartet  $\mathcal{H}(A,B;C,D)$ (see example \ref{HarmCetEnaSamaTockaDKonstr}). Construct the Apollonius circle $k_{\frac{5}{2}}$ over the diameter $CD$. The point $L$ is then the one for which $L\in k\cap l$.

By the previous theorem (\ref{ApolonijevaKroznica}) we have:
$$\frac{AL}{LB}=\frac{AC}{CB}=\frac{5}{2}.$$

The task has 0, 1 or 2 solutions, depending on the number of intersections of the circle $k$ and the line $l$.
\kdokaz



            \bizrek \label{HarmOhranjaVzporProj}
             A parallel projection preserves the relation of a harmonic conjugate points,
              i.e. if $A$, $B$, $C$, $D$ and $A'$, $B'$, $C'$, $D'$ are two quartets
            of collinear points such that $AA'\parallel BB'\parallel CC'\parallel DD'$, then:
            $$\mathcal{H}(A,B;C,D)\hspace*{1mm}\Rightarrow\hspace*{1mm}
            \mathcal{H}(A',B';C',D').$$
            \eizrek

\begin{figure}[!htb]
\centering
\input{sl.pod.7.4.5.pic}
\caption{} \label{sl.pod.7.4.5.pic}
\end{figure}

 \textit{\textbf{Solution.}}
 (Figure \ref{sl.pod.7.4.5.pic})

Let $\mathcal{H}(A,B;C,D)$ or $\frac{\overrightarrow{AC}}{\overrightarrow{CB}}:
\frac{\overrightarrow{AD}}{\overrightarrow{DB}}=-1$. By Tales' theorem \ref{TalesovIzrek} we have:
$$\frac{\overrightarrow{A'C'}}{\overrightarrow{C'B'}}:
\frac{\overrightarrow{A'D'}}{\overrightarrow{D'B'}}=
\frac{\overrightarrow{AC}}{\overrightarrow{CB}}:
\frac{\overrightarrow{AD}}{\overrightarrow{DB}}=-1,$$
which means that $\mathcal{H}(A',B';C',D')$ is also true.
\kdokaz

It is interesting that the central projection also preserves the relation of harmonic points. Therefore, the mentioned relation is the subject of research in projective geometry (see \cite{Mitrovic}).

\bizrek \label{HarmCetSimKota}
            Suppose that a line $BC$ intersects the bisectors of the interior and
            the exterior angle  at the vertex $A$ of a triangle $ABC$ at points $E$ and $F$. Then:

        a) $BE:CE=BF:CF=AB:AC$,

        b) $\mathcal{H}(B,C;E,F)$.
        \eizrek

\begin{figure}[!htb]
\centering
\input{sl.pod.7.4.6.pic}
\caption{} \label{sl.pod.7.4.6.pic}
\end{figure}

 \textit{\textbf{Solution.}} Let $L$ be an arbitrary point for which $\mathcal{B}(C,A,L)$ holds.
We mark with $M$ and $N$ the points in which the parallel to the line $AC$ through the point $B$ intersects in order the simetrals $AE$ and $AF$ of the internal and the external angle (Figure \ref{sl.pod.7.4.6.pic}).

\textit{a)} By izrek \ref{KotiTransverzala} we have:
 \begin{eqnarray*}
\angle BMA &\cong& \angle CAM=\angle CAE\cong\angle BAE=\angle BAM\\
\angle BNA &\cong& \angle LAN=\angle LAF\cong\angle BAF=\angle BAN
\end{eqnarray*}

Therefore, $AMB$ and $ANB$ are isosceles triangles with the bases $AM$ and $AN$ (izrek \ref{enakokraki} and it holds:
 $$BM\cong BA\cong BN.$$
From this and from the consequence of Tales' theorem \ref{TalesovIzrekDolzine} (because $AC\parallel NM$) we get:
\begin{eqnarray*}
\frac{BE}{EC}&=& \frac{BM}{AC}=\frac{BA}{AC},\\
\frac{BF}{FC}&=& \frac{BN}{AC}=\frac{BA}{AC}.
\end{eqnarray*}

\textit{b)} Because $\mathcal{B}(B,E,C)$ and $\neg\mathcal{B}(B,F,C)$, we have:
\begin{eqnarray*}
\frac{\overrightarrow{BE}}{\overrightarrow{EC}}&=& \frac{BA}{AC},\\
\frac{\overrightarrow{BF}}{\overrightarrow{FC}}&=& -\frac{BA}{AC}.
\end{eqnarray*}
Therefore:
\begin{eqnarray*}
\frac{\overrightarrow{BE}}{\overrightarrow{EC}}=
-\frac{\overrightarrow{BF}}{\overrightarrow{FC}},
\end{eqnarray*}
which means that $\mathcal{H}(B,C;E,F)$ holds.
\kdokaz

We will continue by using the previous theorem.

            \bzgled \label{HarmTrikZgl1}
            Construct a triangle $ABC$, with given: $a$, $t_a$, $b:c=2:3$.
            \ezgled

\begin{figure}[!htb]
\centering
\input{sl.pod.7.4.8.pic}
\caption{} \label{sl.pod.7.4.8.pic}
\end{figure}

 \textit{\textbf{Solution.}}
Let $ABC$ be a triangle, in which the side $BC\cong a$, the median $AA_1\cong t_a$ and $AC:AB=2:3$ (Figure \ref{sl.pod.7.4.8.pic}). So we can
construct the side $BC$ and its center $A_1$.
From the given conditions, the vertex $A$ lies on the circle
$k(A_1,t_a)$ and the Apollonius circle $\mathcal{A}_{BC,\frac{3}{2}}$, because $AB:AC=3:2$ (statement \ref{ApolonijevaKroznica}). So we get the vertex $A$ as one of the intersections of the circles
$k$ and $\mathcal{A}_{BC,\frac{3}{2}}$.

Although we do not need this fact in the construction, we mention that
the Apollonius circle $\mathcal{A}_{BC,\frac{3}{2}}$ represents the circle with the diameter $EF$, where $E$ and $F$ are defined as in the statement \ref{HarmCetSimKota}.
\kdokaz



            \bzgled
            Construct a parallelogram, with the sides congruent to
            given line segments $a$ and $b$, and the diagonals in the ratio $3:7$.
            \ezgled

\begin{figure}[!htb]
\centering
\input{sl.pod.7.4.9.pic}
\caption{} \label{sl.pod.7.4.9.pic}
\end{figure}

 \textit{\textbf{Solution.}}  (Figure \ref{sl.pod.7.4.9.pic})

Let $S$ be the intersection of the diagonals of the sought parallelogram $ABCD$, in which $AB\cong a$, $BC\cong b$ and $AC:BD=3:7$. We also mark with $P$ the center of the side $AB$.
Since the diagonals of a
parallelogram are divided in half, the line segments $SA$ and $SB$ are in the ratio $2:3$. This means that we can first construct
the triangle $ASB$, similar to the previous example \ref{HarmTrikZgl1}:
$AB\cong a$, $SP=\frac{1}{2}b$ and $SA:SB=3:7$.
 \kdokaz

\bizrek \label{harmVelNal}
                    Suppose that $k(S, r)$ is the incircle and $k_a (S_a, r_a)$ the excircle of a triangle $ABC$
            and $P$ and $P_a$ the touching points of these circles with the aide $BC$.
            Let $A'$ be the foot of the altitude $AA'$ and $E$
            intersection of the bisectors of the
            interior  angle at the vertex $A$ and the side $BC$.
            If $L$ and $L_a$ are foots of the perpendiculars from the points $S$ and $S_a$ on the line $AA'$, then:

        \hspace*{2mm} (i) $\mathcal{H}(A,E;S,Sa)$ \hspace*{2mm} (ii)
        $\mathcal{H}(A,A';L,La)$ \hspace*{2mm} (iii)
        $\mathcal{H}(A',E;P,Pa)$.
         \eizrek


\begin{figure}[!htb]
\centering
\input{sl.pod.7.3.7.pic}
\caption{} \label{sl.pod.7.3.7.pic}
\end{figure}

\textbf{\textit{Proof.}} (Figure \ref{sl.pod.7.3.7.pic})

\textit{(i)}  The lines $CS$ and $CS_a$ are the altitudes of the
triangle $ACE$, therefore by  \ref{HarmCetSimKota} we have $\mathcal{H}(A,E;S,S_a)$.

\textit{(ii)}  The points $A$, $A'$, $L$ and $L_a$
are the orthogonal projections of the points $A$, $E$, $S$ and $S_a$ on the line $AA'$. By  \ref{HarmOhranjaVzporProj} we have $\mathcal{H}(A,A';L,La)$.

\textit{(iii)} Similarly to the previous statement, since
the points $A'$, $E$, $P$ and $P_a$
 are the orthogonal projections of the points $A$, $E$, $S$ and $S_a$ on the line $BC$.
\kdokaz


            \bzgled Construct a triangle $ABC$, with given:

            \hspace*{4mm} (i) $r$, $a$, $v_a$ \hspace*{5mm} (ii) $v_a$,
             $r$, $b-c$
             \ezgled

\begin{figure}[!htb]
\centering
\input{sl.pod.7.4.10.pic}
\caption{} \label{sl.pod.7.4.10.pic}
\end{figure}

\textbf{\textit{Solution.}}
We use the notation from the big exercise \ref{velikaNaloga} and the statement \ref{harmVelNal} (Figure \ref{sl.pod.7.4.10.pic}). In both cases \textit{(i)} and \textit{(ii)} we can use the fact $\mathcal{H}(A,A';L,La)$ from the statement \ref{harmVelNal}, because from $AA'\cong v_a$ and $LA'\cong r_a$ we can plot the fourth point in the harmonic quadruple $\mathcal{H}(A,A';L,La)$. So we get $r_a\cong A'L_a$.

\textit{(i)} We use the relation $RR_a\cong a$ (big exercise \ref{velikaNaloga}), we plot the circles $k(S,SR)$ and $k_a(S_a,R_a)$, and then their three common tangents (example \ref{tang2ehkroz}).


\textit{(ii)} We use the relation $PP_a\cong b-c$ (big exercise \ref{velikaNaloga}), we plot the circles $k(S,SP)$ and $k_a(S_a,P_a)$, and then their three common tangents.
\kdokaz

In the previous example we saw how, with the help of the proven relation $\mathcal{H}(A,A';L,La)$, we can get $r_a$ from the elements of the triangle $v_a$ and $r$. In a similar way, from each known pair of a triple $(v_a,r,r_a)$ we get the third element. We will write this fact in the form $\langle v_a,r,r_a\rangle$.





            \bzgled
            Construct a cyclic quadrilateral $ABCD$, with the sides congruent to
            given line segments $a$, $b$, $c$ in $d$.
            \ezgled


\begin{figure}[!htb]
\centering
\input{sl.pod.7.4.11.pic}
\caption{} \label{sl.pod.7.4.11.pic}
\end{figure}

\textbf{\textit{Solution.}} (Figure \ref{sl.pod.7.4.11.pic})

Let $ABCD$ be a cyclic quadrilateral with
sides congruent to
given line segments $a$, $b$, $c$ and $d$, $h_{A,k}$ central stretch with center $A$
and coefficient $k=\frac{AD}{AB}=\frac{d}{a}$, $\mathcal{R}_{A,\alpha}$ rotation with
center $A$ for the oriented angle $\alpha=\angle BAD$ and the composite:
 $$\rho_{A,k,\alpha}=\mathcal{R}_{A,\alpha}\circ h_{A,k}$$
rotational stretch with center $A$, coefficient $k$ and angle $\alpha$.

Let $B'=h_{A,k}(B)$ and $C'=h_{A,k}(C)$. Because $|AB'|=k\cdot |AB|=\frac{|AD|}{|AB|}\cdot |AB|= |AD|$ and $\angle B'AD=\angle BAD= \alpha$, it follows that $R_{A,\alpha}(B')=D$, so $\rho_{A,k,\alpha}(B)=D$. Let $E=R_{A,\alpha}(C')$
or $E=\rho_{A,k,\alpha}(C)$. Then the triangle $ADE$ is the image of the triangle $ABC$ under the rotational stretch $\rho_{A,k,\alpha}$ (which is a similarity transformation), so these
two triangles are similar with a similarity coefficient of $k$. From this and from the statement \ref{TetivniPogoj} it follows that:
$$\angle EDC=\angle EDA+\angle ADC=\angle CBA+\angle ADC=180^0,$$
which means that the points $E$, $D$ and $C$ are collinear. The distance $ED$ can be constructed (using Tales' theorem \ref{TalesovIzrek}), because:
$$ED\cong B'C'=k\cdot BC=\frac{AD}{AB}\cdot BC=\frac{b\cdot d}{a}.$$
After the construction of the points $C$, $D$ and $E$, we can also construct the point $A$, because:
$$\frac{AE}{AC}=\frac{AC'}{AC}=k,$$
so the point $A$ lies on the Apollonius circle $\mathcal{A}_{EC,k}$, as well as on the circle $k(D,d)$.
\kdokaz


        \bnaloga\footnote{44. IMO Japan - 2003, Problem 4.}
         $ABCD$ is cyclic. The foot of the perpendicular from $D$ to the
        lines $AB$, $BC$, $CA$ are $P$, $Q$, $R$, respectively. Show that the angle bisectors of
        $\angle ABC$ and $\angle CDA$ meet on the line $AC$ if and only if $RP \cong RQ$.
        \enaloga

\begin{figure}[!htb]
\centering
\input{sl.pod.7.3.IMO1.pic}
\caption{} \label{sl.pod.7.3.IMO1.pic}
\end{figure}

\textbf{\textit{Solution.}} Without loss of generality, we assume that
$\mathcal{B}(B,C,Q)$ and $\mathcal{B}(B,P,A)$ hold. According to
\ref{SimpsPrem}, the points $P$, $Q$ and $R$ are collinear and lie on
\index{premica!Simsonova} Simson's line (Figure
\ref{sl.pod.7.3.IMO1.pic}). From the proof of this theorem it also follows that
 $\angle CDQ\cong \angle ADP$,
 which means that $CDQ$ and $ADP$ are similar right triangles,
 so we have: $$CD:CQ=AD:AP \textrm{ or } CD:AD=CQ:AP.$$
 Let us denote $P_1=\mathcal{S}_A(P)$, from the previous relation we get:
  \begin{eqnarray} \label{eqn.pod.7.3.IMO1}
  CD:AD=CQ:AP_1.
   \end{eqnarray}
($\Rightarrow$) First, let us assume that the altitudes of the angles $ABC$
and $CDA$ intersect at the point $E$, which lies on the line
 $AC$. According to \ref{HarmCetSimKota}, we then have:
 $BC:BA=CE:EA=DC:DA$. From this and from the proven relation
 \ref{eqn.pod.7.3.IMO1} it follows that $BC:BA=CQ:AP_1$ or $BC:CQ=BA:AP_1$.
 Because $\mathcal{B}(B,C,Q)$
 and $\mathcal{B}(B,A,P_1)$ hold, from the previous relation we get
 $BC:BQ=BA:BP_1$. By the converse of Tales' theorem
 \ref{TalesovIzrekObr}, $CA\parallel QP_1$ or $AR\parallel QP_1$,
  so by Tales'
  theorem
 \ref{TalesovIzrek} $PR:RQ=PA:AP_1=1:1$ or $PR\cong RQ$.

 ($\Leftarrow$) Now, let $PR\cong RQ$ hold. Because $A$ is the center of the line segment $PP_1$,
  $RA$ is the median
 of the triangle $PQP_1$. From this it follows that $AR\parallel QP_1$ or
 $CA\parallel QP_1$. By Tales' theorem, we have $BC:CQ=BA:AP_1$
 or $BC:BA=CQ:AP_1$. From the proven relation
 \ref{eqn.pod.7.3.IMO1} we now get $BC:BA=CD:AD$. Let $E_1$
 and $E_2$ be the intersections of the altitudes of the angles $ABC$ and $CDA$ with the line $AC$. If we use
 the previous relation and \ref{HarmCetSimKota}, we get:\\
 $CE_1:AE_1=BC:BA=CD:AD=CE_2:AE_2$. Because the points $E_1$ and $E_2$ both lie
  on the line segment $AC$,  $E_1=E_2$ (\ref{HarmCetEnaSamaDelitev}), so the altitudes of the angles
 $ABC$ and $CDA$ intersect on the line $AC$.
 \kdokaz

%________________________________________________________________________________
 \poglavje{The Right Triangle Altitude Theorem. Euclid's Theorems}  \label{odd7VisinEvkl}

In algebra and mathematical analysis, the concepts of arithmetic or geometric mean of two (or more) numbers are known.
At this point, we will first define the so-called arithmetic and geometric mean of two line segments.
We say that $x$ \index{aritmetična sredina daljic} \pojem{arithmetic mean} of line segments $a$ and $b$, if it holds:

\begin{eqnarray} \label{eqnVisEvkl1}
|x|=\frac{1}{2}\left( |a|+|b| \right).
\end{eqnarray}
Similarly, $y$ \index{geometrijska sredina daljic}\pojem{geometric (also geometrical) mean} of line segments $a$ and $b$, if it holds:

\begin{eqnarray} \label{eqnVisEvkl2}
|y|= \sqrt{|a|\cdot |b|}.
\end{eqnarray}

Therefore, arithmetic or geometric mean of two line segments is a line segment with a length that is equal to the arithmetic or geometric mean of the lengths of these two line segments. Relations \ref{eqnVisEvkl1} and \ref{eqnVisEvkl2} will often be written in a shorter form:
\begin{eqnarray*}
x=\frac{1}{2}\left( a+b \right)\hspace*{1mm}  \textrm{ or }\hspace*{1mm}
y=\sqrt{ab}.
\end{eqnarray*}

It is clear that we can construct the arithmetic mean of two given line segments in a very simple way. In this section we will derive the construction of the geometric mean of two given line segments. We will first prove the main statement that relates to right-angled triangles.




                \bizrek \index{izrek!višinski}\label{izrekVisinski}
                The altitude on the hypotenuse of a right-angled triangle
                 is the geometric mean
                 of the line segments into which it divides the hypotenuse.\\
                (The right triangle altitude theorem)
                \eizrek

\begin{figure}[!htb]
\centering
\input{sl.pod.7.6E.1.pic}
\caption{} \label{sl.pod.7.6E.1.pic}
\end{figure}

\textbf{\textit{Proof.}}
Let $ABC$ be a right-angled triangle with
hypotenuse $AB$ and altitude $CC'$. We denote $v_c=|CC'|$, $b_1=|AC'|$ and $a_1=|BC'|$ (Figure \ref{sl.pod.7.6E.1.pic}).

First, $\angle ACC'=90^0-\angle CAC'=\angle CBC'$ and
 $\angle BCC'=90^0-\angle CBC'=\angle CAC'$.
From the similarity of the right-angled triangles $CC'A$ and $BC'C$ (statement \ref{PodTrikKKK}) it follows that:
$$\frac{CC'}{BC'}=\frac{C'A}{C'C},$$
hence:
\begin{eqnarray} \label{eqnVisinski}
v_c^2=a_1\cdot b_1.
\end{eqnarray}
\kdokaz



                \bizrek \index{izrek!Evklidov}\label{izrekEvklidov}
                Each leg of a right-angled triangle
                 is the geometric mean
                  of the hypotenuse and the line segment of the hypotenuse adjacent to the leg.\\
                (Euclid's\footnote{Starogrški filozof in matematik \index{Evklid}\textit{Evklid iz Aleksandrije} (3. st. pr. n. š.).} theorems)
                \eizrek

\begin{figure}[!htb]
\centering
\input{sl.pod.7.6E.2.pic}
\caption{} \label{sl.pod.7.6E.2.pic}
\end{figure}

\textbf{\textit{Proof.}}
Let $ABC$ be a right-angled triangle with
hypotenuse $AB$ and altitude $CC'$. We denote $a=|BC|$,  $b=|AC|$, $c=|AB|$,  $b_1=|AC'|$ and $a_1=|BC'|$ (Figure \ref{sl.pod.7.6E.2.pic}).

As in the proof of statement \ref{izrekVisinski}, $\angle ACC'=90^0-\angle CAC'=\angle CBA$ and
 $\angle CAC'=\angle CAB$.
From the similarity of the right-angled triangles $CC'A$ and $BCA$ (statement \ref{PodTrikKKK}) it follows that:
$$\frac{CA}{BA}=\frac{C'A}{CA},$$
or:
\begin{eqnarray} \label{eqnEvklidov1}
b^2=b_1\cdot c.
\end{eqnarray}
Similarly, we have:
\begin{eqnarray} \label{eqnEvklidov2}
a^2=a_1\cdot c.
\end{eqnarray}
\kdokaz



                \bzgled \label{EvklVisPosl}
                If $c$ is the length of the hypotenuse, $a$ and $b$ the lengths of the legs and $v_c$
                the length of the altitude on the hypotenuse of a right-angled triangle $ABC$, then
                $$c\cdot v_c=a\cdot b.$$
                \ezgled

\textbf{\textit{Proof.}}
 By multiplying relations \ref{eqnEvklidov1} and \ref{eqnEvklidov2} from Euclid's theorem \ref{izrekEvklidov} and inserting the relation \ref{eqnVisinski} from the altitude theorem \ref{izrekVisinski} we get:
 $$a^2b^2=a_1b_1 c^2=v^2c^2,$$
 from which it follows that $ab=cv_c$.
 \kdokaz

 Now we will derive the predicted construction.



                \bzgled
                Construct a line segment $x$ that is the geometric mean of given line segments
                $a$ and $b$, i.e. $x=\sqrt{ab}$.
                \ezgled


\textbf{\textit{Proof.}} We will derive the construction in two ways.

\begin{figure}[!htb]
\centering
\input{sl.pod.7.6E.3.pic}
\caption{} \label{sl.pod.7.6E.3.pic}
\end{figure}

\textit{1)}
 First, let's draw points $P$, $Q$ and $R$ so that $PQ\cong a$, $QR\cong b$ and $\mathcal{B}(P,Q,R)$ (Figure \ref{sl.pod.7.6E.3.pic}), then the circle $k$ over the diameter $PR$ and finally the point $X$ as the intersection of the circle $k$ with the rectangle line $PR$ at point $Q$.

 We prove that $x=QX$ is the desired line segment. By Theorem \ref{TalesovIzrKroz2}, $PXR$ is a right triangle with hypotenuse $PR$, $XQ$ is its altitude on this hypotenuse. By the altitude theorem \ref{izrekVisinski}, $x$ is the geometric mean of line segments $a$ and $b$.

\textit{2)} Without loss of generality, let $a>b$ (if $a=b$ then also $x=a$).


\begin{figure}[!htb]
\centering
\input{sl.pod.7.6E.4.pic}
\caption{} \label{sl.pod.7.6E.4.pic}
\end{figure}

 First, let's draw points $P$, $Q$ and $R$ so that $PQ\cong a$, $QR\cong b$ and $\mathcal{B}(P,R,Q)$ (Figure \ref{sl.pod.7.6E.4.pic}), then the circle $k$ over the diameter $PQ$ and finally the point $X$ as the intersection of the circle $k$ with the rectangle line $PQ$ at point $R$.

We will now prove that $x=QX$ is the desired distance. According to the \ref{TalesovIzrKroz2} theorem, $PXQ$ is a right angled triangle with hypotenuse $PQ$, and $XR$ is its height on that hypotenuse. According to Euclid's \ref{izrekEvklidov} theorem, $x$ is the geometric mean of the distances $a$ and $b$.
\kdokaz

In the next example, we will consider another important construction.

                \bzgled \label{konstrKoren}
                Construct a line segment $x=e\sqrt{6}$, for a given line segment $e$.
                \ezgled

\textbf{\textit{Proof.}} Similarly to the previous example, we will carry out the construction in two ways.

\begin{figure}[!htb]
\centering
\input{sl.pod.7.6E.5.pic}
\caption{} \label{sl.pod.7.6E.5.pic}
\end{figure}

\textit{1)}
 First, let's draw points $P$, $Q$ and $R$, such that: $PQ=3e$, $QR=2e$ and $\mathcal{B}(P,Q,R)$ (Figure \ref{sl.pod.7.6E.5.pic}), then the circle $k$ over the diameter $PR$ and finally the point $X$ as the intersection of the circle $k$ with the perpendicular line $PR$ at point $Q$.

 We will now prove that $x=QX$ is the desired distance. According to the \ref{TalesovIzrKroz2} theorem, $PXR$ is a right angled triangle with hypotenuse $PR$, and $XQ$ is its height on that hypotenuse. According to the \ref{izrekVisinski} theorem, $x$ is the geometric mean of the distances $PQ=3e$ and $QR=2e$, i.e. $x=\sqrt{3e\cdot 2e}=e\sqrt{6}$.


\begin{figure}[!htb]
\centering
\input{sl.pod.7.6E.6.pic}
\caption{} \label{sl.pod.7.6E.6.pic}
\end{figure}

\textit{2)}
 First, let's draw points $P$, $Q$ and $R$, such that: $PQ=3e$, $QR=2e$ and $\mathcal{B}(P,R,Q)$ (Figure \ref{sl.pod.7.6E.6.pic}), then the circle $k$ over the diameter $PQ$ and finally the point $X$ as the intersection of the circle $k$ with the perpendicular line $PQ$ at point $R$.

We will prove that $x=QX$ is the desired distance. By the \ref{TalesovIzrKroz2} theorem, $PXQ$ is a right angled triangle with hypotenuse $PQ$, $XR$ is its height on that hypotenuse. By \ref{izrekEvklidov} theorem, $x$ is the geometric mean of the distances $PQ=3e$ and $QR=2e$, i.e. $x=\sqrt{3e\cdot 2e}=e\sqrt{6}$.
\kdokaz

 In the previous example, we have described the process of constructing the distance $x=e\sqrt{n}$, where $e$ is a given distance, and $n\in \mathbb{N}$ is a given natural number. We always have two options for construction - using the altitude theorem or the Pythagorean theorem. But if $n$ is a composite number, for example $n=12$, we can choose $PQ=3e$ and $QR=4e$, or $PQ=6e$ and $QR=2e$. If $n$ is a prime number, for example $n=7$, it is most advantageous to choose $PQ=7e$ and $QR=e$.


%________________________________________________________________________________
  \poglavje{Pythagoras' Theorem} \label{odd7Pitagora}

We will prove the famous Pythagorean theorem.



        \bizrek \label{PitagorovIzrek}
         \index{izrek!Pitagoras}
         The square of the length of the hypotenuse of a right-angled triangle is
            equal to the sum of the squares of the lengths of both legs, i.e. for a right-angled triangle
            $ABC$ with the hypotenuse of length $c$ and the legs of lengths $a$ and $b$ is
         $$a^2+b^2=c^2$$
        (Pythagoras'\footnote{Predpostavlja se, da je bil ta izrek znan že Egipčanom (pribl. 3000 let pr. n. š.) in Babiloncem (pribl. 2000
        let pr. n. š.), toda starogrški filozof in matematik \index{Pitagora}\textit{Pitagora z otoka Samosa} (582--497 pr. n. š.) ga je verjetno prvi dokazal. Prvi pisni dokument dokaza Pitagorovega izreka je dal \index{Evklid}\textit{Evklid iz Aleksandrije} (3. st. pr. n. š.) v svojem delu ‘‘Elementi’’.
        Pri Starih Grkih se je Pitagorov izrek običajno nanašal na  zvezo med ploščinami kvadratov nad stranicami pravokotnega trikotnika.} theorem)
        \eizrek

\begin{figure}[!htb]
\centering
\input{sl.pod.7.6.0.pic}
\caption{} \label{sl.pod.7.6.0.pic}
\end{figure}

\textbf{\textit{Proof.}} (Figure \ref{sl.pod.7.6.0.pic})

If we use the statement \ref{izrekEvklidov} and the labels from the proof of this statement, we get:
$$a^2 + b^2 = c\cdot a_1 + c\cdot b_1 = c\cdot (a_1 + b_1) = c^2,$$ which was to be proven. \kdokaz

In the previous form, Pythagorean Theorem applies to the squares of the lengths of the sides of a right-angled triangle. The second form, which applies to the relationship between the areas of the squares over the sides of a right-angled triangle, will be considered in section \ref{odd8PloTrik}.

Pythagorean Theorem allows us to calculate the third side if the first two sides of a right-angled triangle are given. If we denote the length of the hypotenuse with $c$ and the length of the catheti with $a$ and $b$, from Pythagorean Theorem we get:
\begin{eqnarray*}
c&=&\sqrt{a^2+b^2}\\
a&=&\sqrt{c^2-b^2}\\
b&=&\sqrt{c^2-a^2}
\end{eqnarray*}



        \bizrek \label{PitagorovIzrekObrat}
         \index{izrek!Obratni Pitagorov}
         Let $ABC$ be an arbitrary triangle. If
            $$|AC|^2+|BC|^2=|AB|^2,$$
            then $ABC$ is a right-angled triangle with the right angle at the vertex $C$.\\
        (Converse of Pythagorean Theorem)
        \eizrek

\begin{figure}[!htb]
\centering
\input{sl.pod.7.6.0a.pic}
\caption{} \label{sl.pod.7.6.0a.pic}
\end{figure}

\textbf{\textit{Proof.}} We mark: $c=|AB|$,  $b=|AC|$ and $a=|BC|$ (Figure \ref{sl.pod.7.6.0a.pic}). So $a^2+b^2=c^2$ is true, which means:
\begin{eqnarray} \label{eqnPitagObrat1}
|AB|=c=\sqrt{a^2+b^2}.
\end{eqnarray}
Let $A'B'C'$ be such a right-angled triangle with a right angle at the vertex $C'$, that
$A'C'\cong AC$ and $B'C'\cong BC$. By Pythagoras' theorem
$|A'B'|^2= |A'C'|^2+|B'C'|^2=b^2+a^2$
or:
\begin{eqnarray} \label{eqnPitagObrat2}
|A'B'|=\sqrt{a^2+b^2},
\end{eqnarray}
 therefore from \ref{eqnPitagObrat1} and  \ref{eqnPitagObrat2}  it follows that $A'B'\cong AB$.
The triangles $ABC$ and $A'B'C'$ are therefore congruent (by the \textit{SSS} \ref{SSS} theorem), from which it follows that $\angle ACB\cong \angle A'C'B'=90^0$, which means that $ABC$ is a right-angled triangle with a right angle at the vertex $C$.
\kdokaz



If the lengths of the sides $a$, $b$ and $c$ of the right-angled triangle $ABC$ (where $c$ is the length of the hypotenuse) are natural numbers, we say that the triple $(a,b,c)$ represents the \index{pitagorejska trojica}\pojem{Pythagorean triplet}. According to Pythagoras' \ref{PitagorovIzrek} theorem, for the Pythagorean triplet $(a, b, c)$ it is true that $a^2+b^2=c^2$. According to the converse Pythagoras' theorem, if for natural numbers $a$, $b$ and $c$ it is true that $a^2+b^2=c^2$, then $(a,b,c)$ is a Pythagorean triplet. The most famous Pythagorean triplet\footnote{This triplet was known to the Ancient Egyptians and Babylonians. It is also called the \index{trikotnik!egipčanski}\pojem{Egyptian triangle}, because the Ancient Egyptians used it to determine the right angle on the ground.} is $(3, 4, 5)$ (Figure \ref{sl.pod.7.6.0b.pic}), because $3^2+4^2=5^2$. We often encounter the Pythagorean triplets $(5, 12, 13)$ and $(7, 24, 25)$ as well.

In reality, there are an infinite number of Pythagorean triples. From just one such triple $(a,b,c)$, we can get an infinite number of them: $(ka, kb, kc)$ (for any $k\in \mathbb{N}$). Obviously, in this case, we are dealing with similar right-angled triangles.
We say that a \index{pitagorejska trojica!primitivna}\pojem{Pythagorean triple} is \pojem{primitive}, if $a$, $b$ and $c$ do not have a common divisor. It turns out that there are also an infinite number of primitive Pythagorean triples.

We can calculate new Pythagorean triples using the following method.
If $m$ and $n$ are any natural numbers ($m > n$), then:
 \begin{eqnarray*}
    a &=& m^2 - n^2,\\
    b &=& 2mn,\\
    c &=& m^2 + n^2.
 \end{eqnarray*}
 With a simple calculation, we can convince ourselves that $(a, b, c)$ really is a Pythagorean triple.

 We will use Pythagoras' theorem to calculate other shapes in the following sections.




        \bzgled \label{PitagorovPravokotnik}
        If $a$ and $b$ are the lengths of the sides and $d$ the lengths of the diagonal of
            a rectangle, then:
        $$d=\sqrt{a^2+b^2}.$$
        \ezgled

\begin{figure}[!htb]
\centering
\input{sl.pod.7.6.0c.pic}
\caption{} \label{sl.pod.7.6.0c.pic}
\end{figure}

\textbf{\textit{Proof.}} Let $ABCD$ be any rectangle. We label: $a=|AB|$,  $b=|BC|$ and $d=|AC|$ (Figure \ref{sl.pod.7.6.0c.pic}). Because $ABC$ is a right-angled triangle with hypotenuse $AC$, by Pythagoras' theorem \ref{PitagorovIzrek}, $d^2=a^2+b^2$ or  $d=\sqrt{a^2+b^2}$.
\kdokaz

  A direct consequence (for $a=b$) is the following statement.




        \bzgled \label{PitagorovKvadrat}
        If $a$ is the length of the side and $d$ the length of the diagonal of a square,
            then (Figure \ref{sl.pod.7.6.0d.pic}):
        $$d=a\sqrt{2}.$$
        \ezgled


\begin{figure}[!htb]
\centering
\input{sl.pod.7.6.0d.pic}
\caption{} \label{sl.pod.7.6.0d.pic}
\end{figure}

The fact that $\sqrt{2}$ is an irrational number ($\sqrt{2}\notin \mathbb{Q}$) or that it cannot be written in the form of a fraction, results in the fact that the diagonal $d$ of a square and the side $a$ are not \index{proportional lines}\pojem{proportional} or \index{comparable lines}\pojem{comparable} lines\footnote{The ancient Greeks did not yet know of irrational numbers. The Pythagoreans - a philosophical school founded by the Greek mathematician \index{Pythagoras}
        \textit{Pythagoras of Samos}
         (582--497 BC) - were of the belief that everything is a number. They meant rational numbers and believed that any two lines are proportional. The first to discover the opposite was the Greek mathematician and philosopher from the Pythagorean school \index{Hipas}\textit{Hipas of Metapontum} (5th century BC). Several legends are associated with this discovery, which was therefore in complete contrast to Pythagorean philosophy - from his suicide to the fact that the Pythagoreans drowned him or that they simply excluded him from their circle.}. This means that there is no line $e$ as a unit such that for some natural numbers $n,m\in \mathbb{N}$ it would hold that $d=n\cdot e$ and $a=m\cdot e$.



        \bzgled \label{PitagorovEnakostr}
        Suppose that $a$ is the length of the side of an equilateral triangle $ABC$.
        If $v$ is the length of the altitude, $R$ the circumradius and $r$ the inradius
         of that triangle, then
        $$v=\frac{a\sqrt{3}}{2},\hspace{2mm} R=\frac{a\sqrt{3}}{3},
        \hspace{2mm} r=\frac{a\sqrt{3}}{6}.$$
        \ezgled

\begin{figure}[!htb]
\centering
\input{sl.pod.7.6.0e.pic}
\caption{} \label{sl.pod.7.6.0e.pic}
\end{figure}

\textbf{\textit{Proof.}} Let $AA'$ be the altitude of the triangle $ABC$ (Figure \ref{sl.pod.7.6.0e.pic}).

In section \ref{odd3ZnamTock} we have proven that in an equilateral triangle all four characteristic points are equal ($O=S=T=V$). This means that the center $T$ of the triangle $ABC$ is at the same time the center of the circumscribed and inscribed circle of this triangle. Also, $T$ is at the same time the altitude point of this triangle, which means that $AA'$ is also the median, therefore the point $A'$ is the center of the side $BC$, or $|BA'|=\frac{a}{2}$. By the theorem \ref{tezisce} it follows that $AT:TA'=2:1$. Therefore:
 \begin{eqnarray} \label{eqnPitagEnakostr1}
 R=|TA|=\frac{2}{3}|AA'|=\frac{2}{3}v \hspace*{1mm} \textrm{ and } \hspace*{1mm}
 r=|TA'|=\frac{1}{3}|AA'|=\frac{1}{3}v.
 \end{eqnarray}

Since $ABA'$ is a right triangle with hypotenuse $AB$, by the Pythagorean theorem \ref{PitagorovIzrek}
\begin{eqnarray*}
 v=\sqrt{a^2-\left(\frac{a}{2} \right)^2}=\sqrt{\frac{3a^2}{4}}=\frac{a\sqrt{3}}{2}.
 \end{eqnarray*}
 If we use the relation from \ref{eqnPitagEnakostr1}, we get $R=\frac{a\sqrt{3}}{3}$
        and $r=\frac{a\sqrt{3}}{6}$.
\kdokaz



        \bzgled \label{PitagorovRomb}
        If $a$ is the length of the side and $e$ and $f$ the lengths of the diagonals of a rhombus, then
        $$\left(\frac{e}{2}\right)^2+\left(\frac{f}{2}\right)^2=a^2.$$
        \ezgled

\begin{figure}[!htb]
\centering
\input{sl.pod.7.6.0f.pic}
\caption{} \label{sl.pod.7.6.0f.pic}
\end{figure}

\textbf{\textit{Proof.}} Let $S$ be the intersection of the diagonals of the rhombus $ABCD$ (Figure \ref{sl.pod.7.6.0f.pic}). By the theorem \ref{paralelogram} $S$ is the center of the diagonals $AC$ and $BD$. Therefore $|SA|=\frac{e}{2}$ and $|SB|=\frac{f}{2}$. By the theorem \ref{RombPravKvadr} the diagonals $AC$ and $BD$ are perpendicular, therefore $ASB$ is a right triangle with hypotenuse of length $|AB|=c$ and the cathets of lengths $|SA|=\frac{e}{2}$ and $|SB|=\frac{f}{2}$. The desired relation is now a direct consequence of the Pythagorean theorem \ref{PitagorovIzrek}.
\kdokaz

\bzgled
                If $c$ is the length of the hypotenuse, $a$ and $b$ the lengths of the legs
                 and $v_c$ the length of the altitude on the hypotenuse of a right-angled
                triangle, then
                $$\frac{1}{a^2}+\frac{1}{b^2}=\frac{1}{v_c^2}.$$
                \ezgled


\textbf{\textit{Proof.}}
 If we use the Pythagorean Theorem \ref{PitagorovIzrek} and the statement from Example \ref{EvklVisPosl}, we get:
  $$\frac{1}{a^2}+\frac{1}{b^2}=\frac{a^2+b^2}{a^2b^2}
  =\frac{c^2}{a^2b^2}=\frac{c^2}{c^2v_c^2}=\frac{1}{v_c^2},$$ which was to be proven. \kdokaz

In Example \ref{konstrKoren} we considered two ways of constructing the line segment $x=e\sqrt{n}$ ($n\in \mathbb{N}$) using the altitude and Euclid's theorem. At this point we will solve this task with the help of the Pythagorean theorem - again in two ways.


                \bzgled \label{konstrKorenPit}
                Construct a line segment $e\sqrt{7}$, for a given line segment $e$.
                \ezgled

\textbf{\textit{Solution.}} We will carry out the construction in two ways.


\textit{1)} (Figure \ref{sl.pod.7.6.3.pic}).
The idea is to design a right-angled triangle $ABC$ with the cathetus  $a=e\sqrt{7}$, where the other cathetus $b=n\cdot e$ and the hypotenuse $c=m\cdot e$ ($n$ and $m$ are natural or at least rational numbers).
By the Pythagorean Theorem \ref{PitagorovIzrek} we have $\left(e\sqrt{7}\right)^2+b^2=c^2$ or $m^2 e^2-n^2e^2=7e^2$ or equivalently $(m+n)(m-n)=7$. We get one solution of this equation by $m$ and $n$ if we solve the system:
\begin{eqnarray*}
&& m+n=7\\
&& m-n=1
\end{eqnarray*}
By adding and subtracting the equations from this system, we therefore get one possibility $m=4$ and $n=3$; from this we get $c=4e$ and $b=3e$. This idea allows us to construct it.

Let's first draw a right angled triangle $ABC$ with hypotenuse $AB=4e$ and cathetus $AC=3e$. According to Pythagoras' theorem \ref{PitagorovIzrek} $BC^2=\left(4e\right)^2-\left(3e\right)^2=7e^2$ or $BC=e\sqrt{7}$.

\begin{figure}[!htb]
\centering
\input{sl.pod.7.6.3.pic}
\caption{} \label{sl.pod.7.6.3.pic}
\end{figure}

\textit{2)} (Figure \ref{sl.pod.7.6.3.pic}).
Let's first construct a right angled triangle $A_0A_1A_2$ with cathetuses $A_0A_1=A_1A_2=e$, then a right angled triangle $A_0A_2A_3$ with cathetuses $A_0A_2$ and $A_2A_3=e$. We continue the process and construct a sequence of right angled triangles $A_0A_{n-1}A_n$ with cathetuses $A_0A_{n-1}$ and $A_{n-1}A_n=e$. According to Pythagoras' theorem:
 \begin{eqnarray*}
A_0A_2&=&\sqrt{A_0A_1^2+A_1A_2^2}=\sqrt{e^2+e^2}=e\sqrt{2}\\
A_0A_3&=&\sqrt{A_0A_2^2+A_2A_3^2}=\sqrt{2e^2+e^2}=e\sqrt{3}\\
&\vdots&\\
A_0A_n&=&\sqrt{A_0A_{n-1}^2+A_{n-1}A_n^2}=\sqrt{(n-1)e^2+e^2}=e\sqrt{n}
\end{eqnarray*}
We therefore get the desired distance from the sixth right angled triangle: $A_0A_7=e\sqrt{7}$.
\kdokaz

Some of the following examples relate to the use of Pythagoras' theorem in circles.



            \bzgled \label{PitagoraCofman}
            (Example from the book \cite{Cofman})
            Let $a$ and $b$ be two circles of the same
            radius $r=1$, touching each other externally at the point $P$,
             and $t$ their common external tangent.
             If $c_1$, $c_2$, $\ldots$, $c_n$,... is a sequence of circles touching
            circles $a$ and $b$, the first of them touching the line $t$ and each circle from the sequence
            touches the previous one. Calculate the diameter of the circle $c_n$.\\

            \ezgled

\begin{figure}[!htb]
\centering
\input{sl.pod.7.6.4.pic}
\caption{} \label{sl.pod.7.6.4.pic}
\end{figure}

\textbf{\textit{Proof.}}
 Let us denote: with $A$ and $B$ the centers of the circles $a$ and $b$, with $A'$ and $B'$ the points of tangency of these circles with
the tangent $t$, with $S_n$ the centers of the circles $c_n$ ($n =1,2,\ldots$), with $r_n$ their radii, with $d_n=2r_n$ their
diameters, with $P_n$ the points of tangency of the circles $c_n$ and $c_{n-1}$ (or the circles $c_1$ and the line $t$) and with $S_n'$ and $X_n$ the perpendicular projections of the points $S_n$
or $P_n$ on the line $AA'$ (Figure \ref{sl.pod.7.6.4.pic}). Let $x_n=|AX_n|$. If we use the Pythagorean Theorem \ref{PitagoraovIzrek} for the triangle
$AS_nS_n'$, we get $|AS_n|^2=|S_nS_n'|^2+|S_n'A|^2$ or
$\left(1+r_n \right)^2=1+\left(x_n-r_n \right)^2$. From this, solving for $r_n$, we get:
 \begin{eqnarray} \label{eqnPitagoraCofman1}
 r_n=\frac{x_n^2}{2\left(1+x_n \right)}.
 \end{eqnarray}
 On the other hand, it is clear that $x_n-x_{n+1}=d_n=2r_n$. If we connect this with the relation \ref{eqnPitagoraCofman1}, we get:
 \begin{eqnarray*}
 x_{n+1}=x_n-2r_n=x_n-\frac{x_n^2}{1+x_n}=\frac{x_n}{1+x_n}.
 \end{eqnarray*}
 Because in this case $x_1=1$, by direct calculation we get $x_2=\frac{1}{2}$, $x_3=\frac{1}{3}$, $\ldots$ From this we intuitively conclude that:
 \begin{eqnarray} \label{eqnPitagoraCofman2}
 x_n=\frac{1}{n}.
 \end{eqnarray}
We will prove the relation \ref{eqnPitagoraCofman2} formally - by using mathematical induction. The relation \ref{eqnPitagoraCofman2} is clearly true for $n=1$. We assume that the relation \ref{eqnPitagoraCofman2} is true for $n=k$ ($x_k=\frac{1}{k}$) and prove that from this it follows that it is true for $n=k+1$ ($x_{k+1}=\frac{1}{k+1}$):
 \begin{eqnarray*}
 x_{k+1}=\frac{x_k}{1+x_k}=\frac{\frac{1}{k}}{1+\frac{1}{k}}=\frac{1}{k+1}.
 \end{eqnarray*}
 With this we have proven that the relation \ref{eqnPitagoraCofman2} is true for every $n\in \mathbb{N}$.

At the end, from relations \ref{eqnPitagoraCofman1} and \ref{eqnPitagoraCofman2} it follows:
\begin{eqnarray*}
 d_n=2r_n=\frac{x_n^2}{1+x_n}=
 \frac{\left(\frac{1}{n}\right)^2}{1+\frac{1}{n}}=\frac{1}{n(n+1)}
 \end{eqnarray*}
or
\begin{eqnarray} \label{eqnPitagoraCofman3}
 d_n=\frac{1}{n(n+1)},
 \end{eqnarray}
 which was needed to calculate.
\kdokaz

With the help of the statement from the previous example or relation \ref{eqnPitagoraCofman3} we can come to an interesting infinite sequence. Namely, $c_1, c_2, \ldots
c_n,\ldots$ is an infinite sequence of circles and in this case it holds:
\begin{eqnarray*}
 d_1+d_2+\cdots + d_n+\cdots=|PP_1|=1.
 \end{eqnarray*}
Now, from relation \ref{eqnPitagoraCofman3} we get:
\begin{eqnarray*}
 \frac{1}{1\cdot 2}+\frac{1}{2\cdot 3}+\cdots +\frac{1}{n\cdot (n+1)}+\cdots=1,
 \end{eqnarray*}
 or
\begin{eqnarray*}
 \sum_{n=1}^{\infty}\frac{1}{n (n+1)}=1.
 \end{eqnarray*}



        \bizrek \label{Jung}
       Let $\mathcal{P}=\{A_1,A_2,\ldots A_n\}$ be a finite set of
            points in the plane and $d=\max \{|A_iA_j|;\hspace*{1mm}1\leq i,j\leq n\}$.
             Prove that there exists a circle with radius $\frac{d}{\sqrt{3}}$,
             which contains all the points of this set (\index{izrek!Jungov}Jung's theorem\footnote{Nemški matematik
        \index{Jung, H. W. E.}\textit{H. W. E. Jung} (1876--1953)
        je leta 1901 dokazal
        splošno trditev za $n$-razsežni primer.}).
        \eizrek



\begin{figure}[!htb]
\centering
\input{sl.pod.7.6.1a.pic}
\caption{} \label{sl.pod.7.6.1a.pic}
\end{figure}

\textbf{\textit{Proof.}}  (Figure \ref{sl.pod.7.6.1a.pic})

According to the statement \ref{lemaJung} it is enough to prove that for every three of these $n$ points there exists an appropriate circle. Let these points be $A$, $B$ and $C$. According to the assumption none of the sides of the triangle $ABC$ exceeds $d$. Without loss of generality let $BC$ be the longest side of this triangle. Then $|AC|, |AB| \leq |BC| \leq d$. We will consider two cases:

\textit{1)} In the case when the triangle $ABC$ is rectangular or top-rectangular, the circle $\mathcal{K}'(S, \frac{|BC|}{2})$ (where $S$ is the center of the side $BC$) contains all three points $A$, $B$ and $C$. Because $\frac{|BC|}{2}\leq \frac{d}{2}<\frac{d}{\sqrt{3}}$, points $A$, $B$ and $C$ are also in the circle $\mathcal{K}(S, \frac{d}{\sqrt{3}})$.

\textit{2)} If the triangle $ABC$ is acute, then $\angle BAC \geq 60^0$, because $BC$ is the longest side of this triangle. Let $BA'C$ be a right triangle, so that points $A$ and $A'$ are on the same side of the line $BC$, and $\mathcal{K}'(S,R)$ is a circle defined by the circumscribed circle of this triangle. Because of the condition $\angle BAC \geq 60^0$, point $A$ is either on the edge of the circle $\mathcal{K}'$ or its inner point. The radius of this circle is, according to the \ref{PitagorovEnakostr} equality, equal to $R=\frac{|BC|\sqrt{3}}{3}=\frac{|BC|}{\sqrt{3}}$. Because $|BC|\leq d$, points $A$, $B$ are also in the circle $\mathcal{K}(S, \frac{d}{\sqrt{3}})$.
\kdokaz



        \bizrek
        Let $M$ and $N$ be common points of two congruent circles
            $k$ and $l$ with a radius $r$. Points $P$ and $Q$ are the intersections of these two circles with
            a line defined by their centres such that $P$ and $Q$ are on the same side of the line $MN$. Prove that
             $$|MN|^2 + |PQ|^2 = 4r^2.$$
        \eizrek

\begin{figure}[!htb]
\centering
\input{sl.pod.7.6.1.pic}
\caption{} \label{sl.pod.7.6.1.pic}
\end{figure}

\textbf{\textit{Proof.}}
Let $O$ and $S$ be the centers of these two circles and $\overrightarrow{v}
= \overrightarrow{OS}$ (Figure \ref{sl.pod.7.6.1.pic}). The translation
$\mathcal{T}_{\overrightarrow{v}}$
maps the circle $k$ to the circle $l$ and the point $M$ to some point $M'$.
Since the point $M$ lies
on the circle
$k$, its image $M'$ lies on the circle $l$. For the same reasons,
   $\mathcal{T}_{\overrightarrow{v}}(P )=Q$. Therefore,
$\overrightarrow{MM'} = \overrightarrow{v} = \overrightarrow{OS} = \overrightarrow{PQ}$, which means
  that the lines $MM'$ and $OS$ are parallel and $MM'\cong OS\cong PQ$. Because the line
$MN$ is perpendicular to the line $OS$ ($OS$ is the perpendicular bisector of the segment $MN$),
the line $MN$ is also perpendicular to the parallel
$MM'$ of the line $OS$.
Therefore,  $\angle NMM'$ is a right angle, so $NM'$ is the diameter of the circle $l$.
 If we use the Pythagorean Theorem \ref{PitagorovIzrek}, we get:
$$|MN|^2 + |PQ|^2 = |MN|^2 + |MM'|^2 = |NM'|^2 = 4r^2,$$ which was to be proven. \kdokaz



            \bzgled
            Lines $b$ and $c$ and a point $A$ in the same plane are given. Construct
            a square $ABCD$ such that the vertices $B$ and $C$ lie on the lines $b$
            and $c$, respectively.
            \ezgled

\begin{figure}[!htb]
\centering
\input{sl.pod.7.6.2.pic}
\caption{} \label{sl.pod.7.6.2.pic}
\end{figure}

\textbf{\textit{Solution.}} (Figure \ref{sl.pod.7.6.2.pic})

We use the rotational dilation $\rho_{A,\sqrt{2},45^0}$. Since $\rho_{A,\sqrt{2},45^0}(B)=C$, we can plan the point $C$ from the condition
$C\in c\cap\rho_{A,\sqrt{2},45^0}(b)$.
\kdokaz



        \bnaloga\footnote{1. IMO, Romania - 1959, Problem 4.}
        Construct a right triangle with given hypotenuse $c$ such that the median
        drawn to the hypotenuse is the geometric mean of the two legs of the triangle.
        \enaloga

\begin{figure}[!htb]
\centering
\input{sl.pod.7.6.IMO1.pic}
\caption{} \label{sl.pod.7.6.IMO1.pic}
\end{figure}

\textbf{\textit{Solution.}}
 Let's assume that $ABC$ is a right angled triangle with hypotenuse
  $AB\cong c$ and its center of gravity is on the hypotenuse $CM=t_c$
  geometrical middle of the cathetus $AC=b$ and $BC=a$ or
  $t_c=\sqrt{a b}$ (Figure \ref{sl.pod.7.6.IMO1.pic}).

  From the \ref{TalesovIzrKroz} follows $t_c=\frac{1}{2}\cdot c$. If
  we use the Pythagorean theorem (\ref{PitagorovIzrek}), we get:
  $$(a+b)^2=a^2+b^2+2ab=c^2+2t^2_c=c^2+2\left( \frac{1}{2}\cdot c\right)^2=
  \frac{6}{4}\cdot c^2$$
  or
  $$a+b=\frac{\sqrt{6}}{2}\cdot c.$$
  The latter relation allows the construction.

  First, we describe the auxiliary construction of the distance, which has the length
  $\frac{\sqrt{6}}{2}\cdot c$. First, we draw the line $UV=c$
  and its center $W$, then the right angled triangle $TUW$
  ($TU=c$ and $\angle TUW=90^0$) and the right angled triangle
  $TWZ$ ($WZ=\frac{1}{2}\cdot c$
  and $\angle TWZ=90^0$).

  Now we will construct the triangle $ABC$. Draw the line $DB\cong
  TZ$ and  $\angle BDX=45^0$. The point $A$ will be one of
  the intersections of the circle $k(B,c)$ and the arc $DX$. In the end, draw the point
  $C$ as the intersection of the line $BD$ with the line
  $s_{DA}$ of the line $DA$.

  We prove that the triangle $ABC$ satisfies all the conditions of the task.
  First, from the construction it follows that $A\in k(B,c)$, which means that
  $AB=c$. The point $C$ by construction lies on the line
  $DA$, so $DC\cong AC$ or (\ref{enakokraki}) $\angle
  DAC\cong \angle ADC =45^0$. From the triangle $ACD$ is by \ref{VsotKotTrik}
   $\angle ACD=90^0$. Because by construction $\mathcal{B}(D,C,B)$, is
   also $\angle ACB=90^0$ and $ABC$ is a right angled triangle with
   hypotenuse $AB=c$.

We will now prove that its
   centroid $CM=t_c$
   is the geometrical mean of its cathets $AC=b$ and $BC=a$ or $t_c=\sqrt{a b}$.
   By construction, the triangles $TUW$ and $TWZ$ are right angled and
   $UW=WZ=\frac{1}{2}\cdot c$ and $TW=c$,
   so from Pythagoras' theorem it follows:
    \begin{eqnarray*}
     |TZ|^2&=&|TW|^2+|WZ|^2=\\
     &=& |TU|^2+|UW|^2+|WZ|^2=\\
     &=&
     c^2+\left(\frac{c}{2}\right)^2+\left(\frac{c}{2}\right)^2=\\
     &=& \frac{6}{4}\cdot c^2
     \end{eqnarray*}
  By construction, $BD\cong TZ$, so $|BD|^2=\frac{6}{4}\cdot
  c^2$. Therefore:
 \begin{eqnarray*}
     \frac{6}{4}\cdot c^2&=&|BD|^2=\left(|BC|+|CD|\right)^2=\\
        &=&\left(|BC|+|CA|\right)^2=(a+b)^2=\\
        &=&a^2+b^2+2ab=\\
        &=&c^2+2ab\\
 \end{eqnarray*}
 From this it follows that $\left(\frac{c}{2} \right)^2=ab$. Because according to the theorem
 \ref{TalesovIzrKroz} $t_c=\frac{c}{2}$, in the end we get
 $t_c^2=ab$.

 The number of solutions to the task
 depends on the number of intersections of the circle $k(B,c)$ and the line
 $DX$.
  \kdokaz



      \bnaloga\footnote{23. IMO, Hungary - 1982, Problem 5.}
        The diagonals $AC$ and $CE$ of the regular hexagon $ABCDEF$ are
        divided by the inner points $M$ and $N$, respectively, so that
        $$\frac{AM}{AC}=\frac{CN}{CE}=r.$$
        Determine $r$ if $B$, $M$ and $N$ are collinear.
        \enaloga


\begin{figure}[!htb]
\centering
\input{sl.skk.4.2.IMO2.pic}
\caption{} \label{sl.skk.4.2.IMO2.pic}
\end{figure}

\textbf{\textit{Solution.}} Let $O$ be the centre of the regular
hexagon $ABCDEF$ (Figure \ref{sl.skk.4.2.IMO2.pic}) and $a$ the
length of its side.
 From the assumption $\frac{AM}{AC}=\frac{CN}{CE}=r$ and from the fact $AC\cong CE$
  it follows that $CM\cong EN$ and $AM\cong CN$.
The triangles $ACB$ and $CED$ are congruent (the \textit{SSS}
theorem \ref{SSS}), therefore $\angle ACB\cong\angle CED$. This means that the triangles $BCM$ and $DEN$ are also congruent (the \textit{ASA} theorem \ref{KSK} or $\angle BMC\cong \angle DNE$.
 The triangle $ACE$ is regular, therefore $\angle ACE=60^0$. So:
 \begin{eqnarray*}
 \angle BND&=&\angle BNC+\angle CND\\
 &=&\angle BMC- \angle ECA+180^0-\angle DNE=120^0
 \end{eqnarray*}
 Because also $\angle BOD=120^0$ and $CO\cong CD\cong CB$,
 the point $N$ lies on the circle $k(C,CO)$. Therefore $AM\cong CN\cong
 CB=a$. Let $v$ be the length of the altitude of the regular triangle
 $OCD$ and use the Pythagorean theorem:
  $$r=\frac{CN}{CE}=\frac{a}{2v}=\frac{\sqrt{3}}{3},$$ which had to be calculated. \kdokaz

Let $ABC$ be an acute-angled triangle with circumcentre $O$. Let $P$ on $BC$ be the foot of the altitude from $A$.
Suppose that $\angle BCA>\angle ABC+30^0$.
Prove that $\angle CAB+\angle COP<90^0$.

      \bnaloga\footnote{42. IMO, USA - 2001, Problem 1.}
        Let $ABC$ be an acute-angled triangle with circumcentre $O$. Let $P$ on $BC$ be the foot of the altitude from $A$.
        Suppose that $\angle BCA>\angle ABC+30^0$.
        Prove that $\angle CAB+\angle COP<90^0$.
        \enaloga


\begin{figure}[!htb]
\centering
\input{sl.pod.7.6.IMO3.pic}
\caption{} \label{sl.pod.7.6.IMO3.pic}
\end{figure}

\textbf{\textit{Solution.}} Let $A_1$ be the centre of the side
$BC$, $P'$ and $A'$ the other intersections of the lines $AP$ and $AO$ with
the circumscribed circle $k(O,R)$ of the triangle $ABC$, and $Q$ the orthogonal projection of the point $A'$ on the line $BC$ (Figure
\ref{sl.pod.7.6.IMO3.pic}). Let $\alpha$, $\beta$ and
$\gamma$ be the internal angles of the triangle $ABC$ at the vertices $A$, $B$ and
$C$.

Since $AA'$ is the diameter of the circle $k$, $\angle AP'A'=90^0$ (by the statement
\ref{TalesovIzrKroz}). This means that $PP'A'Q$ is a rectangle and
it holds:
 \begin{eqnarray}
PQ\cong A'P'. \label{pod.7.6.IMO31}
 \end{eqnarray}
 The point $O$ is the center of the diameter $AA'$, $A_1$ is by Tales' statement
\ref{TalesovIzrek} the center of the line $PQ$. The point $A_1$ is
thus the common center of the lines $BC$ and $PQ$.

By the statement \ref{TockaNbetagama} it is also $\angle A'AP=\angle
OAP=\gamma-\beta$. From the given condition $\angle BCA>\angle ABC+30^0$
or $\gamma-\beta>30^0$ it thus follows:
 \begin{eqnarray}
\angle A'AP>30^0. \label{pod.7.6.IMO32}
\end{eqnarray}
By the statement \ref{SredObodKot} it is $\angle BOC=2\alpha$. Since $BOC$
is an isosceles triangle with the base $BC$, $\angle
OCB=\frac{1}{2}\left( 180^0-2\alpha\right)=90^0-\alpha$ (by the statement
\ref{enakokraki} and \ref{VsotKotTrik}). The condition $\angle CAB+\angle
COP<90^0$ or $\angle COP<90^0-\alpha$, which we want to prove, is
thus equivalent to the condition $\angle COP<\angle OCB=\angle OCP$
(it holds $\angle OCB=\angle OCP$, because $ABC$ is an acute-angled triangle and $\mathcal{B}(B,P,C)$ is
obtuse-angled). The latter is by the statement
\ref{vecstrveckot} equivalent to the condition $CP<OP$. It is thus enough to prove:
 \begin{eqnarray}
CP<OP. \label{pod.7.6.IMO3a}
 \end{eqnarray}
If we use Pythagoras' statement \ref{PitagorovIzrek} for the right-angled
triangle $OA_1P$, we get:
$|OP|^2=|OA_1|^2+|PA_1|^2=|OA_1|^2+\frac{|PQ|^2}{4}$. On the other hand,

$|CP|^2=\frac{\left(|BC|-|PQ|\right)^2}{4}$. Therefore the inequality
\ref{pod.7.6.IMO3a}, which we want to prove, is equivalent to
$\frac{|BC|^2}{4}-\frac{|BC|\cdot|PQ|}{2}<|OA_1|^2$ or:
\begin{eqnarray}
4|OA_1|^2+2|BC|\cdot|PQ|>|BC|^2. \label{pod.7.6.IMO3b}
 \end{eqnarray}

From inequality \ref{pod.7.6.IMO32} and from \ref{SredObodKot} it follows that $\angle A'OP'>60^0$, so in the equilateral triangle $OA'P'$ $\angle A'OP'$ is the largest angle (\ref{enakokraki} and \ref{VsotKotTrik}). This means that $A'P'>OA'=R$ is true. From \ref{pod.7.6.IMO31} it follows that $PQ\cong A'P'>R$. In the end:
\begin{eqnarray*}
\hspace*{-1.5mm} 4|OA_1|^2+2|BC|\cdot|PQ|>2|BC|\cdot|PQ|>2|BC|\cdot R=|BC|\cdot
2R>|BC|^2,
 \end{eqnarray*}
which had to be proven.
 \kdokaz



%_______________________________________________________________________________
 \poglavje{Napoleon Triangles} \label{odd7Napoleon}

We will first define two triangles, which we will adapt to any triangle $ABC$.
We will construct equilateral triangles on the outside of its sides. The centers of these triangles are the vertices of the so-called \index{trikotnik!Napoleonov zunanji}\pojem{outer Napoleon triangle} (Figure \ref{sl.pod.7.7n.1.pic}). If we construct equilateral triangles on the inside in the same way, we get the so-called \index{trikotnik!Napoleonov notranji}\pojem{inner Napoleon triangle}.


\begin{figure}[!htb]
\centering
\input{sl.pod.7.7n.1.pic}
\caption{} \label{sl.pod.7.7n.1.pic}
\end{figure}



               \bizrek
                The outer and inner Napoleon triangles are equilateral.
                \eizrek


\begin{figure}[!htb]
\centering
\input{sl.pod.7.7n.2.pic}
\caption{} \label{sl.pod.7.7n.2.pic}
\end{figure}

\textbf{\textit{Proof.}}
Let $P$,$ Q$ and $R$ be the centers of the regular triangles, which are constructed on the outside of the sides of any triangle $ABC$, so $PQR$ is the outer Napoleon triangle of triangle $ABC$ (Figure \ref{sl.pod.7.7n.2.pic}). We will prove that $PQR$ is an equilateral triangle.

Let:
$$\mathcal{I}=\mathcal{R}_{R,120^0}\circ
\mathcal{R}_{P,120^0}\circ\mathcal{R}_{Q,120^0}.$$
Since $120^0+120^0+120^0=360^0$, the composite $\mathcal{I}$ is either the identity or a translation (statement \ref{rotacKomp2rotac}). However, the latter option is ruled out, since $\mathcal{I}(A)=A$. Therefore $\mathcal{I}=\mathcal{E}$ or:
 $$\mathcal{R}_{R,120^0}\circ
\mathcal{R}_{P,120^0}\circ\mathcal{R}_{Q,120^0}=\mathcal{E}.$$
It follows that:
 $$\mathcal{R}_{R,120^0}\circ\mathcal{R}_{P,120^0}=
 \mathcal{R}^{-1}_{Q,120^0}=\mathcal{R}_{Q,-120^0}=\mathcal{R}_{Q,240^0}.$$
 Let $Q'$ be the third vertex of the equilateral triangle $PQ'R$. By statement \ref{rotacKomp2rotac}, we have:
 $$\mathcal{R}_{R,120^0}\circ\mathcal{R}_{P,120^0}=\mathcal{R}_{Q',240^0}.$$
 Therefore $\mathcal{R}_{Q,240^0}=\mathcal{R}_{Q',240^0}$ or $Q=Q'$, which means that $PQR$ is an equilateral triangle.
 In a similar way, we prove (by using rotations for an angle of $-120^0$) that the inner Napoleon triangle is also equilateral.
        \kdokaz


We shall now prove an additional property of the two Napoleon triangles.



                \bizrek
                The outer and inner Napoleon triangles of an arbitrary
                 triangle $ABC$ have the same centre, which is at the same time the centroid of the triangle $ABC$.
                \eizrek


\begin{figure}[!htb]
\centering
\input{sl.pod.7.7n.3.pic}
\caption{} \label{sl.pod.7.7n.3.pic}
\end{figure}

\textbf{\textit{Proof.}}
Let $PQR$ and $P'Q'R'$ be the outer and inner Napoleon triangles of the triangle $ABC$ (Figure \ref{sl.pod.7.7n.3.pic}). We denote by $Y$ and $A_1$ the centres of the lines $PR$ and $BC$. We shall first prove:

\begin{enumerate}
  \item The triangles $QP'C$ and $RBP'$ are congruent and similar to the triangle $ABC$.
  \item The quadrilaterals $ARP'Q$ and $CP'RQ'$ are parallelograms.
  \item $\overrightarrow{YA_1}=\frac{1}{2}\overrightarrow{AQ}$.
\end{enumerate}

\textit{1}) First, from $\angle BCP'\cong \angle ACQ=30^0$ it follows that $\angle P'CQ\cong\angle BCA$ (with the rotation $\mathcal{R}_{C.30^0}$, the angle $\angle P'CQ$ is mapped into the angle $\angle BCA$). It is also true that $\frac{CQ}{CA}=\frac{CP'}{CB}=\frac{\sqrt{3}}{3}$ (by the Pythagorean equality \ref{PitagorovEnakostr}, because $CQ$ and $CP'$ are the radii of the circumscribed circles of two isosceles triangles with sides $CA$ and $CB$), so the triangles $ABC$ and $QP'C$ are similar (by the similarity theorem \ref{PodTrikSKS}) with the similarity coefficient $k=\frac{\sqrt{3}}{3}$.
 Similarly, the triangles $ABC$ and $RBP'$ are also similar with the same coefficient of similarity, which means that the triangles $QP'C$ and $RBP'$ are congruent.

\textit{2}) From the congruence of the triangles $QP'C$ and $RBP'$ and from the fact that the triangles $RAB$ and $QAC$ are of the same size, it follows: $QP'\cong RB\cong RA$ and $P'R\cong CQ\cong QA$, so the quadrilateral $ARP'Q$ is a parallelogram. Similarly, we prove that the quadrilateral $CP'RQ'$ is a parallelogram.


\textit{3}) The line $YA_1$ is the median of the triangle $RPP'$ with the base $RP'$, so by the theorems \ref{srednjicaTrikVekt} and \ref{vektParalelogram}:

$$\overrightarrow{YA_1}=\frac{1}{2}\overrightarrow{RP'}=\frac{1}{2}\overrightarrow{AQ}$$

We only need to prove the initial claim. Because $\overrightarrow{YA_1}=\frac{1}{2}\overrightarrow{AQ}$,
 the centroid $AA_1$ of the triangle $ABC$ and
the centroid $QY$ of the triangle $PQR$ intersect in a point which they divide in the ratio $2:1$. So this is the common centroid of the triangles $ABC$ and $PQR$ (by the theorem \ref{tezisce}). Similarly, it can be proven that the triangles
$ABC$ and $P'Q'R'$ also have a common centroid.
 \kdokaz


%________________________________________________________________________________
 \poglavje{Ptolemy's Theorem} \label{odd7Ptolomej}

We will now prove a very important theorem which is related to the cord quadrilaterals.

\bizrek \label{izrekPtolomej}
                \index{izrek!Ptolomejev}(Ptolemy’s\footnote{\index{Ptolomej Aleksandrijski}\textit{Ptolomej Aleksandrijski} (2. st.) je dokazal ta izrek v svojemu delu ‘‘Veliki zbornik’’.} theorem)
                    If $ABCD$ is a cyclic quadrilateral then
                     the product of the lengths of its diagonals
                     is equal to the sum of the products of the lengths of the pairs of opposite sides,
                      i.e.
                $$|AC|\cdot |BD|=|AB|\cdot |CD|+|BC|\cdot |AD|.$$

                \eizrek

\begin{figure}[!htb]
\centering
\input{sl.pod.7.7.1.pic}
\caption{} \label{sl.pod.7.7.1.pic}
\end{figure}

\textbf{\textit{Proof.}}
Brez škode za splošnost predpostavimo, da je $\angle CBD\leq\angle ABD$.
Naj bo $L$ takšna točka diagonale $AC$, da velja
$\angle ABL\cong \angle CBD$ (Figure \ref{sl.pod.7.7.2.pic}). Ker je še $\angle BAL=\angle BAC\cong\angle BDC$ (izrek \ref{ObodObodKot}), sta trikotnika
$ABL$ in $DBC$ podobna (izrek \ref{PodTrikKKK}), zato je $AB:DB=AL:DC$ oziroma:
\begin{eqnarray} \label{eqnPtolomej1}
|AB|\cdot |CD|=|AL|\cdot |BD|.
\end{eqnarray}
Ker je $\angle BCL=\angle BCA\cong\angle BDA$ (izrek \ref{ObodObodKot}) in $\angle LBC\cong\angle ABD$ (iz predpostavke $\angle CBD\cong\angle ABL$), velja
tudi $\triangle BCL\sim \triangle BDA$ (izrek \ref{PodTrikKKK}), zato je $BC:BD=CL:DA$ oziroma:
\begin{eqnarray} \label{eqnPtolomej2}
|BC|\cdot |AD|=|CL|\cdot |BD|.
\end{eqnarray}
 Po seštevanju relacij \ref{eqnPtolomej1} in \ref{eqnPtolomej2} dobimo iskano enakost.
\kdokaz


Trditev iz zgleda \ref{zgledTrikABCocrkrozP} bomo na tem mestu dokazali na bolj enostaven način - z uporabo Ptolomejevega izreka.

\bzgled \label{zgledTrikABCocrkrozPPtol}
Let  $k$ be the circumcircle of a regular triangle $ABC$.
 If $P$ is an arbitrary point lying
on the shorter arc $BC$ of the circle $k$, then
 $$|PA|=|PB|+|PC|.$$
\ezgled

\begin{figure}[!htb]
\centering
\input{sl.pod.7.7.2a.pic}
\caption{} \label{sl.pod.7.7.2a.pic}
\end{figure}

\textbf{\textit{Solution.}}
The quadrilateral
$ABPC$ is a cyclic quadrilateral (Figure \ref{sl.pod.7.7.2a.pic}), therefore from Ptolomey's theorem \ref{izrekPtolomej} it follows that
$|PA|\cdot |BC|=|PB|\cdot |AC|+|PC|\cdot |AB|$. Because $ABC$ is an equilateral triangle, i.e. $|BC|=|AC|= |AB|$, from the previous relation it follows that $|PA|=|PB|+|PC|$.
 \kdokaz

 Another application of Ptolomey's theorem relates to an isosceles trapezium.




                \bzgled
                Let $a$ and $c$ be the bases, $b$ the leg and $d$ the diagonal
                isosceles trapezium. Prove that
                $$ac=d^2-b^2.$$
                \ezgled

\begin{figure}[!htb]
\centering
\input{sl.pod.7.7.4.pic}
\caption{} \label{sl.pod.7.7.4.pic}
\end{figure}

\textbf{\textit{Solution.}} (Figure \ref{sl.pod.7.7.4.pic})

By theorem \ref{trapezTetivEnakokr} the isosceles trapezium is cyclic, therefore we can use Ptolomey's theorem \ref{izrekPtolomej} and obtain $d^2=ac+b^2$ or $ac=d^2-b^2$.
\kdokaz

It is interesting that if we take the special case $a=c$ in the previous theorem (Figure \ref{sl.pod.7.7.4.pic}), when the isosceles trapezium is a rectangle, we obtain the formula for the diagonal of a rectangle $d^2=a^2+b^2$, which is otherwise a direct consequence of Pythagoras' theorem \ref{PitagorovIzrek}. In this sense, we can also say that Pythagoras' theorem is a consequence of Ptolomey's theorem, if we complete the right-angled triangle to a rectangle.

\begin{figure}[!htb]
\centering
\input{sl.pod.7.7.5.pic}
\caption{} \label{sl.pod.7.7.5.pic}
\end{figure}

Of course, the use of Ptolemy's theorem in a square leads to the well-known relationship (theorem \ref{PitagorovKvadrat}) between the diagonal $d$ and the side $a$ of the square: $d^2=2a^2$  or $d=a\sqrt{2}$ (Figure \ref{sl.pod.7.7.5.pic}).

We know that for every regular $n$-gon there is a circumscribed circle \ref{sredOcrtaneKrozVeck}. This allows us to use Ptolemy's theorem, if in a regular $n$-gon (in the case $n>4$) we choose the appropriate four vertices.
Now we will use this idea in the cases $n=5$ and $n=7$.



                    \bzgled \label{PtolomejPetkotnik}
                    Let $a$ be the side and $d$  the diagonal of a regular pentagon ($5$-gon).
                    Prove that
                    $$d=\frac{1+\sqrt{5}}{2}a.$$
                    \ezgled

\begin{figure}[!htb]
\centering
\input{sl.pod.7.7.3.pic}
\caption{} \label{sl.pod.7.7.3.pic}
\end{figure}

\textbf{\textit{Solution.}}
 Let $A_1A_2A_3A_4A_5$ be a regular pentagon with side $a$ and diagonal $d$
 (Figure \ref{sl.pod.7.7.3.pic}). If we use Ptolemy's theorem \ref{izrekPtolomej} for the quadrilateral $A_1A_2A_3A_5$, we get: $d^2=ad+a^2$ or $d^2-ad-a^2=0$. The positive solution of this quadratic equation in $d$ is precisely $d=\frac{1+\sqrt{5}}{2}a$.
 \kdokaz

The previous example allows us to construct a regular pentagon with a side that is congruent to the line $a$.
Indeed, if $ABCDE$ is the desired regular pentagon, we can first
plan the triangle $ABC$, where $AB\cong BC\cong a$ and $AC =\frac{1+\sqrt{5}}{2}a$. To construct $\frac{1+\sqrt{5}}{2}a$, we use the Pythagorean Theorem (\ref{PitagorovIzrek}) for a right triangle with sides $a$ and $2a$ - its hypotenuse measures $a\sqrt{5}$. We add this hypotenuse to the line $a$ and divide the resulting line into two lines. The center of the drawn circle of the regular pentagon $ABCDE$ is also the center of the drawn circle of the triangle $ABC$  (Figure \ref{sl.pod.7.7.3a.pic}).


\begin{figure}[!htb]
\centering
\input{sl.pod.7.7.3a.pic}
\caption{} \label{sl.pod.7.7.3a.pic}
\end{figure}


In a similar way, we can construct a regular pentagon that is inscribed in a given circle. We do this by first constructing a regular pentagon with any side, and then using the central extension (Figure \ref{sl.pod.7.7.3a.pic}).

We will talk more about the construction of regular $n$-gons in section \ref{odd9LeSestilo}.



                \bzgled
                Let $a$ be the side, $d$ the shorter and $D$ the longer diagonal
                of a regular heptagon ($7$-gon). Prove that
                    $$\frac{1}{a}=\frac{1}{D}+\frac{1}{d}$$
                \ezgled

\begin{figure}[!htb]
\centering
\input{sl.pod.7.7.2.pic}
\caption{} \label{sl.pod.7.7.2.pic}
\end{figure}

\textbf{\textit{Solution.}}

Let $A_1A_2A_3A_4A_5A_6A_7$ be a regular heptagon and $k$ the drawn
circle  (Figure \ref{sl.pod.7.7.2.pic}). The quadrilateral
$A_1A_2A_3A_5$ is a tetragon, so from Ptolomey's theorem \ref{izrekPtolomej} it follows
that $ad+aD=dD$ or $\frac{1}{a}=\frac{1}{D}+\frac{1}{d}$.
 \kdokaz

If we use the Ptolomey Theorem multiple times, we get one generalization of task \ref{zgledTrikABCocrkrozPPtol}:

\bzgled
            Let $P$ be an arbitrary point of the shorter arc $A_1A_{2n+1}$
                of the circumcircle of a regular polygon $A_1A_2\cdots A_{2n+1}$.
                If we denote $d_i=PA_i$ ($i\in \{1,2,\cdots , 2n+1 \}$),
                prove that (Figure \ref{sl.pod.7.7.6.pic})
            $$d_1+d_3+\cdots +d_{2n+1}=d_2+d_4+\cdots +d_{2n}.$$
            \ezgled


\begin{figure}[!htb]
\centering
\input{sl.pod.7.7.6.pic}
\caption{} \label{sl.pod.7.7.6.pic}
\end{figure}

%________________________________________________________________________________
 \poglavje{Stewart's Theorem} \label{odd7Stewart}

The next theorem refers to a very interesting and important metric property of a triangle.



            \bizrek \label{StewartIzrek}
            \index{izrek!Stewartov} (Stewart's\footnote{\index{Stewart, M.}\textit{M. Stewart} (1717--1785),
            English mathematician, who in 1746 proved and published this theorem. His teacher - English mathematician \index{Simson, R.}\textit{R. Simson} (1687--1768) - introduced him to the theorem, which he published only in 1749. It is assumed
            that the theorem was known to \index{Arhimed} \textit{Archimedes of Syracuse} (3rd century BC) already.} theorem)
            If $X$ is an arbitrary point of the side
            $BC$ of a triangle $ABC$, then
            $$|AX|^2=\frac{|BX|}{|BC|}|AC|^2+\frac{|CX|}{|BC|}|AB|^2-|BX|\cdot |CX|.$$
            \eizrek


\begin{figure}[!htb]
\centering
\input{sl.pod.7.8.1.pic}
\caption{} \label{sl.pod.7.8.1.pic}
\end{figure}

\textbf{\textit{Proof.}}
Let $a$, $b$ and $c$ denote the lengths of the sides $BC$, $AC$ and $AB$, and $p$, $q$ and $x$ the lengths of the lines $BX$, $CX$ and $AX$, in order. With $A'$ we denote the
intersection point of the altitude $v_a$ from the vertex $A$ of the triangle $ABC$ (Figure \ref{sl.pod.7.8.1.pic}). We assume that $\mathcal{B}(B,A',C)$.

We first prove the claim for the case when $X=A'$. From Pythagoras' theorem \ref{PitagorovIzrek} it follows that $v_a^2 =b^2- q^2$
and $v_a^2 =c^2- p^2$. By multiplying the first equality by $p$ and the second by $q$ and adding the resulting relations and taking into account that $p+q=a$, we obtain $av_a^2=pb^2+qc^2-apq$ or:
$$v_a^2=\frac{p}{a}b^2+\frac{q}{a}c^2-pq,$$
which means that in the case of altitudes the claim is true.

Let now $X\neq A'$. The line $v_a$ is the altitude of the triangles $ABX$ and $ABC$ from the vertex $A$. Without loss of generality, we assume that $\mathcal{B}(B,A',X)$. If we denote the length of the line $BA'$ with $y$ and use the already proven part of the claim for altitudes,
we obtain:
\begin{eqnarray*}
v_a^2&=&\frac{y}{p}x^2+\frac{p-y}{p}c^2-y(p-y),\\
v_a^2&=&\frac{y}{a}b^2+\frac{a-y}{a}c^2-y(a-y).
\end{eqnarray*}
If we equate the right sides of these two equalities and simplify, we obtain:
$$x^2=\frac{p}{a}b^2+\frac{q}{a}c^2-pq,$$
which means that the claim is also true in the case when $X\neq A'$.

The proof of the altitude $v_a$ is similar also in the case when $\mathcal{B}(B,A',C)$ is not true, only then we obtain $v_a^2=\frac{p}{a}b^2+\frac{q}{a}c^2+pq$.
\kdokaz


Stewart's theorem \ref{StewartIzrek} can also be written in another form:

\bizrek \label{StewartIzrek2}
Let $a=|BC|$, $b=|AC|$ and $c=|AB|$ be the length of
the sides of a triangle $ABC$. If $X$ is the point that divides the side $BC$ of this triangle
in the ratio $n:m$ and $x=|AX|$ then
$$x^2=\frac{n}{n+m}b^2+\frac{m}{n+m}c^2-\frac{mn}{(m+n)^2}bc.$$
\eizrek

The most well-known use of Stewart's theorem is for the centroids of a triangle.



                \bizrek \label{StwartTezisc}
                If $a$, $b$ and $c$ are the sides and $t_a$ the triangle median
                on the side $a$, then
                $$t_a^2=\frac{1}{2}b^2+\frac{1}{2}c^2-\frac{1}{4}a^2.$$
                \eizrek



\begin{figure}[!htb]
\centering
\input{sl.pod.7.8.2.pic}
\caption{} \label{sl.pod.7.8.2.pic}
\end{figure}


\textbf{\textit{Proof.}}  The theorem is a direct consequence of Stewart's theorem \ref{StewartIzrek2}, since in the case of the centroid $n=m=1$ (Figure \ref{sl.pod.7.8.2.pic}).
\kdokaz


A direct consequence is the following theorem.



                \bzgled \label{StwartTezisc2}
              If $a$, $b$ and $c$ are the sides and $t_a$,  $t_b$ and $t_c$ the triangle medians
                on those sides, respectively, then
                $$t_a^2+t_b^2+t_c^2=\frac{3}{4}\left(a^2+ b^2+c^2\right).$$
                \ezgled


\begin{figure}[!htb]
\centering
\input{sl.pod.7.8.3.pic}
\caption{} \label{sl.pod.7.8.3.pic}
\end{figure}


\textbf{\textit{Proof.}}  (Figure \ref{sl.pod.7.8.3.pic})

If we use the proven relation from the example \ref{StwartTezisc} three times, we get:
 \begin{eqnarray*}
 t_a^2&=&\frac{1}{2}b^2+\frac{1}{2}c^2-\frac{1}{4}a^2\\
 t_b^2&=&\frac{1}{2}a^2+\frac{1}{2}c^2-\frac{1}{4}b^2\\
 t_c^2&=&\frac{1}{2}a^2+\frac{1}{2}b^2-\frac{1}{4}c^2
 \end{eqnarray*}
 By adding all three equations we get the desired relation.
\kdokaz

In the case of an isosceles triangle, the equality on the right side of Stewart's theorem is simplified.



                \bzgled \label{StewartEnakokraki}
                If $ABC$ is an isosceles triangle with the base $BC$ and
                $X$  an arbitrary point of this base, then
            $$|AX|^2=|AB|^2-|BX|\cdot |CX|.$$
                \ezgled

\begin{figure}[!htb]
\centering
\input{sl.pod.7.8.8.pic}
\caption{} \label{sl.pod.7.8.8.pic}
\end{figure}

\textbf{\textit{Proof.}}  (Figure \ref{sl.pod.7.8.8.pic})

 Therefore, if $AB\cong AC$, directly from Stewart's theorem (\ref{StewartIzrek}) it follows:
\begin{eqnarray*}
|AX|^2 &=& \frac{|BX|}{|BC|}|AC|^2+\frac{|CX|}{|BC|}|AB|^2-|BX|\cdot |CX|=\\
&=&\left(\frac{|BX|}{|BC|}+\frac{|CX|}{|BC|}\right)|AB|^2-|BX|\cdot |CX|=\\
&=&\frac{|BX|+|CX|}{|BC|}|AB|^2-|BX|\cdot |CX|=\\
&=&|AB|^2-|BX|\cdot |CX|,
\end{eqnarray*}
 which had to be proven. \kdokaz


We will continue a little bit more using Stewart's theorem.



                \bzgled
                Let $E$ be the intersection of the side $BC$ with the bisector of the interior angle
                 $BAC$ of a triangle $ABC$. If we denote
                 $a=|BC|$, $b=|AC|$, $c=|AB|$, $l_a=|AE|$ and $s=\frac{a+b+c}{2}$, then
                $$l_a=\frac{2\sqrt{bc}}{b+c}\sqrt{s(s-a)}.$$
                \ezgled

\begin{figure}[!htb]
\centering
\input{sl.pod.7.8.4.pic}
\caption{} \label{sl.pod.7.8.4.pic}
\end{figure}


\textbf{\textit{Proof.}}  (Figure \ref{sl.pod.7.8.4.pic})

 According to the statement \ref{HarmCetSimKota} it is: $BE:CE=c:b$. If we use Stewart's statement \ref{StewartIzrek2} for the triangle $ABC$ and the distance $AE$, we get:
$$l_a^2=\frac{c}{b+c}b^2+\frac{b}{b+c}c^2-\frac{bc}{(b+c)^2}a^2.$$
After a simple rearrangement and simplification of the expression on the right side of the equality, we get the desired relation.
\kdokaz


                \bizrek \label{izrekEulerStirik}\index{izrek!Eulerjev za štirikotnike}
                (Euler's\footnote{Švicarski matematik \index{Euler, L.}\textit{L. Euler} (1707--1783).} theorem for quadrilaterals)
                If $P$ and $Q$ are the midpoints of
                the diagonals $e$ and $f$ of an arbitrary quadrilateral with sides $a$, $b$, $c$ and $d$, then
                $$|PQ|^2=\frac{1}{4}\left(a^2+b^2+c^2+d^2-e^2-f^2 \right).$$
                \eizrek

\begin{figure}[!htb]
\centering
\input{sl.pod.7.8.5.pic}
\caption{} \label{sl.pod.7.8.5.pic}
\end{figure}

\textbf{\textit{Proof.}}
Let $P$ and $Q$ be the centers of the diagonals $AC$ and $BD$ of an arbitrary quadrilateral $ABCD$ (Figure \ref{sl.pod.7.8.5.pic}).
We also mark $a=|AB|$, $b=|BC|$, $c=|CD|$, $d=|DA|$, $e=|AC|$ and $f=|BD|$. The distance $PQ$ is the median of the triangle $AQC$, so according to the statement \ref{StwartTezisc} it holds:
 \begin{eqnarray} \label{eqnEulStirik}
  |PQ|^2=\frac{1}{2}|AQ|^2+\frac{1}{2}|CQ|^2-\frac{1}{4}e^2.
 \end{eqnarray}
 Similarly, the distances $QA$ and $QC$ are the medians of the triangles $ABD$ and
$CBD$, so (from the statement \ref{StwartTezisc}):
 \begin{eqnarray*}
  |AQ|^2&=&\frac{1}{2}d^2+\frac{1}{2}a^2-\frac{1}{4}f^2,\\
  |CQ|^2&=&\frac{1}{2}c^2+\frac{1}{2}b^2-\frac{1}{4}f^2.
 \end{eqnarray*}
 If we insert the last two equalities into \ref{eqnEulStirik}, we get the desired relation.
 \kdokaz

 A direct consequence of the previous statement is obtained if we choose a parallelogram for the quadrilateral.

                \bizrek
                A quadrilateral is a parallelogram if and only if the sum of
                the squares of all of its sides is equal to the sum of the squares of its diagonals.
                \eizrek



\begin{figure}[!htb]
\centering
\input{sl.pod.7.8.6.pic}
\caption{} \label{sl.pod.7.8.6.pic}
\end{figure}


\textbf{\textit{Proof.}}
Let $P$ and $Q$ be the centers of the diagonals $AC$ and $BD$ of an arbitrary quadrilateral $ABCD$. We also mark $a=|AB|$, $b=|BC|$, $c=|CD|$, $d=|DA|$, $e=|AC|$ and $f=|BD|$. According to the statement \ref{paralelogram} the quadrilateral $ABCD$ is a parallelogram exactly when $P=Q$ (Figure \ref{sl.pod.7.8.6.pic}) or $|PQ|=0$. The latter, according to the previous statement \ref{izrekEulerStirik}, is true exactly when
$\frac{1}{4}\left(a^2+b^2+c^2+d^2-e^2-f^2 \right)=0$ or $a^2+b^2+c^2+d^2=e^2+f^2$.
\kdokaz

\bizrek \label{GMTmnl}
                Let $A$ and $B$ be given points in the plane and
                $m, n, l\in R^+\setminus \{0\}$.
                 Determine a set of all points of this plane such that
                $$m|AX|^2 + n|BX|^2 = l^2.$$
                \eizrek


\begin{figure}[!htb]
\centering
\input{sl.pod.7.8.7.pic}
\caption{} \label{sl.pod.7.8.7.pic}
\end{figure}


\textbf{\textit{Proof.}} (Figure \ref{sl.pod.7.8.7.pic})

 According to the statement from Example \ref{izrekEnaDelitevDaljice} there exists only one such point $S$ on the line $AB$, that satisfies
$\overrightarrow{AS}:\overrightarrow{SB}=n:m$. Let $X$ be an arbitrary point. If we
use Stewart's Theorem \ref{StewartIzrek2} for the triangle $AXB$ and the line $XS$, we get:
\begin{eqnarray*}
|XS|^2=\frac{m}{n+m}|AX|^2+\frac{n}{n+m}|BX|^2-\frac{nm}{(n+m)^2}|AB|^2, \textrm{ i.e.}
\end{eqnarray*}
 \begin{eqnarray} \label{eqnStewMnozTock}
|XS|^2=\frac{1}{n+m}\left(m|AX|^2+n|BX|^2\right)-\frac{nm}{(n+m)^2}|AB|^2.
\end{eqnarray}
The point $X$ lies on the desired set of points exactly when $m|AX|^2 + n|BX|^2 = l^2$. Because of \ref{eqnStewMnozTock} this is exactly when:
 \begin{eqnarray*}
|XS|^2=\frac{1}{n+m}l^2-\frac{nm}{(n+m)^2}|AB|^2=c,
\end{eqnarray*}
where $c$ is a constant, that is not dependent on the point $X$. So if $c>0$,  the desired
set of points is a circle $k(S, c)$. If $c=0$, the set of points is only $\{S\}$, but if $c<0$, the set
of points is an empty set.
\kdokaz

\bnaloga\footnote{50. IMO, Germany - 2009, Problem 2.}
                 Let $ABC$ be a triangle with circumcentre $O$. The points $P$ and $Q$ are interior points
                of the sides $CA$ and $AB$, respectively. Let $K$, $L$ and $M$ be the midpoints of the segments $BP$, $CQ$
                and $PQ$, respectively, and let $l$ be the circle passing through $K$, $L$ and $M$. Suppose that the line
                $PQ$ is tangent to the circle $l$. Prove that $|OP|=|OQ|$.
                \enaloga

\begin{figure}[!htb]
\centering
\input{sl.pod.7.8.IMO2.pic}
\caption{} \label{sl.pod.7.8.IMO2.pic}
\end{figure}

\textbf{\textit{Solution.}} We denote with $k(O,R)$ the circumscribed
circle of the triangle $ABC$ (Figure \ref{sl.pod.7.8.IMO2.pic}).
 The lines $MK$ and $ML$ are the medians of the triangles $QPB$ and $PQC$,
 so (from the statement \ref{srednjicaTrikVekt}):
  \begin{eqnarray} \label{eqn72}
  \overrightarrow{MK}=\frac{1}{2}\overrightarrow{QB} \hspace*{3mm}
  \textrm{and} \hspace*{3mm} \overrightarrow{ML}=\frac{1}{2}\overrightarrow{PC}
  \end{eqnarray}
 From  the statements \ref{KotiTransverzala} and \ref{KotaVzporKraki} it follows:
 \begin{eqnarray} \label{eqn73}
 \angle AQP \cong \angle QMK,\hspace*{3mm}
 \angle APQ \cong \angle PML\hspace*{3mm} \textrm{and} \hspace*{3mm}
 \angle BAC \cong  \angle KML
  \end{eqnarray}
 Because, by assumption, the line $PQ$ is tangent to the circle $l$, by
  the statement \ref{ObodKotTang}:
  $$\angle MLK\cong\angle QMK \hspace*{3mm} \textrm{and} \hspace*{3mm}
   \angle MKL\cong\angle PML.$$
 From this and \ref{eqn73} it follows:
 \begin{eqnarray*}
 \angle AQP \cong \angle MLK,\hspace*{3mm}
 \angle APQ \cong \angle MKL\hspace*{3mm} \textrm{and} \hspace*{3mm}
  \angle BAC \cong\angle KML,
  \end{eqnarray*}
 which means that the triangles $AQP$ and $MLK$ are similar (it is enough to prove
  the congruence of two pairs of corresponding angles, by
  the statement \ref{PodTrikKKK}). So, from the definition of similarity
  of figures it follows:
  $\frac{AQ}{ML}=\frac{AP}{MK}$. If we combine this relation with the relation
    \ref{eqn72}, we get $\frac{AQ}{AP}=\frac{ML}{MK}=
    \frac{\frac{1}{2}\cdot CP}{\frac{1}{2}\cdot
    BQ}=\frac{CP}{BQ}$. So it holds:
 \begin{eqnarray} \label{eqn74}
 |AQ|\cdot |BQ| = |AP|\cdot |CP|
  \end{eqnarray}
 From Stewart's statement \ref{StewartIzrek} for the isosceles triangle $AOB$
  ($|OA|=|OB|=R$) it follows:
 $$|OQ|^2=|OA|^2\cdot \frac{QB}{AB}+|OB|^2\cdot \frac{QA}{AB}
 -|AQ|\cdot |BQ|=R^2-|AQ|\cdot |BQ|.$$
 Similarly, from the triangle $AOC$ by the same statement we get:
  $$|OP|^2=R^2-|AP|\cdot |CP|.$$
From the proven relation \ref{eqn74} at the end it follows $|OQ|^2=|OP|^2$
or  $|OQ|=|OP|$.
 \kdokaz

%________________________________________________________________________________
\poglavje{Desargues' Theorem} \label{odd7Desargues}

The next theorem is historically connected to the development of projective geometry.



             \bizrek \label{izrekDesarguesEvkl} \index{izrek!Desarguesov}
            (Desargues’\footnote{
             \index{Desargues, G.} \textit{G. Desargues} (1591--1661), French architect, who was one of
             the founders of projective geometry.} theorem)
            Let $ABC$ and $A'B'C'$ be two triangles in the plane such that the lines $AA'$, $BB'$ and $CC'$ intersect
                at a point $S$
                 (i.e the triangles are \index{perspective triangles}\pojem{perspective with respect to the centre}
                  \color{blue}  $S$).
                   If $P=BC\cap B'C'$, $Q=AC\cap A'C'$ and $R=AB\cap A'B'$,
                 then the points $P$, $Q$ and
                $R$ are collinear
                 (i.e. triangles are \pojem{perspective with respect to the axis} \color{blue}  $PQ$).
             \eizrek


\begin{figure}[!htb]
\centering
\input{sl.pod.7.10D.1.pic}
\caption{} \label{sl.pod.7.10D.1.pic}
\end{figure}


\textit{\textbf{Proof.}} (Figure
\ref{sl.pod.7.10D.1.pic})

If we use Menelaus' theorem \ref{izrekMenelaj} for
the triangles $SA'B'$, $SA'C'$ and $SB'C'$, we get:
 \begin{eqnarray*}
\hspace*{-2mm} \frac{\overrightarrow{SA}}{\overrightarrow{AA'}}\cdot
 \frac{\overrightarrow{A'R}}{\overrightarrow{RB'}}\cdot
 \frac{\overrightarrow{B'B}}{\overrightarrow{BS}}=-1, \hspace*{1mm}
\frac{\overrightarrow{SA}}{\overrightarrow{AA'}}\cdot
 \frac{\overrightarrow{A'Q}}{\overrightarrow{QC'}}\cdot
 \frac{\overrightarrow{C'C}}{\overrightarrow{CS}}=-1, \hspace*{1mm}
 \frac{\overrightarrow{SC}}{\overrightarrow{CC'}}\cdot
 \frac{\overrightarrow{C'P}}{\overrightarrow{PB'}}\cdot
 \frac{\overrightarrow{B'B}}{\overrightarrow{BS}}=-1.
 \end{eqnarray*}
 From these three relations it follows:
  \begin{eqnarray*}
  \frac{\overrightarrow{A'Q}}{\overrightarrow{QC'}}\cdot
   \frac{\overrightarrow{C'P}}{\overrightarrow{PB'}}\cdot
   \frac{\overrightarrow{B'R}}{\overrightarrow{RA'}}=-1.
   \end{eqnarray*}
Therefore, according to Menelaus' theorem \ref{izrekMenelaj} (in the opposite direction) for the triangle $A'B'C'$
the points $P$, $Q$ and $R$ are collinear.
 \kdokaz

We prove the converse statement in a similar way.



            \bizrek \label{izrekDesarguesObr} \index{izrek!Desarguesov obratni}
            Let $ABC$ and $A'B'C'$ be two triangles in the
            plane such that the lines $AA'$ and $BB'$ intersect at the point $S$.
            If $P=BC\cap B'C'$,
             $Q=AC\cap A'C'$ and $R=AB\cap A'B'$ are collinear points, then also
            $S\in CC'$.\\
             (Converse of Desargues’ theorem)
            \eizrek


The following theorems are in a certain way similar to Desargues' theorem \ref{izrekDesarguesEvkl}.



            \bizrek \label{izrekDesarguesZarkVzp}
             Let $ABC$ and $A'B'C'$ be two triangles in the plane such that the lines $AA'$, $BB'$ and $CC'$
            are parallel to each other.
                   If $P=BC\cap B'C'$, $Q=AC\cap A'C'$ and $R=AB\cap A'B'$,
                 then the points $P$, $Q$ and
                $R$ are collinear.
            \eizrek

\begin{figure}[!htb]
\centering
\input{sl.pod.7.10D.2.pic}
\caption{} \label{sl.pod.7.10D.2.pic}
\end{figure}

\textbf{\textit{Proof.}}
  (Figure \ref{sl.pod.7.10D.2.pic})

   After using Tales' Theorem \ref{TalesovIzrek} three times, we get:
 \begin{eqnarray*}
 \frac{\overrightarrow{BP}}{\overrightarrow{PC}}=
 \frac{\overrightarrow{BB'}}{\overrightarrow{C'C}},\hspace*{4mm}
 \frac{\overrightarrow{CQ}}{\overrightarrow{QA}}=
 \frac{\overrightarrow{CC'}}{\overrightarrow{A'A}},\hspace*{4mm}
 \frac{\overrightarrow{AR}}{\overrightarrow{RB}}=
 \frac{\overrightarrow{AA'}}{\overrightarrow{B'B}}.
 \end{eqnarray*}
 After multiplying these three relations, we get:
  \begin{eqnarray*}
  \frac{\overrightarrow{BP}}{\overrightarrow{PC}}\cdot
   \frac{\overrightarrow{CQ}}{\overrightarrow{QA}}\cdot
   \frac{\overrightarrow{AR}}{\overrightarrow{RB}}=-1,
   \end{eqnarray*}
and therefore, according to Menelaus' Theorem \ref{izrekMenelaj} (in the opposite direction), for the triangle $ABC$
the points $P$, $Q$ and $R$ are collinear.
 \kdokaz


                   If $P=BC\cap B'C'$, $Q=AC\cap A'C'$ and $R=AB\cap A'B'$,
                 then the points $P$, $Q$ and
                $R$ are collinear.

                \bizrek \label{izrekDesarguesOsNesk}
                Let $ABC$ and $A'B'C'$ be two triangles in the plane such that
                the lines $AA'$, $BB'$ and $CC'$ intersect
                at a point $S$.
                If
                $BC\parallel B'C'$ and
                 $AC\parallel A'C'$, then also $AB\parallel A'B'$.
                \eizrek

\begin{figure}[!htb]
\centering
\input{sl.pod.7.10D.3.pic}
\caption{} \label{sl.pod.7.10D.3.pic}
\end{figure}


 \textbf{\textit{Proof.}} (Figure \ref{sl.pod.7.10D.3.pic})

Because $BC\parallel B'C'$ and
  $AC\parallel A'C'$, from Tales' theorem \ref{TalesovIzrek} it follows
  $\frac{\overrightarrow{SB}}{\overrightarrow{SB'}}=
  \frac{\overrightarrow{SC}}{\overrightarrow{SC'}}$ and
    $\frac{\overrightarrow{SA}}{\overrightarrow{SA'}}=
  \frac{\overrightarrow{SC}}{\overrightarrow{SC'}}$.
  From the previous two relations it follows first
    $\frac{\overrightarrow{SB}}{\overrightarrow{SB'}}=
  \frac{\overrightarrow{SA}}{\overrightarrow{SA'}}$, so by
  Tales' theorem (in the opposite direction) \ref{TalesovIzrekObr} it is also true that $AB\parallel A'B'$.
  \kdokaz

 The formulations of the previous three theorems are very similar, even though the
  proofs of these
  theorems are essentially different. The formulations would not differ at all, if we
  had assumed that all parallels (in one direction) of a plane intersect at
  the same point at infinity, and that all points at infinity of a plane
  determine exactly one line at infinity. This was actually the main motivation for the development of projective geometry,
  in which every two lines in a plane intersect (see \cite{Mitrovic}).

  In the following we will see some consequences of Desargues' theorem.



              \bzgled \label{zgled 3.2}
             Let $p$, $q$ and $r$ be three lines in the plane which
            intersect at the same point, and points $A$, $B$ and $C$ of this plane which do not belong
            to these lines. Construct a triangle whose vertices belong to the given
            lines, and the sides contain the given points.
             \ezgled



\begin{figure}[!htb]
\centering
\input{sl.pod.7.10D.4.pic}
\caption{} \label{sl.pod.7.10D.4.pic}
\end{figure}


 \textbf{\textit{Solution.}} (Figure \ref{sl.pod.7.10D.4.pic})

  Let $PQR$ be such a triangle, that
 its vertices $P$, $Q$ and $R$ belong to the lines $p$, $q$ and
 $r$, and sides $QR$, $PR$ and $PQ$ contain points $A$, $B$ and
 $C$. Let $S$ be the common point of lines $p$, $q$ and $r$.

If $P'Q'R'$ is an arbitrary triangle that is perspective to the triangle $PQR$ with respect to the center $S$, whose sides $R'Q'$ and $R'P'$ contain the points $A$ and $B$ (the condition regarding the point $C$ is omitted), then by Desargues' Theorem \ref{izrekDesarguesEvkl} the triangle $PQR$ and $P'Q'R'$ are perspective with respect to some line $s$. Therefore, the line $s$ contains the points $A$, $B$ and $Z=PQ\cap P'Q'$.

We construct the triangle $PQR$ by first constructing the auxiliary triangle $P'Q'R'$, where the point $R'\in r$ is arbitrary. Then we construct the point $Z$ as the intersection of the lines $AB$ and $P'Q'$. With the points $Z$ and $C$ the side $PQ$ is determined.

In the proof that $PQR$ is the desired triangle, we use the converse of Desargues' Theorem \ref{izrekDesarguesObr}.
  \kdokaz



%________________________________________________________________________________
\poglavje{Power of a Point} \label{odd7Potenca}

One of the most interesting characteristics of a circle that highlights some of its metric properties is the power of a point\footnote{The term power was first used in this sense by the Swiss mathematician \index{Steiner, J.}\textit{J. Steiner} (1769--1863).}. Before moving on to the definition, we prove the following theorem.



        \bizrek \label{izrekPotenca}
        Suppose that $P$ is an arbitrary point in the plane of a circle $k(S,r)$.
            For any line of this plane containing the point $P$ and intersecting the circle $k$ at points
            $A$ and $B$, the expression $\overrightarrow{PA}\cdot \overrightarrow{PB}$
             (Figure \ref{sl.pod.7.12.1b.pic})
         is constant, furthermore:
        $$\overrightarrow{PA}\cdot \overrightarrow{PB} = |PS|^2 - r^2.$$
       If $P$ is an exterior point of the circle $k$ and $PT$ its tangent at a point $T$, then:
           $$\overrightarrow{PA}\cdot \overrightarrow{PB} = |PT|^2.$$
        \eizrek



\begin{figure}[!htb]
\centering
\input{sl.pod.7.12.1b.pic}
\caption{} \label{sl.pod.7.12.1b.pic}
\end{figure}


\textbf{\textit{Proof.}} We will consider three possible cases.


\textit{1)} (Figure \ref{sl.pod.7.12.1.pic})

Let $P$ be an external point of the circle $k$. In this case, there is no $\mathcal{B}(A,P,B)$, so (the equivalence of \ref{eqnMnozVektRelacijaB} from section \ref{odd5DolzVekt}):
\begin{eqnarray} \label{eqnPotenIzr1}
\overrightarrow{PA}\cdot \overrightarrow{PB}>0.
\end{eqnarray}
Without loss of generality, assume that $\mathcal{B}(P,A,B)$ is true. Because $\angle PTA\cong\angle TBA=\angle TBP$ (the statement of \ref{ObodKotTang}) and $\angle TPA=\angle BPT$,
the triangles $PAT$ and $PTB$ are similar (the statement of \ref{PodTrikKKK}), so
$PA:PT=PT:PB$. If we use the Pythagorean theorem, we get:
 $$|PA|\cdot |PB| = |PT|^2 = |PS|^2 - r^2.$$
 From this, due to relation \ref{eqnPotenIzr1}, it follows:
$$\overrightarrow{PA}\cdot \overrightarrow{PB} = |PS|^2 - r^2.$$

\begin{figure}[!htb]
\centering
\input{sl.pod.7.12.1.pic}
\caption{} \label{sl.pod.7.12.1.pic}
\end{figure}

\textit{2)} (Figure \ref{sl.pod.7.12.1a.pic})

If the point $P$ lies on the circle $k$, then $P=A$ or $P=B$, therefore:
$$\overrightarrow{PA}\cdot \overrightarrow{PB} = 0 = |PS|^2 - r^2.$$

\begin{figure}[!htb]
\centering
\input{sl.pod.7.12.1a.pic}
\caption{} \label{sl.pod.7.12.1a.pic}
\end{figure}


\textit{3)} (Figure \ref{sl.pod.7.12.1a.pic})

Let $P$ be an inner point of the circle $k$. In this case, $\mathcal{B}(A,P,B)$, so (the equivalence of \ref{eqnMnozVektRelacijaB} from section \ref{odd5DolzVekt}):
\begin{eqnarray} \label{eqnPotenIzr2}
\overrightarrow{PA}\cdot \overrightarrow{PB}<0.
\end{eqnarray}
Let $A_1$ and $B_1$ be the intersection points of the line $SP$ with the circle $k$ (without loss of generality, let $\mathcal{B}(A_1,S,P)$). Because of the compatibility of the corresponding
circumferential angles (the statement of \ref{ObodObodKot}), $\triangle APA_1\sim \triangle B_1PB$ (the statement of \ref{PodTrikKKK}), so $AP:B_1P = PA_1:PB$, therefore
$$|PA|\cdot |PB|=|PA_1|\cdot |PB_1|=
\left(r+|PS|\right)\cdot\left(r-|PS|\right)=r^2-|PS|^2.$$
 From this, due to relation \ref{eqnPotenIzr1}, it follows
$$\overrightarrow{PA}\cdot \overrightarrow{PB} = |PS|^2 - r^2,$$ which had to be proven. \kdokaz

The constant product $\overrightarrow{PA}\cdot \overrightarrow{PB}$ from the previous statement of \ref{izrekPotenca} is called
\index{potenca točke}\pojem{potenca točke} $P$ with respect to the circle $k$ and is denoted $p(P,k)$.

According to the previous statement of \ref{izrekPotenca}, the power of the point $P$ with respect to the circle $k(S,r)$ is the number $|PS|^2 - r^2$.
This number is positive, negative or zero, depending on whether $P$ is an outer or inner or point on the circle $k$. So:
 \begin{eqnarray*}
 p(P,k)\hspace*{1mm}\left\{
                      \begin{array}{ll}
                        >0, & \textrm{if } OP>r; \\
                        =0, & \textrm{if } OP=r; \\
                        <0, & \textrm{if } OP<r.
                      \end{array}
                    \right.
\end{eqnarray*}


In the following, we will be interested in what the set of such points represents, for which the powers with respect to two given circles are equal. We first prove an auxiliary statement.

\bizrek \label{PotencOsLema}
                Let $A$ and $B$ be points and $d$  a line segment in the plane.
                A set of all points $X$ in this plane such that
                $$|AX|^2-|BX|^2=|d|^2,$$
                is a line perpendicular to the line $AB$.
                \eizrek

\begin{figure}[!htb]
\centering
\input{sl.pod.7.12.3.pic}
\caption{} \label{sl.pod.7.12.3.pic}
\end{figure}

\textbf{\textit{Proof.}}
Let $D$ be a point of this plane, such that:
$DB\perp BA$ and $DB\cong d$. With $X_0$ we mark the intersection
of the line $AB$ and the perpendicular of the line $AD$ (Figure \ref{sl.pod.7.12.3.pic}).
Because $DBX_0$ is a right angled triangle with hypotenuse $X_0D$, from Pythagoras' theorem (\ref{PitagorovIzrek}) it follows that for
the point $X_0$ it holds:
\begin{eqnarray} \label{eqnPotencaLema1}
 |AX_0|^2-|BX_0|^2=|DX_0|^2-|BX_0|^2=|DB|^2=|d|^2.
\end{eqnarray}
We prove that the sought set of points is perpendicular to the line
$AB$ in the point $X_0$. It is enough to prove the equivalence:
$$X\in x \hspace*{1mm} \Leftrightarrow \hspace*{1mm} |AX|^2-|BX|^2=|d|^2$$

($\Rightarrow$) If $X\in x$, by Pythagoras' theorem (\ref{PitagorovIzrek}) for the right angled triangles $XX_0A$ and $XX_0B$ and by statement \ref{eqnPotencaLema1}:
 \begin{eqnarray*}
 |AX|^2-|BX|^2&=&\left(|AX_0|^2+|XX_0|^2\right) -\left(|BX_0|^2-|XX_0|^2\right)=\\
&=&|AX_0|^2-|BX_0|^2=|d|^2
 \end{eqnarray*}

($\Leftarrow$) Let us now assume that $|AX|^2-|BX|^2=|d|^2$.
 Let $X'$ be the orthogonal projection of point $X$ on the line $AB$. By the Pythagorean theorem (\ref{PitagorovIzrek}) for the right-angled triangle $XX'A$ and $XX'B$ it holds:
\begin{eqnarray*}
 |d|^2=|AX|^2-|BX|^2&=&\left(|AX'|^2+|XX'|^2\right) -\left(|BX'|^2-|XX'|^2\right)=\\
&=&|AX'|^2-|BX'|^2.
 \end{eqnarray*}
 If we connect this with the relation \ref{eqnPotencaLema1}, we get:
\begin{eqnarray*}
 |d|^2=|AX'|^2-|BX'|^2=|AX_0|^2-|BX_0|^2,
 \end{eqnarray*}
or (if we take into account the property \ref{eqnMnozVektDolzina} of multiplication of collinear vectors):
\begin{eqnarray*}
&&|AX'|^2-|X'B|^2=|AX_0|^2-|X_0B|^2\\
&\Rightarrow&
\left(\overrightarrow{AX'}- \overrightarrow{X'B}\right)\left(\overrightarrow{AX'}+ \overrightarrow{X'B}\right)=
\left(\overrightarrow{AX_0}- \overrightarrow{X_0B}\right)\left(\overrightarrow{AX_0}+ \overrightarrow{X_0B}\right)\\
  &\Rightarrow&
\left(\overrightarrow{AX'}- \overrightarrow{X'B}\right)\overrightarrow{AB}=
\left(\overrightarrow{AX_0}- \overrightarrow{X_0B}\right)\overrightarrow{AB}\\
  &\Rightarrow&
\overrightarrow{AX'}- \overrightarrow{X'B}=
\overrightarrow{AX_0}- \overrightarrow{X_0B}\\
  &\Rightarrow&
2\overrightarrow{AX'}- \overrightarrow{AB}=
2\overrightarrow{AX_0}- \overrightarrow{AB}\\
  &\Rightarrow&
\overrightarrow{AX'}=
\overrightarrow{AX_0}\\
  &\Rightarrow&
X'=X_0.
 \end{eqnarray*}
This means that $XX_0\perp AB$ or $X\in x$.
\kdokaz



        \bizrek \label{PotencnaOs}
        A set of all points that have the same power with respect to
             two non-concentric circles is a line perpendicular
             to the line containing the centres of these two circles
         (Figure \ref{sl.pod.7.12.4.pic}).\\
        If the circles intersect, this line is their common secant.\\
        However, if the circles touch each other, this is their common tangent through their touching point.
        \eizrek

\begin{figure}[!htb]
\centering
\input{sl.pod.7.12.4.pic}
\caption{} \label{sl.pod.7.12.4.pic}
\end{figure}


\textbf{\textit{Proof.}}

Let $k_1(S_1,r_1)$ and $k_2(S_2,r_2)$ (for the sake of generality, let $r_1\geq r_2$) be any two circles. A point $P$ belongs to the desired set
precisely when $p(P,k_1)=p(P,k_2)$ or $|S_1P|^2-r_1^2=|S_2P|^2-r_2^2$; the latter relation is equivalent to: $$|S_1P|^2-|S_2P|^2=r_1^2-r_2^2.$$
Since by assumption $r_1^2-r_2^2\geq 0$. If $r_1^2-r_2^2>0$, by the previous statement (\ref{PotencOsLema}) the desired set
is a rectangle of the line $S_1S_2$. In the case $r_1=r_2$ it is clear that $|S_1P|=|S_2P|$, thus the desired set is the line $S_1S_2$'s symmetry.
We denote this line with $p$.

In the special case when the circles intersect in points $A$ and $B$, we have $p(A,k_1)=p(A,k_2)=0$ and $p(B,k_1)=p(B,k_2)=0$. This means that points $A$ and $B$ lie on the desired line $p$, thus this is the line $AB$.

If the circles touch in point $T$, we have $p(T,k_1)=p(T,k_2)=0$. This means that the point $T$ lies on the line $p$, which is the rectangle of the line $S_1S_2$. Thus the line $p$ is the common tangent of two given circles through their point of contact $T$.
\kdokaz

 The line $p$ from the previous statement is called the \index{potent!line}
   \pojem{potent line} of two
 circles. We will denote the potent line of circles $k_1$ and $k_2$ with $p(k_1,k_2)$.

It is interesting to find out how to construct the power line of two given circles $k_1$ and $k_2$ effectively.
In the cases where the circles intersect or touch, we have already given the answer in Theorem \ref{PotencnaOs} (Figure \ref{sl.pod.7.12.4.pic}).
It remains to construct the power line in the case where the circles have no common points. One option is to use Theorem \ref{PotencOsLema} directly. A slightly faster procedure is related to the construction
of the auxiliary circle $l$, which intersects the given circle at points $A$ and $B$ or $C$ and $D$. Then the intersection
of the lines $AB$ and $CD$ - point $X$ - lies on the desired power line $p(k_1,k_2)$. Indeed, from $p(k_1,l)=AB$ and $p(k_2,l)=CD$ it follows that $X\in p(k_1,l)$ and $X\in p(k_1,l)$ or $p(X,p_1)=p(X,l)=p(X,p_2)$.

Theorem \ref{PotencnaOs} does not consider one case - when
$k_1(S,r_1)$ and $k_2(s,r_2)$ are concentric circles. In this case, the mentioned set of points is an empty set. We get this from the condition
$|S_1P|^2-|S_2P|^2=r_1^2-r_2^2$. Namely, if $S_1=S_2$ and $r_1\neq r_2$, we get the condition $0=r_1^2-r_2^2\neq 0$, which is not satisfied for any point $P$.


Let us also define some concepts related to the power line of two circles.

Let the set of all such circles of a plane be such that each has two power lines $p$ that are equal for each pair of circles of this set. We call this set \index{set of circles}\pojem{set of circles}. The line $p$ is the \index{power!line}\pojem{power line} of this set of circles (Figure \ref{sl.pod.7.12.5.pic}).

\begin{figure}[!htb]
\centering
\input{sl.pod.7.12.5.pic}
\caption{} \label{sl.pod.7.12.5.pic}
\end{figure}

Let $p$ be the power line of a set of circles. From Theorem \ref{PotencnaOs} it is clear that all the centers of the circles of this set lie on the same line $s$, which is perpendicular to the line $p$.
We will consider three cases.

\textit{1)} If at least two circles of a family intersect in points $A$ and $B$, then $p=AB$, so all circles of this family go through points $A$ and $B$. In this case we say that it is a \index{family of circles!elliptic}\pojem{elliptic family of circles}.

\textit{2)} If at least two circles of a family touch in point $T$, then $p$ is a perpendicular of line $s$ in point $T$, so all circles have a common tangent $p$ in point $T$. In this case we say that it is a \index{family of circles!parabolic}\pojem{parabolic family of circles}.

\textit{3)} If no two circles of a family have common points, we say that it is a \index{family of circles!hyperbolic}\pojem{hyperbolic family of circles}. In this case the following property is valid: if an arbitrary circle intersects circles of this family (it is not necessary that it intersects all of them) in points $A_i$ and $B_i$, then all lines $A_iB_i$ go through one point that lies on line $p$.




        \bizrek \label{PotencnoSr}
        Let $k$, $l$ and $j$ be three non-concentric circles with non-collinear centres.
             Then there is exactly one point that has the same power with respect to all three circles.
             This point is the intersection of their three radical axes $p(k,l)$, $p(l,j)$ and $p(k,j)$.
        \eizrek


\textbf{\textit{Proof.}} (Figure \ref{sl.pod.7.12.6.pic}) Because the centres of circles $k$, $l$ and $j$  are three non-linear points, no two of lines $p(k,l)$, $p(l,j)$ and $p(k,j)$ are parallel. Let $P=p(k,l)\cap p(l,j)$. Then
$$p(P,k)=p(P,l)=p(P,j),$$
 i.e. $P\in p(k,j)$, which means that the power lines  $p(k,l)$, $p(l,j)$ and $p(k,j)$ intersect in point $P$.

If for another point $\widehat{P}$ of this plane it is valid that $p(\widehat{P},k)=p(\widehat{P},l)=p(\widehat{P},j)$, then $\widehat{P}\in p(k,l),\hspace*{1mm}p(l,j),\hspace*{1mm}p(k,j)$, so $\widehat{P}=P$.
\kdokaz

The point from the previous statement (\ref{PotencnoSr}) is called the \index{središče!potenčno}
          \pojem{potential center} of three circles. We will denote the potential center of circles $k$, $l$ and $j$ with $p(k,l,j)$.


\begin{figure}[!htb]
\centering
\input{sl.pod.7.12.6.pic}
\caption{} \label{sl.pod.7.12.6.pic}
\end{figure}

 In the case when the centers of three circles are three collinear points, and the circles are not from the same pencil, all three radical axes are parallel, because they are all perpendicular to the common center of these circles (statement \ref{PotencnaOs}).
 From this and statement \ref{PotencnoSr} we directly get the following statement (Figure \ref{sl.pod.7.12.6.pic}).


Let $k$, $l$ and $j$ be three non-concentric circles with non-collinear centres.
             Then there is exactly one point that has the same power with respect to all three circles.
             This point is the intersection of their three radical axes $p(k,l)$, $p(l,j)$ and $p(k,j)$.

                Radical axes of three circles in the plane that are not
                from the same pencil  and no two of them are concentric,
                belong to the same family of lines.


                \bizrek \label{PotencnoSrSop}
                Radical axes of three circles in the plane that are not
                from the same pencil  and no two of them are concentric,
                belong to the same family of lines.
                \eizrek

A special case of statement \ref{PotencnoSr}, when each two lines intersect, is the following theorem.


                \bzgled
                Let $k$, $l$, and $j$ be three circles of some plane with nonlinear
                centres and:
                \begin{itemize}
                  \item  $A$ and $B$ intersections of the circles $k$ and $l$,
                  \item  $C$ and $D$ intersections of the circles $l$ and $j$,
                 \item  $E$ and $F$ intersections of the circles $j$ and $k$.
                \end{itemize}
                Prove that the lines $AB$, $CD$ and $EF$ intersect at a single point.
                \ezgled

\begin{figure}[!htb]
\centering
\input{sl.pod.7.12.7.pic}
\caption{} \label{sl.pod.7.12.7.pic}
\end{figure}


\textbf{\textit{Proof.}} (Figure \ref{sl.pod.7.12.7.pic})

According to the \ref{PotencnaOs} theorem, $p(k,l)=AB$, $p(l,j)=CD$ and $p(j,k)=EF$ are the corresponding power axes. According to the \ref{PotencnoSr} theorem, they intersect at one point - the power center $P=p(k,l,j)$ of these three circles.
\kdokaz

The next statement is very similar.




                \bzgled
                Let $k$, $l$, and $j$ be three circles of some plane with nonlinear
                centres and:
                \begin{itemize}
                  \item $t_1$ the common tangent of the circles  $k$ in $l$,
                  \item $t_2$ the common tangent of the circles $l$ in $j$,
                  \item $t_3$ the common tangent of the circles $j$ in $k$.
                \end{itemize}
                Prove that the lines $t_1$, $t_2$ and $t_3$ intersect at a single point.
                \ezgled

\begin{figure}[!htb]
\centering
\input{sl.pod.7.12.8.pic}
\caption{} \label{sl.pod.7.12.8.pic}
\end{figure}


\textbf{\textit{Proof.}} (Figure \ref{sl.pod.7.12.8.pic})

According to the \ref{PotencnaOs} theorem, $p(k,l)=t_1$, $p(l,j)=t_2$ and $p(j,k)=t_3$ are the corresponding power axes. According to the \ref{PotencnoSr} theorem, they intersect at one point - the power center $P=p(k,l,j)$ of these three circles.
\kdokaz

\bzgled
                   a) Suppose that circles $k$ and $l$ are touching each other externally, and a line $t$ is
                    is the common tangent of these circles at their common point. Let $AB$ be a second
                    common tangent of these circles at touching points $A$ and $B$. Prove that the midpoint
                    the line segment $AB$ lies on the tangent $t$.\\
                   b) Suppose that circles $k$ and $l$ intersect at points $P$ and $Q$. Let $AB$ be a common tangent
                    of these circles at touching points  $A$ and $B$. Prove that the midpoint of the line segment $AB$ lies on
                    line $PQ$.
                \ezgled

\begin{figure}[!htb]
\centering
\input{sl.pod.7.12.9.pic}
\caption{} \label{sl.pod.7.12.9.pic}
\end{figure}


\textbf{\textit{Proof.}} (Figure \ref{sl.pod.7.12.9.pic})

\textit{a)} Let $S$ be the midpoint of the line segment $AB$. Then we have:
$p(S,k)= |SA|^2 = |SB|^2 = p(S,l)$,
so the point $S$ lies on the power curve of circles $k$ and $l$ or on the line $t$ (statement \ref{PotencnaOs}).

\textit{b)} Just as in the previous example, only that the power curve of circles $k$ and $l$ is the line $AB$ in this case.
\kdokaz



        \bizrek \label{EulerjevaFormula}
        \index{formula!Eulerjeva}
        (Euler's\footnote{\index{Euler, L.}
        \textit{L. Euler}
        (1707--1783), švicarski matematik.} formula) If $k(S,r)$ is the incircle and $l(O,R)$ the circumcircle
of an arbitrary triangle, then
        $$|OS|^2=R^2- 2Rr.$$
        \eizrek

\begin{figure}[!htb]
\centering
\input{sl.pod.7.12.10.pic}
\caption{} \label{sl.pod.7.12.10.pic}
\end{figure}


\textbf{\textit{Proof.}} (Figure \ref{sl.pod.7.12.10.pic})

We denote by $A$, $B$ and $C$ the vertices of a triangle, $NM$ the diameter of the circumscribed circle $l$, which is perpendicular to
the side $BC$ (and also $A,N\div BC$). By \ref{TockaN} the point $N$ lies on the bisector of the internal angle at the vertex $A$ or on the altitude
$AS$ (\ref{SredVcrtaneKrozn}). By \ref{TockaN.NBNC} we have $NS\cong NC$. If we use the power of the point $S$ with respect to the circle $l$ (\ref{PotencnoSr}) and the fact $\mathcal{B}(A,S,N)$, we get $p(S,l)= |SO|^2 - R^2 = \overrightarrow{SA}\cdot \overrightarrow{SN}
 = -|SA|\cdot |SN|
= -|SA|\cdot |CN|$, thus:
 \begin{eqnarray} \label{eqnEulFormOS}
  |SO|^2 - R^2 =  -|SA|\cdot |CN|.
\end{eqnarray}
We denote by $Q$ the point of contact of the inscribed circle $k$  with the side $AC$ of the triangle $ABC$.
By \ref{TangPogoj} and \ref{TalesovIzrKroz2} we have $\angle AQS\cong\angle MCN=90^0$, from \ref{ObodObodKot} we also get $\angle SAQ=\angle NAC\cong\angle NMC$.
Therefore, the triangles $AQS$ and $MCN$ are similar (\ref{PodTrikKKK}), so:
$$\frac{AS}{MN}=\frac{SQ}{NC}$$ or $|AS|\cdot |NC|=|MN|\cdot |SQ|=2Rr$. If we insert this into \ref{eqnEulFormOS}, we get:
$$|SO|^2 = R^2-|SA|\cdot |CN|=R^2-2Rr,$$
which had to be proven. \kdokaz


The next task is a special case of the previous formula and is therefore
its consequence (Figure \ref{sl.pod.7.12.10a.pic}).

\begin{figure}[!htb]
\centering
\input{sl.pod.7.12.10a.pic}
\caption{} \label{sl.pod.7.12.10a.pic}
\end{figure}


            \bnaloga\footnote{4. IMO, Czechoslovakia - 1962, Problem 6.}
            Consider an isosceles triangle. Let $r$ be the radius of its circumscribed circle
            and $\rho$ the radius of its inscribed circle. Prove that the distance $d$ between
            the centres of these two circles is $$d = \sqrt{r(r-2\rho)}.$$
          \enaloga

The next statement is also a direct consequence of \ref{EulerjevaFormula}.

\bizrek
If $k(S,r)$ is the incircle and $l(O,R)$ the circumcircle
of an arbitrary triangle, then
$$R\geq 2r.$$
Equality is achieved for an equilateral triangle.
\eizrek

The next design task is one of the ten Apollonius' problems on the tangency of circles, which we will investigate in more detail in section \ref{odd9ApolDotik}.


\bzgled
Construct a circle through two given points $A$ and $B$ and tangent to a given line $t$.
\ezgled

\begin{figure}[!htb]
\centering
\input{sl.pod.7.12.11.pic}
\caption{} \label{sl.pod.7.12.11.pic}
\end{figure}


\textbf{\textit{Solution.}} (Figure \ref{sl.pod.7.12.11.pic})

Let $k$ be the desired circle, which passes through points $A$ and $B$ and is tangent to the line $t$ in point $T$.

If the lines $AB$ and $t$ are parallel,
the third point $T$ of the circle $k$ is obtained as the intersection of the line
$t$ with the perpendicular of the line $AB$.

Let $P$ be the intersection of the lines $AB$ and $t$.

We will use the power of point $P$ with respect to
 the circle $k$. With $l$ we denote any
circle, which passes through points $A$ and $B$. By Theorem \ref{PotencnaOs} the line $AB$
is the power line of the circles $k$ and $l$, therefore $p(P,k)=p(P,l)$. We denote with $PT$ and
$PT_1$ the tangents of the circles $k$ and $l$ in their points $P$ and $P_1$. Then it holds (Theorem \ref{izrekPotenca}):
 $$|PT|^2=p(P,k)=p(P,l)=|PT_1|^2$$
or $|PT|=|PT_1|$.

The last relation allows us to construct the third point $T$ of the circle $k$.
 \kdokaz

\bzgled
            Let $E$ be the intersection of the bisector of the interior angle at the vertex
            $A$ with the side $BC$ of a triangle $ABC$ and $A_1$ the midpoint of this side.
            Let $P$ and $Q$
            be intersections of the circumcircle  of the triangle $AEA_1$ with the sides $AB$
            and $AC$ of the triangle $ABC$. Prove that:
             $$BP\cong CQ.$$
            \ezgled

\begin{figure}[!htb]
\centering
\input{sl.pod.7.12.12.pic}
\caption{} \label{sl.pod.7.12.12.pic}
\end{figure}


\textbf{\textit{Proof.}} (Figure \ref{sl.pod.7.12.12.pic})

We'll mark the circumcircle of the triangle $AEA_1$ with $k$. If we use the power of points $B$ and $C$ according to the circle $k$, the relation $\mathcal{B}(B,P,A)$ and $\mathcal{B}(C,Q,A)$ (because of the assumption that points $P$ and $Q$ lie on the sides $AB$ and $AC$ of the triangle $ABC$) and
the equivalence \ref{eqnMnozVektRelacijaB} from section \ref{odd5DolzVekt}, we get:
 \begin{eqnarray*}
p(B,k)&=&|BP|\cdot |BA|=|BE|\cdot |BA_1|,\\
p(C,k)&=&|CP|\cdot |CA|=|CE|\cdot |CA_1|.
\end{eqnarray*}
 From this and the relation $BA_1\cong CA_1$ and the theorem \ref{HarmCetSimKota} we get:
\begin{eqnarray*}
\frac{|BP|\cdot |BA|}{|CP|\cdot |CA|}
=\frac{|BE|\cdot |BA_1|}{|CE|\cdot |CA_1|}
=\frac{|BE|}{|CE|}
=\frac{|BA|}{|CA|}.
\end{eqnarray*}
Therefore $\frac{|BP|\cdot |BA|}{|CP|\cdot |CA|}=\frac{|BA|}{|CA|}$ or $|BP|=|CP|$.
 \kdokaz

            \bzgled
            Construct a circle that is perpendicular to three given circles
            $k$, $l$ in $j$.
            \ezgled

\begin{figure}[!htb]
\centering
\input{sl.pod.7.12.13.pic}
\caption{} \label{sl.pod.7.12.13.pic}
\end{figure}


\textbf{\textit{Solution.}} (Figure \ref{sl.pod.7.12.13.pic})

We will first assume that the centers of these circles $k$, $l$ and $j$ are nonlinear points. Let $x$ be the desired circle with center $P$, which is perpendicular to the circles $k$, $l$ and $j$, and $A\in x\cap k$, $B\in x\cap l$ and $C\in x\cap j$. Because $x\perp k,j,l$, $PA$, $PB$ and $PC$ are tangents of these circles from point $P$ (by \ref{pravokotniKroznici}). By \ref{izrekPotenca} we have $p(P,k)=|PA|^2$, $p(P,l)=|PB|^2$ and $p(P,j)=|PC|^2$. Because points $A$, $B$ and $C$ lie on the circle $x$ with center $P$, we have $|PA|^2=|PB|^2=|PC|^2$, i.e. $p(P,k)=p(P,l)= p(P,j)$. This means that $P=p(k,l,j)$ is the power center of the circles $k$, $l$ and $j$.

Therefore, the desired circle $x$ can be constructed by first drawing its center $P=p(k,l,j)$, then the radius $PA$, where the line $PA$ is tangent to the circle $k$ at point $A$.
It is clear that in the case where $P$ is an internal point of one of the circles $k$, $l$, $j$, the task has no solution.

Even in the case where the centers of the
circles are collinear points, the desired circle
does not exist (it is a "degenerate circle" or a line representing their common center).
 \kdokaz



            \bzgled
            Circles $k(O,r)$ and $l(S,\rho)$ and a point $P$ in the same
             plane are given.
            Construct a line passing through the point $P$, which determine congruent chords
            on the circles $k$ and $l$.
            \ezgled


\begin{figure}[!htb]
\centering
\input{sl.pod.7.12.14.pic}
\caption{} \label{sl.pod.7.12.14.pic}
\end{figure}


\textbf{\textit{Solution.}} (Figure \ref{sl.pod.7.12.14.pic})

Let $p$ be a line that goes through the point $P$,
the circles $k$ and $l$ intersect in such points $A$ and $B$ or $C$ and $D$, that $AB\cong CD$. Let
$\overrightarrow{v}$ be a vector that is determined by
the centers of the lines $AB$ and $CD$.
Then
$\mathcal{T}_{\overrightarrow{v}}:\hspace*{1mm}A,B\mapsto C,D$,
the circle $k$ is transformed by this translation  into
the circle $k'$, that goes through the points $C$ and $D$. So
the circles $k'$ and $l$ intersect in the points $C$ and $D$.
Thus the problem of constructing the line $p$ is translated into the problem
of constructing the vector
 $\overrightarrow{v}$ or the point
$O'= \mathcal{T}_{\overrightarrow{v}}(O)$, which represents the center of the circle $k'$.

The line, that
is determined by the centers $S$ and $O'$ of the circles $l$ and $k'$, is perpendicular to their common chord $CD$.
Because $\overrightarrow{OO'}= \overrightarrow{v}  \parallel p$, it follows
 $$\angle OO'S=90^0.$$
By  \ref{TalesovIzrKroz2} the point $O'$ lies on the circle above the diameter $OS$.

The point $P$ lies on the power line $p$ of the circles $k'$ and $l$ (\ref{PotencnaOs}). From this it follows $PL\cong PK$, where $PL$ and
$PK$ are tangents to the circles $l$ and $k'$ in the points $L$ and $K$.
This means that we can construct a triangle that is similar to the right triangle $PKO'$ ($PK\cong PL$, $\angle PO'K=90^0$ and $KO'\cong r$), and thus also
 $d$, which is similar to the distance $PO'$. So the point $O'$ belongs
to the intersection of the circle with center $P$ and radius $d$ and the circle above the diameter $OS$.
 \kdokaz



        \bnaloga\footnote{36. IMO Canada - 1995, Problem 1.}
        Let $A$, $B$, $C$, $D$ be four distinct points on a line, in that order $\mathcal{B}(A,B,C,D)$. The
        circles with diameters $AC$ and $BD$ intersect at $X$ and $Y$. The line $XY$
        meets $BC$ at $Z$. Let $P$ be a point on the line $XY$ other than $Z$. The
        line $CP$ intersects the circle with diameter $AC$ at $C$ and $M$, and the
        line $BP$ intersects the circle with diameter $BD$ at $B$ and $N$. Prove
        that the lines $AM$, $DN$, $XY$ are concurrent.
        \enaloga

\begin{figure}[!htb]
\centering
\input{sl.pod.7.12.IMO1.pic}
\caption{} \label{sl.pod.7.12.IMO1.pic}
\end{figure}

\textbf{\textit{Solution.}} Let $k_1$ and $k_2$ be the circles
over the segments $AC$ and $BD$ (Figure \ref{sl.pod.7.12.IMO1.pic}). By
the theorem \ref{KroznPresABpravokOS} the line $XY$ is perpendicular to
the central line $BC$ of these two circles and is also their power line
(theorem \ref{PotencnaOs}). Let $S_1=AM\cap XY$ and $S_2=DN\cap
XY$. It is enough to prove $S_1=S_2$.

  Because $M\in k_1$, by the theorem \ref{TalesovIzrek} $\angle
  AMC=90^0$ or $AS_1\perp MC$. Because of this the
  $\angle MCA\cong\angle AS_1Z$ (angle with perpendicular legs
   -  theorem \ref{KotaPravokKraki}). From this similarity of angles
   it follows that $\frac{AZ}{PZ}=\frac{ZS_1}{ZC}$ or
   $|ZS_1|=\frac{|ZC|\cdot |ZA|}{|PZ|}$. But $|ZC|\cdot
   |ZA|=p(Z,k_1)=|ZX|\cdot
   |ZY|=|ZX|^2$. From this it follows:
   $$|ZS_1|=\frac{|ZX|^2}{|PZ|}.$$
   In the same way it can be proven that:
 $|ZS_2|=\frac{|ZX|^2}{|PZ|}$. Because $S_1$ and $S_2$ are on
 the same line segment $ZP$, $S_1=S_2$.
  \kdokaz
     


                \bzgled 
                Let $P$ be an arbitrary point in the plane of a triangle $ABC$ which
                does not lie on any of lines containing altitudes of this triangle. Suppose $A_1$ is a point,
                in which a perpendicular line  of the line $AP$ at the point $P$ intersects the line $BC$. Analogously
                we can also define points $B_1$ and $C_1$.
                Prove that $A_1$, $B_1$ and $C_1$ are three collinear points.
                \ezgled

\textbf{\textit{Solution.}} We mark with $A_C$ and $B_C$ the
orthogonal projections of points $A$ and $B$ on the line $CP$.
Similarly, $A_B$ and $C_B$ are the orthogonal projections of
points $A$ and $C$ on the line $BP$, and $B_A$ and $C_A$ are the
orthogonal projections of points $B$ and $C$ on the line $AP$
(Figure \ref{sl.pd.7.4.6.pic}). By Tales' theorem \ref{TalesovIzrek}
it is:
 \begin{eqnarray} \label{4.1}\frac{AC_1}{C_1B}
 \cdot \frac{BA_1}{A_1C}
 \cdot \frac{CB_1}{B_1A}=
\frac{A_CP}{PB_C}
 \cdot
\frac{B_AP}{PC_A}
 \cdot
\frac{C_BP}{PA_B}
 \end{eqnarray}

\begin{figure}[!htb]
\centering
\input{sl.pd.7.4.6.pic}
\caption{} \label{sl.pd.7.4.6.pic}
\end{figure}


 From $\angle AA_CC\cong\angle AC_AC=90^0$ it follows that
 points $A_C$ and $C_A$ lie on the circle with diameter $AC$. Therefore
 the power of point $P$ on this circle is equal (by theorem \ref{izrekPotenca})
 $\overrightarrow{PC}\cdot \overrightarrow{PA_C}=
  \overrightarrow{PA}\cdot \overrightarrow{PC_A}$.
   Similarly,  $\overrightarrow{PB}\cdot \overrightarrow{PC_B}=
  \overrightarrow{PC}\cdot \overrightarrow{PB_C}$ and
   $\overrightarrow{PB}\cdot \overrightarrow{PA_B}=
  \overrightarrow{PA}\cdot \overrightarrow{PB_A}$. From these relations
  we get: $\frac{PA_C}{PC_A}=\frac{PA}{PC}$,
   $\frac{PC_B}{PB_C}=\frac{PC}{PB}$ and
   $\frac{PB_A}{PA_B}=\frac{PB}{PA}$. If we insert these relations in
   \ref{4.1}, we get  $\frac{AC_1}{C_1B}
 \cdot \frac{BA_1}{A_1C}
 \cdot \frac{CB_1}{B_1A}=1$.
 Because $\frac{\overrightarrow{AC_1}}{\overrightarrow{C_1B}}
 \cdot \frac{\overrightarrow{BA_1}}{\overrightarrow{A_1C}}
 \cdot \frac{\overrightarrow{CB_1}}{\overrightarrow{B_1A}}<0$,
 it is
 $\frac{\overrightarrow{AC_1}}{\overrightarrow{C_1B}}
 \cdot \frac{\overrightarrow{BA_1}}{\overrightarrow{A_1C}}
 \cdot \frac{\overrightarrow{CB_1}}{\overrightarrow{B_1A}}=-1$. By
 Menelaus' theorem points $A_1$, $B_1$ and $C_1$ are collinear.
\kdokaz

\bnaloga\footnote{41. IMO, S. Korea - 2000, Problem 1.}
        $AB$ is tangent to the circles $CAMN$ and $NMBD$. $M$ lies
        between $C$ and $D$ on the line $CD$, and $CD$ is parallel to $AB$. The chords
        $NA$ and $CM$ meet at $P$; the chords $NB$ and $MD$ meet at $Q$. The rays $CA$
        and $DB$ meet at $E$. Prove that $PE\cong QE$.
        \enaloga

\begin{figure}[!htb]
\centering
\input{sl.pod.7.12.IMO3.pic}
\caption{} \label{sl.pod.7.12.IMO3.pic}
\end{figure}

\textbf{\textit{Solution.}} We mark with $L$ the intersection of the lines $MN$
and $AB$ and with $k$ and $l$ the circumscribed circles
        $CAMN$ and $NMBD$ (Figure \ref{sl.pod.7.12.IMO3.pic}).

By Theorem \ref{PotencnaOs} the line $MN$ is the power line of the circles $k$
and $l$, therefore for its point $L\in MN$ we have
$|LA|^2=p(L,k)=p(L,l)=|LB|^2$, which means that $L$ is the center
of the line segment $AB$. Because, by assumption, $AB\parallel CD$, i.e.
$AB\parallel PQ$, by Tales' theorem $MP:MQ=LA:LB=1$. Therefore
the point $M$ is the center of the line segment $PQ$ or $MP\cong MQ$.

If we use Theorem \ref{ObodKotTang} and \ref{KotiTransverzala},
we get:
 \begin{eqnarray*}
 \angle EAB &\cong& \angle AMC \cong\angle MAB\\
 \angle EBA &\cong& \angle BMQ \cong\angle MBA
 \end{eqnarray*}
This means that the triangles $AEB$ and $AMB$ are similar (by the \textit{ASA} Theorem \ref{KSK}),
namely, they are symmetric with respect to the line $AB$. This
means that $EM\perp AB$. Because $AB\parallel PQ$, $EM\perp PQ$ or $\angle PME\cong\angle QME =90^0$.
We have already proven that $MP\cong MQ$, therefore the triangles $PME$ and $QME$ are similar
(by the \textit{SAS} Theorem \ref{SKS}), from which it follows that $PE\cong QE$.
 \kdokaz

\bnaloga\footnote{40. IMO, Romania - 1999, Problem 5.}
        Two circles $k_1$ and $k_2$ are contained inside the circle $k$, and are tangent to $k$
        at the distinct points $M$ and $N$, respectively. $k_1$ passes through the center of
        $k_2$. The line passing through the two points of intersection of $k_1$ and $k_2$ meets
        $k$ at $A$ and $B$. The lines $MA$ and $MB$ meet $k_1$ at $C$ and $D$, respectively.
        Prove that $CD$ is tangent to $k_2$.
        \enaloga

\begin{figure}[!htb]
\centering
\input{sl.pod.7.12.IMO4.pic}
\caption{} \label{sl.pod.7.12.IMO4.pic}
\end{figure}

\textbf{\textit{Solution.}} We mark with $O_1$ and $O_2$ the centers
of the circles $k_1$ and $k_2$, with $r_1$ and $r_2$ their radii and
with $E$ the other point of intersection of the line $AN$ with the
circle $k_2$ (Figure \ref{sl.pod.7.12.IMO4.pic}). Without loss of
generality we assume $r_1\geq r_2$.

We will first prove that the line $CE$ is the common tangent of the circles $k_1$ and $k_2$. Let $\widehat{E}$ be the other intersection of the drawn circle $k'$ of the triangle $CMN$ and the circle $k_2$. The point $A$ lies on the power axes of the circles $k_1$ and $k_2$ or $k_1$ and $k'$, so according to the \ref{PotencnoSr} theorem, the point $A$ is the power center of the circles $k_1$, $k_2$ and $k'$. This means that the point $A$ lies on the power line of the circles $k_2$ and $k'$ - the line $N\widehat{E}$. From this it follows that $\widehat{E}\in AN\cap k_2$ or $\widehat{E}=E$. So the points $M$, $C$, $E$ and $N$ are conciliatory and according to the \ref{ObodKotTang} theorem, $\angle ACE \cong\angle ANM$. We mark with $L$ any point of the common tangent of the circles $k$ and $k_1$ in the point $M$, which lies in the plane with the edge $AC$, in which the points $B$ and $D$ are not. According to the same theorem \ref{ObodKotTang} (with respect to the circles $k$ and $k_1$), we have $\angle LMA \cong\angle MBA$ and $\angle LMC \cong\angle MDC$. From the \ref{ObodObodKot} theorem (for the circle $k$ and the cord $AM$), we also get $\angle ANM \cong\angle ABM$. If we connect the proven relations, we get:
$$\angle ACE \cong \angle ANM
\cong \angle ABM\cong \angle LMA \cong\ \angle CDM. $$
 From $\angle ACE \cong \angle CDM$ and according to the \ref{ObodKotTang} theorem, it follows that $EC$ is the tangent of the circle $k_1$. Because in the proof we have not yet used the fact that $O_2\in k_1$, we analogously prove that $CE$ is the tangent of the circle $k_2$.

Let $T$ be the intersection of line $O_2O_1$ and circle $k_2$. It is enough to
prove that $T\in CD$ and $\angle CTO_2=90^0$. Let $O'_2$
be the orthogonal projection of point $O_2$ on line $O_1C$. Because $CE$
is the common tangent of circles $k_1$ and $k_2$, the radii $O_1C$ and
$O_2E$ are perpendicular to this tangent. Therefore, $CEO_2O'_2$
is a rectangle and $O'_2C\cong O_2E=r_2$ holds. From this it follows
that $O_1O'_2=r_1-r_2=O_1T$. This means that the triangle
$O_1O'_2O_2$ is similar (by the \textit{SAS} theorem \ref{SKS}), to triangle
$O_1TC$, so $\angle CTO_1\cong\angle O_2O'_2O_1=90^0$ or $\angle
CTO_2=90^0$. Line $CT$ and line $AB$ are perpendicular to
the center $O_1O_2$ of the two circles. Therefore, $CT\parallel AB$. Because
(due to the already proven relation $\angle ABM\cong \angle CDM$), also $CD\parallel AB$. By Playfair's axiom \ref{Playfair},
line $CT$ and line $CD$ are the same line, or $T\in CD$. This means that line
$CD$ is tangent to circle $k_2$ at point $T$.
\kdokaz




%________________________________________________________________________________
\poglavje{The Theorems of Pappus and Pascal} \label{odd7PappusPascal}

Theorems in this section are historically connected to the development of \index{geometry!projective}projective geometry.

            

            \bizrek \label{izrek Pappus} \index{theorem!Pappus'}(Pappus'\footnote{\index{Pappus} \textit{Pappus} from Alexandria (3rd century), one of the last
            great ancient Greek geometers. He proved this theorem in the Euclidean case,
            using metric. But the fundamental role of Pappus' theorem
                in projective geometry was discovered only sixteen centuries
            later.} theorem)
            Let $A$, $B$ and $C$ be three different
            points of a line $p$ and $A'$, $B'$ and $C'$ three different points of another line
            $p'$ in the same plane. Then the points
            $X=BC'\cap B'C$, $Y=AC'\cap A'C$ and $Z=AB'\cap A'B$ are collinear. 
            \eizrek

\begin{figure}[!htb]
\centering
\input{sl.pod.7.10.1.pic}
\caption{} \label{sl.pod.7.10.1.pic}
\end{figure}


 \textbf{\textit{Proof.}}
Let $L=AB'\cap BC'$, $M=AB'\cap CA'$ and $N=CA'\cap BC'$ (Figure \ref{sl.pod.7.10.1.pic}). We use Menelaus' theorem (\ref{izrekMenelaj}) five times
with respect to the triangle $LMN$ and the lines $BA'$, $AC'$, $CB'$, $AB$, $A'B'$:

\begin{eqnarray*}
& & \frac{\overrightarrow{LZ}}{\overrightarrow{ZM}}\cdot \frac{\overrightarrow{MA'}}{\overrightarrow{A'N}}\cdot \frac{\overrightarrow{NB}}{\overrightarrow{BL}}=-1,\\
& & \frac{\overrightarrow{LA}}{\overrightarrow{AM}}\cdot \frac{\overrightarrow{MY}}{\overrightarrow{YN}}\cdot \frac{\overrightarrow{NC'}}{\overrightarrow{C'L}}=-1,\\
& & \frac{\overrightarrow{LB'}}{\overrightarrow{B'M}}\cdot \frac{\overrightarrow{MC}}{\overrightarrow{CN}}\cdot \frac{\overrightarrow{NX}}{\overrightarrow{XL}}=-1,\\
& & \frac{\overrightarrow{LA}}{\overrightarrow{AM}}\cdot \frac{\overrightarrow{MC}}{\overrightarrow{CN}}\cdot \frac{\overrightarrow{NB}}{\overrightarrow{BL}}=-1,\\
& & \frac{\overrightarrow{LB'}}{\overrightarrow{B'M}}\cdot \frac{\overrightarrow{MA'}}{\overrightarrow{A'N}}\cdot \frac{\overrightarrow{NC'}}{\overrightarrow{C'L}}=-1.
\end{eqnarray*}
From these five relations (if we multiply the first three, then insert the fourth and fifth into the resulting relation) we get:
 $$\frac{\overrightarrow{LZ}}{\overrightarrow{ZM}}\cdot \frac{\overrightarrow{MY}}{\overrightarrow{YN}}\cdot \frac{\overrightarrow{NX}}{\overrightarrow{XL}}=-1.$$
 By the converse of Menelaus' theorem (\ref{izrekMenelaj}) the points $X$, $Y$ and $Z$ are collinear.
 \kdokaz

  We now prove Pascal's\footnote{It is not known how the French mathematician and philosopher \index{Pascal, B.}
  \textit{B. Pascal} (1623--1662) proved this statement for a circle, because
  the original proof is lost. However, we can assume that he used
   the results
   and methods of that time, which means that he probably used Menelaus' theorem.}
    theorem for
  a circle.

\bizrek \index{izrek!Pascalov} \label{izrekPascalEvkl}
         Let $A$, $B$, $C$, $D$, $E$ and  $F$ be arbitrary points on
        some circle $k$. Then the points $X=AE\cap BD$, $Y=AF\cap CD$ and
        $Z=BF\cap CE$ are collinear. 
        \eizrek

\begin{figure}[!htb]
\centering
\input{sl.pod.7.10.2.pic}
\caption{} \label{sl.pod.7.10.2.pic}
\end{figure}


 \textbf{\textit{Proof.}}
We mark $L=AE\cap BF$, $M=AE\cap CD$ and $N=CD\cap BF$.
 If we use Menelaus' theorem (\ref{izrekMenelaj}) three times for the triangle $LMN$ and the lines $BD$,
$AF$ and $CE$, we get (Figure \ref{sl.pod.7.10.2.pic}):
 \begin{eqnarray*}
  \hspace*{-2mm} \frac{\overrightarrow{LX}}{\overrightarrow{XM}}\cdot
   \frac{\overrightarrow{MD}}{\overrightarrow{DN}}\cdot
   \frac{\overrightarrow{NB}}{\overrightarrow{BL}}=-1,\hspace*{1mm}
   \frac{\overrightarrow{LA}}{\overrightarrow{AM}}\cdot
   \frac{\overrightarrow{MY}}{\overrightarrow{YN}}\cdot
   \frac{\overrightarrow{NF}}{\overrightarrow{FL}}=-1,\hspace*{1mm}
   \frac{\overrightarrow{LE}}{\overrightarrow{EM}}\cdot
   \frac{\overrightarrow{MC}}{\overrightarrow{CN}}\cdot
   \frac{\overrightarrow{NZ}}{\overrightarrow{ZL}}=-1.
  \end{eqnarray*}
 If we use the powers of the points $M$, $N$ and $L$ with respect to the circle
 $k$  (theorem \ref{izrekPotenca}) or consider the similarity of the corresponding triangles, we get:
  \begin{eqnarray*}
  \overrightarrow{MC}\cdot\overrightarrow{MD}=
  \overrightarrow{MA}\cdot\overrightarrow{ME},\hspace*{3mm}
  \overrightarrow{NC}\cdot\overrightarrow{ND}=
  \overrightarrow{NF}\cdot\overrightarrow{NB},\hspace*{3mm}
  \overrightarrow{LA}\cdot\overrightarrow{LE}=
  \overrightarrow{LB}\cdot\overrightarrow{LF}.
  \end{eqnarray*}
 From the previous six relations it follows:
 \begin{eqnarray*}
  \frac{\overrightarrow{LX}}{\overrightarrow{XM}}\cdot
   \frac{\overrightarrow{MY}}{\overrightarrow{YN}}\cdot
   \frac{\overrightarrow{NZ}}{\overrightarrow{ZL}}=-1.
   \end{eqnarray*}
Therefore, according to Menelaus' theorem (\ref{izrekMenelaj}, the inverse direction), the points $X$, $Y$ and
$Z$ are collinear.
 \kdokaz

The previous statement, which refers to a circle,
   can be generalized to any conic section\footnote{The study of conic sections began
with the Ancient Greeks. The terms ellipse, parabola, hyperbola
were first used by the Greek mathematician \index{Apolonij} \textit{Apolonij} from Perga (262--200 BC) in
 his
famous work \textit{Razprava o presekih stožca} (Discussion of the Intersections of a Cone), which
consists
 of
eight books, in which he defines a conic section as the intersection of a plane with
a circular cone. A commentary on this work by Apolonij was written by the Greek philosopher and mathematician, and the last representative of ancient science, \index{Hipatija} \textit{Hipatija} from Alexandria (370--415). The interest in conic sections was revived
 by the German astronomer \index{Kepler, J.} \textit{J. Kepler} (1571--1630) and
the French mathematician and philosopher \index{Pascal, B.} \textit{B.
Pascal} (1623--1662) in the 17th century.}
  or a curve of the second class.
   In the Euclidean case, these are: ellipse, parabola and hyperbola.
  We can determine this by defining
   a conic section
   as the intersection of the generators of all
   sides of the cone and some plane.
   Because of this, we generally look at a conic section $\mathcal{K}$ as
    the central
   projection of some circle $k$ onto some plane. The center of projection is
    the apex of the cone $S$. We have already mentioned that
    a central projection preserves collinearity. If $A'$, $B'$, $C'$,
     $D'$, $E'$ and  $F'$ are points on the conic section $\mathcal{K}$ and $X'$, $Y'$ and $Z'$
     are the corresponding points, defined as in statement \ref{izrekPascalEvkl},
     then these points are the images of some points $A$, $B$, $C$, $D$, $E$, $F$, $X$, $Y$ and
     $Z$.
    In this case, the first six points are on the circle $k$, and the last three are
    collinear by statement \ref{izrekPascalEvkl}. It follows that the points $X'$, $Y'$ and $Z'$ are also collinear (Figure \ref{sl.pod.7.10.3.pic}). This means that statement \ref{izrekPascalEvkl}
    really holds in the general case for any conic section. This general statement is
     known as Pascal's statement for conic sections\footnote{\index{Pascal, B.} \textit{B. Pascal}
(1623--1662), French mathematician and philosopher, who
  as a sixteen-year-old proved this important statement about
  conic sections, and published it
in 1640, but at that time the statement did not directly apply to
projective geometry.}.

\begin{figure}[!htb]
\centering
\input{sl.pod.7.10.3.pic}
\caption{} \label{sl.pod.7.10.3.pic}
\end{figure}

 The ideas we have discussed in this section lead us to the conclusion that we can also derive Pascal's theorem in projective geometry. Even more - in projective geometry we can define and study also the conics, but it is not possible to distinguish between ellipse, hyperbola or parabola (see \cite{Mitrovic}).


%________________________________________________________________________________
 \poglavje{The Golden Ratio} \label{odd7ZlatiRez}

 We say that point $Z$ of the line $AB$ divides this line in the ratio of the \index{golden!ratio}\pojem{golden ratio}\footnote{Such a division was already considered by \index{Pythagoras}\textit{Pythagoras from the island of Samos} (582--497 BC), an ancient Greek philosopher and mathematician. The first known records of the golden ratio were created by the ancient Greek mathematician \index{Euclid}
 \textit{Euclid of Alexandria} (3rd century BC). In his famous work \textit{Elements} he posed the problem: ‘‘\textit{Given a line, divide it into two unequal parts so that the area of the rectangle whose length is equal to the length of the whole line and whose height is equal to the length of the shorter part of the line is equal to the area of the square drawn on the longer part of the line.}’’
The term golden ratio, which we use today, was introduced by \index{Leonardo da Vinci}\textit{Leonardo da Vinci} (1452--1519),
an Italian painter, architect and inventor. The golden ratio has been used by people for thousands of years in painting and architecture.}, if the ratio of the length of the whole line to the longer part is equal to the ratio of the longer part to the shorter part (Figure \ref{sl.pod.7.15.1.pic}) or:

            \begin{eqnarray} \label{eqnZlatiRez}
            AB:AZ=AZ:ZB.
             \end{eqnarray}

\begin{figure}[!htb]
\centering
\input{sl.pod.7.15.1.pic}
\caption{} \label{sl.pod.7.15.1.pic}
\end{figure}

If
we now insert the shorter part "into" the longer part, we get the same
ratio. Indeed, because
$AZ:ZB=AB:AZ=(AZ+ZB):AZ$, it also holds that
$$ZB:(AZ-ZB)=AZ:ZB.$$
We can continue this process (Figure \ref{sl.pod.7.15.2.pic})

\begin{figure}[!htb]
\centering
\input{sl.pod.7.15.2.pic}
\caption{} \label{sl.pod.7.15.2.pic}
\end{figure}

Of course, the question arises as to how to construct such a point $Z$. We will describe this construction in the next example.



\bzgled
                For a given line $AB$, construct a point $Z$ that divides the line segment into
                the golden ratio.
                \ezgled


\begin{figure}[!htb]
\centering
\input{sl.pod.7.15.3.pic}
\caption{} \label{sl.pod.7.15.3.pic}
\end{figure}


 \textbf{\textit{Solution.}}
First, let's draw a circle $k(S,SB)$ with radius
$|SB|=\frac{1}{2}|AB|$, which touches the line $AB$ at point $B$ (Figure \ref{sl.pod.7.15.3.pic}). Let's also construct the intersections
of this circle with the line $AS$ - we'll mark them with $X$ and $Y$ (so that $\mathcal{B}(A,X,S)$ is true). We now get the point $Z$ as
the intersection of the line $AB$ and the circle
$l(A,AX)$.

We will prove that $Z$ is the desired point. If we use the power of point $A$ with respect to the circle $k$ (statement \ref{izrekPotenca}), we get:
$$\hspace*{-1.5mm} |AB|^2=p(A,k)=|AX|\cdot |AY|=|AX|\cdot(|AX|+|XY|)=|AZ|\cdot(|AZ|+|AB|).$$
 Therefore:
\begin{eqnarray} \label{eqnZlatiRez2}
  |AB|^2=|AZ|\cdot(|AZ|+|AB|)
  \end{eqnarray}
 Next:
 \begin{eqnarray*}
 |AB|^2=|AZ|\cdot(|AZ|+|AB|)\hspace*{1mm}&\Leftrightarrow &\hspace*{1mm}
 \frac{|AB|}{|AZ|}=\frac{|AZ|+|AB|}{|AB|}\\
 \hspace*{1mm}&\Leftrightarrow &\hspace*{1mm}
 \frac{|AB|}{|AZ|}=\frac{|AZ|}{|AB|}+1\\
 \hspace*{1mm}&\Leftrightarrow &\hspace*{1mm}
 \frac{|AB|}{|AZ|}-1=\frac{|AZ|}{|AB|}\\
 \hspace*{1mm}&\Leftrightarrow &\hspace*{1mm}
 \frac{|AB|-|AZ|}{|AZ|}=\frac{|AZ|}{|AB|}\\
 \hspace*{1mm}&\Leftrightarrow &\hspace*{1mm}
 \frac{|BZ|}{|AZ|}=\frac{|AZ|}{|AB|}
  \end{eqnarray*}
 which is equivalent to relation \ref{eqnZlatiRez}. This means that point $Z$ divides the line segment $AB$ in the golden ratio.
 \kdokaz

 In the next example we will calculate the value of the ratio determined by the golden ratio.

                

                 \bizrek \label{zlatiRezStevilo}
                If a point $Z$ divides a line segment $AB$ into the golden ratio
                ($AZ$ is the longer part), then
                \begin{eqnarray*}
                && AZ:ZB=AB:AZ=\frac{\sqrt{5}+1}{2}, \hspace*{1mm}\textrm{ i.e.}\\
                && AZ=\frac{\sqrt{5}-1}{2}AB.
                \end{eqnarray*}
                \eizrek

 \textbf{\textit{Proof.}}
 Relation \ref{eqnZlatiRez2} from the previous statement is equivalent to the relation:
 $$|AB|^2-|AB|\cdot |AZ|-|AZ|^2=0.$$
 If we solve this quadratic equation for $|AZ|$, we get:
 $$|AZ|=\frac{\sqrt{5}+1}{2}|AB|,$$
 from which the desired equalities follow.
 \kdokaz

The number $$\Phi=\frac{\sqrt{5}+1}{2}$$ from the previous equation, which therefore represents the value of the ratio determined by the golden cut, is called the \index{number!golden}\pojem{golden number}. Of course, it is an irrational number ($\Phi\notin \mathbb{Q}$). Its approximate value is:
$$\Phi=\frac{\sqrt{5}+1}{2}\doteq 1.62.$$

Similarly, it is:
$$\frac{\sqrt{5}-1}{2}\doteq 0.62,$$
which means that the longer part $AZ$ of the golden cut is approximately $62\%$ of the entire distance $AB$.

                . 

                 \bzgled \label{zlatiRezKonstr}
                Construct a golden rectangle $ABCD$ with two sides
                $a$ and $b$ in the golden ratio -  so-called  \index{golden!rectangle}\pojem{golden rectangle}\color{green1}.
                \ezgled


\begin{figure}[!htb]
\centering
\input{sl.pod.7.15.4.pic}
\caption{} \label{sl.pod.7.15.4.pic}
\end{figure}

 \textbf{\textit{Solution.}} We draw any line segment $AB$, then the point $Z$, which divides the line segment $AB$ in the golden cut so that $AZ$ is the longer part (see the previous example \ref{zlatiRezKonstr}). In the end, we get $D=\mathcal{R}_{A,90^0}(Z)$ and $C=\mathcal{T}_{\overrightarrow{AB}}(D)$
 (Figure \ref{sl.pod.7.15.4.pic}).
 \kdokaz

 If we divide the golden rectangle along its longer side in the golden ratio, we get a square and another golden rectangle. We divide the new golden rectangle in the same way and continue the process. If we connect the corresponding vertices (the places of the diagonals of the squares that determine the cusp), so that we draw circular arcs with a central angle of $90^0$, we get an approximate construction of the so-called \index{golden!spiral}\pojem{golden spiral} or \index{logarithmic!spiral}\pojem{logarithmic spiral} (Figure \ref{sl.pod.7.15.5.pic}).



\begin{figure}[!htb]
\centering
\input{sl.pod.7.15.5.pic}
\caption{} \label{sl.pod.7.15.5.pic}
\end{figure}

Logarithmic spiral\footnote{The proposal for the name logarithmic spiral was made by the French mathematician \index{Varignon, P.}\textit{P. Varignon} (1645--1722). The logarithmic spiral has the property that each straight line from the center of the spiral intersects it at the same angle. It appears in various forms in nature: from spirals in sunflower petals, snail shells and spider webs to distant galaxies. We also call it the spiral mirabilis (miraculous spiral) - this name was proposed by the Swiss mathematician \index{Bernoulli, J.}\textit{J. Bernoulli} (1667–-1748) because of its wonderful properties.} is a curve defined by equations in parametric form:
\begin{eqnarray*}
x&=&a e^{bt}\cdot\cos t \\
y&=&a e^{bt}\cdot\sin t,
\end{eqnarray*}
where $t\in \mathbb{R}$ is a parameter, $a$ and $b$ are arbitrary real constants and $e\doteq 2,72$ is the \index{number!Euler's }\pojem{Euler's\footnote{Number $e$ is an irrational number and represents the value of the limit $\lim_{n\rightarrow \infty}\left(1+\frac{1}{n} \right)^n$ =e. It was named after the Swiss mathematician \index{Euler, L.}\textit{L. Euler} (1707--1783).} number.}.
The previous construction (with circular arcs) is, as we have already mentioned, approximate, but it represents a very good approximation of this curve (Figure \ref{sl.pod.7.15.5a.pic}).


\begin{figure}[!htb]
\centering
\input{sl.pod.7.15.5a.pic}
\caption{} \label{sl.pod.7.15.5a.pic}
\end{figure}

We will continue with a regular pentagon.

                

                \bzgled \label{zlatiRezPravPetk1}
                A diagonal and a side of a regular pentagon are in the golden ratio. 
                \ezgled


 \textbf{\textit{Proof.}}  (Figure \ref{sl.pod.7.15.6.pic})

 Let's mark with $a$ the side and with $d$ the diagonal of the regular pentagon $ABCDE$. From \ref{PtolomejPetkotnik} it follows:
                    $$d=\frac{1+\sqrt{5}}{2}a.$$
Therefore, according to \ref{zlatiRezStevilo}, the segments $d$ and $a$ are in the golden ratio.
\kdokaz

\begin{figure}[!htb]
\centering
\input{sl.pod.7.15.6.pic}
\caption{} \label{sl.pod.7.15.6.pic}
\end{figure}

\bzgled
                Two diagonals of a regular pentagon intersect at a point that
                divides them in the golden ratio\footnote{The Pythagoreans knew this property of the pentagon. The Pythagorean school was founded by the Greek philosopher and mathematician \index{Pitagora}\textit{Pitagora from the island of Samos} (582--497 BC) in Croton in southern
                Italy. His students were engaged in philosophy, mathematics and natural sciences. For their distinctive sign they
                chose the pentagram, which is composed of the pentagon's diagonals. The influence of the Pythagorean school on the mathematics of the Ancient
                Greeks lasted for several centuries after Pythagora's death.}.
                \ezgled


 \textbf{\textit{Proof.}}
We denote by $P$ the intersection of the diagonals $AC$ and $BD$ of a regular pentagon $ABCDE$ and by $k$
its circumscribed circle (Figure \ref{sl.pod.7.15.6.pic}). The angles $EAD$, $DAC$, $CAB$ and $DBC$, which are adjacent to the
chords $ED$, $CD$ and $CB$, are congruent (statement \ref{SklTetSklObKot}). Because according to statement \ref{pravVeckNotrKot} the interior angle $EAB$ of a regular pentagon is equal to
$108^0$, the angles $EAD$, $DAC$, $CAB$ (or $PAB$) and
$DBC$ are equal to $36^0$. From this we obtain $\angle PBA=\angle DBA=\angle CBA-\angle CBD=108^0-36^0= 72^0$.  From the sum of the angles of triangle $ABP$ according to statement \ref{VsotKotTrik} it follows
$\angle APB=180^0-\angle PBA-\angle PAB=72^0$. Therefore, the triangle $PAB$ is isosceles with the base $PB$ (statement \ref{enakokraki}),
or $AP\cong AB$. From this it follows:
$$AC:AP=AC:AB.$$
If we use the previous statement \ref{zlatiRezPravPetk1}, we conclude that the point $P$ divides the diagonal $AC$ in the golden ratio.
\kdokaz

If in a regular pentagon $ABCDE$ we choose the vertices $A$, $B$ and $D$, we get an isosceles triangle $ABD$ with a base that is equal to the side $a$ of the pentagon $ABCDE$, and the legs are equal to the diagonals $d$ of this pentagon. The leg and the base of this triangle are therefore in the golden ratio (Figure \ref{sl.pod.7.15.7.pic}). Therefore, the triangle $ABD$ is called the \index{golden!triangle}\pojem{golden triangle}. The angles of the golden triangle measure $72^0$, $72^0$ and $36^0$.

\begin{figure}[!htb]
\centering
\input{sl.pod.7.15.7.pic}
\caption{} \label{sl.pod.7.15.7.pic}
\end{figure}

Similarly to the golden rectangle, if we use the similitudes of the corresponding internal angles at the base of the golden triangle, we get a sequence of golden triangles, all of which are similar. With the help of the corresponding circular arcs, we can also derive another approximate construction of the golden (or logarithmic) spiral (Figure \ref{sl.pod.7.15.7.pic}).

%________________________________________________________________________________
 \poglavje{Morley's Theorem and Some More Theorems} \label{odd7Morly}

We say that the segments $SP$ and $SQ$ \index{trisektrisa}\pojem{trisektrisi} of the angle $ASB$, if the points $P$ and $Q$ lie in this angle and it holds $$\angle ASP\cong\angle PSQ\cong\angle QSB,$$
or if it is a segment that divides the angle into three congruent angles
(Figure \ref{sl.pod.7.16.0.pic}).



\begin{figure}[!htb]
\centering
\input{sl.pod.7.16.0.pic}
\caption{} \label{sl.pod.7.16.0.pic}
\end{figure}

 We first prove an auxiliary assertion - lemma.

\bizrek \label{izrekMorleyLema}
 Let $Y'$, $Z$, $Y$ and $Z'$ be points in the plane such that
                    $Y'Y\cong ZY\cong ZZ'$ and
                     $$\angle Z'ZY\cong \angle ZYY'=180^0-2\alpha>60^0.$$
                If $A$ is a point in this plane that is on the different side
                of the line $Z'Y'$ with respect to
                the point $Z$ and also $Y'AZ'=3\alpha$, then the points $A$, $Y'$, $Z$, $Y$ and $Z'$ are
                concyclic and also
                $$\angle Z'AZ\cong \angle ZAY\cong \angle YAY'=\alpha.$$
\eizrek

\begin{figure}[!htb]
\centering
\input{sl.pod.7.16.1a.pic}
\caption{} \label{sl.pod.7.16.1a.pic}
\end{figure}


 \textbf{\textit{Proof.}} The given condition $180^0-2\alpha>60^0$ is equivalent to the condition $\alpha<60^0$.

Let $s$ be the perpendicular bisector of the line segment $ZY$, and $\mathcal{S}_s$ be the reflection over the line $s$ (Figure \ref{sl.pod.7.16.1a.pic}). Thus $\mathcal{S}_s(Z)=Y$. Because $\mathcal{S}_s$ is an isometry that preserves angles and lengths of line segments, $\mathcal{S}_s(Z')=Y'$ as well. From $ZY,Z'Y'\perp s$ it follows that $ZY\parallel Z'Y'$, which means that the trapezoid $ZYY'Z'$ is isosceles.
 By the \ref{trapezTetivEnakokr} theorem, this trapezoid is tangent - we denote its inscribed circle with $k$. By the \ref{KotiTransverzala} theorem, the angles $YZZ'$ and $Y'Z'Z$ are supplementary, so we have:
  \begin{eqnarray} \label{eqnMorleyLema1}
  \angle ZZ'Y'=180^0-\angle Z'ZY=2\alpha.
  \end{eqnarray}
  From $Z'Z\cong ZY$ by the \ref{SklTetSklObKot} theorem it follows that $\angle ZZ'Y\cong \angle YZ'Y'$, so by the relation \ref{eqnMorleyLema1} $\angle ZZ'Y\cong \angle YZ'Y'=\alpha$. The triangle $Z'ZY$ is isosceles with the base $Z'Y$, so by the \ref{enakokraki} theorem it is also $\angle ZYZ'\cong \angle ZZ'Y=\alpha$. Analogously, $\angle YY'Z\cong \angle ZY'Z'=\alpha$ and $\angle YZY'\cong \angle YY'Z=\alpha$. If we put everything together, we get $\angle ZZ'Y\cong \angle YZ'Y'\cong \angle ZYZ'\cong
  \angle YY'Z\cong \angle ZY'Z'\cong\angle YZY'=\alpha$, so:
   \begin{eqnarray} \label{eqnMorleyLema2}
  \angle Z'Y'Z \cong \angle Y'Z'Y= \alpha.
  \end{eqnarray}
  Because $\angle Z'ZY'=\angle Z'ZY-\angle Y'ZY=180^0-2\alpha-\alpha=180^0-3\alpha$ (this angle exists because $\alpha<60^0$) or $\angle Z'ZY'+\angle Z'AY'=180^0-3\alpha+3\alpha=180^0$, by the \ref{TetivniPogoj} theorem the quadrilateral $Z'ZY'A$ is tangent, which means that the point $A$ lies on the circle $k$.

 From the relation \ref{eqnMorleyLema2} and the \ref{ObodObodKot} theorem it follows:
  \begin{eqnarray*}
  &&\angle Z'AZ\cong \angle Z'Y'Z=\alpha,\\
  &&\angle Y'AY\cong \angle Y'Z'Y=\alpha.
  \end{eqnarray*}
  Finally, we have:
  \begin{eqnarray*}
  \angle ZAY=\angle Z'AY'-\angle Z'AZ-\angle Y'AY =3\alpha-\alpha-\alpha=\alpha,
  \end{eqnarray*}
  which was to be proven.  \kdokaz

Now we are ready for the basic theorem.


               


                \bizrek \label{izrekMorley}\index{izrek!Morleyev}
                 If $X$, $Y$ and $Z$ are three points of intersection of the adjacent angle trisectors
                 of a triangle $ABC$, then $XYZ$ is an equilateral triangle.\\
                (Morley's\footnote{\index{Morley, F.}\textit{F. Morley} (1860--1937), English mathematician, discovered this property of a triangle in 1904, but published it only 20 years
                later. At that time, the theorem was published as a task in the journal \textit{Educational Times}. Here we will give one of the solutions proposed at that time.}  theorem)
                \eizrek



\begin{figure}[!htb]
\centering
\input{sl.pod.7.16.1.pic}
\caption{} \label{sl.pod.7.16.1.pic}
\end{figure}


 \textbf{\textit{Proof.}}
  (Figure \ref{sl.pod.7.16.1.pic})

  Let $X$ and $V$ be the points of intersection of the respective trisectors of the internal angles $ABC$ and $ACB$ of the triangle $ABC$, so that $X$ is the internal point of the triangle $BVC$. We also denote by $Z_1$ and $Y_1$ the points of the trisector $BV$ and $CV$, for which $\angle Z_1XV \cong \angle VXY_1=30^0$. We first prove that $XY_1Z_1$ is an equilateral triangle.
Since the lines $BX$ and $CX$ are the angle bisectors of the angles $VBC$ and $VCB$, the point $X$ is the center of the inscribed circle of the triangle $BVC$, so the line $VX$ is the angle bisector of the angle $BVC$ (theorem \ref{SredVcrtaneKrozn}). From the similarity of the triangles $VXZ_1$ and
$VXY_1$ (theorem \textit{ASA} \ref{KSK}) it follows that $XZ_1\cong XY_1$. Since $\angle Z_1XY_1=60^0$, $XY_1Z_1$ is an equilateral triangle (theorem \ref{enakokraki}). It is enough to prove that $Y_1=Y$ and $Z_1=Z$ or $\angle BAZ_1\cong\angle Z_1AY_1\cong\angle Z_1AC$.

From the congruence of the triangles $VXZ_1$ and
$VXY_1$ it follows that $Z_1V \cong Y_1V$. Therefore, $VZ_1Y_1$ is an isosceles triangle with a base $Z_1Y_1$, so according to the theorem \ref{enakokraki}:
\begin{eqnarray} \label{eqnMorley1}
\angle VZ_1Y_1\cong\angle VY_1Z_1.
 \end{eqnarray}
Let $Z'$ and $Y'$ be such points on the sides $AB$ and $AC$ of the triangle $ABC$, that $BZ'\cong BX$ and $CY'\cong CX$.
The triangles $BZ'Z_1$ and $BXZ_1$ are congruent (theorem \textit{SAS} \ref{SKS}), so $Z'Z_1\cong XZ_1$. Also, $\angle Z'Z_1B\cong\angle XZ_1B$ or the line $BV$ is the perpendicular bisector of the angle $Z'Z_1X$.
Similarly, $Y'Y_1\cong XY_1$. From this and from the fact that $XY_1Z_1$ is an equilateral triangle, it follows:
\begin{eqnarray} \label{eqnMorley2}
Z'Z_1\cong Z_1Y_1\cong Y_1Y'.
 \end{eqnarray}
We denote with $3\alpha$, $3\beta$, $3\gamma$ the measures
of the internal angles of the triangle $ABC$, at the vertices $A$, $B$ and $C$.
It is clear that $3\alpha+3\beta+3\gamma=180^0$, therefore:
\begin{eqnarray} \label{eqnMorley3}
2\alpha+2\beta+2\gamma=120^0.
 \end{eqnarray}
If we use the fact that the line $BV$  is the perpendicular bisector of the convex or non-convex angle $Z'Z_1X$ and the relations \ref{eqnMorley1} and \ref{eqnMorley3}, after simple calculations we get:
\begin{eqnarray*}
 \angle Z'Z_1Y_1&=&\angle Z'Z_1V+\angle VZ_1Y_1=\\
 &=&\angle VZ_1X+\angle VZ_1Y_1=\\
 &=&60^0+2\angle VZ_1Y_1=\\
 &=&60^0+\angle VZ_1Y_1+\angle VY_1Z_1=\\
 &=&60^0+180^0-\angle Z_1VY_1=\\
 &=&60^0+2\beta+2\gamma=\\
 &=&60^0+120^0-2\alpha=\\
 &=&180^0-2\alpha.
 \end{eqnarray*}
 Similarly, $\angle Y'Y_1Z_1=180^0-2\alpha$, so (if we take into account the relation \ref{eqnMorley2}) by theorem \ref{izrekMorleyLema}
  $\angle Z'AZ_1\cong\angle Z_1AY_1\cong\angle Y_1AY'= \alpha$, which was to be proven.
 \kdokaz

\bizrek (Leibniz's\footnote{\index{Leibniz, G. W.}\textit{G. W. Leibniz} (1646--1716), German mathematician.} theorem)
                \label{izrekLeibniz}\index{theorem!Leibniz's}
                If $T$ is the centroid and $X$ an arbitrary
                point in the plane of a triangle $ABC$, then
                $$|XA|^2 + |XB|^2 + |XC|^2 = \frac{1}{3}\left(|AB|^2 +|BC|^2 +|CA|^2\right) + 3|XT|^2
                ,\textrm{ i.e.}$$
                $$|XA|^2 + |XB|^2 + |XC|^2 = |TA|^2 +|TB|^2 +|TC|^2 + 3|XT|^2.$$
                \eizrek


\begin{figure}[!htb]
\centering
\input{sl.pod.7.16.2.pic}
\caption{} \label{sl.pod.7.16.2.pic}
\end{figure}


 \textbf{\textit{Proof.}} (Figure \ref{sl.pod.7.16.2.pic})
  Let $A_1$ be the center
of the side $BC$ or $|AA_1|=t_a$ its
centroid. Because $AT:TA_1=2:1$ (statement \ref{tezisce}), by
Stewart's theorem \ref{StewartIzrek2}
with respect to the triangle
$AXA_1$ it follows:
\begin{eqnarray} \label{eqnLeibniz1}
|XT|^2=\frac{1}{3}|XA|^2+\frac{2}{3}|XA_1|^2-\frac{2}{9}t_a^2
\end{eqnarray}
If we use Stewart's theorem  \ref{StewartIzrek2} again (or its consequence for the centroid \ref{StwartTezisc}) with respect to the triangle $BXC$, we get:

\begin{eqnarray} \label{eqnLeibniz2}
|XA_1|^2=\frac{1}{2}|XB|^2+\frac{1}{2}|XC|^2-\frac{1}{4}|BC|^2
\end{eqnarray}
if we insert \ref{eqnLeibniz2} in \ref{eqnLeibniz1}:
\begin{eqnarray} \label{eqnLeibniz3}
|XT|^2=\frac{1}{3}\left(|XA|^2+|XB|^2+|XC|^2\right)-\frac{1}{6}|BC|^2-\frac{2}{9}t_a^2
\end{eqnarray}
and similarly:
\begin{eqnarray} \label{eqnLeibniz4a}
\hspace*{-4mm} |XT|^2&=&\frac{1}{3}\left(|XA|^2+|XB|^2+|XC|^2\right)-\frac{1}{6}|AC|^2-\frac{2}{9}t_b^2\\
\hspace*{-4mm} |XT|^2&=&\frac{1}{3}\left(|XA|^2+|XB|^2+|XC|^2\right)-\frac{1}{6}|AB|^2-\frac{2}{9}t_c^2\label{eqnLeibniz4}
\end{eqnarray}
By adding the equality from \ref{eqnLeibniz3} - \ref{eqnLeibniz4} we get:
\begin{eqnarray*}
3|XT|^2&=&|XA|^2+|XB|^2+|XC|^2-\frac{1}{6}\left(|BC|^2+|AC|^2+|AB|^2\right)-\\
&-&\frac{2}{9}\left(t_a^2+t_b^2+t_c^2\right)
\end{eqnarray*}
Finally, from the statement \ref{StwartTezisc2} it follows:
\begin{eqnarray*}
3|XT|^2=|XA|^2+|XB|^2+|XC|^2-\frac{1}{3}\left(|BC|^2+|AC|^2+|AB|^2\right)
\end{eqnarray*}
or both relations from the statement.
\kdokaz

A direct consequence of Leibniz's formula is the following statement.




                \bizrek
                A point in the plane of a triangle for which the squared distances
                 from its vertices has a minimum value is its centroid. 
                \eizrek

 \textbf{\textit{Proof.}} By Leibnitz's formula for any point $X$ in the plane of the triangle $ABC$ with centroid $T$ it holds:
$$|XA|^2 + |XB|^2 + |XC|^2 = \frac{1}{3}\left(|AB|^2 +|BC|^2 +|CA|^2\right) + 3|XT|^2.$$
The minimum of the sum $|XA|^2 + |XB|^2 + |XC|^2$ with respect to $X$ is therefore achieved when $|XT|$ is the smallest, which is when $X=T$.
\kdokaz

\bizrek (Carnot's\footnote{\index{Carnot, L. N. M.}\textit{L. N. M. Carnot} (1753--1823), 
                French mathematician.} theorem)\index{theorem!Carnot's}
                Let $P$, $Q$ and $R$ be points on the lines containing the
                sides $BC$, $CA$ and $AB$ of a triangle $ABC$. Perpendicular lines on the lines $BC$, $CA$.
                and $AB$ at the points $P$, $Q$ and $R$ intersect at one point if and only if
                \begin{eqnarray} \label{eqnCarnotIzrek1}
                |BP|^2 - |PC|^2 + |CQ|^2 - |QA|^2 + |AR|^2 - |RB|^2 = 0.
                \end{eqnarray}

                \eizrek


\begin{figure}[!htb]
\centering
\input{sl.pod.7.16.3.pic}
\caption{} \label{sl.pod.7.16.3.pic}
\end{figure}

 \textbf{\textit{Proof.}} We mark with $p$, $q$ and $r$ the perpendiculars of the lines $BC$, $CA$ and $AB$ through the points $P$, $Q$ and $R$ (Figure \ref{sl.pod.7.16.3.pic}).

($\Rightarrow$) We first assume that the lines $p$, $q$ and $r$ intersect at some point $L$. If we use the Pythagorean theorem \ref{PitagorovIzrek} six times, we get:
 \begin{eqnarray*}
\begin{array}{cc}
  |AR|^2+|RL|^2=|AL|^2 & \hspace{6mm} -|AQ|^2-|QL|^2=-|AL|^2 \\
  |BP|^2+|PL|^2=|BL|^2 & \hspace{6mm} -|BR|^2-|RL|^2=-|BL|^2\\
  |CQ|^2+|QL|^2=|CL|^2 & \hspace{6mm} -|CP|^2-|PL|^2=-|CL|^2
\end{array}
\end{eqnarray*}
 If we add all six equalities, we get relation \ref{eqnCarnotIzrek1}.

($\Leftarrow$) Now let's assume that the
relation \ref{eqnCarnotIzrek1} is true.
The rectangles $q$ and $r$ of the sides $AC$ and $AB$ are not parallel (because otherwise, as a result of Playfair's axiom  \ref{Playfair1}, the points $A$, $B$ and $C$ would be collinear). We denote by $\widehat{L}$ the intersection of the lines $q$ and $r$. We denote by $\widehat{P}$ the orthogonal projection of the point  $\widehat{L}$ onto the line $BC$. Because the rectangles of the sides of the triangle $ABC$ intersect at the points $\widehat{P}$, $Q$ and $R$ in the point $\widehat{L}$, from the first part of the proof ($\Rightarrow$) it follows:
    \begin{eqnarray} \label{eqnCarnotIzrek2}
    |B\widehat{P}|^2 - |\widehat{P}C|^2 + |CQ|^2 - |QA|^2 + |AR|^2 - |RB|^2 = 0.
    \end{eqnarray}
 From \ref{eqnCarnotIzrek1} and \ref{eqnCarnotIzrek2} it follows:
$$|B\widehat{P}|^2 - |\widehat{P}C|^2=|BP|^2 - |PC|^2.$$
 In the same way as at the end of the second part of the proof of the  \ref{PotencOsLema} theorem, we get $P=\widehat{P}$, i.e. the lines $p$, $q$ and $r$ intersect in one point.
\kdokaz


                

                \bizrek (Butterfly theorem\footnote{An English mathematician \index{Horner, W. J.}\textit{W. J. Horner} (1786--1837) published a proof of this theorem in 1815.})
                \index{theorem!butterfly}
                Let $S$ be the midpoint of chord $PQ$ of a circle $k$. Suppose that $AB$ and $CD$
                are arbitrary chords of this circle passing through the point $S$. If $X$ and $Y$
                are the points of intersection of the chords $AD$ and $BC$ with the chord $PQ$,
                then $S$ is the midpoint of the line segment $XY$.
                \eizrek



\begin{figure}[!htb]
\centering
\input{sl.pod.7.16.4.pic}
\caption{} \label{sl.pod.7.16.4.pic}
\end{figure}

\textbf{\textit{Proof.}} Let's mark $x=|SX|$ and $y=|SY|$
(Figure \ref{sl.pod.7.16.4.pic}).
 Let
$X_1$ and $X_2$ or $Y_1$ and $Y_2$
be the perpendicular projections of points $X$ or $Y$ on the lines $AB$
and $CD$.
By Tales' theorem (\ref{TalesovIzrekDolzine}) it is:
\begin{eqnarray*}
 \frac{x}{y}=\frac{XX_1}{YY_1}=\frac{XX_2}{YY_2}
\end{eqnarray*}
 If we use the latter equality, then
similarity of triangles $AXX_1$ and $CYY_2$ or
triangles $DXX_2$ and $BYY_1$ (statement \ref{PodTrikKKK}) and finally the power
of points $X$ and $Y$ with respect to the circle $k$ (statement \ref{izrekPotenca}), we get:
\begin{eqnarray*}
 \frac{x^2}{y^2}&=& \frac{|XX_1|}{|YY_1|}\cdot\frac{|XX_2|}{|YY_2|}=
 \frac{|XX_1|}{|YY_2|}\cdot\frac{|XX_2|}{|YY_1|}=\\
 &=& \frac{|AX|}{|CY|}\cdot\frac{|DX|}{|BY|}=\frac{|AX|\cdot |DX|}{|CY|\cdot |BY|}=\\
 &=& \frac{|XP|\cdot |XQ|}{|YP|\cdot |YQ|}=
 \frac{\left(|PS|-x\right)\cdot\left(|PS|+x\right)}
 {\left(|PS|+y\right)\cdot\left(|PS|-y\right)}=\\
 &=& \frac{|PS|^2-x^2}{|PS|^2-y^2}.
\end{eqnarray*}
 From $$\frac{x^2}{y^2}=\frac{|PS|^2-x^2}{|PS|^2-y^2}$$ it finally follows that $x=y$.
\kdokaz


                \bizrek \index{krožnica!Taylorjeva} \label{izrekTaylor}
                Let $A'$, $B'$ and $C'$ be the foots of the altitudes of a triangle $ABC$.
                The foot of the perpendiculars from the points $A'$,
                $B'$ and $C'$ on the lines containing the  adjacent sides
                of that triangle lie on a circle, so-called \pojem{Taylor\footnote{\index{Taylor, B.} \textit{B. Taylor} (1685--1731), angleški matematik.}  circle} \color{blue} of that triangle.
                \eizrek

\begin{figure}[!htb]
\centering
\input{sl.pod.7.16.5.pic}
\caption{} \label{sl.pod.7.16.5.pic}
\end{figure}

\textbf{\textit{Proof.}}
 Let $A_c$ and $A_b$ be the orthogonal projections of points $C'$ and $B'$ on the line $BC$,
 $B_a$ and $B_c$ be the orthogonal projections of points $A'$ and $C'$ on the line $AC$ and
 $C_a$ and $C_b$ be the orthogonal projections of points $A'$ and $B'$ on the line $AB$ (Figure \ref{sl.pod.7.16.5.pic}).

 Triangle $A'B'C'$ is
 the pedal triangle of triangle $ABC$, so from the proof of Theorem \ref{PedalniVS} it follows
$\angle B'A'C\cong\angle BAC=\alpha$ and $\angle C'B'A\cong\angle CBA=\beta$. Because $\angle B'B_aA'=\angle B'A_bA'=90^0$, by Theorem \ref{TalesovIzrKroz2} $A'A_bB_aB'$ is a trapezoid, so $\angle A_bB_aC\cong B'A'C=\alpha$ (Theorem \ref{TetivniPogojZunanji}). From this, by Theorem \ref{KotiTransverzala} it follows $A_bB_a \parallel BA$.

Triangle $C'A_cB_c$ and $VA'B'$ are perspective with respect to point $C$ (this means that the lines $C'V$, $A_cA'$ and $B_cB'$ intersect at point $C$). Because $C'A_c\parallel VA'$ and $C'B_c\parallel VB'$, by the generalization of Desargues' Theorem \ref{izrekDesarguesOsNesk} it follows that $A_cB_c\parallel A'B'$. By Theorem \ref{KotiTransverzala1} it follows that $\angle B_cA_cA'\cong B'A'C=\alpha$. Because therefore $\angle B_cA_cA'=\alpha=A_bB_aC$, $A_cA_bB_aB_c$ is a trapezoid (Theorem \ref{TetivniPogojZunanji}); we denote its circumscribed circle with $k$.

It remains to prove that the points $C_a$ and $C_b$ also lie on the circle $k$.  Analogously to the proven $A_cB_c\parallel A'B'$, it is also $C_aB_a\parallel C'B'$ and analogously to the proven, that $\angle A_bB_aC\cong B'A'C=\alpha$, it is also $\angle C_aA_cB\cong\angle A'C'B=\gamma$.
From the parallelism $C_aB_a\parallel C'B'$ it follows (from the statement \ref{KotiTransverzala1})
$\angle C_aB_aA \cong\angle C'B'A = \beta$, therefore:
 $$\angle C_aB_aA_b=180^0-\angle A_bB_aC-\angle C_aB_aA= 180^0-\alpha-\beta=\gamma.$$
 Therefore $\angle C_aA_cB\cong\angle C_aB_aA_b=\gamma$, which means (from the statement \ref{TetivniPogojZunanji}), that $C_aA_cA_bB_a$ is a cyclic quadrilateral, therefore the point $C_a$ lies on the circle $k$. Analogously we prove that also the point $C_b$ lies on the circle $k$.
\kdokaz

 A direct consequence is the following theorem.


            
                \bizrek
             Let $A'$, $B'$ and $C'$ be the foots of the altitudes of a triangle $ABC$.
                The foot of the perpendiculars from the points $A'$,
                $B'$ and $C'$ on the lines containing the  adjacent sides
                of that triangle determine three congruent line segments.
                \eizrek


\begin{figure}[!htb]
\centering
\input{sl.pod.7.16.6.pic}
\caption{} \label{sl.pod.7.16.6.pic}
\end{figure}


 \textbf{\textit{Proof.}} (Figure \ref{sl.pod.7.16.6.pic})

In the proof of the previous statement \ref{izrekTaylor} we have determined that $A_bB_a\parallel BA$, therefore the quadrilateral $C_bC_aA_bB_a$ is a trapezoid. This
trapezoid is also a cyclic (the previous statement \ref{izrekTaylor}) and therefore it is also an isosceles trapezoid (\ref{trapezTetivEnakokr}). Its diagonals are congruent (the statement \ref{trapezEnakokraki}), or. it is
$A_bC_b\cong C_aB_a$. Analogously it is also $A_bC_b\cong A_cB_c$.
\kdokaz



 \vspace*{12mm}


%Tales

\poglavje{The basics of Geometry} \label{osn9Geom}

It remains to prove that the points $C_a$ and $C_b$ also lie on the circle $k$.  Analogously to the proven $A_cB_c\parallel A'B'$, it is also $C_aB_a\parallel C'B'$ and analogously to the proven, that $\angle A_bB_aC\cong B'A'C=\alpha$, it is also $\angle C_aA_cB\cong\angle A'C'B=\gamma$.
From the parallelism $C_aB_a\parallel C'B'$ it follows (from the statement \ref{KotiTransverzala1})
$\angle C_aB_aA \cong\angle C'B'A = \beta$, therefore:
 $$\angle C_aB_aA_b=180^0-\angle A_bB_aC-\angle C_aB_aA= 180^0-\alpha-\beta=\gamma.$$
 Therefore $\angle C_aA_cB\cong\angle C_aB_aA_b=\gamma$, which means (from the statement \ref{TetivniPogojZunanji}), that $C_aA_cA_bB_a$ is a cyclic quadrilateral, therefore the point $C_a$ lies on the circle $k$. Analogously we prove that also the point $C_b$ lies on the circle $k$.
\kdokaz

 A direct consequence is the following theorem.


            
                \bizrek
             Let $A'$, $B'$ and $C'$ be the foots of the altitudes of a triangle $ABC$.
                The foot of the perpendiculars from the points $A'$,
                $B'$ and $C'$ on the lines containing the  adjacent sides
                of that triangle determine three congruent line segments.
                \eizrek


\begin{figure}[!htb]
\centering
\input{sl.pod.7.16.6.pic}
\caption{} \label{sl.pod.7.16.6.pic}
\end{figure}


 \textbf{\textit{Proof.}} (Figure \ref{sl.pod.7.16.6.pic})

In the proof of the previous statement \ref{izrekTaylor} we have determined that $A_bB_a\parallel BA$, therefore the quadrilateral $C_bC_aA_bB_a$ is a trapezoid. This
trapezoid is also a cyclic (the previous statement \ref{izrekTaylor}) and therefore it is also an isosceles trapezoid (\ref{trapezTetivEnakokr}). Its diagonals are congruent (the statement \ref{trapezEnakokraki}), or. it is
$A_bC_b\cong C_aB_a$. Analogously it is also $A_bC_b\cong A_cB_c$.
\kdokaz



 \vspace*{12mm}


%Tales

\poglavje{The basics of Ge

\item
Let $S$ be the intersection of the diagonal $AC$ and $BD$ of the trapezoid $ABCD$. Let $P$ and $Q$ be the intersections of the parallels to the bases $AB$ and $CD$ through the point $S$ with the sides of this trapezoid. Prove that $S$ is the center of the line $PQ$.

\item
Let $ABCD$ be a trapezoid with the base $AB$, the point $S$ be the intersection of its diagonals and $E$ be the intersection of the altitudes of this trapezoid. Prove that the line $SE$ goes through the centers of the bases $AB$ and $CD$.

\item
Let $P$, $Q$ and $R$ be points in which an arbitrary line through the point $A$ intersects the altitudes of the sides $BC$ and $CD$ and the diagonal $BD$ of the parallelogram $ABCD$. Prove that $|AR|^2=|PR|\cdot |QR|$.

\item
The points $D$ and $K$ lie on the sides $BC$ and $AC$ of the triangle $ABC$ so that $BD:DC=2:5$ and $AK:KC=3:2$. Calculate the ratio in which the line $BK$ divides the line $AD$.

\item
Let $P$ be a point on the side $AD$ of the parallelogram $ABCD$ such that $\overrightarrow{AP}=
\frac{1}{n}\overrightarrow{AD}$, and $Q$ be the intersection of the lines $AC$ and $BP$. Prove that:
 $$\overrightarrow{AQ}=\frac{1}{n + 1}\overrightarrow{AC}.$$



%Homotetija

\item
Given are: the point $A$, the lines $p$ and $q$ and the lines $m$ and $n$. Draw a line $s$ through the point $A$,
which intersects the lines $p$ and $q$ in such points $X$ and $Y$, that $XA:AY= m:n$.

\item Given are: the point $S$, the lines $p$, $q$ and $r$ and the lines $m$ and $n$. Draw a line $s$ through the point $S$,
which intersects the lines $p$, $q$ and $r$ in such points  $X$, $Y$ and $Z$, that $XY:YZ=m:n$.

 \item In the given triangle $ABC$ draw such a rectangle $PQRS$, that the side $PQ$ lies on the side $BC$,
the points $R$ and $S$ lie on the sides $AB$ and $AC$, and also $PQ=2QR$.

\item
Draw:\\
(\textit{a}) a rhombus, if given are the side $a$ and the ratio of the diagonals $e:f$;\\
 (\textit{b}) a trapezoid, if given are: the angles $\alpha$ and $\beta$ at one of the bases, the ratio of this base and the height $a:v$ and the other base $c$.

\item
Let $A$ be a point inside the angle $pSq$, $l$ a line and $\alpha$ an angle in some plane. Draw a triangle
$APQ$, such that: $P\in p$, $Q\in q$, $\angle PAQ\cong \alpha$ and
$PQ\parallel l$.

\item
Draw a circle $l$, that touches the given circle $k$ and the given line $p$, if the point of contact is given:\\
(\textit{a}) $l$ and $p$;\hspace*{3mm}
(\textit{b}) $l$ and $k$.

\item
Draw a circle, that touches the arms of the angle $pSq$ and:\\
(\textit{a}) goes through the given point,\\
(\textit{b}) touches the given circle.

\item Let $p$ and $q$ be lines, $S$ a point ($S\notin p$ and $S\notin q$) and the lines $m$ and $n$.
Draw circles $k$ and $l$, that touch each other from the outside in the point $S$, the first one touches the line $p$,
the second line $q$, and the ratio of the radii is $m:n$.


\item
In a plane are given: a line $p$ and  points $B$ and $C$, that lie on the circle $k$. Draw a point $A$
on the circle $k$, so that the centroid of the triangle $ABC$ lies on the line $p$.

\item
Let $k$ be a circle with the diameter $PQ$. Draw a square $ABCD$, so that $A,B\in PQ$ and $C,D\in k$.

\item
In the same plane are given: lines $p$ and $q$, a point $A$ and lines $m$ and $n$. Draw a rectangle $ABCD$, so that $B\in p$, $D\in q$
and $AB:AD=m:n$.

\item In the given triangle $ABC$ draw a triangle, so that its sides are parallel to the given lines $p$, $q$ and $r$.

\item Let $p$, $q$ and $r$ be three lines of some plane. Draw a line $t$, that
is perpendicular to the line $p$ and intersects the lines $p$, $q$ and $r$ in such points $P$, $Q$ and $R$, that $PQ\cong QR$.




%Similarity of triangles

\item Let $P$ be an inner point of the triangle $ABC$ and $A_1$, $B_1$ and $C_1$ be the orthogonal projections of
the point $P$ on the sides of the triangle $BC$, $AC$ and $AB$. Similarly, the points $A_2$, $B_2$ and $C_2$ are determined by the point $P$
and the triangle $A_1B_1C_1$,..., the points $A_{n+1}$, $B_{n+1}$ and $C_{n+1}$  with the point $P$ and the triangle $A_nB_nC_n$... Which of the triangles $A_1B_1C_1$, $A_2B_2C_2$, ... are similar to the triangle $ABC$?

\item
Let $k$ be the inscribed circle of the quadrilateral $ABCD$, $E$ be the intersection of its diagonals and $CB\cong CD$. Prove that $\triangle ABC \sim\triangle BEC$.

\item
Let $ABCD$ be a parallelogram. In the points $E$ and $F$ of the triangle $ABC$ the inscribed circle intersects the line $AD$ and $CD$. Prove that $\triangle EBC\sim\triangle EFD$.

\item
Let $AA'$ and $BB'$ be the altitudes of the acute angled triangle $ABC$. Prove that
$\triangle ABC\sim\triangle A'B'C$.

\item
Let the altitude $AD$ of the triangle $ABC$ be tangent to the inscribed circle of this triangle. Prove that $|AD|^2=|BD|\cdot |CD|$.

\item
In the triangle $ABC$ let the internal angle at the vertex $A$ be twice
as large as the internal angle at the vertex $B$. Prove that $|BC|^2= |AC|^2+|AC|\cdot |AB|$.

\item Prove that the radii of the inscribed circles of two similar triangles are proportional to the corresponding sides of these two triangles.


 \item The quadrilateral $ABCD$ is inscribed in a circle with center $S$. The diagonals of this quadrilateral are perpendicular and intersect at the point $E$. The line through the point $E$ and perpendicular to the side $AD$ intersects the side $BC$ at the point $M$. \\
(\textit{a}) Prove that the point $M$ is the center of the segment $BC$.\\
(\textit{b}) Determine the set of all points $M$, if the diagonal $BD$ changes and is always perpendicular to the diagonal $AC$.

\item Let $t$ be the tangent to the inscribed circle $l$ of the triangle $ABC$ at the vertex $A$.
Let $D$ be such a point of the line $AC$, that $BD\parallel t$. Prove that
$|AB|^2=|AC|\cdot |AD|$.

\item The altitude point of an acute angled triangle should divide its altitude in an equal ratio (from the vertex to the knife point of the altitude). Prove that it is an isosceles triangle.

\item In the triangle $ABC$, the altitude $BD$ touches the circumscribed circle of this triangle.
Prove:\\
(\textit{a}) that the difference of the angles at the base $AC$ is equal to $90^0$,\\
(\textit{b}) that $|BD|^2=|AD|\cdot |CD|$.

 \item The circle with the center on the base $BC$ of the isosceles triangle $ABC$ touches
the sides $AB$ and $AC$. The points $P$ and $Q$ are the intersections of these sides with any
tangent of this circle. Prove that $4\cdot |PB|\cdot |CQ|=|BC|^2$.

\item Let $V$ be the altitude point of the acute angled triangle $ABC$, the point $V$ the center of the altitude
$AD$, and the altitude $BE$ the point $V$ divides in the ratio $3:2$. Calculate the ratio in which $V$ divides the altitude $CF$.

\item Let $S$ be an external point of the circle $k$. $P$ and $Q$
are the points in which the circle $k$ touches its tangent from the point $S$, $X$ and $Y$ are the intersections of these circles with any line passing through the point $S$. Prove that $XP:YP=XQ:YQ$.

\item Let $D$ be a point lying on the side $BC$ of the triangle $ABC$. The points $S_1$ and $S_2$ shall be the centers of the circumscribed circles of the triangles $ABD$ and $ACD$. Prove that $ \triangle ABC\sim\triangle AS_1S_2$.

\item The point $P$ lies on the hypotenuse $BC$ of the triangle  $ABC$. The perpendicular of the line $BC$ at the point $P$ intersects the line $AC$ and $AB$ at the points $Q$ and $R$ and the circumscribed circle of the triangle $ABC$ at the point $S$. Prove that $|PS|^2=|PQ|\cdot |PR|$.

\item The point $A$ lies on the leg $OP$ of the right angle $POQ$. Let
$B$, $C$ and $D$ be such points of the leg $OQ$ that $\mathcal{B}(O,B,C)$, $\mathcal{B}(B,C,D)$ and
$OA\cong OB\cong BC\cong CD$. Prove that also $\triangle ABC\sim\triangle DBA$.

\item Draw a triangle if the following data is known:\\
(\textit{a}) $\alpha$, $\beta$, $R+r$, \hspace*{3mm}
 (\textit{b}) $a$, $b:c$, $t_c-v_c$,\hspace*{3mm}
 (\textit{c}) $v_a$, $v_b$, $v_c$.

\item Let $AB$ and $CD$ be the bases of an isosceles trapezoid $ABCD$, and let $r$ be the radius of the inscribed circle. Prove that $|AB|\cdot |CD|=4r^2$.




%Harmonic cetverica

\item Given is a circle $k$ and points $A$ and $B$. Draw a point $X$ on the circle $k$ such that $AX:XB=2:5$.

\item Draw a triangle with the given data:\\
(\textit{a}) $a$, $v_a$, $b:c$, \hspace*{3mm}
 (\textit{b}) $a$, $t_a$, $b:c$,\hspace*{3mm}
 (\textit{c}) $a$, $b$, $b:c$,\\
 (\textit{d}) $a$, $\alpha$, $b:c$,\hspace*{3mm}
  (\textit{e}) $a$, $l_a$, $b:c$.

\item Draw a triangle if the following data is given:\\
(\textit{a}) $v_a$, $r$, $\alpha$, \hspace*{3mm}
(\textit{b}) $v_a$, $r_a$, $a$, \hspace*{3mm}
(\textit{c}) $v_a$, $t_a$, $b-c$.

\item Draw a parallelogram, where one side and the corresponding altitude are consistent with the given distances $a$ and $v_a$, and the diagonals are in the ratio $3:5$.

\item Let $E$ be the intersection of the internal angle $BAC$ of the triangle $ABC$ with its side $BC$. Prove that
    $$\overrightarrow{AE}=\frac{|AC|}{|AB|+|AC|}\cdot\overrightarrow{AB}+
    \frac{|AB|}{|AB|+|AC|}\cdot\overrightarrow{AC}.$$

\item Given are four collinear points for which $\mathcal{H}(A,B;C,D)$ holds. Draw a point $L$, from which the distances $AC$, $CB$ and $BD$ are seen at the same angle.

\item Let $AE$ ($E\in BC$) be the internal angle of the triangle $ABC$ and let $a=|BC|$, $b=|AC|$ and  $c=|AB|$. Prove that
$$|BE|=\frac{ac}{b+c} \hspace*{1mm} \textrm{ and } \hspace*{1mm}  |CE|=\frac{ab}{b+c}.$$

\item Let $AE$ ($E\in BC$) and $BF$ ($F\in AC$) be the internal angle bisectors and $S$ the center of the inscribed circle of triangle $ABC$. Prove that $ABC$ is an isosceles triangle (with the base $AB$) exactly when $AS:SE=BS:SF$.




%Menelaus' Theorem


\item Prove that the external angle bisectors of any triangle intersect the opposite sides in three collinear points.



%Pythagoras' Theorem

\item Given $a$, $b$ and $c$ ($a>b$), construct such a distance $x$, that:\\
(\textit{a}) $x=\sqrt{a^2+b^2}$, \hspace*{3mm}
(\textit{b}) $x=\sqrt{a^2-b^2}$, \hspace*{3mm}
(\textit{c}) $x=\sqrt{3ab}$,\\
(\textit{d}) $x=\sqrt{a^2+bc}$, \hspace*{3mm}
(\textit{e}) $x=\sqrt{3ab-c^2}$, \hspace*{3mm}
(\textit{f}) $x=\frac{a\sqrt{ab+c^2}}{b+c}$.



%Stewart's Theorem

\item Let $a$, $b$ and $c$ be the sides of a triangle and $a^2+b^2=5c^2$. Prove that the centroids corresponding to sides $a$ and $b$ are perpendicular to each other.

\item Let $a$, $b$, $c$ and $d$ be the sides, $e$ and $f$ the diagonals and $x$ the distance determined by the midpoints of sides $b$ and $d$ of a quadrilateral. Prove:
$$x^2 = \frac{1}{4} \left(a^2 +c^2 -b^2 -d^2 +e^2 +f^2 \right).$$

\item Let $a$, $b$ and $c$ be the sides of triangle $ABC$. Prove that the distance of the centroid $A_1$ of side $a$ from the foot $A'$ of the altitude on that side is equal to:
$$|A_1A'|=\frac{|b^2-c^2|}{2a}.$$



%Pappus and Pascal

\item Let ($A$, $B$, $C$) and ($A_1$, $B_1$, $C_1$) be two triples of collinear points of a plane that are not on the same line. If $AB_1\parallel A_1B$ and $AC_1\parallel A_1C$, then also $CB_1\parallel C_1B$. (\textit{Pappus' Theorem}\footnote{Pappus of Alexandria\index{Pappus} (3rd century BC), Greek mathematician. This is a generalization of Pappus' Theorem (see Theorem \ref{izrek Pappus}), if we choose points $X$, $Y$ and $Z$ at infinity.})



%Desargues' Theorem

\item Let $P$, $Q$ and $R$ be points of sides $BC$, $AC$ and $AB$ of triangle $ABC$, such that lines $AP$, $BQ$ and $CR$ are from the same pencil. Prove: If $X=BC\cap QR$, $Y=AC\cap PR$ and $Z=AB\cap PQ$, then points $X$, $Y$ and $Z$ are collinear.


\item Let $AA'$, $BB'$ and $CC'$ be altitudes of triangle $ABC$ and $X=B'C'\cap BC$,  $Y=A'C'\cap AC$ and $Z=A'B'\cap AB$. Prove that points $X$, $Y$ and $Z$ are collinear points.

\item Let $A$ and $B$ be points outside line $p$. Draw the intersection of lines $p$ and $AB$ without drawing line $AB$ directly.

\item Let $p$ and $q$ be lines of a plane that intersect in point $S$, which is "outside the paper", and $A$ is a point of that plane. Draw the line that goes through points $A$ and $S$.

\item Draw a triangle so that its vertices lie on three given parallel lines and the carriers of its sides go through three given points.




%Ppotenca

\item Given is a circle $k(S,r)$.\\
(\textit{a}) Which values can the power of a point have with respect to circle $k$?\\
 (\textit{b}) What is the smallest value of this power and for which point is this minimal value achieved?\\
(\textit{c}) Determine the set of all points for which the power with respect to circle is equal to $\lambda\in \mathbb{R}$.

\item Let $k_a(S_a,r_a)$ and $l(O,R)$ be drawn and dashed circles of some triangle. Prove the equality\footnote{The statement is a generalization of Euler's formula for a circle (see Theorem \ref{EulerjevaFormula}). \index{Euler, L.}
        \textit{L. Euler}
        (1707--1783), Swiss mathematician.}:
   $$S_aO^2=R^2+2r_aR.$$


\item Draw a circle that goes through given points $A$ and $B$ and touches a given circle $k$.

\item Prove that the centers of lines that are determined by common tangents of two circles are collinear points.

\item Draw a circle that is perpendicular to two given circles and intersects the third given circle in points that determine the diameter of that third circle.

\item Let $M$ and $N$ be the intersection points of sides $AB$ and $AC$ of triangle $ABC$ with a line,
which passes through the center of the inscribed circle of this triangle and is parallel to its
side $BC$. Express the length of the line $MN$ as a function of the lengths of the sides of triangle $ABC$.

\item Let $AA_1$ be the median of triangle $ABC$. Points $P$ and $Q$ shall be the intersection points
of the altitudes $AA_1B$ and $AA_1C$ with sides $AB$ and $AC$. Prove that $PQ\parallel BC$.

\item In triangle $ABC$ let the sum (or difference) of the internal angles $ABC$ and $ACB$ be equal to a right
angle. Prove that $|AB|^2+|AC|^2=4r^2$, where $r$ is the radius of the circumscribed circle of this triangle.

\item Let $AD$ be the altitude of triangle $ABC$. Prove that
the sum (or difference) of the internal angles $ABC$ and $ACB$ is equal to a right angle exactly when:
$$\frac{1}{|AB|^2}+\frac{1}{|AC|^2}=\frac{1}{|AD|^2}.$$

\item Express the distance between the centroid and the center of the circumscribed circle of a triangle as
a function of the lengths of its sides and the radius of the circumscribed circle.

\item Prove that in triangle $ABC$ the altitude of the external angle at the vertex $A$ and the altitudes of
the internal angles at the vertices $B$ and $C$ intersect the opposite sides in three collinear points.

\item Prove that in triangle $ABC$ with altitude $AD$, the center of the circumscribed circle and the point, in which the side $BC$ touches the circumscribed circle of this triangle, are three
collinear points.

\item Prove Simson's theorem \ref{SimpsPrem} by using Menelaus' theorem \ref{izrekMenelaj}.

\item Through point $M$ of side $AB$ of triangle $ABC$ a line is constructed, which intersects
the line $AC$ in point $K$. Calculate the ratio, in which the line $MK$ divides the side $BC$,
if $AM:MB=1:2$ and $AK:AC=3:2$.

\item Let $A_1$ be the center of side $BC$ of triangle $ABC$ and let $P$ and $Q$ be such points of sides
$AB$ and $AC$, that $BP:PA=2:5$ and $AQ:QC=6:1$. Calculate the ratio, in which the line  $PQ$ divides the median $AA_1$.

\item Prove that in any triangle, the lines determined by the vertices and the points of contact of one of the inscribed circles of the opposite sides intersect in a common point.

\item What does the set of all points representing the intersection of two lines tangent to two given circles with respect to the centers and the points of contact of the circles represent?

\item Let $PP_1$ and $QQ_1$ be the external tangents of the circles $k(O,r)$ and $k_1(O_1,r_1)$ (the points $P$, $P_1$, $Q$ and $Q_1$ are the corresponding points of contact). Let $S$ be the intersection of these two tangents, $A$ one of the intersections of the circles $k$ and $k_1$, and $L$ and $L_1$ the intersections of the line $SO$ with the lines $PQ$ and $P_1Q_1$. Prove that $\angle LAO\cong\angle L_1AO_1$.

\item Prove that the side of a regular pentagon is equal to the larger part of the division of the radius of the inscribed circle of the pentagon in the golden ratio.

\item Let $a_5$, $a_6$ and $a_{10}$ be the sides of a regular pentagon, hexagon and decagon, which are inscribed in the same circle. Prove that:
 $$a_5^2=a_6^2+a_{10}^2.$$


\item Let $t_a$, $t_b$ and $t_c$ be the centroids and $s$ the semiperimeter of a triangle. Prove that:
    $$t_a^2+t_b^2+t_c^2\geq s^2.$$ % zvezek - dodatni MG


\end{enumerate}




% DEL 8 - - - - - - - - - - - - - - - - - - - - - - - - - - - - - - - - - - - - - - -
%________________________________________________________________________________
% PLOŠČINA
%________________________________________________________________________________

 \del{Area of Figures} \label{pogPLO}


%________________________________________________________________________________
\poglavje{Area of Figures. Definition}  \label{odd8PloLik}

 In this section we will define the concept of the area of a certain class of figures. The area is based on the concept of the length of a line in a certain system of measurement and on the theory of infinitesimal calculus\footnote{The first steps in this direction were made by the Greek mathematician \index{Euclid}\textit{Euclid} (408--355 BC) and \index{Archimedes}\textit{Archimedes} (287--212 BC) in the calculation of the volume of bodies.}.

Let $\Omega_0$ be a square with a side length of $1$ in a given measuring system. The aforementioned square represents the \index{unit of measurement of area}\pojem{unit of measurement of area} - we also call it the \index{unit square}\pojem{unit square}. Intuitively, the area of a figure $\Phi$ represents the number of unit squares that are consistent with square $\Omega_0$ and with which we can ''cover'' figure $\Phi$.

Now we will approach the definition of area more formally. Let $\Phi$ be an arbitrary figure in a plane. The unit square $\Omega_0$ defines the tiling $(4,4)$ (see section \ref{odd3Tlakovanja}) - we denote it with $\mathcal{T}_0$. With $\underline{S}_0$ we denote the number of squares of tiling $\mathcal{T}_0$, which are in figure $\Phi$, or are its subset.
With $\overline{S}_0$ we denote the number of squares of tiling $\mathcal{T}_0$, which have at least one common point with figure $\Phi$ (Figure \ref{sl.plo.8.1.1.pic}). It is clear that it holds:
 $$0\leq\underline{S}_0\leq\overline{S}_0.$$


\begin{figure}[!htb]
\centering
\input{sl.plo.8.1.1.pic}
\caption{} \label{sl.plo.8.1.1.pic}
\end{figure}


If we divide each side of the square into $10$ lines, we can divide the unit square $\Omega_0$ into $10^2$ consistent squares, which are all consistent with one of these squares $\Omega_1$, which has a side length of $\frac{1}{10}$. Square $\Omega_1$ defines a new tiling  $\mathcal{T}_1$. Similarly to the previous example, let $\underline{S}_1$ be the number of squares of tiling $\mathcal{T}_1$, which are in figure $\Phi$, and $\overline{S}_1$ be the number of squares of tiling $\mathcal{T}_1$, which have at least one common point with figure $\Phi$. It is clear that $\underline{S}_1\leq\overline{S}_1$ and also:
 $$0\leq\underline{S}_0\leq\frac{\underline{S}_1}{10^2}
\leq\frac{\overline{S}_1}{10^2}\leq\overline{S}_0.$$

If we continue the process, we get a sequence of squares $\Omega_n$, which have a side length of $\frac{1}{10^n}$, tiling $\mathcal{T}_n$ and pairs of numbers $\underline{S}_n$ and $\overline{S}_n$, for which the following is true:
$$0\leq\underline{S}_0\leq\frac{\underline{S}_1}{10^2}\leq \cdots \leq \frac{\underline{S}_n}{10^{2n}}\leq\cdots\leq \frac{\overline{S}_n}{10^{2n}}\leq\cdots
\leq\frac{\overline{S}_1}{10^2}\leq\overline{S}_0.$$

The sequence $\frac{\underline{S}_n}{10^{2n}}$ is increasing and bounded from above, so it is convergent by a known theorem of mathematical analysis and has the following limit:
$$\underline{S}=\lim_{n\rightarrow\infty}\frac{\underline{S}_n}{10^{2n}}.$$

Similarly, the sequence $\frac{\overline{S}_n}{10^{2n}}$ is decreasing and bounded from below, so it is convergent and has the following limit:
$$\overline{S}=\lim_{n\rightarrow\infty}\frac{\overline{S}_n}{10^{2n}}.$$

If $\underline{S}=\overline{S}$, we say that the figure $\Phi$ is measurable. The number $S=\underline{S}=\overline{S}$ is its \index{area of a figure}\pojem{area}. We denote it by $p(\Phi)$ or $p_{\Phi}$.

It is intuitively clear that the area of a figure $\Phi$ is not dependent on the location of the unit square - that is, it is only dependent on the system of measuring lengths, in which the side of the unit square has a length of 1. We will not formally prove this fact here.

We will now prove the first important property of area.



                \bizrek \label{ploscDaljice}
                The area of an arbitrary point is 0.
                The area of an arbitrary line segment is 0.
                \eizrek

\begin{figure}[!htb]
\centering
\input{sl.plo.8.1.2.pic}
\caption{} \label{sl.plo.8.1.2.pic}
\end{figure}

 \textbf{\textit{Proof.}}

\textit{1)} Let $A$ be an arbitrary point. It is clear that for every $n\in \mathbb{N}$ it holds: $\underline{S}_n=0$ and $0\leq\overline{S}_n\leq4$ (the point lies in at most four paving tiles $T_n$). So it holds:
$$\underline{S}=\lim_{n\rightarrow\infty}\frac{\underline{S}_n}{10^{2n}}
=\lim_{n\rightarrow\infty}\frac{0}{10^{2n}}=0$$
and
$$0\leq\overline{S}=\lim_{n\rightarrow\infty}\frac{\overline{S}_n}{10^{2n}}\leq
\lim_{n\rightarrow\infty}\frac{4}{10^{2n}}=0,$$
thus $p_A=S=\underline{S}=\overline{S}=0$

\textit{2)} Let $AB$ be an arbitrary distance (Figure \ref{sl.plo.8.1.2.pic}) and $|AB|=d$. With $k$ we denote the integer part of the number $d$, i.e. $k=[d]$. We choose a unit square $\Omega_0$, so that one of its vertices is the point $A$, one of its sides lies on the distance $AB$. In this case $\underline{S}_n=0$ and $\overline{S}_n\leq (k+1)\cdot 10^n$. Then it holds:
$$\underline{S}=\lim_{n\rightarrow\infty}\frac{\underline{S}_n}{10^{2n}}
=\lim_{n\rightarrow\infty}\frac{0}{10^{2n}}=0$$
and
$$0\leq\overline{S}=\lim_{n\rightarrow\infty}\frac{\overline{S}_n}{10^{2n}}\leq
\lim_{n\rightarrow\infty}\frac{(k+1)10^n}{10^{2n}}=
\lim_{n\rightarrow\infty}\frac{k+1}{10^n}=0,$$
thus $p_{AB}=S=\underline{S}=\overline{S}=0.$
\kdokaz

The next theorem, which refers to the basic properties of the area, we will state without a proof (see \cite{Lucic}).

\bizrek \label{ploscGlavniIzrek}
            Let $p$ be an area, defined on a set $\mu$ of measurable figures
            of a plane.
             Then it holds:
            \begin{enumerate}
              \item $p(\Omega_0)=1$;
              \item $(\forall \Phi\in\mu)\hspace*{1mm}p(\Phi)\geq 0$;
              \item $(\forall \Phi_1,\Phi_2\in\mu)\hspace*{1mm}
            \left(\Phi_1\cong\Phi_2
            \hspace*{1mm}\Rightarrow\hspace*{1mm}p(\Phi_1)=p(\Phi_2)\right)$;
              \item $(\forall \Phi_1,\Phi_2\in\mu)\hspace*{1mm}
            \left(p(\Phi_1\cap\Phi_2)=0
            \hspace*{1mm}\Rightarrow\hspace*{1mm}p(\Phi_1\cup\Phi_2)=
            p(\Phi_1)+p(\Phi_2)\right)$.
            \end{enumerate}
            \eizrek

The following theorem is very useful.


                \bizrek \label{ploscLomljenke}
                The area of any break is equal to 0.
                \eizrek


\begin{figure}[!htb]
\centering
\input{sl.plo.8.1.3.pic}
\caption{} \label{sl.plo.8.1.3.pic}
\end{figure}

 \textbf{\textit{Proof.}} (Figure \ref{sl.plo.8.1.3.pic})

The claim is a direct consequence of theorems \ref{ploscDaljice} and \ref{ploscGlavniIzrek} (\textit{4}).
\kdokaz



%________________________________________________________________________________
 \poglavje{Area of Rectangles, Parallelogram, and Trapezoids} \label{odd8PloParalel}


 In the following we will derive formulas for the area of certain quadrilaterals.

                \bizrek \label{ploscPravok}
                If $p$ is the area of a rectangle with sides of lengths $a$ and $b$, then
                it is:
                $$p=ab.$$
                \eizrek

\begin{figure}[!htb]
\centering
\input{sl.plo.8.2.1a.pic}
\caption{} \label{sl.plo.8.2.1a.pic}
\end{figure}


 \textbf{\textit{Proof.}}
Let $ABCD$ be a rectangle with sides $AB$ and $AD$ of lengths $|AB|=a$ and $|AD|=b$ (Figure \ref{sl.plo.8.2.1a.pic}).
Choose a unit square $\Omega_0$, so that one of its vertices is point $A$, one of its sides lies on the line $AB$. Let $\mathcal{T}_n$ ($n\in \mathbb{N}\cup\{0\}$) be the corresponding tiling that is determined by the unit square $\Omega_0$.

We denote with $a_n=[a\cdot 10^n]$ and $b_n=[b\cdot 10^n]$  (where $[x]$ represents the integer part of number $x$). First, we have:
 \begin{eqnarray} \label{eqnPloscPrav1a}
 \frac{a_n}{10^n}\leq a\leq \frac{a_n+1}{10^n} \hspace*{2mm} \textrm{ and }  \hspace*{2mm}
\frac{b_n}{10^n}\leq b\leq \frac{b_n+1}{10^n},
 \end{eqnarray}
then:
 \begin{eqnarray} \label{eqnPloscPrav2a}
 \underline{S}_n=\frac{a_n\cdot b_n}{10^{2n}} \hspace*{2mm} \textrm{ and }  \hspace*{2mm}
\overline{S}_n=\frac{\left(a_n+1\right)\cdot \left(b_n+1\right)}{10^{2n}}.
 \end{eqnarray}
From \ref{eqnPloscPrav2a} and \ref{eqnPloscPrav1a} we get:
\begin{eqnarray*}
0&\leq&\lim_{n\rightarrow\infty}\left(\overline{S}_n-\underline{S}_n\right)=\\
&=&\lim_{n\rightarrow\infty}\left(\frac{\left(a_n+1\right)\cdot \left(b_n+1\right)}{10^{2n}}-\frac{a_n\cdot b_n}{10^{2n}}\right)=\\
&=& \lim_{n\rightarrow\infty}\frac{a_n+ b_n+1}{10^{2n}}\leq\\
&\leq& \lim_{n\rightarrow\infty}\frac{a+b+1}{10^n}=\\
&=& 0.
 \end{eqnarray*}
Therefore:
\begin{eqnarray*}
\lim_{n\rightarrow\infty}\left(\overline{S}_n-\underline{S}_n\right)=0,
 \end{eqnarray*}
which means that $\underline{S}=\lim_{n\rightarrow\infty}\underline{S}_n
=\lim_{n\rightarrow\infty}\overline{S}_n=\overline{S}$, which means that the rectangle $ABCD$ is a measurable figure with the area $p=S=\underline{S}=\overline{S}$.

We will now prove $p=ab$. From \ref{eqnPloscPrav1a} we get:
 \begin{eqnarray*}
 \frac{a_n\cdot b_n}{10^{2n}} \leq ab\leq \frac{\left(a_n+1\right)\cdot \left(b_n+1\right)}{10^{2n}},
 \end{eqnarray*}
Since this is true for every $n\in \mathbb{N}$, it follows:
 \begin{eqnarray*}
 \hspace*{-2mm} \underline{S}=\lim_{n\rightarrow\infty}\underline{S}_n=
\lim_{n\rightarrow\infty}\frac{a_n\cdot b_n}{10^{2n}} \leq ab\leq \lim_{n\rightarrow\infty}\frac{\left(a_n+1\right)\cdot \left(b_n+1\right)}{10^{2n}}=\lim_{n\rightarrow\infty}\overline{S}_n=\overline{S}.
 \end{eqnarray*}
From $p=S=\underline{S}=\overline{S}$ at the end it follows $p=ab$.
\kdokaz

A direct consequence of the previous \ref{ploscPravok} is the following claim (Figure \ref{sl.plo.8.2.2.pic}).

                \bizrek \label{ploscKvadr}
                If $p_{\square}$ is the area of a square with a side of length $a$, then:
                $$p_{\square}=a^2.$$
                \eizrek



\begin{figure}[!htb]
\centering
\input{sl.plo.8.2.2.pic}
\caption{} \label{sl.plo.8.2.2.pic}
\end{figure}

We will now derive formulas for the area for some other quadrilaterals.


                \bizrek \label{ploscParal}
                If $p$ is the area of a parallelogram with sides of length $a$ and $b$ and
                the corresponding heights of length $v_a$ and $v_b$, then
                it is:
                $$p=av_a=bv_b.$$
                \eizrek

\textbf{\textit{Proof.}}
Let $ABCD$ be a parallelogram with side $AB$ of length $|AB|=a$ and the corresponding height $v_a$. We mark with $E$ and $F$ the orthogonal projections of the vertices $D$ and $C$ on the line $AB$. The quadrilateral $EFCD$ is a rectangle with sides $|CD|=a$ and $|FC|=v_a$, so according to the theorem \ref{ploscPravok}:
\begin{eqnarray}
p_{EFCD}=av_a. \label{eqnPloscPrav2}
\end{eqnarray}
The right-angled triangle $AED$ and $BFC$ are congruent (theorem \textit{ASA} \ref{KSK}) so according to the theorem \ref{ploscGlavniIzrek} \textit{3)}:
\begin{eqnarray}
p_{AED}=p_{BFC}. \label{eqnPloscPrav1}
\end{eqnarray}
Without loss of generality, we assume that $\angle BAD\leq 90^0$. In this case, the points $B$ and $E$ are on the same side of the point $A$ (in the case $E=A$ it is a rectangle $ABCD$ and the statement follows directly from the theorem \ref{ploscPravok}). We will consider several possible cases: $\mathcal{B}(A,E,B)$, $E=B$ and $\mathcal{B}(A,B,E)$.

\textit{1)} Let $\mathcal{B}(A,E,B)$ (Figure \ref{sl.plo.8.2.3.pic}).

\begin{figure}[!htb]
\centering
\input{sl.plo.8.2.3.pic}
\caption{} \label{sl.plo.8.2.3.pic}
\end{figure}

If we use the relations \ref{eqnPloscPrav2} and \ref{eqnPloscPrav1} and theorems \ref{ploscGlavniIzrek} \textit{4)} and \ref{ploscDaljice}, we get:
 \begin{eqnarray*}
 p_{ABCD}=p_{AED}+p_{EBCD}=p_{BFC}+p_{EBCD}=p_{EFCD}=av_a.
\end{eqnarray*}

\textit{2)} In the case $E=B$ (Figure \ref{sl.plo.8.2.3a.pic}) similarly from the relations \ref{eqnPloscPrav1} and \ref{eqnPloscPrav1} and theorems \ref{ploscGlavniIzrek} \textit{4)} and \ref{ploscDaljice} we get:
 \begin{eqnarray*}
 p_{ABCD}=p_{AED}+p_{BCD}=p_{BFC}+p_{BCD}=p_{EFCD}=av_a.
\end{eqnarray*}


\begin{figure}[!htb]
\centering
\input{sl.plo.8.2.3a.pic}
\caption{} \label{sl.plo.8.2.3a.pic}
\end{figure}



\textit{3)} We assume that $\mathcal{B}(A,B,E)$ (Figure \ref{sl.plo.8.2.3b.pic}).

\begin{figure}[!htb]
\centering
\input{sl.plo.8.2.3b.pic}
\caption{} \label{sl.plo.8.2.3b.pic}
\end{figure}

The lines $BC$ and $DE$ intersect in some point $L$. By the formulas \ref{ploscGlavniIzrek} \textit{4)} and \ref{ploscDaljice} it is:
$p_{AED}=p_{ABLD}+p_{BEL}$ and $p_{BFC}=p_{EFCL}+p_{BEL}$. Because by the relation \ref{eqnPloscPrav1} $p_{AED}=p_{BFC}$, it is also:
 \begin{eqnarray*}
 p_{ABLD}=p_{EFCL}.
\end{eqnarray*}
From this and from the formulas \ref{ploscGlavniIzrek} \textit{4)} and \ref{ploscDaljice} it follows:
 \begin{eqnarray*}
 p_{ABCD}=p_{ABLD}+p_{DLC}=p_{EFCL}+p_{DLC}=p_{EFCD}=av_a,
\end{eqnarray*}
which was to be proven. \kdokaz

                \bizrek \label{ploscTrapez}
                If $p$ is the area of a trapezoid with bases of lengths $a$ and $c$ and
                 height of length $v$, then
                it is:
                $$p=\frac{a+c}{2}\cdot v.$$
                \eizrek


\begin{figure}[!htb]
\centering
\input{sl.plo.8.2.4.pic}
\caption{} \label{sl.plo.8.2.4.pic}
\end{figure}

\textbf{\textit{Proof.}} Let $ABCD$ be a trapezoid with bases $AB$ and $CD$ of lengths $|AB|=a$ and $|CD|=c$ and a height of length $v$ (Figure \ref{sl.plo.8.2.4.pic}).
With $S$ we mark the center of the leg $BC$ and $\mathcal{S}_S:\hspace*{1mm}A,D\mapsto A', D'$. Because $S$ is the common center of the lines $AA'$ and $DD'$, by \ref{paralelogram} $AD'A'D$ is a parallelogram. Because $\mathcal{S}_S(B)=C$ and $\mathcal{B}(A,B,D')$, $\mathcal{B}(A',C,D)$ is also true. The isometry $\mathcal{S}$ maps the trapezoid $ABCD$ into a similar trapezoid $A'CBD'$, therefore $|BD'|=|CD|=c$ or $|AD'|=a+c$, and by \ref{ploscGlavniIzrek} \textit{3)} also $p_{ABCD}=p_{A'CBD'}$. So the parallelogram $AD'A'D$ with a base $AD'$ of length $|AD'|=a+c$ and a height which is equal to the height of the trapezoid $ABCD$ of length $v$, is divided into two similar trapezoids $ABCD$ and $A'CBD'$ with equal areas.

From this and from \ref{ploscGlavniIzrek} \textit{4)}, \ref{ploscDaljice} and \ref{ploscParal} it follows:
 \begin{eqnarray*}
 2\cdot p_{ABCD}=p_{ABCD}+p_{A'CBD'}=p_{AD'A'D}=\left(a+c\right)v,
\end{eqnarray*}
or the desired relation.
\kdokaz


If we use \ref{srednjTrapez}, we see that the expression $\frac{a+c}{2}$ represents the length of the median of the trapezoid, so for the area of the trapezoid the formula:
 $$p=mv,$$
where $m$ is the length of the median of the trapezoid and $v$ is its height
  (Figure \ref{sl.plo.8.2.4a.pic}) is also true.


\begin{figure}[!htb]
\centering
\input{sl.plo.8.2.4a.pic}
\caption{} \label{sl.plo.8.2.4a.pic}
\end{figure}




                \bzgled \label{ploscStirikPravok}
                If $p$ is the area of a quadrilateral with perpendicular diagonals of lengths $e$ and
                $f$, then:
                $$p=\frac{ef}{2}.$$
                \ezgled

\begin{figure}[!htb]
\centering
\input{sl.plo.8.2.5.pic}
\caption{} \label{sl.plo.8.2.5.pic}
\end{figure}

 \textbf{\textit{Proof.}} Let $ABCD$ be a quadrilateral with perpendicular diagonals $AC$ and $BD$ of lengths $|AC|=e$ and $|BD|=f$ (Figure \ref{sl.plo.8.2.5.pic}). The parallels of these diagonals through the vertices of the quadrilateral $ABCD$ determine the rectangle $A'B'C'D'$ with sides of lengths $|A'B'|=f$ and $|B'C'|=e$. From the similarity of triangles $ASD$ and $DA'A$ (the \textit{SAS} theorem \ref{SKS}) it follows that $p_{ASD}=p_{DA'A}$ (the \ref{ploscGlavniIzrek} \textit{3)}). Similarly, $p_{ASB}=p_{BB'A}$, $p_{CSB}=p_{BC'C}$ and $p_{CSD}=p_{DD'C}$. From this and from theorems \ref{ploscGlavniIzrek} \textit{4)}, \ref{ploscDaljice} and \ref{ploscPravok} it follows:
 \begin{eqnarray*}
 2\cdot p_{ABCD}&=&2\cdot\left(p_{ASD}+p_{ASB}+p_{CSB}+p_{CSD}\right)=\\
&=&2\cdot p_{ASD}+2\cdot p_{ASB}+2\cdot p_{CSB}+2\cdot p_{CSD}=\\
&=& p_{ASD}+ p_{DA'A}+ p_{ASB}+p_{BB'A}+\\
 && + p_{CSB}+p_{BC'C} +p_{CSD}+p_{DD'C}=\\
&=& p_{A'B'C'D'}= ef,
\end{eqnarray*}

or the desired relation.
\kdokaz


As a consequence, we have the following theorem (Figure \ref{sl.plo.8.2.5a.pic}).


                \bzgled \label{ploscDeltoid}
                If $p$ is the area of a deltoid with diagonals of lengths $e$ and
                $f$, then:
                $$p=\frac{ef}{2}.$$
                \ezgled

 \textbf{\textit{Proof.}}

     The theorem is a direct consequence of theorem \ref{ploscStirikPravok} and the definition of a deltoid.
\kdokaz


\begin{figure}[!htb]
\centering
\input{sl.plo.8.2.5a.pic}
\caption{} \label{sl.plo.8.2.5a.pic}
\end{figure}

\bzgled \label{ploscRomb}
If $p$ is the area of a rhombus with diagonals of lengths $e$ and
$f$, then:
$$p=\frac{ef}{2}.$$
\ezgled

\textbf{\textit{Proof.}}
The statement is a direct consequence of Theorems \ref{ploscStirikPravok} and \ref{RombPravKvadr}.
\kdokaz



\bzgled \label{ploscRomb1}
If: $a$ is the length of a side, $v$ is the height, and $e$ and
$f$ are the lengths of the diagonals of the rhombus, then:
$$av=\frac{ef}{2}.$$
\ezgled

\begin{figure}[!htb]
\centering
\input{sl.plo.8.2.6.pic}
\caption{} \label{sl.plo.8.2.6.pic}
\end{figure}

 \textbf{\textit{Proof.}} (Figure \ref{sl.plo.8.2.5a.pic})

   The statement is a direct consequence of Theorems \ref{ploscRomb1} and \ref{ploscParal}.
\kdokaz


%________________________________________________________________________________
 \poglavje{Area of Triangles} \label{odd8PloTrik}

In this section we will derive several formulas for the area of a triangle.

            \bizrek \label{PloscTrik} If $p_\triangle$ is the
           area of the triangle $ABC$, $v_a$, $v_b$ and $v_c$ are the lengths of the heights corresponding to the sides $BC$, $AC$ and $AB$ with lengths $a$, $b$ and $c$, then:
           $$p_\triangle=\frac{a\cdot v_a}{2}=\frac{b\cdot v_b}{2}
           =\frac{c\cdot v_c}{2}.$$
           \eizrek

\begin{figure}[!htb]
\centering
\input{sl.plo.8.3.1a.pic}
\caption{} \label{sl.plo.8.3.1a.pic}
\end{figure}

\textbf{\textit{Proof.}}  Let $D$ be the fourth vertex of the parallelogram $ABCD$ (Figure \ref{sl.plo.8.3.1a.pic}). The triangles $ABC$ and $ADC$ are congruent (by \ref{paralelogram} and \ref{SSS}), therefore $p_{ABC}=p_{ADC}$ (by \ref{ploscGlavniIzrek} \textit{3)}). The parallelogram $ABCD$ has the side $BC$ and the corresponding altitude of lengths $a$ and $v_a$, therefore by  \ref{ploscParal} $p_{ABCD}=av_a$. If we use \ref{ploscGlavniIzrek} \textit{4)} and \ref{ploscDaljice} as well, we get:
 \begin{eqnarray*}
 2\cdot p_{ABC}=p_{ABC}+p_{ADC}=p_{ABCD}=av_a,
\end{eqnarray*}
or the desired relation $p_\triangle=\frac{a\cdot v_a}{2}$. Similarly, $p_\triangle=\frac{b\cdot v_b}{2}$ or $p_\triangle=\frac{c\cdot v_c}{2}$.
\kdokaz

          \bizrek \label{PloscTrikVcrt} If $p_\triangle$ is the area
           of the triangle $ABC$ with
          the semi-perimeter $s=\frac{a+b+c}{2}$  and the radius of the inscribed circle
          $r$, then
          $$p_\triangle=sr.$$
          \eizrek

\begin{figure}[!htb]
\centering
\input{sl.plo.8.3.1.pic}
\caption{} \label{sl.plo.8.3.1.pic}
\end{figure}

\textbf{\textit{Solution.}}  Let $S$ be the center of the inscribed circle of the triangle $ABC$ (Figure \ref{sl.plo.8.3.1.pic}).

By  \ref{ploscGlavniIzrek} \textit{4)}, \ref{ploscDaljice} and \ref{PloscTrik} we have:
\begin{eqnarray*}
 p_{ABC}=p_{SBC}+p_{ASC}+p_{ABS}&=&
 \frac{a\cdot r}{2}+\frac{b\cdot r}{2}+\frac{c\cdot r}{2}=\\
&=&
 \frac{a+b+c}{2}\cdot r=sr,
\end{eqnarray*}
 which was to be proven. \kdokaz


        \bizrek \label{PloscTrikPricrt}  If $p_\triangle$ is the
        area of the triangle $ABC$ with
        the semi-perimeter $s=\frac{a+b+c}{2}$ and the radius of the circumscribed circle
        $r_a$, then
        $$p_\triangle=(s-a)r_a.$$
        \eizrek

\textbf{\textit{Proof.}} We use the notation from the big task \ref{velikaNaloga} (Figure \ref{sl.plo.8.3.2.pic}). From the same task we have: $AR=s-a$ and $AR_a=s$.

The right angled triangle $ARS$ and $AR_aS_a$ are similar (statement \ref{PodTrikKKK}), so $\frac{SR}{S_aR_a}=\frac{AR}{AR_a}$, i.e. $\frac{r}{r_a}=\frac{s-a}{s}$. If we use the previous statement \ref{PloscTrikVcrt}, we get:
\begin{eqnarray*}
 p_{ABC}=sr=(s-a)r_a,
\end{eqnarray*}
 which was to be proven. \kdokaz

\begin{figure}[!htb]
\centering
\input{sl.plo.8.3.2.pic}
\caption{} \label{sl.plo.8.3.2.pic}
\end{figure}


        \bizrek \label{PloscTrikHeron}  If $p_\triangle$
        is the area of the triangle $ABC$ with
        the semiperimeter $s=\frac{a+b+c}{2}$, then it holds
        \index{formula!Heronova}
        (Heron's\footnote{\index{Heron}
        \textit{Heron from Alexandria}(20--100),
        ancient Greek matemetician.} formula):
        $$p_\triangle=\sqrt{s(s-a)(s-b)(s-c)}.$$
        \eizrek


 \textbf{\textit{Proof.}} Again, we use the notation from the big task \ref{velikaNaloga} (Figure \ref{sl.plo.8.3.2.pic}). From the same task we have: $BP=s-b$ and $BP_a=s-c$.

The angle $SBP$ and $BS_aP_a$ are complementary, because they have a pair of perpendicular sides (statement \ref{KotaPravokKraki}). Therefore, the right angled triangle $SBP$ and $BS_aP_a$ are similar  (statement \ref{PodTrikKKK}), so $\frac{SP}{BP_a}=\frac{BP}{S_aP_a}$ i.e. $\frac{r}{s-c}=\frac{s-b}{r_a}$ i.e. $rr_a=(s-b)(s-c)$.
 If we use the statement \ref{PloscTrikVcrt} and \ref{PloscTrikPricrt}, we get:
\begin{eqnarray*}
p_{ABC}^2=sr(s-a)r_a=s(s-a)rr_a=s(s-a)(s-b)(s-c),
\end{eqnarray*}
 which was to be proven. \kdokaz

\bizrek \label{PloscTrikOcrt}  If $p_\triangle$ is the area of the triangle $ABC$ with the side lengths $a$, $b$ and $c$ and $R$ is the radius of the circumscribed circle, then the following is true:
$$p_\triangle=\frac{abc}{4R}.$$
\eizrek


\begin{figure}[!htb]
\centering
\input{sl.plo.8.3.3.pic}
\caption{} \label{sl.plo.8.3.3.pic}
\end{figure}


 \textbf{\textit{Proof.}} (Figure \ref{sl.plo.8.3.3.pic})

 The statement is a direct consequence of the formulas \ref{PloscTrik} and \ref{izrekSinusni}:
\begin{eqnarray*}
p_{ABC}=\frac{av_a}{2}=\frac{a}{2}\cdot v_a=\frac{a}{2}\cdot \frac{bc}{2R}=\frac{abc}{4R},
\end{eqnarray*}
 which was to be proven. \kdokaz


The previous formulas for the area of a triangle can be used to calculate the radius of the circumscribed, inscribed and excribed circles of a triangle as functions of its sides.

        \bzgled \label{PloscTrikOcrtVcrt}  If $R$ and $r$ are the radius of the circumscribed and inscribed circles, $r_a$, $r_b$ and $r_c$ are the radius of the excribed circles and
         $s=\frac{a+b+c}{2}$ is the perimeter
          of the triangle $ABC$,
         then the following is true:

 (i) $R=\frac{abc}{4\sqrt{s(s-a)(s-b)(s-c)}}$,\hspace*{7mm}
 (ii) $r=\sqrt{\frac{(s-a)(s-b)(s-c)}{s}}$,\\
 (iii) $r_a=\sqrt{\frac{s(s-b)(s-c)}{s-a}}$,\hspace*{2.7mm}
 (iv) $r_b=\sqrt{\frac{(s-a)s(s-c)}{s-b}}$,\hspace*{2.7mm}
 (v) $r_c=\sqrt{\frac{(s-a)(s-b)s}{s-c}}$.

        \ezgled



 \textbf{\textit{Proof.}}
 The statement is a direct consequence of the formulas \ref{PloscTrikOcrt}, \ref{PloscTrikVcrt}, \ref{PloscTrikPricrt} and \ref{PloscTrikHeron}.
\kdokaz

\bzgled \label{PloscTrikVcrtPricrt} If $r$ is the radius of the inscribed circle, $r_a$, $r_b$ and $r_c$ are the radii of the circumscribed circles, and $p_{\triangle}$ is the area of the triangle $ABC$, then:
$$p_{\triangle}=\sqrt{rr_ar_br_c}.$$
\ezgled

 \textbf{\textit{Proof.}}
 If we use the statements from the previous example \ref{PloscTrikOcrtVcrt}, we get:
 \begin{eqnarray*}
rr_ar_br_c=s(s-a)(s-b)(s-c)=p_{\triangle}^2,
\end{eqnarray*}
which was to be proven. \kdokaz

            \bizrek \label{ploscTrikPedalni}
            Let $A'B'C'$ be the pedal triangle of the obtuse triangle $ABC$, $s'$ be the perimeter
             of the triangle $A'B'C'$, $R$ be the radius of the circumscribed circle, and $p_{\triangle}$ be the area
             of the triangle $ABC$, then:
             $$p_{\triangle} = s'R.$$
            \eizrek



\begin{figure}[!htb]
\centering
\input{sl.plo.8.3.3a.pic}
\caption{} \label{sl.plo.8.3.3a.pic}
\end{figure}


 \textbf{\textit{Proof.}} We mark with $O$ the center of the circumscribed circle of the triangle $ABC$ (Figure \ref{sl.plo.8.3.3a.pic}). By the statement \ref{PedalniLemaOcrtana}, $OA\perp B'C'$, so from the statement \ref{ploscStirikPravok} it follows that $p_{AC'OB'}=\frac{|AO|\cdot |B'C'|}{2}=\frac{R\cdot |B'C'|}{2}$. Analogously, $p_{BA'OC'}=\frac{R\cdot |A'C'|}{2}$ and $p_{CB'OA'}=\frac{R\cdot |A'B'|}{2}$. Because $ABC$ is an obtuse triangle, the point $O$ is in its interior (see section \ref{odd3ZnamTock}), so by the statements \ref{ploscGlavniIzrek} \textit{4)} and \ref{ploscDaljice}, it holds:
\begin{eqnarray*}
p_{\triangle}&=&p_{AC'OB'}+p_{BA'OC'}+p_{CB'OA'}=\\
&=&\frac{R}{2}\left(|B'C'|+|A'C'|+|A'B'| \right)=s'R,
\end{eqnarray*}
 which was to be proven. \kdokaz

\bzgled \label{ploscTrikPedalni1}
                    Let $R$ and $r$ be the radii of the circumscribed and inscribed circles of the acute angled triangle $ABC$ and $s$ and $s'$ be the semi-perimeter of this triangle and its pedal triangle.
                    Prove that:
                    $$\frac{R}{r}=\frac{s}{s'}.$$
                \ezgled

\textbf{\textit{Proof.}} By the formulas \ref{PloscTrikVcrt} and \ref{ploscTrikPedalni} we have:
\begin{eqnarray*}
p_{ABC}=sr=s'R.
\end{eqnarray*}
From this we obtain the desired relation.
\kdokaz

In a special case we obtain formulas for the area of a right angled and an equilateral triangle.

            \bzgled \label{PloscTrikPravokotni} If $p_\triangle$ is the area
           of the right angled triangle $ABC$ with the sides  of lengths $a$ and $b$, then:
           $$p_\triangle=\frac{ab}{2}.$$
           \ezgled


\begin{figure}[!htb]
\centering
\input{sl.plo.8.3.4a.pic}
\caption{} \label{sl.plo.8.3.4a.pic}
\end{figure}


 \textbf{\textit{Proof.}} (Figure \ref{sl.plo.8.3.4a.pic})
       The statement is a direct consequence of the formula \ref{PloscTrik}, since in this case $v_a=b$.
\kdokaz

            \bizrek \label{PloscTrikEnakostr} If $p_\triangle$ is the area
           of the equilateral triangle $ABC$ with the side  of length $a$, then:
           $$p_\triangle=\frac{a^2\sqrt{3}}{4}.$$
           \eizrek


\begin{figure}[!htb]
\centering
\input{sl.plo.8.3.4.pic}
\caption{} \label{sl.plo.8.3.4.pic}
\end{figure}

\textbf{\textit{Proof.}} (Figure \ref{sl.plo.8.3.4.pic})
The statement is a direct consequence of theorems \ref{PloscTrik} and \ref{PitagorovEnakostr}:
\begin{eqnarray*}
p_{ABC}=\frac{av_a}{2}=\frac{a}{2}\cdot v_a=\frac{a}{2}\cdot \frac{a\sqrt{3}}{2}=\frac{a^2\sqrt{3}}{4},
\end{eqnarray*}
which had to be proven. \kdokaz

We will continue using the formulas for the area of a triangle.





                \bzgled \label{CarnotOcrtLema}
                Let $P$ be an inner point of the triangle $ABC$ and $x$, $y$ and $z$ the distances of this point from the sides $BC$, $AC$ and $AB$ with lengths $a$, $b$ and $c$. If $r$ is the radius of the inscribed circle of this triangle, then:
                $$xa+yb+zc=2\cdot p_{\triangle ABC}=r(a+b+c).$$
                \ezgled

\begin{figure}[!htb]
\centering
\input{sl.plo.8.3.6.pic}
\caption{} \label{sl.plo.8.3.6.pic}
\end{figure}

\textbf{\textit{Proof.}}   (Figure \ref{sl.plo.8.3.6.pic})

$x$, $y$ and $z$ are the lengths of the altitudes of the triangles $PBC$, $APC$ and $ABP$, so
according to theorems \ref{PloscTrik} and \ref{PloscTrikVcrt} it follows:
 \begin{eqnarray*}
 xa+yb+zc &=& 2\cdot p_{\triangle PBC}+2\cdot p_{\triangle APC}+2\cdot p_{\triangle ABP}\\
  &=& 2\cdot p_{\triangle ABC}\\
   &=& r(a+b+c),
 \end{eqnarray*}
 which had to be proven. \kdokaz


                \bzgled \index{izrek!Vivianijev} (Vivianijev\footnote{\index{Viviani, V.}\textit{V. Viviani} (1622--1703),
            Italian mathematician and physicist.} izrek)
                Let $P$ be an inner point of the equilateral triangle $ABC$ and $x$, $y$ and $z$ the distances of this point from the sides $BC$, $AC$ and $AB$. If $v$ is the height of this triangle, then:
                $$x+y+z=v.$$
                \ezgled

\begin{figure}[!htb]
\centering
\input{sl.plo.8.3.6a.pic}
\caption{} \label{sl.plo.8.3.6a.pic}
\end{figure}

\textbf{\textit{Proof.}}
By the statement from the previous example \ref{CarnotOcrtLema} and the formula \ref{PloscTrik}:
$xa+ya+za=2\cdot p_{\triangle ABC}=v_aa$.
\kdokaz

                \bizrek \label{CarnotOcrt}\index{izrek!Carnotov o očrtani krožnici}(Carnotov\footnote{\index{Carnot, L. N. M.}\textit{L. N. M. Carnot} (1753--1823), francoski matematik.} izrek o očrtani krožnici.)
                The sum of the distances from the center of the circumscribed circle to the sides of the triangle is equal to the sum of the radii of the circumscribed and inscribed circle of that triangle. So if $l(O,R)$ is the circumscribed and $k(S,r)$ the inscribed circle and $A_1$, $B_1$ and $C_1$ are the centers of the sides of the triangle $ABC$, it holds:
                $$|OA_1|+|OB_1|+|OC_1|=R+r.$$

                \eizrek

\begin{figure}[!htb]
\centering
\input{sl.plo.8.3.7.pic}
\caption{} \label{sl.plo.8.3.7.pic}
\end{figure}

\textbf{\textit{Proof.}} We mark $x=|OA_1|$, $y=|OB_1|$ and $z=|OC_1|$  (Figure \ref{sl.plo.8.3.7.pic}).  According to the statement from the example \ref{CarnotOcrtLema} it is:
\begin{eqnarray} \label{eqnCarnotOcrt1}
xa+yb+zc=r(a+b+c)
\end{eqnarray}
 From the similarity of triangles $BA_1O$ and $CA_1O$ (\textit{SAS} \ref{SKS}) and \ref{SredObodKot} it follows that $\angle BAC=\frac{1}{2}\angle BOC=\angle BOA_1$. This means that $\triangle OA_1B\sim \triangle AB'B \sim \triangle AC'C$ (\ref{PodTrikKKK}) is valid, therefore:
 \begin{eqnarray*}
 \frac{OA_1}{AB'}=\frac{OB}{AB} \hspace*{2mm} \textrm{ and } \hspace*{2mm}
 \frac{OA_1}{AC'}=\frac{OB}{AC}
\end{eqnarray*}
or:
\begin{eqnarray*}
 cx=R\cdot |AB'| \hspace*{2mm} \textrm{ and } \hspace*{2mm}
 bx=R\cdot |AC'|
\end{eqnarray*}
After adding the last two equations we get:
\begin{eqnarray*}
 (b+c)x=R(|AB'|+|AC'|)
\end{eqnarray*}
and similarly:
\begin{eqnarray*}
 (a+c)y=R\cdot (|BA'|+|BC'|)\\
 (a+b)z=R\cdot (|CA'|+|CB'|).
\end{eqnarray*}

After adding the last three relations we get:
\begin{eqnarray} \label{eqnCarnotOcrt2}
(b+c)x+(a+c)y+(a+b)z=R(a+b+c).
\end{eqnarray}
If we finally add the equality \ref{eqnCarnotOcrt1} and \ref{eqnCarnotOcrt2} and divide the obtained equality by $a+b+c$, we get:
\begin{eqnarray*}
x+y+z=r+R,
\end{eqnarray*}
 which had to be proven. \kdokaz


            \bnaloga\footnote{12. IMO, Hungary - 1970, Problem 1.}
             Let $M$ be a point on the side $AB$ of triangle $ABC$. Let $r_1$, $r_2$ and $r$ be the radii
            of the inscribed circles of triangles $AMC$, $BMC$ and $ABC$. Let $q_1$, $q_2$ and $q$
            be the radii of the escribed circles of the same triangles that lie in the angle
            $ACB$. Prove that
            $$\frac{r_1}{q_1}\cdot\frac{r_2}{q_2}=\frac{r}{q}.$$
            \enaloga

\begin{figure}[!htb]
\centering
\input{sl.plo.8.3.IMO1.pic}
\caption{} \label{sl.plo.8.3.IMO1.pic}
\end{figure}

\textbf{\textit{Solution.}} First, we mark $a=|BC|$,  $b=|AC|$,
$c=|AB|$,  $x=|AM|$, $y=|BM|$ and $m=|CM|$  (Figure
\ref{sl.plo.8.3.IMO1.pic}).

 If we use the expression \ref{PloscTrikVcrt} and
 \ref{PloscTrikPricrt}, we get:
  \begin{eqnarray*}
  && p_{\triangle ABC}=\frac{a+b+x+y}{2}\cdot r=\frac{a+b-x-y}{2}\cdot q\\
  && p_{\triangle AMC}=\frac{b+m+x}{2}\cdot r_1=\frac{b+m-x}{2}\cdot q_1\\
  && p_{\triangle ABC}=\frac{a+m+y}{2}\cdot r_2=\frac{a+m-y}{2}\cdot q_2
  \end{eqnarray*}
It follows that:
  \begin{eqnarray*}
  \frac{r}{q}=\frac{a+b+x+y}{a+b-x-y},\hspace*{6mm}
   \frac{r_1}{q_1}=\frac{b+m+x}{b+m-x},\hspace*{6mm}
   \frac{r_2}{q_2}=\frac{a+m+y}{a+m-y}
  \end{eqnarray*}
Therefore, the relation
$\frac{r_1}{q_1}\cdot\frac{r_2}{q_2}=\frac{r}{q}$, which we want to
prove, is equivalent to:
$$(a+b+x+y)(a+m-y)(b+m-x)=(a+b-x-y)(a+m+y)(b+m+x),$$
which, after rearranging, is equivalent to:
$$(a+b)(a+m)x+(a+b)(b+m)y=c(a+m)(b+m)+(x+y)xy,$$
or, if we take into account $x+y=c$ and rearrange further, to:
$$m^2=a^2\frac{x}{c}+b^2\frac{y}{c}-xy.$$
 The last relation is true, because it represents Stewart's theorem
 \ref{StewartIzrek} for the triangle $ABC$ and the distance $CM$.
 \kdokaz


%________________________________________________________________________________
 \poglavje{Area of Polygons} \label{odd8PloVeck}

The area of any polygon is obtained by dividing it with its diagonals into a union of triangles (Figure \ref{sl.plo.8.4.1.pic}). According to the expressions \ref{ploscGlavniIzrek} and \ref{ploscDaljice}, the area of this polygon is equal to the sum of the areas of these triangles.



\begin{figure}[!htb]
\centering
\input{sl.plo.8.4.1.pic}
\caption{} \label{sl.plo.8.4.1.pic}
\end{figure}

The following theorem applies specifically to a regular hexagon.

\bizrek
If $p$ is the area and $a$ is the length of a side of a regular hexagon, then:
$$p=\frac{3\sqrt{3}\cdot a^2}{2}.$$
\eizrek


\begin{figure}[!htb]
\centering
\input{sl.plo.8.4.2.pic}
\caption{} \label{sl.plo.8.4.2.pic}
\end{figure}

 \textbf{\textit{Proof.}} (Figure \ref{sl.plo.8.4.2.pic})

The theorem is a direct consequence of formulas \ref{ploscGlavniIzrek}, \ref{ploscDaljice} \ref{PloscTrikEnakostr} and the fact that a regular hexagon can be divided into six regular triangles (section \ref{odd3PravilniVeck}).
\kdokaz

A similar formula to that in the case of triangles applies for the area of tangent polygons (formula \ref{PloscTrikVcrt}).

            \bizrek \label{ploscTetVec}
            If: $s$ is the circumference, $k(S,r)$ is the inscribed circle of the tangent polygon
             $A_1A_2\ldots A_n$ and $p$
            is its area, then $$p = sr.$$
            \eizrek


\begin{figure}[!htb]
\centering
\input{sl.plo.8.4.3.pic}
\caption{} \label{sl.plo.8.4.3.pic}
\end{figure}

 \textbf{\textit{Proof.}} (Figure \ref{sl.plo.8.4.3.pic})

The triangles $A_1SA_2$, $A_2SA_3$, ... , $A_{n-1}SA_n$ and $A_nSA_1$ all have the same length of altitude from the vertex $S$, which is equal to $r$. If we use formulas \ref{ploscGlavniIzrek} \textit{4)}, \ref{ploscDaljice} and \ref{PloscTrik}, we get:
\begin{eqnarray*}
 p &=& p_{A_1SA_2}+p_{A_2SA_3}+\cdots +p_{A_{n-1}SA_n} +p_{A_nSA_1}=\\
 &=& \frac{|A_1A_2|\cdot r}{2}+\frac{|A_2A_3|\cdot r}{2}+\cdots+
\frac{|A_{n-1}A_n|\cdot r}{2}+\frac{|A_n A_1|\cdot r}{2}=\\
&=&sr,
\end{eqnarray*}
which is what needed to be proven. \kdokaz

\bnaloga\footnote{30. IMO, Germany  - 1989, Problem 2.}
             Let $S$, $S_a$, $S_b$ and $S_c$ be incentre and excentres of an acute-angled triangle $ABC$.
                $N_a$, $N_b$ and $N_c$ are midpoints of line segments $SS_a$, $SS_b$ and $SS_c$, respectively.
            Let $\mathcal{S}_{X_1X_2\ldots X_n}$ be the area of a polygon $X_1X_2\ldots X_n$.
            Prove that:
              $$\mathcal{S}_{S_aS_bS_c} =
                2\mathcal{S}_{AN_cBN_aCN_b} \geq 4\mathcal{S}_{ABC}.$$
            \enaloga

\begin{figure}[!htb]
\centering
\input{sl.plo.8.3.IMO2.pic}
\caption{} \label{sl.plo.8.3.IMO2.pic}
\end{figure}

\textbf{\textit{Solution.}}  Let $l$ be the circle circumscribed
about the triangle $ABC$ (Figure \ref{sl.plo.8.3.IMO2.pic}). By the
great theorem (\ref{velikaNaloga}) the point $N_a$ is the centre
of that arc $BC$ of the circle $l$, on which the point $A$ does not
lie - we denote this arc with $l_{BC}$. Similarly, $N_b$ and $N_c$
are the centres of the corresponding arcs $AC$ and $AB$ on the
circle $l$.

 First, from $SN_a\cong N_aS_a$ it follows that
  $\mathcal{S}_{SCN_a}=\mathcal{S}_{N_aCS_a}$ or
  $\mathcal{S}_{SCS_a}=2\cdot\mathcal{S}_{SCN_a}$. If we use
  analogous equalities and take into account that $S$ is an inner point of the triangle
  $S_aS_bS_c$, we get:
 \begin{eqnarray*}
\mathcal{S}_{S_aS_bS_c} &=&
\mathcal{S}_{SCS_a}+\mathcal{S}_{SBS_a}+
\mathcal{S}_{SCS_b}+\mathcal{S}_{SAS_b}+
\mathcal{S}_{SAS_c}+\mathcal{S}_{SBS_c}=\\
 &=&
2\cdot\left(\mathcal{S}_{SCN_a}+\mathcal{S}_{SBN_a}+
\mathcal{S}_{SCN_b}+\mathcal{S}_{SAN_b}+
\mathcal{S}_{SAN_c}+\mathcal{S}_{SBN_c}\right)=\\
&=& 2\mathcal{S}_{AN_cBN_aCN_b}.
 \end{eqnarray*}

We will now prove that $2\mathcal{S}_{AN_cBN_aCN_b} \geq 4\mathcal{S}_{ABC}$, or $\mathcal{S}_{AN_cBN_aCN_b} \geq 2\mathcal{S}_{ABC}$. Let $V$ be the altitude point of triangle $ABC$ and $V_a=S_{BC}(V)$. Triangle $ABC$ is acute by assumption, so point $V$ lies within it. By \ref{TockaV'} point $V_a$ lies on the arc $l_{BC}$. We have already mentioned that $N_a$ is the center of this arc, so the altitude of triangle $BNC$ is longer than the altitude of triangle $BVC$ by the same base $BC$. It follows that $\mathcal{S}_{BNC}\geq \mathcal{S}_{BVC}$. Similarly, we have $\mathcal{S}_{ANC}\geq \mathcal{S}_{AVC}$ and $\mathcal{S}_{ANB}\geq \mathcal{S}_{AVB}$. Since $V$ is an internal point of triangle $ABC$, we have:
 \begin{eqnarray*}
\mathcal{S}_{AN_cBN_aCN_b}&=&\mathcal{S}_{ABC}+
\mathcal{S}_{BNC}+\mathcal{S}_{ANC}+\mathcal{S}_{ANB}\geq\\
 &\geq& \mathcal{S}_{ABC}+
\mathcal{S}_{BVC}+\mathcal{S}_{AVC}+\mathcal{S}_{AVB}=\\
&=& 2\mathcal{S}_{ABC},
 \end{eqnarray*}
 which was to be proven.  \kdokaz



%________________________________________________________________________________
\poglavje{Koch Snowflake} \label{odd8PloKoch}

In this section we will consider a figure which is bounded (a subset of a circle), has a finite area, but its perimeter is infinite.

First, we will define a special type of polygon. An equilateral triangle $ABC$ with a side length $a=a_0$ is denoted by $\mathcal{K}_0(a)$. The polygon $\mathcal{K}_1(a)$ is obtained if each side of the triangle $\mathcal{K}_0(a)$ is divided into three equal parts and an equilateral triangle with a side length $a_1=\frac{1}{3}\cdot a_0$ is drawn over the middle. The process is continued and the polygon $\mathcal{K}_n(a)$ is obtained if each side of the triangle $\mathcal{K}_{n-1}(a)$ is divided into three equal parts and an equilateral triangle with a side length $a_n=\frac{1}{3}\cdot a_{n-1}$ is drawn over the middle (Figure \ref{sl.plo.8.5.Koch1.pic}).


\begin{figure}[!htb]
\centering
\input{sl.plo.8.5.Koch1.pic}
\caption{} \label{sl.plo.8.5.Koch1.pic}
\end{figure}

 If the described process is continued to infinity, we get the so-called
\index{Kochova snežinka} \pojem{Koch's\footnote{This figure was defined by the Swedish mathematician \index{Koch, H.}\textit{Helge von Koch} (1870--1924) in 1904. Today, Koch's snowflake is classified
as a \textit{fractal}.} snowflake} $\mathcal{K}(a)$.

First, let's calculate the perimeter of the polygon $\mathcal{K}_n(a)$.

From the process by which we obtained the polygon $\mathcal{K}_n(a)$, it is clear that the polygon $\mathcal{K}_n(a)$ has four times as many sides as $\mathcal{K}_{n-1}(a)$ (each side of the polygon $\mathcal{K}_{n-1}(a)$ is replaced by four sides of the polygon $\mathcal{K}_n(a)$). We denote by $s_n$ the number of sides of the polygon $\mathcal{K}_n(a)$. Obviously, $s_n$ is a geometric sequence with a quotient of $q=4$ and an initial term of $s_0=3$, so:
\begin{eqnarray} \label{eqnKoch1}
s_n=3\cdot 4^n
\end{eqnarray}
Each side $a_n$ of the polygon $\mathcal{K}_n(a)$ is, by construction, three times shorter than the side $a_{n-1}$ of the polygon $\mathcal{K}_{n-1}(a)$. So we have a geometric sequence $a_n$ again with a quotient of $q=\frac{1}{3}$ and an initial term of $a_0=a$, so:
\begin{eqnarray} \label{eqnKoch2}
a_n=\left( \frac{1}{3} \right)^n\cdot a
\end{eqnarray}
Because for the perimeter $o_n$ of the polygon $\mathcal{K}_n(a)$ it holds that $o_n=s_n\cdot a_n$, from \ref{eqnKoch1} and \ref{eqnKoch2} we get:
\begin{eqnarray} \label{eqnKoch3}
o_n=3\cdot \left( \frac{4}{3} \right)^n\cdot a
\end{eqnarray}
If we want to obtain the perimeter $o$ of the Koch snowflake $\mathcal{K}(a)$, we calculate it as $o=\lim_{n\rightarrow\infty}o_n$, but from \ref{eqnKoch3} we get (because $\frac{4}{3}>1$):
\begin{eqnarray} \label{eqnKoch4}
o=\lim_{n\rightarrow\infty}o_n=\lim_{n\rightarrow\infty}3\cdot \left( \frac{4}{3} \right)^n\cdot a=3a\cdot\lim_{n\rightarrow\infty}\left( \frac{4}{3} \right)^n=\infty
\end{eqnarray}
which means that the Koch snowflake $\mathcal{K}(a)$ has no finite perimeter, that is, the perimeter $o_n$ of the polygon $\mathcal{K}_n(a)$ increases without bound as the number of steps $n$ increases.

Let's now calculate the area $p$ of the Koch snowflake $\mathcal{K}(a)$. We'll denote the area of the polygon $\mathcal{K}_n(a)$ with $p_n$.
The area $p_n$ is obtained by adding a certain number of smaller triangle areas to the area $p_{n-1}$. How many of these triangles are there? By the construction of the polygon $\mathcal{K}_n(a)$, we get it from the polygon $\mathcal{K}_{n-1}(a)$ by drawing a smaller triangle at each side of the polygon $\mathcal{K}_{n-1}(a)$. So the number of these triangles is equal to $s_{n-1}$. Because of this and the \ref{PloscTrikEnakostr} equation, we have:
\begin{eqnarray*}
p_n=p_{n-1}+s_{n-1}\cdot \frac{a_n^2\cdot \sqrt{3}}{4}=p_{n-1}
+\frac{3}{16}\cdot \left(\frac{4}{9} \right)^n\cdot a^2\sqrt{3}.
\end{eqnarray*}
The area $p$ of the Koch snowflake $\mathcal{K}(a)$ is therefore the infinite sum:
\begin{eqnarray*}
p&=&p_0+ \sum_{n=1}^{\infty}\frac{3}{16}\cdot \left(\frac{4}{9} \right)^n\cdot a^2\sqrt{3}=\\
&=&\frac{a^2\cdot \sqrt{3}}{4}+ \frac{3}{16} a^2\sqrt{3}\cdot\sum_{n=1}^{\infty} \left(\frac{4}{9}\right)^n=\\
&=&\frac{a^2\cdot \sqrt{3}}{4}+ \frac{3}{16} a^2\sqrt{3}\cdot\left(\frac{4}{9}+\left(\frac{4}{9}\right)^2+\cdots + \left(\frac{4}{9}\right)^n+\cdots \right).
\end{eqnarray*}
If we calculate the sum of an infinite geometric sequence and simplify it, we get:
\begin{eqnarray} \label{eqnKoch5}
p=\frac{2a^2\sqrt{3}}{5}.
\end{eqnarray}


%________________________________________________________________________________
\poglavje{Circumference and Area of a Circle} \label{odd8PloKrog}

%OBSEG

In section \ref{odd3NeenTrik} we have already defined the circumference of a polygon as the sum of all its sides. In this sense, the circumference represented a distance. In the following, we will understand the circumference also as the length of this distance, or the sum of the lengths of all the sides of the polygon. Here we will deal with the circumference of a circle. Although it is intuitively clear, this new concept needs to be defined first.

Let $k(S,r)$ be an arbitrary circle. The corresponding circle is denoted by $\mathcal{K}(S,r)$. The circumference of this circle intuitively represents the length of the circle $k$. Let $A_1A_2\ldots A_n$ ($n\in \mathbb{N}$ and $n\geq 3$) be a regular polygon inscribed in the circle $k$. The tangents of this circle at the vertices of the polygon $A_1A_2\ldots A_n$ determine the sides of the regular polygon $B_1B_2\ldots B_n$, which is circumscribed to the circle $k$ (Figure \ref{sl.plo.8.5.1.pic}).

\begin{figure}[!htb]
\centering
\input{sl.plo.8.5.1.pic}
\caption{} \label{sl.plo.8.5.1.pic}
\end{figure}

We denote by $\underline{o}_n$ the circumference of the polygon  $A_1A_2\ldots A_n$ and $\overline{o}_n$ the circumference of the polygon $B_1B_2\ldots B_n$. By the triangle inequality (statement \ref{neenaktrik}) we have $|A_1B_1|+|B_1A_2|>|A_1A_2|$, $|A_2B_2|+|B_2A_3|>|A_2A_3|$,..., $|A_nB_n|+|B_nA_1|>|A_nA_1|$ (Figure \ref{sl.plo.8.5.1a.pic}). If we add all these inequalities we get $\overline{o}_n>\underline{o}_n$. Similarly, $\underline{o}_n<\underline{o}_{n+1}$ and $\overline{o}_n>\overline{o}_{n+1}$. Therefore, for every $n\in \mathbb{N}$, $n\geq 3$ we have:
\begin{eqnarray*}
\underline{o}_1<\underline{o}_2<\cdots<
\underline{o}_n<\overline{o}_n<\cdots<\overline{o}_2<\overline{o}_1.
\end{eqnarray*}
 This means that $\underline{o}_n$ is an increasing sequence bounded from above by $\overline{o}_1$, so it is a known statement of mathematical analysis that it is convergent and has its limit:
\begin{eqnarray*}
\underline{o}=\lim_{n\rightarrow\infty}\underline{o}_n.
\end{eqnarray*}
 Similarly, $\overline{o}_n$ is a decreasing sequence bounded from below by $\underline{o}_1$, so it has its limit:
\begin{eqnarray*}
\overline{o}=\lim_{n\rightarrow\infty}\overline{o}_n.
\end{eqnarray*}

\begin{figure}[!htb]
\centering
\input{sl.plo.8.5.1a.pic}
\caption{} \label{sl.plo.8.5.1a.pic}
\end{figure}

We will not formally prove the intuitively clear fact that for a sufficiently large natural number $n$, the difference $\overline{o}_n-\underline{o}_n$ is arbitrarily small, i.e. $\underline{o}=\overline{o}$. \index{obseg!kroga}\pojem{Circumference of a circle} $\mathcal{K}$ with the designation $o$ then represents $o=\underline{o}=\overline{o}$.

We prove the following important statement.

                \bizrek
                If $o$ is the circumference of a circle with radius $r$, then the ratio
                $o:2r$
                 is constant and thus independent of the choice of the circle.
                \eizrek

\begin{figure}[!htb]
\centering
\input{sl.plo.8.5.1b.pic}
\caption{} \label{sl.plo.8.5.1b.pic}
\end{figure}


\textbf{\textit{Proof.}} (Figure \ref{sl.plo.8.5.1b.pic})

Let $\mathcal{K}(S,r)$ and $\mathcal{K}'(S',r')$ be two arbitrary circles with circumferences $o$ and $o'$, and $k(S,r)$ and $k'(S',r')$ the corresponding circles. According to Theorem \ref{RaztKroznKrozn1}, there exists a central stretch that maps the circle $k$ to the circle $k'$. According to Theorem \ref{RaztTransPod}, this central stretch is a similarity transformation (denoted by $f$) with some coefficient $\lambda$. First, $r':r=\lambda$ or:
\begin{eqnarray}
 r'=\lambda r. \label{eqnPloscKrogrr'oo'}
\end{eqnarray}
Let $A_1A_2\ldots A_n$ and $B_1B_2\ldots B_n$ ($n\in \mathbb{N}$ and $n\geq 3$) be inscribed and circumscribed regular polygons of the circle $k$ with circumferences $\underline{o}_n$ and $\overline{o}_n$. We denote:
$$f:S,A_1,A_2,\ldots,A_n,B_1,B_2,\ldots,B_n\mapsto S',A_1',A_2',\ldots,A_n',B_1',B_2',\ldots,B_n'.$$
The polygons $A_1'A_2'\ldots A_n'$ and $B_1'B_2'\ldots B_n'$ ($n\in \mathbb{N}$ and $n\geq 3$) are inscribed and circumscribed regular polygons of the circle $k$ and for their circumferences $\underline{o'}_n$ and $\overline{o'}_n$ it holds that $\underline{o'}_n=\lambda\underline{o}_n$ and $\overline{o'}_n=\lambda\overline{o}_n$. From this and from the definition of the circumference of a circle it follows:
\begin{eqnarray}\label{eqnPloscKrogrr'oo'1}
 o'=\underline{o'}=\lim_{n\rightarrow\infty}\underline{o'}_n=
\lim_{n\rightarrow\infty}\lambda\underline{o}_n=
\lambda\lim_{n\rightarrow\infty}\underline{o}_n=\lambda \underline{o}=\lambda o.
\end{eqnarray}
From \ref{eqnPloscKrogrr'oo'} and \ref{eqnPloscKrogrr'oo'1} it follows:
$$\frac{o'}{2r'}=\frac{\lambda o}{2\lambda r}=\frac{o}{2r},$$ which was to be proven. \kdokaz

The constant from the previous theorem is called
\index{number!$\pi$}\pojem{number $\pi$} (\index{Archimedes' constant}\pojem{Archimedes' constant} or \index{number!Ludolf's}\pojem{Ludolf's number}) and we also denote it by $\pi$.

From the previous theorem and from the definition of the number $\pi$ we get the following theorem.

\bizrek \label{obsegKtoznice}
If $o$ is the circumference of a circle with radius $r$, then:
$$o=2r\pi.$$
\eizrek

First, we will give a rough estimate of the number $\pi$.

\bizrek \label{stevPiOcena}
For the number $\pi$ it holds:
$$3<\pi<4.$$
\eizrek


\begin{figure}[!htb]
\centering
\input{sl.plo.8.5.2.pic}
\caption{} \label{sl.plo.8.5.2.pic}
\end{figure}


\textbf{\textit{Proof.}} (Figure \ref{sl.plo.8.5.2.pic})

Let $\mathcal{K}$ be an arbitrary circle with radius $r$ and circumference $o$, and $k$ the corresponding circle. By \ref{obsegKtoznice}, $o=2r\pi$.
We choose the regular inscribed polygon $n=6$ and the regular circumscribed polygon $n=4$. From the definition of the circumference of a circle it follows:
$$\underline{o}_6<o<\overline{o}_4,$$ therefore:
$$6r<2r\pi<8r.$$
From this, after dividing the inequality by $2r$, it follows that $3<\pi<4$.
\kdokaz

With different methods, we can more accurately determine the approximate value of the number $\pi$\footnote{The Sumerians (around 2000 BC) did not find a better approximation for the number $\pi$, than the later biblical $\pi \doteq 3$.\\
The Greek mathematician and philosopher \index{Archimedes}\textit{Archimedes of Syracuse} (287--212 BC) found the approximation $\pi \doteq 3.14185110664$ (that is, three decimal places exactly) around 230 BC in his work \textit{On the Measurement of the Circle}, by using the inscribed and circumscribed regular $n$-gons for $n\in\{6, 12, 24, 48, 96\}$, or $n=3\cdot 2^k$ ($k\in\{1,2,3,4,5\}$). We also call the number $\pi$ the \textit{Archimedean constant}.
The French mathematician  \index{Viète, F.} \textit{F. Viète}
(1540--1603) improved Archimedes' results and in 1579, with an $n$-gon with $n$ sides equal to $n = 393216 = 3\cdot 2^{17}$, he calculated the number $\pi$ to 9 decimal places exactly.\\
The development of calculus in the 17th century found new methods for even faster calculation of the number $\pi$. Thus, the Dutch-German mathematician \index{Ludolph van Ceulen} \textit{Ludolph van Ceulen} (1540--1610), who devoted a great part of his life to the approximate calculation of the number $\pi$, calculated 20 correct decimal places. This result was later improved to 35 decimal places. We also call the number $\pi$ the \textit{Ludolph number}.
 The Swiss mathematician \index{Euler, L.}\textit{L. Euler} (1707–-1783) calculated   128 decimal places.
 The Slovenian mathematician \index{Vega, J.}\textit{J. Vega} (1754–-1820) achieved the then world record and calculated the number $\pi$ to 140 decimal places.\\
 The English mathematician \index{Ferguson, D. F.}\textit{D. F. Ferguson} in 1946 used a computer to calculate 620 correct decimal places. On 2 August 2010, the Japanese systems engineer \index{Kondo, S.}\textit{S. Kondo} (1955-- ), with a self-made adapted personal computer, calculated 5,000,000,000,000 decimal places of the number $\pi$. The calculation, together with the verification, took 90 days.
}."

The number $\pi$ is an irrational number\footnote{This property was proven by the French mathematician \index{Lambert, J. H.}\textit{J.
H. Lambert} (1728--1777) in 1761. The number $\pi$ is actually \index{število!transcendentno}transcendent (there does not exist a polynomial with rational coefficients whose root is $\pi$), which was proven by the German mathematician \index{Lindemann, C. L. F.}\textit{C. L. F. Lindemann} (1852–-1939) in 1882.}, ($\pi\notin \mathbb{Q}$), so it cannot be written as a fraction. In decimal form, it has an infinite number of decimals without a period:
$$\pi= 3,14159 26535 89793 23846 26433 83279 50288 41971 69399 37510 58\ldots$$
Its approximate value is therefore:
$$\pi\doteq 3,14.$$

In addition to the most used approximation
there is an approximation $\pi\doteq \frac{22}{7}$ (also to two decimal places). A very good approximation (to six decimal places) is given by the fraction $355/113$ (we can remember it like this: write the number 113355, the numerator is determined by the last three digits of this number, the denominator by the first three).

The exact value of the number $\pi$ in the form of an infinite series is given by
\index{formula!Leibnizova}\pojem{Leibnizova\footnote{\index{Leibniz, G. W.}\textit{G. W. Leibniz} (1646--1716), German mathematician.} formula}:
$$\frac{\pi}{4}=\sum_{k=0}^{\infty}\frac{(-1)^k}{2k+1}$$
or
$$\pi=4\cdot\left(1-\frac{1}{3}+\frac{1}{5}-\frac{1}{7}+\frac{1}{9}-\frac{1}{11}
+\cdots\right),$$
and
 \index{formula!Ramanujanova }\pojem{Ramanujanova\footnote{\index{Ramanujan, S.}\textit{S. Ramanujan}  (1887--1920), Indian mathematician.} formula}:
$$\frac{1}{\pi}=
\frac{2\sqrt{2}}{9801}\sum_{k=0}^{\infty}
\frac{(4k)!\cdot(1103+26390k)}{(k!)\cdot396^{4k}}.$$

At this point, we will not formally introduce the concept of the length of a circular arc. The idea would be similar to the definition of the circumference of a circle. Without proof, we will also accept the fact, which is intuitively clear, that the length of a circular arc $l$ and the circumference $o$ of the corresponding circle are proportional to the corresponding central angle of that arc with a measure $\alpha$ (in degrees) and the full angle, which measures $360^0$ (Figure \ref{sl.plo.8.5.5a.pic}). According to the \ref{obsegKtoznice} therefore holds:
$l:o=\alpha:360^0$ or $l:2r\pi=\alpha:360^0$. It follows that the following statement.

                \bizrek \label{obsegKrozLok}
                If $l$ is the length of a circular arc with the corresponding central angle with a measure
                 (in degrees) $\alpha$ and
                radius $r$, then:
                    $$l=\frac{\pi r \alpha}{180^0}.$$
                \eizrek


\begin{figure}[!htb]
\centering
\input{sl.plo.8.5.5a.pic}
\caption{} \label{sl.plo.8.5.5a.pic}
\end{figure}




%PlOSCINA


In the following we will consider the area of a circle. We prove the first auxiliary statement.

                \bizrek \label{ploscKrogLema}
                If $\Phi_1$ and $\Phi_2$ are two figures in the same plane, then it holds:
                $$\Phi_1\subseteq \Phi_2\hspace*{1mm}\Rightarrow\hspace*{1mm}
                 p_{\Phi_1}\leq  p_{\Phi_2}.$$
                \eizrek


\begin{figure}[!htb]
\centering
\input{sl.plo.8.5.3a.pic}
\caption{} \label{sl.plo.8.5.3a.pic}
\end{figure}


\textbf{\textit{Proof.}} (Figure \ref{sl.plo.8.5.3a.pic})

Since $\Phi_1\subseteq \Phi_2$, it follows that $\Phi_2 = \Phi_1 \cup \left(\Phi_2\setminus \Phi_1\right)$ and $\Phi_1 \cap \left(\Phi_2\setminus \Phi_1\right)=\emptyset$ or $p_{\Phi_1 \cap \left(\Phi_2\setminus \Phi_1\right)}=0$. By \ref{ploscGlavniIzrek} \textit{4)} and \textit{2)}:
\begin{eqnarray*}
p_{\Phi_2} =  p_{\Phi_1}+ p_{\Phi_2\setminus \Phi_1}  \geq p_{\Phi_1},
\end{eqnarray*}
which had to be proven. \kdokaz

                \bizrek \label{ploscKrog}
                If $p_0$ is the area of the circle $\mathcal{K}$ with radius $r$, then
                $$p_0=r^2\pi.$$
                \eizrek


\begin{figure}[!htb]
\centering
\input{sl.plo.8.5.3.pic}
\caption{} \label{sl.plo.8.5.3.pic}
\end{figure}


\textbf{\textit{Proof.}} (Figure \ref{sl.plo.8.5.3.pic})

Let $\underline{\mathcal{V}}_n=A_1A_2\ldots A_n$ be a regular inscribed polygon and $\overline{\mathcal{V}}_n=B_1B_2\ldots B_n$ be a regular circumscribed polygon of a circle $k$, with $a_n$ and $b_n$ being the lengths of the sides of these two $n$-sided polygons.
By the previous statement (\ref{ploscKrogLema}), we have:
 \begin{eqnarray} \label{eqnPloscKrog1}
p_{\underline{\mathcal{V}}_n} \leq  p_0 \leq p_{\overline{\mathcal{V}}_n}.
\end{eqnarray}
 Let $S$ be the center of the circle $\mathcal{K}$. The altitude of the triangle $B_1SB_2$ from the vertex $S$ is equal to the radius $r$ of the circle $\mathcal{K}$, and with $v_n$ we denote the altitude of the triangle $A_1SA_2$ from the vertex $S$.
 By the statements \ref{ploscGlavniIzrek} 3) and 4) and \ref{PloscTrik}, we have:
\begin{eqnarray*}
p_{\underline{\mathcal{V}}_n} &=&n\cdot p_{A_1SA_2}=n\cdot \frac{a_nv_n}{2}=\frac{1}{2}na_nv_n=\frac{1}{2}\underline{o}_nv_n;\\
p_{\overline{\mathcal{V}}_n} &=&n\cdot p_{B_1SB_2}=n\cdot \frac{b_nr}{2}=\frac{1}{2}nb_nr=\frac{1}{2}\overline{o}_nr.
\end{eqnarray*}
From \ref{eqnPloscKrog1} and the two previous relations, we obtain:
 \begin{eqnarray} \label{eqnPloscKrog2}
\frac{1}{2}\underline{o}_nv_n \leq  p_0 \leq \frac{1}{2}\overline{o}_nr.
\end{eqnarray}
 Because this relation is true for every $n\in \mathbb{N}$ and
 $$\lim_{n\rightarrow\infty} \underline{o}_n=\lim_{n\rightarrow\infty} \overline{o}_n=o,$$ where $o$ is the circumference of the circle $\mathcal{K}$, and $$\lim_{n\rightarrow\infty} v_n=r,$$ from \ref{eqnPloscKrog2} we obtain:
\begin{eqnarray*}
\frac{1}{2}or \leq  p_0 \leq \frac{1}{2}or,
\end{eqnarray*}
or (if we use the statement \ref{obsegKtoznice} as well):
\begin{eqnarray*}
  p_0 = \frac{1}{2}or=\frac{1}{2}2r\pi r=r^2\pi,
\end{eqnarray*}
 which is what had to be proven. \kdokaz

At this point, we will define a new term. Let $\mathcal{K}_1(S,r_1)$ and $\mathcal{K}_2(S,r_2)$ ($r_2>r_1$) be two circles with the same center (the corresponding circles $k_1$ and $k_2$ are concentric). The set $\mathcal{K}_2\setminus \mathcal{K}_1$ is called
\index{krožni!kolobar}\pojem{circular sector}.


                \bizrek \label{ploscKrozKolob}
                If $p_{kl}$ is the area of the circular sector determined by the circles $k_1(S,r_1)$
                and  $k_2(S,r_2)$ ($r_2>r_1$), then:
                $$p_{kl}=\left( r_2^2-r_1^2 \right)\cdot\pi.$$
                \eizrek


\begin{figure}[!htb]
\centering
\input{sl.plo.8.5.4.pic}
\caption{} \label{sl.plo.8.5.4.pic}
\end{figure}


\textbf{\textit{Proof.}} (Figure \ref{sl.plo.8.5.4.pic})

We mark with  $\mathcal{K}_1(S,r_1)$ and  $\mathcal{K}_2(S,r_2)$ the corresponding circles of the circles  $k_1(S,r_1)$ and  $k_2(S,r_2)$.

From $\mathcal{K}_2=\mathcal{K}_2\setminus \mathcal{K}_1\cup \mathcal{K}_1 $  and $\left( \mathcal{K}_2\setminus \mathcal{K}_1 \right) \cap \mathcal{K}_1=\emptyset$ according to the formulas \ref{ploscGlavniIzrek} \textit{4)} and \ref{ploscKrog} we get
 $p_{\mathcal{K}_2}=p_{\mathcal{K}_2\setminus \mathcal{K}_1} + p_{\mathcal{K}_1}$ i.e.:
$$p_{kl}=p_{\mathcal{K}_2\setminus \mathcal{K}_1}= p_{\mathcal{K}_2}- p_{\mathcal{K}_1}=r_2^2\pi-r_1^2\pi
=\left( r_2^2-r_1^2 \right)\cdot\pi,$$ which had to be proven. \kdokaz

In a similar way as for the length of the circular arc (formula \ref{obsegKrozLok}) we get the following statement.

\bizrek \label{ploscKrozIzsek}
If $p_i$ is the area of the circular section with the corresponding central angle with a measure (in degrees) $\alpha$ and with a radius $r$, then:
                    $$p_i=\frac{r^2\pi\cdot\alpha}{360^0}.$$
                \eizrek


\begin{figure}[!htb]
\centering
\input{sl.plo.8.5.5.pic}
\caption{} \label{sl.plo.8.5.5.pic}
\end{figure}



                \bzgled
                If in the formula for the area of the circle we replace the number $\pi$ with its approximate value $3$, we get the formula for the area of the inscribed regular dodecagon\footnote{This task was solved by the Chinese mathematician
                \index{Liu, H.}\textit{H. Liu} (3rd century)}.
                \ezgled


\begin{figure}[!htb]
\centering
\input{sl.plo.8.5.6.pic}
\caption{} \label{sl.plo.8.5.6.pic}
\end{figure}


\textbf{\textit{Proof.}} (Figure \ref{sl.plo.8.5.6.pic})

Let $A_1A_2\ldots A_{12}$ be a regular dodecagon, which is inscribed in the circle $k(S,r)$. Then $A_1A_3\ldots A_{11}$ is a regular hexagon, which is inscribed in the same circle. We denote by $p_{12}$ the area of the mentioned dodecagon. By \ref{ploscGlavniIzrek} \textit{4)} we have $p_{12}=12\cdot p_{SA_1A_2}$.

Since the triangle $A_1SA_3$ is a right triangle, the height $A_1P$
 of the triangle $SA_1A_2$ is equal to half of the side $A_1A_3$
of the hexagon $A_1A_3\ldots A_{11}$, that is:
  $$|A_1P|=\frac{1}{2}\cdot |A_1A_3|=\frac{r}{2}.$$
Then, by \ref{PloscTrik} we have:
 $$p_{12}=12\cdot p_{SA_1A_2}=12\cdot\frac{|SA_2|\cdot |A_1P|}{2}=12\cdot\frac{r^2}{4}=3\cdot r^2,$$ which had to be proven. \kdokaz



                \bzgled \label{HipokratoviLuni}
                Above the catheti of a right triangle we construct circles
                    and from the thus enlarged triangle we cut out a circle above the hypotenuse
                     (Figure \ref{sl.plo.8.5.7.pic}). Show that the area of the remainder
                    is equal to the area of the original
                    right triangle\footnote{A special claim (the next example in this section), if the given triangle is isosceles, was proven by \index{Hipokrat}\textit{Hipokrat from Kios} (5th century BC), an ancient
Greek mathematician. The aforementioned figure is called \index{Hipokratovi luni}\pojem{Hipokratovi luni}. Hipokrat was the first to discover that some figures with a curved edge can be converted into a square with a straight edge and a hexagon.}.
                \ezgled


\begin{figure}[!htb]
\centering
\input{sl.plo.8.5.7.pic}
\caption{} \label{sl.plo.8.5.7.pic}
\end{figure}

\textbf{\textit{Solution.}}
Let $ABC$ be the aforementioned triangle with hypotenuse $c$ and sides $a$ and $b$.
Let $p$ be the desired area, $p_{\triangle}$ the area of the triangle $ABC$, and $p_a$, $p_b$, and $p_c$ the areas of the corresponding semicircles. By the formulas \ref{ploscGlavniIzrek} \textit{4)}, \ref{ploscKrozIzsek} and \ref{PloscTrik} and the Pythagorean theorem \ref{PitagorovIzrek}, we have:
 \begin{eqnarray*}
 p&=& p_a+p_b-p_c+p_{\triangle}=\\
    &=& \left(\frac{a}{2} \right)^2\cdot \pi
+\left(\frac{a}{2} \right)^2\cdot \pi-\left(\frac{a}{2} \right)^2\cdot \pi+p_{\triangle}=\\
       &=& \frac{\pi}{4}\cdot \left(a^2+b^2-c^2 \right)+p_{\triangle}=\\
   &=& \frac{\pi}{4}\cdot 0+p_{\triangle}=\\
   &=& p_{\triangle},
 \end{eqnarray*}
 which had to be proven. \kdokaz

                \bzgled \label{HipokratoviLuni2}
                The day is the square $PQRS$.
                Draw a square that has the same area as the figure that results when
                the circle drawn around the square $PQRS$
                is cut by
                a circle with
                radius $PQ$
                (Figure \ref{sl.plo.8.5.8.pic}).
                \ezgled


\begin{figure}[!htb]
\centering
\input{sl.plo.8.5.8.pic}
\caption{} \label{sl.plo.8.5.8.pic}
\end{figure}

\textbf{\textit{Solution.}}
If we choose the isosceles right triangle $QSQ'$ (where $Q'=\mathcal{S}_p(Q)$), by the previous formula (\ref{HipokratoviLuni}) the area of the desired figure is equal to the area of the isosceles right triangle $SPQ$ (Figure \ref{sl.plo.8.5.8.pic}), and its area
 is equal to the area of the square $PLSN$, where $L$ is the center of the line $SQ$ and $N=\mathcal{S}_{SP}(L)$
 (Figure \ref{sl.plo.8.5.8a.pic}).

\begin{figure}[!htb]
\centering
\input{sl.plo.8.5.8a.pic}
\caption{} \label{sl.plo.8.5.8a.pic}
\end{figure}
\kdokaz


        \bzgled
           Let $ABCD$ be a square which we increase by four semicircles above the sides, and from the resulting figure we cut out the circle circumscribed to the square  (Figure \ref{sl.plo.8.5.9.pic}). Prove that
        the area of the remainder is equal to the area of the original square.
        \ezgled


\begin{figure}[!htb]
\centering
\input{sl.plo.8.5.9.pic}
\caption{} \label{sl.plo.8.5.9.pic}
\end{figure}

\textbf{\textit{Proof.}} Let $S$ be the center of the square $ABCD$.
The statement is a direct consequence of the statement of the previous theorem (\ref{HipokratoviLuni2}) - we use the appropriate relation for the triangles $ASB$, $BSC$, $CSD$ and $DSA$ and add everything up.
\kdokaz


            \bzgled \label{arbelos}
            Let $C$ be an arbitrary point on the diameter $AB$ of the semicircle $k$ and let $D$ be the point of intersection of this
            semicircle with the rectangle with diameter $AB$ at point $C$. Let $j$ and $l$
            be semicircles with diameters $AC$ and $BC$. Prove that the area of the figure bounded by semicircles $k$, $l$ and $j$ is equal to the area of the circle with diameter
            $CD$\footnote{This task was published in his 'Book of Lemmas' by the Greek mathematician \index{Arhimed}\textit{Archimedes of Syracuse} (3rd century BC).
He named the aforementioned figure \index{arbelos}\textit{arbelos} (which in Greek means 'shoemaker's knife').}.
            \ezgled


\begin{figure}[!htb]
\centering
\input{sl.plo.8.5.10.pic}
\caption{} \label{sl.plo.8.5.10.pic}
\end{figure}

\textbf{\textit{Proof.}}  (Figure \ref{sl.plo.8.5.10.pic})

Let's mark $a=|AC|$ and $b=|BC|$ the appropriate diameters, $p$ the area of the aforementioned figure and $p_0$ the area of the circle with diameter $|CD|$. According to \ref{izrekVisinski}
it is $|CD|^2 = ab$. If we use \ref{ploscKrozIzsek} and \ref{ploscKrog} as well,
we get:
\begin{eqnarray*}
p&=&\frac{1}{2}\left(\frac{a+b}{2}\right)^2\cdot\pi-
\frac{1}{2}\left(\frac{a}{2}\right)^2\cdot\pi-
\frac{1}{2}\left(\frac{b}{2}\right)^2\cdot\pi=\\
&=& \frac{ab}{4}\cdot\pi=\\
&=& \left(\frac{|CD|}{2}\right)^2\cdot\pi=p_0,
\end{eqnarray*}
which was to be proven. \kdokaz


                \bzgled
                Let $k$ be a drawn circle of a right triangle and $\mathcal{K}$ the corresponding circle. Then draw three
                smaller circles, which touch the circle $k$ and two sides of
                the triangle, then three smaller circles, which touch the previous three
                 circles and two sides of the triangle, and so on until infinity. Is the area of all the corresponding circles, except for the circle $\mathcal{K}$,
                smaller
                than half the area of the circle $\mathcal{K}$?
                \ezgled



\begin{figure}[!htb]
\centering
\input{sl.plo.8.5.11.pic}
\caption{} \label{sl.plo.8.5.11.pic}
\end{figure}

\textbf{\textit{Solution.}}  (Figure \ref{sl.plo.8.5.11.pic})

Let $ABC$ be a given triangle and $k(S,r)$ the inscribed circle of this triangle. With $k_i(S_i ,r_i)$ ($i\in\mathbb{N}$) we denote the circles from the given sequence of circles at the vertex $A$ and with $p_i$ ($i \in \mathbb{N}$) the areas of the corresponding circles. We also denote the circle $k$ with $k_0(S_0 ,r_0)$ and the area of the corresponding circle $\mathcal{K}$ with $p=p_0$. Let $k_n(S_n ,r_n)$ and $k_{n+1}(S_{n+1} ,r_{n+1})$ be the consecutive circles from the mentioned sequence. We denote their tangent points with the side $AB$ of the triangle $ABC$ with $T_n$ and $T_{n+1}$, and the orthogonal projection of the center $S_{n+1}$ onto the line $S_nT_n$ with $L_n$. In the right triangle $S_{n+1}L_nS_n$ the length of the hypotenuse is $|S_nL_n| = r_n- r_{n+1}$ and the hypotenuse $|S_{n+1}S_n| = r_n+ r_{n+1}$. As $L_nS_{n+1}S_n$ is equal to half of the internal angle at the vertex $A$, namely $30°$. From this it follows that $\angle S_{n+1}S_nL_n=60^0$. Let $S_n'=\mathcal{S}_{S_{n+1}L_n}(S_n)$. The triangles $S_{n+1}L_nS_n$ and $S_{n+1}L_nS_n'$ are congruent (by the \textit{SAS} \ref{SKS}), so also $\angle S_{n+1}S_n'L_n=60^0$. Therefore, $S_{n+1}S_n'S_n$ is an isosceles triangle, which means that $|S_{n+1}S_n| =|S_n'S_n| =2\cdot|S_nL_n|$ or
$$r_n+ r_{n+1}=2\cdot(r_n-r_{n+1}).$$
If we express $r_{n+1}$ from the last equality, we get:
 \begin{eqnarray*}
 r_{n+1}=\frac{1}{3}\cdot r_n.
\end{eqnarray*}
From this, by \ref{ploscKrog}, it follows:
 \begin{eqnarray*}
 p_{n+1}=\frac{1}{9}\cdot p_n.
\end{eqnarray*}
The sequence $p_n$ is therefore a geometric sequence with the coefficient $q=\frac{1}{9}$ and the initial value $p_0=p$, so:
 \begin{eqnarray*}
 p_n=\left( \frac{1}{9} \right)^n \cdot p.
\end{eqnarray*}
This means that the total area $p_A$ of the circles from the sequence at the vertex $A$ represents the sum of an infinite geometric sequence $p_n$ ($n\in \mathbb{N}$). Therefore:
\begin{eqnarray*}
 p_A&=&p_1+p_2+p_3+\cdots=\\
&=&\frac{1}{9}\cdot p+\left(\frac{1}{9}\right)^2\cdot p+
\left(\frac{1}{9}\right)^3\cdot p+\cdots=\\
&=&\frac{1}{9}p\cdot \frac{1}{1-\frac{1}{9}}=\\
&=& \frac{1}{8}\cdot p.
\end{eqnarray*}
The sum of the areas of all circles at the vertices $A$, $B$ and $C$ is then equal to $\frac{3}{8}\cdot p$ and is not greater than half of $p$.
\kdokaz

\bnaloga\footnote{6. IMO, USSR - 1964, Problem 3.}
             A circle is inscribed in triangle $ABC$ with sides $a$, $b$, $c$. Tangents to the circle
            parallel to the sides of the triangle are constructed. Each of these tangents
            cuts off a triangle from triangle $ABC$. In each of these triangles, a circle is inscribed.
            Find the sum of the areas of all four inscribed circles (in terms of $a$, $b$, $c$).
            \enaloga

\begin{figure}[!htb]
\centering
\input{sl.plo.8.5.IMO1.pic}
\caption{} \label{sl.plo.8.5.IMO1.pic}
\end{figure}

\textbf{\textit{Solution.}} Let $k(S,\rho)$ be the inscribed circle
of triangle $ABC$ with altitudes $v_a$, $v_b$ and $v_c$ and $p$ the
area of the corresponding circle. We denote by $AB_aC_a$, $A_bBC_b$
and $A_cB_cC$ the new triangles, $k_a$, $k_b$ and $k_c$ the inscribed
circles of these triangles with radii $\rho_a$, $\rho_b$ and $\rho_c$
and $p_a$, $p_b$ and $p_c$ the areas of the corresponding circles.
(Figure \ref{sl.plo.8.5.IMO1.pic}).

The triangles $ABC$ and $AB_aC_a$ are similar with the coefficient
$\frac{v_a-2\rho}{v_a}$. Therefore:
 $$\rho_a=\frac{v_a-2\rho}{v_a}\cdot \rho.$$
 Because according to the statement \ref{PloscTrikVcrt} for the area of the triangle $ABC$
 it holds $p_\triangle=s\cdot \varrho$
 (where $s=\frac{a+b+c}{2}$ is the semi-perimeter of the triangle $ABC$), it follows:
$$\rho_a^2=\left(1-\frac{2\rho}{v_a}\right)^2\cdot \rho^2=
 \left(1-\frac{2\frac{P_\triangle}{s}}{2\frac{P_\triangle}{a}}\right)^2\cdot \rho^2=
 \frac{(s-a)^2}{s^2}\cdot \rho^2.$$
Similarly we obtain $\rho_b^2=
 \frac{(s-b)^2}{s^2}\cdot \rho^2$ and $\rho_c^2=
 \frac{(s-c)^2}{s^2}\cdot \rho^2$. From the statement \ref{PloscTrikOcrtVcrt}
 it follows
  $\rho^2=\frac{(s-a)(s-b)(s-c)}{s}$, therefore:
\begin{eqnarray*}
 && p+p_a+p_b+p_c= \pi (\rho^2+\rho_a^2+\rho_b^2+\rho_c^2)=\\
 &&=\pi
 \rho^2(1+\frac{(s-a)^2}{s^2}+\frac{(s-b)^2}{s^2}+\frac{(s-c)^2}{s^2})=\\
 &&=
 \frac{(s-a)(s-b)(s-c)(s^2+(s-a)^2+(s-b)^2+(s-c)^2)}{s^3}\cdot \pi,
\end{eqnarray*}
 which had to be expressed. \kdokaz


%________________________________________________________________________________
\poglavje{Pythagoras' Theorem and Area} \label{odd8PloPitagora}

 In the section \ref{odd7Pitagora} we have proven Pythagoras' statement \ref{PitagorasIzrek} and predicted that we will express it in the form that relates to the areas.

            \bizrek \index{statement!Pythagoras for areas} \label{PitagorasIzrekPlosc}(Pythagoras' statement in the form of areas)\\
            The area of the square over the hypotenuse of a right-angled triangle is equal to the sum of the areas of the squares over the two cathets of this triangle.

            \eizrek


\begin{figure}[!htb]
\centering
\input{sl.plo.8.6.1.pic}
\caption{} \label{sl.plo.8.6.1.pic}
\end{figure}

\textbf{\textit{Proof.}} At this point we will provide two proofs and one idea of the proof of this statement.

\textit{1)} The statement is a direct consequence of the Pythagorean Theorem \ref{PitagorovIzrek} and the formula for the area of a square \ref{ploscKvadr} (Figure
\ref{sl.plo.8.6.1.pic}).


\textit{2)\footnote{This proof was published by \index{Euclid}\textit{Euclid of Alexandria} (3rd century BC) in his famous work 'Elements', which consists of 13 books.}} Let $AMNB$, $BPQC$ and $CKLA$ be squares constructed on the hypotenuse and
the legs of the right triangle $ABC$. Let $D$ and $E$ be the projections of point $C$ on the lines $AB$ and $MN$ (Figure
\ref{sl.plo.8.6.1.pic}).

Since the area of the square $AMNB$ is equal to the sum of the areas of the rectangles $NBDE$ and $EDAM$, it is enough to prove that the areas of the squares $BPQC$ and
$CKLA$ are equal to the areas of the rectangles $NBDE$ and $EDAM$. We will prove only the first equality $p_{BPQC}=p_{NBDE}$ (the
second equality $p_{CKLA}=p_{EDAM}$ is then analogous).

For this relation $p_{BPQC}=p_{NBDE}$ it is enough to prove the equality of the areas of the triangles $BPQ$ and
$NBD$, since $p_{BPQC}=2\cdot p_{BPQ}$ and $p_{NBDE}=2\cdot p_{NBD}$.
 Then:
 \begin{eqnarray*}
   p_{BPQ}&=&p_{BPA}=\hspace*{3mm} \textrm{(the triangles have the same base and height)}\\
   &=&p_{NBC}=\hspace*{3mm} \textrm{(the triangles are congruent due to } \mathcal{R}_{B,90^0+\angle CBA}\textrm{)}\\
   &=&p_{NBD}\hspace*{4mm} \textrm{(the triangles have the same base and height)}
 \end{eqnarray*}
Therefore, the triangles $BPQ$ and $NBD$ have the same area, which was enough to prove.


\begin{figure}[!htb]
\centering
\input{sl.plo.8.6.2.pic}
\caption{} \label{sl.plo.8.6.2.pic}
\end{figure}

\textit{3)\footnote{This proof (based on the illustration) is attributed to the Ancient Indians. We can say that it only needs the comment: "Look!" - so it is simple and elegant.}} The idea of the proof is given by the
picture \ref{sl.plo.8.6.2.pic}. We will leave the formal proof to the reader.
\kdokaz

\bzgled
            The area of a circular sector, determined by the inscribed and circumscribed circle of a regular $n$-gon with a side length $a$, is not dependent on the number $n$.
            \ezgled


\begin{figure}[!htb]
\centering
\input{sl.plo.8.6.3.pic}
\caption{} \label{sl.plo.8.6.3.pic}
\end{figure}

\textbf{\textit{Proof.}} Let $AB$ be one side, $S$ the center of this side and $O$ the center of the inscribed or circumscribed circle of any regular $n$-gon $\mathcal{V}_n(a)$ (Figure \ref{sl.plo.8.6.3.pic}). According to the assumption, $|AB|=a$, therefore $|AS|=\frac{a}{2}$. We mark with $R$ and $r$ the radius of the inscribed and circumscribed circle of the $n$-gon $\mathcal{V}_n(a)$. From the congruence of triangles $OSA$ and $OSB$ (\textit{SSS} \ref{SSS}) it follows that $\angle OSA\cong \angle OSB=90^0$. Therefore $OSA$ is a right triangle with hypotenuse $OA$, so by the Pythagorean Theorem \ref{PitagorovIzrek} $|OA|^2-|OS|^2=|AS|^2$ or $R^2-r^2=\left(\frac{a}{2}\right)^2$. From this relation and the formula for the area of a circular sector \ref{ploscKrozKolob} we get:
 $$p_k=\left(R^2-r^2\right)\cdot \pi=\left(\frac{a}{2}\right)^2\cdot \pi=\frac{a^2\pi}{4},$$
which means that the area of the sector is only dependent on the length of the side of the regular $n$-gon $\mathcal{V}_n(a)$.
\kdokaz

            \bzgled
            Using the statement from Example \ref{PitagoraCofman}, prove the inequality:
            \begin{eqnarray*}
            \frac{1}{1^2\cdot 2^2}+\frac{1}{2^2\cdot 3^2}+\cdots +\frac{1}{n^2 \left(n+1\right)^2}+\cdots<\frac{2}{3}
            \end{eqnarray*}
             or
            \begin{eqnarray*}
                \sum_{n=1}^{\infty}\frac{1}{n^2 \left(n+1\right)^2}<\frac{2}{3}.
             \end{eqnarray*}
            \ezgled

\begin{figure}[!htb]
\centering
\input{sl.plo.8.6.4.pic}
\caption{} \label{sl.plo.8.6.4.pic}
\end{figure}

\textbf{\textit{Proof.}}  (Figure \ref{sl.plo.8.6.4.pic})

We use the notation from the example \ref{PitagoraCofman}. According to the relation \ref{eqnPitagoraCofman3} from this example, $d_n=\frac{1}{n(n+1)}$, therefore $r_n=\frac{1}{2n(n+1)}$. If we denote the area of the corresponding circle of the circle $c_n$ with $p_n$, it is (formula \ref{ploscKrog}):
\begin{eqnarray*}
p_n=\pi r_n^2=\frac{\pi}{4}\frac{1}{n^2\left(n+1\right)^2}.
\end{eqnarray*}
Therefore, the sum of the areas of all corresponding circles is equal to:
\begin{eqnarray} \label{eqnPitagoraCofmanPlo2}
\sum_{n=1}^{\infty} p_n=\frac{\pi}{4}\sum_{n=1}^{\infty}\frac{1}{n^2\left(n+1\right)^2}.
\end{eqnarray}
But this sum is smaller than the area $p$ of the figure bounded by the line
$A'B'$ and the circular arc $A'P$ and $B'P$ of the circles $a$ and $b$ with the central angle $90^0$. The area $p$ is equal to the difference of the area of the rectangle $A'ABB'$ and the area of two corresponding circular sectors with the central angle $90^0$. Therefore (formula \ref{ploscPravok} and \ref{ploscKrozIzsek}):
   \begin{eqnarray} \label{eqnPitagoraCofmanPlo3}
p=2-2\cdot\frac{\pi}{4}=2-\frac{\pi}{2}.
\end{eqnarray}
Because $\sum_{n=1}^{\infty} p_n<p$, from \ref{eqnPitagoraCofmanPlo2} and \ref{eqnPitagoraCofmanPlo3} we get:
\begin{eqnarray*}
\frac{\pi}{4}\sum_{n=1}^{\infty}\frac{1}{n^2\left(n+1\right)^2}<2-\frac{\pi}{2}.
\end{eqnarray*}
If we also use the fact that $\pi>3$ (formula \ref{stevPiOcena}), we get:
\begin{eqnarray*}
\sum_{n=1}^{\infty}\frac{1}{n^2\left(n+1\right)^2}<\frac{8}{\pi}-2<\frac{8}{3}-2=\frac{2}{3},
\end{eqnarray*}
which was to be proven. \kdokaz



% Ttrikotniki

\item Let $T$ be the centroid of the triangle $ABC$. Prove:
   $$p_{TBC}=p_{ATC}=p_{ABT}.$$

\item Let $c$ be the hypotenuse, $v_c$ be the corresponding altitude, and $a$ and $b$ be the catheti of a right triangle. Prove that $c+v_c>a+b$.
% zvezek - dodatni MG


  \item Draw a square that has the same area as a rectangular triangle with catheti that are consistent with the given distances $a$ and $b$. % (Hipokratovi luni)

\item Let $ABC$ be an isosceles right triangle with the length of the catheti $|CB|=|CA|=a$. We mark with $A_1$, $B_1$, $C_1$ and $C_2$ the points for which it holds: $\overrightarrow{CA_1}=\frac{n-1}{n}\cdot \overrightarrow{CA}$, $\overrightarrow{CB_1}=\frac{n-1}{n}\cdot \overrightarrow{CB}$, $\overrightarrow{AC_1}=\frac{n-1}{n}\cdot \overrightarrow{AB}$ and  $\overrightarrow{AC_2}=\frac{n-2}{n}\cdot \overrightarrow{AB}$ for some $n\in \mathbb{N}$. Express the area of the quadrilateral determined by the lines $AB$, $A_1B_1$, $CC_1$ and $CC_2$, as a function of $a$ and $n$.

\item Given is a right triangle $ABC$ with hypotenuse $AB$ and area $p$. Let $C'=\mathcal{S}_{AB}(C)$,  $B'=\mathcal{S}_{AC}(B)$ and  $A'=\mathcal{S}_{BC}(A)$. Express the area of the triangle $A'B'C'$ as a function of $p$.

\item Let $R$ and $Q$ be points in which the inscribed circle of the triangle $ABC$ intersects its sides $AB$ and $AC$. Let the internal angle bisector of $ABC$ intersect the line $QR$ in the point $L$. Determine the ratio of the areas of the triangles $ABC$ and $ABL$.
    % pripremni zadaci - naloga 193


\item Determine a point inside the triangle $ABC$, for which the product of its distances from the sides of this triangle is maximal. % zvezek - dodatni MG

\item   The triangle
             $ABC$ with sides of lengths $a$, $b$ and $c$ has an inscribed circle. We draw
             the tangents of this circle that are parallel
             to the sides
             of the triangle.
             Each of the tangents inside the triangle determines the appropriate distances of lengths $a_1$, $b_1$ and $c_1$. Prove that:
 $$\frac{a_1}{a}+\frac{b_1}{b}+\frac{b_1}{b}=1.$$ % zvezek - dodatni MG

\item Let $T$ and $S$ be the centroid and the center of the inscribed circle of the triangle $ABC$. Let also $|AB|+|AC|=2\cdot |BC|$. Prove that $ST\parallel BC$. % zvezek - dodatni MG

\item Let $p$ be the area of the triangle $ABC$, $R$ the radius of the circumscribed circle and $s'$ the perimeter of the pedal triangle. Prove that $p=Rs'$. %Lopandic - nal 918

% Sstirikotniki


\item Let $L$ be an arbitrary point inside the parallelogram $ABCD$. Prove that:
        $$p_{LAB}+p_{LCD}=p_{LBC}+p_{LAD}.$$ %Lopandic - nal 890

\item Let $P$ be the center of the leg $BC$ of the trapezoid $ABCD$. Prove:
 $$p_{APD}=\frac{1}{2}\cdot p_{ABCD}.$$

\item Let $o$ be the perimeter, $v$ the height and $p$ the area of the tangent trapezoid. Prove that: $p=\frac{o\cdot v}{4}$.

\item Draw a line $p$, which goes through the vertex $D$ of the trapezoid $ABCD$ ($AB>CD$), so that it divides this trapezoid into two area-equal figures.


\item Draw lines $p$ and $q$, which go through the vertex $D$ of the square $ABCD$ and divide it into area-equal figures.

\item Let $ABCD$ be a square, $E$ the center of its side $BC$ and $F$ a point, for which $\overrightarrow{AF}=\frac{1}{3}\cdot \overrightarrow{AB}$. The point $G$ is the fourth vertex of the rectangle $FBEG$. What part of the area of the square $ABCD$ does the area of the triangle $BDG$ represent?

\item Let $\mathcal{V}$ and $\mathcal{V}'$ be similar polygons with the similarity coefficient $k$. Then:
 $$p_{\mathcal{V}'}=k^2\cdot p_{\mathcal{V}}.$$ Prove.

\item Let $a$, $b$, $c$ and $d$ be the lengths of the sides, $s$ the semiperimeter and $p$ the area of an arbitrary quadrilateral. Prove that:
        $$p=\sqrt{(s-a)(s-b)(s-c)(s-d)}.$$ %Lopandic - nal 924

\item Let $a$, $b$, $c$ and $d$ be the lengths of the sides and $p$ the area of the string-tangent quadrilateral. Prove that:
        $$p=\sqrt{abcd}.$$ %Lopandic - nal 925

% Veckotniki

\item There is a rectangle $ABCD$ with sides of lengths $a=|AB|$ and $b=|BC|$. Calculate the area of the figure that represents the union of the rectangle $ABCD$ and its image under reflection across the line $AC$.

\item Let $ABCDEF$ be a regular hexagon, and let $P$ and $Q$ be the centers of its sides $BC$ and $FA$. What part of the area of this hexagon is the area of the triangle $PQD$?


% KKrog

\item In a square, four congruent circles are drawn so that each circle touches two sides and two circles. Prove that the sum of the areas of these circles is equal to the area of the square inscribed circle.

\item Calculate the area of the circle that is inscribed in the triangle with sides of lengths 9, 12 and 15. %resitev 54

\item Let $P$ be the center of the base $AB$ of the trapezoid $ABCD$, for which $|BC|=|CD|=|AD|=\frac{1}{2}\cdot |AB|=a$. Express the area of the figure determined by the base $CD$ and the shorter circular arcs $PD$ and $PC$ of the circles with centers $A$ and $B$, as a function of the base $a$.

\item The cord $PQ$ ($|PQ|=d$) of the circle $k$ touches its conchoid $k'$. Express the area of the lune determined by the circles $k$ and $k'$, as a function of the cord $d$.

\item Let $r$ be the radius of the inscribed circle of the polygon $\mathcal{V}$, which is divided into  triangles $\triangle_1,\triangle_2,\ldots,\triangle_n$, so that no two triangles have common interior points.  Let $r_1,r_2,\ldots , r_n$ be the radii of the inscribed circles of these triangles. Prove that:
     $$\sum_{i=1}^n r_i\geq r.$$

\end{enumerate}


% DEL 3 - - - - - - - - - - - - - - - - - - - - - - - - - - - - - - - - - - - - - - -
%________________________________________________________________________________
% INVERZIJA
%________________________________________________________________________________


  \del{Inversion} \label{pogINV}
\poglavje{Definition and Basic Properties of Inversion}
\label{odd9DefInv}

In this chapter we will define inversion - a new mapping, which in a certain way represents the "reflection" over a circle.
We expect this new mapping to have similar properties to the reflection over a line. It is clear that inversion cannot be an isometry, but despite this, we will list the following desired common properties of the "two reflections" (Figure \ref{sl.inv.9.1.1.pic}):

\begin{figure}[!htb]
\centering
\input{sl.inv.9.1.1.pic}
\caption{} \label{sl.inv.9.1.1.pic}
\end{figure}

\begin{itemize}
  \item the reflection is a bijective mapping,
  \item the reflection is an involution,
  \item the reflection maps one of the areas of the plane, which the (circle) axis of the reflection determines, into the other area,
  \item all
fixed points of the reflection lie on the (circle) axis of this
reflection,
  \item if $X'$ is the image of the point $X$ under the reflection, then the line
  $XX'$
is perpendicular to the (circle) axis of this reflection.
\end{itemize}

 The mapping, which we will now
define, should therefore satisfy all these properties. Instead of the name
 "reflection over a circle" we will rather use
 the term "inversion"\index{inversion}\footnote{Inversion was first introduced by the German mathematician
\index{Magnus, L. I.} \textit{L. I. Magnus} (1790--1861) in 1831.
The first ideas of inversion appeared already in the ancient Greek mathematician
\index{Apolonij} \textit{Apolonij iz Perge} (3.--2. st. pr. n.
š.) and the Swiss mathematician \index{Steiner, J.} \textit{J.
Steiner} (1796--1863)}.


 In the Euclidean plane we will define inversion in the following way.
Let $k(O,r)$ be a circle in the Euclidean plane $\mathbb{E}^2$. The point $X'$ is the image of some point $X\in \mathbb{E}^2\setminus
\{O\}$ in the \index{inversion} \pojem{inversion} $\psi_k$ with respect to
the circle $k$ (its \index{inverse image} \pojem{inverse image}), if it belongs to the segment $OX$ and $|OX|\cdot
|OX'|=r^2$, $k$ is the \index{circle!inversion}\pojem{circle
of inversion}, $O$ is the \index{center!inversion} \pojem{center
of inversion}.

We will often denote the inversion $\psi_k$ with respect to the circle $k(O,r)$ by $\psi_{O,r}$.

From the definition itself it immediately follows that the only fixed points of the inversion are precisely the points on the inversion circle. And all points of the inversion circle are fixed. Therefore, the equivalence holds:
$$\psi_k(X)=X\hspace*{1mm}\Leftrightarrow\hspace*{1mm}X\in k.$$

The inversion $\psi_k$ is a bijective mapping of the set $\mathbb{E}^2\setminus \{O\}$. It also holds that $\psi_k^{-1}=\psi_k$, therefore $\psi_k^2$ is the identical mapping. Both properties follow directly from the definition. It also directly follows that the external points of the circle $k$ with the inversion $\psi_k$ are mapped to internal points and vice versa (without the point O).

From the definition it follows that the line $XX'$ is perpendicular to the inversion circle, which means that the last desired property is also fulfilled.

Now we will construct the image $X'=\psi_k(X)$ of an arbitrary point $X$ with the inversion. From everything said above (especially from the fact that $\psi_k^{-1}=\psi_k$) it follows that it is enough to describe the construction when $X$ is outside the circle $k$. Let in this case $OX$ be the tangent of the circle $k$ in its point $T$, then from $|OX|\cdot |OX'|=r^2=|OT|^2$ it follows that the triangles $OTX$ and $OX'T$ are similar and therefore $\angle TX'O=90^0$ (Figure \ref{sl.inv.9.1.2.pic}). Since the point $X'$ is also on the segment $OX$, we get this point as the perpendicular projection of the point $T$ on the segment $OX$.

From the relation in the definition of inversion $|OX|\cdot |OX'|=r^2$ it follows that if the point $X$ approaches the center $O$, its image $X'$ "moves away to infinity". It also holds - if the point $X$ is close to the circle $k$, its image is also close to this circle, of course on the other side. We can write all these findings in a more formal way.

\bizrek \label{invUrejenost} If $\psi_k:A,B \mapsto A',B'$ and
$\mathcal{B}(O,A,B)$,
 then $\mathcal{B}(O,B',A')$.
 \eizrek

 The point $O$ has no image. Intuitively, its image is a point at infinity. We will say more about this in chapter \ref{odd9InvRavn}.

We prove the following important property of inversion.

\bizrek \label{invPodTrik} Let $O$, $A$ and $B$ be three
non-collinear points and $\psi_k$ be the inversion with respect to the circle
$k(O,r)$. If $A'$ and $B'$ are the images of points $A$ and $B$ in this
inversion, then the triangles $OAB$ and $OB'A'$ are similar, that is:
$$\psi_{O,r}:A,B\mapsto A',B' \Rightarrow \triangle OAB \sim \triangle
OB'A'.$$
 \eizrek


\begin{figure}[!htb]
\centering
\input{sl.inv.9.1.3.pic}
\caption{} \label{sl.inv.9.1.3.pic}
\end{figure}

 \textbf{\textit{Proof.}} Because $\psi_{O,r}:A,B\mapsto A',B'$,
  it is also
$OA\cdot OA'=OB\cdot OB'=r^2$ (Figure \ref{sl.inv.9.1.3.pic}).
The similarity of the triangles now follows from $OA:OB'=OB:OA'$ and $\angle AOB
\cong \angle B'OA'$.
 \kdokaz

 The next statement is a direct consequence of the previous proposition.

\bizrek Let $O$, $A$ and $B$ be three non-collinear points and
$\psi_k$ be the inversion with respect to the circle $k(O,r)$. If $A'$ and
$B'$ are the images of points $A$ and $B$ in this inversion, then:

 (i) $\angle
OAB \cong \angle OB'A'$ and $\angle OBA \cong \angle OA'B'$,

(ii) points $A$, $B$, $B'$ and $A'$ are concircular,

(iii) the circle passing through points $A$, $B$, $B'$ and $A'$ is
orthogonal to the inversion circle.
 \eizrek

\begin{figure}[!htb]
\centering
\input{sl.inv.9.1.4.pic}
\caption{} \label{sl.inv.9.1.4.pic}
\end{figure}

\textbf{\textit{Proof.}}
 \textit{(i)} The angle congruence follows from the similarity of triangles $OAB$ and
 $OB'A'$(the previous proposition \ref{invPodTrik}).

\textit{(ii)} Let $l$ be a circle drawn through the triangle
$ABA'$ (Figure \ref{sl.inv.9.1.4.pic}). From the definition of
inversion it follows that $OA\cdot OA'=OB\cdot
OB'=r^2$. This means that the power of the point $O$ on the circle $l$
is equal to $OA\cdot OA'=OB\cdot
OB'$, thus the point $B'$ lies on this circle.

We mention that the tautness of the quadrilateral $AA'B'B$ also follows directly
from $\angle
OAB \cong \angle OB'A'$, because then $\angle A'AB + \angle BB'A'
= 180^0 $.

\textit{(iii)} From $\psi_{O,r}(A)=A'$ it follows that one of the points
$A$ and $A'$ is in the interior, the other in the exterior of the inversion circle $k$. By the consequence of Dedekind's axiom
(\ref{DedPoslKrozKroz}) the circles $k$ and $l$ have two common
points; one of them we denote by $T$. Because $OA\cdot OA'=OB\cdot
OB'=r^2=OT^2$, it follows that $OT$ is a tangent to the circle $k$,
which means that the circles $k$ and $l$ are perpendicular (by the statement
\ref{pravokotniKroznici}).
\kdokaz

In the same way as in part (\textit{iii}) of the previous statement,
we also prove the following claim.

\bizrek \label{invPravKrozn} Let
$X'$ be the image of the point $X$ ($X\neq X'$) under the inversion $\psi_k$
with respect to the circle $k(O,r)$. Then every circle that goes
through the points $X$ and $X'$ (also the line $XX'$) is perpendicular to
the inversion circle $k$.
 \eizrek

 In the next example we will see how we can translate an inversion with the help of
 certain stretches into an inversion with a concentric
 inversion circle.

\bzgled \label{invRazteg} The composition of the inversion $\psi_{S,r}$ and the dilation $h_{S,k}$ with the same center $S$ and positive coefficient ($k>0$) is an inversion with center $S$ and coefficient $r\sqrt{k}$.
 \ezgled
\textbf{\textit{Proof.}} Let $f=h_{S,k}\circ \psi_{S,r}$. For any point $X$, denote $X'=f(X)$. First, it is clear that the point $X'$ lies on the line segment $SX$. Also, denote $X_1=\psi_{S,k}(X)$. From the definition of inversion and dilation, it follows that $|SX_1| \cdot |SX| =r^2$ and $|SX'| = k\cdot |SX_1| $. Therefore, $|SX'| \cdot |SX| =k\cdot r^2=\left(r\sqrt{k}\right)^2$ or $\psi_{S,r\sqrt{k}}(X)=X' =f(X)$.
Since this is true for any point $X$, we have $f=h_{S,k}\circ \psi_{S,r}=\psi_{S,r\sqrt{k}}$.
 \kdokaz


%________________________________________________________________________________
 \poglavje{Image of a Circle or Line Under an Inversion} \label{odd9SlokaKrozPrem}

In this section, we will determine that an inversion is not a collinearity, which means that it does not preserve the relation of collinearity of points. We will consider the images of lines and circles under inversion. Since the domain of an inversion is an Euclidean plane without the center of inversion, it makes sense to introduce the following notation: if $\Phi$ is an arbitrary figure of the Euclidean plane and $S$ is a point of this plane, then
  $$\Phi^S= \Phi \setminus
 \{S\}.$$

\bizrek \label{InverzKroznVkrozn}
   Let $\psi_i$ be an inversion with respect to the circle $i(S,r)$ of the Euclidean plane.
   If $p$ is a line and $k$
is a circle of this plane, then (Figure \ref{sl.inv.9.2.1.pic}):

(i) if $S\in p$, then $\psi_i(p^S)=p^S$,

(ii) if $S\notin p$, then $\psi_i(p)=j^S$, where $j$ is a circle that
passes through the point $S$,

(iii) if $S\in k$, then $\psi_i(k^S)=q$, where $q$ is a line that
does not pass through the point $S$;

(iv) if $S\notin k$, then $\psi_i(k)=k'$, where $k'$ is a circle that
does not pass through the point $S$.
   \eizrek


\begin{figure}[!htb]
\centering
\input{sl.inv.9.2.1.pic}
\caption{} \label{sl.inv.9.2.1.pic}
\end{figure}

 \textbf{\textit{Proof.}}

(\textit{i}) From the definition of inversion it follows that the image of any
  point
   $X \in p^S$ lies on the open ray $SX$, therefore $\psi_i(X)\in
   p^S$. Because $\psi_i^{-1}=\psi_i$, any point $Y\in
   p^S$ is the image of the point $\psi_i^{-1}(Y)=\psi_i(Y)\in
   p^S$. Therefore $\psi_i(p^S)=p^S$.

   (\textit{ii}) Let $P$ be the orthogonal projection of the point
   $S$ onto the line $p$ (Figure \ref{sl.inv.9.2.2.pic}).
   Because $S\notin p$, then $P\neq S$, therefore
there exists an image of the point $P$ under the inversion $\psi_i$ -- we denote it with $P'$.
Let $X$ be any point on the line $p$ ($X\neq P$) and $X'=\psi_i(X)$.
By  \ref{invPodTrik} the triangles $SPX$ and $SX'P'$ are similar,
therefore $\angle SX'P' \cong \angle SPX=90^0$. Therefore the point $X'$ lies
on the circle above the diameter $SP'$ - we denote it with $j$. Because $X'\neq
S$, it follows that $X'\in j^S$ or $\psi_i(p)\subseteq j^S$. With the reverse
procedure we prove that each point of the set $j^S$ is the image of some point
on the line $p$, therefore
 $\psi_i(p)= j^S$.


   (\textit{iii}) A direct consequence of (\textit{ii}), because $\psi_i$  is an involution,
   or $\psi_i^{-1}=\psi_i$.

(\textit{iv}) (Figure \ref{sl.inv.9.2.2.pic}) Let $P$ and $Q$ be the
   intersection points of the circle $k$ with the line that goes through the point $S$,
   and the center
   of the circle
   $k$ (for example, when the circles $i$ and $k$ are concentric, it is easy).
    Without loss of generality, assume that $\mathcal{B}(S,P,Q)$ is true.
     Let $P'=\psi_i(P)$,
$Q'=\psi_i(Q)$ and $X$ be any point on the circle $k$, which is
different from the points $P$ and $Q$ and $X'=\psi_i(X)$. By
the statement
\ref{invPodTrik} we have $\triangle SPX \sim \triangle SX'P'$ and $\triangle SQX
\sim \triangle SX'Q'$. Therefore,
 $\angle SX'P' \cong \angle SPX$ and $\angle SX'Q'\cong \angle SQX$, which means:
$$\angle P'X'Q'=\angle SX'P'-\angle SX'Q'
=\angle SPX-\angle SQX=\angle PXQ=90^0.$$
 Then the point $X'$ lies on
the circle above the diameter $P'Q'$ - we denote it with $k'$. By reversing
the process, we can also prove that every point on the circle $k'$
is the image of a point on the circle $k$, so $\psi_i(k)=k'$

We prove the statement similarly in the case when
$\mathcal{B}(P,S,Q)$ is true.
 \kdokaz

The previous statement also provides an effective way of constructing the image
of a line or a circle in different cases. We will use
the notation from this statement (\ref{InverzKroznVkrozn}).

In case (\textit{ii}) - figure \ref{sl.inv.9.2.2a.pic} - it is enough
to determine the image of the perpendicular projection of $P$ onto the center of inversion $S$
on the line $p$, which we map. The point $P'=\psi_i(P)$ with
the center of inversion determines the diameter of the circle $j$. But if the line
$p$ intersects the circle of inversion $i(S,r)$ for example in the points $M$ and $N$,
both points are fixed and the image of the line $p$ is the circumscribed circle
of the triangle $SMN$ (without the point $S$). The tangent of the circle of inversion
$i$ is mapped into a circle (without the point $S$), which from the inside
touches the circle $i$.


\begin{figure}[!htb]
\centering
\input{sl.inv.9.2.2a.pic}
\caption{} \label{sl.inv.9.2.2a.pic}
\end{figure}

In the case of (\textit{iii}) we do the inverse of the construction from
(\textit{ii}) and we get the orthogonal projection on the line that is the image of the given circle. The other special cases (when the circle intersects or touches the circle of inversion) are the inverses of (\textit{ii}).

 Also in the case of (\textit{iv}), when the circle is mapped into a circle, the construction process is clear from the statement itself - the image $k'$ is determined by the points $P'$ and $Q'$.
 We just have to be careful that in this case the center of the circle $k$ is not mapped into the center of the new circle $k'$!
 Where is the image of the center of this circle located in this case? We denote with $O'$ the image of the center $O$ of the circle $k$ under the inversion $\psi_i$. Let $t_1=ST_1$ and $t_2=ST_2$ be the tangents of the circle $k$ at the points $T_1$ and $T_2$, and $T'_1=\psi_i(T_1)$ and $T'_2=\psi_i(T_2)$ (Figure \ref{sl.inv.9.2.3.pic}). It is clear that $T'_1, T'_2 \in k'$, but $t_1$ and $t_2$ are also tangents of the circle $k'$ (if, for example, $Y\in k'\cap t_1$, then also $X=\psi_i(Y)\in k\cap t_1=\{T_1\}$, i.e. $X=T'_1$). By \ref{invPodTrik} we have $\triangle ST_1O \sim \triangle SO'T'_1$ or $\angle SO'T'_1 \cong \angle ST_1O=90^0$. For the same reasons $\angle SO'T'_2 \cong \angle ST_2O=90^0$. This means that the points $T'_1$, $O'$ and $T'_2$ are collinear. So we can find the point $O'$ as the intersection of the line $T'_1T'_2$ and the segment $SO$.

\begin{figure}[!htb]
\centering
\input{sl.inv.9.2.3.pic}
\caption{} \label{sl.inv.9.2.3.pic}
\end{figure}

Another question arises in case of (\textit{iv}).
Which circles are fixed under inversion? One such is of course the circle of inversion $i$ itself, since all of its points are fixed. The next theorem will reveal all other possibilities.

\bizrek \label{InverzKroznFiks}
   Let $\psi_i$ be the inversion with respect to the circle $i(S,r)$.
   The only fixed circles of this inversion are the circle $i$ and the circles
   that are perpendicular to this circle, that is:
   $$\psi_i(k)=k \Leftrightarrow k=i \vee k\perp i.$$
   \eizrek

\begin{figure}[!htb]
\centering
\input{sl.inv.9.2.4.pic}
\caption{} \label{sl.inv.9.2.4.pic}
\end{figure}

\textbf{\textit{Proof.}}

 ($\Leftarrow$) If $k=i$, the statement is trivial. Let $k$ and
 $i$
be perpendicular circles (Figure \ref{sl.inv.9.2.4.pic}). With $T$
we mark one of their intersections. The line $ST$ is tangent
to the circle $k$ (statement \ref{TangPogoj}). Let $X$ be an arbitrary point
on the circle $k$ ($X\notin i$) and $X'$ the other intersection of this
circle with the segment $SX$. Then we have:
 $$p(S,k)= |SX|\cdot |SX'| = |OT|^2 = r^2,$$
so $\psi_i(X)=X'\in k$. Similarly, every point $Y$
on the circle $k$ is the image of some point on this circle, so
$\psi_i(k)=k$ holds.

 ($\Rightarrow$) Let $\psi_i(k)=k$. If $k\neq i$, then
  there exists a point $X$ on
the circle $k$, which does not lie on the circle $i$. Let $X'=\psi_i(X)$.
It is clear that $X'\notin i$ also holds. From the definition of inversion it follows that
the points $X$ and $X'$ are on the segment with the initial point $S$, which
means that $S$ is an external point of the circle $k$ (because $\mathcal{B}(X,S,X')$
is not true). So there exists a tangent from the point $S$ to
the circle $k$ -- with $T$ we mark the point of tangency. Then we have:
 $$p(S,k)=|ST|^2 = |SX| \cdot |SX'| = r^2,$$
 so the point $T$ lies on
the circle $i$. This means that the circles $k$ and $i$ are perpendicular
(statement \ref{TangPogoj}).
 \kdokaz

 We mention that a similar statement also holds for fixed lines, only that
 the perpendicularity of a line with the inversion circle means that the line goes through
 the center of inversion. So the only fixed lines of inversion
 are those that go through the center of inversion (part (\textit{i}) of statement
 \ref{InverzKroznVkrozn}).

The previous statement also holds if instead of inversion we talk about
 reflection over a line. The only fixed lines (circles)
 of this reflection are the reflection axis and those lines (circles)
 that are perpendicular to that axis. This will be the motivation to
  equalize lines and circles in a certain way, thus also the reflection axis and
 inversion. We will learn more about this in the next section.

 \bzgled Let $\gamma$ be an arc and $c$ and $p$ be lines.
Draw a triangle $ABC$, if $AB\cong c$, $\angle BCA\cong \gamma$
and $|BA'|\cdot|BC|=p^2$ (the line $AA'$ is the altitude of the triangle $ABC$).
 \ezgled

\begin{figure}[!htb]
\centering
\input{sl.inv.9.2.5.pic}
\caption{} \label{sl.inv.9.2.5.pic}
\end{figure}

\textbf{\textit{Solution.}} Let $\triangle ABC$ be the desired triangle,
which satisfies the conditions of the task (Figure \ref{sl.inv.9.2.5.pic}).
First, from the condition $\angle BCA\cong \gamma$ it follows that the point $C$
lies on the locus $l$, which is determined by the string $AB$ and the angular
dimension $\gamma$. Since $A'$ is the vertex of the altitude, the point $A'$ lies
on the circle $k$ above the diameter $AB$. Let $\psi_i$ be the inversion with
respect to the circle $i(B,p)$. From the condition
 $|BA'|\cdot|BC|=p^2$ it follows that $\psi_i(A')=C$. Since $A'$ is also in $k$,
 it follows that $C$ is in $k'$, where $k'=\psi_i(k)$ is a line (statement
 \ref{InverzKroznVkrozn}).

 So if we first draw the line $AB\cong c$, we get the point
 $C$ as the intersection of the line $k'=\psi_i(k)$ with the locus $l$.
\kdokaz

\bzgled \label{MiquelKroznice}(Miquel's\footnote{\index{Miquel, A.}\textit{A. Miquel} (1816–-1851), French
mathematician, who published this statement in 1840.} statement about six
circles)
 Let $k_1$, $k_2$, $k_3$ and $k_4$ be such circles that
 the circles $k_1$ and $k_2$ intersect in points $A$ and
$A_0$, the circles $k_2$ and $k_3$ in points $B$ and $B_0$,
the circles $k_3$ and $k_4$ in points $C$ and $C_0$, the circles
$k_4$ and $k_1$ in points $D$ and $D_0$. If the points $A$, $B$, $C$
and $D$ are concircular, the points $A_0$, $B_0$, $C_0$ and $D_0$
are concircular or collinear. \index{statement!Miquel's.}
 \ezgled

\begin{figure}[!htb]
\centering
\input{sl.inv.9.3.7a.pic}
\caption{} \label{sl.inv.9.3.7a.pic}
\end{figure}

\textbf{\textit{Proof.}}  Let $k$ be the circle determined by the
points $A$, $B$, $C$ and $D$. Let $\psi_i$ be the inversion with respect
to any circle $i$ with center at point $A$ (Figure
\ref{sl.inv.9.3.7a.pic}) and
 $$\psi_i:\hspace*{1mm}B,C,D,A_0,B_0,C_0,D_0 \mapsto
 B',C',D',A'_0,B'_0,C'_0,D'_0.$$
   Because the circles $k$, $k_1$ and
$k_2$ go through point $A$, their images $k'$, $k'_1$ and $k'_2$ are
lines under inversion $\psi_i$ that determine the vertices of the
triangle $A_0'B'D'$. Circles $k_3$ and $k_4$, however, do not go
through point $A$, so their images $k'_3$ and $k'_4$ are circles that
intersect at points $C'\in B'D'$ and $C'_0$. Let $k'_0$ be the circle
drawn through the triangle $A'_0D'_0B'_0$. From Example
\ref{Miquelova točka} it follows that the circles $k'_3$, $k'_4$ and
$k'_0$ intersect at one point - $C'_0$, i.e. $A'_0,B'_0,C'_0,D'_0\in
k'_0$. Therefore, $A_0,B_0,C_0,D_0\in k_0=\psi_i(k'_0)$. By Theorem
\ref{InverzKroznVkrozn} $k_0$ is a circle or a line.
 \kdokaz


%________________________________________________________________________________
 \poglavje{The Inversive Plane}
\label{odd9InvRavn}

The motivation for the following discussion is the fact that the
domain of definition is not the whole Euclidean plane under inversion -
the center of inversion does not have its image. This often makes our
work more difficult, as, for example, in the formulation of Theorem
\ref{InverzKroznVkrozn}, where we always have to be careful when a line
or a circle goes through the center of inversion. Also, the images of
a line or a circle are not always a line or a circle, because we have
to exclude the center of inversion in certain cases.

Because of everything mentioned above, another solution is offered; instead of excluding the center of inversion from the definition domain, we can formally add one point to the Euclidean plane that will be the image of the center of inversion at this inversion. From the definition of inversion for the image $X'$ of the point $X$ at the inversion $\psi_{S,k}$ it holds: $|SX'|\cdot |SX|=r^2$. If $S=X$, then we formally get $|SX'|\cdot 0=r^2$, which means that $|SX'|=\infty$. So, the new point is intuitively seen as a "point at infinity", so we will formally mark it with the symbol $\infty$ and call it the \index{point!at infinity}\pojem{point at infinity}. The set that arises from adding this new point to the Euclidean plane is called the \index{inversive plane} \pojem{inversive plane}, which we will mark with $\widehat{E}^2$. So: $$\widehat{E}^2=E^2 \cup \{\infty\}.$$ But we must be careful - we must not mix the inversive plane with the \index{extended Euclidean plane} \pojem{extended Euclidean plane}, which we get if we add one (infinitely distant) line to the Euclidean plane and define that the parallel lines intersect on it. The extended Euclidean plane is a model of so-called projective geometry, which we mentioned in the introduction.

In the case of the inversive plane, we have added only one point $\infty$ and required that for every inversion $\psi_{S,k}$ it holds that $\psi_{S,k}(S)=\infty$. Since $\psi_{S,k}$ is an involution (it holds that $\psi_{S,k}^{-1}=\psi_{S,k}$), by agreement it also holds that $\psi_{S,k}(\infty)=S$.

In section (\textit{iii}) we saw the statement \ref{InverzKroznVkrozn} that
the image of a circle $k^S$ (a circle without point $S$) under an inversion $\psi_{S,k}$ is a line $q$, which does not go through point $S$. But if we add point $S$ to the circle $k^S$, we will get the set $q \cup \{\infty\}$ as the image. Therefore, we will naturally assign point $\infty$ to lines in the inversive plane. If we call such lines with added point $\infty$ $i$-circles, we can translate the statement \ref{InverzKroznVkrozn} and say that an $i$-circle is mapped to an $i$-circle under an inversion. But we also saw certain analogies with reflection over a line and inversion. Therefore, we will call both mappings in the inversive plane $i$-inversions. It is not hard to see that we can generalize the statement \ref{InverzKroznVkrozn} and write it in a simpler form.

 \bizrek \label{InverzInvRavKvK}
   An $i$-circle is mapped to an $i$-circle under an $i$-inversion.
   \eizrek

   The fact that every line in the inversive plane contains point
   $\infty$ means that two parallels intersect in this point. Lines,
   which intersect (in a regular point of the Euclidean plane) have two
   common points in
   the inversive plane.
   Since parallels
   have only one common point, we also say that they are tangent in
   point $\infty$. The image of two parallels under an $i$-inversion will
   either be parallels (if it is the axis of reflection) or
   two circles, which are tangent in the center of inversion (if it is
   an inversion). Two circles, which are tangent (because they have only one
   common point), are mapped either to parallels or to
   circles, which are tangent. The same goes for the case of a line and a circle being tangent. In all these cases in the inversive plane we can formulate them much shorter in the following way.

        \bizrek \label{InverzDotik}
         Let $\psi_i$ be an arbitrary inversion.
        If $i$-circles $k$ and $l$ are tangent, then
        $i$-circles $\psi_i(k)$ and  $\psi_i(l)$ are tangent.
        \eizrek

With the help of this expression, we will first prove the following important
expression.

          \bizrek \label{InverzKonf}
          Inversion is a \index{conformal mapping}conformal mapping, which means:

   The angle under which the lines $p$ and $q$ intersect in the point
$A$, is equal to the angle under which their inverse images $p'$
and $q'$ intersect in the corresponding point $A'$.
   \eizrek


\begin{figure}[!htb]
\centering
\input{sl.inv.9.3.1.pic}
\caption{} \label{sl.inv.9.3.1.pic}
\end{figure}

\textbf{\textit{Proof.}} Let $\psi_i$ be the inversion with respect to the
circle  $i(S,r)$, in which the lines $p$ and $q$ and their
intersection $A$  are mapped to $p'$, $q'$ and $A'$ (Figure
\ref{sl.inv.9.3.1.pic}).

The case when $A=S$, is trivial, because then the lines $p$ and
$q$ are stationary.

 We first assume that $A\notin i$. In this case it is clear that $A\neq
 A'$.

  Let $P$ be the intersection of the rectangle of the line $p$ in the point $A$ and the simetral of the line segment $AA'$ and $k_p$ the circle with the center $P$, which goes through the point $A$. From the construction of the circle $k_p$ it is clear that also the point  $A'$ lies on this circle and that the line $p$ is its
tangent in the point $A$. If in a similar way we mark with $Q$ the intersection of the rectangle of the line $q$ in the point $A$ and the simetral of the line segment $AA'$ and $k_q$ the circle with the center $Q$, which goes through the point $A$, it follows that the point  $A'$ lies on the circle $k_q$ and that the line $q$ is its
tangent in the point $A$.

Because both circles $k_p$ and $k_q$ go through the pair of points $A$, $A'$ of the inversion $\psi_i$, $k_p$ and $k_q$ are perpendicular to the circle of inversion $i$ (expression \ref{invPravKrozn}). According to expression \ref{InverzKroznFiks}, then $\psi_i(k_p)=k_p$ and
$\psi_i(k_q)=k_q$.

The lines $p$ and $q$ intersect the circles $k_p$ and $k_q$ in that order.
By \ref{InverzDotik} the intersections $i$ of the circles $p'$ and
$\psi_i(k_p)=k_p$, or $q'$ and $\psi_i(k_q)=k_q$, are in correspondence. Therefore we have:
 $$\angle p,q \cong \angle k_p,k_q \cong \angle p',q'.$$

  If $A\in i$, we apply the previous
  proof (with an inversion with respect to the concentric circle) using
  \ref{invRazteg}, because a dilation
  preserves the size of angles.
 \kdokaz

 It is clear that the proof would be practically the same if $p$ and $q$
 were circles or a circle and a line. Therefore we can write the previous statement
 in a more general form.

        \bizrek
          Every $i$-inversion preserves the angle between two $i$-circles.
        \eizrek

 A special case of the previous statement concerns right angles. From
 this it follows that perpendicularity is an invariant of inversion and the following statement holds.

      \bizrek \label{InverzKonfPrav}
      Every $i$-inversion maps two perpendicular $i$-circles to
      a perpendicular $i$-circle.
      \eizrek

 Now we will use the proven statements that hold
 in the Euclidean plane.
 The fact that the circle passing through the center
 of inversion is mapped to a line by this inversion allows
 us to solve various design tasks where instead of the sought circle
 we will first construct its image - a line. This
 also applies to other tasks where we will use inversion to translate a statement that
 concerns a circle into an equivalent statement that
 concerns a line. In both cases, it is desirable to have
 at least one point of the circle, which we will then choose as the center of inversion.

 We should also mention that because of all this, we will often say
 that the image of a line $p$ ($S\notin p$) under inversion $\psi_{S,r}$ is just the circle $j$ ($j\ni S$) instead of $j^S$. And vice versa - in the same
 case we will write $\psi_{S,r}(j)=p$.

\bzgled Given are the circle $k$ and the points $A$ and $B$. Draw
       the circle (line) $x$, which is perpendicular to the circle $k$ and
      passes through the points $A$ and $B$.
      \ezgled


\begin{figure}[!htb]
\centering
\input{sl.inv.9.3.2.pic}
\caption{} \label{sl.inv.9.3.2.pic}
\end{figure}

\textbf{\textit{Solution.}}  Let $x$ be the circle, which is
perpendicular to the circle $k$ and
 passes through the points $A$ and $B$, and $\psi_i$
inversion with respect to any circle $i$ with center $A$ (Figure
\ref{sl.inv.9.3.2.pic}).

First, we assume that $A\notin k$. In this case, $k'=\psi_i(k)$ is
the circle and $x'=\psi_i(x)$ is the line that goes through the
point $B'=\psi_i(B)$ (statement \ref{InverzKroznVkrozn}). By
statement \ref{InverzKonfPrav} from $x \perp k$ it follows that
$x' \perp k'$. Therefore, $x'$ is the line that goes through the
point $B'$ and is perpendicular to the circle $k'$ or goes through
its center.

The line $x'$ can therefore be drawn as the line that goes through
the points $B'=\psi_i(B)$ and the center of the circle
$k'=\psi_i(k)$. In the end, $x=\psi_i(x')$.

But if $A\in k$, then $k'$ is the line (statement
\ref{InverzKroznVkrozn}) and the line $x'$ is drawn as its
perpendicular through the point $B'$.

Since we can always draw one single perpendicular $x'$, there is
exactly one solution also for the $i$-circle $x$. But $x$ is the
circle exactly when $S\notin x'$  or when the points $A$, $B$ and
the center of the circle $k$ are not collinear. In the case of
collinearity of these points, the solution $x$ is the line.
 \kdokaz

     \bzgled \label{TriKroznInv}
    If among three circles $k_1$, $k_2$ and $k_3$
     of a plane two touch each other from the outside,
     the circle $k$, which is determined by the points of contact,
     is perpendicular to each of the circles $k_1$, $k_2$ and $k_3$.
     \ezgled

\begin{figure}[!htb]
\centering
\input{sl.inv.9.3.3.pic}
\caption{} \label{sl.inv.9.3.3.pic}
\end{figure}

\textbf{\textit{Proof.}}   Let $A$ be the point of intersection of
the circles $k_2$ and $k_3$, $B$ the point of intersection of the
circles $k_1$ and $k_3$, and $C$ the point of intersection of the
circles $k_2$ and $k_1$ (Figure \ref{sl.inv.9.3.3.pic}). Then the
circle $k$ is the circumscribed circle of the triangle $ABC$.

Let $\psi_{A,r}$ be the inversion with center $A$ and arbitrary
radius $r$. The circles $k_2$ and $k_3$ touch at the center of the
inversion $A$, so they are mapped by this inversion into the lines
$k'_2$ and $k'_3$ (statement \ref{InverzKroznVkrozn}), which have
no common points, i.e. $k'_2\parallel k'_3$. The circle $k_1$, which
does not pass through the point $A$ and touches the circles $k_2$
and $k_3$ at the points $C$ and $B$, is mapped by the inversion
$\psi_{A,r}$ into the circle $k'_1$ (statement
\ref{InverzKroznVkrozn}), which touches the lines $k'_2$ and $k'_3$
(statement \ref{InverzDotik}) at the points $C'=\psi_{A,r}(C)$ and
$B'=\psi_{A,r}(B)$. Therefore, the lines $k'_2$ and $k'_3$ are
parallel tangents of the circle $k'_1$ at the points $C'$ and $B'$,
and the line segment $B'C'$ is their common perpendicular, the
distance $B'C'$ being the diameter of the circle $k'_1$. The line
$B'C'$ is the image of the circle $k$, which passes through the
center of the inversion $A$ and through the points $B$ and $C$
(statement \ref{InverzKroznVkrozn}), i.e. $\psi_{A,r}(k)=k'=B'C'$.
Since the line $k'=B'C'$ is perpendicular to the line $k'_2$ and
$k'_3$ and to the circle $k'_1$, by statement \ref{InverzKonfPrav}
the circle $k$ is perpendicular to each of the circles $k_1$,
$k_2$ and $k_3$.
 \kdokaz

         \bzgled Two of the four circles touch each other from the
        outside at the points $A$, $B$, $C$ and $D$. Prove that the
        points $A$, $B$, $C$ and $D$ are either concilic or collinear.
         \ezgled

\begin{figure}[!htb]
\centering
\input{sl.inv.9.3.3c.pic}
\caption{} \label{sl.inv.9.3.3c.pic}
\end{figure}

\textbf{\textit{Proof.}}  We mark with $k$ the circle (or line),
which goes through the points $B$, $C$ and $D$. Let $k_1$, $k_2$, $k_3$
and $k_4$ be such circles, that $k_1$ and $k_2$ touch in the point
$A$, $k_1$ and $k_4$ in the point $B$, $k_3$ and $k_4$ in the point $C$ and
$k_3$ and $k_2$ in the point $D$. We mark with $\psi_i$ the inversion with respect to
any circle $i$ with the center in $A$ (Figure
\ref{sl.inv.9.3.3c.pic}). Let $B'$, $C'$ and $D'$ be the images of the points
$B$, $C$ and $D$ and $k'_1$, $k'_2$, $k'_3$ and $k'_4$ be the images
of the circles  $k_1$, $k_2$, $k_3$ and $k_4$ in this inversion. From 
\ref{InverzKroznVkrozn} and \ref{InverzDotik} it follows, that $k'_3$
and $k'_4$ are the circles, which touch in the point $C'$, $k'_1$ and
$k'_2$ are the parallel lines, which are tangent to the circles $k'_4$ and
$k'_3$ in the points $B'$ and $D'$. From the parallelity of the lines  $k'_1$ and
$k'_2$ it follows
 $\angle k'_2,D'C' \cong \angle k'_1,B'C'$, so according to 
 \ref{ObodKotTang}
 they are also consistent with the central angle $D'S_2C'$ and $B'S_1C'$. Because of this
 is $\angle D'C'S_2 \cong \angle B'C'S_1$, which means, that the points $B'$, $C'$ and $D'$ are collinear or $k'$ is a line. According to
  \ref{InverzKroznVkrozn} its image with respect to the inversion $k=\psi_i(k')$ goes through the center of the inversion, so $A,B,C,D\in k$.
 \kdokaz

      \bzgled Let $k$, $l$ and $j$ be three mutually perpendicular
      circles with the common tangents $AB$, $CD$ and $EF$. Prove that
      the circumscribed circles of the triangles $ACE$ and $ADF$ touch in
      the point $A$.
      \ezgled


\begin{figure}[!htb]
\centering
\input{sl.inv.9.3.3a.pic}
\caption{} \label{sl.inv.9.3.3a.pic}
\end{figure}

\textbf{\textit{Proof.}} Let $AB$ be the common chord of the circles $k$ and
$l$,  $CD$ be the common chord of the circles $l$ and $j$ and  $EF$ be
the common chord of the circles $j$ and $k$ (Figure \ref{sl.inv.9.3.3a.pic}).
We denote by $x$ and $y$ the ocrtani circles of the triangles $ACE$ and
$ADF$ and by  $\psi_i$ the inversion with respect to any circle $i$ with
center $A$. Let:
  \begin{eqnarray*}
  && \psi_i:\hspace*{1mm}B,C,D,E,F \mapsto B',C',D',E',F' \hspace*{2mm}
  \textrm{ in
  }\\
  && \psi_i:\hspace*{1mm}k,l,j,x,y \rightarrow k',l',j',x',y'.
  \end{eqnarray*}
   Because $A\in k, l, x, y$ and $A\notin j$,
    by izreku \ref{InverzKroznVkrozn}
   $k'=E'F'$, $l'=C'D'$, $x'=E'C'$ and
   $y'=D'F'$ are lines, $j'\ni C',D',E',F'$ is a circle. From
   the mutual perpendicularity of the circles $k$, $l$ and $j$ and
    the fact that $B\in k\cap l$ it follows that the lines $k'$ and $l'$
    are perpendicular in the point $B'$, and the circle $j'$ is perpendicular to
    both lines $k'$ and $l'$ (izrek \ref{InverzKonfPrav}). This
    means that $C'D'$ and $E'F'$ are perpendicular diameters of the
    circle $j'$, so the quadrilateral $E'C'F'D'$ is a square and $E'C' \parallel
    D'F'$. From the parallelism of the lines $x'$ and $y'$ it follows that their
    inverse images (circles $x$ and $y$) touch at the center
    of inversion $A$.
 \kdokaz

        \bzgled Let $k$ and $l$ be the circles of some plane with centers
        $O$ and $S$. Let $t_i$ be the tangents of the circle $k$,
       which intersect the circle $l$ in the points $A_i$ and $B_i$. Prove that
       there exists a circle that touches all the ocrtani circles of the triangles
       $SA_iB_i$.
        \ezgled

\begin{figure}[!htb]
\centering
\input{sl.inv.9.3.4.pic}
\caption{} \label{sl.inv.9.3.4.pic}
\end{figure}

\textbf{\textit{Proof.}} Let $\psi_l$ be the inversion with respect to the circle $l$ (Figure \ref{sl.inv.9.3.4.pic}). The lines $t_i$ are mapped by this inversion to the circles $t'_i$, which are the inscribed circles of the triangles $SA_iB_i$ (statement \ref{InverzKroznVkrozn}). Because the lines $t_i$ touch the circle $k$, all the circles $t'_i$ touch the circle $k'=\psi_l(k)$  (statement \ref{InverzDotik}).
 \kdokaz

      \bzgled
      Let $ST$ be the diameter of the circle $k$, $t$ the tangent of this circle at the point $T$, and $PQ$ and $PR$ its tangents at the points $Q$ and $R$.
      Prove that if $L$, $Q'$ and $R'$ are the intersections of the altitudes $SP$, $SQ$ and
      $SR$ with the line $t$, then the point $L$ is the midpoint of the segment
      $Q'R'$.
     \ezgled

\begin{figure}[!htb]
\centering
\input{sl.inv.9.3.5.pic}
\caption{} \label{sl.inv.9.3.5.pic}
\end{figure}

\textbf{\textit{Proof.}} Let $\psi_i$ be the inversion with respect to the circle $i(S,ST)$ (Figure \ref{sl.inv.9.3.5.pic}). The point $T$ is
the orthogonal projection of the center of inversion $S$ onto the line $t$. Because
$\psi_i(T)=T$, it follows from  statement \ref{InverzKroznVkrozn} that
$\psi_i(t)=k$ or $k'=\psi_i(k)=t$. Therefore, $\psi_i(Q)=Q'$
and $\psi_i(R)=R'$. We denote $P'=\psi_i(P)$. From $L\in SP$ it follows that the points $S$, $P$, $L$ and $P'$ are collinear. The tangents $q=PQ$ and
$r=PR$ of the circle $k$ are mapped to the circles $q'$ and $r'$, which pass through the point $S$ and touch the line $k'$ at the points $Q'$ and $R'$ (statements
\ref{InverzKroznVkrozn} and \ref{InverzDotik}). Therefore, the line
$t=k'$ is the common tangent of the circles $q'$ and $r'$, which intersect at
the points $S$ and $P'$. The point $L$ lies on their power axis
$SP'$, so
 $|LQ'|^2=|LR'|^2$, that is, the point $L$ is the midpoint of the segment $Q'R'$.
\kdokaz

Let $a$, $b$ and $c_0$ be circles with diameters $PQ$, $PR$ and $RQ$, where $\mathcal{B}(P,R,Q)$. Let $c_0$, $c_1$, $c_2$, ... $c_n$, ... be a sequence of circles on the same side of the line $PQ$, which each circle in the sequence touches the previous one. Prove that the distance from the center of the circle $c_n$ to the line $PQ$ is $n$ times greater than the diameter of this circle\footnote{This problem was considered by \index{Pappus} \textit{Pappus of Alexandria} (4th century). The pattern that is formed by the semicircles $a$, $b$, $c_0$, was investigated by \index{Archimedes} \textit{Archimedes of Syracuse} (3rd century BC) - see Example \ref{arbelos}.}\\
(an example from the book \cite{Cofman}).
\ezgled

\begin{figure}[!htb]
\centering
\input{sl.inv.9.3.6.pic}
\caption{} \label{sl.inv.9.3.6.pic}
\end{figure}

\textbf{\textit{Proof.}} (Figure \ref{sl.inv.9.3.6.pic}). Let $i$ be a circle with its center in point $P$, which is perpendicular to the circle $c_n$ (circle $i$ goes through the points of tangency from point $P$ to circle $c_n$). With the inversion $\psi_i$ with respect to this circle, the line $l=PQ$ and the circles $a$ and $b$ are mapped into lines $l$, $a'$ and $b'$ (statement \ref{InverzKroznVkrozn}); in this case, the lines $a'$ and $b'$ are perpendicular to the line $l$ (statement \ref{InverzKonfPrav}). The circles $c_0$, $c_1$, $c_2$, ..., $c_n$, ... are mapped into congruent circles $c'_0$, $c'_1$, $c'_2$, ..., $c'_n$, ... with the same inversion, which are all tangent to the parallels $a'$, $b'$ (statement \ref{InverzDotik}) and are all congruent circles $c_n$, because $c'_n=\psi_i(c_n)=c_n$ (statement \ref{InverzKroznFiks}). Since $c_0'\perp l$, the center of the circle $c'_0$ lies on the line $l$, therefore the distance from the center of the circle $c_n=c'_n$ to this line is $n$ times larger than the diameter of this circle.
 \kdokaz



        \bzgled
        Let $t$ be a common external tangent circle
         of the circles $k$ and $l$, which are externally tangent in
       point $A$, and $c_0$, $c_1$, $c_2$, ... $c_n$, ... a sequence
       of circles, each circle in
       the sequence is tangent to the previous one, circle $c_0$ is also
       tangent to the circle $t$. Prove that there exists a circle (or a line), which is
       perpendicular to each circle from the given sequence.
        \ezgled


\begin{figure}[!htb]
\centering
\input{sl.inv.9.3.8.pic}
\caption{} \label{sl.inv.9.3.8.pic}
\end{figure}

\textbf{\textit{Proof.}} Let $i$ be an arbitrary circle with center
$A$ and $\psi_i$ the inversion with respect to that circle (Figure
\ref{sl.inv.9.3.8.pic}). In this inversion, the circles $k$ and $l$
are mapped to the lines $k'$ and $l'$ (by \ref{InverzKroznVkrozn}),
which are parallel because the circles $k$ and $l$ touch at the
center of inversion $A$. The circles of the sequence $c_0$, $c_1$,
$c_2$, ..., $c_n$, ... are mapped to the circles $c'_0$, $c'_1$,
$c'_2$, ..., $c'_n$, ..., where each two consecutive ones touch and
all the circles touch the parallels $k'$ and $l'$ (by
\ref{InverzDotik}). Therefore, all the circles of this sequence are
mutually tangent. The line $n'$, determined by their centers, is
the perpendicular of the lines $k'$ and $l'$. The image $n=\psi_i(n')$
of $n'$ is perpendicular to the circles $c'_0$, $c'_1$, $c'_2$, ...,
$c'_n$, ... (by \ref{InverzKonf}). In the end, we can conclude that $n$
is a circle exactly when the line $n'$ does not go through the point
$A$, i.e. when the circles $k$ and $l$ are not tangent, otherwise $n$
is a line. \kdokaz


    \bzgled Let $A$, $B$, $C$ and $D$ be four arbitrary coplanar
    points. Prove that the angle, under which the circumscribed
    circles of the triangles $ABC$ and $ABD$ intersect, is equal to
    the angle, under which the circumscribed
    circles of the triangles $ACD$ and $BCD$ intersect.
    \ezgled


\begin{figure}[!htb]
\centering
\input{sl.inv.9.3.9.pic}
\caption{} \label{sl.inv.9.3.9pic}
\end{figure}

\textbf{\textit{Proof.}} We mark with $k_A$, $k_B$, $k_C$ and $k_D$
the circumscribed circles of the triangles $BCD$, $ACD$, $ABD$ and
$ABC$ (Figure \ref{sl.inv.9.3.9pic}). By an inversion  $\psi_i$ with
respect to an arbitrary circle $i$ with center $A$, our statement is
transformed to the equivalent \ref{ObodKotTang}.
 \kdokaz

\bzgled Let $A$, $B$ and $C$ be points that lie on the line
    $l$,
    and $P$ be a point outside of this line. Prove that the point $P$ and
    the centers of the circumscribed circles of the triangles $APB$, $BPC$ and $APC$
    are four concircular points.
    \ezgled


\begin{figure}[!htb]
\centering
\input{sl.inv.9.3.10.pic}
\caption{} \label{sl.inv.9.3.10pic}
\end{figure}

\textbf{\textit{Proof.}} We mark with $K_A$, $K_B$, $K_C$
the centers of the circumscribed circles of the triangles $BPC$, $APC$ and $APB$
(Figure \ref{sl.inv.9.3.10pic}). If $X=\mathcal{S}_{K_A}(P)$,
$Y=\mathcal{S}_{K_B}(P)$ and $Z=\mathcal{S}_{K_C}(P)$ or
$h_{P,2}(K_A)=X$, $h_{P,2}(K_B)=Y$ and $h_{P,2}(K_C)=Z$, from
the concircularity of the points $X$, $Y$, $Z$ and $P$ it follows that the points
$K_A$, $K_B$, $K_C$ and $P$ are concircular (from $X,Y,Z,P\in k$ it follows that
$K_A,K_B,K_C,P\in h^{-1}_{P,2}$). We thus prove that the points
$X$, $Y$, $Z$ and $P$ are concircular.

Let $\psi_{P,r}$ be an inversion with the center $P$ and an arbitrary
radius $r$ and
 $$\psi_{P,r}:\hspace*{1mm} A,B,C,X,Y,Z\mapsto A',B',C',X',Y',Z'.$$
The inversion $\psi_{P,r}$ maps the circumscribed circle of the triangle $BPC$,
$APC$ and $APB$ into the lines $B'C'$, $A'C'$ and $A'B'$ (the statement
\ref{InverzKroznVkrozn}). Because $PX$, $PY$ and $PZ$ are the diameters of these
circles, the points $X'$, $Y'$ and $Z'$ are the orthogonal projections of the center
of inversion $P$ onto the lines $B'C'$, $A'C'$ and $A'B'$. The line $l$ is mapped by
the statement \ref{InverzKroznVkrozn} into the circumscribed circle of the triangle
$A'B'C'$, which also goes through the point $P$. By the statement \ref{SimpsPrem}
the points $X'$, $Y'$ and $Z'$ lie on the \index{line!Simson's} Simson's line $s$,
which means (the statement \ref{InverzKroznVkrozn}), that the points $X$, $Y$ and
$Z$ lie on the circle $\psi_{P,r}(s)$, which goes through the point $P$, therefore
the points  $X$, $Y$, $Z$ and $P$ are concurrent.
 \kdokaz

%________________________________________________________________________________
 \poglavje{Metric Properties of Inversion} \label{odd9MetrInv}

It is clear from the definition itself that inversion is not an isometry. In the
previous chapter we have found that inversion does not preserve the relation of
collinearity of points, which means that it is not even a transformation of
similarity. However, we are interested in how the distance between points changes,
although in general the distance $AB$ is mapped by inversion into the segment
$A'B'$. The answer will be given by the following important statement.

 \bizrek \label{invMetr} If $A'$ and $B'$ ($A',B'\neq S$) are the images of the points $A$ and
 $B$ by the inversion $\psi_{S,r}$, then it holds:
  $$|A'B'|=\frac{r^2\cdot |AB|}{|SA|\cdot |SB|}$$
  \eizrek

\begin{figure}[!htb]
\centering
\input{sl.inv.9.4.1.pic}
\caption{} \label{sl.inv.9.4.1.pic}
\end{figure}

 \textbf{\textit{Proof.}} We will consider two possible
cases.

\textit{(i)} Let the points $S$, $A$ and $B$ be nonlinear (Figure
\ref{sl.inv.9.4.1.pic}). According to Theorem \ref{invPodTrik}, the
triangles $ASB$ and $B'SA'$ are similar and $A'B':AB=SB':SA$. Since
$B'$ is the image of point $B$ under the inversion $\psi_{S,r}$, it
also holds that $|SB|\cdot |SB'|=r^2$ or $|SB'|=\frac{r^2}{|SB|}$.
From these two relations it follows:
 $$|A'B'|=\frac{|SB'|\cdot |AB|}{|SA|}=
 \frac{r^2\cdot |AB|}{|SA|\cdot |SB|}.$$


\textit{(ii)} Let $S$, $A$ and $B$ be collinear points. Without loss
of generality, we assume that $\mathcal{B}(S,A,B)$ holds. Then
$\mathcal{B}(S,B',A')$ (Theorem \ref{invUrejenost}) holds, so
 $$|A'B'|=|SA'|-|SB'|=
\frac{r^2}{|SA|}-\frac{r^2}{|SB|}= \frac{(|SB|- |SA|)\cdot
r^2}{|SA|\cdot |SB|}
 = \frac{r^2\cdot |AB|}{|SA|\cdot |SB|} ,$$ which had to be proven. \kdokaz

 From the previous theorem we see that the distance between the images of points $A'$ and $B'$
   increases if at least one of the originals $A$ and $B$
 approaches the center of inversion $S$, which is logical, because we saw that the image of this center is a point at infinity.

 In section \ref{odd7Ptolomej} we have already proven Ptolomej's \index{theorem!Ptolomej's general} theorem (\ref{izrekPtolomej}) which relates to
  the chordal
 quadrilaterals. Now we will generalize this statement.

\bizrek
 If $ABCD$ is an arbitrary convex quadrilateral, then:
 $$|AB|\cdot |CD|+|BC|\cdot |AD|\geq |AC|\cdot |BD|.$$
  Equality holds if and only if $ABCD$ is a chordal
quadrilateral.
 \eizrek

\begin{figure}[!htb]
\centering
\input{sl.inv.9.4.2a.pic}
\caption{} \label{sl.inv.9.4.2a.pic}
\end{figure}

\textbf{\textit{Proof.}} Let $\psi_{A,r}$ be the inversion with the center in the point $A$ (Figure \ref{sl.inv.9.4.2a.pic}). With $k$ we denote the circumscribed circle of the triangle $ABD$. Let $B'$, $C'$ and $D'$ be the images of the points $B$, $C$ and $D$ and $k'$ be the image of the circle $k$ under the inversion $\psi_{A,r}$. By the theorem \ref{InverzKroznVkrozn} the $k'$ is a line, which contains the points $B'$ and $D'$.
 From the previous theorem \ref{invMetr} it follows:
 $$|B'C'|=\frac{r^2\cdot |BC|}{|AB|\cdot |AC|}, \hspace*{2mm}
  |C'D'|=\frac{r^2\cdot |CD|}{|AC|\cdot |AD|} \hspace*{1mm} \textrm{ and }
 \hspace*{1mm} |B'D'|=\frac{r^2\cdot |BD|}{|AB|\cdot |AD|}.$$
  For the points $B'$, $C'$ and $D'$ the triangle inequality
  \ref{neenaktrik} holds:
   $$B'C'+C'D'\geq B'D',$$
where the equality holds exactly when the points $B'$, $C'$ and $D'$
are collinear (and $\mathcal{B}(B',C',D')$) or when $C'\in k'$ (and
$\mathcal{B}(B',C',D')$). This is exactly the case when the circle $k$
contains the point $C$ or when the quadrilateral $ABCD$ is cyclic (and
convex).
 We can write the previous inequality
also in the form:
 $$\frac{r^2\cdot |BC|}{|AB|\cdot |AC|}+\frac{r^2\cdot |CD|}{|AC|\cdot |AD|}
  \geq\frac{r^2\cdot |BD|}{|AB|\cdot |AD|}, \hspace*{2mm} \textrm{or}$$
 $$|AB|\cdot |CD|+|BC|\cdot |AD|\geq |AC|\cdot |BD|,$$ which was to be proven. \kdokaz

 The next theorem will be a generalization of the example \ref{zgledTrikABCocrkrozP} or
 \ref{zgledTrikABCocrkrozPPtol}.

\bzgled \label{zgledABCPinv} Let $k$ be the circumscribed circle of the regular triangle $ABC$. If $P$ is an arbitrary point in the plane of this triangle, then the following equivalence holds: the point $P$ does not lie on the circle $k$, exactly when there exists a triangle with sides, which are congruent to
the distances $PA$, $PB$ and $PC$.
 \ezgled

\begin{figure}[!htb]
\centering
\input{sl.inv.9.4.3.pic}
\caption{} \label{sl.inv.9.4.3.pic}
\end{figure}

\textbf{\textit{Proof.}} Let $\psi_{P,r}$ be the inversion with the center in
the point $P$ (Figure \ref{sl.inv.9.4.3.pic}). Let $A'$, $B'$ and
$C'$ be the images of the points $A$, $B$ and $C$ and $k'$ be the image of the circle $k$ under the inversion $\psi_{P,r}$.
 From  the formula \ref{invMetr} it follows:
 $$|A'B'|=\frac{r^2\cdot |AB|}{|PA|\cdot |PB|}, \hspace*{2mm}
  |B'C'|=\frac{r^2\cdot |BC|}{|PB|\cdot |PC|} \hspace*{1mm} \textrm{ and }
 \hspace*{1mm} |A'C'|=\frac{r^2\cdot |AC|}{|PA|\cdot |PC|}.$$
Because the triangle $ABC$ is right, from the previous three relations it follows:
$$A'B':B'C':A'C'=PC:PA:PB.$$

Now we can start with the proof of the desired equivalence:
 \begin{eqnarray*}
 P \notin k & \Leftrightarrow& \psi_{P,r}(k) \textrm{ represents a circle
 \hspace*{2mm}(formula \ref{InverzKroznVkrozn})}\\
 & \Leftrightarrow& \textrm{the points }A', B', C' \textrm{ are non-collinear}\\
 & \Leftrightarrow& \textrm{the distances }A'B', B'C', A'C' \textrm{ are the sides
  of a triangle}\\
 & \Leftrightarrow& \textrm{the distances }PC, PA, PB \textrm{ are the sides
  of a triangle,}
 \end{eqnarray*}
  which was to be proven. \kdokaz

  Another generalization will be related to the Torricelli point (formula
  \ref{izrekTorichelijev}).

   \bizrek \label{izrekToricheliFerma}
    The point, for which the sum of the distances from
 the vertices of a triangle is minimal, is
the \index{Torricelli, E.} \textit{E. Torricelli}'s\footnote{(1608--1647), Italian mathematician and physicist.} \index{točka!Torricellijeva}\index{točka!Fermatova}\pojem{Torricelli point} of this triangle.
 \eizrek

\begin{figure}[!htb]
\centering
\input{sl.inv.9.4.4.pic}
\caption{} \label{sl.inv.9.4.4.pic}
\end{figure}

\textbf{\textit{Proof.}} Use the same notation as in Theorem
 \ref{izrekTorichelijev} (Figure \ref{sl.inv.9.4.4.pic}). We have
proven that the circles $k$, $l$ and $j$ intersect in Torricelli's
point $P$ of the triangle $ABC$. Because $BEC$ is a right triangle,
from Theorem \ref{zgledTrikABCocrkrozP} it follows that
$|PB|+|PC|=|PE|$ i.e.:
 $$|PA|+|PB|+|PC|=|AE|.$$
 We will prove that this sum is minimal only for Torricelli's point
 $P$. Let $P'\neq P$. Then the point $P'$ does not lie on any of the circles $k$, $l$ and $j$. Without loss of generality, let $P' \notin k$. From the previous Theorem \ref{zgledABCPinv} it follows that there exists a triangle with sides $P'E$,
$P'B$ and $P'C$. Because of this and the triangle inequality (Theorem
\ref{neenaktrik}) we have $|P'B| + |P'C|
> |P'E|$ i.e.:
 $$|P'A| + |P'B| + |P'C| > |P'A| + |P'E| \geq |AE|=|PA|+|PB|+|PC|,$$ which was to be proven. \kdokaz




              \bnaloga\footnote{37. IMO, India - 1996, Problem 2.}
                Let $P$ be a point inside triangle $ABC$ such that
             $$\angle APB -\angle ACB = \angle APC -\angle ABC.$$
         Let $D$, $E$ be the incentres of triangles $APB$, $APC$, respectively. Show
that $AP$, $BD$, $CE$ meet at a point.
        \enaloga

\begin{figure}[!htb]
\centering
\input{sl.inv.9.4.5.pic}
\caption{} \label{sl.inv.9.4.5.pic}
\end{figure}

 \textbf{\textit{Proof.}}
Let $r>\max\{|AB|, |AC|\}$ be an arbitrary number and $\psi_{A,r}$ be
the inversion with the center in the point $A$ and the radius $r$
and $B'$, $C'$ and $P'$ be the images of the points $B$, $C$ and $P$
under the inversion $\psi_{A,r}$ (Figure \ref{sl.inv.9.4.5.pic}).

According to the statement \ref{invPodTrik} the triangles $ABC$ and $AC'B'$ are similar, so:
 $$\angle AC'B'\cong \angle ABC \textrm{ in } \angle
AB'C'\cong \angle ACB.$$
 From the similarity of $\triangle ABP \sim \triangle
AP'B'$ and $\triangle ACP \sim \triangle AP'C'$ it follows:
 $$\angle AB'P'\cong \angle APB \textrm{ in } \angle
AC'P'\cong \angle APC.$$
 If we use the four relations of congruence of angles and
 the initial condition from the task ($\angle APB -\angle ACB = \angle APC -\angle
 ABC$), we get:
  \begin{eqnarray*}
   \angle C'B'P' &=& \angle AB'P'-\angle AB'C' = \angle APB - \angle
   ACB = \\ &=& \angle APC -\angle ABC = \angle AC'P' -\angle AC'B'
   =\\
   &=& \angle B'C'P'
  \end{eqnarray*}
Now from $\angle C'B'P'\cong \angle B'C'P'$ it follows $P'B'\cong
P'C'$. If we use the relation from the statement \ref{invMetr}, we get
 $\frac{|PB| \cdot r^2}{|AP|\cdot |AB|}=\frac{|PC| \cdot r^2}{|AP|\cdot
 |AC|}$ or:
 $$\frac{|PB| }{ |AB|}=\frac{|PC| }{ |AC|}.$$
 Let $BD \cap AP =X$ and $CE \cap AP =Y$. It is also necessary to
 prove that $X=Y$. Because $D$ and $E$ are the centers
 of inscribed circles of triangles $APB$
and $APC$, the lines $BD$ and $CE$ are the (internal) angle bisectors of
$ABP$ and $ACP$. Therefore (statement \ref{HarmCetSimKota})
 $$ \frac{\overrightarrow{PX}}{\overrightarrow{XA}}=\frac{PB}{BA}
 =\frac{PC}{CA}=\frac{\overrightarrow{PY}}{\overrightarrow{YA}}.$$
 From $ \frac{\overrightarrow{PX}}{\overrightarrow{XA}}
 =\frac{\overrightarrow{PY}}{\overrightarrow{YA}}$ it follows (statement
 \ref{izrekEnaDelitevDaljice}) $X=Y$, which means that the lines $AP$, $BD$ and $CE$ intersect in the same
point $X=Y$.
 \kdokaz

%________________________________________________________________________________
\poglavje{Inversion and Harmonic Conjugate Points}
\label{odd9InvHarm}

In section \ref{odd7Harm} we saw that a harmonic quadruple of points
is defined in two equivalent ways. For four collinear points
$A$, $B$, $C$ and $D$, $\mathcal{H}(A,B;C,D)$, if one of the two (equivalent) conditions is fulfilled:
\begin{itemize}
  \item $\frac{\overrightarrow{AC}}{\overrightarrow{CB}}=-
  \frac{\overrightarrow{AD}}{\overrightarrow{DB}}$,
  \item there exists a quadrilateral $ABCD$, such that
  $PQ \cap RS=A$, $PS \cap QR=B$, $PR \cap AB=C$ and $QS \cap AB=D$.
\end{itemize}


Now we will investigate some possibilities for equivalent definitions
 of the relation of harmonic quadruple of points - first with the help of inversion.

 \bizrek \label{invHarm} Let $AB$ be the diameter of the circle $i$ and $C$ and $D$ the points of the line $AB$, which are different from $A$ and $B$. If $\psi_i$
 is the inversion with respect to the circle $i$, then it holds:
 $$\mathcal{H}(A,B;C,D) \hspace*{1mm} \Leftrightarrow \hspace*{1mm}
 \psi_i(C)=D.$$
 \eizrek
\textbf{\textit{Proof.}} Let $O$ be the center of the line $AB$ (Figure
\ref{sl.inv.9.6.1.pic}). Then it is:
 \begin{eqnarray*}
 \mathcal{H}(A,B;C,D) &\hspace*{1mm} \Leftrightarrow
 \hspace*{1mm}&
  \frac{\overrightarrow{AC}}{\overrightarrow{CB}}=-
  \frac{\overrightarrow{AD}}{\overrightarrow{DB}}\\
 &\hspace*{1mm} \Leftrightarrow
 \hspace*{1mm}&
  \frac{\overrightarrow{OC}-\overrightarrow{OA}}
  {\overrightarrow{OB}-\overrightarrow{OC}}=-
  \frac{\overrightarrow{OD}-\overrightarrow{OA}}
  {\overrightarrow{OB}-\overrightarrow{OD}}\\
 &\hspace*{1mm} \Leftrightarrow
 \hspace*{1mm}&
  \frac{\overrightarrow{OC}+\overrightarrow{OB}}
  {\overrightarrow{OB}-\overrightarrow{OC}}=-
  \frac{\overrightarrow{OD}+\overrightarrow{OB}}
  {\overrightarrow{OB}-\overrightarrow{OD}}\\
 &\hspace*{1mm} \Leftrightarrow
 \hspace*{1mm}& \overrightarrow{OC} \cdot \overrightarrow{OD} =
 OB^2\\
 &\hspace*{1mm} \Leftrightarrow
 \hspace*{1mm}& \psi_i(C)=D,
  \end{eqnarray*}
  which had to be proven.  \kdokaz

\begin{figure}[!htb]
\centering
\input{sl.inv.9.6.1.pic}
\caption{} \label{sl.inv.9.6.1.pic}
\end{figure}

The next statement will give the fourth possibility of an equivalent definition
 of the relation of harmonic quadruple of points.

\bizrek \label{harmPravKrozn}
 Let $A$, $B$, $C$ and $D$ be different collinear points and $k$ and $l$
  be circles over diameters $AB$ and
$CD$. Then the equivalence holds:
$$\mathcal{H}(A,B;C,D) \Leftrightarrow k\perp l.$$
 \eizrek

 \textbf{\textit{Proof.}} Let $O$ and $S$ be the centers of the circles $k$ and $l$
 (Figure \ref{sl.inv.9.6.1.pic}). From both sides of the equivalence
 it follows that the circles
intersect. One of their intersection points is denoted by $T$. From the previous statement \ref{invHarm} it is:
 $$\mathcal{H}(A,B;C,D) \hspace*{1mm} \Leftrightarrow \hspace*{1mm}
 \psi_k(C)=D.$$
 So we only need to prove:
$$k\perp l \hspace*{1mm} \Leftrightarrow \hspace*{1mm}
 \psi_k(C)=D.$$
 But now we have:
 \begin{eqnarray*}
 k\perp l \hspace*{1mm} &\Leftrightarrow& \hspace*{1mm}
 OT\perp TS \hspace*{2mm}
 \textrm{ (statement \ref{pravokotniKroznici})}\\
 &\Leftrightarrow& \hspace*{1mm} OT \textrm{ is a tangent of the circle
 }l \hspace*{2mm}\textrm{ (statement \ref{TangPogoj})}\\
&\Leftrightarrow& \hspace*{1mm} \overrightarrow{OC}\cdot \overrightarrow{OD}
= OT^2\hspace*{2mm}\textrm{ (statement \ref{izrekPotenca})}\\
&\Leftrightarrow& \hspace*{1mm} \psi_k(C)=D,
 \end{eqnarray*}
 which was to be proven.  \kdokaz

 From the previous statement \ref{harmPravKrozn} it directly follows (already proven before) that
  from $\mathcal{H}(A,B;C,D)$ it follows $\mathcal{H}(C,D;A,B)$ or
 $\mathcal{H}(B,A;C,D)$.

We already know that for three collinear points $A$, $B$ and $C$,
  where point $C$ is not the midpoint of line segment $AB$,
 there is only one point $D$, so that $\mathcal{H}(A,B;C,D)$ holds. So for three
given points in a harmonic quadruple of points the fourth is uniquely determined.
One of the possible constructions of this point gives us the previous statement
\ref{invHarm} ($D=\psi_k(C)$). But only two points $A$ and $B$ are not
enough to determine the other pair of points $C$ and $D$. There are infinitely many such pairs of points $(C,D)$, for which $\mathcal{H}(A,B;C,D)$ holds (Figure
\ref{sl.inv.9.6.2.pic}). For their determination
 another condition is needed. We will consider such conditions in
 the next two examples.

\begin{figure}[!htb]
\centering
\input{sl.inv.9.6.2.pic}
\caption{} \label{sl.inv.9.6.2.pic}
\end{figure}

 \bzgled
  Given are points $A$, $B$ and $S$. Construct such points $C$ and $D$,
    that $S$ is the midpoint
of line segment $CD$ and $\mathcal{H}(A,B;C,D)$ holds.
 \ezgled

 \textbf{\textit{Solution.}}
Let $k$ be a circle with diameter $AB$. If $l$ is a circle with
diameter $CD$, then $S$ is the center of this circle. From 
\ref{harmPravKrozn} it follows that $k \perp l$ (Figure
\ref{sl.inv.9.6.3.pic}). So it is enough to construct the circle $l$, because
then the points $C$ and $D$ are the intersections of this circle with the line segment $AB$. But we can construct the circle $l$, if we first construct the tangent
to the circle $k$ from point $S$ in point $T$.
 \kdokaz

\begin{figure}[!htb]
\centering
\input{sl.inv.9.6.3.pic}
\caption{} \label{sl.inv.9.6.3.pic}
\end{figure}


 \bzgled \label{harmDaljicad}
  Given is a line segment $d$ and points $A$ and $B$. Construct such points $C$
   and $D$, that $\mathcal{H}(A,B;C,D)$ holds and $CD\cong d$.
 \ezgled
  \textbf{\textit{Solution.}}

It is enough to construct the center $S$ of the line $CD$, and then continue as in
the previous example. The right triangle $OTS$ can
be constructed because both sides are known $|OT| = \frac{1}{2}\cdot
|AB|$ and $|ST| = \frac{1}{2}\cdot |CD|= \frac{1}{2}\cdot |d|$. From
it we get the hypotenuse $OS$ (Figure \ref{sl.inv.9.6.3.pic}).
 \kdokaz

In the next examples we will look at the use of the previous two
constructions.



\bzgled
  Draw a triangle with the data $v_a$, $l_a$ and $b-c$.
 \ezgled

\textbf{\textit{Solution.}}
 We will use the labels from the big task \ref{velikaNaloga}
(Figure \ref{sl.inv.9.6.4.pic}). First we can draw the right triangle $AA'E$, because $AA'\cong v_a$ and $AE\cong l_a$. By
the example \ref{harmVelNal} is $\mathcal{H}(A',E;P,Pa)$. From the big
task it follows $PP_a=b -c$, so we can by the previous example
\ref{harmDaljicad}  construct the points $P$ and $P_a$. Then
we draw the center $S$ of the inscribed circle of the triangle $ABC$, at
the end  the inscribed circle, their tangents from the point $A$ and
mark $B$ and $C$.
 \kdokaz

\begin{figure}[!htb]
\centering
\input{sl.inv.9.6.4.pic}
\caption{} \label{sl.inv.9.6.4.pic}
\end{figure}



\bzgled
  Draw a triangle with the data $v_a$, $a$ and $r+r_a$.
 \ezgled

\textbf{\textit{Solution.}} Also in this case we will use
the labels from the big task \ref{velikaNaloga} (Figure
\ref{sl.inv.9.6.4.pic}). By the example \ref{harmVelNal} is
$\mathcal{H}(A,L;A',La)$. Because $AA'\cong v_a$ and $LL_a =r+r_a$,
we can by the example \ref{harmDaljicad} first draw the line
$AA'\cong v_a$, then the points $L$ and $L_a$. So we get
$r\cong LA'$ and $r_a\cong L_aA'$.

From the great task it follows that $RR_a = a$. This means that we can draw a rectangular trapezoid $SRR_aS_a$ ($RR_a=a$, $SR=r$ and $S_aR_a=r_a$). Then we construct the inscribed circle $k(S,r)$ and the circumscribed circle $k_a(S_a,r_a)$ and finally their common tangents (two external and one internal), which are the sides of the triangle $ABC$.
 \kdokaz

%________________________________________________________________________________
 \poglavje{Feuerbach Points}
\label{odd9Feuerbach}

In this section we will consider some properties of the inscribed and circumscribed circles of a triangle.

 \bzgled Let $ABC$ be a triangle with a semiperimeter $s$ and $D$ and $E$ such points on the line $BC$ that $|AD|=|AE|=s$. Prove that the circumscribed circle of the triangle $ADE$ and the circumscribed circle of the triangle $ABC$ touch the side $BC$.
 \ezgled

\begin{figure}[!htb]
\centering
\input{sl.inv.9.5.0.pic}
\caption{} \label{sl.inv.9.5.0.pic}
\end{figure}

 \textbf{\textit{Proof.}} Let $l$ be the circumscribed circle of the triangle $AED$ and $P_a$, $Q_a$ and $R_a$ the points in which the circumscribed circle $k_a$ of the triangle $ABC$ touches the side $BC$ and the sides $AC$ and $AB$ (Figure \ref{sl.inv.9.5.0.pic}). With $i$ we denote the circle with center $A$ and radius $AE$ (or $AD$).
From the great task \ref{velikaNaloga} it follows that $|AR_a|=|AQ_a|=s=|AE|=|AD|$, which means that the points $R_a$ and $Q_a$ lie on the circle $i$.
By the theorem \ref{pravokotniKroznici} the circles $i$ and $k_a$ are perpendicular. Therefore, the circle $k_a$ is mapped to itself by the inversion $\psi_i$ (theorem \ref{InverzKroznFiks}), and the line $BC$ is mapped to the circle $l$ (theorem \ref{InverzKroznVkrozn}).
Since the line $BC$ touches the circle $k_a$ at the point $P_a$, their images (the circles $l$ and $k_a$) touch at the point $T=\psi_i(P_a)$ (theorem \ref{InverzDotik}).
  \kdokaz

\bzgled \label{InvOcrtVcrt} Let $P$, $Q$ and $R$ be points in which the
incircle of triangle
 $ABC$ touches the sides of the triangle. Prove that the orthocenter of
 triangle
 $PQR$ and the centers of
 the circumscribed and incircle of triangle $ABC$ are collinear points.
 \ezgled

  \begin{figure}[!htb]
\centering
\input{sl.inv.9.5.0a.pic}
\caption{} \label{sl.inv.9.5.0a.pic}
\end{figure}

 \textbf{\textit{Proof.}} Let $l(O,R)$ and $k(S,r)$ be the circumscribed and
 incircle of triangle
 $ABC$ and $\psi_k$ be the inversion with respect to the circle $k$
 (Figure \ref{sl.inv.9.5.0a.pic}).

We also denote $A'=SA \cap QR$, $B'=SB \cap PR$ and $C'=SC \cap PQ$.
From the similarity of triangles $ARA'$ and $AQA'$ (the \textit{SAS} theorem \ref{SKS})
it follows that the point $A'$ is the center of the side $QR$ and that $AS
\perp RQ$. Similarly, it also follows that points $B'$ and $C'$
are the centers of segments $PR$ and $PQ$ and that $BS \perp PR$ and $CS
\perp PQ$.

From the process of constructing the image of a point in an inversion it follows:
 $$\psi_k:A,B,C \mapsto A',B',C'.$$
 Therefore, the circumscribed circle $l$ with inversion $\psi_k$ is mapped  to
 the circumscribed circle of triangle $A'B'C'$ (the \ref{InverzKroznVkrozn} theorem). This is Euler's circle $e_1$ of triangle $PQR$, because it goes through
 the centers of its sides. The center of this circle - point
 $E_1$ - lies on Euler's line $e_p$ of triangle $PQR$, which is
 determined by its orthocenter $V_1$ and the center
 of the circumscribed circle $S$.

 We also prove that point $O$ lies on line $e_p$. Although the center
 of circle $l$ (point $O$) is not mapped to the center of the circle
 (point $E_1$) by inversion, point $E_1$ lies on line $SO$. This
 also means that point $O$ lies on line $SE_1=e_p$, so points $V_1$, $O$ and $S$ are collinear.
  \kdokaz

 The next simple statement is just an introduction to the so-called Feuerbach's
 theorem.

\bizrek Let $k$ be the inscribed circle of the triangle $ABC$ and $k_a$ the circle drawn through its side $BC$. If $A_1$ is the center of the side $BC$ and $i$ is the circle with center $A_1$ and radius $\frac{1}{2}|b-c|$, then
$$\psi_i:k,k_a\rightarrow k, k_a.$$
 \eizrek
 \textbf{\textit{Proof.}} Let $P$ and $P_a$ be the points of tangency of the circles
$k$ and  $k_a$ with its side $BC$ (Figure
\ref{sl.inv.9.5.1.pic}). From the great task \ref{velikaNaloga}
it follows that the points $P$ and $P_a$ lie on the circle $i$. Therefore, by
the theorem \ref{pravokotniKroznici} the circles $k$ and  $k_a$
are perpendicular to the circle $i$, so $\psi_i:k,k_a\rightarrow k,
k_a$ (theorem \ref{InverzKroznFiks}).
 \kdokaz

\begin{figure}[!htb]
\centering
\input{sl.inv.9.5.1.pic}
\caption{} \label{sl.inv.9.5.1.pic}
\end{figure}


         \bizrek \index{krožnica!Eulerjeva}
         Euler's circle of the triangle intersects
        the
         inscribed
        circle and all three of the drawn
        circles of this triangle\footnote{The points of intersection are called
        \index{točka!Feuerbachova} \pojem{Feuerbach points} of this
        triangle. \index{Feuerbach, K. W.} \textit{K. W. Feuerbach}
        (1800--1834), German mathematician, who proved this theorem in 1822.}.
        \eizrek

\textbf{\textit{Proof.}}
 We will use the notation from the great task
 \ref{velikaNaloga}. Let $e$ be Euler's circle of this
 triangle (Figure \ref{sl.inv.9.5.1.pic}). From the previous theorem the inversion $\psi_i$ with respect to
 the circle $i$ with center  $A_1$ and radius
 $\frac{1}{2}|b-c|$ (or containing the points $P$ and $P_a$) maps the circles
 $k$ and $k_a$ to themselves. We determine the image of Euler's circle $e$
 under the inversion $\psi_i$.

The circle $e$ contains the center of inversion $A_1$, so it is mapped to some line $e'$ by inversion (statement \ref{InverzKroznVkrozn}). We only need to prove that the line $e'$ is tangent to the circles $k$ and $k_a$ (statement \ref{InverzDotik}). First, the line $e'$ contains the point $\psi_i(A')=E$, which is the intersection of the internal angle bisector at the vertex $A$ with the side $BC$. (statements \ref{harmVelNal} and \ref{invHarm}). But the line $e'$ also contains the points $\psi_i(B_1)=B'_1$ and $\psi_i(C_1)=C'_1$. By statement \ref{invPodTrik}, the triangles $A_1B_1C_1$ and $A_1C'_1B'_1$ are similar. It follows that the line $e'$ with the line $AB$ determines the angle $ACB$. Because the line $e'$ intersects the line $BC$ at the point $E$, which lies on the bisector of the angle $BAC$, $e'$ represents the second common tangent to the circles $k$ and $k_a$. We denote by $F'$ and $F'_a$ the points of tangency of the line $e'$ with the circles $k$ and $k_a$. Then the circle $e$ is tangent to the circles $k$ and $k_a$ at the points $F=\psi_i(F')$ and $F_a=\psi_i(F'_a)$. In the same way, we prove that $e$ is also tangent to the circles $k_b$ and $k_c$.
 \kdokaz


%________________________________________________________________________________
 \poglavje{Stainer's Theorem}
\label{odd9Stainer}

In this section, we will elegantly (with the help of inversion) prove Stainer's statement, which is related to the problem of the existence of a sequence of circles that are cyclically tangent to each other, and also tangent to two given, non-concentric, circles. It is clear that such a sequence does not exist in general (for any two non-concentric circles). But we will prove that if there is at least one such sequence for the two circles $k$ and $l$, then we can choose the initial element of this sequence as any circle that is tangent to the circles $k$ and $l$.

 We will first solve two
auxiliary tasks.



 \bzgled \label{StainerjevLema1}
 Let $k_1(S_1,r_1)$ and $k_2(S_2,r_2)$ be two circles of some plane. Draw a circle with
center on the line $S_1S_2$, which is perpendicular to both circles.
 \ezgled

\begin{figure}[!htb]
\centering
\input{sl.inv.9.7.0.pic}
\caption{} \label{sl.inv.9.7.0.pic}
\end{figure}


 \textbf{\textit{Solution.}}
 Let $k_1(S_1,r_1)$ and $k_2(S_2,r_2)$ be any
 circles. In the case that a solution exists, the center
 of the sought circle lies
on the power line of the two circles (Figure
\ref{sl.inv.9.7.0.pic}). This is the case when the intersection $S$
of the power line $l$ and the central line $S_1S_2$ is an external point of both
circles or when the circles $k_1$ and $k_2$ are non-concentric (from the inside or from the outside). If we draw a tangent to one of the circles from the point $S$
(e.g. $k_1$), we also get the radius $r=ST_1$ of the circle
$k(S,r)$, where the point $T_1$ is the point of tangency of the tangent from $S$ on the
circle $k_1$. Let the point $T_2$ be the point of tangency of the tangent from $S$ on the
circle $k_2$. Since the point $S$ lies on the power line of the two circles, we have:
 $$|ST_1|=p(S,k_1)=p(S,k_2)=|ST_2|$$
 the circle $k(S,r)$ is perpendicular to the circle $k_1(S_1,r_1)$ and
 $k_2(S_2,r_2)$ by \ref{pravokotniKroznici}.
 \kdokaz


 %slika

 \bzgled  \label{StainerjevLema2}
 If $k$ is a circle inside the circle $l$, there exists an inversion that
  maps the
circles $k$ and $l$ to two concentric circles.
  \ezgled


 \begin{figure}[!htb]
\centering
\input{sl.inv.9.7.1.pic}
\caption{} \label{sl.inv.9.7.1.pic}
\end{figure}

\textbf{\textit{Proof.}}
 Let $p$ be a line, determined by the centers of the circles $k$ and $l$;
 it is also perpendicular
to both circles. From the previous example \ref{StainerjevLema1} it follows that
there exists a circle $n$ with the center on the line $p$, which is
perpendicular to the circles $k$ and $l$ (Figure
\ref{sl.inv.9.7.1.pic}).
 Because the center of the circle $n$ lies on
the line $p$, the circle $n$ is also perpendicular to the line $p$. With $I$
and $J$ we denote the intersections of the circle $n$ with the line $p$. Let $i$
be an arbitrary circle with the center $I$ and $\psi_i$ the inversion with respect to this circle. Because the inversion preserves angles, the line $p$ and
the circle $n$ (without the point $I$) with $\psi_i$ are mapped to the perpendicular line $p'=p$
and $n'$, the circles $k$ and $l$ to the circles $k'$ and $l'$, which are
perpendicular to these two lines (the statement \ref{InverzKroznVkrozn} and
\ref{InverzKonf}). Therefore, the circles $k'$ and $l'$ are concentric.
 \kdokaz


 \bzgled
 (Steiner's statement \index{statement!Steiner's}
 \footnote{\index{Steiner, J.}
 \textit{J. Steiner} (1769--1863), Swiss mathematician.}.)
 Let $l$ be a circle inside the circle $k$ and $a_1$, $a_2$,
..., $a_n$ a sequence of circles, each of which touches the circles $k$ and $l$,
and each circle touches the adjacent circle in the sequence.
 If the circles $a_n$
and $a_1$ touch, this property is independent of the choice of the first circle
$a_1$ of this sequence.
  \ezgled

 \begin{figure}[!htb]
\centering
\input{sl.inv.9.7.2.pic}
\caption{} \label{sl.inv.9.7.2.pic}
\end{figure}


 \textbf{\textit{Proof.}}
 If we use the previous statement  \ref{StainerjevLema2}, the proposition
 is translated into the case when the circles are concentric (the touch of the circles is
  an invariant of the inversion \ref{InverzDotik}). But in this case the new sequence of circles
 is simply obtained from the first one by rotation with the center in the common center of the circles
  $\psi_i(k)$ and $\psi_i(l)$ (Figure
\ref{sl.inv.9.7.2.pic}).
 \kdokaz


%________________________________________________________________________________
\poglavje{Problem of Apollonius}
\label{odd9ApolDotik}

Now we will deal with so-called \index{problem!Apolonijev}
  Apolonijev's
 problems of the intersection
 of circles, which can be elegantly
solved by using inversion. These are problems of the following form:

\textit{
 \vspace*{2mm}
 Draw a circle $k$, which satisfies three
conditions, each of which has one of the following forms:
\vspace*{2mm}
\begin{itemize}
  \item contains a given point,
  \item touches a given line,
  \item touches a given circle.
\end{itemize}}
 It is clear that all the points, lines and circles mentioned in the above conditions lie
 in the same plane. We have already encountered some of these problems. For example, to construct a circle that
 contains a given point and touches a given line. It is not difficult
 to determine that there are ten Apollonius's problems. We usually list them in the following order:
\begin{enumerate}
  \item draw a circle that contains three given points. $(A,B,C)$,
\item draw a circle that contains a given point and touches a given
line $(A,B,p)$,
  \item draw a circle that contains a given point and touches
  a given
circle $(A,B,k)$,
  \item draw a circle that contains a given point and
touches  two given lines $(A,p,q)$,
  \item draw a circle that contains a given point and
  touches a given line and a given
circle $(A,p,k)$,
  \item draw a circle that contains a given point and
touches two given circles $(A,k_1,k_2)$,
  \item draw a circle that touches three given lines
  $(p,q,r)$,
  \item draw a circle that
touches two given lines and a given circle $(p,q,k)$,
  \item draw a circle that touches a given line and two given circles
$(p,k_1,k_2)$,
  \item draw a circle that touches
three given circles $(k_1,k_2,k_3)$.
\end{enumerate}

We immediately see that the first and seventh problem are trivial,
 and also all the other problems can be
solved without using inversion. However, inversion gives us a general
method for solving them. This method is based on the fact that in a certain case, a circle is transformed into a line by inversion (Theorem
\ref{InverzKroznVkrozn}). We will illustrate it with an example of the fifth Apollonius' problem:

\bzgled
 Draw a circle that contains the given point $A$ and
  touches the given line $p$ and the given
circle $k$.
 \ezgled


 \begin{figure}[!htb]
\centering
\input{sl.inv.9.8.1.pic}
\caption{} \label{sl.inv.9.8.1.pic}
\end{figure}


 \textbf{\textit{Solution.}}

Assume that $x$ is a circle that contains the point $A$ and touches the line $p$ and the circle $k$ (Figure
\ref{sl.inv.9.8.1.pic}). We will consider the general case, so that the point $A$ does not lie on neither the line $p$ nor the circle $k$. With
$i$ we denote the circle with the center $A$ and an arbitrary radius
$r$. Let $\psi_i$ be the inversion with respect to this circle and $p'$,
$k'$ and $x'$ be the images of the line $p$ and the circles $k$ and $x$ under this inversion. Because $A\notin p,k$ and $A\in x$, $p'$ and $k'$ are circles, $x'$ is a line (Theorem \ref{InverzKroznVkrozn}).
The line $p$ and the circle $x$ have exactly one common point, therefore this is also true for the images $p'$ and $x'$, thus the line $x'$ is tangent to the circle $p'$. Similarly, the line $x'$ is also tangent to the circle
$k'$. Therefore, the problem is reduced to constructing the line $x'$, which is the common tangent of the circles $p'$ and $k'$ (Example \ref{tang2ehkroz}).
Then $x= \psi_i^{-1}(x')= \psi_i(x')$.

The task has zero, one, two, three or four solutions, depending on the mutual position of the circles $p'$ and $k'$ or the number of their common tangents.

In the case when $A\in p$ or $A\in k$, the line $x'$ is a line that
touches one circle ($k'$ or $p'$) and is parallel to one
line ($p'$ or $k'$).

If $A\in p \cap k$, the task has no solution or there are infinitely many, depending on whether the circles intersect or touch. In both cases, the line $x'$ is parallel to two lines $p'$ and $k'$, but in the first case the lines $p'$ and $k'$ intersect, in the second case they are parallel.
 \kdokaz

We immediately notice that in solving this problem, it was not important whether $p$ and $k$ were a straight line and a circle, but it was important that the images $p'$ and $k'$ were circles. But $p'$ and $k'$ are also circles in the case when, for example, $p$ and $k$ are two lines and $A\notin p,k$. So the fourth and sixth Apollonius problems are solved in the same way as the fifth problem that we just solved.

We can also solve the second and third problem by using inversion with respect to any circle with center $A$. Both problems are translated into the construction of a tangent from point $B'$ to circle $p'$ (or $k'$).

 Problems 8, 9 and
10 are translated into problems 4, 5 and 6 in order. The idea is to first plan a circle that is concentric with the desired circle and contains the center of one of the given circles - the one with the smallest radius.

In the following, we will solve a problem similar to Apollonius' problem of the intersection of circles.

\bzgled
 Given are the point $A$, the circles $k$ and $l$, and the angles $\alpha$ and $\beta$.
  Draw
  a circle that passes through point $A$,
the circles $k$ and $l$ and intersects at angles $\alpha$ and $\beta$.
 \ezgled


 \begin{figure}[!htb]
\centering
\input{sl.inv.9.8.2.pic}
\caption{} \label{sl.inv.9.8.2.pic}
\end{figure}

\textbf{\textit{Solution.}}

Let's assume that $x$ is a circle that contains point $A$ and with
circles $k$ and $l$ defines angles $\alpha$ and $\beta$ (Figure
\ref{sl.inv.9.8.2.pic}). Again, we will consider the general case,
when point $A$ does not lie on any of the circles $k$ and $l$. With
$i$ we mark the circle with center $A$ and any radius $r$ and with
$\psi_i$ the inversion with respect to that circle. Let $k'$, $l'$
and $x'$ be the images of circles $k$, $l$ and $x$ under that
inversion. Because $A\notin k,l$ and $A\in x$, $k'$ and $l'$ are
circles and $x'$ is a line (Theorem \ref{InverzKroznVkrozn}).
Because inversion is a conformal mapping (Theorem
\ref{InverzKonf}), line $x'$ intersects circles $k'$ and $l'$
under angles $\alpha$ and $\beta$.

Line  $x'$ can be drawn as the common tangent of two circles $k'_1$
and $l'_1$ that are concentric with circles $k'$ and $l'$. When
drawing circles $k'_1$ and $l'_1$ we take into account the fact
that line $x'$ with circles $k'$ and $l'$ defines the chords with
central angles $2\alpha$ and $2\beta$. In the end, $x=
\psi_i^{-1}(x')= \psi_i(x')$.

Also in this case, the number of solutions depends on the mutual
position of circles $k'_1$ and $l'_1$ i.e. the number of their
common tangents.

In the case when $A\in k$ or $A\in l$, $x'$ is a line that with line
$k'$ (or $l'$) defines angle $\alpha$, with circle $p'$ (or $k'$)
angle $\beta$.

If $A\in p \cap k$, the problem in general has no solution,
because it is a line $x'$, that with given lines $k'$ and $l'$
defines angles $\alpha$ and $\beta$. The problem has infinitely
many solutions, when lines $k'$ and $l'$ define angles $|\beta \pm
\alpha|$.
 \kdokaz


%________________________________________________________________________________
\poglavje{Constructions With Compass Alone} \label{odd9LeSestilo}

In Euclidean geometry constructions we always used a ruler and a
compass, which means that we used
elementary constructions\index{konstrukcije!z ravnilom in šestilom} (see section \ref{elementarneKonstrukcije}).

However, by using a ruler and a compass, or the aforementioned elementary constructions, in Euclidean geometry it is not possible to derive all constructions. The most well-known are the following examples\footnote{All three problems were posed by the Ancient Greeks. Later, many famous mathematicians as well as laymen tried to solve these problems. The fact that the aforementioned constructions cannot be derived using only a ruler and a compass was only proven in the 19th century, when the French mathematician \index{Galois, E.} \textit{E. Galois} (1811--1832) developed \index{group}the theory of groups. The proof for the trisection of an angle and the doubling of a cube was first given by the French mathematician \index{Wantzel, P. L.}\textit{P. L. Wantzel} (1814--1848) in 1837. The fact that the quadrature of a circle is impossible is a consequence of the transcendence of the number $\pi$, which was proven by the German mathematician \index{Lindemann, C. L. F.}\textit{C. L. F. Lindemann} (1852–-1939) in 1882.}:
 \begin{itemize}
   \item the division of a given angle into three congruent parts (\pojem{trisection of an angle})\index{trisection of an angle};
   \item the construction of a square with the same area as a given circle (\pojem{quadrature of a circle})\index{quadrature of a circle};
   \item the construction of the edge of a cube that has twice the volume of a cube with a given edge $a$ (\pojem{doubling of a cube})\index{doubling of a cube}.
 \end{itemize}

\index{konstrukcije!pravilnih $n$-kotnikov} In addition, the problem of constructing regular $n$-gon with a ruler and a compass is also known, which cannot be solved for every $n\in\{3,4,\ldots\}$\footnote{The well-known German mathematician \index{Gauss, C. F.}\textit{C. F. Gauss} (1777--1855) proved in 1796 that with a ruler and a compass, a regular $n$-gon can be constructed exactly when $n=2^k\cdot p$, where $p$ is either 1 or a prime number that can be written in the form $2^{2^l}+1$, for $l\in\{0,1,2,\ldots\}$ (i.e. \pojem{Fermatova števila}\normalcolor, which are not all prime - for example, for $n=5$, they are named after the French mathematician \index{Fermat, P.}\textit{P. Fermat} (1601--1665)). A regular $n$-gon can therefore be constructed with a ruler and a compass if $n\in\{3,4,5,6,8,10,12,16,17,\ldots\}$, but if $n\in\{7,9,11,13,14,15,18,19,\ldots\}$, this construction is not possible.}.

 In projective geometry, in which there is neither parallelism nor metric,
 we use only a ruler\index{konstrukcije!z ravnilom} or only those
 elementary constructions
  that can only be done with a ruler. It is clear that in this geometry it is not possible
 to design (or even define) figures such as
 square,
 parallelogram, circle, ...

The question arises, what can we construct if we use
 only a compass  or only those
 elementary constructions
 that can be done only with a compass. We consider that
 a line is drawn if two of its points are drawn, but we cannot
 use the direct construction of the intersection of two lines (with
 a plane). Surprisingly, in this way - only with a compass -
 we can carry out all
 the constructions that can be done with a compass and with
 a plane\footnote{The constructions only with the help of a compass were researched
  by the Italian mathematician \index{Maskeroni, L.}
  \textit{L. Maskeroni} (1750--1800) (after him
  we call them \index{konstrukcije!Maskeronijeve}\index{konstrukcije!s šestilom}
  \pojem{Maskeronijeve konstrukcije}) and a hundred
years before him the Danish mathematician \index{Mor, G.} \textit{G. Mor}
(1640--1697) in his book ‘‘Danish Euclid’’ from 1672. The theoretical
basis of these constructions was given by the Austrian mathematician \index{Adler,
A.} \textit{A. Adler}, who in 1890 proved that every
design problem that can be solved with the help of a plane and a compass, can
also be solved only with the use of  a compass.}!

We will continue with the following constructions, in which we will
use only a compass. In the desire to show that a plane
"can be replaced" with a compass, the main role will be played by inversion. In
the previous use of inversion in constructions, we most often
used the fact that inversion in a certain case transforms
a circle into a line. So we translated the problem of constructing the desired circle with
the help of inversion into the problem of constructing the desired line,
which is usually easier. Now we will try the opposite - problems related to the construction of a line (and the use
of a plane) we will translate with inversion into problems of constructing a circle (and the use
of a compass).

 \bzgled \label{MaskeroniNAB}
 Given are points $A$ and $B$ and $n\in \mathbb{N}$.
 Draw such a point $P_n$, that
 $\overrightarrow{AP_n}=n\cdot \overrightarrow{AB}$.
 \ezgled

\begin{figure}[!htb]
\centering
\input{sl.inv.9.9.1.pic}
\caption{} \label{sl.inv.9.9.1.pic}
\end{figure}

\textbf{\textit{Solution.}} We will carry out the construction
 inductively. For $n=1$ it is clear that $P_1=B$. First, let's draw the point
 $P_2$, for which $\overrightarrow{AP_2}=2\cdot \overrightarrow{AB}$
 (Figure
\ref{sl.inv.9.9.1.pic}). Now let's draw the circles $k_1(A,AB)$
and $k_2(B,BA)$. We mark one of their intersections with $Q_1$.
Then we draw the circle $l_1(Q_1,Q_1B)$. The intersection of
the circles $l_1$ and $k_2$, which is not the point $A$, we mark with $Q_2$. The point $P_2$
we get as one of the intersections of the circles $l_2(Q_2,Q_2B)$ and $k_2$ (that which
is not $Q_1$). \normalcolor The relation $\overrightarrow{AP_2}=2\cdot
\overrightarrow{AB}$ follows from the fact that the triangles $ABQ_1$,
$Q_1BQ_2$ and $Q_2BP_2$ are all right.

 Assume that we have with the described procedure
 drawn the points $P_k$ ($k\leq n-1$), for which
  $\overrightarrow{AP_k}=k\cdot \overrightarrow{AB}$. We can draw the point $P_n$
  in the same way.
  First, we draw the circle $k_n(P_{n-1},P_{n-1}P_{n-2})$. The intersection of the circles $l_{n-1}$ and $k_n$, which is not the point $P_{n-2}$,
we mark with $Q_n$. The point $P_n$ we get as one of the intersections of the circles $l_n$ and $k_n$ (that which
is not $Q_{n-1}$). The triangles
$P_{n-2}P_{n-1}Q_{n-1}$, $Q_{n-1}P_{n-1}Q_n$ and $Q_nP_{n-1}P_n$ are
all right. Therefore,
$\overrightarrow{P_{n-1}P_n}=\overrightarrow{P_{n-2}P_{n-1}}$. If
we use the induction assumption
$\overrightarrow{AP_k}=k\cdot \overrightarrow{AB}$ for $k=n-1$,
we get $\overrightarrow{AP_n}=n\cdot \overrightarrow{AB}$, which
means that $P_n$ is the desired point.
 \kdokaz

 We have already mentioned the importance of inversion in our constructions.
 Now we are ready to prove the process of constructing the image
 of a point under inversion only with the help of a compass.

 \bzgled Given are the circle $i(S,r)$ and the point $X$. Draw the point
 $X'=\psi_i(X)$. \label{MaskeroniInv}
 \ezgled

\begin{figure}[!htb]
\centering
\input{sl.inv.9.9.2.pic}
\caption{} \label{sl.inv.9.9.2.pic}
\end{figure}

 \textbf{\textit{Solution.}} If $X\in i$ it is trivial that $X'=X$.

  First, let us assume that $X$ is an external point of the
  inversion circle $i$ (Figure \ref{sl.inv.9.9.2.pic}). Because the center $S$ is its internal point,
  the circle $k(X,XS)$ intersects the inversion circle in two points
  (statement \ref{DedPoslKrozKroz}) - for example $P$ and $Q$, which we
  can draw. Then the point $X'$ is the second intersection
  of the circles $l_P(P,PS)$ and $l_Q(Q,QS)$ (the first one is the point $S$).
  We will also prove that $X'=\psi_i(X)$. Because by construction $XS\cong
  XP$ and $PX'\cong PS$, it follows that $\angle PX'S\cong \angle PSX'=
   \angle PSX\cong\angle SPX$. This means that the triangles $PSX'$
   and $XPS$ are similar, therefore $\frac{|PX'|}{|XS|}=\frac{|SX'|}{|PS|}$ or
 $|SX|\cdot |SX'|=|PX'|\cdot |PS|=r^2$, thus $X'=\psi_i(X)$.

 If $X$ is an internal point of the
  inversion circle, then there exists a natural number $n$ such that
  $|n\cdot \overrightarrow{SX}|>r$. Let $X_n$ be the point for which
  $\overrightarrow{SX_n}=n\cdot \overrightarrow{SX}$ and
  $X'_n=\psi_i(X_n)$. From the previous example and the first part of the proof
  we can draw the points $X_n$ and $X'_n$.
  For the point $X'=\psi_i(X)$ it holds:
   $$|SX'|=\frac{r^2}{|SX|}=\frac{n\cdot r^2}{|SX_n|}=n\cdot |SX'_n|,$$
 which means that we can also draw the point $X'$ using
 the previous example.
  \kdokaz

 \bzgled Dani sta točki $A$ in $B$. Načrtaj središče daljice
 $AB$. \label{MaskeroniSred}
 \ezgled


\begin{figure}[!htb]
\centering
\input{sl.inv.9.9.3.pic}
\caption{} \label{sl.inv.9.9.3.pic}
\end{figure}

\textbf{\textit{Solution.}} Let $\psi_i$ be the inversion with respect to the circle $i(A,AB)$ (Figure \ref{sl.inv.9.9.3.pic}).
  First, we plan such a point
 $X$, that $\overrightarrow{AX}=2\cdot\overrightarrow{AB}$
 (example \ref{MaskeroniNAB}),  then $X'=\psi_i(X)$ (example \ref{MaskeroniInv}).
 Because $X$ is an external point of the circle $i$, $X'$ is its internal point.
 Therefore $\mathcal{B}(A,X',B)$. For the point $X'$ it also holds
 $|AX'|=\frac{|AB|^2}{|AX|}=\frac{|AB|^2}{2\cdot |AB|}=\frac{1}{2}\cdot |AB|,$
 which means that $X'$ is the center of the line segment $AB$.
 \kdokaz

 \bzgled \label{MaskeroniProj} Given are three non-collinear points $P$, $Q$ and $R$. Plan
 the orthogonal projection of the point $P$ onto the line $QR$.
 \ezgled


\begin{figure}[!htb]
\centering
\input{sl.inv.9.9.4.pic}
\caption{} \label{sl.inv.9.9.4.pic}
\end{figure}

 \textbf{\textit{Solution.}} First, we plan the centers of the line segments
 $PQ$ and $PR$ (example \ref{MaskeroniSred}) and label them in order with $Q_1$
 and $R_1$ (Figure \ref{sl.inv.9.9.4.pic}). We plan the point $P'$
 as the second intersection of the circles $k(Q_1,Q_1P)$ and $l(R_1,R_1P)$. The point $P'$ lies on the circles with diameters $PQ$ and $PR$. Therefore
 $\angle PP'Q=\angle PP'R=90^0$. Therefore the point $R$ lies on
 the line $P'Q$, which means that the points $P'$, $Q$ and $R$
 are collinear. Therefore the point $P'$ is the orthogonal projection of the point $P$
 onto the line $QR$.
  \kdokaz

 \bzgled \label{MaskeroniOcrt} Given are three non-collinear points $A$, $B$ and $C$.
 Plan
 the center of the circumscribed circle and the circumscribed circle of the triangle
 $ABC$.
 \ezgled

\begin{figure}[!htb]
\centering
\input{sl.inv.9.9.5.pic}
\caption{} \label{sl.inv.9.9.5.pic}
\end{figure}

\textbf{\textit{Solution.}}
 Let $l$ be the circle drawn through the triangle $ABC$ with center in
 the point $O$ and $\psi_i$ the inversion with center $A$ and arbitrary
 radius (Figure \ref{sl.inv.9.9.5.pic}). Because $A\in l$, from  the equation
  \ref{InverzKroznVkrozn} it follows,
 that $l'=\psi_i(l)$ is a line. If $P'$ is the orthogonal projection of
 the center of inversion $A$ onto the line $l'$ and $P=\psi_i(P')$, then
 $AP$ is the diameter of the circle $l$ (a consequence of the construction in the proof
  (\textit{ii}) of the equation
 \ref{InverzKroznVkrozn}).

  The proven facts allow us to construct. First, we draw
  an arbitrary circle $i$ with center in the point $A$ and
  points $B'=\psi_i(B)$ and $C'=\psi_i(C)$ (example
  \ref{MaskeroniInv}), then the orthogonal projection $P'$ of the point $A$
  onto the line $B'C'$ (example \ref{MaskeroniProj}) and $P=\psi_i(P')$.
  The center of the drawn circle is obtained as the center of the segment $AP$
  (example \ref{MaskeroniSred}). The circle $l(O,A)$ is the circle drawn
  through the triangle $ABC$.
 \kdokaz

 One of the elementary constructions with only a compass represents the construction of
 the intersection of two circles. With the help of inversion, we
 will use it for the following elementary construction, which we consider to be
 a construction with a ruler and
 a compass.

 \bzgled \label{MaskeroniKrPr} Given are four points $A$, $B$, $C$ and $D$.
 Draw the intersection of the lines $AB$ and $CD$.
 \ezgled

\textbf{\textit{Solution.}} Let's assume that the lines $AB$ and $CD$
 are not parallel.
 Let $S$ be any point that does not lie
 on either of the lines $AB$ and $CD$. It is clear that such a point
 exists, but an effective construction of such a point can be obtained
 if we use the construction of points $Q_1$ and $Q_2 $ from
 the example \ref{MaskeroniNAB} with respect to the distance $AB$.
 Let $\psi_i$ be the inversion with respect to any circle $i$ with
 the center at the point $S$ (Figure \ref{sl.inv.9.9.6.pic}).
 We draw the points $A'=\psi_i(A)$, $B'=\psi_i(B)$, $C'=\psi_i(C)$ and
  $D'=\psi_i(D)$ (example \ref{MaskeroniInv}) and the inscribed circles
  of the triangles $SA'B'$ and $SC'D'$ (example \ref{MaskeroniOcrt}) or
  the images of the lines $AB$ and $CD$ under the inversion $\psi_i$. The point $L'$ is
  the second intersection of two inscribed circles (the first intersection is the point $S$)
   It exists in the case when the lines $AB$ and $CD$
 are not parallel. The intersection $L$ of the lines $AB$ and $CD$ is the image
 of the intersection $L'$ of their images - that is $L=\psi_i(L')$ (example \ref{MaskeroniInv}).
  \kdokaz


 \bzgled \label{MaskeroniKtKr}Given is a circle $k$ and points $A$ and $B$.
  Draw the intersection of the line $AB$ and the circle $k$.
 \ezgled


\begin{figure}[!htb]
\centering
\input{sl.inv.9.9.7.pic}
\caption{} \label{sl.inv.9.9.7.pic}
\end{figure}

 \textbf{\textit{Solution.}} The construction can be carried out in the same
 way as in the previous case, only that $k'$ is the inscribed circle
 of the triangle determined by the points $C'=\psi_i(C)$, $D'=\psi_i(D)$
 and $E'=\psi_i(E)$, where $C$, $D$ and $E$ are any points
 of the circle $k$
  (Figure \ref{sl.inv.9.9.7.pic}).
\kdokaz

We have proven with the last two constructions, that all
elementary constructions related to a line and a compass (see section \ref{elementarneKonstrukcije}),
can be done only with a compass. As we have mentioned before, in doing so we count that
a line is drawn if two of its points are drawn (the same goes for a distance and a segment), but we cannot
use the direct construction of the intersection of two lines or a line and a circle (by using
a ruler). From the last two constructions (\ref{MaskeroniKrPr} and \ref{MaskeroniKtKr}), it follows that the latter is also possible only with a compass.

Since every design task that can be solved with a ruler and a compass,
can be carried out with elementary constructions
related to a line and a compass, from the previous it follows,
that this task can only be solved with a compass.


%NALOGE
%________________________________________________________________________________

\naloge{Exercises}


\begin{enumerate}

  \item
Prove that the composite of two inversions $\psi_{S,r_1}$ and $\psi_{S,r_2}$ with respect to a concentric
  circle represents a dilation. Determine the center and coefficient
  of this dilation.

  \item Let $A$, $B$, $C$ and $D$ be four collinear points.
  Construct such points $E$ and $F$, that $\mathcal{H}(A,B;E,F)$
and $\mathcal{H}(C,D;E,F)$ are true.

 \item In a plane are given a point $A$, a line $p$ and a circle $k$.
 Draw a circle that goes through the point $A$ and is
 perpendicular to the line $p$ and the circle $k$.

  \item Solve the third, fourth, ninth and tenth Apollonius' problem.

  \item Let $A$ be a point, $p$ a line, $k$ a circle
   and $\omega$ an angle in some plane. Draw a circle that goes through the point $A$, touches the line $p$ and with the circle $k$
determines the angle $\omega$.

 \item Determine the geometric location of the points of tangency of two circles that
 touch the arms of a given angle in two given points $A$ and $B$.

  \item Draw a triangle, if the following data are known:
\begin{enumerate}
 \item $a$, $l_a$, $v_a$
 \item $v_a$, $t_a$, $b-c$
 \item $b+c$, $v_a$, $r_b-r_c$
 \end{enumerate}

\item Let $c(S,r)$ and $l$ be a circle and a line in the same plane that do not have any common points. Let $c_1$, $c_2$, and $c_3$ also be circles in this plane that intersect each other (two at a time) and each of them intersects $c$ and $l$. Express the distance from point $S$ to line $l$ with $r$\footnote{Proposal for MMO 1982. (SL 12.)}.

\item Let $ABCD$ be a regular tetrahedron. To any point $M$ on edge $CD$, we assign the point $P = f(M)$ which is the intersection of the rectangle through point $A$ on line $BM$ and the rectangle through point $B$ on line $AM$. Determine the geometric position of all points $P$, if point $M$ takes on all values on edge $CD$.

\item Let $ABCD$ be a tetrahedron with perpendicular edges and $P$, $Q$, $R$, and $S$ be the points of tangency of sides $AB$, $BC$, $CD$, and $AD$ with the inscribed circle of this tetrahedron. Prove that $PR\perp QS$.

\item Prove that the centers of a tetrahedron with perpendicular edges, the centers of the inscribed and circumscribed circles, and the intersection of its diagonals are collinear points (\index{izrek!Newtonov}Newton's Theorem\footnote{\index{Newton, I.}\textit{I. Newton} (1643--1727), English physicist and mathematician}).

\item Let $p$ and $q$ be parallel tangents to circle $k$. Circle $c_1$ intersects line $p$ in point $P$ and circle $k$ in point $A$, circle $c_2$ intersects line $q$ and circles $k$ and $c_1$ in points $Q$, $B$, and $C$. Prove that the intersection of lines $PB$ and $AQ$ is the center of triangle $ABC$ of the circumscribed circle.

\item Circles $c_1$ and $c_3$ intersect each other from the outside in point $P$. Similarly, circles $c_2$ and $c_4$ also intersect each other from the outside in the same point. Circle $c_1$ intersects circles $c_2$ and $c_4$ in points $A$ and $D$, while circle $c_3$ intersects circles $c_2$ and $c_4$ in points $B$ and $C$. Prove that the following holds\footnote{Proposal for MMO 2003. (SL 16.)}:
$$\frac{|AB|\cdot|BC|}{|AD|\cdot|DC|}=\frac{|PB|^2}{|PD|^2}.$$

\item Let $A$ be a point that lies on the circle $k$. With only a
   compass, draw the square $ABCD$ (or its vertices), which is inscribed in the given
circle.

\item Given are the points $A$ and $B$. With only
  a compass, draw such a point $C$, that
   $\overrightarrow{AC}=\frac{1}{3}\overrightarrow{AB}$.

   \item With only the help
   of a compass, divide the given line segment in the ratio $2:3$.
\end{enumerate}









% DEL 10 - - - - - - - - - - - - - - - - - - - - - - - - - - - - - - - - - - - - - - -
%________________________________________________________________________________
% REŠITVE IN NAPOTKI
%________________________________________________________________________________

\del{Solutions and Hints}
\footnotesize
%REŠITVE - Aksiomi evklidske geometrije ravnine
%________________________________________________________________________________

\poglavje{Introduction}

There are no exercises in the first chapter. \color{viol4}$\ddot\smile$ \normalcolor %smile

\poglavje{Axioms of Planar Euclidean Geometry}

\begin{enumerate}

\item \res{Let $P$, $Q$ and $R$ be the inner points of the sides of the triangle $ABC$. Prove that $P$, $Q$ and $R$ are non-collinear.}

Assume the opposite, that points $P$, $Q$ and $R$ lie on
 a line $l$. This means that the line $l$ intersects all three
sides of the triangle $ABC$ and by assumption does not go through any
of its vertices. This is in contradiction with the consequence of \ref{PaschIzrek}
 Pasch's axiom \ref{AksPascheva}, therefore  $P$, $Q$ and
 $R$ are non-collinear points.

\item \res{Let $P$ and $Q$ be points of sides $BC$ and $AC$ of the triangle $ABC$ and at the same time
different from its vertices. Prove that the line segments $AP$ and $BQ$
intersect in one point.}

From the assumption that the point $P$ lies on the side $BC$ of the triangle
 $ABC$, according to the definition of the distance $BC$ it holds $\mathcal{B}(B,P,C)$.
 From the axiom \ref{AksII2} it follows that it does not hold
 $\mathcal{B}(B,C,P)$, which means that the point $B$ does not lie on
 the line $PC$. In the triangle $APC$ the line $BQ$ intersects the side $AC$,
 but does not intersect the sides $PC$, so by the consequence \ref{PaschIzrek}
 of Pasch's axiom \ref{AksPascheva} the line $BQ$ intersects
 the side $AP$ of this triangle. So the line $BQ$ intersects the line
 $AP$ in some point $X$ or $BQ\cap [AP]=\{X\}$. Analogously it
 is also $[BQ]\cap AP=\{\widehat{X}\}$. We prove that
 $X=\widehat{X}$. We assume the opposite, that it holds $X\neq
 \widehat{X}$. Because $[AP]\subset AP$ and $[BQ]\subset BQ$,
 it holds $X,\widehat{X}\in AP\cap BQ$. By the axiom \ref{AksI1} $AP$ and $BQ$ represent
 the same line or the points $A$, $B$,
 $P$ and $Q$ lie on the same line $l$. But on this line, which
 is determined by the points $B$ and $P$ (axiom \ref{AksI1}), according to the assumption also the point $C$ lies, which means that the points $A$, $B$
 and $C$ are collinear. But this is in contradiction with the assumption that $ABC$ is a triangle, so $X=\widehat{X}$. This means $X\in
 [AP]\cap[BQ]=\{X\}$. If the line $[AP]$ and $[BQ]$ had one more
 common point, that point would also be the second common point of the lines $AP$
 and $BQ$. We have shown that this is not possible, so
 $[AP]\cap[BQ]=\{X\}$.

\item  \res{The points $P$, $Q$ and $R$ lie in order on the sides $BC$, $AC$ and $AB$ of the triangle
 $ABC$ and are different from its vertices. Prove that the lines $AP$
and $QR$ intersect in one point.}

We use the previous task first for the lines $AP$ and $BQ$ in
the triangle $ABC$, and then once again for the line $AX$
($\{X\}=[AP]\cap[BQ]$) and $QR$ in the triangle $ABQ$.



\item \res{The line $p$, which lies in the plane of the quadrilateral, intersects its
diagonal $AC$ and does not pass through any vertex of this quadrilateral.
Prove that the line $p$ intersects exactly two sides of this
quadrilateral.}

We use the consequence \ref{PaschIzrek} of Pasch's axiom
 \ref{AksPascheva} twice for the line $p$ and the triangle $ABC$ and $ADC$.

\item \label{nalAks3}\res{Prove that the midline is a convex figure.}

Let $X$ and $Y$ be any points of the midline $pA$. It needs to be proven that the whole line segment $XY$ lies within this midline. By definition of the midline from $X,Y\in pA$ it follows that $X,A\ddot{-} p$ and $Y,A\ddot{-} p$. Because $\ddot{-} p$ is an equivalence relation, it is also true that $X,Y\ddot{-} p$. Let $Z$ be any point of the line segment $XY$. We assume that it is not the case that $Z,A\ddot{-} p$ or that $Z,A\div p$ or $Z,X\div p$ (or $Z \in p$) holds. In this case, the line segment $ZX$ intersects the line $p$ at some point $T$. The point $T$ is then the common point of the line $p$ and the line segment $XY$ (according to statement \ref{izrekAksIIDaljica}), which contradicts what has been proven about $X,Y\ddot{-} p$. Therefore, $Z,A\ddot{-} p$ holds, which means that $Z\in pA$, which in turn means that the midline $pA$ is a convex figure.

\item \label{nalAks4} \res{Prove that the intersection of two convex figures
is a convex figure.}

Let $\Phi_1$ and $\Phi_2$ be convex figures. We prove that
then $\Phi_1\cap\Phi_2$ is also a convex figure. Let $A, B\in
\Phi_1\cap\Phi_2$ be any points. In this case, $A, B\in
\Phi_1$ and $A,B\in\Phi_2$. Because the figures $\Phi_1$ and $\Phi_2$
are convex, it follows that $[AB]\subseteq\Phi_1$ and $[AB]\subseteq\Phi_2$
or $[AB]\subseteq\Phi_1\cap\Phi_2$, which means that $\Phi_1\cap\Phi_2$ is also a convex figure.

\item  \res{Prove that any triangle is a convex figure.}

The triangle $ABC$ is the intersection of the midlines $ABC$, $ACB$ and
$BCA$, so according to the previous two tasks \ref{nalAks3} and
\ref{nalAks4}, it is a convex figure.

\item  \res{If $\mathcal{B}(A,B,C)$ and $\mathcal{B}(D,A,C)$, then
$\mathcal{B}(B,A,D)$ is also true. Prove it.}

According to the statement \ref{izrekAksIIPoltrak}, the point $A$ on the line $AB$ determines two poltraks.
We denote with $p$ the poltrak $AC$ and with $p'$ its complementary (supplementary) poltrak. From $\mathcal{B}(A,B,C)$ it follows that $B,C\ddot{-} A$ from $\mathcal{B}(D,A,C)$ we have $D,C\div A$, therefore $B\in p$ and $D\in p'$. So the points $B$ and $D$ not only lie on the same poltrak with the starting point $A$, so there is no $B,D\ddot{-} A$, which means that $B,D\div A$ or $\mathcal{B}(B,A,D)$ applies.



\item  \res{Let $A$, $B$, $C$ and $D$ be such collinear points that
$\neg\mathcal{B}(B,A,C)$ and $\neg\mathcal{B}(B,A,D)$. Prove that
$\neg\mathcal{B}(C,A,D)$ applies.}

From $\neg\mathcal{B}(B,A,C)$ and $\neg\mathcal{B}(B,A,D)$ it follows that $B,C\ddot{-} A$ or $B,D\ddot{-} A$. Because $\ddot{-} A$ is an equivalence relation, it is also transitive, so $C,D\ddot{-} A$ or $\neg\mathcal{B}(C,A,D)$ applies.

\item \res{Let $A_1A_2\ldots,A_{2k+1}$ be an arbitrary polygon with an even
number of vertices. Prove that there is no line that intersects all
of its sides.}

Assume the opposite, that some line $p$ intersects all sides
of this polygon. Prove that in this case the points
$A_1,A_3,\ldots,A_{2k+1}$ would be on the same side of the line $p$.

\item \label{nalAks11}\res{If the isometry $\mathcal{I}$ maps the figure $\Phi_1$ and $\Phi_2$
into the figure  $\Phi'_1$ and $\Phi'_2$, then the intersection
$\Phi_1\cap\Phi_2$ with this isometry is mapped into the intersection
$\Phi'_1\cap\Phi'_2$. Prove it.}

If for an arbitrary point $X$ it holds that $X\in \Phi_1\cap\Phi_2$, then $X\in \Phi_1$ and $X\in\Phi_2$. From this it follows that $\mathcal{I}(X)\in \mathcal{I}(\Phi_1)$ and $\mathcal{I}(X)\in \mathcal{I}(\Phi_2)$ or $X'\in \Phi'_1$ and $X'\in \Phi'_2$ (where $X'=\mathcal{I}(X)$). So it holds that $X'\in\Phi'_1\cap\Phi'_2$. In this way we have proven $\mathcal{I}(\Phi_1\cap\Phi_2)\subseteq \Phi'_1\cap\Phi'_2$. The other inclusion $\mathcal{I}(\Phi_1\cap\Phi_2)\supseteq \Phi'_1\cap\Phi'_2$ is proven in the same way using the isometry $\mathcal{I}^{-1}$.

\item  \res{Prove that any two
lines of a plane are parallel to each other.}

We use axiom \ref{aksIII2} and theorem \ref{izrekIzoB}.

\item  \res{Prove that any two
lines of a plane are concurrent to each other.}

We use axiom \ref{aksIII2} and \ref{aksIII1}.


\item  \res{Let $k$ and $k'$ be two circles
of a plane with centers $O$ and $O'$ and radii $AB$ and $A'B'$.
Prove the equivalence: $k\cong k' \Leftrightarrow AB\cong A'B'$.}

($\Leftarrow$) Let $AB\cong A'B'$. We mark with $P$ and $P'$ any points of the circles $k$ and $k'$. Because $OP$ and $O'P'$ are radii of these circles, it holds that $OP\cong O'P'$. By theorem \ref{izrekAB} there exists an isometry $\mathcal{I}$, which maps the points $O$ and $P$ to the points $O'$ and $P'$. Let $X$ be an arbitrary point of the circle $k$ and $X'=\mathcal{I}(X)$. From $\mathcal{I}:O,X,P\rightarrow O',X',P'$ it follows that  $O'X'\cong OX$. Because $X,P\in k$, it also holds that $OX\cong OP$. From the previous relations it follows that $O'X'\cong O'P'$ or $X'\in k'$. So it holds that $\mathcal{I}(k)\subseteq k'$. Similarly $\mathcal{I}(k)\supseteq k'$ (we use $\mathcal{I}^{-1}$), so $\mathcal{I}(k)= k'$.

($\Rightarrow$) Let $k\cong k'$. We assume the opposite - that $AB\not\cong A'B'$. Without loss of generality, let $AB>A'B'$.
 From $k\cong k'$ it follows that there exists an isometry $\mathcal{I}$, which maps the circle $k$ to the circle $k'$. Let $PQ$ be any diameter of the circle $k$. By \ref{premerInS} we have $PQ=2\cdot AB$. Let $P'=\mathcal{I}(P)$ and $Q'=\mathcal{I}(Q)$. Therefore $P'Q'$ is the diameter of the circle, for which $P'Q'\cong PQ =2\cdot AB>2\cdot A'B'$. But the relation $P'Q'>2\cdot A'B'$ is not possible (\ref{premerNajdTetiva}), so $AB\cong A'B'$.

\item  \res{Let $\mathcal{I}$
be a non-identical isometry of a plane with two fixed points $A$ and
$B$. Let $p$ be a line of this plane, which is parallel to the line
$AB$, and $A\notin p$. Prove that there are no fixed points of the
isometry $\mathcal{I}$ on the line $p$.}

We assume the opposite - that there exists a fixed point $C$ of the isometry $\mathcal{I}$, which lies on the line $p$. The point $C$ does not lie on the line $AB$, because in the opposite case $AB$ and $p$ would be parallel, which is in contradiction with the assumption $A\notin p$. So we have three non-collinear fixed points of the isometry $\mathcal{I}$, so by \ref{IizrekABC2} $\mathcal{I}$ is identical. This is in contradiction with the assumption, so there are no fixed points of the isometry $\mathcal{I}$ on the line $p$.

\item   \res{ Let $S$ be the only fixed
point of the isometry $\mathcal{I}$ in some plane. Prove that if this
isometry maps the line $p$ to itself, then $S\in p$.}

We assume that $S\notin p$. We mark with $N$ the orthogonal projection of the point $S$ on the line $p$ and $N'=\mathcal{I}(N)$. From $S\notin p$ it follows that $N\neq S$. Because $S$ is the only fixed point of this isometry, $N'\neq N$. By the assumption $\mathcal{I}:p\rightarrow p$ we also have $N'\in p$. From $\mathcal{I}:SN,p\rightarrow SN', p$ it follows that $\angle SN',p=\angle SN,p=90^0$. In this case, there would be two different perpendiculars from the point $S$ on the line $p$, which by \ref{enaSamaPravokotnica} is not possible. Therefore the assumption $S\notin p$ is false, which means that $S\in p$.

\item  \label{nalAks17}\res{Prove that any two lines of
a plane either intersect or are parallel. }

By Axiom \ref{AksI1}, two different lines $p$ and $q$ of a plane have at most one common point. If they have one common point, they intersect by definition, if they don't have any common points, they are parallel by definition.

\item  \label{nalAks18}\res{If a
line in a plane intersects one of two parallel lines of the same plane,
 then it also intersects the other parallel line. Prove.}

Let $p\parallel q$ and $l\cap p =\{A\}$. If $p=q$, then it is clear that $l\cap q =\{A\}$. The other possibility is $p\cap q =\emptyset$. Then, by Playfair's Axiom, $l\cap q \neq\emptyset$. Because it can't be that $l=q$, then by the previous task \ref{nalAks17} the lines $l$ and $q$ intersect.

\item  \res{Prove that every isometry maps parallel lines to parallel lines.}

Use task \ref{nalAks11}.

\item  \res{Let $p$, $q$ and $r$ be such lines of a plane that
$p\parallel q$ and $r\perp p$. Prove that $r\perp q$.}

The statement is a direct consequence of task \ref{nalAks18} and Theorem \ref{KotiTransverzala}.

\item \res{Prove that a convex $n$-gon can't have more than three
acute angles.}

Use the fact that the sum of the exterior angles of a convex $n$-gon
is equal to $360^0$ (Theorem \ref{VsotKotVeckZuna}).


\end{enumerate}


%REŠITVE - Skladnost trikotnikov
%________________________________________________________________________________

\poglavje{Congruence. Triangles and Polygons}

\begin{enumerate}


 \item \res{Let $S$ be a point that lies in the angle $pOq$, and let $A$ and $B$ be the orthogonal projections of point $S$ onto the sides $p$ and $q$ of this
angle. Prove that $SA\cong SB$ if and only if the line
$OS$ is the angle bisector of the angle $pOq$.}

Prove the congruence of triangles $OSA$ and
$OSB$ in both directions of equivalence. In the proof for ($\Rightarrow$) use Theorem \textit{SSA}
\ref{SSK}, in the proof for ($\Leftarrow$) use Theorem \textit{ASA}
\ref{KSK}.

\item \res{Prove that the sum of the diagonals of a convex quadrilateral is greater than
the sum of its opposite sides.}

We use the triangle inequality for the triangles $ASB$ and $CSD$, where
$S$ is the intersection of the diagonals $AC$ and $BD$ of the convex quadrilateral
$ABCD$.

  \item \res{Prove that in every triangle
  no more than one side is shorter than the corresponding altitude.}

  We mark with $a$, $b$ and $c$ the sides and with $v_a$, $v_b$,  $v_c$
  the corresponding altitudes of the triangle $ABC$. We assume the opposite, that
  for example $a<v_a$ and $b<v_b$. If we use the inequality for
  the right triangles (statement \ref{vecstrveckotHipot}, we
  have $v_a\leq b$ and $v_b\leq a$. If we connect the four inequalities,
  we get $a<v_a\leq b<v_b\leq a$ or $a<a$, which is not possible. So in every triangle
  no more than one side is shorter than the corresponding altitude.

  \item \res{Let $AA_1$ be the centroid of the triangle $ABC$. Prove
   that of the two angles that the centroid $AA_1$ determines with
the sides $AB$ and $AC$, the greater one is the one that the centroid determines with
the shorter side.}

We use the triangle inequality for the triangle $ADC$, where $D=S_{A_1}(A)$ (see the definition of the central reflection in section
\ref{odd6SredZrc}).

  \item \res{Let $BB_1$ and $CC_1$ be the centroids of the triangle $ABC$ and
  $AB<AC$.
  Prove that $BB_1<CC_1$.}

We mark with $AA_1$ the third centroid and with $T$ the centroid of the triangle
$ABC$. If we use  statement \ref{SkladTrikLema} for the triangle
$BA_1A$ and $CA_1A$, we get $\angle BA_1A< \angle CA_1A$. If then we use the same statement for the triangle $BA_1T$ and $CA_1T$, we get
$BT<CT$ or $BB_1<CC_1$ (statement \ref{tezisce}).


  \item \res{Let $a$, $b$ and $c$ be the sides, $t_a$, $t_b$ and $t_c$
   the corresponding centroids and $s$ the semiperimeter of an arbitrary triangle.
Prove that the following statements hold:
 \begin{enumerate}
  \item $s <  t_a  + t_b +  t_c  < 2s$
  \item $ta + tb + tc  >  \frac{3}{4}(a + b + c)$
 \end{enumerate}}

 We use example \ref{neenTezisZgl}.

\item \res{Let $p$ be a line that is parallel to the circle $k$. Prove that all points of this circle are on the same side of the line $p$.}

We assume the opposite, that points $X,Y\in k$ are on different sides of the line $p$. In this case, the segment $XY$ intersects the line $p$ in some point $N$, which, according to \ref{tetivaNotrTocke}, is an interior point of the circle $k$. The line $p$, which contains the interior point $N$ of the circle $k$, is, according to \ref{DedPoslKrozPrem}, its secant, which contradicts the assumption that $p$ is parallel.

\item \res{If the circle $k$ lies in some convex figure $\Phi$, then also the circle, which is determined by this circle, lies in this figure. Prove it.}

It is enough to prove that any interior point $X$ of the circle $k$ lies in the figure $\Phi$. Let $AB$ be any segment that contains the point $X$ (the existence of such a segment follows from \ref{DedPoslKrozPrem}). Because $\Phi$ is a convex figure, from $A,B\in k\subset \Phi$ it follows that $X\in [AB]\subset \Phi$.

\item \res{Let $p$ and $q$ be two different tangents to the circle $k$, which touch it in points $P$ and $Q$. Prove the equivalence: $p \parallel q$ if and only if $AB$ is the diameter of the circle $k$.}

We use the fact $SP\perp p$ and $SQ\perp q$, where $S$ is the center of the circle $k$.

\item \res{If $AB$ is a segment of the circle $k$, then the intersection of the line $AB$ and the circle, which is determined by the circle $k$, is equal to this segment. Prove it.}

We denote by $\mathcal{K}$ the mentioned circle. It is necessary to prove that $[AB]=AB\cap \mathcal{K}$. The inclusion $[AB]\subseteq AB\cap \mathcal{K}$ follows directly from the axiom \ref{AksII1} and \ref{tetivaNotrTocke}. We prove the inclusion $AB\cap \mathcal{K}\subseteq [AB]$. Let $X\in AB\cap \mathcal{K}$. We assume that $X\notin [AB]$. In this case, it would be $\mathcal{B}(X,A,B)$ or $\mathcal{B}(A,B,X)$. It is not difficult to prove that in each of these two cases we would get $SX>SA$, which is not possible, because $X\in \mathcal{K}$. Therefore, $X\in[AB]$, i.e. $AB\cap \mathcal{K}\subseteq [AB]$.

\item \res{Let $S'$ be the orthogonal projection of the center $S$ of the circle $k$ onto the line $p$. Prove that $S'$ is an external point of this circle exactly when the line $p$ does not intersect the circle.}

We use the statement \ref{TangSekMimobKrit}.

\item \res{Let $V$ be the altitude point of the triangle $ABC$, for which $CV \cong AB$. Determine the size of the angle $ACB$.}

We mark with $A'$, $B'$ and $C'$ the points of the altitudes from the vertices $A$, $B$ and $C$ of the triangle $ABC$. First, the angles $C'CB$ and $A'AB$ are congruent (the angle with perpendicular legs - statement \ref{KotaPravokKraki}). From the congruence of the triangles $CVA'$ and $ABA'$ (statement \textit{ASA} \ref{KSK}) it follows that $CA'\cong AA'$. This means that $CAA'$ is an isosceles right triangle with hypotenuse $AC$, so according to statement \ref{enakokraki} $\angle ACB=\angle ACA'=45^0$.

\item \res{Let $CC'$ be the altitude of the right triangle $ABC$ ($\angle ACB = 90^0$). If $O$ and $S$ are the centers of the inscribed circles of the triangles $ACC'$ and $BCC'$, the internal angle bisector of $ACB$ is perpendicular to the line $OS$. Prove it.}

Let $I$ be the center of the inscribed circle of the triangle $ABC$. We prove that $I$ is the altitude point of the triangle $COS$.

\item \res{Let $ABC$ be a triangle in which $\angle ABC = 15^0$ and $\angle ACB = 30^0$. Let $D$ be a point on the side $BC$ such that $\angle BAD=90^0$. Prove that $BD = 2AC$.}

We mark with $S$ the center of the segment $BD$. The point $S$ is the center of the inscribed circle of the right triangle $BAD$ (statement \ref{TalesovIzrKroz2}). Therefore $SA\cong SB\cong SD$. Because $BSA$ is an isosceles triangle with the base $AB$, according to statement \ref{enakokraki} $\angle BAS\cong \angle ABS=15^0$. For the external angle $ASD$ of the triangle $BSA$ it then holds that $\angle ASD=\angle BAS + \angle ABS=30^0$ (statement \ref{zunanjiNotrNotr}). Therefore $\angle ASC=30^0=\angle ACS$, so $ASC$ is an isosceles triangle with the base $SC$ or $AS\cong AC$ (statement \ref{enakokraki}). From this it follows that $BD = 2SB=2AS=2AC$.

\item \res{Prove that there exists a pentagon that can tile the plane.}

We can choose a pentagon $ABCDE$, so that $ABCE$ is a square, $ECD$ is an isosceles right triangle with the base $CE$, and $A,D\div CE$.

\item \res{Prove that there exists a decagon, with which it is possible to tile a plane.}

We make the union of two appropriate regular hexagons - the cells of tiling $(6,3)$.

\item \res{In some plane each point is painted red or black. Prove that there exists a right triangle, which has all vertices of the same color.}

We use the tiling $(3,6)$.

\item \res{Let $l_1,l_2,\ldots, l_n$ ($n > 3$) be arcs, which all lie on the same circle. The central angle of each arc is at most $180^0$. Prove that if each triple of arcs has at least one common point, there exists a point, which lies on each arc.}

We denote with $k(S,r)$ the given circle. Let $I_i$ ($i\in \{1,2,\ldots,n\}$) be the circular segments, which are determined by the arcs $l_k$, and $J_i=I_i\setminus\{S\}$. Because for each $l_i$ the central angle is at most $180^0$, the sets $J_i$ are convex figures (prove it). By the assumption each triple of arcs $l_{i_1}$,  $l_{i_2}$ and $l_{i_2}$ ($i_1,i_2,i_3\in\{1,2,\ldots,n\}$) has a common point $X_{i_1i_2i_3}$. This point also belongs to each of the figures $J_{i_1}$,  $J_{i_2}$ and  $J_{i_2}$. By Helly's theorem \ref{Helly} there exists a point $Y$, which belongs to each of the figures $J_1,J_2,\ldots, J_n$. If we denote with $X$ the intersection of the line segment $SY$ and the circle $k$, it follows that also the open line segment $(SX]$ belongs to each of these figures. Because for each $i$ it holds $l_i=J_i\cap k$, the point $X$ belongs to each of the arcs $l_1,l_2,\ldots, l_n$.

\item
\res{Let $p$ and $q$ be rectangles that intersect at point $A$. If
$B, B'\in p$, $C, C'\in q$, $AB\cong AC'$, $AB'\cong AC$,
$\mathcal{B}(B,A,B')$ and $\mathcal{B}(C,A,C')$, then the rectangle
on the line $BC$ through point $A$ goes through the center of the
line $B'C'$. Prove it.}

First, by the \textit{SAS} theorem \ref{SKS}, the triangles $BAC$
and $C'AB'$ are congruent. From this it follows: $\angle
AC'B'\cong\angle ABC=\beta$ and $\angle AB'C'\cong\angle
ACB=90^0-\beta$. We mark with $S$ the center of the line $B'C'$,
with $P$ the intersection of the lines $BC$ and $AS$. By theorem
\ref{TalesovIzrKroz2}, $SA\cong SB'\cong SC'$, so $\angle
CAP=\angle C'AS\cong\angle AC'S=\angle AC'B'=\beta$ (theorem
\ref{enakokraki}). Because $\angle ACP=\angle ACB=90^0-\beta$,
from triangle $ACP$ by theorem \ref{VsotKotTrik} $\angle
APC=90^0$ or $AS\perp BC$. This means that the rectangle $AP$ of
the line $BC$ goes through the center of the line $B'C'$.

\item
\res{Prove that the internal angle bisectors of a rectangle that is
not a square intersect at points that are the vertices of a
square.}

We mark with $s_{\alpha}$, $s_{\beta}$, $s_{\gamma}$ and
$s_{\delta}$ the internal angle bisectors at the vertices $A$, $B$,
$C$ and $D$ of the rectangle $ABCD$ and $P=s_{\alpha}\cap
s_{\beta}$, $Q=s_{\gamma}\cap s_{\delta}$, $L=s_{\beta}\cap
s_{\gamma}$ and $K=s_{\alpha}\cap s_{\delta}$. We prove that
$PKQL$ is a square. First, from $\angle PAB=45^0$ and $\angle
PBA=45^0$ it follows that $ABP$ is an isosceles right triangle
with the base $AB$ (theorem \ref{enakokraki} and \ref{VsotKotTrik}).
Therefore $\angle APB=90^0$ and $AP\cong BP$. Analogously, all
the internal angles of the quadrilateral $PKQL$ are right angles.
It is enough to prove that $PK\cong PL$. From the congruence of
the triangles $AKD$ and $BLC$ (theorem \ref{KSK}) it follows that
$AK\cong BL$. If we connect this with the already proven $AP\cong
BP$, we get  $PK\cong PL$.

\item
 \res{Prove that the internal angle bisectors of a parallelogram that is not a rhombus intersect at the vertices of a rectangle. Prove also that the diagonals of this rectangle are parallel to the sides of the parallelogram and are equal to the difference of the adjacent sides of the parallelogram.}

As in the previous task, we denote by $s_{\alpha}$,
$s_{\beta}$, $s_{\gamma}$ and $s_{\delta}$ the internal angle
bisectors at the vertices $A$, $B$, $C$ and $D$ of the rectangle
$ABCD$ and $P=s_{\alpha}\cap s_{\beta}$, $Q=s_{\gamma}\cap
s_{\delta}$, $L=s_{\beta}\cap s_{\gamma}$ and $K=s_{\alpha}\cap
s_{\delta}$. We prove that $PKQL$ is a rectangle. Let $E$ and $F$
be the midpoints of the sides $AD$ and $BC$ of the parallelogram
$ABCD$. In the triangle $PAB$ we have
 $\angle APB=180^0-\angle PAB-\angle
PBA=180^0-\frac{1}{2}\left(\angle DAB+\angle CBA
\right)=180^0-\frac{1}{2}\cdot 180^0=90^0$ (by the
 \ref{VsotKotTrik} and \ref{paralelogram}). Similarly, all the other internal angles of the quadrilateral $PKQL$ are right angles. By the
 \ref{TalesovIzrKroz2} the point $E$ is the center of the inscribed circle
 of the right triangle $AKD$ with hypotenuse $AD$ and we have
 $EK\cong EA$. Therefore $\angle EKA\cong \angle EAK \cong\angle
 KAB$ (by the \ref{enakokraki}) or, by the
 \ref{KotiTransverzala}, $EK\parallel AB$. Similarly, $FL\parallel AB$. Since also $EF\parallel AB$ (by the \ref{srednjTrapez}) we have
 $KL\parallel AB$. Without loss of generality, we assume that $AB>AD$. We denote by $T$ the intersection of the bisector $s_{\alpha}$ with the side $CD$. By the \ref{KotiTransverzala}
 we have $\angle DTA\cong\angle TAB\cong\angle DAT$. This means that
 $ADT$ is an isosceles triangle, therefore $AD\cong DT$ (by the
 \ref{enakokraki}). Since $KL\parallel AB\parallel CT$ and
 $s_{\alpha}\parallel s_{\gamma}$, the quadrilateral $KLCT$
 is a parallelogram. Therefore $PQ\cong KL\cong CT=CD-DT=CD-AB$.


\item
\res{Prove that the angle bisectors of two adjacent angles are perpendicular to each other.}

The angle bisectors determine the angle that is equal to half of the corresponding extended angle.

\item \res{Let $B'$ and $C'$ be the altitudes of the triangle $ABC$ from the vertices $B$ and $C$. Prove that $AB\cong AC \Leftrightarrow BB'\cong CC'$.}

Prove the equivalence in both directions. In the direction ($\Rightarrow$) use the \textit{ASA} theorem \ref{KSK}, in the direction ($\Leftarrow$) use the \textit{SSA} theorem \ref{SSK}.

\item \res{Prove that a triangle is right if and only if the center of the circle drawn through the triangle and its altitude point coincide. Does a similar statement hold for any two characteristic points of this triangle?}

Use the fact that the sides of the triangle are perpendicular to the altitude in this case. In a similar way, one could prove that a similar statement holds for any two characteristic points of this triangle.

\item \res{Prove that the obtuse triangles $ABC$ and $A'B'C'$ are congruent if and only if they have congruent altitudes $CD$ and $C'D'$, sides $AB$ and $A'B'$, and angles $ACD$ and $A'C'D'$.}

From the congruence of the triangles $ABC$ and $A'B'C'$ it follows directly first that $AB\cong A'B'$, $AC\cong A'C'$, and $\angle BAC\cong \angle B'A'C'$, then that $\triangle ACD\cong\triangle A'C'D'$ (by the \textit{ASA} theorem \ref{KSK}) or $CD\cong C'D'$, and $\angle ACD\cong \angle A'C'D'$.

Assume that $CD\cong C'D'$, $AB\cong A'B'$, and $\angle ACD\cong \angle A'C'D'$. From the congruence of the triangles $ACD$ and $A'C'D'$ (by the \textit{ASA} theorem \ref{KSK}) it follows that there exists an isometry $\mathcal{I}$, for which $\mathcal{I}:A,C,D \mapsto A',C',D'$ (by theorem \ref{IizrekABC}. This maps the segment $AD$ onto the segment $A'D'$. Because $B$ and $B'$ lie on the segments $AD$ and $A'D'$, and $AB\cong A'B'$, also $\mathcal{I}(B)=B'$. Therefore $\mathcal{I}:A,B,C \mapsto A',B',C'$, which means that $\triangle ABC\cong\triangle A'B'C'$.

\item \res{If $ABCD$ is a rectangle and $AQB$ and $APD$ are right triangles with the same orientation, then the segment $PQ$ is congruent to the diagonal of this rectangle. Prove it.}

We prove that $PAQ$ and $DAB$ are congruent triangles.

\item \res{Let $BB'$ and $CC'$ be the altitudes of the triangle $ABC$ ($AC>AB$) and
 $D$ be such a point on the line segment $AB$, that $AD\cong AC$. The point
$E$ is the intersection of the line $BB'$ with the line that goes through the point $D$ and is
parallel to the line $AC$. Prove that $BE=CC'-BB'$.}

Let $F$ be the intersection of the line $ED$ with the line that is in the point $B'$
parallel to the line $AB$. The quadrilateral $ADFB'$ is a parallelogram,
so $FB'\cong DA$ and $\angle DFB' \cong\angle DAB'$ (by the statement
\ref{paralelogram}). Because according to the assumption $AD\cong AC$,
it also holds that $FB'\cong AC$. This means that the right-angled
triangles $FEB'$ and $AC'C$ are similar (by the \textit{ASA} statement \ref{KSK}), so
$EB'\cong C'C$ or $BE=EB'-BB'=CC'-BB'$.

\item \res{Let $ABCD$ be a convex quadrilateral, for which
 $AB\cong BC\cong CD$ and $AC\perp BD$. Prove that $ABCD$
 is a rhombus.}

 Because $ABCD$ is a convex quadrilateral, its diagonals
 intersect in some point $S$. Prove $\triangle ABS\cong\triangle CBS$ and
 $\triangle CBS\cong\triangle CDS$.

\item \res{Let $BC$ be the base of the isosceles triangle $ABC$. If $K$ and
$L$ are such points, that $\mathcal{B}(A,K,B)$, $\mathcal{B}(A,C,L)$ and $KB\cong LC$, then
the center of the line segment $KL$ lies on the base $BC$. Prove.}

Let $O$ be the center of the line $KL$. We mark with $M$ the fourth vertex of the parallelogram $CKBM$, the common center of their diagonals $BC$ and $KM$ (statement \ref{paralelogram}) with $S$. If we use the fact that the triangles $ABC$ and $MLC$ are of the same base with the bases $BC$ and $ML$, from the statement \ref{enakokraki} we get $\angle ABC\cong\angle ACB$ and $\angle CML\cong\angle CLM$. From the congruence of the triangles $BKC$ and $CMB$ (statement \ref{paralelogram} and \textit{SSS} \ref{SSS}) it follows that $\angle ABC=\angle KBC\cong\angle MCB$. As $ACM$ is the external angle of the triangle $MLC$, then by the statement \ref{zunanjiNotrNotr} $\angle ACM =\angle CML+\angle CLM=2\angle CML$. Therefore $\angle CML=\frac{1}{2}\angle ACM=\frac{1}{2}\left(\angle ACB+\angle BCM\right)\frac{1}{2}\left(\angle ACB+\angle ABC\right)=\angle ABC$. This means that the lines $ML$ and $BC$ are parallel (statement \ref{KotiTransverzala}). The line $OS$ is the median of the triangle $KML$ with the base $ML$, therefore by the statement \ref{srednjicaTrik} $SO\parallel MN$. By Playfair's axiom \ref{Playfair} the lines $SO$ and $BC$ are the same (coincide), therefore the point $O$ lies on the line $BC$. By Pasch's axiom \ref{AksPascheva} for the triangle $ABC$ and the line $KL$ the point $O$ lies on the side $BC$.

\item \res{Let $S$ be the center of the triangle $ABC$ of the inscribed circle. The line passing through the point $S$ and parallel to the side $BC$ of this triangle intersects the sides $AB$ and $AC$ in order. Prove that $BM+NC=NM$.}

We prove that the triangles $BSM$ and $SCN$ are of the same base with the bases $BS$ and $SC$.

\item \res{Let $ABCDEFG$ be a convex heptagon. Calculate the sum of the convex angles determined by the broken line $ACEGBDFA$.}

We use the fact that the double sum of the external angles of the heptagon determined by the intersections of the diagonals of the heptagon $ABCDEFG$ is equal to $720^0$. Result: $540^0$.

\item \label{nalSkl34}
\res{Prove that the centers of the sides and the vertex of any altitude of a triangle in which no two sides are congruent are the vertices of a parallelogram.}

We denote with $A_1$, $B_1$ and $C_1$ the centers of sides $BC$, $CA$ and
$BA$ of the triangle $ABC$ and with $A'$ the intersection point of the altitude of this triangle with
the point $A$. From $|AB|\neq |AC|$ it follows that $A'\neq A_1$. We prove that
the quadrilateral $A'A_1B_1C_1$ is a parallelogram. The distance $B_1C_1$
is the median of the triangle $ABC$ with the base $BC$, therefore by
the statement \ref{srednjicaTrik} $B_1C_1\parallel BC$. Because $A',A_1\in
BC$, also $B_1C_1\parallel A',A_1$. So the quadrilateral $A'A_1B_1C_1$ is a parallelogram. We prove also that it is a rhombus, or
$C_1A'\cong B_1A_1$. But this fact follows from 
the statements \ref{TalesovIzrKroz2} and \ref{srednjicaTrik}, because:
$C_1A'=\frac{1}{2}AB= B_1A_1$.

We mention also that by using this statement we can another way prove the statement about Euler's circle \ref{EulerKroznica}.

 \item \res{Let $ABC$ be a right angled triangle with the right angle at the point $C$.
The points $E$ and $F$ are the intersections of the internal angle bisectors at
the points $A$ and $B$ with the opposite sides,  $K$ and $L$ are
the orthogonal projections of the points $E$ and $F$ on the hypotenuse of this
triangle. Prove that $\angle LCK=45^0$.}

Let $\alpha$ and $\beta$ be the inner angles at the vertices $A$ and
$B$ of the triangle $ABC$. According to the theorem \ref{VsotKotTrik},
$\alpha+\beta=90^0$. According to the theorem \textit{ASA} \ref{KSK},
$\triangle ACE\cong\triangle AKE$ and $\triangle BCF\cong\triangle BLF$, therefore
$EC\cong EK$ and $FC\cong FL$. So the triangles $CEK$ and $CFL$
 are congruent with the bases $CK$ and $CL$, which means that $\angle ECK\cong\angle EKC$ and
$\angle FCL\cong\angle FLC$. If
we use the theorem \ref{zunanjiNotrNotr} for the triangle $CFL$ and
the theorem \ref{VsotKotTrik} for the triangle $ALF$, we get: $\angle
FCL=\frac{1}{2}\angle LFA=\frac{1}{2}\left(90^0-\alpha \right)$.
Similarly, $\angle ECK=\frac{1}{2}\left(90^0-\beta
\right)$. From this it follows that $\angle FCL+\angle
ECK=\frac{1}{2}\left(90^0-\alpha +90^0-\beta\right)=45^0$. So
$\angle LCK=90^0-(\angle FCL+\angle ECK)=90^0-45^0=45^0$.

\item \res{Let $M$ be the center of the side $CD$ of the square $ABCD$ and $P$ such a point
 of the diagonal $AC$, that $3AP=PC$. Prove that $\angle BPM$
is a right angle.}

Let $S$ be the center of the diagonal $AC$ and $V$ the center of the line $SB$. Prove that $V$
 is the altitude of the triangle $PBC$ (see also the example
\ref{zgledPravokotnik}).

 \item \res{Let $P$, $Q$ and $R$ be the centers of the sides $AB$, $BC$ and $CD$
  of the parallelogram $ABCD$. The lines $DP$ and $BR$ intersect the line
$AQ$ in the points $K$ and $L$. Prove that $KL= \frac{2}{5} AQ$.}

Let $S$ be the center of the line $AD$ and $M$ the intersection of the lines $SC$
and $BR$. Prove first that $CM\cong AK$, the line $LQ$
is the median of the triangle $CBM$ and the line $PK$ is the median of the triangle
$LAB$.

 \item  \res{Let $D$ be the center of the hypotenuse $AB$ of the right-angled
triangle $ABC$ ($AC>BC$). The points $E$ and $F$ are the intersections of
the altitudes $CA$ and $CB$ with a line through $D$ and perpendicular
to the line $CD$. The point $M$ is the center of the line $EF$. Prove that
$CM\perp AB$.}

Let $T$ be the intersection of the lines $AB$ and $CM$, and $\angle
CAB=\alpha$ and $\angle CBA=\beta$. By assumption,
$\alpha+\beta=180^0$. Because $D$ is the center of the hypotenuse $AB$
of the right triangle $ABC$, by Tales' theorem
\ref{TalesovIzrKroz2} $DC\cong DA$. This means that $\triangle
CDA$ is an isosceles triangle and $\angle DCE=\angle
DCA\cong\angle DAC=\alpha$. From $FE\perp CD$ and $FC\perp CE$
it follows that $\angle CFE\cong\angle DCE=\alpha$ (by
\ref{KotaPravokKraki}). The point $M$ is the center of the hypotenuse $FE$
of the right triangle $FCE$, so (similarly to the triangle $ABC$) $\angle FCM\cong\angle CFM=\angle CFE=\alpha$.
In the triangle $BCT$ the sum of the two internal angles $\angle
CBT+\angle BCT=\angle CBA+\angle FCM=\alpha+\beta=90^0$. By
\ref{VsotKotTrik} it follows that $\angle CTB=90^0$ or
$CM\perp AB$.

\item \res{Let $A_1$ and $C_1$ be the centers of the sides $BC$ and $AB$ of the triangle $ABC$.
 The perpendicular to the internal angle at the vertex $A$ intersects the line
$A_1C_1$ at the point $P$. Prove that $\angle APB$ is a right angle.}

By \ref{KotiTransverzala}, $\angle C_1PA\cong\angle PAC$. Therefore $\angle C_1PA\cong\angle PAC\cong\angle PAC_1$, which
means that $PAC_1$ is an isosceles triangle with the base $AP$ or $C_1A\cong C_1P$ (by \ref{enakokraki}). Because $C_1$ is the center
of the side $AB$, $C_1B\cong C_1A\cong C_1P$. Therefore
the point $P$ lies on the circle with the diameter $AB$, so by Tales' theorem
\ref{TalesovIzrKroz2} $\angle APB=90^0$.

 \item \res{Let $P$ and $Q$ be such points of the sides $BC$ and $CD$ of the square $ABCD$,
  that the line $PA$ is the perpendicular to the angle $BPQ$. Determine the size of the angle
  $PAQ$.}

  Let $A'=pr_{\perp PQ}(A)$. Prove first that $\triangle
  ABP\cong\triangle AA'P$ and $\triangle ADQ\cong\triangle AA'Q$.
  Result: $\angle PAQ=45^0$.

\item \res{Prove that  the center of the circumscribed circle lies closest to the longest side
of the triangle.}

Let $O$ be the center of the inscribed circle of the triangle $ABC$ and $A_1$,
$B_1$ and $C_1$ be the centers of its sides $BC$, $AC$ and $AB$.
It is enough to prove, for example, that from $AC>AB$ it follows that $OB_1<OC_1$.
Assume that $AC>AB$. By the theorem \ref{vecstrveckot} in this case $\angle ABC>\angle ACB$. Because $B_1C_1$ is the median of the triangle $ABC$, $B_1C_1\parallel BC$ (theorem
\ref{srednjicaTrik}). Therefore, by the theorem \ref{KotiTransverzala}
$\angle AC_1B_1\cong \angle ABC$ and $\angle AB_1C_1\cong \angle
ACB$. From $\angle ABC>\angle ACB$ it now follows: $\angle
B_1C_1O=90^0-\angle AC_1B_1= 90^0-\angle ABC<90^0-\angle
ACB=90^0-\angle AB_1C_1=\angle C_1B_1O$ or $\angle
B_1C_1O<\angle C_1B_1O$. By the theorem \ref{vecstrveckot}  for
the triangle $B_1OC_1$ it is then $OB_1<OC_1$.

 \item \res{Prove that the center of the inscribed circle is closest to the vertex
of the largest internal angle of the triangle.}

Let $\alpha$, $\beta$ and $\gamma$ be the internal angles at
the vertices $A$, $B$ and $C$ of the triangle $ABC$, $S$ the
center of the inscribed circle of this triangle and $P$ the
point where this circle touches the side $BC$. It is enough to
prove, for example, that from $\beta>\gamma$ it follows that
$SB<SC$. From the assumption $\beta>\gamma$ it follows first
of all that $\angle
SBP=\frac{1}{2}\beta>\frac{1}{2}\gamma=\angle SCP$. Let $B'$ be
the point on the segment $PC$, for which $PB'\cong PB$. From
the similarity of the triangles $SPB'$ and $SPB$ (the
\textit{SAS} theorem \ref{SKS}) it follows that $SB'\cong SB$
and $\angle SB'P\cong \angle SBP>\angle SCP$. We prove that
$\mathcal{B}(P,B',C)$ or $PB'<PC$. In the opposite case, from
$B'=B$ it would follow that $\beta=\gamma$ (the similarity of
the triangles $SBP$ and $SCP$ - the \textit{SAS} theorem
\ref{SKS}), from $\mathcal{B}(P,C,B')$ that $\beta<\gamma$ (in
this case $\angle SCP>\angle SB'P\cong\angle SBP$ - the
\ref{zunanjiNotrNotrVecji} theorem for the triangle $SCB'$).
Since $\mathcal{B}(P,B',C)$, it follows that $\angle
SB'C=180^0-\angle SB'P=180^0-\frac{1}{2}\beta>90^0$. In the
right-angled triangle $SB'C$ it then follows that $SC<SB'\cong
SB$ (the \ref{vecstrveckot} theorem).

\item \res{Let $ABCD$ be a convex quadrilateral. Find a point $P$ so that
the sum $AP+BP+CP+DP$ is minimal.}

With the help of the triangle inequality (the \ref{neenaktrik}
theorem) we prove that the point $P$ is the intersection of
the diagonals of this quadrilateral.

\item \res{The diagonals $AC$ and $BD$ of an isosceles trapezoid $ABCD$ with
the base $AB$ intersect in the point $O$ and $\angle AOB=60^0$.
The points $P$, $Q$ and $R$ are, in order, the centers of the
distances $OA$, $OD$ and $BC$. Prove that $PQR$ is an
equilateral triangle.}

First of all, we prove that the triangles $AOB$ and $COD$ are
equilateral, then the relation $PQ=\frac{1}{2}AD$ and the fact
that the triangles $CPB$ and $CQB$ are right-angled with the
common hypotenuse $BC$.

\item \res{Let $P$ be an arbitrary internal point of the triangle $ABC$, for which
it holds that $\angle PBA\cong \angle PCA$. Points $M$ and $L$ are the perpendicular
projections of point $P$ on sides $AB$ and $AC$, point $N$ is the center of side
$BC$. Prove that $NM\cong NL$\footnote{Proposal for MMO 1982 (SL 9.).}.}

Let $\angle PBA\cong \angle PCA=\varphi$. Let $Q$ and
 $R$ be the centers of the lines $PC$ and $PB$. The line $NQ$ is the median of the triangle $PBC$ with the base $BP$, so according to the theorem
 \ref{srednjicaTrik} $NQ\parallel BP$ and
 $|NQ|=\frac{1}{2}|BP|=|RP|$ or $NQ\parallel RP$ and $NQ\cong
 RP$. From the theorem \ref{paralelogram} it follows that the quadrilateral
 $NQPR$ is a parallelogram. By the same theorem, $RN\cong PQ$ and
 $\angle PRN\cong\angle PQN$. So we have:
    \begin{eqnarray} \label{relacijaNalSkl42a}
      NQ\cong
     RP,\hspace*{2mm} NR\cong QP\hspace*{1mm} \textrm{ and }
     \hspace*{1mm}\angle PRN\cong\angle PQN
     \end{eqnarray}
  The line $RM$ is the altitude of the right triangle $BPM$ with the hypotenuse $BP$, so according to Tales' theorem
  \ref{TalesovIzrKroz2} $MR\cong RP\cong RB$. The triangle $MBR$ is an isosceles triangle with the base $MB$ and $\angle
  BMR=\angle MBR=\varphi$ (theorem \ref{enakokraki}. For the external
  angle $MRP$ of the triangle $MBR$ according to the theorem \ref{zunanjiNotrNotr}
  it holds that $\angle MRP=\angle BMR+\angle RBM=2\varphi$. So we have:
 \begin{eqnarray} \label{relacijaNalSkl42b}
      MR\cong RP  \hspace*{1mm}\textrm{ and }
     \hspace*{1mm}\angle MRP=2\varphi
     \end{eqnarray}
Similarly, from the right triangle $CPL$ and the isosceles triangle $LQC$ we get:
 \begin{eqnarray} \label{relacijaNalSkl42c}
      LQ\cong QP \hspace*{1mm} \textrm{ and }
     \hspace*{1mm}\angle LQP=2\varphi
     \end{eqnarray}
 From the relations  \ref{relacijaNalSkl42a},  \ref{relacijaNalSkl42b}
 and  \ref{relacijaNalSkl42c} it follows that $MR\cong RP \cong NQ$, $NR\cong
 QP \cong LQ$ and $\angle MRP\cong\angle LQP=2\varphi$. So we have:
   \begin{eqnarray} \label{relacijaNalSkl42d}
      MR \cong NQ,\hspace*{2mm} NR \cong LQ\hspace*{1mm} \textrm{ and }
     \hspace*{1mm}\angle MRP\cong\angle LQP
     \end{eqnarray}
 From  the third relation  from \ref{relacijaNalSkl42a} and the third
 relation from \ref{relacijaNalSkl42d} it follows that $\angle MRN=\angle
 MRP+\angle PRN=\angle LQP+\angle PQN=\angle LQN$. If we
 connect this with the first two relations from \ref{relacijaNalSkl42d},
 we get:
 \begin{eqnarray} \label{relacijaNalSkl42e}
      MR \cong NQ,\hspace*{2mm} NR \cong LQ\hspace*{1mm} \textrm{ and }
     \hspace*{1mm}\angle MRN\cong\angle LQN
     \end{eqnarray}
 This means that according to the theorem \textit{SAS} \ref{SKS} the triangles
 $MRN$ and $NQL$ are congruent, so $NM\cong
NL$.

\item \res{Let $P$, $Q$ and $R$ be the midpoints of sides $BC$, $AC$ and $AB$ of triangle $ABC$ ($AB<AC$) and let $D$ be the altitude vertex from vertex $A$. Prove that $\angle DRP\cong \angle DQP=\angle ABC-\angle ACB$.}

By problem \ref{nalSkl34}, quadrilateral $DPQR$ is an isosceles trapezoid with the base $DP$. From the congruence of triangles $DPR$ and $PDQ$ (the \textit{SSS} theorem \ref{SSS}), it follows that $\angle DRP\cong\angle DQP$. We will also prove that $\angle DQP=\angle ABC-\angle ACB=\beta-\gamma$. Quadrilateral $BPQR$ is a parallelogram (theorem \ref{srednjicaTrik}), so $\angle RQP=\angle ABC=\beta$. Since $\angle ADP=90^0$, by the Tales theorem \ref{TalesovIzrKroz2}, $QD\cong QC\cong QA$. Therefore, $DCQ$ is an isosceles triangle with the base $DC$ and $\angle QDC\cong \angle QCD=\gamma$ (theorem \ref{enakokraki}). From $DP\parallel RQ$ by the theorem \ref{KotiTransverzala}, it follows that $\angle RQD\cong \angle QDP$. In the end, we have: $\angle DQP=\angle RQP-\angle DQP=\beta-\angle QDP=\beta-\angle QDC=\beta-\gamma$.

\item \res{Let $AD$ be the altitude of the internal angle at vertex $A$ ($D\in BC$) of triangle $ABC$ and let $E$ be a point on side $AB$ such that $\angle BDE\cong\angle BAC$. Prove that $DE\cong DC$.}

Let $F$ be a point on segment $AC$ such that $AF\cong AE$. First, we will prove that $\triangle AED\cong\triangle AFD$ and $\triangle FDC$ is an isosceles triangle with the base $FC$.

Let $P$, $Q$ and $R$ be any points that divide the perimeter of a square $ABCD$ into three equal parts. It is clear that in this case, these three points lie in succession on three sides of the square. Without loss of generality, we assume that $P \in [AB]$, $Q \in [BC]$ and $R \in [AD]$. Let $P_0$ be the center of the line $BC$ and $Q_0 \in [BC]$ and $R_0 \in [AD]$ be such points that also the triplet $P_0$, $Q_0$ and $R_0$ divide the perimeter of the square $ABCD$ into three equal parts. In this case, it is clear that $CQ_0 \cong DR_0$, so from the similarity of the triangles $COQ_0$ and $DOR_0$ (the \textit{SAS} theorem \ref{SKS}) it follows that $OQ_0 \cong OR_0$ and $\angle BQ_0O \cong \angle AR_0O$ (the corresponding outer angles of these triangles).
 Without loss of generality, we can assume that $\mathcal{B}(P_0,P,C)$. We prove that in this case $|OP_0|+|OQ_0|+|OR_0| < |OP|+|OQ|+|OR|$. Because the triplets $P$, $Q$ and $R$ and $P_0$, $Q_0$ and $R_0$ both divide the perimeter of the square $ABCD$ into three equal parts, we have $|P_0P|=|Q_0Q|=|R_0R|=d$, and also $\mathcal{B}(Q_0,Q,C)$ and $\mathcal{B}(D,R_0,R)$ hold. Let $R'$ be such a point on the side $CD$ that $CR' \cong DR$. In this case, we have $Q_0R'=CR'-CQ_0=DR-DR_0=R_0R=Q_0Q=d$, which means that the line $OQ_0$ is the altitude of the triangle $OR'Q$. From the similarity of the triangles $OQ_0R'$ and $OR_0R$ (the \textit{SAS} theorem \ref{SKS}) it also follows that $OR \cong OR'$. If we use the statement from the example \ref{neenTezisZgl}, we get $OQ_0 < \frac{1}{2}\left(|OQ|+|OR'| \right)$, i.e. $|OQ_0|+|OR_0|=2|OQ_0| < |OQ|+|OR'|=|OQ|+|OR|$. Because from the right triangle $OP_0P$ it is also $|OP_0| < |OP|$ (the \textit{Pythagorean theorem}), at the end we have $|OP_0|+|OQ_0|+|OR_0| < |OP|+|OQ|+|OR|$.

\item \res{There is a given finite number of lines that divide the plane into areas.
Prove that the plane can be colored with two colors, so that each area is colored with one color, and adjacent areas are always colored with different colors.}

The proof is by induction on the number of lines. For $n=1$ one line divides the plane into two areas and the statement is true.

Assume that the statement is true for any $n$ arbitrary lines, or that there exists an appropriate coloring for them. We show that the statement is true for $n+1$. Let $p_1,p_2,\ldots , p_n,p_{n+1}$ be arbitrary lines in the plane $\alpha$. The line $p_{n+1}$ determines the half-planes $\alpha_1$ and $\alpha_2$. Consider the areas determined by the lines $p_1,p_2,\ldots , p_n$, that is, without the line $p_{n+1}$. By the induction assumption, these areas can be colored with two colors, so that each area is colored with one color, and adjacent areas are always colored with different colors. We denote this coloring by $\mathcal{B}_n$. The appropriate coloring $\mathcal{B}_{n+1}$ of the areas determined by the lines $p_1,p_2,\ldots , p_n,p_{n+1}$ is defined as follows:
\begin{itemize}
  \item the areas in the half-plane $\alpha_1$ retain the same color that is determined by the coloring $\mathcal{B}_n$,
  \item the areas in the half-plane $\alpha_2$ change the color that is determined by the coloring $\mathcal{B}_n$.
\end{itemize}
The coloring $\mathcal{B}_{n+1}$ satisfies the given conditions. In the case when adjacent areas have a boundary side that lies on the line $p_{n+1}$, by the definition of the coloring $\mathcal{B}_{n+1}$, the areas are of different colors. In the case when adjacent areas do not have a boundary side that lies on the line $p_{n+1}$, both lie in the half-plane $\alpha_1$ or both lie in the half-plane $\alpha_2$. Also in this case, the areas are of different colors, which follows directly from the definitions of the colorings $\mathcal{B}_n$ and $\mathcal{B}_{n+1}$.


\item \res{Draw a triangle $ABC$, if the following data are given (see the labels in section \ref{odd3Stirik}):} \label{nalSklKonstrTrik}

(\textit{a}) \res{$\alpha$, $\beta$, $s$}

Let $E$ and $F$ be such points that $\mathcal{B}(E,A,B,F)$, $EA\cong CA$ and $FB\cong CB$ hold true. This allows for the construction of the triangle $ECF$ ($EF\cong s$, $\angle CEF = \frac{1}{2}\cdot \alpha$ and $\angle CFE = \frac{1}{2}\cdot \beta$).

(\textit{b}) \res{$a-b$, $c$, $\gamma$}

 Let $D$ be such a point that $\mathcal{B}(C,D,B)$ and $CD\cong CA$ hold true. First, we draw the triangle $ADB$ ($AB\cong c$, $DB=a-b$ and $\angle ADB=90^0+\frac{1}{2}\cdot\gamma$).

(\textit{c}) \res{$a$, $\beta-\gamma$, $b-c$}

 We mark with $D$ that point for which $\mathcal{B}(A,D,C)$ and $AD\cong AB$ hold true. First, we draw the triangle $DBC$ ($BC\cong a$, $DC=b-c$ and $\angle DBC=\frac{1}{2}\cdot\left(\beta-\gamma\right)$).

(\textit{d}) \res{$a$, $\beta-\gamma$, $b+c$}

 Let $D$ be such a point that $\mathcal{B}(B,A,D)$ and $AD\cong AC$ hold true. This allows for the first construction of the triangle $DBC$ ($BC\cong a$, $DB=b+c$ and $\angle DCB = 90^0-\frac{1}{2}\cdot\left(\beta-\gamma\right)$).

(\textit{e}) \res{$b$, $c$, $v_a$}

 First, we draw an arbitrary line $p$ and a point $A$, which is $v_a$ units away from the line $p$, then the points $B\in k(A,c)$ and $C\in k(A,b)$.

(\textit{f}) \res{$b$, $v_a$, $v_b$}

 Let $A'=pr_{\perp BC}(A)$. First, we draw the right triangle $AA'C$ with hypotenuse $AC=b$ and leg $AA'=v_a$, then the point $B$ as the intersection of the line $A'C$ and the parallel $l$ of the line $AC$ at a distance $v_b$.

(\textit{g}) \res{$\alpha$, $v_a$, $v_b$}

Let $B'=pr_{\perp AC}(B)$. First, we draw the right triangle $ABB'$ (from the conditions: $\angle AB'B=90^0$, $\angle BAB'\cong\alpha$ and $BB'=v_b$), then the point $C$ as the intersection of the line $AB'$ and the tangent of the circle $k(A,v_a)$ at the point $B$.

(\textit{h}) \res{$c$, $a+b$, $\gamma$}

 See Example \ref{konstrTrik1}.

(\textit{i}) \res{$v_a$, $\alpha$, $\beta$}

Let $A'=pr_{\perp BC}(A)$. First, we draw the right triangle $ABA'$ (from the conditions: $\angle ABB'\cong\beta$, $AA'=v_a$ and $\angle AA'B=90^0$), then the point $C$ as the intersection of the line $BB'$ and the other arm $p$ as $\angle BA,p=\alpha$.

(\textit{j}) \res{$b$, $a+c$, $v_c$}

 Let $C'=pr_{\perp AB}(C)$ and $D$ be such a point that $\mathcal{B}(A,B,D)$ and $BD\cong BC$. We draw in order: the line $AD=a+c$, the parallel $l$ of the line $AD$ at a distance $v_c$, $C\in l\cap k(A,b)$ and finally the point $B$ as the intersection of the line $AD$ and the line $CD$.

(\textit{k}) \res{$b-c$, $v_b$, $\alpha$}


 Let $B'=pr_{\perp AC}(B)$ and $D$ be such a point that $\mathcal{B}(A,D,C)$ and $AD\cong AB$. First, we draw the right triangle $ABA'$ ($\angle BAB'\cong\alpha$, $BB'=v_b$ and $\angle AB'B=90^0$), then the point $D$ on the segment $AC$ with the condition $AD\cong AB$ and finally such a point $C$, that $\mathcal{B}(A,D,C)$ and $DC=b-c$.

(\textit{l}) \res{$a$, $t_b$, $t_c$}

 Let $BB_1$ and $CC_1$ be the midpoints and $T$ be the center of the triangle $ABC$. First, we draw the triangle $BTC$ ($BC\cong a$, $BT=\frac{2}{3}\cdot t_b$ and $CT=\frac{2}{3}\cdot t_c$ - izreka \ref{tezisce} and \ref{izrekEnaDelitevDaljice}), then the points $B_1$ and $C_1$ on the segments $BT$ and $CT$ from the conditions $BB_1\cong t_b$ and $CC_1\cong t_c$ and finally the point $A$ as the intersection of the segments $BC_1$ and $CB_1$.

(\textit{m}) \res{$b$, $c$, $t_a$}

 Let $AA_1$ be the midpoint of the triangle $ABC$ and $D$ be the point that is symmetrical to the point $A$ with respect to the point $A_1$. The quadrilateral $ABA_1C$ is a parallelogram (izrek \ref{paralelogram}), which allows us to first construct the triangle $ADC$ ($AC\cong b$, $CD\cong c$ and $AD=2\cdot t_a$), then the point $B$, which is symmetrical to the point $C$ with respect to the point $A_1$.

 (\textit{n}) \res{$t_a$, $t_b$, $t_c$}

Let $AA_1$ be the altitude and $T$ be the centroid of the triangle $ABC$ and $D$ be the point that is symmetrical to the point $T$ with respect to the point $A_1$. The quadrilateral is a parallelogram (see Theorem \ref{paralelogram}). By Theorem \ref{tezisce}, $TC= \frac{2}{3}\cdot t_c$, $CD\cong BT= \frac{2}{3}\cdot t_b$ and $TD=2\cdot TA_1=\frac{2}{3}\cdot t_a$. This allows us to construct the triangle $TDC$ first, then the vertices $B$ and $A$.

(\textit{o}) \res{$c$, $v_a$, $l_a$}

Let $AA'$ be the altitude and $AE$ ($E\in BC$) be the line of symmetry of the internal angle $BAC$ of the triangle $ABC$. We first construct the right-angled triangle $AA'E$ ($AA'\cong v_a$, $\angle AA'E=90^0$ and $AE=l_a$), then the vertices $B\in AE\cap k(A,c)$ and the vertices $C$ as the intersection of the line $AE$ and the other leg $p$ of the angle $\angle EA,p\cong\angle BAE$.

 (\textit{p}) \res{$c$, $v_a$, $t_b$}

 Let $AA'$ be the altitude, $BB_1$ be the centroid of the triangle $ABC$ and $B_2=pr_{\perp BC}(B_1)$. The distance $B_1B_2$ is the median of the triangle $AA'C$, so $B_1B_2=\frac{1}{2}\cdot AA'$ (see Theorem \ref{srednjicaTrik}). Now we construct the right-angled triangle $ABA'$ ($AA'\cong v_a$, $\angle AA'B=90^0$ and $AB=c$). The point $B_1$ is the intersection of the circle $k(B, t_b)$ and the parallel $l$ of the line $BA'$ at a distance of $\frac{1}{2}\cdot v_b$. The vertices $C$ are the intersection of the lines $BA'$ and $AB_1$.

 (\textit{r}) \res{$b$, $l_a$, $\alpha$}

 Let $AE$ ($E\in BC$) be the line of symmetry of the internal angle $BAC$ of the triangle $ABC$. We first construct the triangle $AEC$ ($AE\cong l_a$, $\angle AEC=\frac{1}{2}\cdot\alpha$ and $AC=b$), then the vertices $A$ as the intersection of the line $CE$ and the other leg $p$ of the angle $EA,p=\frac{1}{2}\cdot\alpha$.

 (\textit{s}) \res{$v_a$, $v_b$, $t_a$}

Let $AA'$ and $BB'$ be the altitude and $AA_1$ the median of the triangle $ABC$ and $A_2=pr_{\perp AC}(A_1)$. The distance $A_1S_2$ is the median of the triangle $BB'C$, so $A_1A_2=\frac{1}{2}\cdot BB'$ (statement \ref{srednjicaTrik}). Now we construct the right triangle $AA'A_1$ ($AA'\cong v_a$, $\angle AA'A_1=90^0$ and $AA_1=t_a$). The vertex $C$ is obtained as the intersection of the line $A'A_1$ and the tangent of the circle $k(A_1,\frac{1}{2}\cdot v_b)$ from the point $A$. The vertex $B$ is the point that is symmetrical to the vertex $C$ with respect to the point $A_1$.

(\textit{t}) \res{$t_a$, $v_b$, $b+c$}

Let $A_1$ be the center of the side $BC$, $D$ the point that is symmetrical to the point $A$ with respect to the point $A_1$, and $E$ the point such that $\mathcal{B}(D,B,E)$ and $EB\cong AB$. Because $ABDC$ is a parallelogram (statement \ref{paralelogram}), $BD\cong AC$. Next, $AD=2\cdot AA_1=2\cdot t_a$, $DE=BD+BE=AC+AB=b+c$, the point $A$ is distant from the line $ED$ by $v_b$. These facts allow the construction of the triangle $AED$. The vertex $B$ is then obtained as the intersection of the line $AE$ and the line $ED$, and the vertex $C$ is the fourth vertex of the parallelogram $ABDC$.

 (\textit{u}) \res{$a$, $b$, $\alpha-\beta$}

 Let $D$ be the point such that $\mathcal{B}(C,D,B)$ and $CD\cong CA$. Then $BD=a-b$ and $\angle DAB=\frac{1}{2}\left(\alpha-\beta \right)$. These facts allow the construction. First, we plan the points $C$, $B$ and $D$ from the conditions $\mathcal{B}(C,D,B)$, $CB\cong a$ and $CD\cong b$. The vertex $A$ is the intersection of the circle $k(C,b)$ and the set of points from which the distance $BD$ is seen at an angle of $\frac{1}{2}\left(\alpha-\beta \right)$ (statement \ref{ObodKotGMT}).


            \item \res{Draw an isosceles triangle $ABC$ , if given:}
In all cases, we assume that $BC$ is the base of this triangle.

        (\textit{a}) \res{Base and sum of the leg and altitude to the base.}

Let $A'$ be the altitude vertex from the vertex $A$.
We first draw the triangle $DA'B$, where $A'B=\frac{1}{2}a$, $DA'=v_a+b$ and $\angle DA'B=90^0$.

        (\textit{b}) \res{Perimeter and altitude on the base.}

Let $A'$ be the altitude vertex from the vertex $A$.
We first draw the triangle $AA'P$, where $A'P=s$, $AA'=v_a$ and $\angle AA'P=90^0$.

        (\textit{c}) \res{Both altitudes.}

Let $AA'=v_a$ and $BB'=v_b$ be the altitudes of this triangle, and $D$ be the orthogonal projection of the point $A'$ onto the side $AC$. Since $A'$ is the center of the base $BC$, $A'D$ is the altitude of the triangle $BCD$ for its side $BB'$, so $A'D=\frac{1}{2}v_b$. This allows us to construct the right triangle $AA'D$.

        (\textit{d}) \res{Angle at the base and the section of its altitude.}

We use the fact that the angles at the base are complementary.

        (\textit{e}) \res{Side and the altitude vertex on it.}

Let's say that the given side is $AC$ and the vertex $B'$ is the altitude $BB'$ on this side. The vertex $B$ is one of the intersections of the circle $k(A,AC)$ with the perpendicular of the side $AC$ at the point $B'$ (the altitude of $BB'$).

        (\textit{f}) \res{Side and the altitude.}

We first draw the right triangle $ABB'$, where $B'$ is the altitude $BB'$ of the triangle $ABC$.


            \item \res{Draw a right triangle $ABC$ with a right angle at the vertex $C$, if the following data are given:}

 (\textit{a}) \res{$\alpha$, $a+b$}

Let $D$ be the point for which $\mathcal{B}(A,C,D)$ and $CD\cong CB$ hold. First, draw the triangle $ABD$ ($DA=a+b$, $\angle DAB=\alpha$ and $\angle BDA=45^0$).

 (\textit{b}) \res{$\alpha$, $a-b$}

Let $D$ be the point for which $\mathcal{B}(A,D,C)$ and $CD\cong CB$ hold. First, draw the triangle $ABD$ ($DA=a-b$, $\angle DAB=\alpha$ and $\angle BDA=135^0$).

 (\textit{c}) \res{$a$, $b+c$}

Let $D$ be a point for which $\mathcal{B}(C,A,D)$ and $CD\cong AB$ holds. First, draw the triangle $CBD$ ($CD=b+c$, $CB=a$ and $\angle BCD=90^0$).

 (\textit{d}) \res{$c$, $a+b$}

Let $D$ be a point for which $\mathcal{B}(A,C,D)$ and $CD\cong CB$ holds. First, draw the triangle $ABD$ ($DA=a+b$, $AB=c$ and $\angle BDA=45^0$).

 (\textit{e}) \res{$t_a$, $t_c$}

Let $AA_1=t_a$ and $CC_1=t_c$ be the altitudes and $T$ be the centroid of the right triangle $ABC$. First, we draw the altitude $AA_1=t_a$, then the centroid $T$ from the condition $A_1T=\frac{1}{3}A_1A$, the vertex $C$ as the intersection of the circle above the diameter $AA_1$ and the circle $k(T,\frac{2}{3}t_c)$, and finally the vertex $B$, which is symmetrical to the vertex $C$ with respect to the point $A_1$.

 (\textit{f}) \res{$a$, $c-b$}

Let $D$ be a point for which $\mathcal{B}(A,C,D)$ and $AD\cong AB$ holds. First, draw the triangle $CBD$ ($CD=c-b$, $BC=a$ and $\angle BCD=90^0$).

(\textit{g}) \res{$a+v_c$, $\alpha$}

First, we draw the right triangle $CC'B$, where $C'$ is the vertex of the triangle $ABC$ from the vertex $C$.

 (\textit{h}) \res{$t_c$, $v_c$}


First, we draw the right triangle $CC'C_1$, where $C'$ is the vertex of the triangle $ABC$ from the vertex $C$ and $C_1$ is the center of its side $AB$. Then we use the fact $C_1C\cong C_1A\cong C_1B$ (Tales' theorem \ref{TalesovIzrKroz2}).

 (\textit{i}) \res{$a$, $t_a$}

Let $AA_1=t_a$ be the altitude of the triangle $ABC$. First, we draw the triangle $AA_1C$.

 (\textit{j}) \res{$v_c$, $l_c$}


First, we draw the right triangle $CC'E$, where $C'$ is the vertex of the triangle $ABC$ from the vertex $C$ and $E$ is the intersection of the altitude of the inner angle $ACB$ with its side $AB$. Then we use the fact $\angle ECA\cong \angle ECB=45^0$.

\item \res{Draw a rectangle $ABCD$, if given:}
        \begin{enumerate}
        \item \res{The diagonal and one side.}

First, we draw the rectangular triangle $ABC$.

        \item \res{The diagonal and the perimeter.}

First, we draw the rectangular triangle $ABC$ from the condition $AC=d$ and $AB+BC=\frac{1}{2}o$.

        \item \res{One side and the angle between the diagonals.}


First, we draw the isosceles triangle $ABS$, where $S$ is the intersection of the diagonals of the rectangle $ABC$.

        \item \res{Perimeter and the angle between the diagonals.}

Let $E$ be such a point that $\mathcal{B}(C,B,E)$ and $BE\cong AB$. First, we draw the triangle $AEC$ from the conditions: $CE=AB+BC$, $\angle ACE=\angle ACB=\frac{1}{2}\angle ASB$ and $\angle AEC=45^0$.

        \end{enumerate}

            \item \res{Draw a rhombus $ABCD$, if given:}

In all cases, let $a$ be the side, $e$ and $f$ the diagonals, and $\alpha$ the internal angle $BAD$ of the rhombus $ABCD$.

        \begin{enumerate}
        \item \res{Side and sum of diagonals.}

Let $S$ be the intersection of the diagonals of the rhombus $ABCD$ and $E$ be such a point that $\mathcal{B}(A,S,E)$ and $SE\cong SB$. First, we draw the triangle $ABE$ from the conditions: $AB=a$, $AE=\frac{e+f}{2}$ and $\angle AEB=45^0$.

        \item \res{Side and difference of diagonals.}


Let $S$ be the intersection of the diagonals of the rhombus $ABCD$ and $E$ be such a point that $\mathcal{B}(A,E,S)$ and $SE\cong SB$. First, we draw the triangle $ABE$ from the conditions: $AB=a$, $AE=\frac{e-f}{2}$ and $\angle AEB=135^0$.

        \item \res{One angle and sum of diagonals.}


Let $S$ be the intersection of the diagonals of the rhombus $ABCD$ and $E$ be such a point that $\mathcal{B}(A,S,E)$ and $SE\cong SB$. First, we draw the triangle $ABE$ from the conditions: $\angle SAB=\frac{1}{2}\alpha$, $AE=\frac{e+f}{2}$ and $\angle AEB=45^0$.

        \item \res{One angle and difference of diagonals.}

Let $S$ be the intersection of the diagonals of the rhombus $ABCD$ and $E$ be such a point that $\mathcal{B}(A,E,S)$ and $SE\cong SB$ hold. First, we draw the triangle $ABE$ from the conditions: $\angle SAB=\frac{1}{2}\alpha$, $AE=\frac{e-f}{2}$ and $\angle AEB=135^0$.

        \end{enumerate}

            \item \res{Draw a parallelogram $ABCD$, if the following is given:}

In all cases, we denote by $a=AB$ and $b=BC$ the sides, $e=AC$ and $f=BD$ the diagonals, and $\alpha$, $\beta$, $\gamma$, $\delta$ the internal angles at the vertices $A$, $B$, $C$ and $D$ of the parallelogram $ABCD$.

        \begin{enumerate}

        \item \res{One side and diagonals.}

First, we draw the triangle $ASB$, where $S$ is the intersection of the diagonals of the parallelogram $ABCD$.

        \item \res{One side and altitude.}

We assume that the side $AB$ is given.
First, we draw the right-angled triangle $ABB'$, where $B'$ is the orthogonal projection of the vertex $B$ onto the line $AD$ of the parallelogram $ABCD$.

        \item \res{One diagonal and altitude.}


We assume that the diagonal $AC=e$ is given.
First, we draw the right-angled triangle $ASS'$, where $S$ is the intersection of the diagonals of the parallelogram $ABCD$ and $S'$ is the orthogonal projection of this point onto the line $AB$ ($SA'=\frac{v_a}{2}$, $\angle SA'A=90^0$ and $AS=\frac{e}{2}$). The vertex $C$ is symmetric to the vertex $A$ with respect to the point $S$.
Then we construct the right-angled triangle $ACC'$, where $B'$ is the orthogonal projection of the vertex $B$ onto the line $AD$ of the parallelogram $ABCD$ (use the fact that $CC'=v_b$ and Tales' theorem \ref{TalesovIzrKroz2}).

        \item \res{The side $AB$, as at the vertex $A$ and the sum $BC+AC$.}

Let $E$ be such a point that $\mathcal{B}(B,C,E)$ and $CE\cong CA$ hold. First, we draw the triangle $ABE$ from the conditions: $AB=a$, $BE=e+b$ and $\angle ABC=180^0-\alpha$.

        \end{enumerate}

                 \item \res{Draw a trapezoid $ABCD$, if the following is given:}

In all examples, let $a=AB$ and $c=CD$ be the bases, $b=BC$ and $d=DA$ the legs, $e=AC$ and $f=BD$ the diagonals, and $\alpha$, $\beta$, $\gamma$, $\delta$ the internal angles at the vertices $A$, $B$, $C$ and $D$ of the trapezoid $ABCD$.

\begin{enumerate}

\item \res{Bases, leg and smaller angle that does not lie on that leg.}

Without loss of generality, let $a=AB$, $c=CD$, $b=BC$ and $\alpha=\angle DAB$ be given. Let $E$ be the fourth vertex of the parallelogram $BCDE$. First, draw the triangle $AED$ ($AE=a-c$, $\angle DAE=\alpha$ and $DE=b$).

\item \res{Bases and diagonals.}


Let $E$ be the fourth vertex of the parallelogram $CDBE$. First, draw the triangle $AEC$ ($AE=a+c$, $AC=e$ and $CE=f$).

\item \res{Bases and angle at the longer base.}

Let $E$ be the fourth vertex of the parallelogram $BCDE$. First, draw the triangle $AED$ ($AE=a-c$, $\angle DAE=180^0-\delta$ and $\angle DEA=180^0-\gamma$).

\item \res{Sum of bases, altitude and angle at the longer base.}

Let $PQ$ be the median of the trapezoid $ABCD$. First, draw the trapezoid $ABPQ$.

\end{enumerate}


\item \res{Draw the deltoid $ABCD$, if the diagonal $AC$, which lies on the altitude of the deltoid, $\angle CAD$ and the sum $AD+DC$ are given.}

Let $E$ be such a point that $\mathcal{B}(A,D,E)$ and $DE\cong DC$ hold. First, draw the triangle $ACE$.


\item \res{Draw the quadrilateral $ABCD$, if the following is given:}

In all cases, we denote by $a=AB$, $b=BC$, $c=CD$ and $d=AD$ the sides, $e=AC$ and $f=BD$ the diagonals, and $\alpha$, $\beta$, $\gamma$, $\delta$ the internal angles at the vertices $A$, $B$, $C$ and $D$ of the quadrilateral $ABCD$.

\begin{enumerate}
\item \res{Four sides and one angle.}

Assume that the angle $DAB=\alpha$ is given. First, draw the triangle $DAB$.

\item \res{Four sides and the angle enclosed by the lines perpendicular to the opposite sides.}

Let $E$ be the fourth vertex of the parallelogram $BCDE$. First, we draw the triangle $ABE$ ($AB=a$, $BE=c$ and $\angle ABE\cong \angle AB,CD$).

\item \res{Three sides and the angle at the fourth side}

Without loss of generality, let the sides $AB$, $BC$, $AD$ and the angles $BCD$ and $CDA$ be given.
Let $E$ be the fourth vertex of the parallelogram $BCDE$. First, we draw this parallelogram ($CB=b$, $CD=c$ and $\angle BCD=\delta$).

        \item \res{The centers of three sides and the distance that is consistent and parallel to the fourth side}

Use the statement \ref{Varignon}.

        \end{enumerate}


\end{enumerate}




%REŠITVE - Skladnost in krožnica
%________________________________________________________________________________

\poglavje{Congruence and Circle}

\begin{enumerate}

\item   \res{The lengths of the sides of a triangle are $6$, $7$ and $9$. Let $k_1$, $k_2$ and $k_3$ be the circles with centers
at the vertices of this triangle. The circles touch each other so that
 the circle with the center at the vertex of the smallest angle of the triangle touches the other two circles from the inside,
the other two circles touch from the outside. Calculate the lengths of the radii of these three circles.}

The lengths of the radii are obtained as the solution of the system $r_1-r_2=9$, $r_1-r_3=7$, $r_2+r_3=6$. So $r_a=11$, $r_2=2$ and $r_3=4$.

\item \res{Prove that the angle determined by the secants of the circle that intersect outside the circle is equal to half the difference of the central angles, adjacent to the vertex of this angle.} \label{nalSkk2}

Let $S$ be the center of the circle, $P$ the intersection of the secants $s_1$ and $s_2$, which intersect the circle in the points $N_1$ and $M_1$ or $N_2$ and $M_2$ (where $\mathcal{B}(P,N_1,M_1)$ and $\mathcal{B}(P,N_2,M_2)$). Let also: $\alpha=\angle s_1,s_2=\angle N_1PN_2$, $\beta=\angle M_1SM_2$, $\gamma=\angle N_1SN_2$, $\varphi=\angle M_1SN_1$ and  $\psi=\angle M_2SN_2$. From the quadrilateral $PN_1SN_2$ we get (from the statement \ref{VsotKotVeck}): $\alpha=360^0-\left( \gamma+90^0-\frac{\varphi}{2} +90^0-\frac{\psi}{2}\right)=180^0-\frac{\varphi+\psi}{2}-\gamma=
    \frac{360^0-\varphi-\psi}{2}-\gamma=\frac{\beta+\gamma}{2}-\gamma=
    \frac{\beta-\gamma}{2}$.

\item \res{The vertex of the angle $\alpha$ is an external point of the circle $k$. Between the arms of this angle, on the circle, there are two
arcs, which are in the ratio $3:10$. The larger of these arcs corresponds to the central angle $40^0$. Determine
the size of the angle $\alpha$.}

First, we determine (with the help of the appropriate extreme ratio), that the other central angle measures $12^0$. From the previous task (\ref{nalSkk2}) it then follows that $\alpha=\frac{40^0-12^0}{2}=14^0$.

\item  \res{Prove that the angle between the tangents of the circle is equal to half the difference of the central angles of the arcs lying between the arms of this angle.}

    Let $PT_1$ and $PT_2$ be the tangents that touch the circle with center $S$ in the points $T_1$ and $T_2$. With $\beta$ and $\gamma$ ($\beta>\gamma$) we mark the central angles. Because according to the statement \ref{TangPogoj} $\angle ST_1P=\angle ST_2P=90^0$ we get (from the quadrilateral $PT_1ST_2$ - statement \ref{VsotKotVeck}): $\angle T_1PT_2=180^0-\gamma=
    \frac{\beta+\gamma}{2}-\gamma=
    \frac{\beta-\gamma}{2}$.

\item \res{Let $L$ be the orthogonal projection of any point $K$ of the circle $k$
 on its tangent $t$ through the point $T\in k$ and $X$ the point that is
symmetrical to the point $L$ with respect to the line $KT$. Determine the geometric
position of the points $X$.}

Let's mark the center of the circle $k$ with $S$ and $\alpha=\angle LTK$. We will first prove that $S$, $X$ and $K$ are collinear points. From the congruence of the triangles $TLK$ and $TXK$ it follows that $\angle XTK\cong \angle LTK=\alpha$, $\angle TXK\cong \angle TLK=90^0$ and $\angle TKX\cong \angle TKL=90^0-\alpha$. According to the formulas \ref{ObodKotTang} and \ref{SredObodKot} $\angle TSK=2\angle LTK=2\alpha$. Therefore, from the isosceles triangle $TSK$ (formula \ref{enakokraki} and \ref{VsotKotTrik}) we get: $\angle TKS=\frac{180^0-2\alpha}{2}=90^0-\alpha=\angle TKX$ or $\angle TKS\cong\angle TKX$, which means that $X\in SK$. In the end $\angle SXT\cong\angle KXT=90^0$, which according to the formula \ref{TalesovIzrKroz2} means that the point $X$ lies on the circle $l$ with the diameter $TS$. It is not difficult to prove that the converse is also true - each point $X\in l$ can be obtained from some point $K\in k$ according to the described procedure.

\item \res{Let $BB'$ and $CC'$ be the altitudes of the triangle $ABC$ and $t$
the tangent of the circumscribed circle of this triangle at point $A$. Prove that
$B'C'\parallel t$.}

Let's mark any point of the tangent $t$ with $P$, so that $P$ and $C$ are on different sides of this tangent. According to the formula \ref{ObodKotTang} $\angle PAC'=\angle PAB\cong\angle ACB$. Because $BCC'B'$ is a trapezoid (formula \ref{TalesovIzrKroz2}), according to the formula \ref{TetivniPogojZunanji} $\angle AC'B'\cong\angle ACB$. Therefore $\angle AC'B'\cong\angle PAC'$, which according to the formula \ref{KotiTransverzala} means that $PA\parallel B'C'$ or $B'C'\parallel t$.

\item \res{In the right triangle $ABC$ above the leg $AC$ with the angle as the diameter
the circle is drawn, which intersects the hypotenuse $AB$ at point $E$. The tangent of this
circle at point $E$ intersects the other leg $BC$ at point $D$. Prove that $BDE$
is an isosceles triangle.}

If we use the formula \ref{ObodKotTang}, we get $\angle DEC\cong\angle EAC$. Therefore:
$\angle BED=90^0-\angle DEC=90^0-\angle EAC=90^0-\angle BAC=\angle ABC=\angle EBD$. From
$\angle BED\cong\angle EBD$ according to the formula \ref{enakokraki} it follows that $BD\cong ED$, which means that $BDE$ is an isosceles triangle.

\item \res{In a right angle with the top $A$ is inscribed a circle that touches the sides of this angle in
the points $B$ and $C$. Any tangent to this circle intersects the line $AB$ and $AC$ in succession in the points $M$ and $N$ (so that $\mathcal{B}(A,M,B)$). Prove that:
$$\frac{1}{3}\left(|AB|+|AC|\right) < |MB|+|NC| <
\frac{1}{2}\left(|AB|+|AC|\right).$$}

We denote with $T$ the point of intersection of the aforementioned circle with the line $MN$ and $s=\frac{1}{2}(|MN|+|NA|+|AM|)$ the perimeter of the triangle $MAN$. The given circle is the inscribed circle of the triangle $MAN$, so according to the great theorem it holds:
\begin{eqnarray} \label{nalSkk8Eqn1}
|AB|=|AC|=s.
\end{eqnarray}
We prove first the inequality $|MB|+|NC| <\frac{1}{2}\left(|AB|+|AC|\right)$.
If we use the statement \ref{TangOdsek} and the triangle inequality \ref{neenaktrik} and the relation \ref{nalSkk8Eqn1}, we get:
 \begin{eqnarray*}
|MB|+|NC| &=& |MT|+|NT| =|MN|=\\
 &=& \frac{1}{2}\left( |MN|+|MN| \right)\\
 &<& \frac{1}{2}\left( |MN|+|NA|+|AM| \right)=\\
 &=& s=|AB|=\\
 &=& \frac{1}{2}\left(|AB|+|AC|\right).
\end{eqnarray*}
We prove the second inequality: $|MB|+|NC| >\frac{1}{3}\left(|AB|+|AC|\right)$.
Since $MAN$ is a right triangle with hypotenuse $MN$, according to the statement \ref{vecstrveckotHipot} it follows:
\begin{eqnarray} \label{nalSkk8Eqn2}
|MN|>|MA| \hspace*{1mm} \textrm{ and } \hspace*{1mm} |MN|>|NA|.
\end{eqnarray}
If we use the statement \ref{TangOdsek} and the relations \ref{nalSkk8Eqn2} and \ref{nalSkk8Eqn1}, we get:
 \begin{eqnarray*}
|MB|+|NC| &=& |MT|+|NT| =|MN|=\\
 &=& \frac{1}{3}\left( |MN|+|MN|+|MN| \right)\\
 &>& \frac{1}{3}\left( |MN|+|NA|+|AM| \right)=\\
 &=& \frac{1}{3}\cdot2s=\frac{1}{3}\cdot2\cdot|AB|=\\
 &=& \frac{1}{3}\left(|AB|+|AC|\right).
\end{eqnarray*}

\item \res{Prove that in a right triangle the sum of the sides is equal to the sum of
the diameters of the inscribed and circumscribed circle.}

Let $ABC$ be a right angled triangle with a right angle at point $C$. Use the same notation as in the big task \ref{velikaNaloga}. Because $\angle PCQ=\angle BCA=90^0$, the quadrilateral $PCQS$ is a square, so by the big task (\ref{velikaNaloga}) it follows:
 \begin{eqnarray} \label{nalSkk9Eqn}
r=|SQ|=|CP|=s-c.
\end{eqnarray}
 From this and Tales' theorem (\ref{TalesovIzrKroz2}) it follows:
 $$R+r=\frac{c}{2}+s-c=b+c.$$

\item \res{The lines of symmetry of the internal angles of a convex quadrilateral intersect in six different points.
Prove that four of these points are the vertices of the angle bisector quadrilateral.}

Use the criterion for the angle bisector - theorem \ref{TetivniPogoj}.

\item \res{Let: $c$ be the length of the hypotenuse, $a$ and $b$ be the lengths
of the catheti, and $r$ be the radius of the inscribed circle of a right angled triangle. Prove that:}
\begin{enumerate}
 \item \res{$2r + c \geq 2 \sqrt{ab}$}

Use the relation \ref{nalSkk9Eqn} and the inequality between arithmetic and geometric mean.

 \item \res{$a + b + c > 8r$}

Use the relation \ref{nalSkk9Eqn} and Pythagoras' theorem \ref{PitagorovIzrek}.
\end{enumerate}

\item \res{Let $P$ and $Q$ be the centers of the shorter arcs $AB$ and $AC$
of the inscribed circle of a regular triangle $ABC$. Prove that the sides $AB$ and $AC$ of this triangle divide
the line $PQ$ into three congruent segments.}

Let $X$ and $Y$ be the intersection points of the chord $PQ$ with the sides $AB$ and $AC$ of the triangle $ABC$. Prove that $PX\cong XY\cong YQ$ holds. By the theorem \ref{TockaN}, $BQ$ and $CP$ are the altitudes of the internal angles $ABC$ and $ACB$ of the triangle $ABC$. By the theorem \ref{ObodObodKot} we have:
  \begin{eqnarray*}
\angle PAX &=& \angle PAB\cong\angle PCB=\frac{1}{2}\angle ACB=30^0\\
\angle APX &=& \angle APQ\cong\angle ABQ=\frac{1}{2}\angle ABC=30^0.
\end{eqnarray*}
Therefore:
 $\angle PAX = \angle APX=30^0$, which means that $APX$ is an isosceles triangle with the base $AP$ (theorem \ref{enakokraki}) or $PX\cong XA$. By the theorem \ref{zunanjiNotrNotr} we also have $\angle AXY=\angle APX+\angle PAX=60^0$. Since $\angle XAY=\angle BAC= 60^0$, $AXY$ is a right triangle, therefore $XA\cong XY\cong YA$. Similarly, $AQY$ is an isosceles triangle with the base $AQ$ or $QY\cong YA$. If we connect the proven relations of congruence of the segments, we get: $PX\cong XA\cong XY\cong AY\cong YQ$.

\item \res{Let $k_1$, $k_2$, $k_3$, $k_4$ be four circles, each of which externally touches one side   and two vertices of an arbitrary convex
quadrilateral. Prove that the centers of these circles are the points of a concurrence.}

First, prove that the two opposite angles of the quadrilateral have measures $\frac{\alpha+\beta}{2}$ and $\frac{\gamma+\delta}{2}$, where $\alpha$, $\beta$, $\gamma$ and $\delta$ are the internal angles of the initial quadrilateral, and then use theorems \ref{TetivniPogoj} and \ref{VsotKotVeck}.

\item \res{The circles $k$ and $l$ externally touch in the point $A$. The points $B$ and $C$ are the points of
the common external tangent of these two circles. Prove that $\angle BAC$ is a right angle.}

Let $L$ be the intersection of the common tangent of the circles $k$ and $l$ in the point $A$ with the distance $BC$. Because $XA$ and $XB$ are tangents of the circle $k$ in the points $A$ and $B$, $XA\cong XB$ (statement \ref{TangOdsek}). Similarly, $XA$ and $XC$ are tangents of the circle $l$ in the points $A$ and $C$, or $XA\cong XC$. Therefore, $X$ is the center of the circle with the diameter $BC$ and the point $A$ lies on this circle. By Tales' statement \ref{TalesovIzrKroz2}, $\angle BAC=90^0$.

\item \res{Let $ABCD$ be a deltoid ($AB\cong AD$ and $CB\cong CD$). Prove:}
\begin{enumerate}
 \item \res{$ABCD$ is a tangential quadrilateral.}

Use statement \ref{TangentniPogoj}.

 \item  \res{$ABCD$ is a tautological quadrilateral exactly when $AB\perp BC$.}

Use statement \ref{TetivniPogoj}.
\end{enumerate}



\item \res{The circles $k$ and $k_1$ touch each other from the outside in the point $T$, in which they intersect the lines $p$ and $q$. The line $p$ has
also intersections with the circles $P$ and  $P_1$, the line $q$ has $Q$ and $Q_1$ intersections.  Prove that $PQ\parallel P_1Q_1$.}

Let $t$ be the common tangent of the circles $k$ and $k_1$, $X\in t$ any point of this tangent, for which $P,X\div QQ_1$ and $Y$ the point which is symmetrical to the point $X$ with respect to $T$. By statements \ref{ObodKotTang} and \ref{sovrsnaSkladna}, we have:
 \begin{eqnarray*}
 \angle QPP_1= \angle QPT \cong\angle QTX\cong\angle Q_1TY\cong\angle Q_1P_1T
 =\angle Q_1P_1P
 \end{eqnarray*}
Therefore, $\angle QPP_1 \cong\angle Q_1P_1P$, so by statement \ref{KotiTransverzala} $PQ\parallel P_1Q_1$.

\item \res{Let $MN$ be the common tangent of the circles $k$ and $l$ ($M$ and $N$ are touch points),
which intersect in the points $A$ and $B$. Calculate the value of the sum $\angle MAN+\angle MBN$.}

Without loss of generality, assume that $d(A,MN)<d(B,MN)$.
If we use statement \ref{ObodKotTang} and \ref{VsotKotTrik}, we get:
 \begin{eqnarray*}
 \angle MBN &=& \angle MBA +\angle ABN =\\
&=& \angle NMA +\angle MNA = \\
&=&  180^0-\angle MAN.
 \end{eqnarray*}
Therefore, $\angle MAN+\angle MBN=180^0$.

\item \res{Let $t$ be a tangent to the circle with radius $ABC$ drawn in point $A$. The line parallel to the tangent $t$ intersects the sides $AB$ and $AC$ in points $D$ and $E$.
Prove that the points $B$, $C$, $D$ and $E$ are concyclic.}

We mark with $P$ the point of the tangent $t$ so that $T,E\div AB$ holds. If we use the expression \ref{KotiTransverzala} and \ref{ObodKotTang}, we get:
 \begin{eqnarray*}
 \angle ADE \cong \angle PAD =\angle PAB \cong \angle ACB=\angle ECB.
 \end{eqnarray*}
From $\angle ADE \cong \angle ECB$ it follows from \ref{TetivniPogojZunanji} that $BCED$ is a cyclic quadrilateral, so the points $B$, $C$, $D$ and $E$ are concyclic.

\item \res{Let $D$ and $E$ be arbitrary points of the circle drawn over the diameter $AB$. Let $AD\cap BE= \{F\}$ and $AE\cap BD= \{G\}$.
Prove that $FG\perp AB$.}

By Tales' theorem \ref{TalesovIzrKroz2} we have $\angle ADB\cong\angle AEB=90^0$, so $BD\perp AF$ and $AE\perp BF$. The point $G$ is the altitude point of the triangle $ABF$, so according to \ref{VisinskaTocka} the line $FG$ is also the altitude of the third height and $FG\perp AB$.

\item \res{Let $M$ be a point on the circle $k(O,r)$. Determine the geometric center
 of all the tangents of this circle that have one endpoint in point $M$.}

We prove that the sought geometric center $\mathcal{G}$ is equal to the circle $k_1$ over the diameter $OM$ or in other words we prove $\mathcal{G}=k_1$. It is enough to prove that for any point $Y$ of the plane the equivalence holds:
\begin{eqnarray*}
 Y\in\mathcal{G} \hspace*{1mm}\Leftrightarrow \hspace*{1mm}Y\in k_1.
 \end{eqnarray*}
For the points $O$ and $M$ it obviously holds that $O,M\in \mathcal{G}$ and  $O,M\in k_1$. So we assume that $Y\neq O$ and $Y\neq M$.

($\Rightarrow$) We assume that $Y\in\mathcal{G}$. This means that $Y$ is the center of some tangent $MX$ of the circle $k$. The triangles $MYO$ and $XYO$ are similar (\textit{SSS} theorem \ref{SSS}), so $\angle OYM\cong\angle OYX=90^0$, which means (by Tales' theorem \ref{TalesovIzrKroz2}) that $Y\in k_1$ holds.

($\Leftarrow$) Let $Y\in k_1$. We mark the point $X$, which is symmetrical to the point $M$ with respect to the point $Y$. From $Y\in k_1$ follows from Tales' theorem (\ref{TalesovIzrKroz2}) that $\angle OYM=90^0$, i.e. $\angle OYM\cong\angle OYX=90^0$. Triangles $MYO$ and $XYO$ are congruent (\textit{SAS} theorem \ref{SKS}), therefore $OX\cong OM$, which means that the point $X$ lies on the circle $k$. So $Y$ is the center of the chord $AX$ of the circle $k$, i.e. $Y\in\mathcal{G}$.

It is left to the reader to solve the task in a simpler way - by using the central extension $h_{M,\frac{1}{2}}$ (see section \ref{odd7SredRazteg}). Or do we generalize the task in this sense, if we use the central extension $h_{M,k}$ for an arbitrary coefficient $k$?

\item \res{Let $M$ and $N$ be points that are symmetrical to the vertex $A'$ of the altitude of the triangle $ABC$ with respect to the side $AB$ and $AC$, and $K$ is the intersection of the lines $AB$ and $MN$. Prove that the points $A$, $K$, $A'$, $C$ and $N$ are concircular.}

First, we prove that $CK$ is the altitude of the triangle $ABC$, and then that the points $K$, $A'$ and $N$ lie on the circle with diameter $AC$.

\item \res{Let $D$ be the vertex of the altitude from the vertex $A$ of the acute-angled triangle $ABC$ and $O$
the center of the circumscribed circle of this triangle. Prove that $\angle CAD\cong \angle BAO$.}

First, we prove that $\angle BAD\cong\angle OAC =90^0-\angle ABC$.

From $DE\perp AB$ and $CF\perp AB$ it follows directly that $DE\parallel CF$. We will also prove that $\angle EDC+\angle DCF=180^0$. If we use the expressions \ref{ObodObodKot} and \ref{VsotKotTrik}, it holds:
\begin{eqnarray*}
 \angle EDC &=& \angle EDB+\angle BDC=
 90^0-\angle ABD+\angle CAB\\
\angle DCF &=& \angle ACF+\angle DCA=
 90^0-\angle CAB+\angle ABD.
 \end{eqnarray*}
From this it follows that $\angle EDC+\angle DCF=180^0$, which means (according to the expression \ref{paralelogram}), that the quadrilateral $CDEF$
is a parallelogram.

The second proof of this expression is given in the part of the proof of the claim from the example \ref{TetivniVisinska}.

\item \res{The circles with centers $O_1$ and $O_2$ intersect in points $A$ and $B$. The line $p$,
which goes through the point $A$, intersects these two circles in points $M_1$ and $M_2$. Prove
that $\angle O_1M_1B\cong\angle O_2M_2B$.}

Because the triangles $O_1M_1B$ and $O_2M_2B$ are isosceles with the bases $M_1B$ and $M_2B$, it is enough to prove (according to the expressions \ref{VsotKotTrik} and \ref{enakokraki}) that the central angles $M_1O_1B$ and $M_2O_2B$ are congruent. Without loss of generality, we assume that $A, O_1\div BM_1$ and $A,O_2\ddot{-} BM_2$. Let $A'$ be the point that is symmetrical to the point $A$ with respect to the center $O_1$. According to the expressions \ref{SredObodKot} and \ref{ObodObodKotNaspr}, it holds:
 \begin{eqnarray*}
 \angle M_2O_2B &=& 2\angle M_2AB=2(180^0-\angle M_1AB)=\\
&=& 2\angle M_1A'B=\angle M_1O_1B.
 \end{eqnarray*}

\item \res{The semicircle with center $O$ is drawn above the diameter $AB$.
Let $C$ and $D$ be such points on the line $AB$ that $CO\cong OD$. The parallel lines through points $C$ and $D$ intersect the semicircle in points $E$ and $F$.
Prove that the lines $CE$ and $DF$ are perpendicular to the line $EF$.}

Let $S$ be the center of the line $EF$. The line $SO$ is the median of the trapezoid $CDFE$ with bases $CE$ and $DF$, so $OS\parallel EC$ or $OS\parallel FD$ (statement \ref{srednjTrapez}). By the \textit{SSS} statement \ref{SSS}, $\triangle OSE\cong\triangle OSF$, so $\angle OSE\cong\angle OSF=90^0$. Therefore $OS\perp EF$ or $CE\perp EF$ and $DF\perp EF$.

\item \res{On the string $AB$ of the circle $k$ with the center $O$, the point $C$ lies, and the point $D$ is
the other intersection of the circle $k$ with the circumscribed circle of the triangle $ACO$. Prove that
$CD\cong CB$.}

Because $A,B\in k$, $AOB$ is an isosceles triangle with the base $AB$, so $\angle OBA\cong\angle OAB$ (statement \ref{enakokraki}). By statement \ref{ObodObodKot}, $\angle OAC\cong\angle ODC$. If we connect the proven relations, we get:
 \begin{eqnarray*}
 \angle OBC=\angle OBA=\angle OAB=\angle OAC\cong\angle ODC.
 \end{eqnarray*}
From $\angle OBC\cong\angle ODC$, $OC\cong OC$ and $OB\cong OD$ it follows (from the \textit{SSA} statement \ref{SSK}) that $\triangle OBC\cong\triangle ODC$ or $CB\cong CD$.


\item \res{Let $AB$ be a transversal of the circle $k$. The lines $AC$ and $BD$ are tangents
to the circle $k$ in points $C$ and $D$.
Prove that
 $$||AC|-|BD||< |AB| < |AC|+|BD|.$$}

Since $AB$ is a secant of the circle $k$, $A$ is an external point of that circle, so there are two tangents $AC$ and $AC_1$ ($C,C_1\in k$) to that circle from point $A$. According to \ref{TangOdsek} theorem $AC\cong AC_1$. Similarly, from point $B$ there are two tangents $BD$ and $BD_1$ ($D,D_1\in k$) to the circle $k$ and $BD\cong BD_1$ holds. So, without loss of generality, we can choose such a pair of tangents $AC$ and $BD$ that the lines $AC$ and $BD$ intersect in some point $E$. If we use the triangle inequality \ref{neenaktrik}, we get:
 \begin{eqnarray*}
 |AB| < |AE|+|BE| < |AC|+|BD|.
 \end{eqnarray*}
For the second inequality we use the triangle inequality \ref{neenaktrik} again and the relation $EC\cong ED$ (\ref{TangOdsek} theorem):
 \begin{eqnarray*}
 |AB| > ||AE|-|BE||= |\left(|AC|-|EC|\right)-\left(|BD|-|ED|\right)| =  ||AC|-|BD||.
 \end{eqnarray*}


\item \res{Let $S$ be the intersection of the altitudes of the sides $AD$ and $BC$
of the trapezoid $ABCD$ with the base $AB$. Prove that the circumscribed circles of the triangles $SAB$ and
$SCD$ touch each other in point $S$.}

Use \ref{TangOdsek} theorem.

\item \res{The lines $PB$ and $PD$ touch the circle $k(O,r)$ in points $B$ and $D$.
The line $PO$ intersects the circle $k$ in points $A$ and $C$ ($\mathcal{B}(P,A,C)$). Prove that the
line $BA$ is the angle bisector of the angle $PBD$.}


Use \ref{enakokraki} and \ref{TangOdsek} theorems.

\item \res{The quadrilateral $ABCD$ is inscribed in a circle with center $O$. The diagonals $AC$ and
$BD$ are perpendicular. Let $M$ be the perpendicular projection of the center $O$
on the line $AD$. Prove that
 $$|OM|=\frac{1}{2}|BC|.$$}

From the congruence of the triangles $OMA$ and $OMD$ (the SSA theorem \ref{SSK}) it follows that $MA\cong MD$ or that the point $M$ is the center of the line segment $AD$.
We mark with $E$ the intersection of the diagonals $AC$ and $BD$. Let $D'$ be the point that is symmetrical to the point $D$ with respect to the center $O$. It is clear that $D'\in k$. By Tales' theorem \ref{TalesovIzrKroz2} we have $\angle DAD'=90^0$. The line segment $MO$ is the median of the triangle $DAD'$ for the side $AD'$, so $|OM|=\frac{1}{2}|AD'|$. It is enough to prove that $AD'\cong BC$ or (by the theorem \ref{SklTetSklObKot}) that $\angle ADD'\cong \angle BDC$. If we now use the theorem \ref{VsotKotTrik} and \ref{ObodObodKot}, we get:
\begin{eqnarray*}
 \angle ADD' &=& 90^0-\angle AD'D =90^0- \angle ACD =\\
&=& 90^0-\angle ECD =\angle EDC =\\
&=& \angle BDC.
 \end{eqnarray*}


\item \res{Daljici $AB$ in $BC$ sta sosednji stranici pravilnega devetkotnika, ki je včrtan krožnici $k$ s središčem $O$.
Točka $M$ je središče stranice $AB$, točka $N$ pa središče
polmera $OX$ krožnice $k$, ki je pravokoten na premico $BC$. Dokaži, da je
$\angle OMN=30^0$.}

From $\angle AOB\cong\angle BOC=\frac{360^0}{9}=40^0$ it follows that:
 $$\angle AOX=\angle AOB + \angle BOX=\angle AOB + \frac{1}{2}\angle BOC=60^0.$$
Since $OA\cong OB$, the triangle $AOB$ is a right triangle (theorems \ref{enakokraki} and \ref{VsotKotTrik}). Its median $AN$ is also the altitude, so $\angle ONA=90^0$. Since $\angle OMA=90^0$, by Tales' theorem the points $N$ and $M$ lie on the circle with the radius $OA$ or that the quadrilateral $ONMA$ is a square. If we now use the theorem \ref{ObodObodKot}, we get:
\begin{eqnarray*}
 \angle OMN=\angle OAN=\frac{1}{2}\angle OAX=30^0.
 \end{eqnarray*}

\item \res{The circles $k_1$ and $k_2$ intersect in points $A$ and $B$. Let $p$ be a line that goes through point $A$, intersects circle $k_1$ in point $C$ and circle $k_2$ in point $D$, and let $q$ be a line that goes through point $B$, intersects circle $k_1$ in point $E$ and circle $k_2$ in point $F$. Prove that $\angle CBD\cong\angle EAF$.}

If we use \ref{VsotKotTrik} and \ref{ObodObodKot}, we get:
\begin{eqnarray*}
 \angle CBD &=& 180^0-\angle BCD - \angle BDC =\\
             &=& 180^0-\angle BCA - \angle BDA =\\
             &=& 180^0-\angle BEA - \angle BFA =\\
             &=& 180^0-\angle FEA - \angle EFA =\\
             &=& \angle EAF =\\
 \end{eqnarray*}


\item \res{The circles $k_1$ and $k_2$ intersect in points $A$ and $B$. Draw a line $p$ that goes through point $A$, so that the length of the line segment $MN$, where $M$ and $N$ are the intersections of line $p$ with circles $k_1$ and $k_2$, is maximal.}

Let $p$ be an arbitrary line that goes through point $A$ and intersects circles $k_1$ and $k_2$ in points $M$ and $N$.
Let $S_1$ and $S_2$ be the centers of circles $k_1$ and $k_2$ and let $P_1$ and $P_2$ be the centers of line segments $MA$ and $NA$. Because $MN=2\cdot P_1P_2$, the problem of the maximum length of line segment $MN$ can be translated to the maximum length of line segment $P_1P_2$. From the similarity of triangles $MS_1P_1$ and $AS_1P_1$ (\textit{SSS} \ref{SSS}) it follows that $\angle S_1P_1M\cong\angle S_1P_1A=90^0$ or $\angle P_2P_1S_1=90^0$. Similarly, $\angle P_2P_1S_1=90^0$. Therefore, $S_1S_2P_2P_1$ is a right trapezoid with height $P_1P_2$, so $P_1P_2\leq S_1S_2$. Equality is achieved when $S_1S_2P_2P_1$ is a rectangle or $P_1P_2\parallel S_1S_2$. This means that line $p$ (for which $|MN|$ is maximal) is drawn as a parallel to line $S_1S_2$ through point $A$.

\item \res{Let $L$ be the orthogonal projection of an arbitrary point $K$ of the circle $k$ onto its tangent at the point $T\in k$, and let $X$ be the point that is symmetric to the point $L$ with respect to the line $KT$. Determine the geometric position of the point $X$.}

We denote by $S$ the center of the circle $k$ and by $k_1$ the circle with diameter $ST$.
We prove that the geometric position of the point $X$ (which we denote by $\mathcal{G}$) is equal to the circle $k_1$. We need to prove the equivalence:
\begin{eqnarray*}
 X\in \mathcal{G} \hspace*{1mm} \Leftrightarrow \hspace*{1mm} X\in k_1
 \end{eqnarray*}

For the points $T$ and $S$, it is clear from the definition that both $T\in \mathcal{G}$ and $T\in k_1$ hold, as well as $S\in \mathcal{G}$ and $S\in k_1$. We assume further that $X\neq T$ and $X\neq S$.

We first assume $X\in \mathcal{G}$. We denote $\angle LTK=\alpha$.
The triangles $XKT$ and $LKT$ are symmetric with respect to the line $KT$ (see subsection \ref{odd6OsnZrc}), so $\triangle XKT\cong \triangle LKT$ holds, i.e. $\angle XTK\cong \angle LTK=\alpha$ and $\angle KXT\cong \angle KLT=90^0$. We prove that also $\angle SXT=90^0$. It is enough to prove that the points $S$, $X$ and $K$ are collinear, i.e. that $\angle TKX=\angle TKS$ holds. First, we have (from $\triangle XKT\cong \triangle LKT$):
 $\angle TKX\cong\angle TKL =90^0-\alpha$. If we use the statement \ref{enakokraki} of the congruent triangles $KST$ with the base $KT$ and the statements \ref{ObodKotTang} and \ref{SredObodKot}, we get:
 $\angle TKS=\frac{1}{2}(180^0-\angle KST)=\frac{1}{2}(180^0-2\angle LTK)=90^0-\alpha$.
Therefore $\angle TKX=\angle TKS=90^0-\alpha$, which means that the points $S$, $X$ and $K$ are collinear, i.e. $\angle SXT=90^0$. From the latter it follows that $X\in k_1$.

Let's now assume that $X\in k_1$. We'll mark the intersection of the line segment $SX$ with the circle $k$ with $K$ and the point $L$ that is symmetrical to the point $X$ with respect to the line $TK$.
From $X\in k_1$ follows from Tales' theorem \ref{TalesovIzrKroz2} that $\angle SXT=90^0$, so $\angle KXT=90^0$ as well.
To prove that $X\in \mathcal{G}$, it is enough to prove that $X$ lies on the tangent $t$ of the circle $k$ through the point $T$. Let $L'$ be an arbitrary point on the tangent $t$, for which $L,X\ddot{-} ST$ holds. It is enough to prove $\angle LTK\cong\angle L'TK$.
We'll mark $\angle LTK=\alpha$.
The triangles $XKT$ and $LKT$ are symmetrical with respect to the line $KT$, so $\triangle XKT\cong \triangle LKT$ or $\angle XTK\cong \angle LTK=\alpha$ and $\angle KLT\angle KXT\cong =90^0$. If we use the formula \ref{ObodKotTang} and \ref{SredObodKot}, we get:
$\angle L'TK=\frac{1}{2}\angle TSK=\frac{1}{2}(180^0-2\angle SKT)=90^0-\angle SKT=90^0-\angle XKT=\angle XTK=\alpha=\angle LTK$.
Therefore $L'\in t$ or $X\in \mathcal{G}$.

\item \res{Prove that a string polygon with an even number of vertices, which has all the internal angles compliant, is a regular polygon.}

We'll mark the center of the circumscribed circle of this polygon with $O$ and the centers of the sides $A_iB_{i+1}$ with $B_i$ ($i\in \{1,2,\ldots , 2n-1 \}$, $A_0=A_{2n-1}$ and $A_{2n}=A_1$). We'll use the fact that $A_iOA_{i+1}$ is an isosceles triangle with the base $A_iA_{i+1}$ and we'll first prove $\triangle OB_{i-1}A_i\cong \triangle OB_{i+1}A_{i+1}$.

Let $EA$ be the common tangent of the circles $k$ (larger) and $l$ (smaller) and $D$ the other intersection of the line $AB$ with the circle $l$. Let $L$ be the other intersection of the segment $AK$ with the circle $l$, $\alpha=\angle DCB$ and $\beta=\angle DBC$.
By the theorem \ref{ObodKotTang} it follows:
 \begin{eqnarray} \label{nalSkk36Eqn1}
 \angle CAB=\angle CAD\cong \angle DCB=\alpha
\end{eqnarray}
and
 \begin{eqnarray} \label{nalSkk36Eqn2}
 \angle LDA\cong \angle EAL=\angle EAK\cong \angle ABK=\angle DBC=\beta.
\end{eqnarray}
Since $CDA$ is the exterior angle of the triangle $CDB$, it follows from the theorem \ref{zunanjiNotrNotr} that
 $\angle CDA=\angle DCB+\angle CBD=\alpha+\beta$. From this and \ref{nalSkk36Eqn2} it follows:
 \begin{eqnarray} \label{nalSkk36Eqn3}
 \angle LDC= \angle CDA-\angle LDA=\alpha+\beta-\beta=\alpha.
\end{eqnarray}
From the relations \ref{nalSkk36Eqn3} and \ref{nalSkk36Eqn1} and the theorem \ref{ObodObodKot}:
\begin{eqnarray*}
 \angle KAC= \angle LAC\cong \angle LDC=\alpha=\angle CAB.
\end{eqnarray*}
This means that the line $AC$ is the perpendicular bisector of the angle $BAK$.

The task is a special case of a more general task \ref{nalSkk47}.

\item \res{Let $BC$ be the diameter of the circle $k$. Determine the geometric position of the altitude points
of all triangles $ABC$, where $A$ is an arbitrary point on the circle $k$.}

The desired geometric position of the points is the circle that passes through the points $B$ and $C$. Calculate the measure of the angle $BVC$, where $V$ is the altitude point of the triangle $ABC$, and use the theorem \ref{ObodKotGMT}. See also the theorem \ref{TockaV'a}.

\item \res{Let $ABCDEF$ be a tetrahedral hexagon and $AB\cong DE$ and $BC\cong EF$.
Prove that $CD\parallel AF$.}

Let $ABCDEF$ be a tetrahedral hexagon and $AB\cong DE$ and $BC\cong EF$.
Since $\angle BAD>90^0$ and $\angle BCD>90^0$, by the theorem \ref{obodKotGMTZunNotr} $A$ and $C$ are inner points of the circle $k$. This means that $AC<BD$.

Prove that $\angle ASC\cong\angle DSF$, where $S$ is the center of the inscribed circle of the hexagon $ABCDEF$.

\item \res{Let $ABCD$ be a convex quadrilateral, where $\angle ABD=50^0$, $\angle ADB=80^0$, $\angle ACB=40^0$ and $\angle DBC=\angle BDC +30^0$. Calculate the measure of the angle $\angle DBC$.}

In the triangle $ABD$ $\angle DAB=180^0-80^0-50^0=50^0$ (\ref{VsotKotTrik}). Therefore, $\angle DAB\cong \angle DBA =50^0$, which means that $ABD$ is an isosceles triangle (\ref{enakokraki}), thus $DA\cong DB$. From this it follows that the circle $k$ with center $D$, which passes through the point $A$, also passes through the point $B$. Because $\angle ACB=40^0=\frac{1}{2}\angle ADB$, according to \ref{ObodKotGMT} the point $C$ also lies on the circle $k$. From this it follows that $DAC$ is an isosceles triangle or $\angle DAC\cong\angle DCA=x$ (\ref{enakokraki}). From this $\angle DCB=\angle DCA+\angle ACB=x+40^0$.
According to \ref{SredObodKot} $\angle BDC=2\angle BAC=2(50^0-x)=100^0-2x$. From the given condition $\angle DBC=\angle BDC +30^0=130^0-2x$. If we use the sum of the interior angles in the triangle $BCD$ (\ref{VsotKotTrik}), we get the equation:
 \begin{eqnarray*}
 (100^0-2x)+(x+40^0)+(130^0-2x)=180^0.
 \end{eqnarray*}
From the solution of this equation $x=30^0$ we get $\angle DBC=130^0-2x=70^0$.

\item \res{Let $M$ be an arbitrary internal point of the angle with the vertex $A$, points $P$ and $Q$ the orthogonal projections of the point $M$ on the sides of this angle, and point $K$ the orthogonal projection of the vertex $A$ on the line $PQ$. Prove that $\angle MAP\cong \angle QAK$.}

According to Tales' theorem \ref{TalesovIzrKroz2} the quadrilateral $APMQ$ is cyclic.
If we use \ref{ObodObodKot} and \ref{KotaPravokKraki}, we get:
\begin{eqnarray*}
 \angle MAP\cong \angle MQP\cong \angle QAK.
\end{eqnarray*}


\item \res{In the cyclic octagon $A_1A_2\ldots A_8$ it holds that $A_1A_2\parallel A_5A_6$, $A_2A_3\parallel A_6A_7$,
$A_3A_4\parallel A_7A_8$. Prove that $A_8A_1\cong A_4A_5$.}

First, from $A_1A_2\parallel A_5A_6$ and  $A_2A_3\parallel A_6A_7$ by \ref{KotaVzporKraki} it follows that $\angle A_1A_2A_3\cong\angle A_5A_6A_7$. Because $A_1A_2A_3A_4$ and $A_5A_6A_7A_8$ are taut quadrilaterals, by \ref{TetivniPogoj} it holds:
\begin{eqnarray*}
 \angle A_5A_8A_7=180^0- \angle A_5A_6A_7=180^0-\angle A_1A_2A_3= \angle A_1A_4A_3.
\end{eqnarray*}
Since $A_7A_8\parallel A_3A_4$, also $A_5A_8\parallel A_1A_4$ (from \ref{KotaVzporKraki}). Therefore, $A_1A_4A_5A_8$ is a taut trapezoid with bases $A_5A_8$ and $A_1A_4$, so its sides $A_8A_1$ and $A_4A_5$ are congruent (from \ref{trapezTetivEnakokr}).

\item \res{A circle intersects each side of a quadrilateral in two points and thus on all sides of the quadrilateral it determines congruent sides.
Prove that this quadrilateral is tangent.}

Prove that the center of this circle is the same distance from all four sides of the triangle and thus coincides with the center of the inscribed circle.

\item \res{The lengths of the sides of a tangent pentagon $ABCDE$ are natural numbers and at the same time $|AB|=|CD|=1$.
The inscribed circle of the pentagon touches the side $BC$ in the point $K$.
Calculate the length of the line $BK$.}

Let $J$, $L$, $M$ and $N$ be the points of intersection of sides $AB$, $CD$, $DE$ and $EA$ of pentagon $ABCDE$ with its inscribed circle. Let $x=|JB|$, $y=|CL|$ and $z=|EN|$. From the given conditions $|AB|=|CD|=1$ it follows first that $|AJ|=1-x$ and $|LD|=1-y$, then also that $x<1$ and $y<1$. By \ref{TangOdsek} we have $|BK|=|BJ|=x$ and $|CK|=|CL|=y$ or $|BC|=x+y$. Because the lengths of all sides of pentagon $ABCDE$ are natural numbers, also $x+y\in \mathbb{N}$. From this $x<1$ and $y<1$ it follows that $x+y=1$ or $y=1-x$. Therefore $|LD|=1-y=x$.
By \ref{TangOdsek} we have first $|EM|=|EN|=z$, $|AN|=|AJ|=1-x$ and $|DM|=|DL|=x$ or $|DE|=z+x\in \mathbb{N}$ and $|EA|=z+1-x\in \mathbb{N}$. From the last two relations (by subtracting) we get $2x-1\in \mathbb{N}$. Because $x<1$, we get $|BK|=x=\frac{1}{2}$.

\item \res{Prove that the circle that passes through the adjacent vertices $A$ and $B$ of
a regular pentagon $ABCDE$ and its center $O$ also passes through the intersection
of its diagonals $AD$ and $BE$.}

Let $P$ be the intersection of diagonals $AD$ and $BE$. Prove first that $\angle APB\cong\angle AOB=36^0$.

\item \res{Let $H$ be the altitude point of triangle $ABC$, $l$ the circle above the diameter $AH$
and $P$ and $Q$ the intersections of this circle with sides $AB$ and $AC$. Prove that
the tangents of circle $k$ through points $P$ and $Q$ intersect at side $BC$.}

Prove first that the tangent of circle $l$ at point $P$ intersects side $BC$ at its center. Use \ref{TalesovIzrKroz2} and \ref{ObodKotTang}.

\item \label{nalSkk47}
\res{Circle $l$ intersects circle $k$ from the inside at point $C$. Let $M$ be any point
of circle $l$ (different from $C$). The tangent of circle $l$ at point $M$ intersects circle $k$ at
points $A$ and $B$. Prove that $\angle ACM \cong \angle MCB$.}

Let $K$ and $L$ be the centers of the circles $k$ and $l$, with $EC$ their common tangent at point $C$ ($E,A\ddot{-} KL$) and with $F$ and $G$ the other intersections of the line segments $CA$ and $CB$ with the circle $l$. By the theorem \ref{ObodKotTang} it follows:
 \begin{eqnarray*}
 \angle FGC\cong\angle FCE=\angle ACE\cong\angle ABC.
 \end{eqnarray*}
From $\angle FGC\cong\angle ABC$ by the theorem \ref{KotiTransverzala} it follows that $FG\parallel AB$. Since $AB$ is a tangent of the circle $l$ at point $M$, it follows that $LM\perp AB$. From the proven $FG\parallel AB$ it follows that $LM\perp EF$, which means that the line of the radius $LM$ of the circle $l$ is perpendicular to its chord $EF$. The circle $l$ is circumscribed by the triangle $CGF$, and the point $M$ is the intersection of this circle with the perpendicular of the side $FG$ of this triangle, so by the theorem \ref{TockaN} it follows that $CM$ is perpendicular to the angle $FCG$ or the angle $ACB$. Therefore $\angle ACM \cong \angle MCB$.

\item \res{Let $k$ be the circumscribed circle of the triangle $ABC$ and $R$ the center of that arc $AB$ of this circle,
which does not contain the point $C$. The line segments $RP$ and $RQ$ are the chords of this circle. The first one
is parallel, the second one is perpendicular to the perpendicular of the internal angle  $\angle BAC$. Prove:
\begin{enumerate}
\item the line $BQ$ is perpendicular to the internal angle $\angle CBA$,
\item  the triangle, which is determined by the lines $AB$, $AC$ and $PR$, is a right (isosceles) triangle.
 \end{enumerate}}

\textit{(a)}  Use the theorem from the example \ref{tockaNtockePQR}.

\textit{(b)} Let $AED$ be the given triangle. Use the theorem \ref{zunanjiNotrNotr} and \ref{KotaVzporKraki} and prove that $\angle DEA\cong \angle EDA=\frac{1}{2}\angle BAC$.

\item \res{Let $X$ be such an internal point of the triangle $ABC$, that it holds:
 $\angle BXC =\angle BAC+60^0$,  $\angle AXC =\angle ABC+60^0$ and  $\angle AXB =\angle AC B+60^0$. Let
$P$, $Q$ and $R$ be the other intersections of the lines $AX$, $BX$ and $CX$ with the circumscribed circle of the triangle $ABC$. Prove that the triangle $PQR$ is a right triangle.}

The angle $BXC$ is the exterior angle of the triangle $BRX$, so by the statement \ref{zunanjiNotrNotr} $\angle BXC=\angle BRX+\angle RBX$. If we use the statement \ref{ObodObodKot} as well, we get:
 \begin{eqnarray*}
 \angle BXC &=& \angle BRX+\angle RBX= \angle BRC+\angle RBQ=\\
 &=& \angle BAC+\angle RPQ.
\end{eqnarray*}
 From this follows $\angle RPQ=\angle BXC-\angle BAC=60^0$. Similarly, $\angle RQP=60^0$, which means that $PQR$ is a right triangle.

\item \label{nalSkl50}
\res{Prove that the tangent points of an inscribed circle of a triangle $ABC$ divide its sides into segments of lengths $s-a$, $s-b$ and $s-c$ ($a$, $b$ and $c$ are the lengths of the sides, $s$ is the semi-perimeter of the triangle).}

A direct consequence of the great task \ref{velikaNaloga}.


 \item \res{The circles $k$, $l$ and $j$ are tangent to each other in non-collinear points $A$, $B$ and $C$. Prove that the inscribed circle of the triangle $ABC$ is perpendicular to the circles $k$, $l$ and $j$.}

First, prove that the inscribed circle of the triangle $ABC$ is also the circumscribed circle of the triangle determined by the centers of the circles $k$, $l$ and $j$. Use the previous task \ref{nalSkl50}.

Another solution to this task (with the help of inversion) is given in the example \ref{TriKroznInv}.

 \item
 \res{Let $ABCD$ be a quadrilateral with the center of its inscribed circle in the point $O$. With $E$ we denote the intersection of its diagonals $AC$ and $BD$ and with $F$, $M$ and $N$ the centers of the lines $OE$, $AD$ and $BC$. If $F$, $M$ and $N$ are collinear points, then $AC\perp BD$ or $AB\cong CD$. Prove it.}

It is enough to prove that in the case when $AC\perp BD$, it holds that $AB\cong CD$. We mark with $P$ and $Q$ the centers of the diagonal $AC$ and $BD$. The line segments $MP$ and $QN$ are the medians of the triangles $DAC$ and $DBC$ for the same base $DC$, therefore (from the statement \ref{srednjicaTrik}) $|MP|=\frac{1}{2}|DC|=|QN|$ and $MP\parallel DC\parallel QN$, which means that the quadrilateral $PMQN$ is a parallelogram and its diagonals intersect in a common center (from the statement \ref{paralelogram}) - we mark it with $S$. From the congruence of the triangles $OAP$ and $OCP$ (from the statement \textit{SSS} \ref{SSS}) it follows that $OP\perp AC$. Similarly, $OQ\perp BD$. If we mark with $G$ the intersection of the perpendiculars from the points $P$ and $Q$ on the diagonal $BD$ and $AC$, then the quadrilateral $QOPG$ is also a parallelogram. From the assumption that $AC\perp BD$ it follows that $G\neq E$. The line segment $FS$ is now the median of the triangle $GOE$ with the base $GE$, therefore $FS\parallel EG$ (from the statement \ref{srednjicaTrik}).
  Because the points $F$ and $S$ lie on the line $MN$, the line segments $FS$ and $MN$ coincide and it holds that $MN\parallel EG$. But according to the statement \ref{VisinskaTocka}, $G$ is the altitude point of the triangle $PEQ$ (according to the construction of the point $G$, $GP\perp EQ$ and $GQ\perp EP$), therefore $GE\perp PQ$. From $MN\parallel EG$ and $GE\perp PQ$ it follows that $MN\perp PQ$, which means that the parallelogram $PMQN$ is a rhombus (\ref{RombPravKvadr}). If we use the statement \ref{srednjicaTrik} once again, we get that $|AB|=2|PN|=2|NQ|=|CD|$.



\item \res{Draw the triangle $ABC$ (see the labels in the section \ref{odd3Stirik}):}

In all cases we use the labels as in the big task \ref{velikaNaloga}.

 (\textit{a}) \res{$a$, $\alpha$, $r$}

We use the fact that $\angle BSC=90^0+\frac{1}{2}\alpha$ and the statement \ref{ObodKotGMT} and first draw the triangle $BSC$.

 (\textit{b}) \res{$a$, $\alpha$, $r_a$}


We use the fact that $\angle BS_aC=90^0-\frac{1}{2}\alpha$ and the statement \ref{ObodKotGMT} and first draw the triangle $BS_aC$.

 (\textit{c}) \res{$a$, $v_b$, $v_c$}

First we draw the triangles $BB'C$ and $BC'C$ - but the points $B'$ and $C'$ lie on the circle with the radius $BC$.

(\textit{d}) \res{$\alpha$, $v_a$, $s$}

Let $E$ and $D$ be such points that $\mathcal{B}(ED,B,C,A)$, $DB\cong BA$ and $EC\cong CA$ hold. In this case, $BC=s$ and $\angle DAE=90^0-\frac{1}{2}\alpha$. This allows the construction of the triangle $DAE$.

 (\textit{e}) \res{$v_a$, $l_a$, $r$}

First, we plan the right-angled triangle $AA'E$ ($AA'=v_a$, $\angle A'=90^0$ and $AE=l_a$), then the center $S$ of the inscribed circle.

(\textit{f}) \res{$\alpha$, $v_a$, $l_a$}

Again, first plan the right-angled triangle $AA'E$, then use the fact that the line $AE$ is perpendicular to the angle $BAC$.

 (\textit{g}) \res{$\alpha$, $\beta$, $R$}

First, we plan the isosceles triangle $BOC$ ($OB=OC=R$, $\angle BOC=2\alpha$).

 (\textit{h}) \res{$c$, $r$, $R$}

Analogous to Example \ref{konstr_Rra}.

 (\textit{i}) \res{$a$, $v_b$, $R$}

First, we plan the isosceles triangle $BOC$ ($OB=OC=R$, $BC=a$), then the point $B'$ from the conditions $BB'=v_b$ and $\angle BB'C=90^0$.


 \item \res{Draw a circle $k$ so that:}

  \begin{enumerate}
    \item \res{it touches two given non-parallel lines $p$ and $q$, and the cord determined by the points of contact is perpendicular to the given distance $t$}

First, we plan the isosceles triangle determined by the points of contact and the intersection of the given lines.

    \item \res{its center is the given point $S$, and the given line $p$ determines on it a cord that is perpendicular to the given distance $t$}

We use the fact that the center of the cord is the orthogonal projection of the point $S$ onto the line $p$.

    \item \res{it passes through the given points $A$ and $B$, and its center lies on the given circle $l$}

 The center of the desired circle is one of the intersections of the circle $l$ with the perpendicular of the line $AB$.

    \item \res{it has the given radius $r$ and it touches two given circles $l$ and $j$}

 The center of the desired circle is the intersection of the circles $l'$ and $j'$, which are concentric with the circles  $l$ and $j$ with radii increased by $r$.

\item \res{touches the line $p$ at the point $P$ and goes through the given point $A$}

The center of the sought-after circle is the intersection of the perpendicular to the line $p$ at the point $P$ and the perpendicular bisector of the segment $PA$.

  \end{enumerate}

  \item \res{Draw a square $ABCD$, if the vertex $B$ and two points $E$ and $F$ lying on the sides $AD$ and $CD$ are given.}

See example \ref{KvadratKonstr4tocke}.

  \item \res{The line $CD$ and the points $A$ and $B$ ($A,B\notin CD$) are given. Draw on the line $CD$ such a point $M$, that $\angle AMC\cong2\angle BMD$ holds.}

Assume that $M$ is the sought-after point. Let $k$ be the circle with the center $B$, that touches the line $CD$, and $t$ be the other tangent from the point $M$ to this circle. Without loss of generality, assume that $\mathcal{B}(C,M,D)$. Then $\angle t,MD=2\angle BMD=\angle AMC$. This means that the line $t$ is reflected into the line $AM$ across the line $CD$ when reflected (see section \ref{odd6OsnZrc}). Therefore, the point $M$ is the intersection of the line $CD$ and the tangent from the point $A$ to the image of the circle $k$ when reflected across the line $CD$.


  \item \res{Draw a triangle $ABC$ with the data:}

In all cases, use the notation from the big exercise \ref{velikaNaloga}.

   (\textit{a}) \res{$v_a$, $t_a$, $\beta-\gamma$}

First, draw the right triangle $AA'A_1$ ($AA'=v_a$, $AA_1=t_a$ and $\angle A'=90^0$), then use the fact that $\angle A'AO=\beta-\gamma$ (statement \ref{TockaNbetagama}).

   (\textit{b}) \res{$v_a$, $l_a$, $R$}

First, draw the right triangle $AA'E$ ($AA'=v_a$, $AE=l_a$ and $\angle A'=90^0$), then use the fact that $\angle EAO=\angle A'AE$ (statement \ref{TockaNbetagama}).

   (\textit{c}) \res{$R$, $\beta-\gamma$, $t_a$}

First, draw the isosceles triangle $AON$ ($AO=ON=R$ and $\angle OAN=\frac{1}{2}\left( \beta-\gamma\right)$ (statement \ref{TockaNbetagama})), then the point $A_1$ on the line $ON$ from the condition $AA_1=t_a$.

   (\textit{d}) \res{$R$, $\beta-\gamma$, $v_a$}

We use the fact that $\angle A'AE\cong\angle EAO\frac{1}{2}\left( \beta-\gamma\right)$  (statement \ref{TockaNbetagama}). First, we draw the right angled triangle $AA'E$.

   (\textit{e}) \res{$R$, $\beta-\gamma$, $a$}
First, draw the cord $BC=a$ on the circle $k(O,R)$, then the point $N$ on the circle $k$ ($ON\perp BC$) and finally the point $A$ from the condition $\angle OAN\cong\angle ONA=\frac{1}{2}\left( \beta-\gamma\right)$ (statement \ref{TockaNbetagama}).


\item \res{On the hypotenuse of the rectangle $ABCD$ draw the point $E$, from which the sides $AD$ and $DC$ are seen under the same angle. When does the task have a solution?}

Assume that $E$ is the desired point. In this case $\angle CED\cong\angle DEC\cong\angle EDC$ (statement \ref{KotaVzporKraki}). By statement \ref{enakokraki} the triangle $DCE$ is an isosceles triangle with the base $DE$ or $CD\cong CE$. So the point $E$ can be obtained as the intersection of the line $AB$ and the circle $k(C,CD)$. The number of solutions depends on the number of intersections of this line and circle.

    \item \res{In a convex quadrilateral $ABCD$ it holds that $BC\cong CD$. Draw this quadrilateral, if the sides $AB$ and $AD$ and the internal angle at the vertices $B$ and $D$ are given.}

First, draw the triangle $ABD$ - use $\angle ADB-\angle ABD=\angle ADC-\angle ABC$
 (see task \ref{nalSklKonstrTrik}\textit{(u)} - at the end of section \ref{pogSKL}).

    \item \res{In a given circle $k$ draw the triangle $ABC$, if the vertex $A$, the line $p$, which is parallel to the altitude $AA'$, and the intersection $B_2$ of the altitude $BB'$ and this circle are given.}

Use the fact that $\angle BAC=90^0-\angle ABB_2=90^0-\frac{1}{2}\angle AOB_2$, where $O$ is the center of the circle $k$.


    \item \res{Draw the right angled triangle $ABC$, if its side $BC$ is congruent to the distance $a$, the hypotenuses of the sides $AB$ and $AC$ and the altitude of the internal angle $BAC$ go through the given points $M$, $N$ and $P$ consecutively.}

 The vertex $A$ is obtained as the intersection of two geometric loci, from which the distance $MP$ or $NP$ is seen under the angle $30^0$ (statement \ref{ObodKotGMT}).

\item \res{Draw a triangle $ABC$, if the following are given:}

In all cases we use the notation as in the big exercise \ref{velikaNaloga}.

  \begin{enumerate}
    \item \res{vertex $A$, center of the circumscribed circle $O$ and center of the inscribed circle $S$}

First, let's draw point $N$ as the intersection of the circle $k(O, OA)$ and the segment $AS$ (statement \ref{TockaN}), then vertices $B$ and $C$ as the intersection of the circles  $k(O, OA)$ and  $k_1(N, NS)$ (statement \ref{TockaN.NBNC}).

    \item \res{center of the circumscribed circle $O$, center of the inscribed circle $S$ and center of the excribed circle $S_a$}

We use the fact that point $N$ is the center of the segment $SS_a$ (big exercise \ref{velikaNaloga}).

    \item  \res{vertex $A$, center of the circumscribed circle $O$ and altitude point $V$}

We use izrek \ref{TockaV'}.

    \item res{vertices $B$ and $C$ and the perpendicular bisector of the internal angle $BAC$}

When reflecting over the perpendicular bisector $s$ of the internal angle $BAC$, the line $AB$ is mapped to the line $AC$. This means that the image $B'$ of point $B$ when reflected over the line $s$ lies on the line $AC$. So the vertex $A$ can be obtained as the intersection of the lines $s$ and $CB'$.

    \item   \res{vertex $A$, center of the circumscribed circle $O$ and the intersection $E$ of the side $BC$ with the perpendicular bisector of the internal angle $BAC$}

First, let's draw point $N$ as the intersection of the circle $k(O, OA)$ and the segment $AE$ (statement \ref{TockaN}), then vertices $B$ and $C$ as the intersection of the circle  $k(O, OA)$ with the rectangle of the line $ON$ in point $E$.

    \item \res{points $M$, $P$ and $N$, in which the altitude and median lines from the vertex $A$ and the perpendicular bisector of the internal angle $BAC$ intersect the circumscribed circle of the triangle}

First, we draw the circumscribed circle $k(O,R)$ of the triangle $ABC$, which is also the circumscribed circle of the triangle $MPN$. Because $AA'\parallel ON$ ($A'$ is the projection of the point $A$ on the side $BC$), we get the vertex $A$ as the intersection of the parallel line $ON$ through the point $M$. The center of the side $BC$ - the point $A_1$ is the intersection of the lines $ON$ and $AP$. In the end, we get the vertices $B$ and $C$ as the intersections of the perpendicular line $ON$ through the point $A_1$ with the circle $k$.

    \item \res{vertex $A$, center of the circumscribed circle $O$, point $N$, in which the simetrical of the internal angle $BAC$ intersects the circumscribed circle of the triangle, and the distance $a$, which is consistent with the side $BC$}

We use the fact that $BC\perp ON$.

  \end{enumerate}

  \item \res{Draw the triangle $ABC$ with the data:}

   (\textit{a}) \res{$a$, $b$, $\alpha=3\beta$}

Let $D$ and $E$ be such points on the side $BC$, that $\angle BAD\cong\angle DAE \cong\angle EAC$. First, $DAB$ is an isosceles triangle and $DA\cong DB$ (statement \ref{enakokraki}). $\angle CDA$ is the external angle of this triangle, so according to the statement \ref{zunanjiNotrNotr}
 $$\angle CDA=\angle DBA+\angle DAB =2\alpha=\angle CAD.$$
This means that also $CAD$ is an isosceles triangle or $CD\cong CA=b$ (statement \ref{enakokraki}). From this it follows that $DA\cong DB =a-b$. So we can first draw the isosceles triangle $CAD$ ($CA\cong CD=b$ and $DA=a-b$), then the point $B$ from the condition $\mathcal{B}(C,D,B)$ and $DB\cong DA$.

   (\textit{b}) \res{$t_a$, $t_c$, $v_b$}

Let $AA_1$ and $CC_1$ be the medians and $BB'$ the altitude of the triangle $ABC$. We also mark the point $T$ as the center and $D$ as the perpendicular projection of the point $A_1$ on the line $AC$. First, we draw the right triangle $AA_1D$ ($AA_1=t_a$, $\angle D=90^0$ and $A_1D=\frac{1}{2}v_b$), then the vertex $C$ from the condition $TC=\frac{2}{3}t_c$.

\item \res{Through the point $M$, which lies inside the given circle $k$, draw such a cord, that the difference of its segments (from the point $M$) is equal to the given distance $a$.}

First, we draw a concentric circle $k_1$ that goes through the point $M$, then its cord $MN=a$.

\item \res{Draw a triangle $ABC$, if you know:}

In all cases, we use the labels and properties from the big task \ref{velikaNaloga}.

 (\textit{a}) \res{$b-c$, $r$, $r_a$}

First, we draw the line $PP_a=b-c$ and its perpendicular, then the circles $k(S,r)$ and $k(S_a,r_a)$, finally the perpendiculars of the sides $AB$ and $AC$ of the triangle $ABC$, which are the common external tangents of these two circles. The intersection of the external tangents is the vertex $A$, and their intersection with the line $PP_a$ are the points $B$ and $C$.

 (\textit{b}) \res{$a$, $r$, $r_a$}

We use the fact $RR_a=a$.

 (\textit{c}) \res{$a$, $r_b+r_c$, $v_a$}

We use the fact $A_1M=\frac{1}{2}\left(r_b+r_c \right)$. First, we draw the cord $BC$, the point $M$ and the inscribed circle of the triangle $ABC$.

 (\textit{d}) \res{$b+c$, $r_b$, $r_c$}

We use the fact $R_bR_c=b+c$.

 (\textit{e}) \res{$R$, $r_b$, $r_c$}

We use first the fact $A_1M=\frac{1}{2}\left(r_b+r_c \right)$ and draw the circle $k(O,R)$, the points $M$, $N$ and $A_1$ and the side $BC$. Then we use the fact $MM'=\frac{1}{2}\left(r_b-r_c \right)$ and draw the tangent from the vertex $B$ to the circle $k_1(M,\frac{1}{2}\left(r_b-r_c \right))$ - the mentioned tangent is the perpendicular of the side $AB$.


 (\textit{f}) \res{$b$, $R$, $r+r_a$}

First, we draw the rectangular trapezoid $N'NMM'$, then the inscribed circle of the triangle $ABC$.

(\textit{g}) \res{$a$, $v_a$, $r_a-r$}

We use the fact $A_1N=\frac{1}{2}\left(r_a-r \right)$ and draw first the cord $BC$, then the point $N$ and the inscribed circle of the triangle $ABC$.

 (\textit{h}) \res{$\alpha$, $r$, $b+c$}

First, we draw the rectangular triangle $ARS$, then the point $N$ from the condition $AN'=\frac{1}{2}\left(b+c \right)$.

\item \res{Let there be: a circle $k$, its diameter $AB$ and a point $M\notin k$. With only a straightedge, construct a rectangle from point $M$ to line $AB$.}

Let $D$ and $E$ be the intersections of lines $AM$ and $BM$ with circle $k$, and let $F$ be the intersection of lines $AE$ and $BF$. We use the fact that $M$ is an altitude of triangle $ABF$.

\item \res{Let there be: a square $ABCD$ and such points $M$ and $N$ on sides $BC$ and $CD$, that $\angle MAN=45^0$.
 With only a straightedge, construct a rectangle from point $A$ to line $MN$.}

Let $E$ and $F$ be the intersections of the diagonal $BD$ of square $ABCD$ with lines $AM$ and $AN$. We first prove that $MF$ and $NE$ are altitudes of triangle $AMN$.



\end{enumerate}



%Solutions - Vectors
%________________________________________________________________________________

\poglavje{Vectors}


\begin{enumerate}

  %The sum and difference of vectors
    %_____________________________________

  \item \res{Draw any vectors $\overrightarrow{a}$, $\overrightarrow{b}$, $\overrightarrow{c}$ and $\overrightarrow{d}$ so that their sum is equal to:}

  (\textit{a}) \res{one of these four vectors}

The relation $\overrightarrow{a}+\overrightarrow{b}+\overrightarrow{c}+
\overrightarrow{d}=\overrightarrow{d}$ is equivalent to $\overrightarrow{a}+\overrightarrow{b}+\overrightarrow{c}=\overrightarrow{0}$. So for $\overrightarrow{d}$ we can choose any vector, and for the others we can choose $\overrightarrow{a}=\overrightarrow{PQ}$, $\overrightarrow{b}=\overrightarrow{QR}$ and $\overrightarrow{c}=\overrightarrow{RP}$, where $P$, $Q$ and $R$ are any points.

  (\textit{b}) \res{the difference of two of these four vectors}

The relation $\overrightarrow{a}+\overrightarrow{b}+\overrightarrow{c}+
\overrightarrow{d}=\overrightarrow{c}-\overrightarrow{d}$ is equivalent to
 $\overrightarrow{a}+\overrightarrow{b}+
=-2\overrightarrow{d}$. So vectors  $\overrightarrow{a}$, $\overrightarrow{b}$ and $\overrightarrow{c}$ can be chosen arbitrarily, and for vector $\overrightarrow{d}$ we have $\overrightarrow{d}=-\frac{1}{2}\left( \overrightarrow{a}+\overrightarrow{b} \right)$.


   \item \res{Let $ABCDE$ be a pentagon in some plane. Prove that in this plane there exists
a pentagon with sides that determine the same vectors as the
 diagonals of pentagon $ABCDE$ do.}

\end{enumerate}

We use the relation:
$\overrightarrow{AC}+\overrightarrow{CE}+\overrightarrow{EB}+\overrightarrow{BD}
+\overrightarrow{DA}=\overrightarrow{0}$



  \item \res{Let $A$, $B$, $C$ and $D$ be arbitrary points in the plane. Does the following hold in general:}

  (\textit{a}) \res{$\overrightarrow{AB}+\overrightarrow{BD}=\overrightarrow{AD}+\overrightarrow{BC}$?}

The given relation is equivalent to $\overrightarrow{AD}=\overrightarrow{AD}+\overrightarrow{BC}$ or $\overrightarrow{BC}=\overrightarrow{0}$. Therefore, the relation only holds if $B=C$, but it does not hold in general.

  (\textit{b}) \res{$\overrightarrow{AB}=\overrightarrow{DC}\hspace*{1mm}\Rightarrow \hspace*{1mm} \overrightarrow{AC}+\overrightarrow{BD}=2\overrightarrow{BC}$?}

The relation on the left side of the implication is equivalent to $\overrightarrow{AC}+\overrightarrow{CB}=\overrightarrow{DB}+\overrightarrow{BC}$ or
$\overrightarrow{AB}=\overrightarrow{DC}$. Therefore, the implication holds in general - for arbitrary points $A$, $B$, $C$ and $D$.

  \item \label{nalVekt4}
\res{Given is the line segment $AB$. Using only a straightedge (construction in affine geometry) construct a point $C$, so that:}

    (\textit{a}) \res{$\overrightarrow{AC}=-\overrightarrow{AB}$}

We construct the parallelogram $ABPQ$ and $APQC$.

   (\textit{b}) \res{$\overrightarrow{AC}=5\overrightarrow{AB}$}

We construct the parallelograms $ABPQ$, $BQPC_1$, $BQC_1C_2$, $BQC_2C_3$ and $BQC_3C$.

   (\textit{c}) \res{$\overrightarrow{AC}=-3\overrightarrow{AB}$}

We construct the parallelograms $ABPQ$, $BQPC_1$, $BQC_1C_2$, then we use (\textit{a}).


   \item \res{Let $ABCD$ be a quadrilateral and $O$ an arbitrary point in the plane of this quadrilateral. Express the vectors of the sides and
the diagonals of this quadrilateral with the vectors $\overrightarrow{a}=\overrightarrow{OA}$, $\overrightarrow{b}=\overrightarrow{OB}$, $\overrightarrow{c}=\overrightarrow{OC}$ and $\overrightarrow{d}=\overrightarrow{OD}$.}

According to the statement \ref{vektOdsev} it is:
\begin{eqnarray*}
 \overrightarrow{AB}&=&\overrightarrow{OB}-\overrightarrow{OA}=
\overrightarrow{b}-\overrightarrow{a}\\
  \overrightarrow{BC}&=&\overrightarrow{OC}-\overrightarrow{OB}=
\overrightarrow{c}-\overrightarrow{b}\\
  \overrightarrow{CD}&=&\overrightarrow{OD}-\overrightarrow{OC}=
\overrightarrow{d}-\overrightarrow{c}\\
  \overrightarrow{DA}&=&\overrightarrow{OA}-\overrightarrow{OD}=
\overrightarrow{a}-\overrightarrow{d}\\
  \overrightarrow{AC}&=&\overrightarrow{OC}-\overrightarrow{OA}=
\overrightarrow{c}-\overrightarrow{a}\\
  \overrightarrow{BD}&=&\overrightarrow{OD}-\overrightarrow{OB}=
\overrightarrow{d}-\overrightarrow{b}
 \end{eqnarray*}

\item \res{Let $ABCD$ be a quadrilateral and $O$ an arbitrary point in the plane of this quadrilateral. Is the equivalence that the quadrilateral $ABCD$ is a parallelogram exactly when $\overrightarrow{OA}+\overrightarrow{OC}=
    \overrightarrow{OB}+\overrightarrow{OD}$?}

The given relation is equivalent to $\overrightarrow{OC}+\overrightarrow{DO}=
    \overrightarrow{AO}+\overrightarrow{OB}$ or $\overrightarrow{DC}=
    \overrightarrow{AB}$. According to the statement \ref{vektParalelogram} the last relation is equivalent to the statement that $ABCD$ is a parallelogram.

 \item \res{Let $ABCD$ be a parallelogram, $S$
the intersection of its diagonals and $M$ an arbitrary point in the plane of this parallelogram. Prove that it is true:
        $$\overrightarrow{MS} = \frac{1}{4}\cdot
        \left( \overrightarrow{MA}+\overrightarrow{MB}
         +\overrightarrow{MC} + \overrightarrow{MD} \right).$$}

Since $S$ is the center of the lines $AC$ and $BD$ (statement \ref{paralelogram}), according to the statement \ref{vektSredOSOAOB}:
\begin{eqnarray*}
 \overrightarrow{MA}+\overrightarrow{MB}
         +\overrightarrow{MC} + \overrightarrow{MD}&=&
\left( \overrightarrow{MA}+\overrightarrow{MB} \right)
         +\left( \overrightarrow{MC} + \overrightarrow{MD} \right)=\\
2\cdot\overrightarrow{MS}
         +2\cdot\overrightarrow{MS}=4\cdot\overrightarrow{MS}.
 \end{eqnarray*}

\item \res{Let $ABB_1A_2$,
$BCC_1B_2$ and $CAA_1C_2$ be parallelograms that are drawn on the sides of triangle $ABC$. Prove that:
$$\overrightarrow{A_1A_2}+\overrightarrow{B_1B_2}+
\overrightarrow{C_1C_2}=\overrightarrow{0}.$$}

By the \ref{vektParalelogram} theorem:
\begin{eqnarray*}
 \overrightarrow{A_1A_2}+\overrightarrow{B_1B_2}+
\overrightarrow{C_1C_2}
&=&
\overrightarrow{A_1A_2}+\overrightarrow{B_1B_2}+
\overrightarrow{C_1C_2}+\overrightarrow{0}=\\
&=&
\overrightarrow{A_1A_2}+\overrightarrow{B_1B_2}+
\overrightarrow{C_1C_2}+\overrightarrow{AB}+\overrightarrow{BC}+\overrightarrow{CA}=\\
&=&
\overrightarrow{A_1A_2}+\overrightarrow{B_1B_2}+
\overrightarrow{C_1C_2}
+\overrightarrow{A_2B_1}+\overrightarrow{B_2C_1}+\overrightarrow{C_2A_1}=\\
&=&
\overrightarrow{A_1A_2}+\overrightarrow{A_2B_1}+\overrightarrow{B_1B_2}
+\overrightarrow{B_2C_1}+
\overrightarrow{C_1C_2}
+\overrightarrow{C_2A_1}=\\
&=&
\overrightarrow{A_1A_1}=\overrightarrow{0}.
 \end{eqnarray*}


  \item \res{Perpendicular lines $p$ and $q$, that intersect in point $M$, intersect circle $k$ with center $O$ in
points $A$, $B$, $C$ and $D$. Prove that:
$$\overrightarrow{OA}+ \overrightarrow{OB} + \overrightarrow{OC} + \overrightarrow{OD} = 2\overrightarrow{OM}.$$}

Let's assume that the line $p$ intersects the circle $k$ in points $A$ and $B$, and the line $q$ intersects in points $C$ and $D$. We denote with $P$ and $Q$ the midpoints of the segments $AB$ and $CD$.
From the congruence of the triangles $MPA$ and $MPB$ (the \textit{SSS} theorem \ref{SSS}) it follows that $\angle MPA\cong \angle MPB=90^0$ or $\angle MPO=90^0$. Similarly,  $\angle MQO=90^0$, which means (because $p\perp q$ as well), that $MPOQ$ is a rectangle (also a parallelogram). If we use the theorem \ref{vektParalelogram} and \ref{vektSredOSOAOB}, we get:
\begin{eqnarray*}
 \overrightarrow{OA}+ \overrightarrow{OB} + \overrightarrow{OC} + \overrightarrow{OD}
 = 2\cdot\overrightarrow{OP}+ 2\cdot\overrightarrow{OQ} = 2\cdot\left( \overrightarrow{OP}+ \overrightarrow{OQ} \right)=
2\overrightarrow{OM}.
 \end{eqnarray*}


\item \res{Naj bodo $A$, $B$, $C$ in $D$ poljubne točke v neki ravnini. Ali lahko vseh šest daljic, ki jih določajo te točke, orientiramo tako, da je vsota ustreznih šestih vektorjev enaka vektorju nič?}

"

\item \res{Are there any points $A$, $B$, $C$ and $D$ in some plane, such that all six lines determined by these points can be oriented so that the sum of the corresponding six vectors is equal to the vector of zero?}

Let $\overrightarrow{DA}=\overrightarrow{a}$, $\overrightarrow{DB}=\overrightarrow{b}$ and
$\overrightarrow{DC}=\overrightarrow{c}$.
In this case,
$\overrightarrow{AB}=\overrightarrow{b}-\overrightarrow{a}$,
$\overrightarrow{AC}=\overrightarrow{c}-\overrightarrow{a}$ and
$\overrightarrow{BC}=\overrightarrow{c}-\overrightarrow{b}$.
It is necessary to determine whether, for any vectors $\overrightarrow{a}$, $\overrightarrow{b}$ and $\overrightarrow{c}$, we can choose such an orientation of the vectors:
$\pm\overrightarrow{a}$, $\pm\overrightarrow{b}$, $\pm\overrightarrow{c}$,
$\pm\left( \overrightarrow{b}-\overrightarrow{a} \right)$,
$\pm\left( \overrightarrow{c}-\overrightarrow{a} \right)$ and
$\pm\left( \overrightarrow{c}-\overrightarrow{b} \right)$
 (for each $+$ or $-$), that their sum is equal to $\overrightarrow{0}$. If we add the first three vectors (in all different orientations), we get possible sums:\\
$\pm\left( \overrightarrow{a}+\overrightarrow{b}+\overrightarrow{c} \right)$,
$\pm\left( \overrightarrow{a}+\overrightarrow{b}-\overrightarrow{c} \right)$,
$\pm\left( \overrightarrow{a}-\overrightarrow{b}+\overrightarrow{c} \right)$ and
$\pm\left( -\overrightarrow{a}+\overrightarrow{b}+\overrightarrow{c} \right)$,\\
if we add the second triple, we get:\\
$\pm 2\left( \overrightarrow{b}-\overrightarrow{a} \right)$,
$\pm 2\left( \overrightarrow{c}-\overrightarrow{a} \right)$,
$\pm 2\left( \overrightarrow{c}-\overrightarrow{b} \right)$ and
$\overrightarrow{0}$.\\
It is clear that if $\overrightarrow{a}$, $\overrightarrow{b}$ are arbitrary vectors or $A$, $B$, $C$ and $D$ are arbitrary points, we do not get as a final sum of the vector $\overrightarrow{0}$.

The sum of zero could only be obtained in the case when, for example, $\overrightarrow{c}=\overrightarrow{a}+\overrightarrow{b}$ or when $ABCD$ is a parallelogram.
In this case, we can choose:
$\overrightarrow{AB}+\overrightarrow{BC}+\overrightarrow{CD}+\overrightarrow{DA}
+\overrightarrow{BD}+\overrightarrow{CA}=\overrightarrow{0}$.



\item \res{Let $A_1$, $B_1$ and $C_1$ be the midpoints of sides $BC$, $AC$ and $AB$ of triangle $ABC$ and $M$ an arbitrary point. Prove:}

    (\textit{a}) \res{$\overrightarrow{AA_1}+\overrightarrow{BB_1}+
\overrightarrow{CC_1}=\overrightarrow{0}$}

We use the theorem \ref{vektSredOSOAOB}.

   (\textit{b}) \res{$\overrightarrow{MA}+\overrightarrow{MB}+\overrightarrow{MC}=
   \overrightarrow{MA_1}+\overrightarrow{MB_1}+\overrightarrow{MC_1}$}

We use the theorem \ref{vektSredOSOAOB}.

   (\textit{c}) \res{There exists a triangle $PQR$, such that for its vertices it holds
$\overrightarrow{PQ}=\overrightarrow{CC_1}$, $\overrightarrow{PR}=\overrightarrow{BB_1}$ and $\overrightarrow{RQ}=\overrightarrow{AA_1}$}

A direct consequence of (\textit{a}).


 \item \res{Let $M$, $N$, $P$, $Q$, $R$ and  $S$ be the centers of the sides of an arbitrary hexagon, in order.
Prove that it holds:
$$\overrightarrow{MN}+\overrightarrow{PQ}+
\overrightarrow{RS}=\overrightarrow{0}.$$}

Let $M$, $N$, $P$, $Q$, $R$ and  $S$ be the centers of the sides $AB$, $BC$, $CD$, $DE$, $EF$ and $FA$ of the hexagon $ABCDEF$. The lines $MN$, $PQ$ and $RS$ are the medians of the triangles $ABC$, $CDE$ and $EFA$ for the sides $AC$, $CE$ and $EA$, so by the theorem \ref{srednjicaTrikVekt}:
 \begin{eqnarray*}
\overrightarrow{MN}+\overrightarrow{PQ}+
\overrightarrow{RS}
&=&
 \frac{1}{2}\overrightarrow{AC}+\frac{1}{2}\overrightarrow{CE}+
\frac{1}{2}\overrightarrow{EA}=\\
&=&
\frac{1}{2} \left( \overrightarrow{AC}+\overrightarrow{CE}+
\overrightarrow{EA} \right)=
\overrightarrow{0}.
 \end{eqnarray*}

  %Linearna kombinacija vektorjev
    %_____________________________________

  \item \res{Let $ABCDEF$ be a convex hexagon, for which $AB\parallel DE$, and let $K$ and $L$ be points, which are the centers of the lines, determined by the centers of the remaining pairs of opposite sides. Prove that $K=L$ if and only if $AB\cong DE$.}

We prove that $\overrightarrow{KL}=\frac{1}{4}\left( \overrightarrow{AB}+\overrightarrow{DE} \right)$. See the example \ref{vektPetkoinikZgled}.


 \item \res{Let $P$ and $Q$ be points on the sides $BC$ and $CD$ of the parallelogram $ABCD$, such that $BP:PC=2:3$ and
$CQ:QD=2:5$. The point $X$ is the intersection of the lines $AP$ and $BQ$. Calculate the ratios, in which the point $X$ divides
the line $AP$ and $BQ$.}

Let $\overrightarrow{u}=\overrightarrow{AB}$ and $\overrightarrow{v}=\overrightarrow{AD}$. Then
$\overrightarrow{AP}=\overrightarrow{AB}+\overrightarrow{BP}
 =\overrightarrow{u}+\frac{2}{5}\overrightarrow{v}$ and
$\overrightarrow{AQ}=\overrightarrow{AD}+\overrightarrow{DQ}
 =\frac{5}{7}\overrightarrow{u}+\overrightarrow{v}$. From this it follows that
 $\overrightarrow{BQ}=\overrightarrow{AQ}-\overrightarrow{AB}
=-\frac{2}{7}\overrightarrow{u}+\overrightarrow{v}$.
 Because $A$, $X$ and $P$ are collinear points, for some $\lambda \in \mathbb{R}$ it holds that $\overrightarrow{AX}=\lambda \overrightarrow{AP}$ (statement \ref{vektKriterijKolin}). Similarly, for some $\mu \in \mathbb{R}$ it holds that $\overrightarrow{BX}=\mu \overrightarrow{BQ}$.
Therefore:
\begin{eqnarray*}
\overrightarrow{AX}
&=& \lambda \overrightarrow{AP}=
 \lambda\left( \overrightarrow{u}+\frac{2}{5}\overrightarrow{v} \right)=
 \lambda \overrightarrow{u}+\frac{2}{5}\lambda\overrightarrow{v}
\\
\overrightarrow{AX}
&=& \overrightarrow{AB}+\overrightarrow{BX}=
\overrightarrow{AB}+\mu \overrightarrow{BQ}=
u+ \mu\left( -\frac{2}{7}\overrightarrow{u}+\overrightarrow{v} \right)=
\left(1-\frac{2}{7}\mu\right)\overrightarrow{u}+\mu\overrightarrow{v}.
 \end{eqnarray*}
Because $\overrightarrow{u}$ and $\overrightarrow{v}$ are non-collinear vectors, by statement \ref{vektLinKomb1Razcep}:
\begin{eqnarray*}
\lambda &=& 1-\frac{2}{7}\mu\\
 \frac{2}{5}\lambda &=& \mu.
 \end{eqnarray*}
If we solve the system for $\lambda$ and $\mu$, we get $\lambda=\frac{35}{39}$ and $\mu=\frac{14}{39}$. From this it follows that $\overrightarrow{AX}=\lambda \overrightarrow{AP}=\frac{35}{39}\overrightarrow{AP}$ and $\overrightarrow{BX}=\mu \overrightarrow{BQ}=\frac{14}{39}\overrightarrow{BQ}$, so $AX:XP=35:4$ and $BX:XQ=14:25$.

\item \res{Naj bodo $A$, $B$, $C$ in $D$ poljubne točke neke ravnine. Točka $E$ je središče daljice $AB$, $F$ in
$G$ takšni točki, da velja $\overrightarrow{EF} = \overrightarrow{BC}$ in $\overrightarrow{EG} = \overrightarrow{AD}$,  ter $S$ središče daljice $CD$. Dokaži, da so $G$, $S$ in $F$
kolinearne točke.}

Let $A$, $B$, $C$ and $D$ be any points on a plane. Point $E$ is the center of the line segment $AB$, $F$ and
$G$ are such points that $\overrightarrow{EF} = \overrightarrow{BC}$ and $\overrightarrow{EG} = \overrightarrow{AD}$, and $S$ is the center of the line segment $CD$. Prove that $G$, $S$ and $F$
are collinear points.

We prove that $S$ is also the center of the line $FG$.
Use the statement \ref{vektSredOSOAOB}. Another possibility is to first prove that the quadrilateral $FCGD$ is a parallelogram.

\item \res{Let $K$ and $L$ be such points on the side $AD$ and the diagonal $AC$ of the parallelogram $ABCD$, that $\frac{\overrightarrow{AK}}{\overrightarrow{KD}}=\frac{1}{3}$ and
    $\frac{\overrightarrow{AL}}{\overrightarrow{LC}}=\frac{1}{4}$. Prove that $K$, $L$ and $B$ are collinear points.}

By assumption, $\overrightarrow{AK}=\frac{1}{4}\overrightarrow{AD}$ and $\overrightarrow{AL}=\frac{1}{5}\overrightarrow{AC}$. Express the vector $\overrightarrow{AL}$ as a linear combination of the vectors $\overrightarrow{AB}$ and $\overrightarrow{AK}$:
 \begin{eqnarray*}
 \overrightarrow{AL}
&=&
\frac{1}{5}\overrightarrow{AC}
 = \frac{1}{5}\left( \overrightarrow{AB} + \overrightarrow{AD} \right)=\\
&=&
 \frac{1}{5}\left( \overrightarrow{AB} + 4\overrightarrow{AK} \right)=\\
&=&
 \frac{1}{5} \overrightarrow{AB} +\frac{4}{5}\overrightarrow{AK}.
 \end{eqnarray*}
 Because $\frac{1}{5}+\frac{4}{5}=1$, by statement \ref{vektParamPremica}  $K$, $L$ and $B$ are collinear points.

\item \res{Let $X_n$ and $Y_n$ ($n\in \mathbb{N}$) be such points on the sides $AB$ and $AC$ of the triangle $ABC$, that $\overrightarrow{AX_n}=\frac{1}{n+1}\cdot \overrightarrow{AB}$ and
    $\overrightarrow{AY_n}=\frac{1}{n}\cdot \overrightarrow{AC}$. Prove that there exists a point that lies on all lines
$X_nY_n$  ($n\in \mathbb{N}$).}

Let $D$ be the fourth vertex of the parallelogram $BCAD$. We prove that the point $D$ lies on all lines $X_nY_n$ ($n \in \mathbb{N}$) or that the points $D$, $X_n$ and $Y_n$ are collinear for each $n \in \mathbb{N}$. According to the assumption:
 \begin{eqnarray*}
 \overrightarrow{AX_n}
&=&
\frac{1}{n+1}\overrightarrow{AB}
 = \frac{1}{n+1}\left( \overrightarrow{AD} + \overrightarrow{AC} \right)=\\
&=&
 \frac{1}{n+1}\left( \overrightarrow{AD} + n\overrightarrow{AY_n} \right)=\\
&=&
 \frac{1}{n+1} \overrightarrow{AD} +\frac{n}{n+1}\overrightarrow{AY_n}.
 \end{eqnarray*}
 Because $\frac{1}{n+1}+\frac{n}{n+1}=1$, according to the theorem \ref{vektParamPremica}  $D$, $X_n$ and $Y_n$ are collinear points.


 \item \res{Let $M$, $N$, $P$ and $Q$ be the centers of the sides $AB$, $BC$, $CD$ and $DA$ of the quadrilateral
     $ABCD$. Is the equivalence that the quadrilateral $ABCD$ is a parallelogram exactly when:}

    (\textit{a}) \res{$2\overrightarrow{MP}=\overrightarrow{BC}+\overrightarrow{AD}$ and
    $2\overrightarrow{NQ}=\overrightarrow{BA}+\overrightarrow{CD}$?}\\
   (\textit{b}) \res{$2\overrightarrow{MP}+2\overrightarrow{NQ}=
   \overrightarrow{AB}+\overrightarrow{BC}+\overrightarrow{CD}+\overrightarrow{DA}$?}

The equivalence does not hold, because the statements (\textit{a}) and (\textit{b}) according to the theorem \ref{vektSestSplosno} hold generally for any quadrilateral $ABCD$.

 \item \res{Let $E$, $F$ and $G$ be the centers of the sides $AB$, $BC$ and $CD$ of the parallelogram $ABCD $, the line $BG$ and the line $DE$ intersect the line $AF$ in points $N$ and $M$. Express the vector $\overrightarrow{AF}$, $\overrightarrow{AM}$ and $\overrightarrow{AN}$ as a linear combination of the vectors $\overrightarrow{AB}$ and $\overrightarrow{AD}$. Prove that the points $M$ and $N$ divide the distance $AF$ in the ratio $2:2:1$.}

We express the vector $\overrightarrow{AN}$ in two ways: $\overrightarrow{AN}=\lambda \overrightarrow{AF}$ and $\overrightarrow{AN}=\overrightarrow{AB}+ \mu\overrightarrow{BG}$, then we calculate $\lambda$. Similarly, for the vector $\overrightarrow{AM}$.

\item \res{Points $K$, $L$, $M$ and $N$ lie on sides $AB$, $BC$, $CD$ and $DA$
of quadrilateral $ABCD$. If the quadrilateral $KLMN$ is a parallelogram and it holds
$$\frac{\overrightarrow{AK}}{\overrightarrow{KB}}=
\frac{\overrightarrow{BL}}{\overrightarrow{LC}}
=\frac{\overrightarrow{CM}}{\overrightarrow{MD}}=
\frac{\overrightarrow{DN}}{\overrightarrow{NA}}=\lambda$$ for some $\lambda\neq\pm 1$,
 then the quadrilateral $ABCD$
is a parallelogram as well. Prove it.}

Since $KLMN$ is a parallelogram, it holds $\overrightarrow{KL}=\overrightarrow{NM}$  (statement \ref{vektParalelogram}).

From the given conditions it follows that $\overrightarrow{AK}=\lambda\overrightarrow{KB}$, from this we get $\overrightarrow{AB}+\overrightarrow{BK}=\lambda\overrightarrow{KB}$ or:
 \begin{eqnarray} \label{nalVekt23a}
\overrightarrow{KB}=\frac{1}{1+\lambda}\cdot \overrightarrow{AB}.
\end{eqnarray}
Similarly, from $\overrightarrow{BL}=\lambda\overrightarrow{LC}$ it follows that $\overrightarrow{BL}=\lambda \left( \overrightarrow{LB}+\overrightarrow{BC} \right)$ or:
 \begin{eqnarray} \label{nalVekt23b}
\overrightarrow{BL}=\frac{\lambda}{1+\lambda}\cdot \overrightarrow{BC}.
\end{eqnarray}
From \ref{nalVekt23a} and \ref{nalVekt23b} we get:

\begin{eqnarray} \label{nalVekt23c}
\overrightarrow{KL}=\overrightarrow{KB}+\overrightarrow{BL}=
\frac{1}{1+\lambda}\cdot \overrightarrow{AB}+\frac{\lambda}{1+\lambda}\cdot \overrightarrow{BC}.
\end{eqnarray}
Analogously:
\begin{eqnarray} \label{nalVekt23d}
\overrightarrow{MN}=\overrightarrow{MD}+\overrightarrow{DN}=
\frac{1}{1+\lambda}\cdot \overrightarrow{CD}+\frac{\lambda}{1+\lambda}\cdot \overrightarrow{DA}.
\end{eqnarray}
From $\overrightarrow{KL}=\overrightarrow{NM}$ (or $\overrightarrow{KL}+\overrightarrow{MN}=\overrightarrow{0}$), \ref{nalVekt23c} and \ref{nalVekt23d} we get:
\begin{eqnarray*}
\frac{1}{1+\lambda}\cdot \overrightarrow{AB}+\frac{\lambda}{1+\lambda}\cdot \overrightarrow{BC}+
\frac{1}{1+\lambda}\cdot \overrightarrow{CD}+\frac{\lambda}{1+\lambda}\cdot \overrightarrow{DA}=\overrightarrow{0}
\end{eqnarray*}
or
\begin{eqnarray*}
 \overrightarrow{AB}+\lambda\cdot \overrightarrow{BC}+
 \overrightarrow{CD}+\lambda\cdot \overrightarrow{DA}=\overrightarrow{0}.
\end{eqnarray*}
This implies:
\begin{eqnarray*}
\overrightarrow{0}
&=&
\overrightarrow{AB}+\lambda\cdot \overrightarrow{BC}+
 \overrightarrow{CD}+\lambda\cdot \overrightarrow{DA}=\\
 &=&
\overrightarrow{AB}+\overrightarrow{BC}+\left(\lambda-1\right)\cdot \overrightarrow{BC}+
 \overrightarrow{CD}+\overrightarrow{DA}+\left(\lambda-1\right)\cdot \overrightarrow{DA}=\\
 &=&
\overrightarrow{AC}+\left(\lambda-1\right)\cdot \overrightarrow{BC}+
 \overrightarrow{CA}+\left(\lambda-1\right)\cdot \overrightarrow{DA}=\\
 &=&
\left(\lambda-1\right)\cdot \overrightarrow{BC}+
\left(\lambda-1\right)\cdot \overrightarrow{DA}=\\
 &=&
\left(\lambda-1\right)\cdot \left(\overrightarrow{BC}+
 \overrightarrow{DA}\right).
\end{eqnarray*}
Since by assumption $\lambda\neq 1$, it follows that $\overrightarrow{BC}+
 \overrightarrow{DA}=\overrightarrow{0}$ or $\overrightarrow{BC}=
 \overrightarrow{AD}$, which means (from \ref{vektParalelogram}), that $ABCD$ is a parallelogram.

\item \res{Let $M$ be the center of the side $DE$ of a regular hexagon $ABCDEF$. The point
$N$ is the center of the line $AM$, and the point $P$ is the center of the side $BC$. Express $\overrightarrow{NP}$ as a linear combination of the vectors
$\overrightarrow{AB}$ and $\overrightarrow{AF}$.}

Let $S$ be the center of a regular hexagon $ABCDEF$, $\overrightarrow{u}=\overrightarrow{AB}$ and $\overrightarrow{v}=\overrightarrow{AF}$.
Then:
\begin{eqnarray*}
 \overrightarrow{NP}
 &=&
 \overrightarrow{AP}-\overrightarrow{AN}=
\overrightarrow{AB}+\overrightarrow{BP}-\frac{1}{2}\overrightarrow{AM}=\\
 &=&
 \overrightarrow{u}+\frac{1}{2}\overrightarrow{BC}
-\frac{1}{2}\left( \overrightarrow{AD}+\overrightarrow{DM} \right)= \\
 &=&
 \overrightarrow{u}+\frac{1}{2}\overrightarrow{AS}
-\frac{1}{2}\left( 2\overrightarrow{AS}-\frac{1}{2}\overrightarrow{u} \right)= \\
 &=&
 \frac{5}{4}\overrightarrow{u}-\frac{1}{2}\overrightarrow{AS}= \\
 &=&
 \frac{5}{4}\overrightarrow{u}-
\frac{1}{2}\left( \overrightarrow{u}+\overrightarrow{v} \right)= \\
 &=&
 \frac{3}{4}\overrightarrow{u}-
\frac{1}{2}\overrightarrow{v}.
\end{eqnarray*}
The other way is to use the direct statement \ref{vektSestSplosno} or.:
\begin{eqnarray*}
\overrightarrow{NP}
&=&
\frac{1}{2}\left( \overrightarrow{AB}+\overrightarrow{MC} \right)=\\
&=&
\frac{1}{2}\left( \overrightarrow{u}+\overrightarrow{MD}+\overrightarrow{DC} \right)=\\
&=&
\frac{1}{2}\left( \overrightarrow{u}+\frac{1}{2}\overrightarrow{u}-\overrightarrow{v} \right)=\\
 &=&
 \frac{3}{4}\overrightarrow{u}-
\frac{1}{2}\overrightarrow{v}.
\end{eqnarray*}

  %Dolžina vektorja
    %_____________________________________

  \item \res{Prove that for any points $A$, $B$ and $C$ it holds:}

    (\textit{a}) \res{$|\overrightarrow{AC}|\leq|\overrightarrow{AB}|+|\overrightarrow{BC}|$; \hspace*{6mm}}
   (\textit{b}) \res{$|\overrightarrow{AC}|\geq|\overrightarrow{AB}|-|\overrightarrow{BC}|$\\
  Under which conditions is the equality true?}

A direct consequence of the triangle inequality \ref{neenaktrik}. The equality is true exactly when $\mathcal{B}(A,B,C)$.

  \item \res{ Let $M$ and $N$ be points that lie on the lines $AD$ and $BC$, respectively, so that $\frac{\overrightarrow{AM}}{\overrightarrow{MD}}\cdot \frac{\overrightarrow{CN}}{\overrightarrow{NB}}=1$. Prove that:
      $$|MN|\leq\max\{|AB|, |CD|\}.$$}

From the given condition it follows that $\frac{\overrightarrow{AM}}{\overrightarrow{MD}}= \frac{\overrightarrow{NB}}{\overrightarrow{CN}}=\lambda$ or $\overrightarrow{AM}=\lambda\overrightarrow{MD}$  and $\overrightarrow{NB}=\lambda\overrightarrow{CN}$.
Since $\mathcal{B}(A,M,D)$ and $\mathcal{B}(B,N,C)$, $\lambda\geq 0$. Furthermore:
$\overrightarrow{MN}=\overrightarrow{MA}+\overrightarrow{AB}+\overrightarrow{BN}$ and $\overrightarrow{MN}=\overrightarrow{MD}+\overrightarrow{DC}+\overrightarrow{CN}$.
If we multiply the second relation by $\lambda$ and add the first, we get:
 $(1+\lambda)\overrightarrow{MN}=\overrightarrow{AB}+\lambda \overrightarrow{DC}$ or:
$$\overrightarrow{MN}=\frac{1}{1+\lambda}\overrightarrow{AB}+\frac{\lambda}{1+\lambda} \overrightarrow{DC}.$$
Without loss of generality, we assume that $|AB|\geq|CD|$. If we use the triangle inequality for vectors \ref{neenakTrikVekt}, we get
 \begin{eqnarray*}
 |\overrightarrow{MN}|
 &=&
 |\frac{1}{1+\lambda}\overrightarrow{AB}+\frac{\lambda}{1+\lambda} \overrightarrow{DC}|\leq\\
 &\leq&
 |\frac{1}{1+\lambda}\overrightarrow{AB}|+|\frac{\lambda}{1+\lambda} \overrightarrow{DC}|=\\
 &=&
\frac{1}{1+\lambda}|\overrightarrow{AB}|+\frac{\lambda}{1+\lambda} |\overrightarrow{DC}|\leq\\
 &\leq&
\frac{1}{1+\lambda}|\overrightarrow{AB}|+\frac{\lambda}{1+\lambda} |\overrightarrow{AB}|=\\
 &=&
\left( \frac{1}{1+\lambda}+\frac{\lambda}{1+\lambda} \right)\cdot|\overrightarrow{AB}|=\\
 &=&|\overrightarrow{AB}|.
 \end{eqnarray*}



  %Težišče
    %_____________________________________

  \item \label{nalVekt24}
\res{Točki $T$ in $T'$ sta težišči $n$-kotnikov $A_1A_2...A_n$ in $A'_1A'_2...A'_n$. Izračunaj:
$$\overrightarrow{A_1A'_1}+\overrightarrow{A_2A'_2}+\cdots+\overrightarrow{A_nA'_n}.$$}

Uporabimo relacijo $\overrightarrow{A_kA'_k}
=\overrightarrow{A_kT}+\overrightarrow{TT'}+\overrightarrow{T'A'_k}$ ($k\in \{1,2,\ldots ,n \}$) in definicijo težišča. Rezultat $n\cdot\overrightarrow{TT'}$.

\poglavje{The basics of Geometry} \label{osn9Geom}
From the given condition it follows that $\frac{\overrightarrow{AM}}{\overrightarrow{MD}}= \frac{\overrightarrow{NB}}{\overrightarrow{CN}}=\lambda$ or $\overrightarrow{AM}=\lambda\overrightarrow{MD}$  and $\overrightarrow{NB}=\lambda\overrightarrow{CN}$.
Since $\mathcal{B}(A,M,D)$ and $\mathcal{B}(B,N,C)$, $\lambda\geq 0$. Furthermore:
$\overrightarrow{MN}=\overrightarrow{MA}+\overrightarrow{AB}+\overrightarrow{BN}$ and $\overrightarrow{MN}=\overrightarrow{MD}+\overrightarrow{DC}+\overrightarrow{CN}$.
If we multiply the second relation by $\lambda$ and add the first, we get:
 $(1+\lambda)\overrightarrow{MN}=\overrightarrow{AB}+\lambda \overrightarrow{DC}$ or:
$$\overrightarrow{MN}=\frac{1}{1+\lambda}\overrightarrow{AB}+\frac{\lambda}{1+\lambda} \overrightarrow{DC}.$$
Without loss of generality, we assume that $|AB|\geq|CD|$. If we use the triangle inequality for vectors \ref{neenakTrikVekt}, we get
 \begin{eqnarray*}
 |\overrightarrow{MN}|
 &=&
 |\frac{1}{1+\lambda}\overrightarrow{AB}+\frac{\lambda}{1+\lambda} \overrightarrow{DC}|\leq\\
 &\leq&
 |\frac{1}{1+\lambda}\overrightarrow{AB}|+|\frac{\lambda}{1+\lambda} \overrightarrow{DC}|=\\
 &=&
\frac{1}{1+\lambda}|\overrightarrow{AB}|+\frac{\lambda}{1+\lambda} |\overrightarrow{DC}|\leq\\
 &\leq&
\frac{1}{1+\lambda}|\overrightarrow{AB}|+\frac{\lambda}{1+\lambda} |\overrightarrow{AB}|=\\
 &=&
\left( \frac{1}{1+\lambda}+\frac{\lambda}{1+\lambda} \right)\cdot|\overrightarrow{AB}|=\\
 &=&|\overrightarrow{AB}|.
 \end{eqnarray*}



  %Težišče

\item \res{Prove that a quadrilateral $ABCD$ and  $A'B'C'D'$ have a common center of gravity exactly when:
$$\overrightarrow{AA'}+\overrightarrow{BB'}+\overrightarrow{CC'}+
\overrightarrow{DD'}=\overrightarrow{0}.$$}

 Use the previous task \ref{nalVekt24}.

    \item \res{Let $P$, $Q$, $R$ and $S$ be the centers of gravity of the triangles $ABD$, $BCA$, $CDB$ and $DAC$. Prove that the quadrilateral $PQRS$ and $ABCD$ have a common center of gravity.}

Let $T$ be the center of gravity of the quadrilateral $ABCD$. By definition: $\overrightarrow{TA}+\overrightarrow{TB}
+\overrightarrow{TC}+\overrightarrow{TD}=\overrightarrow{0}$. From this and the assumption that $P$, $Q$, $R$ and $S$ are the centers of gravity of the triangles $ABD$, $BCA$, $CDB$ and $DAC$ and the formula \ref{vektTezTrikXT} we have:
\begin{eqnarray*}
\overrightarrow{TP}+\overrightarrow{TQ}
+\overrightarrow{TR}+\overrightarrow{TS}
&=&
\frac{1}{3}\left( \overrightarrow{TA}+\overrightarrow{TB}+\overrightarrow{TD} \right)+\\
&+&
\frac{1}{3}\left( \overrightarrow{TB}+\overrightarrow{TC}+\overrightarrow{TA} \right)+\\
&+&
\frac{1}{3}\left( \overrightarrow{TC}+\overrightarrow{TD}+\overrightarrow{TB} \right)+\\
&+&
\frac{1}{3}\left( \overrightarrow{TD}+\overrightarrow{TA}+\overrightarrow{TC} \right)=\\
&=&
 \overrightarrow{TA}+\overrightarrow{TB}+\overrightarrow{TC}+ \overrightarrow{TD}=\overrightarrow{0},
\end{eqnarray*}
 which means that the point $T$ is at the same time the center of gravity of the quadrilateral $PQRS$.

  \item \res{Let $A_1A_2A_3A_4A_5A_6$ be an arbitrary hexagon and $B_1$, $B_2$, $B_3$, $B_4$, $B_5$ and $B_6$ in order the centers of gravity of the triangles $A_1A_2A_3$, $A_2A_3A_4$, $A_3A_4A_5$, $A_4A_5A_6$, $A_5A_6A_1$ and $A_6A_1A_2$.
Prove that these centers of gravity determine a hexagon with three pairs of parallel sides.}

Use the formula \ref{vektTezTrikXT}.

\item \res{Let $A$, $B$, $C$ be four different points in $D$. The points $T_A$, $T_B$, $T_C$ and $T_D$
        are the centroids of the triangles $BCD$, $ACD$, $ABD$ and $ABC$. Prove that the lines $AT_A$, $BT_B$, $CT_C$ and $DT_D$ intersect in one point $T$. In what ratio does the point $T$ divide these lines?}

We use the statement \ref{vektTezTrikXT}. The desired ratio is $3:1$.
We also use the example \ref{TetivniTezisce}.

 \item \label{nalVekt29}
 \res{Let $CC_1$ be the centroid of the triangle $ABC$ and $P$ any point on the side
$AB$ of this triangle. The parallel $l$ of the line $CC_1$ through the point $P$ intersects the lines $AC$ and $BC$ in the points $M$ and $N$. Prove that:
$$\overrightarrow{PM} + \overrightarrow{PN}= \overrightarrow{AC} + \overrightarrow{BC}.$$}

First, $\overrightarrow{AC} + \overrightarrow{BC}=-(\overrightarrow{CA} + \overrightarrow{CB})=-2\overrightarrow{CC_1}=2\overrightarrow{C_1C}$ (statement \ref{vektSredOSOAOB}).
By Tales' theorem \ref{TalesovIzrek} we have: $\frac{\overrightarrow{PN}}{\overrightarrow{C_1C}}=
\frac{\overrightarrow{BP}}{\overrightarrow{BC_1}}=
\frac{|BP|}{|BC_1|}$ or $\overrightarrow{PN}=\frac{|BP|}{|BC_1|}\cdot \overrightarrow{C_1C}$.
Similarly, we have:
$\frac{\overrightarrow{PM}}{\overrightarrow{C_1C}}=
\frac{\overrightarrow{AP}}{\overrightarrow{AC_1}}=
\frac{|AP|}{|AC_1|}$ or $\overrightarrow{PM}=\frac{|AP|}{|AC_1|}\cdot \overrightarrow{C_1C}$.
 Therefore:
 \begin{eqnarray*}
 \overrightarrow{PM} + \overrightarrow{PN}
&=&
\frac{|BP|}{|BC_1|}\cdot \overrightarrow{C_1C}+\frac{|AP|}{|AC_1|}\cdot \overrightarrow{C_1C}=\\
&=&
\frac{|BP|+|AP|}{\frac{1}{2}|AB|}\overrightarrow{C_1C}=2\overrightarrow{C_1C}=\\
&=&
\overrightarrow{AC} + \overrightarrow{BC}.
 \end{eqnarray*}

 %Hamilton in Euler
 %______________________________________________________


 \item \res{Let $A$, $B$, $C$ be points of a plane that lie on the same side of the line $p$, and $O$ a point on
the line $p$, for which $|\overrightarrow{OA}| = |\overrightarrow{OB}| = |\overrightarrow{OC}| =1$. Prove that then also: $$|\overrightarrow{OA} + \overrightarrow{OB} + \overrightarrow{OC}| \geq 1.$$}

The point $O$ is the center of the circle $k(O,1)$ of the triangle $ABC$. According to Hamilton's theorem \ref{Hamilton}, $\overrightarrow{OA} + \overrightarrow{OB} + \overrightarrow{OC}=\overrightarrow{OV}$, where $V$ is the altitude point of the triangle $ABC$. Because the points $A$, $B$, $C$ lie on the same side of the line $p$, the center $O$ is an external point of this triangle. This means that $ABC$ is a topocentric triangle. Without loss of generality, let $\angle BAC>90^0$. This means $\mathcal{B}(A',A,V)$ (where $A'$ is the altitude of this triangle) or $V$ is an external point of the circle $k$ and it holds $|\overrightarrow{OV}|>1$.
We can also look at the task \ref{OlimpVekt15} (section \ref{odd5DolzVekt}).

\item \res{Calculate the angles determined by the vectors $\overrightarrow{OA}$, $\overrightarrow{OB}$ and $\overrightarrow{OC}$, if the points $A$, $B$ and $C$ lie on the circle with the center
$O$ and in addition it holds:
$$\overrightarrow{OA} + \overrightarrow{OB} + \overrightarrow{OC} = \overrightarrow{0}.$$}

According to Hamilton's theorem \ref{Hamilton}, $\overrightarrow{0}=\overrightarrow{OA} + \overrightarrow{OB} + \overrightarrow{OC}=\overrightarrow{OV}$, where $V$ is the altitude point of the triangle $ABC$. So $V=O$, which means that $ABC$ is an equilateral triangle and all the angles sought are equal to $120^0$.

 \item \res{Let $A$, $B$, $C$ and $D$ be points that lie on the circle with the center
$O$, and it holds
$$\overrightarrow{OA} + \overrightarrow{OB} + \overrightarrow{OC} + \overrightarrow{OD} = \overrightarrow{0}.$$
Prove that $ABCD$ is a rectangle.}

We use Hamilton's theorem \ref{Hamilton}.

\item \res{Let $\overrightarrow{a}$, $\overrightarrow{b}$ and $\overrightarrow{c}$ be vectors of some plane, for which it holds $|\overrightarrow{a}| = |\overrightarrow{b}| = |\overrightarrow{c}| =x$. Investigate, in which case it also holds $|\overrightarrow{a} + \overrightarrow{b} + \overrightarrow{c}| = x$.}

First, let's assume that $\overrightarrow{a}$, $\overrightarrow{b}$ and $\overrightarrow{c}$ are nonlinear vectors.
Let $O$ be an arbitrary point and $A$, $B$ and $C$ be such points that $\overrightarrow{OA}=\overrightarrow{a}$, $\overrightarrow{OB}=\overrightarrow{b}$ and $\overrightarrow{OC}=\overrightarrow{c}$ (statement \ref{vektAvObst1TockaB}). From the nonlinearity of vectors $\overrightarrow{a}$, $\overrightarrow{b}$ and $\overrightarrow{c}$ it follows that points $A$, $B$ and $C$ are nonlinear. From the condition $|\overrightarrow{a}| = |\overrightarrow{b}| = |\overrightarrow{c}| =x$ it follows that $|\overrightarrow{OA}| = |\overrightarrow{OB}| = |\overrightarrow{OC}| =x$, which means that point $O$ is the center of the circumscribed circle of triangle $ABC$. By Hamilton's statement \ref{Hamilton} it is:
 \begin{eqnarray*}
x
 &=&
|\overrightarrow{a} + \overrightarrow{b} + \overrightarrow{c}|=\\
 &=&
|\overrightarrow{OA} + \overrightarrow{OB} + \overrightarrow{OC}|=\\
 &=& |\overrightarrow{OV}|,
\end{eqnarray*}
 where $V$ is the altitude point, which therefore lies on the circumscribed circle of triangle $ABC$. This is possible only when $ABC$ is a right triangle.

If $\overrightarrow{a}$, $\overrightarrow{b}$ and $\overrightarrow{c}$ are collinear vectors, the given condition is satisfied only when $\overrightarrow{a}$, $\overrightarrow{b}$ and $\overrightarrow{c}$ are not all of the same orientation.


 %Tales
 %______________________________________________________


\item \res{Divide the given line segment $AB$:}

  (\textit{a}) \res{into five equal line segments,}\\
  (\textit{b}) \res{in the ratio $2:5$,}\\
  (\textit{c}) \res{into three line segments, which are in the ratio $2:\frac{1}{2}:1$.}

Use example \ref{izrekEnaDelitevDaljice}.


  \item \res{Given is the line segment $AB$. Only by using a ruler with the ability to draw parallels (constructions in affine geometry), plot point $C$, if:}

(\textit{a}) \res{$\overrightarrow{AC}=\frac{1}{3}\overrightarrow{AB}$,} \hspace*{3mm}
   (\textit{b}) \res{$\overrightarrow{AC}=\frac{3}{5}\overrightarrow{AB}$,} \hspace*{3mm}
   (\textit{c}) \res{$\overrightarrow{AC}=-\frac{4}{7}\overrightarrow{AB}$.}

Use task \ref{nalVekt4} and example \ref{izrekEnaDelitevDaljiceNan}.

\item \res{The lines $a$, $b$ and $c$ are given. Draw the line $x$, so that it will:}

(\textit{a}) \res{$a:b=c:x$;} \hspace*{3mm}
(\textit{b}) \res{$x=\frac{a\cdot b}{c}$,} \hspace*{3mm}
(\textit{c}) \res{$x=\frac{a^2}{c}$,}\hspace*{3mm}\\
(\textit{č}) \res{$x=\frac{2ab}{3c}$,}\hspace*{3mm}
(\textit{č}) \res{$(x+c):(x-c)=7:2$.}

We use the Pythagorean theorem \ref{TalesovIzrek}.

\item \res{Let $M$ and $N$ be points on the arm $OX$, $P$ a point on the arm $OY$ as $XOY$ and $NQ\parallel MP$ and $PN\parallel QS$ ($Q\in OY$, $S\in OX$). Prove that $|ON|^2=|OM|\cdot |OS|$ (for the line $ON$ in this case we say that it is the \index{geometrijska sredina daljic}\pojem{geometric mean} \color{green1} of the lines $OM$ and $OS$).}

By the Pythagorean theorem \ref{TalesovIzrek} it is:
$$\frac{|ON|}{|OM|}=\frac{|OQ|}{|OP|}=\frac{|OS|}{|ON|},$$ from which the desired relation follows.

\item \res{Let $ABC$ be a triangle and $Q$, $K$, $L$, $M$, $N$ and $P$ such points of the sides $AB$, $AC$, $BC$, $BA$, $CA$ and $CB$, that $AQ\cong CP\cong AC$, $AK\cong BL\cong AB$ and $BM\cong CN\cong BC$.
Prove that $MN$, $PQ$ and $LK$ are parallel.}

We use the formulas \ref{vektParamPremica} and \ref{vektDelitDaljice}.

\item \res{Let $P$ be the center of gravity of the centroid $AA_1$ of the triangle $ABC$. The point $Q$ is the intersection of the line $BP$
with the side $AC$. Calculate the ratios $AQ:QC$ and $BP:PQ$.}

Let $R$ be the center of the line $QC$. By the formula \ref{srednjicaTrikVekt} (for the triangle $BQC$ and $AA_aR$) it is:
$$\overrightarrow{PQ}=\frac{1}{2}\overrightarrow{A_1R}=\frac{1}{4}\overrightarrow{BQ},$$
so $BP:PQ=3:1$. From $\overrightarrow{PQ}=\frac{1}{2}\overrightarrow{A_1R}$ it follows that $PQ\parallel A_1R$, so by the Pythagorean theorem $AQ:QR=AP:PA_1=1:1$. Therefore $AQ\cong QR \cong RC$ or $AQ:QC=1:2$.

See also the example \ref{TezisceSredisceZgled}.

\item \res{The points $P$ and $Q$ lie on the sides $AB$ and $AC$ of the triangle $ABC$, and it holds that $\frac{|\overrightarrow{PB}|}{|\overrightarrow{AP}|}
    +\frac{|\overrightarrow{QC}|}{|\overrightarrow{AQ}|}=1$. Prove that the centroid of this triangle lies on the line $PQ$.}

Let $\widehat{T}$ be the intersection of the line $PQ$ with the centroidal line $AA_1$ of this triangle. It is enough to prove that $\widehat{T}=T$.
Since the points $A$, $\widehat{T}$ and $A_1$ are collinear, $\overrightarrow{A\widehat{T}}=\lambda\overrightarrow{AA_1}$ for some $\lambda\in R$. For $\widehat{T}=T$ it is enough to prove that $\lambda=\frac{2}{3}$.

We denote $\alpha=\frac{|\overrightarrow{PB}|}{|\overrightarrow{AP}|}$ and $\beta=\frac{|\overrightarrow{QC}|}{|\overrightarrow{AQ}|}$. By the assumption, $\alpha+\beta=1$. From the given conditions it follows that
 $\overrightarrow{AB}=\left(1+\alpha\right)\cdot\overrightarrow{AP}$ and
 $\overrightarrow{AC}=\left(1+\beta\right)\cdot\overrightarrow{AQ}$. If we use the latter relation and the formula \ref{vektSredOSOAOB}, we get:
 \begin{eqnarray*}
 \overrightarrow{A\widehat{T}} &=& \lambda\overrightarrow{AA_1}=\\
 &=&
\frac{\lambda}{2}\cdot \left(\overrightarrow{AB}+\overrightarrow{AC}\right)=\\
 &=&
\frac{\lambda}{2}\cdot \left(\left(1+\alpha\right)\cdot\overrightarrow{AP}
+\left(1+\beta\right)\cdot\overrightarrow{AQ}\right)=\\
 &=&
\frac{\lambda}{2}\cdot \left(1+\alpha\right)\cdot\overrightarrow{AP}
+\frac{\lambda}{2}\left(1+\beta\right)\cdot\overrightarrow{AQ}.
 \end{eqnarray*}
Since the point $T$ lies on the line $PQ$ and since $\alpha+\beta=1$, by the formula \ref{vektParamPremica}:
 \begin{eqnarray*}
1&=&\frac{\lambda}{2}\cdot \left(1+\alpha\right)
+\frac{\lambda}{2}\left(1+\beta\right)=\\
&=&\frac{\lambda}{2}\cdot \left(1+\alpha+1+\beta\right)=\\
 &=&\frac{3\lambda}{2}\\
 \end{eqnarray*}
i.e. $\lambda=\frac{2}{3}$.


 \item \res{Let $a$, $b$ and $c$ be three segments with a common starting point $S$ and $M$
a point on the segment $a$. If the point $M$ "moves" along the segment $a$,
 the ratio of the distances of this point from the lines $b$ and $c$ is constant.}

We use the Tales' theorem  \ref{TalesovIzrek}.

\item \res{Let $D$ be a point lying on the side $BC$ of the triangle $ABC$ and
 $F$ and $G$ points in which the line passing through the point $D$ and parallel to the median $AA_1$, intersects the lines $AB$ and $AC$.
Prove that the sum $|DF|+|DG|$ is constant if the point $D$ "moves" along the side
$BC$.}

We prove that $|DF|+|DG|=|AA_1|$. See problem \ref{nalVekt29}.


   \item
   \res{Draw a triangle with the following data:}

In both cases use the same notation as in the big problem \ref{velikaNaloga} and the statement \ref{velNalTockP'}.

   (\textit{a}) \res{$v_a$, $r$, $b-c$}

First, we draw the rectangular triangle $SPP_a$ ($SP=r$, $\angle P=90^0$ and $PP_a=b-c$), the point $P'$ and finally use the fact that the vertex $A$ lies on the segment $P_aP'$.

   (\textit{b}) \res{$\beta$, $r$, $b-c$}

Similarly to the previous case.


\end{enumerate}




%REŠITVE -  Izometrije
%________________________________________________________________________________

\poglavje{Isometries}

\begin{enumerate}
  \item
  \res{Given is a line $p$ and points $A$ and $B$, which lie on
  opposite sides of the line $p$. Construct a point
  $X$, which lies on
the line $p$, so that the difference $|AX|-|XB|$ is maximal.}

Let $A'=\mathcal{S}_p(A)$. We prove that $X=A'B\cap p$ is the desired point (see also example \ref{HeronProbl}).

  \item \res{In the plane are given lines $p$, $q$ and $r$. Construct
  an equilateral triangle $ABC$, so that
   the vertex $B$ lies on the line $p$, $C$ on $q$,
the altitude from the vertex $A$ lies on the line $r$.}

Since $\mathcal{S}_r(B)=C$ and $B\in p$, the vertex $C$ we get from
the condition $C\in \mathcal{S}_r(p)\cap q$.

\item \res{Given is a quadrilateral $ABCD$ and a point $S$. Draw a parallelogram
with the center in the point $S$, so
that its vertices lie on the lines
of the given quadrilateral.}

We use the central reflection $\mathcal{S}_S$.

\item \res{Let $\mathcal{I}$ be an indirect isometry of a plane that
maps point $A$ to point $B$,
$B$ to $A$. Prove that $\mathcal{I}$ is the central reflection.}

Let $s$ be the line of symmetry of the segment $AB$. The composition
$\mathcal{I}\circ\mathcal{S}_s$ is a direct isometry with two
fixed points $A$ and $B$, therefore by izreku \ref{izo2ftIdent} it
represents the identity or
$\mathcal{I}\circ\mathcal{S}_s=\mathcal{E}$. If we multiply the
equality by $\mathcal{S}_s$ from the right, we get
$\mathcal{I}=\mathcal{S}_s$.

\item  \res{Let $K$ and $L$ be points that are symmetric to the vertex
$A$ of the triangle $ABC$ with respect to
the lines of symmetry of the internal angles at the vertices  $B$ and
$C$. Let $P$ be the point of
tangency of the inscribed circle of this triangle and the side $BC$.
Prove that $P$ is the center of the segment $KL$.}

We denote by $s_\beta$ and $s_\gamma$ the mentioned lines of symmetry
of the internal angles at the vertices  $B$ and $C$ and by $Q$ and $R$
the points of tangency of the inscribed circle of the triangle $ABC$
with the sides $AC$ and $AB$. By izreku \ref{TangOdsek} we have
$AQ\cong AR$, $BR\cong BP$ and $CP\cong CQ$. Therefore
$\mathcal{S}_{s_\beta}:A,R\mapsto K,P$ and
$\mathcal{S}_{s_\gamma}:A,Q\mapsto L,P$. From this it follows that
$KP\cong AR\cong AQ\cong LP$, which means that $P$ is the center of
the segment $KL$.

\item  \res{Let $k$ and $l$ be circles on different sides of a line $p$.
Draw an isosceles triangle $ABC$, so that its altitude $AA'$ lies on
the line $p$, the vertex $B$ on the circle $k$, and the vertex $C$ on
the circle $l$.}

Since $\mathcal{S}_p(B)=C$ and $B\in k$, we get the point $C$ from
the condition $C\in \mathcal{S}_p(k)\cap l$.

\item \res{Let $k$ be a circle and $a$, $b$ and $c$ lines in the same
plane. Draw a triangle $ABC$, which is
inscribed in the circle $k$, so that its sides $BC$, $AC$ and  $AB$
will be parallel in order to the lines $a$, $b$ and $c$.}

First, we can plan the simetrals of sides $BC$, $AC$ and $AB$ as
lines that go through the center $O$ of the circle $k$ and are
perpendicular in turn to the lines $a$, $b$ and $c$. We mark these
simetrals with $p$, $q$ and $r$. Because the composite
$\mathcal{I}=\mathcal{S}_r\circ\mathcal{S}_q\circ\mathcal{S}_p$
has the fixed points $O$ and $A$, by Theorem \ref{izo1ftIndZrc}
$\mathcal{I}=\mathcal{S}_{OA}$. So we can draw the line $OA$ as the
simetral of the distance $XX'$, where $X$ is an arbitrary point.
The point $A$ we get as the intersection of the line $OA$ and the
circle $k$. Then $B=\mathcal{S}_p(A)$ and $C=\mathcal{S}_r(A)$.

\item  \res{Let $ABCDE$ be a rope pentagon, in which $BC\parallel
DE$ and $CD\parallel EA$.
Prove that the point $D$ lies on the simetral of the distance $AB$.}

We mark with $O$ the center of the inscribed circle of the
pentagon $ABCDE$. Because $BC\parallel DE$, the lines $BC$ and $DE$
have the same perpendicular from the center $S$, which is actually
the common simetral of sides $BC$ and $DE$. We mark it with $p$.
Similarly, sides $CD$ and $EA$ have the common simetral $q$. We
mark with $r$ the simetral of the distance $AB$. We prove that
$D\in r$. The composite
$\mathcal{S}_{OD}\circ\mathcal{S}_q
\circ\mathcal{S}_p\circ\mathcal{S}_r
\circ\mathcal{S}_q\circ\mathcal{S}_p$ is a direct isometry, which
has the fixed points $O$ and $D$, so by Theorem \ref{izo2ftIdent} it
represents the identity. Therefore:
$$\mathcal{S}_{OD}\circ\mathcal{S}_q
\circ\mathcal{S}_p\circ\mathcal{S}_r
\circ\mathcal{S}_q\circ\mathcal{S}_p=\mathcal{E}.$$

Premices $p$, $q$ and $r$ are from the same $\mathcal{X}_O$, so
the composite $\mathcal{S}_q \circ\mathcal{S}_p\circ\mathcal{S}_r$
represents the basic reflection (statement \ref{izoSop}) or
$$\mathcal{S}_q
\circ\mathcal{S}_p\circ\mathcal{S}_r=\mathcal{S}_l
=\mathcal{S}^{-1}_l=\left(\mathcal{S}_q
\circ\mathcal{S}_p\circ\mathcal{S}_r\right)^{-1}=\mathcal{S}_r
\circ\mathcal{S}_p\circ\mathcal{S}_q.$$
 Therefore:
 \begin{eqnarray*}
\mathcal{E}&=&\mathcal{S}_{OD}\circ\mathcal{S}_q
\circ\mathcal{S}_p\circ\mathcal{S}_r
\circ\mathcal{S}_q\circ\mathcal{S}_p =\\
&=&\mathcal{S}_{OD}\circ\mathcal{S}_r
\circ\mathcal{S}_p\circ\mathcal{S}_q
\circ\mathcal{S}_q\circ\mathcal{S}_p =
\mathcal{S}_{OD}\circ\mathcal{S}_r.
\end{eqnarray*}
 From
$\mathcal{E}=\mathcal{S}_{OD}\circ\mathcal{S}_r$ follows
$\mathcal{S}_{OD}=\mathcal{S}_r$ or $OD=r$ and $D\in r$.


\item  \res{Premice $p$, $q$ in $r$ ležijo v isti ravnini. Dokaži
ekvivalenco $\mathcal{S}_r\circ\mathcal{S}_q\circ\mathcal{S}_p
 =\mathcal{S}_p\circ\mathcal{S}_q\circ \mathcal{S}_r$ natanko
 tedaj, ko premice $p$, $q$ in $r$ pripadajo istemu šopu.}

Označimo
$\mathcal{I}=\mathcal{S}_r\circ\mathcal{S}_q\circ\mathcal{S}_p$.

($\Leftarrow$) Predpostavimo, da premice $p$, $q$ in $r$
pripadajo istemu šopu. Po izreku \ref{izoSop} izometrija
$\mathcal{I}$ predstavlja osmo zrcaljenje. Iz tega sledi
$\mathcal{I}=\mathcal{I}^{-1}$ oz.
$\mathcal{S}_r\circ\mathcal{S}_q\circ\mathcal{S}_p
 =\mathcal{S}_p\circ\mathcal{S}_q\circ \mathcal{S}_r$

($\Rightarrow$) Let us assume that
 $\mathcal{S}_r\circ\mathcal{S}_q\circ\mathcal{S}_p
 =\mathcal{S}_p\circ\mathcal{S}_q\circ \mathcal{S}_r$. From this
 condition it follows that $\mathcal{I}=\mathcal{I}^{-1}$ or
$\mathcal{I}^2=\mathcal{E}$. Let us assume the opposite - that
lines $p$, $q$ and $r$ are not from the same sheaf. According to
the statement \ref{izoZrcdrsprq} it
is $\mathcal{I}=\mathcal{G}_{2\overrightarrow{AB}}$
($\overrightarrow{AB}\neq \overrightarrow{0}$). But in this
case $\mathcal{E}=\mathcal{I}^2=\mathcal{G}^2_{2\overrightarrow{AB}}
=\mathcal{T}_{4\overrightarrow{AB}}$ (statement
\ref{izoZrcdrsZrcdrs}), which is not possible, because according to the assumption
$\overrightarrow{AB}\neq \overrightarrow{0}$. Therefore lines
$p$, $q$ and $r$ belong to the same sheaf.

\item \res{Let $O$, $P$ and $Q$ be three non-linear points. Construct
a square $ABCD$ (in the plane $OPQ$) with the center in  point $O$, so
that points $P$ and $Q$ in order
lie on lines $AB$ and $BC$.}

We use the fact that $Q,\mathcal{R}_{O,90^0}(P)\in BC$.

\item \res{Let $\mathcal{R}_{S,\alpha}$ be a rotation and $\mathcal{S}_p$
be a reflection in the same plane and $S\in p$.
Prove that the composition $\mathcal{R}_{S,\alpha}\circ\mathcal{S}_p$
and $\mathcal{S}_p\circ\mathcal{R}_{S,\alpha}$ represent a reflection.}

We use the statement \ref{rotacKom2Zrc} and \ref{izoSop}.

\item \res{Given is a point $A$ and a circle $k$ in the same plane. Draw
a square $ABCD$, so that the vertices of the diagonal $BD$ lie on the circle $k$.}

From $B,D\in k$ and $\mathcal{R}_{A,90^0}(B)=D$ it follows that $D\in k\cap
\mathcal{R}_{A,90^0}(k)$, which allows us to construct the vertex $D$.
Then $B=\mathcal{R}_{A,-90^0}(D)$ and
$C=\mathcal{R}_{D,90^0}(A)$.

\item \res{Let $ABC$ be an arbitrary triangle. Prove:
 $$\mathcal{R}_{C,2\measuredangle BCA}\circ
 \mathcal{R}_{B,2\measuredangle ABC}\circ
 \mathcal{R}_{A,2\measuredangle CAB}=\mathcal{E}.$$}

According to the statement \ref{rotacKom2Zrc} it is:
  \begin{eqnarray*}
 &&\mathcal{R}_{C,2\measuredangle BCA}\circ
 \mathcal{R}_{B,2\measuredangle ABC}\circ
 \mathcal{R}_{A,2\measuredangle CAB}=\\ &&=
 \mathcal{S}_{CA}\circ\mathcal{S}_{CB}\circ\mathcal{S}_{BC}
 \circ\mathcal{S}_{BA}\circ\mathcal{S}_{AB}\circ\mathcal{S}_{AC}=
 \mathcal{E}.
 \end{eqnarray*}

\item \res{Prove that the composite of the central reflection $\mathcal{S}_p$
and the central reflection $\mathcal{S}_S$ ($S\in p$)
represents the central reflection.}

  Let $q$ be the rectangle of the line $p$ through the point $S$. According to the
  statement \ref{izoSrZrcKom2Zrc} it is:
   $$\mathcal{S}_S\circ\mathcal{S}_p=
   \mathcal{S}_q\circ\mathcal{S}_p\circ\mathcal{S}_p=\mathcal{S}_q.$$

\item \res{Let $O$, $P$ and $Q$ be three non-linear points. Construct
the square $ABCD$ (in the plane $OPQ$) with the center in the point $O$, so
 that the points $P$ and $Q$
lie in succession on the lines $AB$ and $CD$.}

We use the fact that $Q,\mathcal{S}_O(P)\in CD$.

\item \res{What does the composite of the translation and the central reflection represent?}

We use the statements \ref{transl2sred} and \ref{izoKomp3SredZrc}.

\item \res{Given are the line $p$ and the circles  $k$ and $l$, which
lie in the same plane. Draw the line, which is parallel to
the line $p$, so that it determines the corresponding tangents on the circles $k$ and $l$.}

Let $K$ and $L$ be the centers of the circles $k$ and $l$. Let $q$ be the desired line that is parallel to the line $p$, and the circles $k$ and $l$ intersect in such points $A$, $B$, $C$ and $D$, so that $AB \cong CD$. Let $\mathcal{B} (A, B, C, D)$. We denote by $K_q$ and $L_q$ the orthogonal projections of the centers $K$ and $L$ onto the line $q$, and $K_p$ and $L_p$ the orthogonal projections of the centers $K$ and $L$ onto the line $p$. From the congruence of the triangles $KAK_q$ and $KBK_q$ (SSA Theorem \ref{SSK}) it follows that $K_q$ is the center of the chord $AB$. Similarly, $L_q$ is the center of the chord $CD$. From $AB \cong CD$ it then follows that $K_qB \cong L_qD$, i.e. $\overrightarrow {K_qB} = \overrightarrow {L_qD}$. Therefore:
$$\overrightarrow {BD} = \overrightarrow {BD} + \overrightarrow {0} = \overrightarrow {BD} + \overrightarrow {K_qB} + \overrightarrow {DL_q} = \overrightarrow {K_qL_q} = \overrightarrow {K_pL_p}.$$ The vector $\overrightarrow {v} = \overrightarrow {K_pL_p}$ can be constructed. This means that the previous analysis allows the construction, because $\mathcal {T} _ {\overrightarrow {v}} (B) = D$, i.e. $D \in \mathcal {T} _ {\overrightarrow {v}} (k) \cap l$.

\item \res {Let $c$ be a line that intersects the parallels $a$ and $b$, and $l$ a distance. Draw an isosceles triangle $ABC$ so that $A \in a$, $B \in b$, $C \in c$ and $AB \cong l$.}

First, we construct an arbitrary isosceles triangle $A_1B_1C_1$ that satisfies the conditions $A_1 \in a$, $B_1 \in b$ and $A_1B_1 \cong l$, then we use the appropriate translation.

\item \res {Prove that the composition of a rotation and an axis reflection of a plane represents a mirror glide exactly when the center of rotation does not lie on the axis of the axis reflection.}

Use Theorems \ref {rotacKom2Zrc}, \ref {izoZrcdrsprq} and \ref {izoSop}.

\item \res {Let $ABC$ be an isosceles triangle. Prove that the composition $\mathcal {S} _ {AB} \circ \mathcal {S} _ {CA} \circ \mathcal {S} _ {BC}$ represents a mirror glide. Also determine the vector and the axis of this glide.}

Let $\mathcal{I}=\mathcal{S}_{AB} \circ\mathcal{S}_{CA}
    \circ\mathcal{S}_{BC}$. By  \ref{izoZrcdrsprq} is
    $\mathcal{I}$ a mirror reflection. Let $A_1$, $B_1$ and $C_1$ be in order the center points of the lines $BC$, $AC$ and $AB$ of the triangle $ABC$.
    Because it is a right triangle is $\mathcal{I}(A_1C_1)=A_1C_1$, which
    means that the line $A_1C_1$ is of this mirror reflection. It is not
    difficult to prove that for the point $A'_1=\mathcal{I}(A_1)$ (both
    lie on the axis $A_1C_1$) is
    $\overrightarrow{A_1A'_1}=3\overrightarrow{A_1C_1}$, so
    $\mathcal{I}=\mathcal{G}_{3\overrightarrow{A_1C_1}}$.

\item  \res{Given are the points $A$ and $B$ on the same side of the line
$p$.
Draw the line  $XY$, which lies on the line $p$ and is consistent
with the given line $l$, so that the sum
$|AX|+|XY|+|YB|$ is minimal.}

Let $A'=\mathcal{G}_{\overrightarrow{MN}}(A)$ (where $M,N\in
p$ and $MN\cong l$). The point $Y$ is obtained as the intersection of the lines $p$
and $X'Y$ (see also example \ref{HeronProbl}).

\item  \res{Let $ABC$ be an isosceles right triangle with a right angle at the vertex $A$. What does the composite
$\mathcal{G}_{\overrightarrow{AB}}\circ \mathcal{G}_{\overrightarrow{CA}}$ represent?}

Let $p$ and $q$ be the simetrali of the sides $CA$ and $AB$ of the triangle
$ABC$. By  \ref{izoZrcDrsKompSrOsn} is:
 $$\mathcal{G}_{\overrightarrow{AB}}\circ
 \mathcal{G}_{\overrightarrow{CA}}=
 \mathcal{S}_q\circ\mathcal{S}_A\circ\mathcal{S}_A\circ\mathcal{S}_p=
 \mathcal{S}_q\circ\mathcal{S}_p.$$ Because $ABC$ is an isosceles
 right triangle with a right angle at the vertex $A$, the lines $p$ and $q$ are perpendicular and intersect at the center $S$
 of the hypotenuse $BC$. Therefore
 $\mathcal{G}_{\overrightarrow{AB}}\circ
 \mathcal{G}_{\overrightarrow{CA}}=\mathcal{S}_q
 \circ\mathcal{S}_p=\mathcal{S}_S$.

\item \res{In the same plane are given the lines
$a$, $b$ and $c$.
Draw the points $A\in a$ and $B\in b$
so that $\mathcal{S}_c(A)=B$.}

From $A\in a$ it follows that $\mathcal{S}_c(A)\in \mathcal{S}_c(a)$ or
$B\in \mathcal{S}_c(a)$. Because $B\in b$ as well, we get the point $B$ from
the condition $B\in \mathcal{S}_c(a)\cap b$. Then $A=\mathcal{S}_c(B)$ is also true.

\item  \res{Given are the lines $p$ and $q$ and the point $A$ in the same plane.
Draw the points $B$ and $C$ so that
the lines $p$ and $q$ will be the internal angle bisectors of the triangle $ABC$.}

We use the fact that the line $BC$ is determined by the points
$\mathcal{S}_p(A)$ and $\mathcal{S}_q(A)$.

\item  \res{Let $p$, $q$ and $r$ be lines and $K$ and $L$ points in the same plane. Draw
the lines $s$ and $s'$, which go through the points $K$ and $L$ in order,
so that $\mathcal{S}_r\circ\mathcal{S}_q\circ\mathcal{S}_p(s)=s'$ is true.}

We denote
$\mathcal{I}=\mathcal{S}_r\circ\mathcal{S}_q\circ\mathcal{S}_p$.
The line $s'$ is determined by the points $L$ and $\mathcal{I}(K)$,
the line $s$ is determined by the points $K$ and $\mathcal{I}^{-1}(K)$. If
$\mathcal{I}(K)=L$, the task has infinitely many solutions - the line $s$
is an arbitrary line, which goes through the point $K$.


\item  \res{Let $s$ be the line which bisects one of the angles determined by the lines $p$
and $q$. Prove that $\mathcal{S}_s\circ\mathcal{S}_p =
\mathcal{S}_q\circ\mathcal{S}_s$.}

We can even prove more - that the equivalence (under
the assumption that the lines $p$ and $q$ intersect) holds. If we multiply the given equality by $\mathcal{S}_s$ from the right, we get the equivalent equality $\mathcal{S}_s\circ\mathcal{S}_p\circ\mathcal{S}_s =
\mathcal{S}_q$. By \ref{izoTransmutacija}, this is equivalent to the equality $\mathcal{S}_{\mathcal{S}_s(p)} =
\mathcal{S}_q$ or $\mathcal{S}_s(p) = q$, which is equivalent to the fact that $s$ bisects one of the angles determined by the lines $p$ and $q$.

\item \label{nalIzo27}
\res{Let $S$ be the center of the triangle $ABC$ inscribed in the circle and $P$ the point,
 in which this circle touches the side $BC$. Prove: $$\mathcal{S}_{SC}
 \circ\mathcal{S}_{SA}\circ\mathcal{S}_{SB} =\mathcal{S}_{SP}.$$}

 The equality is a direct consequence of the statement \ref{izo1ftIndZrc}, because
 is\\
 $\mathcal{S}_{SC}\circ\mathcal{S}_{SA}\circ\mathcal{S}_{SB}:S,P\mapsto
 S,P$.

\item  \res{The lines $p$, $q$ and $r$ of a plane go through the center
$S$ of the circle $k$.
Draw a triangle $ABC$, which is inscribed in this circle, so that
the lines $p$, $q$ and $r$ will be the altitudes of the internal angles at
the vertices $A$, $B$ and $C$ of this triangle.}

We use the previous task \ref{nalIzo27}.

\item  \res{The lines $p$, $q$, $r$, $s$ and $t$ of a plane intersect
in the point $O$, the point $M$ lies on the line $p$.
Draw such a pentagon that $M$ will be the center of one of its sides,
the lines $p$, $q$, $r$, $s$ and $t$ will be the altitudes of the sides.}

The point $M$ lies on one of the altitudes. Without loss of
generality, let $M\in p$. Let $ABCDE$ be such a pentagon that
the lines $p$, $q$, $r$, $s$ and $t$ are in turn the altitudes of
its sides $AB$, $BC$, $CD$, $DE$ and $EA$. Let
$\mathcal{I}=\mathcal{S}_t\circ\mathcal{S}_s\circ\mathcal{S}_r
\circ\mathcal{S}_q\circ\mathcal{S}_p$. Because
$\mathcal{I}:O,A\mapsto O,A$, by the statement \ref{izo1ftIndZrc}
$\mathcal{I}=\mathcal{S}_{OA}$. The line $OA$ is obtained as the altitude of the segment $XX'$, where $X$ is an arbitrary point and
$X'=\mathcal{I}(X)$, then the vertex $A$ as the intersection of the line
$OA$ and the rectangle of the line $p$ through the point $M$.

\item  \res{The point $P$ lies in the plane of the triangle $ABC$. Prove that
the lines, which are symmetrical to
the lines $AP$, $BP$ and $CP$ with respect to the altitudes of
 at the vertices $A$, $B$ and $C$ of this triangle, belong to the same set.}

Let us denote by $s_{\alpha}$, $s_{\alpha}$ and $s_{\alpha}$ the internal angle bisectors at the vertices $A$, $B$ and $C$ of the triangle $ABC$
and $a=\mathcal{S}_{s_{\alpha}}(AP)$, $b=\mathcal{S}_{s_{\beta}}(BP)$ and $c=\mathcal{S}_{s_{\gamma}}(CP)$.
We will prove that the lines $a$, $b$ and $c$ belong to the same pencil. Because $\mathcal{S}_{s_{\alpha}}:AC, p\rightarrow AB,
 a$ is $\measuredangle CAP=\measuredangle a,AP$. Therefore $\mathcal{R}_{A,2\measuredangle CAP}
 =\mathcal{R}_{A,2\measuredangle a,AP}$ or $\mathcal{S}_{AC}\circ \mathcal{S}_{AP}
 =\mathcal{S}_a\circ \mathcal{S}_{AB}$. From this it follows that  $\mathcal{S}_a=\mathcal{S}_{AC}
 \circ\mathcal{S}_{AP}\circ \mathcal{S}_{AB}$. Similarly: $\mathcal{S}_b=\mathcal{S}_{BA}
 \circ\mathcal{S}_{BP}\circ \mathcal{S}_{BC}$ and $\mathcal{S}_c=\mathcal{S}_{CB}\circ\mathcal{S}_{CP}
 \circ \mathcal{S}_{CA}$. Therefore it is (from \ref{izoSop} and \ref{izoTransmutacija}):
 \begin{eqnarray*}
 \mathcal{I}&=&\mathcal{S}_a\circ\mathcal{S}_b\circ\mathcal{S}_c=\\
 &=&
 \mathcal{S}_{AC}\circ\mathcal{S}_{AP}\circ \mathcal{S}_{AB}\circ
 \mathcal{S}_{BA}\circ\mathcal{S}_{BP}\circ \mathcal{S}_{BC}\circ
 \mathcal{S}_{CB}\circ\mathcal{S}_{CP}\circ \mathcal{S}_{CA}=\\
 &=&
 \mathcal{S}_{AC}\circ\mathcal{S}_{AP}\circ \mathcal{S}_{BP}\circ\mathcal{S}_{CP}\circ \mathcal{S}_{CA}=\\
 &=&
 \mathcal{S}_{AC}\circ\mathcal{S}_{AX}\circ \mathcal{S}_{CA}=\\
 &=&
 \mathcal{S}_{\mathcal{S}_{AC}(AX)}.
 \end{eqnarray*}
By  \ref{izoSop} the lines  $a$, $b$ and $c$ belong to the same pencil.

\item  \res{Calculate the angle determined by the lines $p$ and $q$, if it holds that:
$\mathcal{S}_p\circ\mathcal{S}_q\circ\mathcal{S}_p =
\mathcal{S}_q\circ\mathcal{S}_p\circ\mathcal{S}_q$.}

If we multiply the given equality from the left side in order by $\mathcal{S}_q$, $\mathcal{S}_p$ and $\mathcal{S}_q$, we get an equivalent equality: $\mathcal{S}_q\circ\mathcal{S}_p\circ\mathcal{S}_q\circ \mathcal{S}_p\circ\mathcal{S}_q\circ \mathcal{S}_p = \mathcal{E}$ or $\left(\mathcal{S}_q\circ\mathcal{S}_p\right)^3 = \mathcal{E}$.

If $p=q$, the last equality is clearly fulfilled.

In the case $p\parallel q$, $\mathcal{S}_q\circ\mathcal{S}_p$ is a translation, so it is always the case that $\left(\mathcal{S}_q\circ\mathcal{S}_p\right)^3 =\mathcal{T}^3_{\overrightarrow{v}}=\mathcal{T}_{3\overrightarrow{v}}\neq \mathcal{E}$.

If the lines $p$ and $q$ intersect, the composite $\mathcal{S}_q\circ\mathcal{S}_p$ is a rotation $\mathcal{R}_{S,\omega}$ ($p\cap q=\{S\}$ and $\omega=2\measuredangle p,q$). In this case, it is therefore the case that $\left(\mathcal{S}_q\circ\mathcal{S}_p\right)^3 =\mathcal{R}^3_{S,\omega}=\mathcal{R}_{S,3\omega}= \mathcal{E}$. This means that $3\omega=360^0$ or $\omega=120^0$, so that the lines $p$ and $q$ determine an angle of $60^0$.

\item  \res{Let $\mathcal{R}_{A,\alpha}$ and $\mathcal{R}_{B,\beta}$ be rotations in the same plane. Determine all points $X$ in this plane for which it is true that $\mathcal{R}_{A,\alpha}(X)=\mathcal{R}_{B,\beta}(X)$.}

The condition $\mathcal{R}_{A,\alpha}(X)=\mathcal{R}_{B,\beta}(X)$ is equivalent to the condition $\mathcal{R}_{A,\alpha}\circ\mathcal{R}^{-1}_{B,\beta}(X)=X$ or $\mathcal{R}_{A,\alpha}\circ\mathcal{R}_{B,-\beta}(X)=X$. It is therefore necessary to determine the fixed points of the isometry $\mathcal{I}=\mathcal{R}_{A,\alpha}\circ\mathcal{R}_{B,-\beta}$. We will use Theorem \ref{rotacKomp2rotac}. We will consider several cases.

 \textit{1}) If $A=B$ and $\alpha-\beta=k\cdot360^0$ (for some $k\in \mathbb{Z}$), then $\mathcal{I}=\mathcal{E}$, so the condition is clearly true for every point in this plane.

\textit{2}) If $A=B$ and $\alpha-\beta\neq k\cdot360^0$ (for
 every $k\in \mathbb{Z}$), then
 $\mathcal{I}=\mathcal{R}_{A,\alpha-\beta}$ and the condition
  is fulfilled only for the point $X=A$.

 \textit{3}) If $A\neq B$ and $\alpha-\beta=k\cdot360^0$ (for
 some $k\in \mathbb{Z}$), then $\mathcal{I}$ is a translation,
 so the condition does not apply to any point of this plane.

 \textit{4}) If $A\neq B$ and $\alpha-\beta\neq k\cdot360^0$
 (for every $k\in \mathbb{Z}$), then $\mathcal{R}_{C,\alpha-\beta}$
 (where $\measuredangle CAB=\frac{\alpha}{2}$ and
 $\measuredangle BAC=\frac{\beta}{2}$), which means that the condition
 applies only to the point $C$.

\item  \res{The lines $p$ and $q$ intersect at an angle of $60^0$ at the center $O$ of an isosceles
 triangle $ABC$. Prove that
 the segments determined by the sides
 of the triangle $ABC$ on the lines are congruent segments.}

We use the rotation $\mathcal{R}_{O,120^0}$.

\item   \res{Let $S$ be the center of a regular pentagon $ABCDE$.
Prove that:
 $$\overrightarrow{SA} + \overrightarrow{SB} + \overrightarrow{SC}
  + \overrightarrow{SD} + \overrightarrow{SE} = \overrightarrow{0}.$$}

Let $\overrightarrow{SA} + \overrightarrow{SB} +
 \overrightarrow{SC} + \overrightarrow{SD} + \overrightarrow{SE}
  = \overrightarrow{SX}$. Because $ABCDE$ is a regular pentagon,
  we have $\mathcal{R}_{S,72^0}:A,B,C,D,E,S,X\mapsto
  B,C,D,E,A,S,X'$. Therefore, $ \overrightarrow{SB} +
 \overrightarrow{SC} + \overrightarrow{SD} + \overrightarrow{SE}
 +\overrightarrow{SA} = \overrightarrow{SX'}$. This means that
 $\overrightarrow{SX}= \overrightarrow{SX'}$ or
 $X=X'=\mathcal{R}_{S,72^0}(X)$. Because $S$ is the only fixed point
 of the rotation $\mathcal{R}_{S,72^0}$, we have $X=S$. Therefore,
 $\overrightarrow{SA} + \overrightarrow{SB} +
 \overrightarrow{SC} + \overrightarrow{SD} + \overrightarrow{SE}
  = \overrightarrow{SS}=\overrightarrow{0}$.

\item   \res{Prove that the diagonals of a regular pentagon
intersect at points that are also
the vertices of a regular pentagon.}

We use the rotation with center at the center of a regular pentagon by
an angle of $72^0$.

\item   \res{Let $ABP$ and $BCQ$ be two triangles with the same orientation and $\mathcal{B}(A,B,C)$. The points $K$ and $L$ are the centers of the segments $AQ$ and $PC$. Prove that $BLK$ is a triangle.}

By rotating $\mathcal{R}_{B,-60^0}$, the points $A$ and $Q$ are mapped to the points $P$ and $C$. Therefore, the segment $AQ$ and its center $K$ are mapped to the segment $PC$ and its center $L$. From $\mathcal{R}_{B,-60^0}(K)=L$ it follows that $BLK$ is a triangle.

\item   \res{There are three concentric circles and a line in the same plane. Draw a triangle so that its vertices are on these circles in order, one side is parallel to the given line.}

First, we draw an arbitrary triangle without the condition that one of its sides is parallel to the given line. We can do this by choosing an arbitrary vertex $A$ on one of the circles and using the rotation $\mathcal{R}_{A,60^0}$. Then we use the appropriate rotation with the center at the center of the concentric circles, which maps the drawn triangle into a triangle whose one side is parallel to the given line.

\item  \res{The point $P$ is an internal point of a regular triangle $ABC$, so that
$\angle APB=113^0$ and $\angle BPC=123^0$. Calculate the size of the angles of the triangle whose sides are in correspondence with the segments
$PA$, $PB$ and $PC$.}

First, it is clear that $\angle APC=360^0-113^0-123^0=124^0$. We mark $P'=\mathcal{R}_{A,60^0}(P)$. $APP'$ is a regular triangle, so $PA\cong P'P$. Since $C=\mathcal{R}_{A,60^0}(B)$, we have $\triangle ABP\cong\triangle ACP'$, so $PB\cong P'C$ and $\angle APB\cong\angle AP'C$. This means that the sides $P'P$, $P'C$ and $PC$ of the triangle $PCP'$ are in correspondence with the segments $PA$, $PB$ and $PC$ in order. In this case, $\angle P'PC=124^0-60^0=64^0$, $\angle PP'C=113^0-60^0=53^0$ and $\angle PP'C=180^0-64^0-53^0=63^0$.

\item   \res{Given are the nonlinear points $P$, $Q$ and $R$.
Draw a triangle $ABC$, so
that $P$, $Q$ and $R$ are the centers of the squares that are constructed
over the sides $BC$, $CA$ and $AB$ of this triangle.}

We find a fixed point of the composite
$\mathcal{R}_{Q,90^0}\circ\mathcal{R}_{R,90^0}\circ\mathcal{R}_{P,90^0}$.

\item   \res{Let $A$ and $B$ be points and $p$ a line in the same
plane. Prove that
the composite $\mathcal{S}_B\circ\mathcal{S}_p\circ\mathcal{S}_A$
 is a central reflection exactly when $AB\perp p$.}

We represent the central reflections $\mathcal{S}_A$ and $\mathcal{S}_B$
as composites
$\mathcal{S}_A=\mathcal{S}_a\circ\mathcal{S}_{a_1}$ and
$\mathcal{S}_B=\mathcal{S}_{b_1}\circ\mathcal{S}_b$, where $a$ and $b$ are perpendicular to the line $p$.

\item   \res{Let $p$, $q$ and $r$ be tangents to the triangle $ABC$ of inscribed circles,
which are parallel to its sides
$BC$, $AC$ and $AB$. Prove that
the lines $p$, $q$, $r$, $BC$, $AC$ and $AB$ determine such
 a hexagon, in which the pairs of opposite sides are congruent distances.}

We use the central reflection $\mathcal{S}_S$, where $S$
is the center of the inscribed circle of the triangle $ABC$.

\item   \res{Draw a triangle with the data:  $\alpha$, $t_b$, $t_c$.}

First, we can draw the altitude $BB_1\cong t_b$ and the center
$T$. Because the vertex $A$ lies on the arc $l$ over the altitude $BB_1$ and
the obtuse angle $\alpha$, the vertex $B$ lies on the circle
$k(T,\frac{2}{3}t_c)$, and we have $C=\mathcal{S}_{B_1}(A)$, we get the vertex $C$ from the condition $C\in
k(T,\frac{2}{3}t_c)\cap\mathcal{S}_{l}(A)$.

\item \res{Let $ALKB$ and $ACPQ$ be squares that are drawn outside the triangle $ABC$
over the sides $AB$ and
$AC$, and $X$ the center of the side $BC$. Prove that
$AX\perp LQ$ and
$|AX|=\frac{1}{2}|QL|$.}

Let $Q'=\mathcal{S}_A(C)$. In this case it holds that
$R_{A,90^0}:Q,L\mapsto Q',B$. So $QL\cong Q'B$ and $QL\perp
Q'B$ (statement \ref{rotacPremPremKot}). Because $AX$ is the median of triangle $BCQ$ for the base $BQ'$, it follows that
$\overrightarrow{AX}=\frac{1}{2}\overrightarrow{BQ'}$, therefore $AX\perp LQ$ and $|AX|=\frac{1}{2}|QL|$.

\item \res{Let $O$ be the center of the regular triangle $ABC$ and $D$ and
$E$ points on the sides $CA$ and $CB$, so
 that $CD\cong CE$ holds. The point $F$ is the fourth vertex of the parallelogram
 $BODF$. Prove
that the triangle $OEF$ is regular.}

Let $\mathcal{I}=\mathcal{T}_{\overrightarrow{OB}}\circ
\mathcal{R}_{C,-60^0}$. By izrek \ref{izoKompTranslRot} it holds that
$\mathcal{I}$ is a rotation by the same angle $-60^0$ with the center in some
point $\widehat{O}$. So
$\mathcal{T}_{\overrightarrow{OB}}\circ
\mathcal{R}_{C,-60^0}=\mathcal{R}_{\widehat{O},-60^0}$. In this case
$\mathcal{R}_{\widehat{O},-60^0}(E)=\mathcal{T}_{\overrightarrow{OB}}\circ
\mathcal{R}_{C,-60^0}(E)=F$, therefore $\widehat{O}EF$ is a regular
triangle. It remains to be proven that $\widehat{O}=O$ or
that $O$ is a fixed point of the rotation
$\mathcal{R}_{\widehat{O},-60^0}$. If
$O'=\mathcal{R}_{C,-60^0}(O)$, then $O'$ is the center of the isosceles triangle $AB'C$ (where
$C'=\mathcal{S}_{AC}(B)$). So
$\overrightarrow{O'O}=\overrightarrow{OB}$ or
$\mathcal{T}_{\overrightarrow{OB}}(O')=O$. From this it follows that
$\mathcal{R}_{\widehat{O},-60^0}(O)=O$ or $\widehat{O}=O$, which
means that $OEF$
 is a regular triangle.

\item \res{Let $L$ be the point in which the inscribed circle of triangle
$ABC$ touches its side $BC$.
Prove: $$\mathcal{R}_{C,\measuredangle ACB}\circ\mathcal{R}_{A,\measuredangle BAC}
\circ\mathcal{R}_{B,\measuredangle CBA} =\mathcal{S}_L.$$}

Let $M$ and $N$ be the points where the inscribed circle of the
triangle $ABC$ touches its sides $AC$ and $AB$. By the statement in
\ref{rotacKomp2rotac}, $\mathcal{R}_{C,\measuredangle ACB}
\circ\mathcal{R}_{A,\measuredangle BAC}\circ
\mathcal{R}_{B,\measuredangle CBA}=\mathcal{S}_{\widehat{L}}$.
Since:
\begin{eqnarray*}
 \mathcal{S}_{\widehat{L}}(L)&=&
 \mathcal{R}_{C,\measuredangle
ACB} \circ\mathcal{R}_{A,\measuredangle BAC}\circ
\mathcal{R}_{B,\measuredangle CBA}(L)=\\
&=&
 \mathcal{R}_{C,\measuredangle
ACB} \circ\mathcal{R}_{A,\measuredangle BAC}(N)=
 \mathcal{R}_{C,\measuredangle ACB}(M)= L,
 \end{eqnarray*}
  it follows that $\widehat{L}=L$ or $\mathcal{R}_{C,\measuredangle ACB}
\circ\mathcal{R}_{A,\measuredangle BAC}\circ
\mathcal{R}_{B,\measuredangle CBA}=\mathcal{S}_L$.

\item \res{The points $P$ and $Q$ as well as $M$ and $N$ are the centers of two squares that are drawn outside of the opposite sides of any quadrilateral. Prove that $PQ\perp MN$ and $PQ\cong MN$.}

Let $ABCD$ be a given quadrilateral, $P$ and $Q$ the centers of the squares constructed on the sides $AB$ and $CD$, and $M$ and $N$ the centers of the squares constructed on the sides $BC$ and $AD$. The composite $\mathcal{I}=\mathcal{R}_{N,90^0}\circ \mathcal{R}_{P,90^0}$ according to the theorem \ref{rotacKomp2rotac} represents a central reflection. Because in this case $\mathcal{I}(B)=D$, the center of this reflection is actually the center of the diagonal $BD$; we denote it with $S$. So $\mathcal{R}_{N,90^0}\circ \mathcal{R}_{P,90^0}=\mathcal{S}_S$. For the point $S$ according to the same theorem \ref{rotacKomp2rotac} it holds that $\angle NPS=\frac{1}{2}90^0=45^0$ and $\angle PNS=\frac{1}{2}90^0=45^0$. This means that $PNS$ is an isosceles right-angled triangle with the base $NP$ and the right angle at the vertex $S$. Analogously, $MSQ$ is an isosceles right-angled triangle with the base $MQ$ and the right angle at the vertex $S$. From these two facts it follows that $R_{S,90^0}: M,N\mapsto Q,P$, so $MN\cong QP$ and $MN\perp QP$ (theorem \ref{rotacPremPremKot}).


\item  \res{Let $APB$ and $ACQ$ be two right-angled triangles outside the triangle $ABC$, constructed on the sides $AB$ and $AC$. The point $S$ is the center of the side $BC$ and $O$ is the center of the triangle $ACQ$. Prove that $|OP|=2|OS|$.}

Let $\mathcal{I}=\mathcal{R}_{O,120^0}\circ \mathcal{R}_{P,60^0}$. According to the theorem \ref{rotacKomp2rotac} we have:
 $$\mathcal{I}=\mathcal{R}_{O,120^0}\circ
\mathcal{R}_{P,60^0}=\mathcal{R}_{\widehat{S},180^0}
=\mathcal{S}_{\widehat{S}},$$ where $\widehat{S}$ is the vertex of the triangle $OP\widehat{S}$ and $\angle
 \widehat{S}PO=\frac{1}{2}60^0=30^0$ and $\angle
 PO\widehat{S}=\frac{1}{2}120^0=60^0$. Because $\mathcal{S}_{\widehat{S}}(S)=\mathcal{I}(S)=S$ or $\widehat{S}=S$.

 If we denote $O'=\mathcal{S}_S$, then $POO'$ is a right-angled triangle, so $|OP|=|OO'|=2|OS|$.

\item \res{Prove that the reflection in an axis and the translation of a plane commute if and only if the axis of this reflection is parallel to the translation vector.}

If we use the statement \ref{izoTransmutacija}, we get:
 \begin{eqnarray*}
\mathcal{T}_{\overrightarrow{v}}\circ \mathcal{S}_p=
\mathcal{S}_p\circ\mathcal{T}_{\overrightarrow{v}}
&\Leftrightarrow&
\mathcal{T}_{\overrightarrow{v}}\circ \mathcal{T}_p\circ\mathcal{T}^{-1}_{\overrightarrow{v}}=
 \mathcal{S}_p\\
&\Leftrightarrow&
\mathcal{S}_{\mathcal{T}_{\overrightarrow{v}}(p)}=
 \mathcal{S}_p
\Leftrightarrow
\mathcal{T}_{\overrightarrow{v}}(p)=
 p
\Leftrightarrow
\overrightarrow{v}\parallel
 p.
\end{eqnarray*}

\item  \res{In the same plane are given a line $p$, circles $k$ and $l$, and a distance $d$.
Draw a rhombus $ABCD$ with a side that is congruent to the distance $d$, the side $AB$ lies on
the line $p$, the vertices $C$ and $D$ are in turn on the circles $k$ and $l$.}

Use a translation for the vector $\overrightarrow{v}$, which is
parallel to the line $p$ and $|\overrightarrow{v}|=|d|$. In this
case, $D\in \mathcal{T}_{\overrightarrow{v}}(k)\cap l$.

\item  \res{Let $p$ be a line, $A$ and $B$ be points that lie on the same side
of the line $p$, and $d$ be a distance in the same plane.
Draw points $X$ and $Y$ on the line $p$ so that $AX\cong BY$ and $XY\cong d$.}

Let $\overrightarrow{v}$ be a vector that is parallel to the line
$p$ and $|\overrightarrow{v}|=|d|$. Let
$A'=\mathcal{T}_{\overrightarrow{v}}(A)$, $Y$ be the intersection
of the line $A'B$ with the line $p$, and
$X=\mathcal{T}^{-1}_{\overrightarrow{v}}(Y)$.

The point $Y$ lies on the line $A'B$, so $A'Y\cong YB$. Because
$\mathcal{T}_{\overrightarrow{v}}:A,X\mapsto A',Y$ is a parallelogram, $AYYA'$ is a parallelogram, so $AX\cong A'Y\cong BY$. From $\mathcal{T}_{\overrightarrow{v}}(X)=Y$
it follows that $\overrightarrow{XY}=\overrightarrow{v}$ or $|XY|=|\overrightarrow{v}|=|d|$.

\item  \res{Let $H$ be the orthocenter of the triangle $ABC$ and $R$ be the radius of the circumscribed circle of this triangle. Prove that $|AB|^2+|CH|^2=4R^2$.}

Let $O$ be the center of the circle $k(O,R)$ inscribed in the triangle $ABC$
and $A'=\mathcal{S}_O(A)$. The distance $AA'$ is the diameter of the circle $k$,
so $\angle ACA'=90^0$ or $A'C\perp AC$. Because
$BH\perp AC$, $A'C\parallel BH$. Similarly
$A'B\parallel CH$, which means that the quadrilateral $BA'CH$
is a parallelogram. Therefore $CH\cong A'B$. Because $A'BA$
is a right triangle, by the Pythagorean Theorem
\ref{PitagorovIzrek}:
$|AB|^2+|CH|^2=|AB|^2+|A'B|^2=|AA'|^2=4R^2$.

\item  \res{Let $EAB$ be a triangle that is drawn over the side $AB$ of the square
$ABCD$. Let also $M=pr_{\perp AE}(C)$ and $N=pr_{\perp BE}(D)$ and the point $P$
be the intersection of the lines $CM$ and $DN$. Prove that $PE\perp AB$.}

We use a translation for the vector $\overrightarrow{CB}$.

\item  \res{Draw an isosceles triangle $ABC$ so that its vertices in order
lie on three parallel lines $a$, $b$ and $c$ in the same plane,
the center of this triangle lies on the line $s$, which intersects
the lines $a$, $b$ and $c$ through the point $S_1$.}

    First, we draw any regular triangle $A_1B_1C_1$, where $A_1\in a$, $B_1\in b$ and $C_1\in c$,
    then we use a translation for the vector $\overrightarrow{S_1S}$, where $S_1$ is the center of the triangle
    $A_1B_1C_1$, the point $S$ is the intersection of the line $s$ with the parallel line through the point $S_1$.

\item \res{If a pentagon has at least two axes of symmetry, it is regular. Prove it.}

Let $\mathfrak{G}(\mathcal{V}_5)$ be the group of symmetries
    of our pentagon $\mathcal{V}_5$ and $p$ and $q$ its axes of symmetry. It is clear that this group is finite, so according to the theorem \ref{GrupaLeonardo} it represents either the cyclic group $\mathfrak{C}_n$ or the dihedral group $\mathfrak{D}_n$. Because it contains the axes of symmetry, $\mathfrak{G}(\mathcal{V}_5)=\mathfrak{D}_n$. In this case it is clear that $n\leq 5$. Because $\mathcal{S}_p, \mathcal{S}_q \in \mathfrak{G}(\mathcal{V}_5)=\mathfrak{D}_n$, the axes $p$ and $q$ intersect in some point $S$, the composition $\mathcal{S}_q\circ \mathcal{S}_p$ represents the rotation $\mathcal{R}_{S, \alpha}$. Because the group $\mathfrak{G}(\mathcal{V}_5)=\mathfrak{D}_n$ contains at least two axes of symmetry, $n\geq 2$. Therefore $n\in \{2,3,4,5\}$. In this case the basic rotation of this group (for the smallest angle) is $\mathcal{R}_{S, \frac{360^0}{n}}$. Because the pentagon has five vertices, the number $n$ is a divisor of the number 5, which means $n=5$. Therefore $\mathfrak{G}(\mathcal{V}_5)=\mathfrak{D}_5$, therefore $\mathcal{V}_5$ is a regular pentagon.

\item \res{Let $A$, $B$ and $C$ be three collinear points. What does the composition $\mathcal{G}_{\overrightarrow{BC}}\circ \mathcal{S}_A$ represent?}

We denote with $p$ the line passing through the points $A$, $B$ and $C$. Let $b$ and $a$ be the perpendiculars to the line $p$ in the points $A$ and $B$ and $s$ the symmetry of the line segment $BC$. Then:
 $$\mathcal{G}_{\overrightarrow{BC}}\circ \mathcal{S}_A=
 \mathcal{S}_s\circ \mathcal{S}_b\circ \mathcal{S}_p
 \circ \mathcal{S}_p\circ \mathcal{S}_a=
 \mathcal{S}_s\circ \mathcal{S}_b\circ \mathcal{S}_a.$$
 Because the lines $a$, $b$ and $s$ are perpendicular to the line $p$, they are from the group of parallel lines,
 therefore according to the theorem \ref{izoSop} the composition $\mathcal{S}_s
 \circ \mathcal{S}_b\circ \mathcal{S}_a$ (or $\mathcal{G}_{\overrightarrow{BC}}
 \circ \mathcal{S}_A$) is the axis of symmetry.

\item \res{Let $p$, $q$ and $r$ be lines that are not in the same plane, and let $A$ be a point in the same plane. Draw a line $s$ that goes through the point $A$ such that $\mathcal{S}_r\circ \mathcal{S}_q\circ \mathcal{S}_p(s)=s'$ and $s\parallel s'$.}

According to the theorem \ref{izoZrcdrsprq}, the composite $\mathcal{S}_r\circ \mathcal{S}_q\circ \mathcal{S}_p(s)=s'$ is a reflection - we will denote it with $\mathcal{G}_{2\overrightarrow{PQ}}$. The line $s$ is parallel or perpendicular to the axis of the reflection. This is determined by the centers of the lines $XX'$ and $YY'$, where $X$ and $Y$ are any points and $\mathcal{S}_r \circ \mathcal{S}_q\circ \mathcal{S}_p: X, Y\mapsto X', Y'$.

%new tasks
%___________________________________

\item \res{Let $Z$ and $K$ be inner points of the rectangle $ABCD$.
Draw points $A_1$, $B_1$, $C_1$ and $D_1$, which in order lie on
the sides $AB$, $BC$, $CD$ and $DA$ of this rectangle, so that it holds
$\angle ZA_1A\cong\angle B_1A_1B$, $\angle A_1B_1B\cong\angle C_1B_1C$,
$\angle B_1C_1C\cong\angle D_1C_1D$ and  $\angle C_1D_1D\cong\angle KD_1A$.}

    First, we draw points $Z'=S_{CB}\circ S_{AB}(Z)=S_B(Z)$ and $K'=S_{CD}\circ S_{AD}(Z)=S_D(K)$.
     Then we prove and use the fact that points $Z'$, $B_1$, $C_1$ and $K'$ are collinear.


\item \res{The point $A$ lies on the line $a$, and the point $B$ lies on the line $b$.
Determine the rotation that transforms the line $a$ into the line $b$ and the point $A$ into the point $B$.}

    If the lines $a$ and $b$ are parallel, the desired rotation is the central reflection with the center that is the center of the line $AB$. If the lines $a$ and $b$ intersect in the point $O$, the center of the rotation is the intersection of the simetral of the line $AB$ and the simetral of the angle $aOb$, and the rotation is equal to the angle $aOb$.

\item \res{In the center of the square, two rectangles intersect.
Prove that these rectangles intersect the sides of the square in the points that are
the vertices of a new square.}

We use the rotation $\mathcal{R}_{S,90^0}$, where $S$ is the center of the square.

\item \res{Given is a circle $k$ and lines $a$, $b$, $c$, $d$ and $e$, which lie
in the same plane. Draw a pentagon on the circle $k$ with sides, which
in order are parallel to the lines  $a$, $b$, $c$, $d$ and $e$.}


    First, we plan the simetrale $p$, $q$, $r$, $s$ and $t$ of the sides
    of the desired pentagon $ABCDE$, which are actually perpendicular to
    the lines  $a$, $b$, $c$, $d$ and $e$  from the center $O$
    of the circle $k$. The composition $\mathcal{I}= \mathcal{S}_t\circ
    \mathcal{S}_s\circ \mathcal{S}_r\circ \mathcal{S}_q\circ
    \mathcal{S}_p$ is an indirect isometry with fixed  points
    $O$ and $A$, so according to izreku \ref{izo1ftIndZrc}
    $\mathcal{I}=\mathcal{S}_{OA}$. We can plan the line $OA$ as
    the simetral of the line $XX'$, where $X$ is an arbitrary point and
    $X'=\mathcal{I}(X)$. This allows us to construct the vertex $A$,
    and then the other vertices of the pentagon $ABCDE$.

\item \res{The point $P$ lies inside the angle $aOb$. Draw a line $p$ through the point $P$,
 which with the sides $a$ and $b$ determines a triangle with the smallest area.}

We will prove that a triangle has the smallest area,
    if the line $OP$ is its median. We will mark this
    triangle with $OAB$ ($A\in a$ and $B\in b$). Let $X$ and
    $Y$ be the intersections of any line (different from $AB$) through
    the point $P$. We will prove that the area of the triangle $XOY$ is greater
    than the area of the triangle $AOB$. Without loss of generality, let
    $\mathcal{B}(O,X,A)$ or $\mathcal{B}(O,B,Y)$. In this case,
    we will mark points $A$ and $B$ in the following way. Let
    $a'=\mathcal{S}_P(a)$, $\{B\}=b\cap a'$ and
    $A=\mathcal{S}_P(B)$. We will mark $X'=\mathcal{S}_P$. It is clear that $X'\in a'$. Because $\mathcal{S}_P:X,A,P\mapsto X',B,P$, the
    triangles $XAP$ and $X'BP$ are congruent, therefore they have the same
    area. If we mark $p_{V_1V_2\cdots V_n}$ the area
    of any polygon $V_1V_2\cdots V_n$, it is:
    \begin{eqnarray*}
    p_{AOB}&=&p_{AXPB}+p_{XAP}=\\
    &=&p_{AXPB}+p_{X'BP}>p_{AXPB}+p_{X'BP}+
    p_{X'BY}=\\
    &=&p_{XOY}.
    \end{eqnarray*}

\item \res{The parallelogram $PQKL$ is inscribed in the parallelogram $ABCD$ (the vertices of the first one lie on the sides
of the second one). Prove that the parallelograms have a common center.}

    We will use the central reflection $\mathcal{S}_S$, where $S$ is the intersection of the diagonals of the parallelogram $ABCD$.

\item \res{The arcs $l_1, l_2,\cdots , l_n$ lie on the circle $k$ and the sum of their
lengths is less than the radius of this circle. Prove that there exists such a diameter
$PQ$ of the circle $k$, that none of its endpoints lies on any of
the arcs $l_1, l_2,\cdots , l_n$.}

Let $S$ be the center of the circle $k$. Let $\mathcal{S}_S:l_1, l_2,\cdots , l_n\rightarrow l'_1, l'_2,\cdots , l'_n$.
    Assume the contrary, that for each diameter $PQ$ one of its endpoints lies
    on one of the arcs $l_1, l_2,\cdots , l_n$. Because $\mathcal{S}_S(P)=Q$,
    both endpoints lie on some arc $l_1, l_2,\cdots , l_n$, $l'_1, l'_2,\cdots , l'_n$.
    This means that every point of the circle $k$ lies on one of the arcs $l_1, l_2,\cdots ,
    l_n$, $l'_1, l'_2,\cdots , l'_n$, which is not possible, since the total length of these arcs is less than the circumference of the circle $k$.

\item \res{The circle $K(S,20)$ is given. The players $\mathcal{A}$ and $\mathcal{B}$
alternately draw circles with radii $x_i$ ($1<x_i<2$), which lie in
the interior of the circle $K$, so that none of them has common points
with any of the previously drawn circles. The player who
draws the last circle wins. Does there exist a winning strategy for either of
the players $\mathcal{A}$ or $\mathcal{B}$?}

    Player $\mathcal{A}$ has a winning strategy. In the first
    move, he draws a circle with center $S$, and then to each
    move of player $\mathcal{B}$ - circle $K_i$ - responds with
    a move - circle $\mathcal{S}_S(K_i)$.

\item \res{Let $AB$ and $CD$ be the chords of the circle $k$, which have no common
points, and $P$ any point that lies on the chord $CD$. Draw such
a point $X$ on the circle $k$, that the chords $XA$ and $XB$ intersect the chord $CD$
in points $Y$ and $Z$ so that the point $P$ is the center of the segment $ZY$.}

Let's assume that $X$, $Y$ and $Z$ are points that satisfy the conditions of the task.
   From the condition $X\in k$ it follows that the angle $\angle CXD=\omega$ is known - the angle above the string
   $CD$ (statement \ref{ObodObodKot}). Let $C'=\mathcal{S}_P(C)$. The point $P$ is
   the common center of the lines $YZ$ and $CC'$, so the quadrilateral $YC'ZC$ is a parallelogram.
   Therefore, since $CX\parallel C'Z$, it follows that $\angle XZC'\cong \angle YXZ=\angle CXD=\omega$.
   This means that $C'ZD=180^0-\omega$, which allows us to construct the point $Z$ (statement \ref{ObodKotGMT}).

\item \res{Outside the parallelogram $ABCD$, similar triangles are constructed over its sides.
Prove that the centers of these triangles are the vertices of a new parallelogram.}

    Let $APB$, $BQC$, $CMD$ and $AND$ be given similar triangles and $S$
    the center (the intersection of the diagonals) of the parallelogram $ABCD$. Because
     $\mathcal{S}_S:A,B\mapsto C,D$, the reflection
     $\mathcal{S}_S$ maps the triangle $APB$ and $BQC$ to the triangle
     $CMD$ and $AND$
      or points $P$ and $Q$ to points $M$ and $N$.
      This means that the lines $PQ$ and $MN$ have a common center, so $PQMN$ is a parallelogram.

\item \res{Draw a trapezoid so that the bases are consistent with the given lines $a$ and
$c$, and the diagonals are consistent with the given lines $e$ and $f$.}

    If $ABCD$ is the desired trapezoid, first draw the triangle $DBC'$, where $C'=\mathcal{T}_{\overrightarrow{AB}}(C)$.

\item \res{The points (cities) $A$ and $B$ are on different banks of the river
(the strip determined by the parallels $p$ and $q$). It is necessary to build
a bridge (the line $PQ\perp p$, $P\in P$ and $Q\in q$) over the river that will connect
the cities $A$ and $B$, so that the path between the cities is the shortest ($|AP|+|PQ|+|QB|$ minimum).}

    Use the translation  $\mathcal{T}_{\overrightarrow{PQ}}$.

\item \res{In the plane, the lines $a$, $b$ and $p$ and the line $d$ are given.
Draw a parallel $q$ to the line $p$ that, with the lines $a$ and $b$, determines the line
that is consistent with the line $d$.}

Use the translation $\mathcal{T}_{\overrightarrow{v}}$,
    where $\overrightarrow{v}\parallel p$ and $|\overrightarrow{v}|=|d|$.

\item \res{The triangle $ABE$ is drawn outside the rectangle $ABCD$ on the side $AB$.
The rectangles of the lines $AE$ and $BE$ from the points $C$ and $D$ intersect in the point
$P$.  Prove that $PE\parallel BC$.}

    Use the translation $\mathcal{T}_{\overrightarrow{CB}}$ and prove
     that $\mathcal{T}_{\overrightarrow{CB}}(P)$ is the altitude point of the triangle $ABE$.

\item \res{The point $M$ lies inside the square $ABCD$. Prove that there exists
a quadrilateral with perpendicular diagonals and sides that are consistent with
the distances $MA$, $MB$, $MC$ and $MD$.}

    If $M_1=\mathcal{S}_{BC}(M)$ and
    $M_2=\mathcal{T}_{\overrightarrow{AB}}\circ\mathcal{S}_{AD}(M)$,
    then we prove that $BM_1CM_2$ is the desired quadrilateral.

\item \res{The congruent circles intersect in the points $P$ and $Q$. The line $l$ is
parallel to the line $m$, which passes through the centers of two circles, and $l$ intersects
the circles in succession in the points $A$ and $B$ and $C$ and $D$. Prove that the measure
of the angle $APC$ is independent of the choice of the line $l$.}

    If $P_1=\mathcal{T}_{\overrightarrow{AC}}(P)$, first prove $\angle APC=\angle PCP_1$.

\item \label{nalIzo75} \res{What does the composite $\mathcal{S}_A\circ \mathcal{S}_p$ represent?}

    The central reflection $\mathcal{S}_A$ is represented as
    the composition of the basic reflections:
    $\mathcal{S}_A=\mathcal{S}_n\circ\mathcal{S}_q$, where
    $q\parallel p$. In the case $A\in p$ the composite
    $\mathcal{S}_A\circ \mathcal{S}_p$ is the basic reflection, in
    the case $A\notin p$ it is the mirror glide (see also the statement
    \ref{izoZrcDrsKompSrOsn}).



\item \res{Let $t$ be the tangent of the circumscribed circle of the triangle $ABC$ in the vertex $A$.
Prove that:
$\mathcal{G}_{\overrightarrow{CA}} \circ \mathcal{G}_{\overrightarrow{BC}}
\circ \mathcal{G}_{\overrightarrow{AB}} =\mathcal{S}_t $.}

Let $p$, $q$ and $r$ be the altitudes of sides $BC$, $CA$ and $AB$, which intersect in the center $O$ of the circumscribed circle of triangle $ABC$. The composition $\mathcal{S}_q \circ
 \mathcal{S}_p  \circ \mathcal{S}_r$ is an indirect isometry with fixed
  points $O$ and $A$, so by \ref{izo1ftIndZrc}
 $\mathcal{S}_q \circ \mathcal{S}_p  \circ
 \mathcal{S}_r=\mathcal{S}_{OA}$. Therefore, (by \ref{izoZrcDrsKompSrOsn}):
\begin{eqnarray*}
\mathcal{I}&=&\mathcal{G}_{\overrightarrow{CA}} \circ
\mathcal{G}_{\overrightarrow{BC}} \circ \mathcal{G}_{\overrightarrow{AB}}
=
 \mathcal{S}_A \circ \mathcal{S}_q \circ
 \mathcal{S}_p \circ \mathcal{S}_B \circ
 \mathcal{S}_B \circ \mathcal{S}_r\\
 &=&
 \mathcal{S}_A \circ \mathcal{S}_q \circ
 \mathcal{S}_p  \circ \mathcal{S}_r
  =
 \mathcal{S}_A \circ \mathcal{S}_{OA}
 =
 \mathcal{S}_t\circ\mathcal{S}_{OA}\circ \mathcal{S}_{OA}
=\mathcal{S}_t.
\end{eqnarray*}



\item \res{What does the composition $\mathcal{S}_A\circ
\mathcal{S}_B\circ \mathcal{S}_{AB}$ represent?}

    By \ref{transl2sred} we have:
    $\mathcal{S}_A\circ \mathcal{S}_B\circ \mathcal{S}_{AB}=
    \mathcal{T}_{2\overrightarrow{BA}} \circ \mathcal{S}_{AB}=
    \mathcal{G}_{2\overrightarrow{BA}}$.

\item \res{Let $ABC$ be an equilateral triangle. Determine the axis and the vector
of the glide reflection, which is determined by the composition
$\mathcal{S}_{BC}\circ \mathcal{S}_{AB}\circ \mathcal{S}_{CA}$.}

    Let $A_1$ and $B_1$ be the centers of sides $BC$ and $AC$
    of triangle $ABC$. First, we prove that $\mathcal{S}_{BC}\circ
     \mathcal{S}_{AB}\circ \mathcal{S}_{CA}(A_1B_1)=A_1B_1$, then
     we find $\mathcal{S}_{BC}\circ \mathcal{S}_{AB}\circ \mathcal{S}_{CA}(B_1)$.

\item \res{In the same plane, given a convex heptagon $PQRSTUV$ and a circle $k$.
Draw a heptagon $ABCDEFG$, which is inscribed in the given circle,
and its sides are parallel to the sides of the given heptagon.}

Let $p$, $q$, $r$, $s$, $t$, $u$
    and $v$ be the altitudes of the heptagon $ABCDEFG$, which are
    actually the altitudes of the heptagon $PQRSTUV$
      with center $O$ on the circle $k$. The composite
    $\mathcal{I}=\mathcal{S}_v\circ \mathcal{S}_u\circ \mathcal{S}_t
    \circ \mathcal{S}_s\circ \mathcal{S}_r\circ \mathcal{S}_q\circ
    \mathcal{S}_p$ is an indirect isometry with fixed points $O$
    and $A$, so by \ref{izo1ftIndZrc} $\mathcal{I}=\mathcal{S}_{OA}$.
    The line $OA$ can be drawn as the altitude of the line $XX'$, where $X$
    is any point and $X'=\mathcal{I}(X)$. This allows us to construct
    the vertex $A$, and then the other vertices of the heptagon $ABCDEFG$.

    If the solution exists, we get it by the described procedure. If it does not exist, then
    we get a non-simple closed loop with individual lines
    parallel to the heptagon $PQRSTUV$.

\item \res{Prove that:
$\mathcal{S}_A\circ\mathcal{S}_B\circ\mathcal{S}_C=
\mathcal{S}_C\circ\mathcal{S}_B\circ\mathcal{S}_A$.}

    By \ref{izoKomp3SredZrc} the composite of three
    central reflections is a central reflection. Since it is also
    a central reflection involution (\ref{izoSredZrcInv}),
    it holds:
    $\mathcal{S}_A\circ\mathcal{S}_B\circ\mathcal{S}_C=\mathcal{S}_X=
    \mathcal{S}_X^{-1}=
    \left(\mathcal{S}_A\circ\mathcal{S}_B\circ\mathcal{S}_C\right)^{-1}=
    \mathcal{S}_C\circ\mathcal{S}_B\circ\mathcal{S}_A$.

\item \res{Let $A$, $B$ and $C$ be three non-linear points. Determine the point
$S$, for which it holds: $$\mathcal{S}_S\circ\mathcal{S}_A\circ
\mathcal{S}_S\circ\mathcal{S}_B\circ
\mathcal{S}_S\circ\mathcal{S}_C=\mathcal{E}.$$}
    According to the formulas \ref{transl2sred} and \ref{translKomp} we have:
    $\mathcal{E}=\mathcal{S}_S\circ\mathcal{S}_A\circ
\mathcal{S}_S\circ\mathcal{S}_B\circ
\mathcal{S}_S\circ\mathcal{S}_C=
\mathcal{T}_{\overrightarrow{AS}}\circ
\mathcal{T}_{\overrightarrow{BS}}\circ
\mathcal{T}_{\overrightarrow{CS}}=
 \mathcal{T}_{\overrightarrow{CS}+\overrightarrow{BS}+\overrightarrow{AS}}$.
  Therefore we have:
  $\overrightarrow{CS}+\overrightarrow{BS}+\overrightarrow{AS}
  =\overrightarrow{0}$ or
  $\overrightarrow{SC}+\overrightarrow{SB}+\overrightarrow{SA}
  =\overrightarrow{0}$. According to the formula \ref{tezTrikVektObr} the point
  $S$ is the centroid of the triangle $ABC$.

\item \res{If $S$ is the center of the line $AB$, then
$\mathcal{S}_S\circ\mathcal{S}_A\circ\mathcal{S}_S=\mathcal{S}_B$. Prove it.}

Use the formula about transmutation of the central reflection
\ref{izoTransmSredZrc}.

%olimp
%____________________________________________________

\item  \res{Let $ABC$ and $A'B'C'$ be two congruent
isosceles triangles. Prove that the centers of the lines $AA'$, $BB'$ and $CC'$
are either collinear points or the vertices of a new isosceles triangle\footnote{Proposal for MMO 1971. (SL 12.)}.}

    If the triangles are oriented the same way, the statement is a direct consequence of the example \ref{RotacZglVeck}. If the triangles are not oriented the same way, the statement is a direct consequence of the formula \ref{Chasles-Hjelmsleva}.

\item \res{The line $l$ goes through the altitude point of the triangle $ABC$. We denote with $l_a$, $l_b$ and $l_c$ the images of the line $l$ reflected over the lines $BC$, $AC$ and $BC$. Prove that the lines
  $l_a$, $l_b$ and $l_c$ intersect in a common point, which lies on the circumscribed circle of the triangle $ABC$\footnote{Proposal for MMO 1967. (LL 41.)}.}

Let $k$ be a circumscribed circle and $V$ be the altitude point
    of the triangle $ABC$. We denote with $V_a=\mathcal{S}_{BC}(V)$,
     $V_b=\mathcal{S}_{AC}(V)$ and $V_c=\mathcal{S}_{AB}(V)$.
     By \ref{TockaV'} the points $V_a$, $V_b$ and $V_c$
     lie on the circle $k$ and $\measuredangle V_aCV_b =2\measuredangle BCA=2\gamma$.

    By the assumption $\mathcal{S}_{BC}(l)=l_a$,
    $\mathcal{S}_{AC}(l)=l_b$ and $\mathcal{S}_{AB}(l)=l_c$,
     it follows that $V_a\in l_a$, $V_b\in l_b$ and $V_c\in l_c$.
     From $\mathcal{S}_{BC}(l)=l_a$ (or $\mathcal{S}_{BC}(l_a)=l$)
      and $\mathcal{S}_{AC}(l)=l_b$ it also follows (from \ref{rotacKom2Zrc})
      $\mathcal{R}_{C,2\gamma}(l_a)=
     \mathcal{S}_{AC}
    \circ\mathcal{S}_{BC}(l_a)=l_b$,
     so by \ref{rotacPremPremKot}  $\angle l_a, l_b=2\gamma$.

    We denote with $X$ the intersection of the lines $l_a$ and $l_b$
    (by the previous findings $l_a$ and $l_b$ are not parallel).
    Because $V_a\in l_a$ and $V_b\in l_b$, $V_aXV_b=\angle l_a,
     l_b=2\gamma=\measuredangle V_aCV_b$, which means that the point $X$
     lies on the circle $k$. So the lines $l_a$ and $l_b$ intersect
     on the circle $k$. The same goes for the lines $l_a$ and $l_c$ or $l_b$
     and $l_c$. Because $l_c\neq V_aV_b$,  also the line $l_c$  intersects the circle $k$ at the point $X$.

\item \res{Enakostranični trikotniki
 $BAM$, $DCP$, $BCN$ in $DAQ$ are konstruirani nad stranicami konveksnega
 štirikotnika $ABCD$. Prva dva trikotnika sta konstruirana zunaj, druga
 dva pa znotraj tega štirikotnika. Kaj lahko rečemo o štirikotniku $MNPQ$\footnote{Predlog za MMO 1982. (SL 20.)}?}

The isosceles triangles
 $BAM$, $DCP$, $BCN$ and $DAQ$ are constructed over the sides of the convex
 quadrilateral $ABCD$. The first two triangles are constructed outside, the other
 two are inside this quadrilateral. What can we say about the quadrilateral $MNPQ$\footnote{Predlog za MMO 1982. (SL 20.)}?

The line $AC$ is mapped by the rotation $R_{B, 60^0}$ to the line $MN$.
The same line $AC$ is mapped by the rotation $R_{D, 60^0}$ to the line $QP$.
Therefore, the lines $MN$ and $QP$ are congruent and parallel
(by statement \ref{rotacPremPremKot}). So the quadrilateral $MNPQ$ is a parallelogram.

Another way is to use the composition
$\mathcal{I}=\mathcal{R}_{Q,60^0}\circ
\mathcal{R}_{P,-60^0}\circ \mathcal{R}_{N,60^0}\circ
\mathcal{R}_{M,-60^0}$. The isometry $\mathcal{I}$ by
statement \ref{rotacKomp2rotac} represents the identity
or the translation. Because $\mathcal{I}(A)=A$, we have
$\mathcal{I}=\mathcal{E}$. Therefore $\mathcal{I}=
\mathcal{R}_{Q,60^0}\circ \mathcal{R}_{P,-60^0}=
\mathcal{R}^{-1}_{M,-60^0}\circ \mathcal{R}^{-1}_{N,60^0}$
or $\mathcal{I}=\mathcal{R}_{Q,60^0}\circ \mathcal{R}_{P,-60^0}=
\mathcal{R}_{M,60^0}\circ \mathcal{R}_{N,-60^0}$.
It is not difficult to prove (as a consequence of statement \ref{rotacKomp2rotac}),
that then $\overrightarrow{PQ}=\overrightarrow{NM}$, which means that
the quadrilateral $MNPQ$ is a parallelogram.


 \item \res{Let $\mathcal{R}_{D,90^0}\circ
 \mathcal{R}_{C,90^0}\circ\mathcal{R}_{B,90^0}\circ\mathcal{R}_{A,90^0}
 =\mathcal{E}$. Prove that $AC\perp BD$ and $AC\cong BD$.}

The equality $\mathcal{R}_{D,90^0}\circ
 \mathcal{R}_{C,90^0}\circ\mathcal{R}_{B,90^0}\circ\mathcal{R}_{A,90^0}
 =\mathcal{E}$ is equivalent to $\mathcal{R}_{D,90^0}\circ
 \mathcal{R}_{C,90^0}=\mathcal{R}_{A,-90^0}\circ\mathcal{R}_{B,-90^0}$
 or $\mathcal{S}_P=\mathcal{S}_Q$, where $CPD$ and $BQA$
 are congruent right-angled triangles with hypotenuses $CD$ and $AB$
 (by statement \ref{rotacKomp2rotac}). From $\mathcal{S}_P=\mathcal{S}_Q$
 it follows that $P=Q$, which means that $CPD$ and $BPA$ are congruent
 right-angled triangles with hypotenuses $CD$ and $AB$. Therefore
 $\mathcal{R}_{P,90^0}:D,B\mapsto C,A$, so $DB\perp CA$ and
 $DB\cong CA$.

\end{enumerate}


%REŠITVE -  Podobnost
%________________________________________________________________________________

\poglavje{Similarity}


\begin{enumerate}



\poglavje{The basics of Geometry} \label{osn9Geom}
The line $AC$ is mapped by the rotation $R_{B, 60^0}$ to the line $MN$.
The same line $AC$ is mapped by the rotation $R_{D, 60^0}$ to the line $QP$.
Therefore, the lines $MN$ and $QP$ are congruent and parallel
(by statement \ref{rotacPremPremKot}). So the quadrilateral $MNPQ$ is a parallelogram.

Another way is to use the composition
$\mathcal{I}=\mathcal{R}_{Q,60^0}\circ
\mathcal{R}_{P,-60^0}\circ \mathcal{R}_{N,60^0}\circ
\mathcal{R}_{M,-60^0}$. The isometry $\mathcal{I}$ by
statement \ref{rotacKomp2rotac} represents the identity
or the translation. Because $\mathcal{I}(A)=A$, we have
$\mathcal{I}=\mathcal{E}$. Therefore $\mathcal{I}=
\mathcal{R}_{Q,60^0}\circ \mathcal{R}_{P,-60^0}=
\mathcal{R}^{-1}_{M,-60^0}\circ \mathcal{R}^{-1}_{N,60^0}$
or $\mathcal{I}=\mathcal{R}_{Q,60^0}\circ \mathcal{R}_{P,-60^0}=
\mathcal{R}_{M,60^0}\circ \mathcal{R}_{N,-60^0}$.
It is not difficult to prove (as a consequence of statement \ref{rotacKomp2rotac}),
that then $\overrightarrow{PQ}=\overrightarrow{NM}$, which means that
the quadrilateral $MNPQ$ is a parallelogram.


 \item \res{Let $\mathcal{R}_{D,90^0}\circ
 \mathcal{R}_{C,90^0}\circ\mathcal{R}_{B,90^0}\circ\mathcal{R}_{A,90^0}
 =\mathcal{E}$. Prove that $AC\perp BD$ and $AC\cong BD$.}

The equality $\mathcal{R}_{D,90^0}\circ
 \mathcal{R}_{C,90^0}\circ\mathcal{R}_{B,90^0}\circ\mathcal{R

%Tales

\item \label{nalPod1}
\res{Let $S$ be the intersection of the diagonal $AC$ and $BD$ of the trapezoid $ABCD$. Let $P$ and $Q$ be the intersections of the parallels to the bases $AB$ and $CD$ through the point $S$ with the sides of this trapezoid. Prove that $S$ is the center of the segment $PQ$.}

Assume that $P\in AD$ and $Q\in BC$. By Tales' theorem \ref{TalesovIzrek} we have:
 \begin{eqnarray*}
 \frac{\overrightarrow{PS}}{\overrightarrow{AB}}
=\frac{\overrightarrow{DS}}{\overrightarrow{SB}}
=\frac{\overrightarrow{CS}}{\overrightarrow{SA}}
=\frac{\overrightarrow{SQ}}{\overrightarrow{AB}},
 \end{eqnarray*}
 therefore $\overrightarrow{PS}=\overrightarrow{SQ}$, that is, the point $S$ is the center of the segment $PS$.

The proven statement is a special case of the statement from the example \ref{vektTrapezZgled}.

\item
\res{Let $ABCD$ be a trapezoid with the base $AB$, the point $S$ be the intersection of its diagonals and $E$ be the intersection of the altitudes of this trapezoid. Prove that
the line $SE$ passes through the centers of the bases $AB$ and $CD$.}

Let $F$ and $G$ be the intersections of the line $SE$ with the lines $AB$ and $CD$ and $P$ and $Q$ be the points as in the previous task \ref{nalPod1}. By Tales' theorem \ref{TalesovIzrek} we have:
 \begin{eqnarray*}
 \frac{\overrightarrow{AF}}{\overrightarrow{PS}}
=\frac{\overrightarrow{EF}}{\overrightarrow{ES}}
=\frac{\overrightarrow{FB}}{\overrightarrow{SQ}}.
 \end{eqnarray*}
Because according to the previous task \ref{nalPod1} we have $\overrightarrow{PS}=\overrightarrow{SQ}$, we have
 $\overrightarrow{AF}=\overrightarrow{FB}$, that is, the point $F$ is the center of the side $AB$. Similarly, the point $G$ is the center of the side $CD$.

\item
\res{Let $P$, $Q$ and $R$ be points in which an arbitrary line through the point $A$ intersects
the sides $BC$ and $CD$ and the diagonal $BD$ of the parallelogram
$ABCD$. Prove that $|AR|^2=|PR|\cdot |QR|$.}

Because $AD\parallel BC$ and $AB\parallel CD$, by Tales' theorem \ref{TalesovIzrek} we have:
\begin{eqnarray*}
 \frac{AR}{PR}=\frac{DR}{BR}=\frac{QR}{AR},
 \end{eqnarray*}
which implies the desired relation.

\item
 \res{The points $D$ and $K$ lie on the sides $BC$ and $AC$ of the triangle $ABC$ so that $BD:DC=2:5$ and $AK:KC=3:2$. Calculate the ratio in which the line $BK$ divides the distance
$AD$.}

 Solution: $21:4$.

\item
\res{Let $P$ be a point on the side $AD$ of the parallelogram $ABCD$ such that $\overrightarrow{AP}=
\frac{1}{n}\overrightarrow{AD}$, and $Q$ the intersection of the lines $AC$ and $BP$. Prove that it holds:
 $$\overrightarrow{AQ}=\frac{1}{n + 1}\overrightarrow{AC}.$$}


Since $AD\parallel BC$ and $AB\parallel CD$, by  Tales' theorem  \ref{TalesovIzrek}:
\begin{eqnarray*}
 \frac{\overrightarrow{AQ}}{\overrightarrow{QC}}
=\frac{\overrightarrow{AP}}{\overrightarrow{BC}}=
\frac{\overrightarrow{AP}}{\overrightarrow{AD}}=\frac{1}{n}.
 \end{eqnarray*}
Therefore $\overrightarrow{AQ}=\frac{1}{n}\overrightarrow{QC}$, from which the desired relation follows.

%Homotetija

\item \label{nalPod6}
\res{Given: a point $A$, lines $p$ and $q$, and segments $m$ and $n$. Draw a line $s$ through the point $A$,
which intersects the lines $p$ and $q$ in such points $X$ and $Y$, that $XA:AY= m:n$.}

Assume that $s$ is the desired line and $\mathcal{B}(X,A,Y)$.
Let $h_{A,k}$ be the central stretch with center $A$ and coefficient $k=-\frac{|n|}{|m|}$. Then $Y=h_{A,k}(X)$. Because $X\in p$, it also holds that  $Y=h_{A,k}(X)\in h_{A,k}(p)$. So we get the point $Y$ from the condition $Y\in h_{A,k}(p)\cap q$. We get other solutions (for $\mathcal{B}(A,X,Y)$ or $\mathcal{B}(X,Y,A)$) if we choose different values for the coefficient $k$.

\item \res{Given: a point $S$, lines $p$, $q$ and $r$, and segments $m$ and $n$. Draw a line $s$ through the point $S$,
which intersects the lines $p$, $q$ and $r$ in such points  $X$, $Y$ and $Z$, that $XY:YZ=m:n$.}

We use the previous task (\ref{nalPod6}) and first plan the line $s'$, which passes through any point $Y'\in q$ and intersects the lines $p$ and $r$ in such points $X'$ and $Y'$, that $X'Y':Y'Z'=m:n$. As a consequence of Tales's theorem \ref{TalesPosl3}, the desired line $s$ is parallel to the line $s'$.

\item \res{In the given triangle $ABC$ draw such a rectangle $PQRS$, that the side $PQ$ lies on the side $BC$,
the points $R$ and $S$ lie on the sides $AB$ and $AC$, and it also holds that $PQ=2QR$.}

First, we plan any rectangle $P'Q'R'S'$, so that the side $P'Q'$ lies on the side $BC$,
the point $R'$ lies on the side $AB$, and it also holds that $P'Q'=2Q'R'$. Then we use the central extension with the center $B$. See example \ref{sredRaztegZgledKvadrat}.

\item
\res{Plan:}

(\textit{a}) \res{a rhombus, if the given side is $a$ and the ratio of the diagonals is $e:f$}.

Let $ABCD$ be a rhombus, where $AB\cong a$ and $\frac{AC}{BD}=\frac{e}{f}$. We mark with $S$ the intersection of the diagonals $AC$ and $BD$. $ASB$ is a right triangle with hypotenuse $AB\cong a$, and for its cathets it holds that $\frac{AS}{BS}=\frac{e}{f}$. First, we plan any right triangle $A'SB'$ with hypotenuse $A'B'$, where $\frac{A'S}{B'S}=\frac{e}{f}$. The triangle $ASB$ is the image of the triangle $A'SB'$ under the central extension $h_{S,\frac{a}{A'B'}}$. Then $C=\mathcal{S}_S(A)$ and $D=\mathcal{S}_S(B)$.

(\textit{b}) \res{a trapezoid, if the given: the interior angles $\alpha$ and $\beta$ at one of the bases, the ratio of this base and the height $a:v$, and the other base $c$}

Let $a:v=m:n$, where $m$ and $n$ are given lines. First, we plan a trapezoid $A'B'C'D$, where $\angle DA'B'\cong\alpha$, $\angle A'B'C'\cong\beta$,  $A'B'=m$, and the height is consistent with the line $n$. The desired trapezoid is the image of the trapezoid $A'B'C'D$ under the central extension $h_{D,\frac{c}{DC'}}$.

\item
\res{Let $A$ be a point inside the angle $pSq$, $l$ a line and $\alpha$ an angle in some plane. Draw a triangle
$APQ$, such that: $P\in p$, $Q\in q$, $\angle PAQ\cong \alpha$ and
$PQ\parallel l$.}

Let $P'$ and $Q'$ be the intersections of any parallel to the line $l$ with the sides $p$ and $q$ and $A'$ the intersection of the line segment $SA$ with the locus of points $X$, for which $\angle P'XQ'=\alpha$ (statement \ref{ObodKotGMT}). The triangle $APQ$ is then the image of the triangle $A'P'Q'$ under the central projection $h_{S,\frac{SP}{SP'}}$.

\item
\res{Draw a circle $l$, which touches a given circle $k$ and a given line $p$, if the point of contact is given:}

(\textit{a}) \res{$l$ and $p$}

Let $S$ be the center of the circle, which touches the line $p$ in the point $P$ and the circle $k(K,r)$ outside in the point $Q$. We mark with $P'$ such a point of the line $SP$, that $\mathcal{B}(S,P,P')$ and $PP'\cong KQ=r$ holds. Because $|SP'|=|SP|+|PP'|=|SQ|+|QK|=|SK|$, the triangle $SP'K$ is isosceles with the base $P'K$, therefore the point $S$ lies on the perpendicular bisector of the line $P'K$.
The proven facts allow for the construction. In our case the line $p$, the circle $k(K,r)$ and the point $P$ are given. First, we draw such a point $P'$, that: $P'P\perp p$, $P'P\cong r$ and $P', K\div p$, then the point $S$ as the intersection of the line $P'P$ with the perpendicular bisector $s_{P'K}$ of the line $P'K$ and finally the circle $l(S,SP)$.

We will now prove that $l(S,SP)$ is the desired circle. By the construction, $PP'\perp p$, so the line $p$ is tangent to the circle $l(S,SP)$ at point $P$. Let $Q_1$ be the intersection of the line $SK$ and the circle $k$. By the construction, point $S$ lies on the line $s_{P'K}$ of the distance $P'K$, so $SP'\cong SK$. From this and the construction it follows that $|SQ_1|=|SK|-|KQ_1|=|SP'|-r=|SP|+|PP'|-r=|SP|+r-r=|SP|$. This means that $Q_1\in l$. Because the points $S$, $Q_1$ and $K$ are collinear, the circles $l$ and $k$ touch at point $Q_1$ (actually, $Q_1=Q$).

The task has another solution in the general case, which we get if we choose $\mathcal{B}(S,P,P')$ in the construction.


(\textit{b}) \res{$l$ and $k$}

Let $S$ be the center of the circle that touches the line $p$ at point $P$ and the circle $k(K,r)$ outside at point $Q$. We mark with $S_1$ any point on the segment $KQ$ and $l_1(S_1,S_1P_1)$. The circle that touches the line $p$ at point $P_1$ and the line $KS_1$ at point $Q_1$. Let $O$ be the intersection of the lines $p$ and $KQ$.

First, we draw the circle $l_1$, then we use the central extension with center at point $O$ and $\frac{OQ}{OQ_1}$.

Another way is to directly plan the point $T$ from the condition
 $\angle OQP=\angle SQP=\frac{1}{2}\angle OSP=\frac{1}{2}\left( 90^0-\angle KOP \right)$.

We can also solve tasks (\textit{a}) and (\textit{b}) with the help of inversion (see section \ref{odd9ApolDotik}).

\item \label{nalPod12}
\res{Draw a circle that touches the sides of the angle $pSq$ and:}

(\textit{a}) \res{passes through the given point}

This is the Apollonius problem of the circle's tangency (see section \ref{odd9ApolDotik}), which we can solve with inversion.
We can also use the central extension with center at point $S$, so that we first plan any circle that touches the sides $p$ and $q$.

(\textit{b}) \res{touches the given circle}

First, we draw a concentric circle that goes through the center of the given circle and touches the lines $p'$ and $q'$, which are at the radius of the given circle and parallel to the sides $p$ and $q$.

\item
\res{Let there be: $p$ and $q$ lines, $S$ a point ($S\notin p$ and $S\notin q$), and the lines $m$ and $n$.
Draw the circles $k$ and $l$, which touch each other externally in the point $S$, the first one touches the line $p$,
the second line $q$, and the ratio of the radii is equal to $m:n$.}

With the central extension $h_{S,-\frac{n}{m}}$, the desired circle $k$ is mapped into the desired circle $l$, and its tangent $p$ is mapped into the tangent $p'$ of the circle $l$. Because we can draw the line $p'$, the problem is reduced to the construction of the circle $l$, which goes through the point $S$ and touches two lines: $q$ and $p'$. We use the previous task \ref{nalPod12}.

\item
\res{In the plane are given: the line $p$ and the points $B$ and $C$, which lie on the circle $k$. Draw such a point $A$
on the circle $k$, that the centroid of the triangle $ABC$ lies on the line $p$.}

We use the central extension $h_{A_1,2}$, where $A_1$ is the center of the line $BC$.

\item
\res{Let $k$ be a circle with the diameter $PQ$. Draw the square $ABCD$ so that $A,B\in PQ$ and $C,D\in k$.}

Draw an arbitrary square $A'B'C'D'$ so that $A',B'\in PQ$, and the center $S$ of the line $A'B'$ is also the center of the circle $k$. Then use the appropriate central extension with the center in the point $S$.

\item
\res{In the same plane are given: the lines $p$ and $q$, the point $A$, and the lines $m$ and $n$. Draw such a
rectangle $ABCD$, that $B\in p$, $D\in q$
and $AB:AD=m:n$.}

The rotational extension $\rho_{A,\frac{n}{m},90^0}$ maps the point $B$ into the point $D$, so $D\in q\cap \rho_{A,\frac{n}{m},90^0}(p)$.

\item \res{In the given triangle $ABC$ draw a triangle so that its sides are parallel to the given lines $p$, $q$ and $r$.}

First, we draw the triangle $X'Y'Z'$ so that we omit the condition $X'\in BC$. So let it only be: $Y'\in AC$, $Z'\in AB$, $Y'Z'\parallel p$, $X'Z'\parallel q$ and $X'Y'\parallel r$. Then we draw the point $X=AX'\cap BC$ and use the central extension $h_{A,\frac{\overrightarrow{AX}}{\overrightarrow{AX'}}}$, which maps the triangle $X'Y'Z'$ to the desired triangle $XYZ$.


\item \res{Let $p$, $q$ and $r$ be three lines of a plane. Draw a line $t$, which is
perpendicular to the line $p$ and the lines $p$, $q$ and $r$ intersect in such points $P$, $Q$ and $R$, that $PQ\cong QR$.}

If $p\parallel r$, the task is trivial.
Let $S=p\cap r$.

Assume that $t$ is a line, which is perpendicular to the line $p$ and intersects the lines $p$, $q$ and $r$ in such points $P$, $Q$ and $R$, that $PQ\cong QR$.
Let $h_{S,k}$ be a central extension with an arbitrary coefficient $k$ and
 $h_{S,k}:\hspace*{1mm} P,Q,R\mapsto P',Q',R'$. Then it holds: $R'\in r$, $P'\in p$, $P'R'\parallel t$. Because by assumption $Q$ is the center of the line $PR$ and the central extension $h_{S,k}$ is a similarity transformation, $Q'$ is also the center of the line $P'R'$. These ideas allow us to construct.

We draw an arbitrary point $P'$ on the line $p$, the point $R'$ as the intersection of the rectangle $t'$ on the line $p$ in the point $P'$ and the line $r$, the center  $Q'$ of the line $P'R'$, the point $Q$ as the intersection of the lines $SQ'$ and $q$ and the points $P$ and $R$ as the intersection of the rectangle $t$ on the line $p$ in the point $Q$ with the lines $p$ and $r$.

We will now prove that $t$ is the desired line.
By construction, $t$ is the perpendicular line of $p$ that intersects lines $p$, $q$, and $r$ in points $P$, $Q$, and $R$, respectively. We will now prove that $PQ\cong QR$. By construction, $t'\perp p$, so $t'\parallel t$ as well. By the consequence of Tales' theorem \ref{TalesPosl3}, we have: $\frac{\overrightarrow{PQ}}{\overrightarrow{QR}}=
\frac{\overrightarrow{P'Q'}}{\overrightarrow{Q'R'}}$. Because, by construction, point $Q'$ is the midpoint of line segment $P'R'$, we have $\frac{\overrightarrow{P'Q'}}{\overrightarrow{Q'R'}}=1$, so $\frac{\overrightarrow{PQ}}{\overrightarrow{QR}}=1$, which means that $PQ\cong QR$.

The task has only one solution precisely when $p\not\perp q$. In the case $p\perp q$, there is no solution.





%Ppodobnost ttrikotnikov

\item \res{Let $P$ be an inner point of triangle $ABC$ and $A_1$, $B_1$, and $C_1$ be the orthogonal projections of point $P$ onto the sides of triangle $BC$, $AC$, and $AB$, respectively. Analogously, points $A_2$, $B_2$, and $C_2$ are determined by point $P$ and triangle $A_1B_1C_1$, ..., points $A_{n+1}$, $B_{n+1}$, and $C_{n+1}$ are determined by point $P$ and triangle $A_nB_nC_n$, ... Which of the triangles $A_1B_1C_1$, $A_2B_2C_2$, ... are similar to triangle $ABC$?}

We denote:
\begin{eqnarray*}
\angle BAP= \alpha_1, \hspace*{3mm}  \angle PAC= \alpha_2,\\
\angle CBP= \beta_1, \hspace*{3mm}   \angle PBA= \beta_2,\\
\angle ACP= \gamma_1, \hspace*{3mm}  \angle PCB= \gamma_2.
\end{eqnarray*}

The quadrilaterals $AC_1PB_1$, $BA_1PC_1$, and $CB_1PA_1$ are taut (by theorem \ref{TetivniPogoj}), so:
\begin{eqnarray*}
\angle B_1A_1P= \gamma_1, \hspace*{3mm}  \angle PA_1C_1= \beta_2,\\
\angle C_1B_1P= \alpha_1, \hspace*{3mm}   \angle PB_1A_1= \gamma_2,\\
\angle A_1C_1P= \beta_1, \hspace*{3mm}  \angle PC_1B_1= \alpha_2.
\end{eqnarray*}

If we continue this process, we will obtain the inner angles $\angle A_n$, $\angle B_n$, and $\angle C_n$ of triangles $A_nB_nC_n$; $n\in \{0,1,2,\ldots \}$ (where $\triangle A_0B_0C_0=\triangle ABC$):

\vspace*{3mm}

\hspace*{12mm}
\begin{tabular}{|c|c|c|c|c|c|}
  \hline
  % after \\: \hline or \cline{col1-col2} \cline{col3-col4} ...
  $n$ & 0 & 1 & 2 & 3 & $\cdots$ \\
  \hline
  $\angle A_n$ & $\alpha_1+\alpha_2$ & $\gamma_1+\beta_2$ & $\beta_1+\gamma_2$ & $\alpha_1+\alpha_2$ & $\cdots$ \\
  \hline
  $\angle B_n$ & $\beta_1+\beta_2$ & $\alpha_1+\gamma_2$ & $\gamma_1+\alpha_2$ & $\beta_1+\beta_2$ & $\cdots$ \\
  \hline
  $\angle C_n$ & $\gamma_1+\gamma_2$ & $\beta_1+\alpha_2$ & $\alpha_1+\beta_2$ & $\gamma_1+\gamma_2$ & $\cdots$ \\
  \hline
\end{tabular}

\vspace*{3mm}

This means that the triangles $A_3B_3C_3$, $A_7B_7C_7$, ..., $A_{4n-1}B_{4n-1}C_{4n-1}$ ,... are similar to the triangle $ABC$.

\item
\res{Let $k$ be the inscribed circle of the quadrilateral $ABCD$, $E$ the intersection of its diagonals and $CB\cong CD$. Prove that $\triangle ABC \sim\triangle BEC$.}

We use: $\angle EBC=\angle DBC \cong \angle DAC\cong\angle BAC$.

\item
\res{Let $ABCD$ be a parallelogram. In the points $E$ and $F$ of the triangle $ABC$ the inscribed circle intersects the lines $AD$ and $CD$. Prove that $\triangle EBC\sim\triangle EFD$.}

Without loss of generality (in the other cases the proof is similar) assume that $\mathcal{B}(A,D,E)$ and $\mathcal{B}(C,D,F)$ hold.
Since $ABCD$ is a parallelogram, $ABCE$ is a trapezoid, by \ref{paralelogram} and \ref{TetivniPogoj} we have:
\begin{eqnarray*}
 \angle EDF\cong \angle ADC=180^0-\angle DAB=180^0-\angle EAB=\angle ECB.
\end{eqnarray*}
By \ref{ObodObodKot} we also have:
\begin{eqnarray*}
 \angle EFD\cong \angle EFC\cong\angle EBC.
\end{eqnarray*}
From \ref{PodTrikKKK} it follows that $\triangle EFD\sim\triangle EBC$.


\item
 \res{Let $AA'$ and $BB'$ be the altitudes of the acute angled triangle $ABC$. Prove that
$\triangle ABC\sim\triangle A'B'C$.}

First we prove that $AB'A'B$ is a trapezoid, then
 $\angle A'B'C\cong\angle ABC$.

\item \label{nalPod23}
\res{Let the altitude $AD$ of the triangle $ABC$ be the tangent of the circumscribed circle of this triangle at the same time. Prove that $|AD|^2=|BD|\cdot |CD|$.}

Because according to the theorem \ref{ObodKotTang} $\angle DAB\cong\angle BCA=\angle DCA$, the right-angled triangles
$DAB$ and $DCA$ are similar (theorem \ref{PodTrikKKK}). From this it follows
$\frac{AD}{CD}=\frac{BD}{AD}$ or $|AD|^2=|BD|\cdot |CD|$.

The statement is also a direct consequence of the theorem about the power of a point with respect to a circle \ref{izrekPotenca}.

\item
\res{In the triangle $ABC$, let the internal angle at the vertex $A$ be twice as
large as the internal angle at the vertex $B$. Prove that $|BC|^2= |AC|^2+|AC|\cdot |AB|$.}

Let $D$ be the point in which the internal angle $AC$ of the triangle $ABC$ is bisected by its side $BC$. According to the assumption, $\angle BAD\cong\angle DAC\cong\angle CBA$. Therefore, the triangle $DAB$ is isosceles with the base $AB$, so $AD\cong BD$.
From the condition $\angle DAC\cong\angle CBA$ it follows from the theorem \ref{PodTrikKKK} that $\triangle CAD\sim \triangle CBA$. From this similarity and the proven $AD\cong BD$ we get:
 \begin{eqnarray*}
\frac{AC}{BC}=\frac{CD}{CA}
 \hspace*{2mm} \textrm{ and }  \hspace*{2mm}
 \frac{AC}{BC}=\frac{AD}{BA}=\frac{BD}{BA}
\end{eqnarray*}
or:
\begin{eqnarray*}
|CD|\cdot |BC|=|AC|^2
 \hspace*{2mm} \textrm{ and }  \hspace*{2mm}
 |BC|\cdot |BD|=|AC|\cdot |AB|.
\end{eqnarray*}
 If we add the obtained equality and use $|CD|+|BD|=|BC|$ (because $\mathcal{B}(C,D,B)$), we get the desired relation.

\item
\res{Prove that the radii of the circumscribed circles of two similar triangles are proportional to the corresponding sides of these two triangles.}

We use the similarity transformation $f$, which maps the first triangle into the second.


 \item \res{In the circle with center $S$, the quadrilateral $ABCD$ is inscribed. The diagonals of this quadrilateral are perpendicular and intersect at the point $E$. The line passing through the point $E$ and perpendicular to the side $AD$ intersects the side $BC$ at the point $M$.}


(\textit{a}) \res{Prove that the point $M$ is the center of the segment $BC$.}

See example \ref{TetivniLemaBrahm}.

(\textit{b}) \res{Determine the set of all points $M$, if the diagonal $BD$ changes and is always perpendicular to the diagonal $AC$.}

We mark with $k$ the circumscribed circle of the quadrilateral $ABCD$. Because $h_{C,\frac{1}{2}}(B)=M$ and $B\in k$, $M\in k'=h_{C,\frac{1}{2}}(k)$. The converse is also true - for a point $M\in k'\setminus \{ C\}$, $B=h_{C,\frac{1}{2}}^{-1}(M)\in k$. The sought set of all points $M$ is therefore $M\in k'\setminus \{ C\}$. In this case, $k'=h_{C,\frac{1}{2}}(k)$ is actually a circle with a diameter of $CS$ (statement \ref{RaztKroznKrozn}).


\item
\res{Let $t$ be the tangent of the circumscribed circle $l$ of the triangle $ABC$ at the vertex $A$.
Let $D$ be a point on the line $AC$ such that $BD\parallel t$. Prove that
$|AB|^2=|AC|\cdot |AD|$.}

We use the statement \ref{ObodKotTang} and \ref{KotiTransverzala} and prove that $\triangle ABD \sim \triangle ACB$.

 \item
 \res{The altitude point of an obtuse triangle divides its altitude in the same ratio (from the vertex to the altitude point). Prove that it is an isosceles triangle.}

 Let $BB'$ and $CC'$ be altitudes, $V$ the altitude point of the triangle $ABC$, for which $\frac{BV}{VB'}=\frac{CV}{VC'}$ or $\frac{BV}{CV}=\frac{VB'}{VC'}$. Because $\angle C'VB\cong \angle B'VC$, the triangles $C'VB$ and $B'VC$ are similar according to the statement \ref{PodTrikKKK}. It follows that $\frac{BV}{CV}=\frac{VC'}{VB'}$. From the previous two relations it follows that $\frac{BV}{CV}=\frac{VC'}{VB'}=\frac{VB'}{VC'}$ or $VB'\cong VC'$ and $VB\cong VC$ and finally $BB'\cong CC'$. From the similarity of the right triangles $ABB'$ and $ACC'$ (statement \textit{ASA} \ref{KSK}) it follows that $AB\cong AC$, which means that $ABC$ is an isosceles triangle.

\item
 \res{In the triangle $ABC$, the altitude $BD$ touches the circumscribed circle of this triangle.
Prove:}

(\textit{a}) \res{that the difference of the angles at the base $AC$ is equal to $90^0$}

(\textit{b}) \res{that $|BD|^2=|AD|\cdot |CD|$}

See task \ref{nalPod23}.

\item
\res{The circle with the center at the base $BC$ of the isosceles triangle $ABC$ touches
the sides $AB$ and $AC$. The points $P$ and $Q$ are the intersections of these sides with any
tangent to this circle. Prove that $4\cdot |PB|\cdot |CQ|=|BC|^2$.}

We prove $\triangle BDP\sim\triangle CQD$.


\item
\res{Let $V$ be the altitude point of the right angled triangle $ABC$, the point $V$ is the center of the altitude $AD$, and the point $V$ divides the altitude $BE$ in the ratio $3:2$. Calculate the ratio in which $V$ divides the altitude $CF$.}

The solution is $10:1$.

\item
\res{Let $S$ be an external point of the circle $k$. $P$ and $Q$
are the points in which the circle $k$ touches its tangents from the point $S$, $X$ and $Y$ are the intersections of this circle with any line passing through the point $S$. Prove that $XP:YP=XQ:YQ$.}

We use the similarity of the triangles $XPS$ and $PYS$ and the triangles $XQS$ and $QYS$ and prove
 $XP:YP=SX:SP=SX:SQ=XQ:YQ$.

\item
\res{Let $D$ be a point lying on the side $BC$ of the triangle $ABC$. The points $S_1$ and $S_2$ are the centers of the circles drawn through the triangle $ABD$ and $ACD$. Prove that
 $\triangle ABC\sim\triangle AS_1S_2$.}

The distance $AD$ is the common chord of the circles with the centers $S_1$ and $S_2$, therefore the line $S_1S_2$ is the perpendicular bisector of the angle $AS_1S_2$. If we use the statement \ref{SredObodKot}, we get:
 $\angle AS_1S_2=\frac{1}{2}\angle AS_1D=\angle ABD=\angle ABC$ or $\angle AS_1S_2\cong\angle ABC$. Similarly, $\angle AS_2S_1\cong\angle ACB$. By the statement \ref{PodTrikKKK}, we have  $\triangle ABC\sim\triangle AS_1S_2$.

\item
\res{The point $P$ lies on the hypotenuse $BC$ of the triangle  $ABC$. The perpendicular to the line $BC$ at the point $P$ intersects the line $AC$ and $AB$ at the points $Q$ and $R$ and the circle drawn through the triangle $ABC$ at the point $S$. Prove that $|PS|^2=|PQ|\cdot |PR|$.}

We prove $\triangle QPC\sim\triangle BPR$, then we use the statement \ref{izrekVisinski}

\item
\res{Point $A$ lies on the arm $OP$ of the right angle $POQ$. Let
$B$, $C$ and $D$ be such points of the arm $OQ$, that $\mathcal{B}(O,B,C)$, $\mathcal{B}(B,C,D)$ and
$OA\cong OB\cong BC\cong CD$. Prove that also $\triangle ABC\sim\triangle DBA$.}

We mark $|OA|=a$. According to the assumption, $|OB|=|BC|=|CD|=a$, according to Pythagoras's theorem \ref{PitagorovIzrek} $|AB|=a\sqrt{2}$. Therefore $\frac{AB}{DB}=\frac{a\sqrt{2}}{2a}=\frac{\sqrt{2}}{2}$ and $\frac{BC}{BA}=\frac{a}{a\sqrt{2}}=\frac{1}{\sqrt{2}}=\frac{\sqrt{2}}{2}$ or $\frac{AB}{DB}=\frac{BC}{BA}$. Since $\angle ABC\cong\angle DBA$, according to the theorem  \ref{PodTrikSKS} $\triangle ABC\sim\triangle DBA$.

\item \res{Draw a triangle, if the following data are known:}

(\textit{a}) \res{$\alpha$, $\beta$, $R+r$}

We draw an arbitrary similar triangle $A'B'C'$, then we use $a:a'=(R+r):(R+r')$ and Tales's theorem \ref{TalesovIzrekDolzine}.

 (\textit{b}) \res{$a$, $b:c$, $t_c-v_c$}

We draw an arbitrary similar triangle $A'B'C'$, then we use $a:a'=(t_c-v_c):(t'_c-v'_c)$ and Tales's theorem \ref{TalesovIzrekDolzine}.

 (\textit{c}) \res{$v_a$, $v_b$, $v_c$}

We use $2\cdot p_{\triangle}=a\cdot v_a=b\cdot v_b=c\cdot v_c$ (theorem \ref{PloscTrik}). From this, it follows that $a=\frac{2\cdot p_{\triangle}}{v_a}$, $b=\frac{2\cdot p_{\triangle}}{v_b}$ and $c=\frac{2\cdot p_{\triangle}}{v_c}$ or:
$$a:b:c=\frac{1}{v_a}=\frac{1}{v_b}=\frac{1}{v_c}.$$
In this way, we can first draw an arbitrary triangle $A'B'C'$ with sides $a'=\frac{x^2}{v_a}$, $b'=\frac{x^2}{b_a}$ and $c'=\frac{x^2}{v_c}$ for an arbitrary distance $x$ (we use Tales's theorem \ref{TalesovIzrekDolzine}).

\item
\res{Let $AB$ and $CD$ be the bases of the isosceles trapezoid $ABCD$, $r$ be the radius of the inscribed circle. Prove that $|AB|\cdot |CD|=4r^2$.}

Let $T$ be the point of intersection of the inscribed circle $k(S,r)$ of the trapezoid $ABCD$ with its leg $BC$. Because $BS$ and $CS$ are the altitudes of the trapezoid $ABCD$, by the theorem \ref{KotiTransverzala}
 $\angle SBC+\angle SCB=\frac{1}{2}\left( \angle ABC+\angle BCD\right)=\frac{1}{2}\cdot 189^0=90^0$. From this it follows that $\angle BSC=90^0$ (theorem \ref{VsotKotTrik}).
 Therefore, $ST$ is the altitude of the right triangle $CSB$ with hypotenuse $CB$, so by the theorem \ref{izrekVisinski}
\begin{eqnarray} \label{eqnPodNal37}
|ST|^2= |CT|\cdot |TB|.
 \end{eqnarray}
Let $E$ and $F$ be the midpoints of the bases  $AB$ and $CD$. Because it is an isosceles trapezoid, the line $EF$ is the median of this trapezoid and $E$ and $F$ are the points of intersection of the bases $AB$ and $CD$ with the inscribed circle $k$ of this trapezoid. If we use \ref{eqnPodNal37} and the theorem \ref{TangOdsek}, we get:
\begin{eqnarray*}
4r^2=4\cdot |ST|^2= 4\cdot |CT|\cdot |TB|=4\cdot |CF|\cdot |BE|=|AB|\cdot |CD|.
 \end{eqnarray*}


%Harmon cetverica

\item \res{The circle $k$ and the points $A$ and $B$ are given. Draw a point $X$ on the circle $k$ such that $AX:XB=2:5$.}

We use the Apollonius circle or the theorem \ref{ApolonijevaKroznica}.

\item \res{Draw a triangle with the given data:}\\
(\textit{a}) \res{$a$, $v_a$, $b:c$,} \hspace*{3mm}
 (\textit{b}) \res{$a$, $t_a$, $b:c$,}\hspace*{3mm}
 (\textit{c}) \res{$a$, $b$, $b:c$,}\\
 (\textit{d}) \res{$a$, $\alpha$, $b:c$,}\hspace*{3mm}
  (\textit{e}) \res{$a$, $l_a$, $b:c$.}

In all cases we use the theorem \ref{HarmCetSimKota} and the Apollonius circle $\mathcal{A}_{BC,c:b}$ - theorem \ref{ApolonijevaKroznica}.

\item \res{Draw a triangle, if the following data are given:}

(In all cases we use the notation and results of the big task \ref{velikaNaloga} and its consequences \ref{harmVelNal}.)

(\textit{a}) \res{$v_a$, $r$, $\alpha$}

We use $\mathcal{H}(A,A';L,L_a)$ and thus first draw $r_a$.

(\textit{b}) \res{$v_a$, $r_a$, $a$}

We use $\mathcal{H}(A,A';L,L_a)$ and $RR_a=a$.

(\textit{c}) \res{$v_a$, $t_a$, $b-c$}

We use $\mathcal{H}(A',E;P,P_a)$, $PP_a=b-c$ and the fact that $A_1$ is the center of the line $PPa$.


\item \res{Draw a parallelogram, where one side and the corresponding height are consistent with the given distances $a$ and $v_a$, and the diagonals are in the ratio $3:5$.}

First, we draw the side $AB\cong a$ of the parallelogram $ABCD$, then its center $S$ as the intersection of Apollonius's circle $\mathcal{A}_{AB,3:5}$ (statement \ref{ApolonijevaKroznica}) and the parallel line $AB$ at a distance $\frac{1}{2}v$.

\item \res{Let point $E$ be the intersection of the internal angle bisector of triangle $ABC$ with its side $BC$. Prove that it holds:
    $$\overrightarrow{AE}=\frac{|AC|}{|AB|+|AC|}\cdot\overrightarrow{AB}+
    \frac{|AB|}{|AB|+|AC|}\cdot\overrightarrow{AC}.$$}

We use example \ref{vektDelitDaljice} and statement \ref{HarmCetSimKota}.

\item
\res{Given are four collinear points, for which it holds $\mathcal{H}(A,B;C,D)$. Draw a point $L$, from which the lines $AC$, $CB$ and $BD$ are seen under the same angle.}

Let $L$ be a point lying on the intersection of the circles $k_{AB}$ and $k_{CD}$, which are drawn over the diameters $AB$ and $CD$. We prove that $\angle ALC\cong\angle CLB\cong\angle BLD$.

Circle $k_{CD}$ is actually Apollonius's circle $\mathcal{A}_{AB,\frac{AC}{CB}}$ (statement \ref{ApolonijevaKroznica}), so for a point $L\in k_{CD}$ it holds $\frac{LA}{LB}=\frac{CA}{CB}$. From this it follows (statement \ref{HarmCetSimKota}) that $LC$ is the bisector of angle $ALB$, thus $\angle ALC\cong\angle CLB$. Similarly, from $L\in k_{CD}$ it follows
$\angle CLB\cong\angle BLD$, which means that $L$ is the desired point.

\item \label{nalPod44}
\res{Let $AE$ ($E\in BC$) be the bisector of the internal angle of triangle $ABC$ and $a=|BC|$, $b=|AC|$ and  $c=|AB|$. Prove that it holds:
$$|BE|=\frac{ac}{b+c} \hspace*{1mm} \textrm{ and } \hspace*{1mm}  |CE|=\frac{ab}{b+c}.$$}

According to the statement \ref{HarmCetSimKota} it is $\frac{BE}{EC}=\frac{BA}{AC}=\frac{c}{b}$, therefore:
 $$|BE|=|BC|\cdot\frac{BE}{BC}=a\cdot\frac{c}{b+c}=\frac{ac}{b+c}.$$
Similarly, $|CE|=\frac{ab}{b+c}$.

\item \res{Let $AE$ ($E\in BC$) and $BF$ ($F\in AC$) be the simetrals of the inner angles and $S$ the center of the inscribed circle of the triangle $ABC$. Prove that $ABC$ is an isosceles triangle (with the base $AB$) exactly when $AS:SE=BS:SF$.}


We assume that $AS:SE=BS:SF$ is true.
The line $BS$ is also the simetral of the inner angle $ABE$ of the triangle $ABE$, therefore according to the statement \ref{HarmCetSimKota} $\frac{AS}{SE}=\frac{AB}{BE}$. Because $AS$ is also the simetral of the inner angle $BAF$ of the triangle $BAF$, it is also $\frac{BS}{SF}=\frac{BA}{AF}$. Because according to the assumption $\frac{AS}{SE}=\frac{BS}{SF}$, it is $BE\cong AF$. According to the previous task \res{nalPod44} it is:
$|BE|=\frac{ac}{b+c}$ and $|AF|=\frac{bc}{a+c}$. Therefore $\frac{ac}{b+c}=\frac{bc}{a+c}$ or $a(a+c)=b(b+c)$. The last equality is equivalent to $a^2-b^2+ac-bc=0$ or
 $(a-b)(a+b+c)=0$. Because $a+b+c\neq 0$, it is $a=b$, which means that $ABC$ is an isosceles triangle.

If $ABC$ is an isosceles triangle, the relation $AS:SE=BS:SF$ is trivially fulfilled.


%Menelaj Ceva


\item \res{Prove that the simetrals of the outer angles of any triangle intersect the lines of the opposite sides in three collinear points.}

We use Menelaj's statement \ref{izrekMenelaj}.

%Ppitagorov iizrek

\item \res{If $a$, $b$ and $c$ ($a>b$) are given lengths, draw such a length $x$, that it is true:}\\
(\textit{a}) \res{$x=\sqrt{a^2+b^2}$,} \hspace*{3mm}
(\textit{b}) \res{$x=\sqrt{a^2-b^2}$,} \hspace*{3mm}
(\textit{c}) \res{$x=\sqrt{3ab}$,}\\
(\textit{d}) \res{$x=\sqrt{a^2+bc}$,} \hspace*{3mm}
(\textit{e}) \res{$x=\sqrt{3ab-c^2}$,} \hspace*{3mm}
(\textit{f}) \res{$x=\frac{a\sqrt{ab+c^2}}{b+c}$.}

Use the Pythagorean Theorem \ref{PitagorovIzrek} and the altitude theorem \ref{izrekVisinski}.

%Stewart's theorem

\item
\res{Let $a$, $b$ and $c$ be the sides of a triangle and let $a^2+b^2=5c^2$. Prove that the centroids
 $t_a$ and $t_b$ are perpendicular to each other.}

Let $T$ be the centroid of the given triangle $ABC$, and $t_a$ and $t_b$ be the lengths of the corresponding centroids. We prove that $ATB$ is a right triangle.
By \ref{StwartTezisc} of Stewart's theorem \ref{StewartIzrek}, we have:
 \begin{eqnarray*}
t_a^2=\frac{b^2}{2}+\frac{c^2}{2}-\frac{a^2}{4}\\
t_b^2=\frac{a^2}{2}+\frac{c^2}{2}-\frac{b^2}{4}.
 \end{eqnarray*}
From this and from the assumption $a^2+b^2=5c^2$, we get:
\begin{eqnarray*}
|AT|^2+|BT|^2
 &=&
\left(\frac{2}{3}t_a\right)^2+\left(\frac{2}{3}t_b\right)^2=\\
 &=&
\frac{4}{9}\left(\frac{b^2}{2}+\frac{c^2}{2}-\frac{a^2}{4}
+\frac{a^2}{2}+\frac{c^2}{2}-\frac{b^2}{4}\right)=\\
 &=&
\frac{4}{9}\left(\frac{a^2+b^2}{4}+c^2\right)=\\
 &=& c^2=|AB|^2.
 \end{eqnarray*}
By the inverse Pythagorean theorem \ref{PitagorovIzrekObrat}, $ATB$ is a right triangle with hypotenuse $AB$, which means that the corresponding centroids are perpendicular.

\item
\res{Let $a$, $b$, $c$ and $d$ be the sides, $e$ and $f$ the diagonals, and $x$ the distance determined by the centers
 of sides $b$ and $d$ of a quadrilateral. Prove:
$$x^2 = \frac{1}{4} \left(a^2 +c^2 -b^2 -d^2 +e^2 +f^2 \right).$$}

We use the consequence \ref{StwartTezisc} of Stewart's theorem \ref{StewartIzrek}.

\item
\res{Let $a$, $b$ and $c$ be the sides of a triangle $ABC$. Prove that the distance from the center $A_1$ of side $a$ to the vertex $A'$ of the altitude on
this side is equal to:
$$|A_1A'|=\frac{|b^2-c^2|}{2a}.$$}

Without loss of generality, we assume that $b\geq c$.
We denote $x=|A_1A'|$, $v_a=|AA'|$ and $t_a=|AA_1|$. If we use the Pythagorean Theorem \ref{PitagorovIzrek} for the triangle $AA'A_1$ and $AA'B$, we get:
 \begin{eqnarray*}
 v_a^2 &=& t_a^2-x^2;\\
 v_a^2 &=& c^2- \left( \frac{a}{2} -x \right)^2.
 \end{eqnarray*}
 After subtracting the equations and solving the obtained equation for $x$, we get:
  \begin{eqnarray*}
x=\frac{1}{a}\left( t_a^2-c^2+ \frac{a^2}{2} \right).
 \end{eqnarray*}
Finally, we use the relation for $t_a^2$ from the Stewart Theorem \ref{StwartTezisc}:
 \begin{eqnarray*}
x &=& \frac{1}{a}\left( t_a^2-c^2+ \frac{a^2}{4} \right)=\\
 &=& \frac{1}{a}\left( \frac{b^2}{2}+\frac{c^2}{2}-\frac{a^2}{4}-c^2+ \frac{a^2}{4} \right)=\\
 &=& \frac{b^2-c^2}{2a}.
 \end{eqnarray*}

%Pappus and Pascal

\item
\res{Let ($A$, $B$, $C$) and ($A_1$, $B_1$, $C_1$) be two collinear points of a plane that are not on the same line. If $AB_1\parallel A_1B$ and $AC_1\parallel A_1C$, then $CB_1\parallel C_1B$. (\textit{Pappus' Theorem}\footnote{Pappus of Alexandria\index{Pappus} (3rd century BC), Greek mathematician. This is a generalization of Pappus' Theorem (see Theorem \ref{izrek Pappus}), if we choose points $X$, $Y$ and $Z$ at infinity.})}

We use the Tales Theorem \ref{TalesovIzrek} and the converse Tales Theorem \ref{TalesovIzrekObr}.

%Desargues' Theorem


\item \label{nalPodDesarg1}
\res{Let $P$, $Q$ and $R$ be such points of sides $BC$, $AC$ and $AB$
of the triangle $ABC$, that the lines $AP$, $BQ$ and
$CR$ are from the same pencil. Prove: If $X=BC\cap QR$, $Y=AC\cap PR$ and $Z=AB\cap PQ$, then points
$X$, $Y$ and $Z$
are collinear.}

We use the Desargues Theorem \ref{izrekDesarguesEvkl} for the triangle $ABC$ and $PQR$.

\item
\res{Let $AA'$, $BB'$ and $CC'$ be the altitudes of the triangle $ABC$ and $X=B'C'\cap BC$,  $Y=A'C'\cap AC$ and $Z=A'B'\cap AB$. Prove that $X$, $Y$ and $Z$ are collinear points.}

A direct consequence of the \ref{VisinskaTocka} theorem and the previous task \ref{nalPodDesarg1}.

\item
\res{Let $A$ and $B$ be points outside the line $p$. Draw the intersection of the lines $p$ and $AB$ without directly drawing the line
$AB$.}

Draw any collinear points $C$, $C'$ and $S$, then: $Y=p\cap AC$, $X=p\cap BC$, $A'=SA\cap C'Y$ and $B'=SB\cap  C'X$. By the Desargues theorem \ref{izrekDesarguesEvkl} the lines $AB$ and $A'B'$ intersect at the point $Z$, which is collinear with the points $X$ and $Y$, therefore lies on the line $p$. This means that the intersection of the lines $p$ and $AB$ is obtained as the intersection of the lines $p$ and $A'B'$, that is, without directly drawing the line
$AB$.

\item
\res{Let $p$ and $q$ be lines of a plane that intersect at the point $S$, which is "outside the paper", and $A$
a point of this plane. Draw the line that goes through the points $A$ and $S$.}

Use the \ref{izrekDesarguesOsNesk} theorem.

\item
\res{Draw a triangle so that its vertices lie on three given parallel lines and the sides of the triangle pass through three given
points.}

Use the Desargues theorem \ref{izrekDesarguesEvkl}. See example \ref{zgled 3.2}.


%Ppotenca

\item
\res{The circle $k(S,r)$ is given.}

(\textit{a}) \res{What values can the power of the point have with respect to the circle $k$?}

Since according to \ref{izrekPotenca} the power of a point $P$ with respect to the circle $k$ is obtained as: $p(S,k)=|PS|^2-r^2$, the value of the power is on the interval $[-r^2,\infty)$.

 (\textit{b}) \res{What is the smallest value of this power and for which point is this minimum value achieved?}

From the previous example it is clear that the minimum value of the power is $-r^2$ and is achieved for the center $S$ of the circle $k$.

(\textit{c}) \res{Determine the set of all points for which the power with respect to the circle is equal to $\lambda\in \mathbb{R}$.}

The condition $p(S,k)=|PS|^2-r^2=\lambda$ and $|PS|^2=r^2+\lambda$ are equivalent.

In the case when $\lambda>-r^2$, it is a circle $k(S,r^2+\lambda)$. If $\lambda=-r^2$, the desired set is equal to $\{S \}$, and in the case $\lambda<-r^2$, the empty set $\emptyset$.

\item \res{Let $k_a(S_a,r_a)$ be inscribed and $l(O,R)$ be the circumscribed circle of some triangle. Prove the equality\footnote{The statement is a generalization of Euler's formula for a circle (see Theorem \ref{EulerjevaFormula}). \index{Euler, L.}
        \textit{L. Euler}
        (1707--1783), Swiss mathematician.}:
   $$S_aO^2=R^2+2r_aR.$$}

The proof is similar to the proof of Theorem \ref{EulerjevaFormula}.


\item
\res{Draw a circle that passes through the given points $A$ and $B$ and touches the given circle $k$.}

The desired circle is denoted by $x$.
We draw any circle $j$, which passes through the points $A$ and $B$, and the circle $k$ intersects in points $C$ and $D$. We use the fact that the intersection of the lines $AB$ and $CD$ is the power center of the circles $k$, $j$ and $x$.

The task is one of the ten Apollonius' problems about the touch of circles (see Section \ref{odd9ApolDotik}).

\item
\res{Prove that the centers of the lines that are determined by the common tangents of two circles are collinear points.}

The mentioned points lie on the power line of two circles.

\item
\res{Draw a circle that is perpendicular to two given circles, and the third given circle
intersects in points that determine the diameter of this third circle.}

Let the given circles be denoted in order by $k(K, r_k)$, $j(J,r_j)$ and $l(L,r_l)$, and the desired circle by $x(X,r_x)$.

Let $P\in x\cap k$, $Q\in x\cap l$ and $R, R_1\in x\cap j$.
By assumption, $x\perp k, l$. By Theorem \ref{TangPogoj}, $XP$ and $XQ$ are tangents of the circles $k$ and $l$ in points $P$ and $Q$. Since $XP\cong XQ$, the center $X$ of the circle $X$ lies on the power line $p(k,l)$ of the circles $k$ and $l$.

We look for another geometric location of points $X$, so that we include the condition for the circle $j$.
By assumption, $RR_1$ is the diameter of the circle $j$. This means that $J$ is the center of the line $RR_1$, and from the similarity of the triangles $XJR$ and $XJR_1$ (by the \textit{SSS} theorem \ref{SSS}) it follows that $\angle XJR=90^0$. By the Pythagorean theorem \ref{PitagorovIzrek} (for $\triangle XJR$) we then have:
 \begin{eqnarray*}
 |XR|^2=|XJ|^2+r_j^2.
 \end{eqnarray*}
From the right triangle $XQL$ we get by the same theorem:
 \begin{eqnarray*}
 |XQ|^2=r_l^2-|XL|^2.
 \end{eqnarray*}
Since $XR\cong XQ$, from the previous two relations it follows that $|XJ|^2+r_j^2=r_l^2-|XL|^2$ or:
\begin{eqnarray*}
 |XJ|^2+|XL|^2=r_l^2-r_j^2.
 \end{eqnarray*}
Therefore, the point $X$ lies on some circle $g$, which we can plot (see theorem \ref{GMTmnl}). This means that we get the point $S$ as the intersection of the circle $g$ and the power line $p(k,l)$ of the circles $k$ and $l$.



%Various

\item
\res{Let $M$ and $N$ be the intersections of the sides $AB$ and $AC$ of the triangle $ABC$ with a line that goes through the center of the inscribed circle of this triangle and is parallel to its side $BC$. Express the length of the line $MN$ as a function of the lengths of the sides of the triangle $ABC$.}

We use theorem \ref{HarmCetSimKota} and prove $\frac{AS}{SE}=\frac{b+c}{a}$.
Then we use \ref{TalesovIzrek} and prove $\frac{MN}{BC}=\frac{AS}{AE}$. Result: $|MN|=\frac{a(b+c)}{a+b+c}$.

\item
\res{Let $AA_1$ be the altitude of the triangle $ABC$. The points $P$ and $Q$ are the intersections of the altitudes of the angles $AA_1B$ and $AA_1C$ with the sides $AB$ and $AC$. Prove that $PQ\parallel BC$.}

By theorem \ref{HarmCetSimKota} we have:
$$\frac{AP}{PB}=\frac{AA_1}{A_1B}=\frac{AA1}{A_1C}=\frac{AQ}{QC}.$$
By the converse of theorem \ref{TalesovIzrekObr} it follows that $PQ\parallel BC$.

\item
\res{In the triangle $ABC$, let the sum (or difference) of the internal angles $ABC$ and $ACB$ be equal to the right angle. Prove that $|AB|^2+|AC|^2=4r^2$, where $r$ is the radius of the circumscribed circle of this triangle.}

Let $\alpha=\angle BAC$, $\beta=\angle ABC$ and $\gamma=\angle ACB$ and $k(O,r)$ be the circumscribed circle of the triangle $ABC$.

If $\beta+\gamma=90^0$, then $\alpha=90^0$ (by \ref{VsotKotTrik}), which means that $ABC$ is a right triangle with hypotenuse $BC$; the statement is trivial in this case as a consequence of \ref{TalesovIzrKroz2} and Pythagoras' theorem \ref{PitagorovIzrek}.

Let $\beta-\gamma=90^0$ or $\gamma=\beta-90^0$. We mark $C'=\mathcal{S}_O(C)$. By \ref{TalesovIzrKroz2}, we first have $\angle CAC'=90^0$. Because $BCC'A$ is a trapezoid, by \ref{TetivniPogoj}:
 \begin{eqnarray*}
\angle C'CA &=& 90^0-\angle AC'C=\\
            &=& 90^0-(180^0-\beta)=\beta-90^0=\\
            &=& \gamma =\angle BCA.
 \end{eqnarray*}

By \ref{ObodObodKot}, $AC'\cong AB$, so by Pythagoras' theorem \ref{PitagorovIzrek} we get:
 \begin{eqnarray*}
|AB|^2+|AC|^2=|AC'|^2+|AC|^2=|C'C|^2=4r^2.
 \end{eqnarray*}

\item
\res{Let $AD$ be the altitude of the triangle $ABC$. Prove that the sum (or difference) of the internal angles $ABC$ and $ACB$ is equal to the right angle exactly when:
$$\frac{1}{|AB|^2}+\frac{1}{|AC|^2}=\frac{1}{|AD|^2}.$$}

We consider the triangle $ADB$ and $CDA$.

\item
\res{Express the distance between the centroid and the center of the circumscribed circle of the triangle as a function of the lengths of its sides and the radius of the circumscribed circle.}

We use Stewart's theorem \ref{StewartIzrek} for the triangle $OAA_1$ and \ref{StwartTezisc}; $k(O,R)$ is the circumscribed circle, $T$ is the centroid and $A_1$ is the center of the side $BC$ of the triangle $ABC$.

 Result:
$|OT|=\sqrt{R^2-\frac{1}{9} \left(a^2+b^2+c^2 \right)}$.

\item
\res{Prove that in a triangle $ABC$, the external angle bisector at vertex $A$ and the internal angle bisectors at vertices $B$ and $C$ intersect the opposite sides at three collinear points.}

Use Menelaus' theorem \ref{izrekMenelaj} and theorem \ref{HarmCetSimKota}.

\item
\res{Prove that in a triangle $ABC$, the center of the altitude $AD$, the center of the inscribed circle and the point where side $BC$ touches the circumscribed circle of this triangle are three collinear points.}

Use theorem \ref{velNalTockP'}.

\item
\res{Prove Simson's theorem \ref{SimpsPrem} by using Menelaus' theorem \ref{izrekMenelaj}.}

Let $S$ be an arbitrary point of the circumscribed circle of triangle $ABC$ and $P$, $Q$ and $R$ the orthogonal projections of this point on sides $BC$, $AC$ and $BC$. Use:
$\triangle SRA\sim\triangle SPC$, $\triangle SPB\sim\triangle SQA$ and $\triangle SQC\sim\triangle SRB$.


\item
\res{A line through point $M$ of side $AB$ of triangle $ABC$ intersects side $AC$ at point $K$. Calculate the ratio in which line $MK$ divides side $BC$, if $AM:MB=1:2$ and $AK:AC=3:2$.}

Let $P$ be the intersection of lines $MK$ and $BC$. By Menelaus' theorem
$$-1=\frac{\overrightarrow{BP}}{\overrightarrow{PC}}\cdot
\frac{\overrightarrow{CK}}{\overrightarrow{KA}}\cdot
\frac{\overrightarrow{AM}}{\overrightarrow{MB}}=
\frac{\overrightarrow{BP}}{\overrightarrow{PC}}\cdot
\frac{-1}{3}\cdot\frac{1}{2},$$ so $\frac{\overrightarrow{BP}}{\overrightarrow{PC}}=6:1$.

\item
\res{Let $A_1$ be the center of side $BC$ of triangle $ABC$ and let $P$ and $Q$ be such points of sides
$AB$ and $AC$, that $BP:PA=2:5$ and $AQ:QC=6:1$. Calculate the ratio in which line  $PQ$ divides the median $AA_1$.}

Let $R$ be the intersection of lines $PQ$ and $BC$. Use Menelaus' theorem \ref{izrekMenelaj} first for triangle $ABC$ and line $PQ$, then for triangle $AA_1C$ and the same line. The result is $17:60$.

\item
\res{Prove that in an arbitrary triangle, the lines determined by the vertices and
the points of tangency of one circumscribed circle with the opposite sides intersect at a common point.}

We use Ceva's theorem \ref{izrekCeva} and the big task \ref{velikaNaloga}.

\item
\res{What does the set of all points represent, from which the tangent of the two given circles represents two compatible lines?}

Because in this case the power of these points with respect to the circle is equal, the desired set is part of its power line, which is outside both circles.

\item
\res{Let $PP_1$ and $QQ_1$ be the external tangents of the circles $k(O,r)$ and $k_1(O_1,r_1)$ (points $P$, $P_1$, $Q$ and $Q_1$ are the corresponding touch points). Let $S$ be the intersection of these two tangents, $A$ one of the intersections of the circles $k$ and $k_1$ and $L$ and $L_1$ the intersections
of the line $SO$ with the lines $PQ$ and $P_1Q_1$. Prove that $\angle LAO\cong\angle L_1AO_1$.}

Without loss of generality, let $\mathcal{B}(S,O,O_1)$.
Let $h_{S,\lambda}$ be the central extension with coefficient $\lambda=\frac{\overrightarrow{SO_1}}{\overrightarrow{SO}}$. Then $h_{S,\lambda}(k)=k_1$ and:
 $$h_{S,\lambda}:\hspace*{1mm} O,P,Q,L \mapsto O_1,P_1,Q_1,L_1.$$
Let $A_1=h_{S,\lambda}(A)$. From $h_{S,\lambda}(k)=k_1$ and $A\in k$ it follows
 $A_1\in k_1$. By the theorem \ref{homotOhranjaKote} we have $\angle L_1A_1O_1\cong\angle LAO$, so it is enough to prove $\angle L_1A_1O_1\cong\angle L_1AO_1$.

From the similarity of the right-angled triangles $SP_1O_1$ and $SL_1P_1$ (theorem \ref{PodTrikKKK}) and the theorem \ref{izrekPotenca} we get:
 \begin{eqnarray*}
\overrightarrow{SO_1}\cdot \overrightarrow{SL_1}=|SP_1|^2=p(S.k_1)=
\overrightarrow{SA}\cdot \overrightarrow{SA_1}.
 \end{eqnarray*}
This means that $O_1L_1AA_1$ is a string quadrilateral, so $\angle L_1A_1O_1\cong\angle L_1AO_1$.


\item
\res{Prove that the side of a regular pentagon is equal
to the larger part of the division of the radius of the circumscribed circle of this pentagon in the golden ratio.}

Let $k(S,r)$ be the center of the inscribed circle and $AB$ ($a=|AB|$) one side of the regular pentagon. In the triangle $ASB$ measure the internal angles:
 $\angle ASB=36^0$ and $\angle SAB=\angle SBA=72^0$.

We denote by $P$ such a point that $\mathcal{B}(B,A,P)$ and $AP\cong AS$. Because $SAP$ is an isosceles triangle with the base $PS$, by \ref{enakokraki} $\angle SPA\cong\angle PSA$. From \ref{zunanjiNotrNotr} (for the triangle $SAP$) follows
 $\angle APS=\angle PSA=\frac{1}{2}72^0=36^0$. Therefore $\triangle SPB\sim\triangle ASB$ (\ref{PodTrikKKK}), so $\frac{PB}{SB}=\frac{SB}{AB}$ or $\frac{a+r}{r}=\frac{r}{a}$, from which directly follows our claim.


\item \label{nalPod75}
\res{Let $a_5$, $a_6$ and $a_{10}$ be the sides of a regular pentagon, hexagon and decagon, which are inscribed in the same circle. Prove that:
 $$a_5^2=a_6^2+a_{10}^2.$$}

We use the similarity of triangles from the previous task \ref{nalPod75}, but we take into account the appropriate heights of these two triangles.

\item
 \res{Let $t_a$, $t_b$ and $t_c$ be the centroids and $s$ the semi-perimeter of a triangle. Prove that:
    $$t_a^2+t_b^2+t_c^2\geq s^2.$$} % zvezek - dodatni MG

If we use \ref{StwartTezisc2}:
 \begin{eqnarray*}
t_a^2+t_b^2+t_c^2 &=& \frac{3}{4}\left(a^2+ b^2+c^2\right)=\\
  &\geq& \frac{3}{4}\cdot 3\cdot\left(\frac{a+ b+c}{3}\right)^2=\\
 &=& s^2.
 \end{eqnarray*}

\end{enumerate}

Let $A_1$ be the center of side $BC$. Then, according to the statement \ref{PloscTrik}: $p_{AA_1B}=p_{AA_1C}$ and $p_{TA_1B}=p_{TA_1C}$. If we subtract the equality, we get $p_{AA_1B}-p_{TA_1B}$ and  $p_{AA_1C}-p_{TA_1C}$ or, according to the statement \ref{ploscGlavniIzrek} \textit{4)}: $p_{ABT}=p_{ATC}$. Similarly, we also get $p_{TBC}=p_{ATC}$.


\item \res{Let $c$ be the hypotenuse, $v_c$ the corresponding altitude, and $a$ and $b$ the catheti of the right triangle. Prove that $c+v_c>a+b$.} % zvezek - dodatni MG

According to the statement \ref{PloscTrik} for the area $p$ of this triangle, it holds that $p=\frac{1}{2}ab=\frac{1}{2}cv_c$, so $ab=cv_c$, but if we also use the Pythagorean statement \ref{PitagorovIzrek}:
 \begin{eqnarray*}
  (a+b)^2=a^2+2ab+b^2=
c^2+2cv_c<c^2+2cv_c+v_c^2=(c+v_c)^2.
 \end{eqnarray*}


  \item \res{Draw a square that has the same area as the rectangular triangle with sides $a$ and $b$.} % (Hipokratovi luni)

If $x$ is the side of the desired square, then $x^2=\frac{1}{2}ab$. Therefore, the side $x$ can be constructed according to the altitude statement \ref{izrekVisinski} from the relation $x=\sqrt{\frac{a}{2}\cdot b}$.


\item \res{Let $ABC$ be an isosceles right triangle with the lengths of the catheti $|CB|=|CA|=a$. Let $A_1$, $B_1$, $C_1$ and $C_2$ be the points for which the following holds: $\overrightarrow{CA_1}=\frac{n-1}{n}\cdot \overrightarrow{CA}$, $\overrightarrow{CB_1}=\frac{n-1}{n}\cdot \overrightarrow{CB}$, $\overrightarrow{AC_1}=\frac{n-1}{n}\cdot \overrightarrow{AB}$ and  $\overrightarrow{AC_2}=\frac{n-2}{n}\cdot \overrightarrow{AB}$ for some $n\in \mathbb{N}$. Express the area of the quadrilateral determined by the lines $AB$, $A_1B_1$, $CC_1$ and $CC_2$ as a function of $a$ and $n$.}

Let $X$ and $Y$ be the intersection points of the lines $CC_2$ and $CC_1$ with the line $A_1B_1$. Then (from the formula \ref{PloscTrik} and \ref{ploscGlavniIzrek} \textit{4)}):
  \begin{eqnarray*}
  p_{C_1C_2XY}&=&p_{CC_1C_2}-p_{CXY}=\frac{1}{n}p_{ABC}-\frac{1}{n}p_{A_1B_1C}=\\
&=& \frac{1}{n}\cdot \frac{1}{2}\cdot a^2-\frac{1}{n}\cdot \frac{1}{2}\cdot
\left(\frac{n-1}{n}\cdot a \right)^2=\frac{2n-1}{2n^3}\cdot a^2.
 \end{eqnarray*}

\item \res{Dan je pravokotni trikotnik $ABC$ s hipotenuzo $AB$ in ploščino $p$. Naj bo $C'=\mathcal{S}_{AB}(C)$,  $B'=\mathcal{S}_{AC}(B)$ in  $A'=\mathcal{S}_{BC}(A)$. Izrazi ploščino trikotnika $A'B'C'$ kot funkcijo ploščine $p$.}


From  $C'=\mathcal{S}_{AB}(C)$ it follows that $\triangle ACB \cong \triangle AC'B$, from $B'=\mathcal{S}_{AC}(B)$ and  $A'=\mathcal{S}_{BC}(A)$ we have
  $\mathcal{S}_{C}:\hspace*{1mm}A,B,C\mapsto A',B',C$ or $\triangle ACB \cong \triangle A'CB'$. From this it follows that $|A'B'|=|AB|=c$, the altitude of the triangle $A'B'C'$ from the point $C'$ is equal to three times the altitude $v_c$ of the triangle $ABC$ from the point $C$. By the formula \ref{PloscTrik} we have:
 $$p_{A'B'C'}=\frac{1}{2}\cdot c\cdot 3v_c=3\cdot\frac{1}{2}\cdot cv_c=3p_{ABC}.$$


\item \res{Naj bosta $R$ in $Q$ točki, v katerih se trikotniku $ABC$ včrtana krožnica dotika njegovih stranic $AB$ in $AC$. Simetrala notranjega kota $ABC$ naj seka premico $QR$ v točki $L$. Določi razmerje ploščin trikotnikov $ABC$ in $ABL$.}
  % pripremni zadaci - naloga 193

Let $R$ and $Q$ be the points in which the inscribed circle of the triangle $ABC$ touches its sides $AB$ and $AC$. The internal angle bisector of the triangle $ABC$ intersects the line $QR$ in the point $L$. Determine the ratio of the areas of the triangles $ABC$ and $ABL$.

We denote with $\alpha$, $\beta$ and $\gamma$ the internal angles at the vertices $A$, $B$ and $C$, with $A_1$ the center of the side $BC$ and with $S$ the center of the inscribed circle of the triangle $ABC$. The external angle $\angle BRE =\angle BRQ = 90^0+\frac{1}{2}\alpha$ of the triangle $BRQ$ is equal to the angle (from the statement \ref{VsotKotTrik}). Because according to the assumption $\angle EBR =\angle SBA = \frac{1}{2}\beta$, for the triangle $BRE$ it holds:
\begin{eqnarray*}
 \angle QES &=& \angle REB=\\
  &=& 180^0-(90^0+\frac{1}{2}\alpha+\frac{1}{2}\beta)=\\
 &=&  \frac{1}{2}\gamma = \angle ACS= \angle QCS.
 \end{eqnarray*}
 Therefore $\angle QES \cong \angle QCS$, so according to the statement \ref{ObodKotGMT} the quadrilateral $SCEQ$ is a cyclic quadrilateral. This means (from the statement \ref{ObodObodKot}) that it holds:
 $$\angle BEC=\angle SEC=\angle SQC=90^0.$$
 Therefore $EA_1$ is the altitude of the right triangle $BEC$ with the hypotenuse $BC$, so according to the statement \ref{TalesovIzrKroz2} it holds $|EA_1|=\frac{1}{2}|BC|=|BA_1|$. From this it follows that the triangle $BA_1E$ is an isosceles triangle with the base $BC$ or
$\angle A_1EB \cong\angle EBA_1\cong \angle ABE$ (from the statement \ref{enakokraki}) or $A_1E\parallel BA$ (from the statement \ref{KotiTransverzala}). From the latter it follows (from the statement \ref{PloscTrik}):
 $$p_{AEB}=p_{AA_1B}=\frac{1}{2}\cdot p_{ABC}.$$


\item \res{Določi točko v notranjosti trikotnika $ABC$, za katero je produkt njenih razdalj od stranic tega trikotnika maksimalen.} % zvezek - dodatni MG

The basics of Geometry

We denote with , and the internal angles at the vertices , with the center of the side and with the center of the inscribed circle of the triangle . The external angle of the triangle is equal to the angle (from the statement ). Because according to the assumption , for the triangle it holds:



Therefore , so according to the statement the quadrilateral is a cyclic quadrilateral. This means (from the statement ) that it holds:



Therefore is the altitude of the right triangle with the hypotenuse , so according to the statement it holds . From this it follows that the triangle is an isosceles triangle with the base or (from the statement ) or (from the statement ). From the latter it follows (from the statement ):

Let's denote the lengths of sides $BC$, $AC$ and $AB$ with $a$, $b$ and $c$ and the distances of any point $P$ inside the triangle $ABC$ from its sides $BC$, $AC$ and $AB$ with $x$, $y$ and $z$. Because $a$, $b$ and $c$ are constants, the maximum of the product $xyz$ is reached exactly when the maximum of the product $ax\cdot by\cdot cz$ is reached. If we use the known inequality between the geometric and arithmetic mean, we get:
 \begin{eqnarray*}
 ax\cdot by\cdot cz\leq \left( \frac{ax+by+cz}{3}\right)^3=\left( \frac{2p_{ABC}}{3}\right)^3.
 \end{eqnarray*}
In this case, the equality is reached (as well as the maximum of the product $ax\cdot by\cdot cz$, because the expression $\left( \frac{2p_{ABC}}{3}\right)^3$ is constant) exactly when $ax=by=cz$. In this case:
 \begin{eqnarray*}
 \left(ax\right)^3= \left( \frac{2p_{ABC}}{3}\right)^3=\left( \frac{a\cdot v_a}{3}\right)^3,
 \end{eqnarray*}
i.e. $x=\frac{v_a}{3}$. Similarly, $y=\frac{v_b}{3}$ and $z=\frac{v_c}{3}$, which means that the maximum of the product $ax\cdot by\cdot cz$ or $xyz$ is reached exactly when the point $P$ is the centroid of the triangle $ABC$.

\item \res{Trikotniku
             $ABC$ s stranicami $a$, $b$ in $c$ je včrtana krožnica. Načrtane so
             tangente te krožnice, ki so vzporedne  stranicam
             trikotnika.
             Vsaka od tangent v notranjosti trikotnika določa ustrezne daljice dolžin $a_1$, $b_1$ in $c_1$. Prove that:
 $$\frac{a_1}{a}+\frac{b_1}{b}+\frac{b_1}{b}=1.$$} % zvezek - dodatni MG

A circle is drawn through triangle $ABC$ with sides $a$, $b$ and $c$. The tangents of this circle, parallel to the sides of the triangle, are drawn. Each of these tangents, inside the triangle, determines the appropriate distances of lengths $a_1$, $b_1$ and $c_1$. Prove that:
 $$\frac{a_1}{a}+\frac{b_1}{b}+\frac{b_1}{b}=1.$$

Let: $r$ be the radius of the inscribed circle, $s$ be the semi-perimeter, and $v_a$, $v_b$, and $v_c$ be the corresponding altitudes of the triangle $ABC$. If we use the Pythagorean Theorem \ref{TalesovIzrek} and the formulas \ref{PloscTrik} and \ref{PloscTrikVcrt}, we get:
  \begin{eqnarray*}
 \frac{a_1}{a}+\frac{b_1}{b}+\frac{b_1}{b}&=&  \frac{v_a-2r}{v_a}+\frac{v_b-2r}{v_b}+\frac{v_c-2r}{v_b}=\\
&=& 3-\left(\frac{2r}{v_a}+\frac{2r}{v_b}+\frac{2r}{v_b}\right)=\\
&=& 3-\left(\frac{ar}{p_{ABC}}+\frac{br}{p_{ABC}}+\frac{cr}{p_{ABC}}\right)=\\
&=& 3-\frac{2sr}{p_{ABC}}=\\
&=& 3-2=1.
 \end{eqnarray*}

\item \res{Let $T$ and $S$ be the centroid and the center of the inscribed circle of the triangle $ABC$. Let also $|AB|+|AC|=2\cdot |BC|$. Prove that $ST\parallel BC$.} % zvezek - dodatni MG

Let: $r$ be the radius of the inscribed circle, $s$ be the semi-perimeter, $a$, $b$, and $c$ be the lengths of the sides, and $v_a$ be the corresponding altitude of the triangle $ABC$. If we use the given condition $b+c=2a$ and the formulas \ref{PloscTrik} and \ref{PloscTrikVcrt}, we get:
  \begin{eqnarray*}
 r=\frac{p_{ABC}}{s}=\frac{av_a}{3a}=\frac{v_a}{3}.
 \end{eqnarray*}
This means that the points $S$ and $T$ both lie on the line $h_{A,\frac{2}{3}}(BC)$, so the lines $ST$ and $BC$ are parallel.

\item \res{Let $p$ be the area of the triangle $ABC$, $R$ be the radius of the circumscribed circle, and $s'$ be the perimeter of the pedal triangle. Prove that $p=Rs'$.} %Lopandic - nal 918

One proof of this statement is given in the formula \ref{ploscTrikPedalni}. At this point, we will derive another variant of the proof.

Let $A'$, $B'$ and $C'$ denote the points of intersection of the altitudes from the vertices $A$, $B$ and $C$, $A_1$, $B_1$ and $C_1$ the midpoints of the sides $BC$, $AC$ and $AB$, and $O$ the center of the circumscribed circle of the triangle $ABC$. Because $\angle BB'C=\angle CC'B=90^0$, the quadrilateral $BCC'B'$ is a parallelogram (Tales' theorem \ref{TalesovIzrKroz2}), so $\angle B'C'A=\angle BCB'=\angle BCA=\gamma$ (theorem \ref{TetivniPogojZunanji}). From this, according to  theorem \ref{PodTrikKKK}, it follows that $\triangle ABC \sim \triangle AB'C'$. Therefore, it holds:
 \begin{eqnarray} \label{nalPlo1}
BC:B'C'=AB:AB'.
 \end{eqnarray}
But it also holds that $\angle BOA_1=\frac{1}{2}\angle BOC=\angle BAC=\alpha$ (theorem \ref{SredObodKot}), so
$\triangle ABB' \sim \triangle OBA_1$ (theorem \ref{PodTrikKKK}) or:
 \begin{eqnarray} \label{nalPlo2}
AB:AB'=OA:OA_1=R:OA_1.
 \end{eqnarray}
From relations \ref{nalPlo1} and \ref{nalPlo2} it follows that:
 \begin{eqnarray} \label{nalPlo3}
 |B'C'|=\frac{|BC|\cdot |OA_1|}{R}.
 \end{eqnarray}
Similarly, we have:
 \begin{eqnarray} \label{nalPlo4}
 |A'B'|=\frac{|AB|\cdot |OC_1|}{R} \hspace*{1mm} \textrm{ and }
 \hspace*{1mm} |A'C'|=\frac{|AC|\cdot |OB_1|}{R}.
 \end{eqnarray}
Finally, if we use relations \ref{nalPlo3} and \ref{nalPlo4} and theorems \ref{PloscTrik} and \ref{ploscGlavniIzrek} \textit{4)}, we get:
\begin{eqnarray*}
 s'&=& \frac{1}{2}\cdot \left(|A'B'|+|B'C'|+|A'C'|\right)=\\
&=&\frac{1}{R} \cdot
\left(\frac{|AB|\cdot |OC_1|}{2}+\frac{|BC|\cdot |OA_1|}{2} +\frac{|AC|\cdot |OB_1|}{2} \right) =\\
&=&\frac{1}{R} \cdot
\left(p_{OAB}+p_{OBC} +p_{OAC} \right) =\\
&=& \frac{1}{R} \cdot p_{ABC},
 \end{eqnarray*}
 i.e. $p=Rs'$.
% Sstirikotniki

\item \res{Let $L$ be an arbitrary point inside the parallelogram $ABCD$. Prove that it holds:
        $$p_{LAB}+p_{LCD}=p_{LBC}+p_{LAD}.$$} %Lopandic - nal 890
Let $a$ and $b$ be the lengths of the sides and $v_a$ and $v_b$ be the corresponding heights of the parallelogram $ABCD$. We also mark $v_{a_1}$ and $v_{a_2}$ the heights of the triangles $LAB$ and $LCD$ from the vertex $L$ and with $v_{b_1}$ and $v_{b_2}$ the heights of the triangles $LBC$ and $LAD$ from the vertex $L$. Because $v_{a_1}+v_{a_2}=v_a$ and  $v_{b_1}+v_{b_2}=v_b$, according to the formulas \ref{PloscTrik}, \ref{ploscParal} and \ref{ploscGlavniIzrek} \textit{4)} it holds:
\begin{eqnarray*}
&& p_{LAB}+p_{LCD}=\frac{av_{a_1}}{2}+\frac{av_{a_2}}{2}=\frac{1}{2}av_a=\frac{1}{2} p_{ABCD}\\
&& p_{LBC}+p_{LAD}=\frac{av_{b_1}}{2}+\frac{av_{b_2}}{2}=\frac{1}{2}bv_b=\frac{1}{2} p_{ABCD}.
 \end{eqnarray*}

\item \res{Let $P$ be the center of the leg $BC$ of the trapezoid $ABCD$. Prove:
 $$p_{APD}=\frac{1}{2}\cdot p_{ABCD}.$$}

Let $v$ be the height of the trapezoid, $Q$ the center of the leg $AD$ or $|PQ|=m$  the median of the trapezoid $ABCD$.
According to the formulas \ref{PloscTrik}, \ref{PloscTrik}, \ref{ploscTrapez} and \ref{ploscGlavniIzrek} \textit{4)} it holds:

\begin{eqnarray*}
 p_{APD}=p_{APQ}+p_{QPD}=\frac{1}{2}m\frac{v}{2}+\frac{1}{2}m\frac{v}{2}=
 \frac{1}{2}mv=\frac{1}{2} p_{ABCD}.
 \end{eqnarray*}

\item \res{Let $o$ be the circumference, $v$ the height and $p$ the area of the tangent trapezoid. Prove that it holds: $p=\frac{o\cdot v}{4}$.}

We use the formula \ref{ploscTetVec}.

\item \res{Draw a line $p$, which goes through the vertex $D$ of the trapezoid $ABCD$ ($AB>CD$), so that it divides this trapezoid into two figures with equal area.}

Let $v$ be the height of the trapezoid and $a$ and $c$ be the lengths of its bases $AB$ and $CD$.
If $X$ is a point in which the line $p$ intersects the base $AB$, from the given condition it follows:
$p_{AXD}=\frac{1}{2}p_{ABCD}$ or:
\begin{eqnarray*}
 \frac{|AX|\cdot v}{2} =\frac{1}{2}\cdot \frac{a+c}{2}\cdot v.
 \end{eqnarray*}

Therefore, the distance $AX$ is equal to the median of the trapezoid or $|AX|=\frac{a+c}{2}$, which allows the construction of the point $X$ and the line $p$.


\item \res{Draw the line $p$ and $q$, which pass through the vertex $D$ of the square $ABCD$ and divide it into areas of equal size.}

If $P$ and $Q$ are points in which the line $p$ or $q$ intersects the side $AB$ or $BC$ of the square $ABCD$, we prove that $AP:PB=2:1$ and $BQ:QC=1:2$.


\item \res{Let $ABCD$ be a square, $E$ be the center of its side $BC$ and $F$ be a point for which $\overrightarrow{AF}=\frac{1}{3}\cdot \overrightarrow{AB}$. The point $G$ is the fourth vertex of the rectangle $FBEG$. What part of the area of the square $ABCD$ does the triangle $BDG$ represent?}

Let $a$ be the length of the side of the square $ABCD$.
Let $S$ be the center of this square and $H=\mathcal{T}_{\overrightarrow{AF}}(D)$. By formulas \ref{PloscTrik}, \ref{ploscKvadr} and \ref{ploscGlavniIzrek} \textit{4)} we have:
\begin{eqnarray*}
 p_{BDG} &=& p_{BSG}+p_{SDG}= p_{FSG}+p_{SHG}=\\
 &=& p_{FSH}=\frac{1}{2}\cdot |FH| \cdot |GS|= \\
 &=& \frac{1}{2}\cdot a \cdot \left( \frac{a}{2}-\frac{a}{3} \right)=\\
 &=& \frac{a^2}{12} = \frac{1}{12}\cdot p_{ABCD}.
 \end{eqnarray*}


\item \label{nalPloKoef}
\res{Let $\mathcal{V}$ and $\mathcal{V}'$ be similar polygons with a similarity coefficient of $k$. Prove that:
 $$p_{\mathcal{V}'}=k^2\cdot p_{\mathcal{V}}.$$}

First, we prove that the statement is true for triangles (we use the formula \ref{PloscTrik}), then we use the formula \ref{ploscGlavniIzrek} \textit{4)}.

\item \label{nalPloHeronStirik}
 \res{Let $a$, $b$, $c$ and $d$ be the lengths of the sides, $s$ the semiperimeter and $p$ the area of any quadrilateral. Prove that:
        $$p=\sqrt{(s-a)(s-b)(s-c)(s-d)}.$$} %Lopandic - nal 924

Let $A$, $B$, $C$ and $D$ be the vertices of the given quadrilateral, so that $|AB|=a$, $|BC|=b$, $|CD|=c$ and $|DA|=d$. If $ABCD$ is a parallelogram, due to its rigidity it is a rectangle (see \ref{paralelogram} and \ref{TetivniPogoj}) and the statement is trivial.

Without loss of generality, we assume that the sides $BC$ and $AD$ intersect in some point $P$ and that $a>c$. We also mark $|PC|=x$, $|PD|=y$ and $p'=p_{PCD}$.
According to the theorem \ref{TetivniPogojZunanji}, $\angle PCD\cong \angle BAD$, therefore $\triangle CDP \cong \triangle ABP$ (theorem \ref{PodTrikKKK}). The similarity coefficient of these two triangles is $k=\frac{AB}{CD}=\frac{a}{c}$. Thus, according to the statement from the previous task \ref{nalPloKoef}, it holds:
 $$\frac{p_{ABP}}{p_{CDP}}=\frac{a^2}{c^2}$$
or, according to the theorem \ref{ploscGlavniIzrek} \textit{4)}
$$\frac{p_{ABCD}+p_{CDP}}{p_{CDP}}=\frac{a^2}{c^2}.$$
From this, we get:
\begin{eqnarray} \label{nalPloEqnHeronStirik1}
 p=\frac{a^2-c^2}{c^2}\cdot p'.
 \end{eqnarray}
First, let us calculate $p'=p_{PCD}$. From the mentioned similarity $\triangle CDP \cong \triangle ABP$, it also follows:
\begin{eqnarray*}
 \frac{x+b}{y}=\frac{y+d}{x}=\frac{a}{c}.
 \end{eqnarray*}
If we rearrange the equations, we get a system of equations for $x$ and $y$:
 \begin{eqnarray*}
  && ax-cy=cd\\
 && cx-ay=-bc
 \end{eqnarray*}
and its solutions:
\begin{eqnarray*}
  x &=& \frac{c(ad+bc)}{a^2-c^2}\\
 y &=& \frac{c(ab+cd)}{a^2-c^2}
 \end{eqnarray*}
From this, we get:
\begin{eqnarray} \label{nalPloEqnHeronStirik2}
  x+y &=& \frac{c(b+d)}{a+c}\\
 x-y &=& \frac{c(d-b)}{a-c}
 \end{eqnarray}
Let $s'$ be the semi-perimeter of the triangle $PCD$. By using the relations \ref{nalPloEqnHeronStirik2}, we get:
 \begin{eqnarray*}
 s' &=& \frac{x+y+c}{2}=\frac{c}{a-c}\cdot (s-c)\\
 s'-c &=& \frac{x+y-c}{2}=\frac{c}{a-c}\cdot (s-a)\\
 s'-x &=& \frac{c+x+y}{2}=\frac{c}{a+c}\cdot (s-d)\\
 s'-y &=& \frac{c+x+y}{2}=\frac{c}{a+c}\cdot (s-b)
 \end{eqnarray*}
From the previous relations according to Heron's formula for the area of a triangle (theorem \ref{PloscTrikHeron}), it holds:
\begin{eqnarray*}
 p' &=& p_{PCD}=\sqrt{s'(s'-c)(s'-x)(s'-y)}=\\
 &=& \frac{c^2}{a^2-c^2}\sqrt{(s-a)(s-b)(s-c)(s-d)}.
 \end{eqnarray*}
If we insert this into the relation \ref{nalPloEqnHeronStirik1}, we get:
\begin{eqnarray*}
 p = \frac{a^2-c^2}{c^2}\cdot p'=\sqrt{(s-a)(s-b)(s-c)(s-d)}.
 \end{eqnarray*}

\item \res{Let $a$, $b$, $c$ and $d$ be the lengths of the sides and $p$ the area of the tangent quadrilateral. Prove that:
        $$p=\sqrt{abcd}.$$} %Lopandic - nal 925

Let $s$ be the semiperimeter of the given quadrilateral.
Since it is a tangent quadrilateral, for its sides according to \ref{TangentniPogoj} it holds that $a+c=b+d$. Therefore:
\begin{eqnarray*}
 s-a &=& \frac{b+c+d-a}{2}=c\\
 s-b &=& \frac{a+c+d-b}{2}=d\\
 s-c &=& \frac{a+b+d-c}{2}=a\\
 s-d &=& \frac{a+b+c-d}{2}=b.
 \end{eqnarray*}
Since the quadrilateral is also a taut, we can use the statement from the previous task \ref{nalPloHeronStirik} and get:
$$p=\sqrt{(s-a)(s-b)(s-c)(s-d)}=\sqrt{abcd}.$$

% Veckotniki

\item \res{Given is a rectangle $ABCD$ with sides $a=|AB|$ and $b=|BC|$. Calculate the area of the figure which represents the union of the rectangle $ABCD$ and its image after reflection over the line $AC$.}

Let $B'=\mathcal{S}_{AC}(B)$ and $D'=\mathcal{S}_{AC}(D)$, $P$ be the intersection of the lines $AB'$ and $CD$ and $S$ be the intersection of the diagonal $AC$ and $BD$ of the rectangle $ABCD$. We also mark $d=|AC|=|BD|$ and $v=|SP|$. From the similarity of the triangles $AB'C$ and $ASP$ (\ref{PodTrikKKK}) we get $v:b=\frac{d}{2}:a$ or $v=\frac{bd}{2a}$. If we use the obtained relation, \ref{PloscTrik}, \ref{ploscPravok} and \ref{ploscGlavniIzrek} and Pythagoras' theorem \ref{PitagorovIzrek}, for the area $p$ of the sought figure it holds:

\begin{eqnarray*}
 p &=& 2\cdot \left(p_{AB'C}+p_{ADC}-p_{APC}\right)=
2\cdot \left(p_{ABC}+p_{ADC}-p_{APC}\right)=\\
&=& 2\cdot \left(p_{ABCD}-p_{APC}\right)=2\cdot \left( ab-\frac{dv}{2} \right)=\\
&=& 2\cdot \left( ab-\frac{bd^2}{4a} \right)=2ab-\frac{b(a^2+b^2)}{2a}=\\
&=& \frac{b(3a^2-b^2)}{2a}.
 \end{eqnarray*}

\item \res{Let $ABCDEF$ be a regular hexagon, and let $P$ and $Q$ be the centers of its sides $BC$ and $FA$. What part of the area of this hexagon does the area of the triangle $PQD$ represent?}

We use the formula \ref{srednjTrapez}. Result: $p_{PQD}=\frac{3}{8}p_{ABCDEF}$.


% KKrog

\item \res{In a square, four congruent circles are drawn so that each circle touches two sides and two other circles. Prove that the sum of the areas of these circles is equal to the area of the square of the inscribed circle.}

If we denote the length of the square's side with $a$, then $r=\frac{a}{2}$ or $r_1=\frac{a}{4}$ is the radius of the inscribed circle or each of the smaller inscribed circles. Then the aforementioned sum of areas is equal to:
$$4\cdot r_1^2\pi=4\left(\frac{a}{4} \right)^2\cdot \pi=\left(\frac{a}{2} \right)^2\cdot \pi=r^2\pi.$$

\item \res{Calculate the area of the circle inscribed in the triangle with sides of lengths 9, 12 and 15.}

We use Heron's formula \ref{PloscTrikHeron} and the formula \ref{PloscTrikVcrt}. Result: the area of the triangle  $p_{\triangle}=54$; the area of the circle $p=9\pi$.

\item \res{Let $P$ be the center of the base $AB$ of the trapezoid $ABCD$, for which $|BC|=|CD|=|AD|=\frac{1}{2}\cdot |AB|=a$. Express the area of the figure determined by the base $CD$ and the shorter circular arcs $PD$ and $PC$ of the circles with centers $A$ and $B$, as a function of the base $a$.}

Since, by assumption, $\overrightarrow{PB}=\overrightarrow{DC}$, the quadrilateral $PBDC$ is a parallelogram, so $BC\cong PD$. Since, by assumption, $AD\cong BC$ and $AD\cong AP$, we have $PD\cong AD\cong AP$ and $APD$ is an isosceles triangle. Therefore, $\angle PAD= 60^0$. Similarly, $\angle CBP=60^0$. The desired area $p_0$ is the difference between the area of the trapezoid $p$ and twice the area of the circular sector with central angle $60^0$:
$$p_0=\frac{a^2}{12}\left(9\sqrt{3}-4\pi \right).$$

\item \res{The chord $PQ$ ($|PQ|=d$) of the circle $k$ touches its concurring circle $k'$. Express the area of the trapezoid determined by the circles $k$ and $k'$, as a function of the chord $d$.}

We use the Pythagorean Theorem \ref{PitagorovIzrek}. Result: $\frac{d^2\pi}{4}$.

\item \res{Let $r$ be the radius of the inscribed circle of the polygon $\mathcal{V}$, which is divided into triangles $\triangle_1,\triangle_2,\ldots,\triangle_n$, so that no two triangles have common inner points. Let $r_1,r_2,\ldots , r_n$ be the radii of the inscribed circles of these triangles. Prove that:
     $$\sum_{i=1}^n r_i\geq r.$$}

Let $s$ be the semiperimeter and $p$ the area of the polygon $\mathcal{V}$, and let $s_i$ be the semiperimeter and $p_i$ ($i\in \{1,2,\ldots , n\}$) the area of the triangle $\triangle_i$. If we use the formula \ref{ploscTetVec} and \ref{PloscTrikVcrt} and the fact that for each $i\in \{1,2,\ldots , n\}$ we have $s\geq s_i$, we get:
 $$sr=p= \sum_{i=1}^n p_i = \sum_{i=1}^n s_ir_i\leq \sum_{i=1}^n sr_i=s\cdot\sum_{i=1}^n r_i.$$


\end{enumerate}




%REŠITVE -  Inverzija
%________________________________________________________________________________

\poglavje{Inversion}


\begin{enumerate}

 \item \res{Prove that the composition of two inversions $\psi_{S,r_1}$ and $\psi_{S,r_2}$ with respect to a concentric
  circle represents a dilation. Determine the center and the coefficient
  of this dilation.}

 From the definition of inversion it follows that the composition of inversions
 $\psi_{S,r_2}\circ\psi_{S,r_1}$
  is a dilation with center $S$ and coefficient $\frac{r_2^2}{r_1^2}$.
  The statement is also a direct
  consequence of the task \ref{invRazteg}.


  \item \res{Let $A$, $B$, $C$ and $D$ be four collinear points.
  Construct such points $E$ and $F$, that $\mathcal{H}(A,B;E,F)$
and $\mathcal{H}(C,D;E,F)$ hold.}

  We use the formula \ref{harmPravKrozn} - first we draw the circle, which
  is perpendicular to the circle with diameters $AB$ and $CD$,

\item \res{In a plane, given are a point $A$, a line $p$ and a circle $k$.
Draw a circle that goes through point $A$ and is
perpendicular to line $p$ and circle $k$.}

Use inversion with center $A$.

\item \res{Solve the third, fourth, ninth and tenth Apollonius problem.}

For the third and fourth Apollonius problem we use inversion
with center in one of the given points. For the ninth and tenth
Apollonius problem we first draw a circle that is
concentric with the sought circle and goes through the center of one of
the circles. In this way we translate the problem to the fifth or sixth
Apollonius problem.

\item \res{Let: $A$ be a point, $p$ be a line, $k$ be a circle
 and $\omega$ be an angle in some plane. Draw a circle that goes through point $A$, touches line $p$ and with circle $k$
determines angle $\omega$.}

Use inversion with center $A$.

 \item \res{Determine the geometric location of the points of intersection of two circles that
 touch at the arms of a given angle in two given points $A$ and $B$.}

 Use inversion with center $B$.

  \item \res{Draw a triangle, if the following data are known:
\begin{enumerate}
 \item $a$, $l_a$, $v_a$
 \item $v_a$, $t_a$, $b-c$
 \item $b+c$, $v_a$, $r_b-r_c$
 \end{enumerate}}

  Use the great problem (see Theorem \ref{velikaNaloga}) and
  the appropriate harmonic quadruples of points.


\item \res{Let $c(S,r)$  and $l$ be a circle and a line in the same plane that
have no common points. Let $c_1$, $c_2$ and $c_3$ be circles in this
plane that touch each other (two at a time) and each of them
touches also $c$ and $l$. Express the distance of point $S$ from line $l$ with
$r$\footnote{Proposal for MMO 1982. (SL 12.)}.}

First, we prove that there exists a circle $n$, which is perpendicular to
   the line $l$ and the circle $c$. Let $Y$ be one of the intersections
   of the circle $n$ and the rectangle on the line $l$ from the center of the
   circle $c$. We use the composition $f=\psi_i\circ \mathcal{R}\circ \psi_i$,
   where $i$ is an arbitrary circle with center $Y$, $\mathcal{R}$
    is a rotation with center in the center of the circle $l'=\psi_i(l)$,
    which maps the tangent points of the circles $l'$ and $c'_3=\psi_i(c_3$) to the point
    $Y$. If:
    $f:\hspace*{1mm}l, c, c_1, c_2, c_3\mapsto
    \widehat{l}, \widehat{c}, \widehat{c_1}, \widehat{c_2}, \widehat{c_3},$
    we prove that $\widehat{l}=l$, $\widehat{c}=c$, $\widehat{c_3}$
    is a line parallel to the line $l$, and $\widehat{c_1}$ and
     $\widehat{c_2}$ are congruent circles, which touch each other and
     also touch the parallels $l$  and $\widehat{c_3}$.
     In the end, it follows that the distance from the center of the circle $c$ to the line $l$
     is equal to $7r$.


\item \res{Let $ABCD$ be a regular tetrahedron. To any point
$M$, lying on the edge $CD$, we assign the point $P = f(M)$, which is the intersection of the rectangle through the point $A$ on the line $BM$ and
the rectangle through the point $B$ on the line $AM$. Determine the geometric
location of all points $P$, if the point $M$ takes all values on the edge
$CD$.}

The point $P$ is the altitude of the triangle $ABM$. If $S$ is the center of the
edge $AB$, first we prove that $\overrightarrow{SP}\cdot
\overrightarrow{SM}=\frac{a^2}{4}$, where $a$ is the edge
of the regular tetrahedron. Then we use the inversion
$\psi_{S,\frac{a}{2}}$ (in the plane $SCD$). The geometric location of the points
is then the image of the segment $CD$ under this inversion, i.e. the corresponding circular
arc with center $S$, with endpoints in the altitudes of the triangles
$ACD$ and $BCD$.

\item \res{Let $ABCD$ be a tangent-chord quadrilateral and $P$, $Q$,
$R$ and $S$ the points of tangency of sides $AB$, $BC$, $CD$ and $AD$
with the inscribed circle of this quadrilateral. Prove that $PR\perp
QS$.}

We use inversion with respect to the inscribed circle. We prove
that the vertices of the quadrilateral $ABCD$ are the vertices of a
rectangle, and the sides of this rectangle are parallel to the
distances $PR$ and $QS$.


\item \res{Prove that the centers of a tangent-chord quadrilateral, the inscribed and
the circumscribed circle, and the intersection of its diagonals are
collinear points (\index{izrek!Newtonov}Newton's theorem\footnote{\index{Newton,
I.}\textit{I. Newton} (1643--1727), English physicist and mathematician}).}

We use the previous exercise and prove that the center $G$ of the
rectangle from that exercise is also the center of the line determined
by the center of the inscribed circle of the quadrilateral $ABCD$
(the center of inversion) and the intersection of the lines $PR$ and
$QS$. The point $G$ is in fact the center of the image of the
circumscribed circle under that inversion.


\item \res{Let $p$ and $q$ be parallel tangents to the circle $k$.
Circle $c_1$ is tangent to the line $p$ at point $P$ and to the
circle $k$ at point $A$, circle $k_2$ is tangent to the line $q$ and
to the circles $k$ and $k_1$ at points $Q$, $B$ and $C$. Prove that
the intersection of the lines $PB$ and $AQ$ is the center of the
triangle $ABC$ of the circumscribed circle.}

We use inversion with center $B$ and first prove that $PB$ is a
common tangent to the circles $k$ and $k_2$, and then that the
intersection of the lines $PB$ and $AQ$ is the potent center of the
circles $k$, $k_1$ and $k_2$.

\item \res{The circles $k_1$ and $k_3$ touch each other externally in the point $P$. The circles $k_2$ and $k_2$ also touch each other externally in the same point. The circle $k_1$ intersects the circles $k_2$ and $k_4$ also in the points $A$ and $D$, the circle $k_3$ intersects the circles $k_2$ and $k_4$ also in the points $B$ and $C$. Prove that it holds\footnote{A suggestion for MMO 2003. (SL 16.)}:
 $$\frac{|AB|\cdot|BC|}{|AD|\cdot|DC|}=\frac{|PB|^2}{|PD|^2}.$$}

Let $\psi_P$ be an inversion with an arbitrary radius $r$. This
transforms the quadrilateral $A'B'C'D'$ into a parallelogram (by
the statement \ref{InverzDotik}). So $A'B'\cong C'D'$ holds. If we
use the statement \ref{invMetr}, we get
 $\frac{|AB|\cdot r^2}{|PA|\cdot|PB|}=\frac{|CD|\cdot
 r^2}{|PC|\cdot|PD|}$ or
 $\frac{|AB|}{|CD|}=\frac{|PA|\cdot|PB|}{|PC|\cdot|PD|}$. In a
 similar way from $C'B'\cong A'D'$ we get
  $\frac{|CB|}{|AD|}=\frac{|PC|\cdot|PB|}{|PA|\cdot|PD|}$. By
  multiplying the two relations we get:
  $\frac{|AB|\cdot|BC|}{|AD|\cdot|DC|}=\frac{|PB|^2}{|PD|^2}$.

  \item \res{Let $A$ be a point that lies on the circle $k$. With the help of a
   ruler only, draw a square $ABCD$ (or its vertices), that is inscribed in the given
circle.}

  First we draw a regular hexagon  $AB_1B_2CD_1D_2$, that
  is inscribed in the given circle.

 \item \res{Given are the points $A$ and $B$. With the help of
   a ruler only, draw such a point $C$, that
   $\overrightarrow{AC}=\frac{1}{3}\overrightarrow{AB}$.}

 We use a similar procedure as in the problem \ref{MaskeroniSred}.
  First we draw the point $X$, for which
  $\overrightarrow{AX}=3\cdot \overrightarrow{AB}$, then the desired
  point
  $X'=\psi_k(X)$, where $k$ is a circle with the center $A$ and the radius $AB$.

   \item \res{With the help of
   a ruler only, divide the given segment in the ratio $2:3$.}

For a given line $AB$ we first draw a point $X$, for which
   $\overrightarrow{AX}=2\cdot \overrightarrow{AB}$ (similarly to
   the previous task), then
    a point $Y$, for which
   $\overrightarrow{AY}=\frac{1}{5}\cdot \overrightarrow{AX}$ holds. $Y$
   is the desired point, because
   $\overrightarrow{AY}=\frac{2}{5}\cdot \overrightarrow{AB}$ holds.
\end{enumerate}
\newpage


\normalsize

%________________________________________________________________________________
% LITERATURA - - - - - - - - - - - - - - - - - - - - - - - - - - - - - - - - - - - - - - -
%________________________________________________________________________________
\begin{thebibliography}{1}


        \bibitem{Berger}  Berger, M.
\emph{Geometry}, Springer-Verlag, Berlin 1987.

        \bibitem{Cofman}  Cofman, J.
\emph{What to solve?}, Oxford University Press, Oxford, 1990.

        \bibitem{CoxeterRevisited}  Coxeter, H. S. M.; Greitzer, S. L.
\emph{Geometry Revisited}, Random House, New York, 1976.

        \bibitem{Djerasimovic}  Djerasimovi\'c, B.
\emph{Zbirka zadataka iz geometrije}, Stručna knjiga, Beograd, 1987.

        \bibitem{MMO}  Djuki\'c, D.; Jankovi\'c, V.;
        Mati\'c, I.; Petrovi\'c, N.
\emph{The IMO Compendium}, Springer, New York, 2006.

         \bibitem{Efimov}  Efimov, N. V.
\emph{Višja geometrija}, Nauka, Moskva, 1978.

         \bibitem{Evklid}  Evklid, \emph{Elementi}, Naučna knjiga, Beograd, 1949.

         \bibitem{Fetisov}  Fetisov, A. I.
\emph{O euklidskoj i neeuklidskim geometrijama}, Školska knjiga, Zagreb, 1981.

        \bibitem{KratkaZgodCasa} Hawking, S. W.
\emph{Kratka zgodovina časa}, DMFA, Ljubljana, 2003.

        \bibitem{Hilbert}  Hilbert, D. \emph{Osnove geometrije}, Naučno delo, Beograd, 1957.

        \bibitem{ZutaKnjiga}  Lopandi\'c, D.
\emph{Geometrija}, Naučna knjiga, Beograd, 1979.

        \bibitem{Lopandic}  Lopandi\'c, D.
\emph{Zbirka zadataka iz osnova geometrije}, PMF, Beograd, 1971.

        \bibitem{Lucic}  Luči\'c, Z.
\emph{Euklidska i hiperbolička geometrija}, Matematički
fakultet, Beograd, 1994.

\bibitem{Martin} \emph{Martin, G.}, Transformation Geometry, Springer-Verlang, New York, 1982.

        \bibitem{Mitrovic}  Mitrovi\'c, M.
\emph{Projective geometry}, DMFA, Ljubljana, 2009.

        \bibitem{MitrovicMG}  Mitrovi\'c, M.; Ognjanovi\'c, S.; Veljkovi\'c, M.; Petkovi\'c, L.; Lazarevi\'c, N.
\emph{Geometry for the first grade of Mathematical Gymnasium}, Krug, Beograd, 1996.

        \bibitem{Nice}  Niče, V.
\emph{Introduction to synthetic geometry}, Školska knjiga, Zagreb,
1956.

        \bibitem{Prasalov} Prasalov, V. V. \emph{Problems in geometry}, Nauka, Moskva, 1986.

        \bibitem{Prvanovic} Prvanovi\'c, M. \emph{The basics of geometry}, Gradjevinska knjiga, Beograd, 1987.

        \bibitem{Tosic}  Toši\'c, R.;  Petrovi\'c, V.
\emph{A collection of problems in the basics of geometry}, Gradjevinska
knjiga, Novi Sad, 1982.

        \bibitem{Stojanovic}  Stojanović, V. \emph{Matematiskop III}, Nauka, Beograd, 1988.

        \bibitem{Struik}  Struik, D. J.
\emph{A brief history of mathematics}, Državna založba Slovenije,
Ljubljana, 1978.

        \bibitem{Oblika} Weeks, J. R.
\emph{The shape of space}, DMFA, Ljubljana, 1998.

%%%% Pregledano v celoti! Roman